\begin{figure}[H]
    \centering
    \begin{tikzpicture}
        \begin{object}[text width=4 cm]{TestCase}{0 ,1}
            \attribute{TestCaseId : INT}
            \attribute{Name : VARCHAR}
            \attribute{Language : VARCHAR}
        \end{object}
        \begin{object}[text width=4 cm]{Dut}{10,1}
            \attribute{DutId : INT}
            \attribute{Name : VARCHAR}
            \attribute{OS : VARCHAR}
            \attribute{Version : INT}
        \end{object}
        \begin{object}[text width=4 cm]{Run}{5,5}
            \attribute{RunId : INT}
            \attribute{DutId : INT}
            \attribute{TestCaseId : INT}
            \attribute{Value : VARCHAR}
        \end{object}
        \begin{object}[text width = 4 cm]{MeasuringInstrument}{5,1}
            \attribute{InstrumentId : INT}
            \attribute{Name : VARCHAR}
        \end{object}
        \begin{object}{Measurement}{6.5,-2.5}
            \attribute{MeasurementId : INT}
            \attribute{ConfigId : INT}
            \attribute{InstrumentId : INT}
            \attribute{DutId : INT}
            \attribute{TestCaseId : INT}
            \attribute{Runs : INT}
            \attribute{Iteration : INT}
            \attribute{FirstMeasuring : VARCHAR}
            \attribute{StartTime : DATETIME(6)}
            \attribute{EndTime : DATETIME(6)}
        \end{object}
        \begin{object}[text width=4 cm]{RawData}{0, -10}
            \attribute{RawDataId : INT}
            \attribute{MeasurementId : INT}
            \attribute{Value : VARCHAR}
            \attribute{Time : DATETIME(6)}
        \end{object}
        \begin{object}[text width=4 cm]{TimeSeries}{0, -4.5}
            \attribute{TimeSeriesId : INT}
            \attribute{MeasurementId : INT}
            \attribute{Value : VARCHAR}
            \attribute{Time : DATETIME(6)}
        \end{object}
        \begin{object}[text width = 4 cm]{Configuration}{5,-10}
            \attribute{ConfigurationId : INT}
            \attribute{MinTemp : INT}
            \attribute{MaxTemp : INT}
            \attribute{Between : INT}
            \attribute{Duration : INT}
            \attribute{MinBattery : INT}
            \attribute{MaxBattery : INT}
            \attribute{Version : INT}
        \end{object}
        \begin{object}[text width = 4 cm]{Environment}{10, -10}
            \attribute{EnvironmentId : INT}
            \attribute{MeasurementId : INT}
            \attribute{Time : DATETIME(6)}
            \attribute{Name : VARCHAR}
            \attribute{Value : INT}
            \attribute{Type : VARCHAR}
        \end{object}
        
        \association{Run}{}{0..*}{TestCase}{}{1}
        \association{Run}{}{0..*}{Dut}{}{1}
        \association{Measurement}{}{0..*}{Dut}{}{1}
        \association{Measurement}{}{0..*}{MeasuringInstrument}{}{1}
        \association{Measurement}{}{0..*}{Dut}{}{1}
        \association{Measurement}{}{1}{RawData}{}{1}
        \association{Measurement}{}{1}{TimeSeries}{}{1}
        \association{Measurement}{}{0..*}{Configuration}{}{1}
        \association{Measurement}{}{0..*}{Environment}{}{0..*}
        \association{Measurement}{}{0..*}{TestCase}{}{1}
    \end{tikzpicture}
    \caption{An UML diagram representing the tables in the SQL database} \label{fig:uml_diagram}
\end{figure}



