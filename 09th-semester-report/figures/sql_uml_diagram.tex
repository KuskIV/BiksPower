\begin{figure}[H]
    \centering
    \begin{tikzpicture}
        \begin{object}[text width=4 cm]{Test case}{0 ,0}
            \attribute{Test caseId : INT}
            \attribute{Name : VARCHAR}
        \end{object}
        \begin{object}[text width=4 cm]{DUT}{10,0}
            \attribute{DUTId : INT}
            \attribute{Name : VARCHAR}
            \attribute{OS : VARCHAR}
            \attribute{Version : INT}
        \end{object}
        \begin{object}[text width=4 cm]{Run}{5,5}
            \attribute{RunId : INT}
            \attribute{DUTId : INT}
            \attribute{Test caseId : INT}
            \attribute{Value : VARCHAR}
        \end{object}
        \begin{object}[text width = 4 cm]{Measuring instrument}{5,0}
            \attribute{Measuring instrumentId : INT}
            \attribute{Name : VARCHAR}
        \end{object}
        \begin{object}{Experiment}{5,-3}
            \attribute{ExperimentId : INT}
            \attribute{ConfigId : INT}
            \attribute{Measuring instrumentId : INT}
            \attribute{DUTId : INT}
            \attribute{Test caseId : INT}
            \attribute{Runs : INT}
            \attribute{Iteration : INT}
            \attribute{FirstMeasuring instrument : VARCHAR}
            \attribute{Language : VARCHAR}
            \attribute{StartTime : DATETIME(6)}
            \attribute{EndTime : DATETIME(6)}
        \end{object}
        \begin{object}[text width=4 cm]{RawData}{0, -10}
            \attribute{RawDataId : INT}
            \attribute{ExperimentId : INT}
            \attribute{Value : VARCHAR}
            \attribute{Time : DATETIME(6)}
        \end{object}
        \begin{object}[text width = 4 cm]{Configuration}{5,-10}
            \attribute{ConfigurationId : INT}
            \attribute{MinTemp : INT}
            \attribute{MaxTemp : INT}
            \attribute{Between : INT}
            \attribute{Duration : INT}
            \attribute{MinBattery : INT}
            \attribute{MaxBattery : INT}
            \attribute{Version : INT}
        \end{object}
        \begin{object}[text width = 4 cm]{Temperature}{10, -10}
            \attribute{TemperatureId : INT}
            \attribute{ExperimentId : INT}
            \attribute{Time : DATETIME(6)}
            \attribute{Name : VARCHAR}
            \attribute{Value : INT}
        \end{object}
        
        \association{Run}{}{0..*}{Test case}{}{1}
        \association{Run}{}{0..*}{DUT}{}{1}
        \association{Experiment}{}{0..*}{DUT}{}{1}
        \association{Experiment}{}{0..*}{Measuring instrument}{}{1}
        \association{Experiment}{}{0..*}{DUT}{}{1}
        \association{Experiment}{}{0..*}{RawData}{}{1}
        \association{Experiment}{}{0..*}{Configuration}{}{1}
        \association{Experiment}{}{0..*}{Temperature}{}{1}
        \association{Experiment}{}{0..*}{Test case}{}{1}
    \end{tikzpicture}
    \caption{An UML diagram representing the tables in the SQL database} \label{fig:uml_diagram}
\end{figure}



