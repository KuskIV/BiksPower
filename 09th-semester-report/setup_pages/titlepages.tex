\pdfbookmark[0]{English title page}{label:titlepage_en}
\aautitlepage{%
  \englishprojectinfo{
    Measuring Measuring Instruments %title
  }{%
    Programming Technology %theme
  }{%
    Fall Semester 2022 %project period
  }{%
    cs-22-pt-9-03 % project group
  }{%
    %list of group members
    Mads Hjuler Kusk\\ 
    Jamie Baldwin Pedersen\\
    Jeppe Jon Holt
  }{%
    %list of supervisors
    Bent Thomsen\\
    Lone Leth Thomsen
  }{%
    1 % number of printed copies
  }{%
    \today % date of completion
  }%
}{%department and address
  \textbf{Department of Computer Science}\\
  Aalborg University\\
  \href{http://www.aau.dk}{http://www.aau.dk}
}{% the abstract
  Based on observations of how most existing work chose to use RAPL when measuring the energy consumption of software, this work sets out to explore how other measuring instruments perform, including Intel Power Gadget, E3 and Libre Hardware Monitor. This is achieved by constructing a framework able to log and save measurements made by different measuring instruments on different test cases targeting different parts of the system. During the analysis, the impact of temperature, battery levels and R3 validation are explored and illustrated. In order to find the best performing measuring instrument, the correlation between the measuring instruments and the ground truth is found, using a combination of Mann Whitney U tests and  Kendel Tau correlation coefficients. Through this study the different measuring instruments all show promising results and reasonable usability.
}

% \cleardoublepage
% {\selectlanguage{danish}
% \pdfbookmark[0]{Danish title page}{label:titlepage_da}
% \aautitlepage{%
%   \danishprojectinfo{
%     Rapportens titel %title
%   }{%
%     Semestertema %theme
%   }{%
%     Efterårssemestret 2010 %project period
%   }{%
%     XXX % project group
%   }{%
%     %list of group members
%     Forfatter 1\\ 
%     Forfatter 2\\
%     Forfatter 3
%   }{%
%     %list of supervisors
%     Vejleder 1\\
%     Vejleder 2
%   }{%
%     1 % number of printed copies
%   }{%
%     \today % date of completion
%   }%
% }{%department and address
%   \textbf{Elektronik og IT}\\
%   Aalborg Universitet\\
%   \href{http://www.aau.dk}{http://www.aau.dk}
% }{% the abstract
%   Her er resuméet
% }}