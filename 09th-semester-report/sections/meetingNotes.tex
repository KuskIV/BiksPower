% VEJLEDERMØDE 30/09/22

% Hypoteser / forventning til vores resultater burde skrives i begyndelsen af rapporten et sted. 

% Benchmark game okay til vores sanity check. måske også bruge noget større fx rosetta code. for merer repræsentative programmer.

% We will answer (research questions)

% We need research question before describing the rest

% Hvorfor gennemgør vi lige de software based approached til at measure energy. 

% Not quick introduction, maybe short. But maybe just delete in entirety. They don't like unnecessary words

% Første sætning af related work sucks. 

% Lære peter sestoft (staves) måske vores censor. (Microbenchmarks in Java and C#)


% VEJLEDERMØDE 06/10/2022

% istedet for it has been found -> kilde 27 has found that.....

% Istedet for bare at omskrive alt kan det nvære okay at bare citere så vi ikke bliver bonget for plagiat som man ofte gør når man skriver om stattiske metoder. 

% Der mangler kilder i nogle af afsnitene. Fx den her metode er brugt frequenly. 

% deemed er lidt fluff or lidt dårligt argument. Sig kilde 18 fx siger at. ++ Nogle ord på hvorfor det er okay. 

% Metric er en underlig overskrift

% Beskriver vores arbejdsmetode og hvordan vi angriber vores projekt. Det skal vi tænke over så vi eventuelt kan laver ændringer i næste semester, hvis der er noget som ikke fungere så godt.

% Kolmogorov–Smirnov test ???? staves


Vejldermøde 20/10-2022

Begrundt om feetings findings er gode. fx med et eksampel med forskellig terminologi

% Kilde på points fra sestoft. Kilde på lig inde vi opremser bullet points. -> DONE

% Bokhari firstly, secondly, thirdly. --> DONE

% Fjern delen om hvad vi vil gøre i related work. --> DONE

Tydligør i RW kaptiel at underoverskrifterne stadig er related work og vi først senere vælger hvad vi bruger af de forskellige rws.

% when the power meter chip is not available... Not existing --> Done

fjern det med om chipperen er der fra 3.7.1

% formula can be used. WIll be used. --> Done

% Hvor kommer bullet points fra i 3.7.1 ref --> cite added

Diskutere forskellen fra de fx kilder fx sestoft/R3/Feetings og hvorfor vi vælger de forksellige ting. 

ups, I zoned out ),:

opsumering af SOTA 

SUMMARY: FLERE CITES PLS
%%%%%%%%%%%%%%%%%%%%%%%%%%%%%%%%%%%%%%%%%%%%%%%%%%%%%%%%%%%%%%%%%%%%
VEJLEDERMØDE 27/10

% vi bruger should for meget. results should be comparable --- they have to be comparable. -> Removed a lot of shoulds.

% Vi er ikke så selvsikker i E3 afsnittet. Måske skrive fra vores antagelser så assumer vi. --> Done

Gem resultater fra E3 eksperiementer. evt på github.

% HVad er det vi gerne vil vise.  Groudn truth vs rapl, E3, intel power gadget, open HW. --> det står i introduktionen.

% FAhad, senere har vi RAPL In ACtion som er nyere. Så tilføj den til related work.

Vi skal skrive vores vurdereinger om RAPL good or bad. 

Beskrive hvad vi får ud af RW. Hvad tager vi med os videre.

ift til at finde becnharms kig på de gamle opgaver. 

Vi må gerne udvide hilke benchmarks vi bruger vi skal bare prøve at arguemntere for hvorfor vi vil bruge dem. fx en stres test. Vi må gerne være kritiske overfor de benchmarks folk bruger.

% Skriv hvorfor vi valgte OHW. hvorfor vi tænker der andre er de samme. Begrund begrund begrund.  might be a threat to validity though. --> done

% SQL afsnittet. Arugment for hvorfor vi ikke bruger CSV ligesom alle andre. Lige skrive de fleste bruger CSV, men vi gør såden her fordi....

% SQL makes a lot of sense needs source. 
Hvorfor ser structuren ud som den gør.

Skriv noget om hvilke sprog vi bruger. 

% sestoft kommentar om interfaces. Kun for Java???  gælder det for C#

% skriv til aage og CC bent og lone.

% Prioter det de skal læse, hvis nu de ikke kan nå det hele. eller hvad de skal fokuserer på når de læser.

