%How do we systematically measure the energy consumption of a test case on different DUTs and OSs and how do we make the test case measurements comparable?
%How do existing energy measurements instruments compare to each other?
%How does the test case measurement compare between windows and Linux?
%How does the test case measurement compare between different DUTs?

In this section our work is concluded by summarizing the answers to the research questions presented in \cref{ch:introduction} and then forming and overall conclusions to our works main goal of comparing different measuring instruments in order to explore how the measuring instruments on Windows performs in comparison with our ground truth, the clamp measurements and the software-based solution most commonly utilized in the literature, which is RAPL. The first research question, \textbf{RQ1}, was about how to measure the energy consumption the different DUTs and OSs in way that made them comparable. The method we used was presented in \cref{sec:experimental_setup} where our measurement methodology and our framework is described. Furthermore the use of dynamic energy consumption allowed us to compare the energy consumption of the test cases with minimum noise from the OSs. The third research question, \textbf{RQ3}, was about how the measurements compare on the two OSs. In experiment \#1 we observed a generally higher or similar standard deviation for the measurement on Windows. We also saw that Windows in the majority of cases had a higher actual energy consumption as well as dynamic energy consumption. It was unexpected that Windows had a higher dynamic energy consumption, which prompted experiment \#2. When looking at the correlation we found that LHM and IPG had a higher correlation than RAPL with their corresponding clamp measurements. In experiment \#2 where the C-states were disabled our results changed to Windows having a lower dynamic power consumption in the majority of cases which corresponds to our expectations.
%%% CORROLATION exp 2
The fourth research questions, \textbf{RQ4}, was about how the DUTs compare. It is clear that the workstation has a higher energy consumption than the laptops, which is as expected. More over for the laptops the Surface Pro 4 has a higher energy consumption. 

\input{tabels/experiment_results/exp_two/StatQuest/COrrelationPower.tex}
