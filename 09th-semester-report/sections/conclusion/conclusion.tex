In this section our work is concluded by summarizing the answers to the research questions presented in \cref{ch:introduction} and then forming and overall conclusions to our works main goal of comparing different measuring instruments in order to explore how the measuring instruments on Windows performs in comparison with our ground truth, and the software-based solution most commonly utilized in the literature, which is RAPL. 

%The first research question, \textbf{RQ1}, was about how to measure the energy consumption the different DUTs and OSs in way that made them comparable. 
First of all the method we used was presented in \cref{sec:experimental_setup} where our measurement methodology and our framework is described. Furthermore the use of dynamic energy consumption allowed us to compare the energy consumption of the test cases with minimum noise from the OSs and across different DUTs. 

%The third research question, \textbf{RQ3}, was about how the measurements compare on the two OSs. 
In experiment \#1 we observed a higher or similar standard deviation for the measurement on Windows compared to Linux. We also saw that Windows in the majority of cases had a higher actual energy consumption as well as dynamic energy consumption. However, it was unexpected that Windows had a higher dynamic energy consumption than Linux, which prompted experiment \#2. In experiment \#2 where the C-states were disabled our results changed to Windows having a lower dynamic power consumption in the majority of cases which corresponds to our expectations. 

%%% CORROLATION exp 2
%The fourth research question, \textbf{RQ4}, was about how the DUTs compare. 
It is clear that the workstation has a higher energy consumption than the laptops, as expected. For the laptops the Surface Pro 4 has a higher energy consumption compared to the Surface Book. One difference between the laptops was the MAXIM chip, which is in the Surface Book, that supposedly should make E3 more accurate. The E3 measurement for the Surface Book does have a $.02$ and $.01$ higher correlation with IPG and LHM respectively, which could mean that E3 is more accuracte on the Surface Book if IPG and LHM is accurate. Notably the Surface Book has lower correlation between the RAPL measurements and the other software-based measuring instruments than the Surface Pro 4, it is however unclear why.

%The second research question, \textbf{RQ2}, about how do measuring instruments compare.
For the measuring instruments themselves, we found that LHM and IPG are very similar in terms of measurements as well as being highly correlated with each other. RAPL in the majority of cases measured a lower energy consumption than LHM and IPG and it also has a lower standard deviation, but in a few cases it has a higher energy consumption. E3 in most cases report an energy consumption between RAPL and LHM/IPG. However the requirements of having a battery and the method of which E3 functions makes it more tedious to utilizes for our purpose. In experiment \#1, the highest correlation with the clamp on the same OS as the software-based measuring instrument was LHM and IPG with a $.01$ difference. Where as RAPL had a slightly lower correlation, which suggest that LHM and IPG are more accurate than RAPL. However, in experiment \#2 RAPL has the highest correlation with the clamp. Therefore RAPL is more accurate in our test cases, however IPG and LHM are also highly correlated, while E3 has a low to moderate correlation. The higher correlation between LHM and IPG suggest similarities and one reason could be similar methods of getting energy consumption from the CPUs. 
