\section{Experiment \#1}\label{sec:experiment_one}

Following the initial experiments presented in \cref{sec:initial_experiments} the experiments can now be conducted. This will be done based on the findings of \cref{sec:initial_experiments}. This means the measurements will execute for one minute as a result of E3, R3 validation will be disregarded, temperature will not have an upper limit and the battery limits are set between 40-80\%. A few scripts were also introduced, including a script to ensure the framework is running and a script to start the framework when on startup on Windows.

\paragraph*{}
This experiment is conducted with an aim to answer the research questions presented in \cref{ch:introduction}. Through these the research questions, comparisons will be made between measuring instruments, OSs and DUTs and aims to answer the question of how good RAPL is compared to alternatives.

\paragraph*{}
For the initial experiments, the experiments were run with $120$ measurements for each unique test case measurement. This was done with the hope of getting a wide variety of temperatures, and battery levels as well as calculate an estimated standard deviation to use to calculate how many measurements would be required according to Cochran's formula. Based on what has been found so far, no additional measurements are required and no new requirements have been found which would require any changes to the current state of the framework. Because of this, no new measurements are made, and the data used to analyze this experiment will be the same as in the initial experiment.

\paragraph*{}
The next step is thus to present and analyze the data in more depth, from this experiment, which will be done in \cref{ch:results}. Based on observations, one additional experiment will however be conducted, as will be covered in \cref{sec:experiment_two}.