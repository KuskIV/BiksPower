\subsection{Sample size}\label[subsec]{subsec:sampleSize}
%3.05 - Z 2.58, E = 0.01
% w/o outliers: 2.36 
%% we have enough runs so it is also okay if we include outliers.
After the initial experiment phase, an estimation of the standard deviation for each configuration can be calculated. The standard deviation is calculated for each test case measurement using the measurements. In the initial experiment, a $120$ measurements were conducted for each configuration. Using Cochran's formula as shown in \cref{cochransEQ2} we can calculate how many measurements are required. In \cref{subsec:cock} we aimed to use a confidence level of 95\% and a margin of error of 3\%, however, the required measurements came out using Cochran's formula to a low number. Therefore, we tried with stricter requirements than what we initially aimed for in \cref{subsec:cock}. This being a confidence level of 99\%, which gives a Z-value of $2.58$ and a margin of error of 1\% was also used. We calculated the required measurements for each configuration and the highest output was $3.05$ with anomalies and $2.36$ having removed some extreme anomalies. Thus if we include all anomalies and round up the highest even number the maximum amount of required measurements is $4$. Therefore since we have a $120$ measurements, running further measurements is not necessary. As to why such a low number of measurements is required we suspect it is due to the standard deviation being low and each measurement in reality consisting of thousands of samples e.g. Binary Tress in one case has a 100666 samples in a single measurement of one minute. However, the number of samples in a single measurement varies from test case to test case as well as from run to run.

% Example of samples in a measurement: 100666 BinaryTrees
