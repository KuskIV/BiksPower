\subsection{Sample Size}\label[subsec]{subsec:CockUse}
%3.05 - Z 2.58, E = 0.01
% w/o outliers: 2.36 
%% we have enough runs so it is also okay if we include outliers.
After the initial experiment phase, an estimation of the standard deviation for each configuration can be calculated. The standard deviation is calculated for each test case measurement. In the initial experiment, $120$ measurements were conducted for each configuration. Using Cochran's formula as shown in \cref{cochransEQ2} we can calculate how many measurements are required. In \cref{subsec:cock} we aimed to use a confidence level of 95\% and a margin of error of 3\%, however, the required amount of measurements using Cochran's formula was a low number. Therefore, we tried with stricter requirements than what we initially aimed for in \cref{subsec:cock}. The new, stricter requirements are a confidence level of 99\%, which gives a Z-value of $2.58$ and a margin of error of 1\%. We calculated the required measurements for each configuration and the highest output was $3.05$ with anomalies and $2.36$ having removed anomalies. Thus if we include all anomalies and round up, the maximum required measurements are $4$. Therefore since we have $120$ measurements, running further measurements is not necessary. As to why such a low number of measurements is required, it is due to the standard deviation being low since each measurement in reality consists of thousands of samples. This could for example be Binary Trees, which in one case has $100666$ samples in a single measurement of one minute. However, the number of samples in a single measurement varies from test case to test case as well as from run to run.
\todo{Needs update}
%%%%
% with 1.96 and 0.03 = 37.71
% 1.96 and 0.03
%%%%
% Example of samples in a measurement: 100666 BinaryTrees

