\subsection{Sample Size}\label[subsec]{subsec:CockUse}
%3.05 - Z 2.58, E = 0.01
% w/o outliers: 2.36 
%% we have enough runs so it is also okay if we include outliers.
After the initial experiment phase, an estimation of the standard deviation for each configuration can be calculated. The standard deviation is calculated for each test case measurement. In the initial experiment, $120$ measurements were conducted for each configuration. Using Cochran's formula as shown in \cref{cochransEQ2} we can calculate how many measurements are required. In \cref{subsec:cock} we aimed to use a confidence level of 95\% and a margin of error of 3\%, however, the required amount of measurements using Cochran's formula was a low number. Therefore, we tried with stricter requirements than what we initially aimed for in \cref{subsec:cock}. The new, stricter requirements are a confidence level of 99\%, which gives a Z-value of $2.58$ and a margin of error of 1\%. We calculated the required measurements for each configuration and the highest output was $3.05$ with anomalies and $2.36$ having removed anomalies. Thus if we include all anomalies and round up, the maximum required measurements are $4$. Therefore since we have $120$ measurements, running further measurements is not necessary. As to why such a low number of measurements is required, it is due to the standard deviation being low since each measurement in reality consists of thousands of samples.

\paragraph{}
When calculating how many measurements were required for the different test cases, one outlier was found, this being the idle test case. During the one minute of execution, the DUT would sleep for 30 seconds twice, meaning only two samples were made during one minute. When calculating how many measurements were required, this resulted in thousands of samples. In an attempt to avoid having to run this test case thousands of times, the sleep duration of the idle test case was set to $2$ms, resulting in more samples for each measurement. The result of this was that Cochran's formula required $55$ measurements for this test case using a confidence level of 95\% and a margin of error of $0.03$. Given that we had more than $55$ measurements for all test cases, no additional measurements were deemed necessary.


%%%%
% with 1.96 and 0.03 = 37.71
% 1.96 and 0.03
%%%%
% Example of samples in a measurement: 100666 BinaryTrees

