\subsection{Expected Energy Consumption} \label[subsec]{subsec:expec_energy_consumption}

In this section the expected energy consumption of the different components within the DUT's will be covered. For the consumption, the values and calculation will be covered, in addition to the rational behind them. This will be achieved by looking at the expected range of energy consumption of each component individually, this will then provide a lower and upper limit for the consumption of the DUT.

When referring to the energy consumption of pc component, one term which can be used is Thermal Design Power (TDP). In the work by Hennessy\cite{hennessy2011computer}, the TDP is is defined as a representation of the average power a processor will draw, when all cores are active and under a high complexity workload. While under peak workload the processor can reach around $1.5$ times more than its TDP. The lower power limit of a CPU is disputed, but it seems the be around  $10-25$W for most modern CPU's\cite{CPUPowerCum}.

\paragraph{Workstation} The first DUT is the workstations, and the energy consumption for the different components can be seen in \cref{tab:WorkstationEstEnergCon}.

\begin{table}[ht]
    \begin{tabular}{||c | c | c | c | c | c ||}
    \hline
    % \multicolumn{5}{l}{Workstation} \\
     & RAM\cite{buildComputerRAM} & CPU\cite{IntelCPUi7} & SSD\cite{tomshardwareSSD} & Motherboard\cite{CPUPowerCum} & Fans\\
    Minimum consumption & $1.5 * 2$W & $10$W   & $0.2$W & $15$W & \\ \hline
    Average consumption & $2.25 * 2$W & $65$W   & $6$W & $37.5$W & \\ \hline
    Maximum consumption & $3 * 2$ & $97.5$W & $9$W &  $60$W & $2.4W*2$ \\ \hline
    \end{tabular}
    \caption{The energy consumption for the different components in the workstation}
    \label{tab:WorkstationEstEnergCon}    
\end{table}
\todo{Add second SSD}
The power supply does no have a energy specific energy consumption, but instead has a energy efficiency. The energy efficiency represents how much of energy going into the power supply, is used by the system. The power supply used in the workstation has a 80-Plus Gold certification, representing an efficiency of $90\%$ at a load of 20\%, an efficiency of $92\%$ at a load of 50\% and an efficiency of $89\%$ at a load of 100\%. This is highly relevant for the hardware measurements as they are measured before going into the power supply and will thus be around 8-11\% higher that the actual system consumption.

Using these values lower and upper bounds for the expected energy consumption can be created for the DUTs.

\paragraph{Software measurement instruments}: The different software approaches measures the CPU and ram individually so the Min and Max values for these components are the values the measurements should be between. Given the test cases will not put the DUT under a significant load, the measurements would be expected to be close to the following:

$$CPU: 10W < x < 65W $$
$$Ram: 1.5W < y < 2.25W $$

Where $x$ and $y$ represents the measured energy consumption of the CPU and Ram respectively.

\paragraph{Hardware measurements:} The hardware measurements include the energy consumption of the entire system. In order to calculate the lower and upper limits to the energy consumption, the values found in \cref{tab:WorkstationEstEnergCon} are summarized. Based on observations, the CPU never utilizes more than $40\%$. Because of this, and as there is no GPU in the DUT, the efficiency of the power supply is set to the minimum efficiency for the calculations. The lower bound is thus calculated to be:
\todo{Call hardware measurements Clamp measurements}
$$LowerCase = 35.75W =  (MinCPU + MinBoard + MinRam + AvgSSD + CPUFan + CaseFan)*(1+(\frac{1}{MinEff}))$$

$avgSSD$ was used instead of $minSSD$ based on observations, where the SSD is used but the OS all the time.

$$AverageCase = 119.6W = (AvgCPU+AvgBoard+AvgRam+AvgSSD+ CPUFan + CaseFan)*(1+(1/AvgEff))$$ 

The measurements from the hardware are expected to be between the $LowerCase$ and $AverageCase$.

\paragraph{Surface Book 1:} As the Surface Book 1 is laptop and is running on a battery, the hardware measurement setup will not be used on this DUT. Thus only the CPU and RAM will be subject to the measurements. The energy consumption of the components of this DUT can be seen in \cref*{tab:bookjuanEstEnergCon}

$$CPU: 7.5W < x < 14W $$
$$Ram: 2W < y < 2.5W $$

\begin{table}[ht]
    \begin{tabular}{llllll}
    \hline
    \multicolumn{5}{l}{Surface Book 1} \\
     & RAM\cite{buildComputerRAM} & CPU\cite{NotebookCPU} \\
    Minimum consumption & $2$W& $7.5$W \\ \hline
    Average consumption & $2.5$W& $14$W \\ \hline
    Maximum consumption & $3$W & $21$W \\ \hline
    \end{tabular}
    \caption{The energy consumption for the different components in the Surface Book 1}
    \label{tab:bookjuanEstEnergCon}    
\end{table}

\paragraph{Surface Pro 4} The Surface Pro 4 is also a laptop and is hardware wise similar to the Surface book 1. They are similar in therms of the RAM, and the CPU is from the same generation, but one is a better version. The energy consumption for the Surface Pro 4 can be seen in \cref{tab:proQuadroEstEnergCon}, and the upper and lower bounds are as following:

$$CPU: 9.5W < x < 15W $$
$$Ram: 2W < y < 2.5W $$

\begin{table}[ht]
    \begin{tabular}{llllll}
    \hline
    \multicolumn{5}{l}{Surface Pro 4} \\
     & RAM\cite{buildComputerRAM} & CPU\cite{IntelCPUSpec} \\
    Minimum consumption & $1.5 * 2$W& $9.5$W \\ \hline
    Average consumption & $2.25 * 2$W& $15$W \\ \hline
    Maximum consumption & $3 * 2$ & $22.5$W\\ \hline
    \end{tabular}
    \caption{The energy consumption for the different components in the Surface Pro 4}
    \label{tab:proQuadroEstEnergCon}    
\end{table}

These should be considered rough estimates for what the DUTs can consume, values both above and below these estimates might still be valid, but require some further investigation if the deviation is significant.