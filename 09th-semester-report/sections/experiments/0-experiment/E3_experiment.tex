\subsection{E3 Experiments}

The following experiments with E3 is therefore about addressing some of the discoveries made in our initial testing with E3, which  contradict what is stated in the available sources\cite[]{E3Doc,E3Video,E3WinHec}. 

This experiment was conducted based on observations made, which did not match with the sources, with respect to the log file. This is because, in the output log file for E3, each row represents the measurement of one application in a given state. For this, it is claimed that an application needs to run for $1-5$ minutes before it is added to the file. Based on observations this is not entirely true, as the lowest measurements were observed to be as low as either $0$ or $1.554$ seconds. The largest durations found in the test is $902$ minutes, which does not match the claim. Another aspect about the measurements is that E3 counts the same application as different measurements depending on its status.
Example \texttt{App.Status=Focus}, \texttt{App.Status=Visible}, \texttt{App.Status=Minimized} and \texttt{App.Status=NotUnique} will all be counted as different measurements in E3. While the first three seem fairly self-explanatory the last one is less so. By looking at the processes in it and the frequency of their repetition over a period, we have concluded that these are background processes which do not have an user interface. 
Because of the uncertainty about how E3 functions it was deemed necessary to conduct some experiments to determine when a process is included in the measurements and when it is not. To carry out these experiments different scenarios were formulated, which will be covered now.

% \begin{itemize}
%     \item Order of measurement
%     \begin{itemize}
%         \item Scenario 1: A process is started before the measurements and ended after the measurements
%         \item Scenario 2: A process is started during the measurements and ended after the measurements
%         \item Scenario 3: A process is started before the measurements and ended during the measurements
%     \end{itemize}
%     \item State Change
%     \begin{itemize}
%         \item Scenario 4: Several process "state" are swapped between with increasing intervals
%         \item Scenario 5: Several process "state" are swapped between with a fixed interval
%         \item Scenario 6: Change The "state" of a single process during measurements
%     \end{itemize}
%     \item Instances
%     \begin{itemize}
%         \item Scenario 7: A process opened and restarted several times during the measurements
%         \item Scenario 8: Several instances of the same application
%     \end{itemize}
%     \item Measurement resolution
%     \begin{itemize}
%         \item Scenario 9: Taking several measurements with different durations
%     \end{itemize}
% \end{itemize}

\paragraph{Order of measurement}

\begin{itemize}
    \item Scenario 1: A process is started before the measurements and ended after the measurements
    \item Scenario 2: A process is started during the measurements and ended after the measurements
    \item Scenario 3: A process is started before the measurements and ended during the measurements
\end{itemize}

\paragraph {Expectations}
The first three scenarios are designed to see when measurements are recorded by E3. The initial expectation is that E3 uses the start and exit timestamp for the measured process. Because of this, expectations are that scenarios 1 and 2 would not record a process, but that scenario 3 would.

\paragraph{Findings}
Scenario 1 resulted in several measurements of the process, which contradicts the expectations for the experiment. Similar results were obtained from scenario 2, which again was contrary to our expectations. In scenario 3 the process is not found at all which is contrary to our expectation. These results point to that E3 instead of using the start and end times instead take several snapshots of the process over some time period. Exactly how frequent these snapshots are will be tested later. 

\paragraph{State Change}

\begin{itemize}
    \item Scenario 4: Several process "state" are swapped between with increasing intervals
    \item Scenario 5: Several process "state" are swapped between with a fixed interval
    \item Scenario 6: Change The "state" of a single process during measurements
\end{itemize}

\paragraph{Expectations}
Scenarios 4 and 5 are meant to see how E3 handles changes in the process state during the measurements. Scenario 4 specifically attempts to see how granular a new state can be measured. While scenario 5 would use the lowest successful measurement found in scenario 4 to check for consistency. The expectations are that each state change will be measured until a certain threshold where the change will not be registered as the swapping is more frequent than E3's sampling is. Scenario 6 is designed to test something similar to scenario 4 but instead, uses the same process to see how that changes the results.
\paragraph{Findings}
For scenarios 4 and 5 swapping between the states is only being recorded at the first occurrence and then just aggregates the results for further state changes. The speed of the change does not to hinder E3's measurements contrary to our expectations and does not create a new measurement for each state change. Scenario 6 had same results as 4 and 5 but ended up providing some interesting insight. Each application instance has the same id in E3 and cannot be differentiated based on it, but the execution time for each instance is carried over from state to state so using this the two identical processes can be identified since the time reported by E3 is the total execution at the time of collection.

\paragraph{Instances}

\begin{itemize}
    \item Scenario 7: A process opened and restarted several times during the measurements
    \item Scenario 8: Several instances of the same application
\end{itemize}

\paragraph{Expectations}
Scenario 7 is designed to see how E3 looks at multiple starts and shutdowns during measurements. The expectation is that each instance will be a separate measurement. Scenario 8 is designed to see how E3 handles multiple concurrent instances of the process with the same process id the expectation here is that the measurements will be merged into one big measurement. 
\paragraph{Findings}
In scenario 7 a process opening and closing within 1 second seem to not always get picked up by E3, but the instances that do get picked up seem to be aggregates of the instances recorded. This could indicate that E3 might not be able to tell the difference between the instances if they are opened and closed fast enough but still knows that they did execute. In scenario 8 several instances of the same application are not immediately identifiable in E3 since they have the same id, but by looking at the execution time in each recording the different instances can be identified.

\paragraph{Measurement resolution}

\begin{itemize}
    \item Scenario 9: Taking several measurements with different durations
\end{itemize}

\paragraph{Expectations}
Scenario 9 is designed to test if E3 uses a per-application sampling rate or if it has a global sampling rate and exactly how frequent it is. The expectation for Scenario 9 is that measurements with durations less than $1$ minute will be inconsistent as we expect a global sampling around once every minute.

\paragraph{Findings}
Scenario 9 confirmed our understanding of E3 as it showed that a new snapshot was taken exactly at the start of a minute.

\paragraph {Recommended usage}
From the experiments, several aspects of E3 has been uncovered, and an understanding of its inner working has become deep enough to utilize for further experimentation. The important aspects that was learned is summarized below:

%% 

\begin{itemize}
    \item E3 takes the measurements in snapshots, which contain every active process on the system
    \item If a program is active during the snapshot it will be included, unaffected by process start, but it has to be active at the end of the snapshot
    \item Change state will only be recorded once every snapshot
    \item The separate measures of the same instance in different states are linked
    \item Each snapshot is taken at the start of every minute
\end{itemize} 

Taking these discoveries into account a process for recommended usage can be created. Our process for using E3 is the following: Await the start of a new snapshot and then execute the program repeatedly until another snapshot is taken. The energy for the test case is then calculated by going through the measurement experiment id and summing each of their energy usage to get the dynamic energy usage for the whole test case. There must be some time between the snapshot for the next measurement to not include the same process twice, our findings suggest that $2$ minutes is adequate for isolating the measurements.    