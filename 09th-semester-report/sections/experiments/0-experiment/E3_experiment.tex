\subsection{E3 Experiments}\label[subsec]{sec:E3Experiments}

The following experiments with E3 address some of the discoveries made in the initial testing, which contradicts the statements in the available sources\cite[]{E3Doc,E3Video,E3WinHec}. 

This experiment was conducted based on observations where there were deviations from the expected output in the log file according to the sources and the actual output. In the output log file for E3, each row represents the measurement of one application in a given state. It is claimed that an application needs to run for $1-5$ minutes before it is added to the file. Based on observations this is not true, as the lowest measurements were observed to be as low as  $0$ and $1.554$ seconds. The highest value found in the test is $902$ minutes, which also does not match the claim. Another aspect is that E3 counts the same application as different measurements depending on its status.
Example \texttt{App.Status=Focus}, \texttt{App.Status=Visible}, \texttt{App.Status=Minimized} and \texttt{App.Status=NotUnique} will all be counted as different measurements in E3. Where \texttt{App.Status=Focus} is when the application is in focus. \texttt{App.Status=Visible} is when the application is not in focus, but still visible on the screen. \texttt{App.Status=Minimized} is when the application is minimized. \texttt{App.Status=NotUnique} are background processes which do not have a user interface. This has been concluded by looking at the processes with this status and the frequency of their repetition over a period. 
Because of the uncertainty about how E3 functions it was deemed necessary to conduct some further experiments to determine when a process is included in the measurements and when it is not. To carry out these experiments different scenarios were formulated, which will be covered now.

% \begin{itemize}
%     \item Order of measurement
%     \begin{itemize}
%         \item Scenario 1: A process is started before the measurements and ended after the measurements
%         \item Scenario 2: A process is started during the measurements and ended after the measurements
%         \item Scenario 3: A process is started before the measurements and ended during the measurements
%     \end{itemize}
%     \item State Change
%     \begin{itemize}
%         \item Scenario 4: Several process "state" are swapped between with increasing intervals
%         \item Scenario 5: Several process "state" are swapped between with a fixed interval
%         \item Scenario 6: Change The "state" of a single process during measurements
%     \end{itemize}
%     \item Instances
%     \begin{itemize}
%         \item Scenario 7: A process opened and restarted several times during the measurements
%         \item Scenario 8: Several instances of the same application
%     \end{itemize}
%     \item Measurement resolution
%     \begin{itemize}
%         \item Scenario 9: Taking several measurements with different durations
%     \end{itemize}
% \end{itemize}

\paragraph{Order of measurement}

\begin{itemize}
    \item Scenario 1: A process is started before the measurements and ended after the measurements
    \item Scenario 2: A process is started during the measurements and ended after the measurements
    \item Scenario 3: A process is started before the measurements and ended during the measurements
\end{itemize}

\paragraph {Expectations}
The first three scenarios are designed to see when measurements are recorded by E3. The initial expectation is that E3 uses the start and exit timestamp for the measured process. Because of this, expectations are that scenarios 1 and 2 would not record a process but that scenario 3 would.

\paragraph{Findings}
Scenarios 1 and 2 were both contrary to our expectations, where scenario 1 resulted in several measurements of the process, and similar results were obtained from scenario 2. In scenario 3, the process is not found, which is also contrary to our expectations. These results point to E3 instead of using the start and end times instead take several snapshots of the process over some time. Exactly how frequent these snapshots are will be tested later. 

\paragraph{State Change}

\begin{itemize}
    \item Scenario 4: Several process "states" are swapped between with increasing intervals
    \item Scenario 5: Several process "states" are swapped between with a fixed interval
    \item Scenario 6: Change The "state" of a single process during measurements
\end{itemize}

\paragraph{Expectations}
Scenarios 4 and 5 are meant to see how E3 handles changes in the process state during the measurements. Scenario 4 attempts to see how granular a new state is measured. While scenario 5 uses the lowest measurement found in scenario 4 to check for consistency. The expectation is that each state change will be measured, until a certain threshold where the change will no longer be registered as the swapping is more frequent than E3's sampling. Scenario 6 is designed to test something similar to scenario 4 but instead uses the same process to see how that changes the results.
\paragraph{Findings}
For scenarios 4 and 5, swapping between the states is only recorded at the first occurrence, and then the results for further state changes are aggregated together. The speed of the change does not hinder E3's measurements contrary to our expectations and does not create a measurement for each state change. Scenario 6 had the same results as 4 and 5 but provided some insight. Each application instance has the same id in E3 and cannot be differentiated based on it, but the execution time for each instance is carried over from state to state, so using this, two identical processes, can be identified since the time reported by E3 is the total execution at the time of collection.

\paragraph{Instances}

\begin{itemize}
    \item Scenario 7: A process opened and restarted several times during the measurements
    \item Scenario 8: Several instances of the same application
\end{itemize}

\paragraph{Expectations}
Scenario 7 is designed to see how E3 looks at multiple starts and shutdowns during measurements. The expectation is that each instance will be a separate measurement. Scenario 8 is designed to see how E3 handles multiple concurrent instances of the process with the same process id the expectation here is that the measurements will be merged into one big measurement. 
\paragraph{Findings}
In scenario 7 a process opening and closing within 1 second seem to not always get picked up by E3, but the instances that do get picked up seem to be aggregates of the instances recorded. This could indicate that E3 might not be able to tell the difference between the instances if they are opened and closed fast enough but still knows that they did execute. In scenario 8 several instances of the same application are not immediately identifiable in E3 since they have the same id, but by looking at the execution time in each recording the different instances can be identified.

\paragraph{Measurement resolution}

\begin{itemize}
    \item Scenario 9: Taking several measurements with different durations
\end{itemize}

\paragraph{Expectations}
Scenario 9 is designed to test if E3 uses a per-application sampling rate or if it has a global sampling rate and exactly how frequent it is. The expectation for Scenario 9 is that measurements with durations less than $1$ minute will be inconsistent as we expect a global sampling around once every minute.

\paragraph{Findings}
Scenario 9 confirmed our understanding of E3 as it showed that a new snapshot was taken exactly at the start of a minute.

\paragraph {Recommended usage}
From the experiments, several aspects of E3 has been uncovered, and an understanding of its inner working has become deep enough to utilize for further experimentation. The important aspects that were learned are summarized below:

%% 

\begin{itemize}
    \item E3 takes the measurements in snapshots, which contain every active process on the system
    \item If a program is active during the snapshot it will be included, unaffected by process start, but it has to be active at the end of the snapshot
    \item Change state will only be recorded once every snapshot
    \item The separate measures of the same instance in different states are linked
    \item Each snapshot is taken at the start of every minute
\end{itemize} 

Taking these discoveries into account a process for recommended usage can be created. Our process for using E3 is the following: Await the start of a new snapshot and then execute the program repeatedly until another snapshot is taken. Because a snapshot is taken each minute, the test case will be executed for one minute. The energy for the test case is then calculated by going through the measurement experiment id and summing each of their energy usages to get the dynamic energy usage for the whole test case. There must be some time between the snapshot for the next measurement to not include the same process twice, our findings suggest that two minutes is adequate for isolating the measurements.    