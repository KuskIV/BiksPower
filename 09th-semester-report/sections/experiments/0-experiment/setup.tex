\subsection{Experiment Representation}\label[]{subsec:exp_rep}

After the introduction to the framework in \cref*{subsec:framework}, the format of the saved data in \cref*{subsec:sql}, the different DUT's, test cases and measuring instruments in \cref*{sec:hardware}, \cref*{subsec:test_cases}, \cref*{sec:measuring_instruments} respectively, the experiments can now be introduced. This will also be done in the context of the different terms related to the experiments found in \cref*{tab:TerminologyAlert}.

When running the experiments, there will be switched between different experiment setups. These setups will be different DUT's, test cases and measurement instruments. Here the same computer running either windows or linux will count as two DUT's. In the framework, a test case will run for a minimum period of time, before switching to the next setup, and based on what was found in \cref*{sec:E3Experiments}, this period will be one minute. In this period of one minute, the total energy consumption during this minute will be captured, in addition to how many times the test case executed, which means an average energy consumption per test case run an be calculated. After this one minute of execution, there is a cooldown period, this will however be set to zero minutes. This was choosen as no argument could be found for this value, and no effect of this was expected.

During one minute, the test case will execute $n$ times, and this will be $n$ sampels, according to \cref*{tab:TerminologyAlert}. When the test case has executed for one minuted, this will be called a measurement. When the same setup has executed multiple times, i.e. there are multiple measurements with the same DUT, test case and measurement instrument, it is a test case measurement.

In addition to this, R3 validation will also be utilized 

