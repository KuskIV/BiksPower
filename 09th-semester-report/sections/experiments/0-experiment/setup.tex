\subsection{Experiment Representation}\label[subsec]{subsec:exp_rep}

After the introduction to the framework in \cref{subsec:framework}, the format of the data in \cref{subsec:sql}, the different DUTs, test cases and measuring instruments in \cref{sec:hardware}, \cref{subsec:test_cases}, \cref{sec:measuring_instruments} respectively, the experiments can now be introduced. This will be done in the context of the different terms related to the experiments found in \cref{tab:TerminologyAlert}.

\paragraph{}
When running the experiments, it will be done based on different experiment configurations. These configurations will be different DUTs, test cases and measuring instruments. Here the same computer running either Windows or Linux will count as different DUTs. In the framework, a test case will run for a minimum period of time, before switching to the next configuration, and based on what was found in \cref{sec:E3Experiments}, this period will be one minute, since that is required in order to use E3. In this period of one minute, the total energy consumption will be captured, in addition to how many times the test case was executed, which means an average energy consumption per test case run can be calculated. After this one minute of execution, there is an optional cooldown period, which is required for E3. However since E3 is not used in the initial experiment a cooldown period is not used. This was chosen as no argument was found as to why it should be used or what the value should be. However, there is an indirect cooldown period of about $~30$ seconds as the measurements are sent to the database and the setup phases is entered to ensure background processes and WiFi is turned off.

During one minute, the test case will execute $n$ times, resulting in $n$ samples, according to \cref{tab:TerminologyAlert}. When the test case has been executed for one minute, this will be called a measurement. When the same configuration has been executed multiple times, it is a test case measurement according to \cref{tab:TerminologyAlert}.

\paragraph*{}
As was introduced in \cref{subsec:cock}, Cochran's formula will be used to calculate how many measurements are required, in order for the results to be within some pre-defined standard deviation. Before this value is chosen, an arbitrary value of 120 is used. Meaning that we run for one minute a 120 times. 
%In addition to this, a value is also needed for how many measurements $m$ should be made for each measuring instrument between restarts. On start up, one test case is chosen to be executed until the next restart. For this test case, each measuring instrument needs to make $m$ measurement before the DUT restarts. This is done to ensure there will be multiple measurements where each test case is run with each measuring instrument for the $1 \dots m$'th time after a restart. Here $m=30$ was chosen, as this would result in each configuration having four measurement, where it was the first measuring instrument on a specific test case to run.
If based on Cochran's formula the values are not deemed sufficient, additional experiments will be executed.

 %for each $m$ times a test case is executed after a restart.

%For this test case, we need to take $m$ measurements with each instrument before the device under test (DUT) is restarted. This is to ensure that there are multiple measurements for each run after a restart. We chose $m=30$ so that each run after a restart would have four measurements. If the sample size is not large enough according to Cochran's formula, we will conduct additional experiments.

\paragraph*{}
In addition to this, R3 validation\cite*[]{Bokhari2020r3} will also be utilized in an adapted manner. The idea from the original article was to compare different implementations of the same test case, to see which version performed best. The idea was to switch between which implementation was the first to run after a restart, to make the comparison fairer. In our work, it is a bit different, where different measuring instruments will be compared. When running the experiment, with three measuring instruments $A, B$ and $C$ and test cast $T$ and DUT $D$, $D$ will execute $T$ $m$ times in a row. The first time $A$ will measure the energy consumption, then $B$ etc. When $D$ has reached a state where it needs to restart (after $m$ measurements), it will save which measuring instrument was used first. When $D$ has restarted, it will again execute $T$, but this time $B$ will measure first, then $C$ and so on. This will continue until the measuring instruments have 120 measurements each.\todo{I have moved part about m to here instead of cochrans} Therefore, a value is also needed for how many measurements $m$ should be made for each measuring instrument between restarts. This is done to ensure there will be multiple measurements where a test case is run with a measuring instrument for the first time after a restart. Here $m=30$ was chosen, as this would result in each configuration having four measurement, where it was the first measuring instrument on a specific test case to run, which is important to avoid measurements for specific configurations being affected by variables in changing system states.
