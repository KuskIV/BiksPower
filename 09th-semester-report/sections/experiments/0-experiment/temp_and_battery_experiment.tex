\subsection{Temperature and Battery Experiment}\label[subsec]{subsec:TempBat}

One thing to consider before running the experiments is to determine some of the values for the configurations. This includes at what range of temperature and what range of battery percentage the system can be within without degrading performance. The reason why these ranges are chosen is first of all based on the work by Bokhari et al.\cite*[]{Bokhari2020r3} where experiments were only run when the battery percentage was above a certain limit, as the DUT would otherwise enter power saving mode. Because of this, the hypothesis will be that a similar observation can be made on laptops. Another parameter able to make the DUT underperform is the temperature, as was noted in the work by Lindholt et al.\cite*[]{Lindholt2022}. For this experiment, some expectations will first be presented, before the results will be discussed.

\paragraph{Expectations:} Based on these two parameters, the assumption would be that there is some lower limit on the battery where the CPU is underclocked, and some upper limit for the temperature where the CPU will thermal throttle, resulting in a much worse performance. To find these values, a test case will be executed on all DUTs, where the battery and temperature will be measured before and after the measurement.

\paragraph{Results:} When running this set of experiments to determine the temperature and battery, some different issues came up. These were as following:

\begin{itemize}
    \item Intel Power Gadget: This measurement instrument would sometimes cause the framework to crash. This occurred more frequently on some devices, but without any pattern. The error could not be caught within C\#, despite sources claiming this\cite*[]{cpp_exceptions}. The error could not be replicated inside and IDE, and would close the window down immediatly, meaning it could not be read. And since it could not be caught, it could not be logged either. In order to solve this, different versions of Intel Power Gadget was tested, but without success, but it was most likely related to Intel Power Gadget accessing some memory they were no supposed to.
    \item Start the framework automatically: When reading certain values and disabeling the WIFI, the framwork was required to run in admin. It was however not possible to start the framework automatically as admin, despite disabeling notifications about changes on the computer. This was most likely due to security reasons.
    \item Battery level: One of the DUT's were unable to charge fully up. This was most likely due to the DUT being an older device.
    \item No battery: During the experiments, there were issues with the battery powered DUT's running out of power, and shutting down. This could be related to the DUT's not being new, and thus shutting down before reaching 0\% battery.
\end{itemize}

In order to solve this, a few measures were taken:

\begin{itemize}
    \item Background script: In order to handle cases where the framework crashed without any warning, a script was introduced. This script would run every 20 minutes, to ensure the framework was running. If it was not, it would restart the computer.
    \item AutoHotKey: In order to solve the issue of the framwork not being able to run on startup, a AutoHotKey script was used. This would execute the framework a few minutes after the DUT was started, to ensure it was in a stable condition. After it had executed once, it would shut down.
    \item Batterycharge: Because of the issues both with low and high battery levels, this experiment is performed based where the DUT will not go below 40\% battery, and not above 80\%. 40 was choosen as the issues had occured at both 10 and 20\%, and we needed to make sure it would not crash during the night. 80\% was choosen as the oldest DUT could charge just above 90\%, and we needed to make sure it would not charge forever. These limits were set for both DUT's, to make them more comparable.
\end{itemize}

A thing to note is that both scripts descibed are used in all experiments. Following this, the experiments were conducted. To ensure enough data was collected, each test case measurement was performed 120 times


                            \begin{figure}
                                \centering
                                \begin{tikzpicture}[]
                                    \pgfplotsset{%
                                        width=.7\textwidth,
                                        height=.15\textheight
                                    }
                                    \begin{axis}[xlabel={Average dynamic energy consumption (Watts)}, title={Dram - FannkuchRedux - Dynamic Energy - without outliers}, ytick={1, 2, 3},
                                    yticklabels={
                                        IntelPowerGadget , HardwareMonitor , RAPL 
                                        },
                                        xmin=0,xmax=10,
                                        ]
                                    
                                    \addplot+ [boxplot prepared={
                                    lower whisker=0.20538501792059694,
                                    lower quartile=0.21746331766250407,
                                    median=0.22181397206089515,
                                    upper quartile=0.22689881660268849,
                                    upper whisker=0.2574096342559279},
                                    ] table[row sep=\\,y index=0] {\\};
                                    
                                    \addplot+ [boxplot prepared={
                                    lower whisker=0.20011075295825292,
                                    lower quartile=0.21351009682521976,
                                    median=0.21882287282828888,
                                    upper quartile=0.22279414070357745,
                                    upper whisker=0.24916373243628814},
                                    ] table[row sep=\\,y index=0] {\\};
                                    
                                    \addplot+ [boxplot prepared={
                                    lower whisker=-36.9590502759173,
                                    lower quartile=0.42743760995280056,
                                    median=36.77656912882534,
                                    upper quartile=76.49716528340782,
                                    upper whisker=116.30089979379153},
                                    ] table[row sep=\\,y index=0] {\\};
                                    
                                    \end{axis}
                                \end{tikzpicture}
                            \caption{A comparison of of Dram dynamic energy consumption for test case FannkuchRedux for the SurfaceBook (without outliers)} \label{fig:FannkuchRedux_Dram_comparison_dynamic_energy_without_outliers_SurfaceBook_avg_watts}
                            \end{figure}
                            

\paragraph{Battery level:}


                            \begin{figure}
                                \centering
                                \begin{tikzpicture}[]
                                    \pgfplotsset{%
                                        width=.7\textwidth,
                                        height=.15\textheight
                                    }
                                    \begin{axis}[xlabel={Average dynamic energy consumption (Watts)}, title={Dram - FannkuchRedux - Dynamic Energy - without outliers}, ytick={1, 2, 3},
                                    yticklabels={
                                        IntelPowerGadget , HardwareMonitor , RAPL 
                                        },
                                        xmin=0,xmax=10,
                                        ]
                                    
                                    \addplot+ [boxplot prepared={
                                    lower whisker=0.20538501792059694,
                                    lower quartile=0.21746331766250407,
                                    median=0.22181397206089515,
                                    upper quartile=0.22689881660268849,
                                    upper whisker=0.2574096342559279},
                                    ] table[row sep=\\,y index=0] {\\};
                                    
                                    \addplot+ [boxplot prepared={
                                    lower whisker=0.20011075295825292,
                                    lower quartile=0.21351009682521976,
                                    median=0.21882287282828888,
                                    upper quartile=0.22279414070357745,
                                    upper whisker=0.24916373243628814},
                                    ] table[row sep=\\,y index=0] {\\};
                                    
                                    \addplot+ [boxplot prepared={
                                    lower whisker=-36.9590502759173,
                                    lower quartile=0.42743760995280056,
                                    median=36.77656912882534,
                                    upper quartile=76.49716528340782,
                                    upper whisker=116.30089979379153},
                                    ] table[row sep=\\,y index=0] {\\};
                                    
                                    \end{axis}
                                \end{tikzpicture}
                            \caption{A comparison of of Dram dynamic energy consumption for test case FannkuchRedux for the SurfaceBook (without outliers)} \label{fig:FannkuchRedux_Dram_comparison_dynamic_energy_without_outliers_SurfaceBook_avg_watts}
                            \end{figure}
                            


\paragraph{Temperature:}







