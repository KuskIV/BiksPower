\subsection{Temperature and Battery Experiment}\label[subsec]{subsec:TempBat}

One thing to consider before running the experiments is to determine some of the values for the conditions the DUT perform measurements within. This includes at what range of temperature and what range of battery percentage the DUT can be within without degrading performance. The reason why these ranges are chosen is first of all based on the work by Bokhari et al.\cite*[]{Bokhari2020r3} where experiments were only run when the battery percentage was above a certain limit, as the DUT would otherwise enter power saving mode. Because of this, the hypothesis will be that a similar observation can be made on laptops. Another parameter able to make the DUT underperform is the temperature, as was noted in the work by Lindholt et al.\cite*[]{Lindholt2022}. For this experiment, some expectations will first be presented, before the results will be discussed.

\paragraph{Expectations:} Based on these two parameters, the assumption would be that there is some lower limit on the battery where the CPU is underclocked, and some upper limit for the temperature where the CPU will thermal throttle, resulting in a much worse performance. To find these values, a test case will be executed on all DUTs, where the battery and temperature will be measured before and after the measurement.

\paragraph{Issues:} When running this set of experiments to determine the temperature and battery, some different issues came up. These were as following:

\begin{itemize}
    \item Intel Power Gadget: This measurement instrument would sometimes cause the framework to crash. This occurred more frequently on some DUT's, but without any pattern. The errors was caused by the Intel Power Gadget API, implemented in c++, which ment he error could not be caught within C\#, despite sources claiming this\cite[]{cpp_exceptions}. The error could not be replicated inside and IDE, and would close the window down immediatly, which ment we were unable to see or log the specific error. In order to solve this, different versions of Intel Power Gadget was tested, but without success, but it was most likely related to Intel Power Gadget accessing some memory it were no supposed to.
    \item Startup: When reading certain values and disabeling the WIFI, the framwork was required to run as administrator. It was however not possible to start the framework automatically as admin, despite disabeling notifications about changes on the computer. This was most likely due to security reasons.
    \item Battery level: One of the DUT's were unable to charge fully up. This was most likely due to the DUT being an older device.
    \item No battery: During the experiments, there were issues with the battery powered DUT's running out of power, and shutting down. This could be related to the DUT's being older devices, and thus shutting down before reaching 0\% battery.
\end{itemize}

In order to solve this, a few measures were taken:

\begin{itemize}
    \item Background script: In order to handle cases where the framework crashed without any warning, a script was introduced. This script would run every 20 minutes, to ensure the framework was running. If it was not, it would restart the computer. This script was implemented in PowerShell, and would affect the energy measurements, but as the energy consumption of the script is consistent across all measurements, it should be removed from the energy measurements when using dynamic energy.
    \item AutoHotKey: In order to solve the issue of the framework not being able to run on startup, a AutoHotKey script was used. This would execute the framework a few minutes after the DUT was started, to ensure it was in a stable condition. After it had executed once, it would shut down, thus not affecting the energy measurements.
    \item Battery level: Because of the issues both with low and high battery levels, some limits were set before the experiments was run, to ensure the DUT would not stop prematurely because a DUT would charge forever, or it would shut down. These values were set similarly across all DUT's to make them more comparable. The DUT facing issues with charging to 100\%, could charge to almost 90\%. Because of this, 80\% was chosen, to ensure it would work. For the lower limit, some different values were tested, where the issues occurred at both 5\%, 10\%, 20\% and 30\%. Because of this, 40\% was chosen as the battery level the DUT could not go below.
\end{itemize}

A thing to note is that both scripts described are used in all experiments. Following this, the experiments were conducted.


                            \begin{figure}
                                \centering
                                \begin{tikzpicture}[]
                                    \pgfplotsset{%
                                        width=.85\textwidth,
                                        height=.15\textheight
                                    }
                                    \begin{axis}[xlabel={Average energy consumption (Watts)}, title={Cores - FannkuchRedux - Energy - without outliers}, ytick={},
                                    yticklabels={
                                        
                                        },
                                        xmin=0,xmax=20,
                                        ]
                                    
                                    \end{axis}
                                \end{tikzpicture}
                            \caption{A comparison of of Cores energy consumption for test case FannkuchRedux for the Surface4Pro,  experiment \#2 (without outliers)} \label{fig:FannkuchRedux_Cores_comparison_energy_without_outliers_Surface4Pro_avg_watts_exp2}
                            \end{figure}
                            

\paragraph{Battery level:} When considering the impact of the battery level on the performance, FannkuchRedux was picked arbitrarily to illustrate this, and can be seen in \cref{fig:FannkuchRedux_Cores_charge}. A similar graph for all other test cases can be found in \cref{app:charge}. When looking at the graph, a slight decrease in energy consumption can be observed when the battery level increases, like Intel Power Gadget in Surface 4 Pro. The overall conclusion for the impact of energy is however concluded to be negligible, where the expected decrease in performance would most likely occur at lower battery levels than 40\%, which unfortunately was not possible to test. Based on this, the experiments will be executed with a lower limit of 40\% and an upper limit of 80\%.


                            \begin{figure}
                                \centering
                                \begin{tikzpicture}[]
                                    \pgfplotsset{%
                                        width=.85\textwidth,
                                        height=.15\textheight
                                    }
                                    \begin{axis}[xlabel={Average energy consumption (Watts)}, title={Cores - FannkuchRedux - Energy - without outliers}, ytick={},
                                    yticklabels={
                                        
                                        },
                                        xmin=0,xmax=20,
                                        ]
                                    
                                    \end{axis}
                                \end{tikzpicture}
                            \caption{A comparison of of Cores energy consumption for test case FannkuchRedux for the Surface4Pro,  experiment \#2 (without outliers)} \label{fig:FannkuchRedux_Cores_comparison_energy_without_outliers_Surface4Pro_avg_watts_exp2}
                            \end{figure}
                            


\paragraph{Temperature:} When looking at the impact of the temperature on the energy performance, FannkuchRedux was choosen again to illustrate this, as can be observed in \cref*{fig:FannkuchRedux_Cores_temperature}. The other test cases can be found in \cref{app:temperature}. When considering the graph in \cref{fig:FannkuchRedux_Cores_temperature}, a few things can be observed. The temperatures for example never reach a temperature above 65 on any of the DUTs, and the highest temperature for the workstation is even lower. This is most likely due to the workstation having full fan speed during all experiments, where this was not possible in the laptops. Based on the expectations, a increase in power was expected to be observed when the temperature increased, which did not occur, most likely due to the temperatures never rising above 65. When looking at the results, no impact can be observed, which is why no limits are set going forward.





