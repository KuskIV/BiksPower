\subsection{Temperature and Battery Experiment}

One thing to consider before running the experiments, is to determine some of the values for the configurations. This includes at what range of temperature and what range of battery percentage the system should be within without a degrading performance. The reason why these ranges are chosen is first of all based on the work by Bokhari et al.\cite*[]{Bokhari2020r3} where experiments were only run when the battery percentage was above a certain limit, as phones cpu will enter power saving mode otherwise. Because of this, the hypothesis will be that a similar observation can be made on laptops. Another parameter able to make the DUT underperform is the temperature, as was noted in the work by Lindholt et al.\cite*[]{Lindholt2022}. 

Based on these two parameters, the assumption would be that there is some lower limit on the battery where the CPU is under clocked, and some upper limit for the temperature where the CPU will thermal throttle, resulting in a much worse performance. In order to find these values, a test case will be executed on all DUTs, where the battery will decrease, and the temperature will increase during the experiments. These values will then be plottet for each DUT, where the X-axis will represent the temperature/battery and the Y-axis will be the energy consumption in joules.