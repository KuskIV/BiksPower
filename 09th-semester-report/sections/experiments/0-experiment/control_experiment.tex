\subsection{Control Experiment}\label[subsec]{subsec:control_experiment}

The last part of the initial experiments will be a control experiment. This will be done before the rest of the experiments can be conducted, to see how the energy measurements in our work, compared to other works. This control experiment will compare the energy measurements from the framework implemented in our work, to the results in the work by Pereira et al.\cite[]{Pereira2017} on the same test case. It should however be noted that the exact implementation of the different test cases might not be the same, as our work and the work by Pereira et al.\cite[]{Pereira2017} both picked the fastest implementation of each test case at the time of writing. Given the work by Pereira et al.\cite[]{Pereira2017} is from 2017, these will most likely be different when compared to the fastest implementation in 2022. Our work and the work by Pereira et al.\cite[]{Pereira2017} also differs on other areas, including:

\begin{itemize}
    \item Temperature: The work by Pereira et al.\cite[]{Pereira2017} does not consider the temperature when running the test cases, which could impact the results.
    \item Software: When comparing RAPL in 2017 to RAPL in 2022, the performance of the measuring instrument could have changed as a result of updates to the software. The impact of these updates to RAPL are unknown.
    \item Hardware: The DUT in the work by Pereira et al.\cite[]{Pereira2017} had 16GB of ram and a Haswell Intel(R) Core(TM) i5-4460 CPU, which is an older CPU also compared to all DUT's used in our work.
\end{itemize}

When considering the results in the work by Pereira et al.\cite[]{Pereira2017}, the energy consumption of a given test case is given in joules and a duration in seconds. When comparing the energy consumption against the measurements in our work, it is done so by calculating the average energy consumption in joules, given the time by Pereira\cite[]{Pereira2017}, e.i. if Fasta was reported to use $50$ joules in $10$ seconds and the measurements from our work were $25$ joules for $5$ seconds, the energy consumption will be scaled to have the same time, i.e. $25J*2=50$. Following this, it is time to present the expectations.

\paragraph*{Expectations:} The CPUs used in our work are all newer than the one used in the work by Pereira et al.\cite[]{Pereira2017}, However, only the workstation has a higher clock frequency so we expect it to be able to execute the test case more times during the same period. Because of this, higher energy consumption is expected, at least for the workstation. For the two Surface devices, both CPUs have a lower clock frequency of 2.4 and 2.2GHz against 3.2GHz. Because of this, lower energy consumption is expected for both surface devices.

\begin{table}[ht]
    \centering
    \begin{tabular}{|| c | c | c | c | c | c ||}
        \hline
        \textbf{} & \textbf{Time} & \textbf{Pereira} & \textbf{Workstation} & \textbf{Surface Pro 4} & \textbf{Surface Book} \\ [0.5ex] \hline\hline
        Binary Trees & $10797$ms & $189.74$j & $613.87$j ($3.24$) & $86.68$j ($0.45$) & $55.69$j ($0.29$) \\
        FannkuchRedux & $10840$ms & $399.33$j & $551.34$j ($1.38$) & $78.78$j (0.19) & $56.09$j ($0.14$) \\
        Fasta & $1549$ms & $45.35$j & $58.28$j ($1.29$) & $12.10$j ($0.26$) & $8.40$j ($0.18$)  \\ \hline
    \end{tabular}
    \caption{A comparison of the RAPL energy measurements reported by our work against the work by Pereira et al.\cite[]{Pereira2017}}
    \label{tab:sanity_check}
\end{table}


\paragraph*{Results:} The results from the control experiments can be seen in \cref{tab:sanity_check}, where RAPL measurements from our work are presented. In \cref{tab:sanity_check} there are five columns, where the two first shows the time and energy consumption as reported by Pereira et al.\cite[]{Pereira2017}, and the last three are the energy consumption from the DUTs in our work, over the time reported by Pereira et al.\cite[]{Pereira2017}. When looking at the three columns for the DUTs of our work, two values can be seen, where the first is the energy consumption, and the second is how the energy consumption compares against the energy consumption of our work. An example of this could be the number $3.24$ for Binary Trees on the workstations, which means that the energy consumption in our work is $3.24$ times higher compared to the number reported by Pereira et al.\cite[]{Pereira2017}. As a result, a few observations can be made. This is first of all how the workstation always has a higher energy consumption and the surface devices have a lower energy consumption, where this is as expected. When considering each test case, Binary Trees is for all DUTs the test case deviating the most from the measurements made by Pereira et al.\cite[]{Pereira2017}. This could be related to what the test case targets, where the Binary Trees target memory as can be seen in \cref{tab:benchmarks}, but exactly why, is a subject for potential future work. When comparing the energy measurements, it is difficult to say if the measurements made in our work are more or less correct compared to the numbers provided by Pereira et al.\cite[]{Pereira2017} as the hardware and software have changed a lot since 2017.