\subsection*{Control Experiment}\label[subsec]{subsec:control_experiment}

The last part of the initial experiments, will be a control experiment. This will be done before the rest of the experiments can be conducted, to see how the energy measurements in this work, compares to other works. This control experiment will compare the energy measurements from the framework implemented in this work, to the results in the work by Pereira et al.\cite*[]{Pereira2017} on the same test case. It should however be noted that the exact implementation of the different test cases might not be the same, as this work and the work by Pereira et al.\cite*[]{Pereira2017} both picked the fastest implementation of each test case at the time of writing. Given the work by Pereira et al.\cite*[]{Pereira2017} is from 2017, these will most likely be different when comparing against the fastest implementation in 2022. This work and the work by Pereira et al.\cite*[]{Pereira2017} also differs on another areas, including:

\begin{itemize}
    \item Temperature: The work by Pereira et al.\cite*[]{Pereira2017} does not consider the temperature when running the test cases, which could impact the results.
    \item Software: Since 2017, the where work by Pereira et al.\cite*[]{Pereira2017} is from, most of the software used will have received updated, this includes both RAPL the Linux. The exact effect of this, is however unknown.
    \item Hardware: The DUT in the work by Pereira et al.\cite*[]{Pereira2017} had 16GB of ram and a Haswell Intel(R) Core(TM) i5-4460 CPU, which is an older CPU also compared to all DUT's used in this work.
\end{itemize}

When considering the results in work by Pereira et al.\cite*[]{Pereira2017}, they are presented in a table where the energy in joules and the time is present. When comparing the energy consumption against the measurements in this work, it is done so by calculating the average energy consumption in joules given the time by Pereira\cite*[]{Pereira2017}, e.i. if Fasta was reported use $50$ joules in $10$ seconds, and the measurements from this work were $25$ joules for $5$ seconds, the energy consumption will be scaled to have the same time, i.e. $25J*2=50$. Following this, it is time to present the expectations.

\paragraph*{Expectations:} Given the CPU used in this work for all DUT's are newer than the one used in the work by Pereira et al.\cite*[]{Pereira2017}, the DUT's are expected to be able to execute the test case more times during the same period of time. Because of this, a higher energy consumption is expected, at least for the workstation. For the two Surface devices, both CPU's has a lower clock frequency of 2.4 and 2.2GHz against 3.2GHz. Because of this, a lower energy consumption is expected.

\begin{table}[ht]
    \centering
    \begin{tabular}{|| c | c | c | c | c | c ||}
        \hline
        \textbf{} & \textbf{Time} & \textbf{Pereira} & \textbf{Workstation} & \textbf{Surface Pro 4} & \textbf{Surface Book} \\ [0.5ex] \hline\hline
        Binary Trees & $10797$ms & $189.74$j & $613.87$j ($3.24$) & $86.68$j ($0.45$) & $55.69$j ($0.29$) \\
        FannkuchRedux & $10840$ms & $399.33$j & $551.34$j ($1.38$) & $78.78$j (0.19) & $56.09$j ($0.14$) \\
        Fasta & $1549$ms & $45.35$j & $58.28$j ($1.29$) & $12.10$j ($0.26$) & $8.40$j ($0.18$)  \\ \hline
    \end{tabular}
    \caption{A comparison of the RAPL energy measurements reported by this work against the work by Pereira et al.\cite*[]{Pereira2017}}
    \label{tab:sanity_check}
\end{table}


\paragraph*{Results:} The results from the control experiments can be seen in \cref{tab:sanity_check}, where RAPL measurements from this work is presented. In this table each cell for the different DUT's in this work has two values, where one is the energy consumption in joules, the other represents the relation between the DUT and the DUT used in the work by Pereira et al.\cite*[]{Pereira2017}, where the number $3.24$ for Binary Trees on the workstations means that the energy consumption in this work is $3.24$ times higher compared the the number reported by Pereira et al.\cite*[]{Pereira2017}. In the result, a few observations can be made. This is first of all how the workstations always has a higher energy consumption and the surface devices has a lower energy consumption, where this is as expected. When considering each test case, Binary Trees is for all DUT's the test case deviation the most from the measurements made by Pereira et al.\cite*[]{Pereira2017}. This could be related to what the test case targets, where the Binary Trees target memory as can be seen in \cref*{tab:benchmarks}, but the exactly why, is a subject for a potential future work. When comparing the energy measurements, it is difficult to say if the measurements made in this work are more or less correct compared to the numbers provided by Pereira et al.\cite*[]{Pereira2017} as the hardware and software has changed a lot since 2017.