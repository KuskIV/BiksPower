\subsection{Anomalies Detection}\label[]{subsec:anomalies_detection}

Anomaly detection was found in \cref*{subsec:anomalies} to be an important tool to avoid corrupted results. For this, DBScan was found to be the algorithm of choice in order to find anomalies in the data.

For this, an example is shown in \cref{fig:anomaly_detection_example} for a test case measurement for the workstation, on test case fasta, measured by the HardwareMonitor, where the same operation was conducted for all other test case measurements.

The algorithm is described in the work by Ester et al.\cite{ester1996density}. In order to use this algorithm, the value of two parameters needs to be decided, this being \texttt{Eps} and \texttt{MinPts}. In order to do this, a \texttt{k-dist} function is defined, where $k$ represents the distance from a point to its $k$'th neighbor. This function is applied to all measurements, and is plotted based on descending order. This is called the sortd k-dist graph, an can be seen in \cref{fig:4-dist}. The next step is to choose a threshold point, which is the first point in the first "valley" on the graph. All points with a higher k-dist value will then be considered to be noise. The \texttt{MinPts} will in this case be set to $k$, and \texttt{Eps}, will be the y value of \texttt{k-dist(p)}. When finding this "valley", it is noted in the work by Ester et al.\cite{ester1996density} to be very difficult to do this in an automatically manner, but it is however easy for a user to see this. Because of this, this process was done manually for all cases. In this example the first valley is marked in \cref{fig:4-dist} with a dotted line, where the \texttt{Eps} parameter is set to $0.00105$. Based on this, the outliers can be found, and removed. This is illustrated in \cref{fig:outliers}, where it can be seen how two outliers (marked in red) are found.

\begin{figure}[h]  
    \centering 
    \begin{subfigure}[b]{0.4\linewidth}
    \begin{tikzpicture}
        \pgfplotsset{%
        width=1.1\linewidth,
        % height=1\textheight
        }
        \begin{axis}[%
        scatter/classes={%
            a={mark=o,draw=black},
            b={mark=o,draw=red}}]
        \addplot[scatter,only marks,%
            scatter src=explicit symbolic]%
        table[meta=label] {
        x y label
        0.01327654701428432 0.30597463760729066 a
        0.013287214504152108 0.3059995922112346 a
        0.0132497346021537 0.30596372124789123 a
        0.013158449009557673 0.30580162704335556 a
        0.01313797768053797 0.30577598940559264 a
        0.013194376966146148 0.30571795237416227 a
        0.01317422646947403 0.3058300898661178 a
        0.013201657630442327 0.30586281213220495 a
        0.013220625039478722 0.3058242782693748 a
        0.013111049404297848 0.3066686419551269 a
        0.01312265426964118 0.30576849611613394 a
        0.013152599919953963 0.3057226231278105 a
        0.013200018666587876 0.3057086112950043 a
        0.013137469913893819 0.3061576279311716 a
        0.013167521215536533 0.3061826119775994 a
        0.013181599932276147 0.3061342090903991 a
        0.013269674132419307 0.3060261109383523 a
        0.01311017363450352 0.30582499541499397 a
        0.013211053243522152 0.3056063205371642 a
        0.013188753232014442 0.30565837911598326 a
        0.01345422403020393 0.30589200454524607 a
        0.013497149786396233 0.30585388220908905 a
        0.013226068012040314 0.3055277046869195 a
        0.013181406587386865 0.30556658420629346 a
        0.01306641583877133 0.30664984368934023 a
        0.013218005320030996 0.3059559243252059 a
        0.013108030686125585 0.30660599003023614 a
        0.013050415923486849 0.30594345007542706 a
        0.01314818663809047 0.306654543039727 a
        0.013309818005141062 0.30682265393009195 a
        0.013297092067359819 0.3068698700847362 a
        0.013184565376598055 0.30549971755868927 a
        0.01343656295037907 0.3061697038317752 a
        0.013401106220029024 0.30616585067319463 a
        0.013433911438417769 0.30610257299325905 a
        0.013294815371253803 0.30570436305862225 a
        0.013092400605551505 0.30589668069667664 a
        0.013271570412161964 0.3058094161037585 a
        0.013313110996657983 0.3057479577077434 a
        0.013496671666510841 0.3058269738249768 a
        0.013444013781810277 0.3057533437235787 a
        0.01337261019607976 0.3068128385975672 a
        0.013093291984656643 0.3059766154598926 a
        0.013157021043144632 0.3060280885988836 a
        0.013391326336925628 0.3057735897109385 a
        0.01324913552760757 0.30616189640751545 a
        0.013346689323381145 0.3060592928700706 a
        0.013236915000708909 0.3051347999349382 a
        0.0131490784675332 0.30625133913540314 a
        0.013403327460909524 0.30627477590086066 a
        0.013571995634517737 0.3058986588398353 a
        0.013518267968393399 0.30572998630018383 a
        0.01316795231659627 0.3060472277167153 a
        0.013350905036135742 0.3056599354372244 a
        0.013246347686254153 0.3066984105899363 a
        0.013276364511579115 0.3066138201208456 a
        0.013299190336140692 0.3055883610596404 a
        0.01348382688818378 0.30624665221272795 a
        0.013264491226657547 0.3067043358758347 a
        0.013215459756840872 0.3051321280261867 a
        0.01327922032662637 0.30545308373742563 a
        0.013482687699285891 0.30554325530734505 a
        0.013294572104089282 0.3062415508225992 a
        0.013340509808413392 0.30632322325182676 a
        0.013414588545749902 0.30589979821039115 a
        0.013582545043481276 0.30589979821039115 a
        0.013135649231639781 0.3064885866429771 a
        0.013044427510689101 0.3064514482322072 a
        0.013390966133024594 0.3055494759987581 a
        0.013335815135352123 0.30692031615865195 a
        0.01342418175669348 0.3068720318578461 a
        0.013217550976775332 0.30658134629821615 a
        0.013173354014287096 0.3050510696681742 a
        0.01369835175249967 0.30627831517080933 a
        0.013614464384301946 0.30634198118021677 a
        0.013627092807553247 0.30623727879774115 a
        0.013141867668741525 0.30529195781079954 a
        0.013485781246519469 0.3056490415213462 a
        0.013320834605748389 0.3051425550624965 a
        0.013179177472184738 0.3064823276544256 a
        0.013065531362449931 0.306236864121452 a
        0.013157375180113808 0.3053960146715436 a
        0.013275402611545174 0.3064056757860351 a
        0.013316138143520681 0.30650736514258 a
        0.013211263419468915 0.306315408126298 a
        0.013795925127810109 0.30622238999357254 a
        0.01348044625275882 0.3054803582332587 a
        0.013786089422695706 0.30694188085754753 a
        0.013372137872438886 0.3066647595792829 a
        0.01380992516526564 0.30632634941369774 a
        0.013313621379928108 0.30544220456095006 a
        0.013584278123237004 0.3063587649910692 a
        0.013060007456381894 0.3055848261327713 a
        0.013651242810114615 0.30646124630534327 a
        0.013480012690027934 0.3053525195957253 a
        0.013897221754789638 0.3069689671570357 a
        0.013068652976625036 0.3063705351902791 a
        0.013522374980954304 0.30732960973936546 a
        0.013406118632536861 0.30726196935109074 a
        0.013247543424122377 0.30529506295896985 a
        0.013871734254857921 0.3078749647408775 a
        0.013375217026535664 0.30650308126681675 a
        0.013954420136811033 0.30780234937010015 a
        0.013492579173059393 0.30658134629821615 a
        0.013515428950389227 0.3053334554036458 a
        0.013155313038019353 0.3068508835150456 a
        0.012992287424136745 0.3056754995213557 a
        0.013477218358724912 0.30716961051188113 a
        0.013902639044753893 0.30609872376008934 a
        0.013903225865023718 0.3068148133004876 a
        0.01373000864252581 0.30701763358154 a
        0.01380632195165261 0.3065652800049025 a
        0.013507313866522477 0.3074271901837977 a
        0.013749929956988928 0.3078086623903357 a
        0.013895329707810858 0.3076757532281205 a
        0.013721992946543811 0.30664671086909884 a
        0.013766372106681574 0.3059418908657252 a
        0.013700977108123838 0.3072187423237534 a
        0.013961186125864153 0.3070804517739493 a
        0.013742963092909751 0.30724232700250265 a
        0.0140643951024605 0.3061400377377735 a
        0.013406279749500588 0.30488994078031834 a
        0.01333807695082713 0.307547285805037 a
        0.013579416927623324 0.30771438823107283 a
        0.013281288750921228 0.30788325136037503 a
        0.013902835127531035 0.30814594497576986 a
        0.01444845749099448 0.30716920888488636 a
        0.013407502190774884 0.3040267810567578 a
        0.01779387677628413 0.32026269319206374 a
        0.020303868317634872 0.3218492486020191 a
        0.01779387677628413 0.32026269319206374 b
        0.020303868317634872 0.3218492486020191 b
            };
        \end{axis}
    \end{tikzpicture}
    \caption{The outliers (red) based on the 4-dist sorted graph} \label{fig:outliers}  
\end{subfigure}
\begin{subfigure}[b]{0.4\linewidth}
    \begin{tikzpicture}
        \pgfplotsset{%
        width=1.1\linewidth,
        % height=1\textheight
        }
        \begin{axis}[%
        scatter/classes={%
            a={mark=o,draw=black},
            b={mark=o,draw=red}}]
        \addplot[scatter,only marks,%
            scatter src=explicit symbolic]%
        table[meta=label] {
        x y label
        0 2.8949479214654294e-05 a
        1 3.179477203046875e-05 a
        2 3.267322064601172e-05 a
        3 3.280798350852557e-05 a
        4 3.280798350852557e-05 a
        5 4.203733493201637e-05 a
        6 4.2699125166050646e-05 a
        7 4.294905342573423e-05 a
        8 4.676111584622105e-05 a
        9 4.843067177105661e-05 a
        10 4.8774191374681006e-05 a
        11 4.9445616498532656e-05 a
        12 4.9445616498532656e-05 a
        13 4.9958989436862796e-05 a
        14 5.0408826155301995e-05 a
        15 5.0408826155301995e-05 a
        16 5.193014842759746e-05 a
        17 5.634406741910958e-05 a
        18 5.6633789754044637e-05 a
        19 5.6633789754044637e-05 a
        20 5.7410217372431694e-05 a
        21 5.7410217372431694e-05 a
        22 5.921367982846027e-05 a
        23 5.921367982846027e-05 a
        24 6.0456090984576906e-05 a
        25 6.145988034834584e-05 a
        26 6.272460722809678e-05 a
        27 6.284972805645797e-05 a
        28 6.300710443137074e-05 a
        29 6.355470081318446e-05 a
        30 6.359467248323007e-05 a
        31 6.694121683315767e-05 a
        32 6.718318238640716e-05 a
        33 7.127585237345829e-05 a
        34 7.127585237345829e-05 a
        35 7.155322529164951e-05 a
        36 7.38556711976285e-05 a
        37 7.41805540625215e-05 a
        38 7.41805540625215e-05 a
        39 7.765820244366868e-05 a
        40 7.784120665026037e-05 a
        41 7.850775711089268e-05 a
        42 7.993973309373286e-05 a
        43 8.191994222015357e-05 a
        44 8.230819546665794e-05 a
        45 8.420232050680483e-05 a
        46 8.426994000106784e-05 a
        47 8.427717433224613e-05 a
        48 8.479177316626968e-05 a
        49 8.527074308148876e-05 a
        50 8.721724394873936e-05 a
        51 8.722066451929488e-05 a
        52 8.804553811338067e-05 a
        53 8.830233135155855e-05 a
        54 8.975832698738875e-05 a
        55 8.975832698738875e-05 a
        56 8.994826250782899e-05 a
        57 9.030466847320144e-05 a
        58 9.129116493983801e-05 a
        59 9.134194523707361e-05 a
        60 9.161579912448281e-05 a
        61 9.193227245685756e-05 a
        62 9.370516727749144e-05 a
        63 9.370516727749144e-05 a
        64 9.447030026752446e-05 a
        65 9.695684147183935e-05 a
        66 9.84919485426008e-05 a
        67 9.84919485426008e-05 a
        68 9.967369188253414e-05 a
        69 0.00010069773319988507 a
        70 0.00010069773319988507 a
        71 0.00010084232469055083 a
        72 0.00010512257877014406 a
        73 0.00010531121163963648 a
        74 0.00010531121163963648 a
        75 0.00010546120599483379 a
        76 0.00010572136596011746 a
        77 0.00010583143723479704 a
        78 0.00010588947949861364 a
        79 0.00010619424500734341 a
        80 0.00010702116161766447 a
        81 0.00010720818806042881 a
        82 0.00010954500810094572 a
        83 0.00010954500810094572 a
        84 0.00011073430522929063 a
        85 0.00011246416770882411 a
        86 0.0001130661635295037 a
        87 0.00011438558850046904 a
        88 0.00011469125335755415 a
        89 0.00012147392693952117 a
        90 0.00012237591171992 a
        91 0.0001224201334093466 a
        92 0.00012276203358687853 a
        93 0.0001248071419298498 a
        94 0.00012783937274188738 a
        95 0.00012853116577443856 a
        96 0.00013370751331335452 a
        97 0.00013450189840040735 a
        98 0.00013450189840040735 a
        99 0.00013535715902440975 a
        100 0.00013868052735234137 a
        101 0.00013946592819106236 a
        102 0.00013970777345427544 a
        103 0.000146505571066363 a
        104 0.00015101069658064682 a
        105 0.00015706270827900125 a
        106 0.00016013706955956582 a
        107 0.0001662494310569234 a
        108 0.0001669486955830121 a
        109 0.00017282103511564564 a
        110 0.0001741512152900111 a
        111 0.0001867419363296455 a
        112 0.00019374828758415263 a
        113 0.00019483919084249817 a
        114 0.00019699221542950795 a
        115 0.00019850109877225447 a
        116 0.00019912614342020588 a
        117 0.0002031933960156964 a
        118 0.0002232952178907816 a
        119 0.00022506654839991077 a
        120 0.000280816704139863 a
        121 0.0002832259109375601 a
        122 0.00028521681835927984 a
        123 0.00029354421334281766 a
        124 0.0003426297800563272 a
        125 0.00034744633170887413 a
        126 0.0005864789735718681 a
        127 0.0010507104872560743 a
        128 0.012726185306264882 a
        129 0.015124607637026024 a
        };
        \draw[gray, dashed] (127,0) -- (127,1);
        \draw[gray, dashed] (0,0.0010507104872560743) -- (129,0.0010507104872560743);
        \end{axis}
        % \draw[gray, thick] (128,-1) -- (128,2);
    \end{tikzpicture}
    \caption{A 4-dist sorted graph} \label{fig:4-dist}  
\end{subfigure}
\caption{An example of how outliers are found for on the workstation for Fasta measured by HardwareMonitor}
\end{figure}


