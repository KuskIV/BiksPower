\section{Comparison}\label{sec:comparison}

In this section of the results, a comparison between the different DUTs and measuring instruments will be performed. This will be done in a way where the different measuring instruments will be compared for each DUT, before comparing all DUTs and all measuring instruments. Before this, the expectations will be presented.


\subsection{Expectations:} Based on what was seen in \cref{sec:iterations}, similar observations are expected. This will include a clamp with a high standard deviation compared to the different software measuring instruments, and cases where Intel Power Gadget and LHM measurements a close to each other. For RAPL, a low standard deviation is expected, in addition to lower measured energy consumption in most cases compared to the other measuring instruments.


\subsection{Results}
\paragraph{Workstation}


                            \begin{figure}
                                \centering
                                \begin{tikzpicture}[]
                                    \pgfplotsset{%
                                        width=.7\textwidth,
                                        height=.2\textheight
                                    }
                                    \begin{axis}[xlabel={Average energy consumption (Watts)}, title={Cores - Fasta - Energy - without outliers}, ytick={1, 2},
                                    yticklabels={
                                        IntelPowerGadget , HardwareMonitor 
                                        },
                                        xmin=0,xmax=80,
                                        ]
                                    
                                    \addplot+ [boxplot prepared={
                                    lower whisker=54.22445449466433,
                                    lower quartile=54.45751347572185,
                                    median=54.54624547297937,
                                    upper quartile=54.72020125409266,
                                    upper whisker=55.103791721157386},
                                    ] table[row sep=\\,y index=0] {\\};
                                    
                                    \addplot+ [boxplot prepared={
                                    lower whisker=51.45082608392138,
                                    lower quartile=51.933860038023106,
                                    median=52.121433545941585,
                                    upper quartile=52.479201309170854,
                                    upper whisker=54.95103920614709},
                                    ] table[row sep=\\,y index=0] {\\};
                                    
                                    \end{axis}
                                \end{tikzpicture}
                            \caption{A comparison of of Cores energy consumption for test case Fasta for the workstation (without outliers)} \label{fig:Fasta_Cores_comparison_energy_without_outliers_PowerKomplett_avg_watts_exp2}
                            \end{figure}
                            

                \begin{figure}[H]
                    \centering
                    \begin{tikzpicture}
                        \pgfplotsset{%
                            width=1\textwidth,
                            height=0.4\textheight
                        }
                        \begin{axis}[
                            xlabel={Start battery level},
                            ylabel={Average dynamic energy (watt)},
                            ymin=0,ymax=20,
                        ]
                        
                            \addplot [mark=none, ultra thick, red]  coordinates {
                            (40, 0.006595684402444291)(45, 0.007295220674193181)(50, 0.007999659608910881)(55, 0.007202467971243639)(60, 0.007280865079495958)(65, 0.006858423509955077)(70, 0.008369444141141925)(75, 0.007144940647010992)(80, 0.004753236834232667)
                            };
                            \addlegendentry{Surface4Pro - IntelPowerGadget}
                            
                            \addplot [mark=none, ultra thick, blue]  coordinates {
                            (40, 0.002385328427460912)(45, 0.0017856511573015649)(50, 0.0025901992189954056)(55, 0.00210998144366709)(60, 0.00286452646862575)(65, 0.0020038487280194628)(70, 0.002908483770774353)(75, 0.0005098111150931342)(80, -0.005995916826693459)
                            };
                            \addlegendentry{Surface4Pro - HardwareMonitor}
                            
                            \addplot [mark=none, ultra thick, orange]  coordinates {
                            (50, 251.8661014299577)(55, 214.7597259251014)(60, 162.23880995564176)(65, 107.43421593849766)(70, 53.9187387386668)(75, 0.7554852536149699)(80, -44.463085003568885)
                            };
                            \addlegendentry{Surface4Pro - RAPL}
                            
                            \addplot [mark=none, dashdotted, red]  coordinates {
                            (40, -0.004038062354887025)(45, -0.004312668500277529)(50, -0.003808663911021498)(55, -0.0037057407527755254)(60, -0.004478257932982471)(65, -0.0026308734995501644)(70, -0.0034090674446925874)(75, -0.003041497079697436)(80, -0.0020334307266356875)
                            };
                            \addlegendentry{SurfaceBook - IntelPowerGadget}
                            
                            \addplot [mark=none, dashdotted, blue]  coordinates {
                            (40, -0.002443518930616523)(45, -0.0029256880137447055)(50, -0.002551777773312929)(55, -0.0027567211782433486)(60, -0.002206859154402231)(65, -0.0026388300848279207)(70, -0.0024597736945479324)(75, -0.002445350799195827)(80, -0.000541271265011134)
                            };
                            \addlegendentry{SurfaceBook - HardwareMonitor}
                            
                            \addplot [mark=none, dashdotted, orange]  coordinates {
                            (40, 101.92702155010147)(45, 86.7212700795567)(50, 69.40754598562454)(55, 51.343785407669614)(60, 32.443112755251185)(65, 14.52577919077786)(70, -4.520517378551423)(75, -23.811706977384954)(80, -35.700372165212755)
                            };
                            \addlegendentry{SurfaceBook - RAPL}
                            
                        \end{axis}
                    \end{tikzpicture} 
                \caption{A graph illustrating the energy consumption of Dram for test case Nbody with regards to the battey level of the DUT (with outliers)} \label{fig:Nbody_Dram_charge}
                \end{figure}
                

The first DUT to consider is the workstation. For this DUT, Fasta and NBody measurements can be seen in \cref{fig:Fasta_Cores_comparison_dynamic_energy_without_outliers_PowerKomplett_avg_watts} and \cref{fig:Nbody_Cores_comparison_dynamic_energy_without_outliers_PowerKomplett_avg_watts} respectively, and the rest can be found in \cref{app:comparison_workstation}. When comparing the software measuring instruments, Intel Power Gadget and LHM measure close to each other in most cases. This includes BinaryTrees, Fasta and NBody where for FannkuchRedux a bigger difference can be observed. When considering RAPL compare to this, RAPL will in all test cases except NBody measure an energy consumption lower than its windows equivalents. When comparing the software measuring instruments against the hardware measurements, RAPL will against the clamp on Linux in all cases except NBody get a measurement higher than the clamp. For Windows, both Intel Power Gadget and LHM will get a measurement higher than the clamp.

\paragraph{Surface Pro 4}


                            \begin{figure}
                                \centering
                                \begin{tikzpicture}[]
                                    \pgfplotsset{%
                                        width=.7\textwidth,
                                        height=.2\textheight
                                    }
                                    \begin{axis}[xlabel={Average energy consumption (Watts)}, title={Cores - Fasta - Energy - without outliers}, ytick={1, 2},
                                    yticklabels={
                                        IntelPowerGadget , HardwareMonitor 
                                        },
                                        xmin=0,xmax=80,
                                        ]
                                    
                                    \addplot+ [boxplot prepared={
                                    lower whisker=54.22445449466433,
                                    lower quartile=54.45751347572185,
                                    median=54.54624547297937,
                                    upper quartile=54.72020125409266,
                                    upper whisker=55.103791721157386},
                                    ] table[row sep=\\,y index=0] {\\};
                                    
                                    \addplot+ [boxplot prepared={
                                    lower whisker=51.45082608392138,
                                    lower quartile=51.933860038023106,
                                    median=52.121433545941585,
                                    upper quartile=52.479201309170854,
                                    upper whisker=54.95103920614709},
                                    ] table[row sep=\\,y index=0] {\\};
                                    
                                    \end{axis}
                                \end{tikzpicture}
                            \caption{A comparison of of Cores energy consumption for test case Fasta for the workstation (without outliers)} \label{fig:Fasta_Cores_comparison_energy_without_outliers_PowerKomplett_avg_watts_exp2}
                            \end{figure}
                            

                            \begin{figure}
                                \centering
                                \begin{tikzpicture}[]
                                    \pgfplotsset{%
                                        width=.85\textwidth,
                                        height=.15\textheight
                                    }
                                    \begin{axis}[xlabel={Average energy consumption (Watts)}, title={Cores - FannkuchRedux - Energy - without outliers}, ytick={},
                                    yticklabels={
                                        
                                        },
                                        xmin=0,xmax=20,
                                        ]
                                    
                                    \end{axis}
                                \end{tikzpicture}
                            \caption{A comparison of of Cores energy consumption for test case FannkuchRedux for the Surface4Pro,  experiment \#2 (without outliers)} \label{fig:FannkuchRedux_Cores_comparison_energy_without_outliers_Surface4Pro_avg_watts_exp2}
                            \end{figure}
                            

The next DUT is the Surface Pro 4, where FannkuchRedux and Fasta are illustrated in \cref{fig:FannkuchRedux_Cores_comparison_dynamic_energy_without_outliers_Surface4Pro_avg_watts} and \cref{fig:Fasta_Cores_comparison_dynamic_energy_without_outliers_Surface4Pro_avg_watts} respectively, and the rest can be found in \cref{app:comparison_surfacepro4}. On the Surface Pro 4, a few things can be observed. When comparing the standard deviation, Intel Power Gadget deviates the most, which for example can be observed in the FannkuchRedux in \cref{fig:FannkuchRedux_Cores_comparison_dynamic_energy_without_outliers_Surface4Pro_avg_watts}. Intel Power Gadgets measurements are close the those made by LHM. When comparing RAPL against the measuring instruments on Windows, the difference changes from test case to test case,  where the measurements differ by $~6$W in FannkuchRedux in \cref{fig:FannkuchRedux_Cores_comparison_dynamic_energy_without_outliers_Surface4Pro_avg_watts} and $~1$W in Fasta in \cref{fig:Fasta_Cores_comparison_dynamic_energy_without_outliers_Surface4Pro_avg_watts}.

\paragraph{Surface Book}


                            \begin{figure}
                                \centering
                                \begin{tikzpicture}[]
                                    \pgfplotsset{%
                                        width=.85\textwidth,
                                        height=.15\textheight
                                    }
                                    \begin{axis}[xlabel={Average energy consumption (Watts)}, title={Cores - BinaryTrees - Energy - without outliers}, ytick={},
                                    yticklabels={
                                        
                                        },
                                        xmin=0,xmax=20,
                                        ]
                                    
                                    \end{axis}
                                \end{tikzpicture}
                            \caption{A comparison of of Cores energy consumption for test case BinaryTrees for the Surface4Pro,  experiment \#2 (without outliers)} \label{fig:BinaryTrees_Cores_comparison_energy_without_outliers_Surface4Pro_avg_watts_exp2}
                            \end{figure}
                            

                \begin{figure}[H]
                    \centering
                    \begin{tikzpicture}
                        \pgfplotsset{%
                            width=1\textwidth,
                            height=0.4\textheight
                        }
                        \begin{axis}[
                            xlabel={Start battery level},
                            ylabel={Average dynamic energy (watt)},
                            ymin=0,ymax=20,
                        ]
                        
                            \addplot [mark=none, ultra thick, red]  coordinates {
                            (40, 0.006595684402444291)(45, 0.007295220674193181)(50, 0.007999659608910881)(55, 0.007202467971243639)(60, 0.007280865079495958)(65, 0.006858423509955077)(70, 0.008369444141141925)(75, 0.007144940647010992)(80, 0.004753236834232667)
                            };
                            \addlegendentry{Surface4Pro - IntelPowerGadget}
                            
                            \addplot [mark=none, ultra thick, blue]  coordinates {
                            (40, 0.002385328427460912)(45, 0.0017856511573015649)(50, 0.0025901992189954056)(55, 0.00210998144366709)(60, 0.00286452646862575)(65, 0.0020038487280194628)(70, 0.002908483770774353)(75, 0.0005098111150931342)(80, -0.005995916826693459)
                            };
                            \addlegendentry{Surface4Pro - HardwareMonitor}
                            
                            \addplot [mark=none, ultra thick, orange]  coordinates {
                            (50, 251.8661014299577)(55, 214.7597259251014)(60, 162.23880995564176)(65, 107.43421593849766)(70, 53.9187387386668)(75, 0.7554852536149699)(80, -44.463085003568885)
                            };
                            \addlegendentry{Surface4Pro - RAPL}
                            
                            \addplot [mark=none, dashdotted, red]  coordinates {
                            (40, -0.004038062354887025)(45, -0.004312668500277529)(50, -0.003808663911021498)(55, -0.0037057407527755254)(60, -0.004478257932982471)(65, -0.0026308734995501644)(70, -0.0034090674446925874)(75, -0.003041497079697436)(80, -0.0020334307266356875)
                            };
                            \addlegendentry{SurfaceBook - IntelPowerGadget}
                            
                            \addplot [mark=none, dashdotted, blue]  coordinates {
                            (40, -0.002443518930616523)(45, -0.0029256880137447055)(50, -0.002551777773312929)(55, -0.0027567211782433486)(60, -0.002206859154402231)(65, -0.0026388300848279207)(70, -0.0024597736945479324)(75, -0.002445350799195827)(80, -0.000541271265011134)
                            };
                            \addlegendentry{SurfaceBook - HardwareMonitor}
                            
                            \addplot [mark=none, dashdotted, orange]  coordinates {
                            (40, 101.92702155010147)(45, 86.7212700795567)(50, 69.40754598562454)(55, 51.343785407669614)(60, 32.443112755251185)(65, 14.52577919077786)(70, -4.520517378551423)(75, -23.811706977384954)(80, -35.700372165212755)
                            };
                            \addlegendentry{SurfaceBook - RAPL}
                            
                        \end{axis}
                    \end{tikzpicture} 
                \caption{A graph illustrating the energy consumption of Dram for test case Nbody with regards to the battey level of the DUT (with outliers)} \label{fig:Nbody_Dram_charge}
                \end{figure}
                

For the Surface Book, test case BinaryTrees and Nbody can be seen in \cref{fig:BinaryTrees_Cores_comparison_dynamic_energy_without_outliers_PowerKomplett_avg_watts} and \cref{fig:Nbody_Cores_comparison_dynamic_energy_without_outliers_PowerKomplett_avg_watts} respectively, where the other test cases can be found in \cref{app:comparison_surfacebook}. On the Surface Book, the patterns are not as clear as they were on the Surface Pro 4. This is first of all because of the increased uncertainty in some of the results on the software measuring instruments on Windows, like for BinaryTrees in \cref{fig:BinaryTrees_Cores_comparison_dynamic_energy_without_outliers_PowerKomplett_avg_watts}. In one case, the difference between the median value for Intel Power Gadget and LHM stands out, with a difference of $~2$W for Nbody in \cref{fig:Nbody_Cores_comparison_dynamic_energy_without_outliers_PowerKomplett_avg_watts}. When comparing RAPL against Windows measurements, RAPL measures the highest in all cases except for FannkuchRedux.

\subsection{DUT and Measuring Instrument}


                            \begin{figure}
                                \centering
                                \begin{tikzpicture}[]
                                    \pgfplotsset{%
                                        width=.7\textwidth,
                                        height=.2\textheight
                                    }
                                    \begin{axis}[xlabel={Average energy consumption (Watts)}, title={Cores - Fasta - Energy - without outliers}, ytick={1, 2},
                                    yticklabels={
                                        IntelPowerGadget , HardwareMonitor 
                                        },
                                        xmin=0,xmax=80,
                                        ]
                                    
                                    \addplot+ [boxplot prepared={
                                    lower whisker=54.22445449466433,
                                    lower quartile=54.45751347572185,
                                    median=54.54624547297937,
                                    upper quartile=54.72020125409266,
                                    upper whisker=55.103791721157386},
                                    ] table[row sep=\\,y index=0] {\\};
                                    
                                    \addplot+ [boxplot prepared={
                                    lower whisker=51.45082608392138,
                                    lower quartile=51.933860038023106,
                                    median=52.121433545941585,
                                    upper quartile=52.479201309170854,
                                    upper whisker=54.95103920614709},
                                    ] table[row sep=\\,y index=0] {\\};
                                    
                                    \end{axis}
                                \end{tikzpicture}
                            \caption{A comparison of of Cores energy consumption for test case Fasta for the workstation (without outliers)} \label{fig:Fasta_Cores_comparison_energy_without_outliers_PowerKomplett_avg_watts_exp2}
                            \end{figure}
                            

                \begin{figure}[H]
                    \centering
                    \begin{tikzpicture}
                        \pgfplotsset{%
                            width=1\textwidth,
                            height=0.4\textheight
                        }
                        \begin{axis}[
                            xlabel={Start battery level},
                            ylabel={Average dynamic energy (watt)},
                            ymin=0,ymax=20,
                        ]
                        
                            \addplot [mark=none, ultra thick, red]  coordinates {
                            (40, 0.006595684402444291)(45, 0.007295220674193181)(50, 0.007999659608910881)(55, 0.007202467971243639)(60, 0.007280865079495958)(65, 0.006858423509955077)(70, 0.008369444141141925)(75, 0.007144940647010992)(80, 0.004753236834232667)
                            };
                            \addlegendentry{Surface4Pro - IntelPowerGadget}
                            
                            \addplot [mark=none, ultra thick, blue]  coordinates {
                            (40, 0.002385328427460912)(45, 0.0017856511573015649)(50, 0.0025901992189954056)(55, 0.00210998144366709)(60, 0.00286452646862575)(65, 0.0020038487280194628)(70, 0.002908483770774353)(75, 0.0005098111150931342)(80, -0.005995916826693459)
                            };
                            \addlegendentry{Surface4Pro - HardwareMonitor}
                            
                            \addplot [mark=none, ultra thick, orange]  coordinates {
                            (50, 251.8661014299577)(55, 214.7597259251014)(60, 162.23880995564176)(65, 107.43421593849766)(70, 53.9187387386668)(75, 0.7554852536149699)(80, -44.463085003568885)
                            };
                            \addlegendentry{Surface4Pro - RAPL}
                            
                            \addplot [mark=none, dashdotted, red]  coordinates {
                            (40, -0.004038062354887025)(45, -0.004312668500277529)(50, -0.003808663911021498)(55, -0.0037057407527755254)(60, -0.004478257932982471)(65, -0.0026308734995501644)(70, -0.0034090674446925874)(75, -0.003041497079697436)(80, -0.0020334307266356875)
                            };
                            \addlegendentry{SurfaceBook - IntelPowerGadget}
                            
                            \addplot [mark=none, dashdotted, blue]  coordinates {
                            (40, -0.002443518930616523)(45, -0.0029256880137447055)(50, -0.002551777773312929)(55, -0.0027567211782433486)(60, -0.002206859154402231)(65, -0.0026388300848279207)(70, -0.0024597736945479324)(75, -0.002445350799195827)(80, -0.000541271265011134)
                            };
                            \addlegendentry{SurfaceBook - HardwareMonitor}
                            
                            \addplot [mark=none, dashdotted, orange]  coordinates {
                            (40, 101.92702155010147)(45, 86.7212700795567)(50, 69.40754598562454)(55, 51.343785407669614)(60, 32.443112755251185)(65, 14.52577919077786)(70, -4.520517378551423)(75, -23.811706977384954)(80, -35.700372165212755)
                            };
                            \addlegendentry{SurfaceBook - RAPL}
                            
                        \end{axis}
                    \end{tikzpicture} 
                \caption{A graph illustrating the energy consumption of Dram for test case Nbody with regards to the battey level of the DUT (with outliers)} \label{fig:Nbody_Dram_charge}
                \end{figure}
                

The last comparison in this experiment is between all DUTs and measuring instruments. The test cases Fasta and Nbody are used, where the rest can be found in \cref{app:comparison}. When comparing the different DUTs, some of the same tendencies can be observed. This could be the similarity between Intel Power Gadget and LHM in most cases, with a few outliers like Nbody on Surface Book in \cref{fig:Nbody_Cores_comparison_dynamic_energy_without_outliers_avg_watts}. An outlier like this, is however not consistent for the DUT, meaning, the Intel Power Gadget and LHM measurements for the workstation and Surface Pro 4 for Nbody are still similar. Another thing to consider is the correlation between when RAPL measurements are either higher/lower compared to the Windows measurements across the same DUT. Here it is however more difficult to find a pattern. In most cases, RAPL will measure a lower value across all measuring instruments for all DUTS, with some exceptions. Exceptions include Surface Book on Fasta in \cref{fig:Fasta_Cores_comparison_dynamic_energy_without_outliers_avg_watts}, where the RAPL measurement is higher. For the other two DUT's, RAPL will however measure lower energy consumption. Another example is the energy consumption for the test case Nbody in \cref{fig:Nbody_Cores_comparison_dynamic_energy_without_outliers_avg_watts}, where both the Surface Book and the Workstation, RAPL reports a higher energy consumption, this is however not the case for the Surface Pro 4.

