\section{Time series}\label{sec:timeseries}
In this section, we present some of the time series graphs containing the average energy consumption in joules at a specific time stamp over 120 measurements. These graphs are not showing dynamic energy consumption, but rather the raw data. Not all of the graphs will be shown, because of the vast amount of graphs that would be, however some additional graphs are shown in \cref{ch:appAlabel}. This section is split up into different parts that focus on either the cases, the DUT or the measuring instrument. They will each include our expectations and then the actual results will be presented.
%Add expectations
\subsection{Comparing the test cases}\label[subsec]{subsec:comparing_test_case}
In this subsection, light will be shed on the different test cases.

\paragraph{Expectations:}
When looking at the different test cases with their real measured energy consumption and not dynamic energy consumption, we expect that the energy consumption of TestCaseIdle is a lot less than the other test cases. Furthermore, the energy consumption should be consistent as there should not be many background processes that can interfere with the energy consumption. For BinaryTrees and Fasta there could potentially be some visible spikes due to garbage collection since those test cases are memory-heavy.
\paragraph{Results:}
%Sammenligne alle test cases samme DUT og MI. se tendens.
In \cref{fig:time_series_Workstation_RAPL} some time series graphs are shown where the DUT is the workstation and the measuring instrument is RAPL. Each graph is an average of all the measurements of a given configuration. In \cref{fig:time_series_Workstation_RAPLIdleCase} TestCaseIdle is shown. What can be seen, is a lower average than the other test cases as we expected. However, there are some spikes, which were not expected. We can also observe that BinaryTrees consumes the most energy, followed by FannkuckRedux, Fasta, Nbody and finally TestCaseIdle. Nbody has a lower variance than the other test cases, but like the others, it does have some spikes. Since these graphs show an average over the measurements, the spikes are consistent over many measurements. Notably, all of the test cases in \cref{fig:time_series_Workstation_RAPL} have a spike at around $~28$ seconds into a test case run. This is an interesting observation, in an attempt to figure out why this occurs a look at a second configuration is done. With the same DUT, but using Intel Power Gadget as shown in \cref{fig:time_series_Workstation_IntelPowerGadget} the spike at around $~28$ does not appear. However, it could have something to do with the measuring instrument and not the DUT so a third configuration is looked at, where the Surface Pro 4 and RAPL are used, as shown in \cref{fig:time_series_Surface4Pro_RAPL}. Here the spikes also do not appear. The cause of the spike is as such currently unknown, but we speculated that there is some process that would start at the same interval in each run. To further investigate this is future work.\todo{Any ideas as to why this spike occurs?} Another observation which appears in many configurations is that the measurement will start with a measurement close to $~0$, this could be due to the measuring instrument being started by our framework one line of code before the test case. As such the measuring instrument could potentially conduct a measurement before the test case is executed. Therefore in theory it could be a measurement of the DUT being idle. However, looking at \cref{fig:time_series_Workstation_RAPLIdleCase} there is also an initial measurement which is lower than the remaining measurements. This trend is consistent with all the RAPL measurements on TestCaseIdle as can be seen in \cref{fig:time_series_TestCaseIdle_RAPL}, however, it is not present on the other measuring instruments for TestCaseIdle as can be seen for Intel Power Gadget in \cref{fig:time_series_TestCaseIdle_IPG}. %We speculate that the reason could be that the actual increase in the TestCaseIdle is the command prompt being opened, however, this does not explain why it does not appear on the other measuring instruments. 


\begin{figure}[H]  
    \centering 
    \begin{subfigure}[b]{0.49\linewidth}
        \begin{tikzpicture}
            \pgfplotsset{%
        width=1\linewidth,
        % height=1\textheight
            }
            \begin{axis}[ymax=120,
            xlabel={Time (Seconds)},
            ylabel={Energy Consumption (Joules)},
            ]
            \addplot[color=blue, mark=none,] coordinates { %% AVG value
            (0.0, 0.0)(0.1001669449081803, 57.7940715)(0.2003338898163606, 63.80245441666663)(0.3005008347245409, 59.70699591666668)(0.4006677796327212, 59.425786)(0.5008347245409015, 59.939936333333335)(0.6010016694490818, 60.0352364166667)(0.7011686143572621, 60.04882149999998)(0.8013355592654424, 60.205361500000016)(0.9015025041736228, 60.99890783333334)(1.001669449081803, 60.196038666666674)(1.1018363939899833, 59.58499166666666)(1.2020033388981637, 60.108549583333335)(1.3021702838063438, 60.137952999999996)(1.4023372287145242, 60.19749333333335)(1.5025041736227045, 60.137505583333336)(1.6026711185308848, 60.11290833333333)(1.7028380634390652, 59.96604850000001)(1.8030050083472455, 60.178674250000014)(1.9031719532554257, 60.059162083333334)(2.003338898163606, 60.21851441666669)(2.1035058430717863, 59.967731583333325)(2.2036727879799667, 60.44271633333332)(2.303839732888147, 60.356377250000016)(2.4040066777963274, 60.16260675000001)(2.5041736227045073, 60.182259166666704)(2.6043405676126876, 60.18638516666666)(2.704507512520868, 59.915150166666656)(2.8046744574290483, 60.25216524999998)(2.9048414023372287, 60.28135)(3.005008347245409, 60.37591325)(3.1051752921535893, 60.00882358333335)(3.2053422370617697, 60.118803583333374)(3.30550918196995, 60.27501241666671)(3.4056761268781304, 60.079827333333334)(3.5058430717863107, 60.18649691666666)(3.606010016694491, 60.402595999999996)(3.7061769616026714, 60.418673750000025)(3.8063439065108513, 60.19580424999995)(3.906510851419032, 60.338544999999996)(4.006677796327212, 60.292071833333345)(4.106844741235393, 60.250333999999995)(4.207011686143573, 60.26594908333335)(4.3071786310517535, 60.40605925)(4.407345575959933, 60.18090724999999)(4.507512520868114, 60.35219133333332)(4.607679465776294, 60.266396166666674)(4.707846410684475, 60.24959600000001)(4.808013355592655, 60.25648875)(4.908180300500835, 60.23536033333334)(5.0083472454090145, 60.51462591666663)(5.108514190317195, 60.49057766666667)(5.208681135225375, 60.24679933333335)(5.308848080133556, 60.27384224999999)(5.409015025041736, 60.40763708333332)(5.509181969949917, 60.169212833333326)(5.609348914858097, 60.32672949999999)(5.709515859766277, 60.57405424999999)(5.809682804674457, 60.15213391666669)(5.909849749582638, 60.368400500000014)(6.010016694490818, 60.28043975)(6.110183639398999, 60.165140166666646)(6.210350584307179, 60.371716333333325)(6.3105175292153595, 60.29423841666666)(6.410684474123539, 60.330803916666696)(6.510851419031719, 60.33832608333334)(6.6110183639399, 60.38646791666665)(6.71118530884808, 60.32951624999999)(6.811352253756261, 60.17364933333333)(6.911519198664441, 60.38885241666666)(7.011686143572621, 60.29364383333334)(7.111853088480801, 60.17931433333334)(7.212020033388982, 60.44979691666668)(7.312186978297162, 60.38820700000005)(7.412353923205343, 60.52332258333332)(7.512520868113523, 60.32151666666667)(7.612687813021703, 60.417839333333355)(7.712854757929884, 60.561485083333345)(7.813021702838064, 60.31200524999999)(7.913188647746244, 60.49232249999997)(8.013355592654424, 60.356611249999986)(8.113522537562606, 60.489397916666704)(8.213689482470786, 60.328565249999976)(8.313856427378965, 59.986001833333354)(8.414023372287145, 60.52388816666666)(8.514190317195327, 60.40567291666665)(8.614357262103507, 60.29866374999998)(8.714524207011687, 60.36799891666669)(8.814691151919867, 60.18856691666665)(8.914858096828048, 60.416669833333344)(9.015025041736228, 60.300845166666676)(9.115191986644408, 60.51730633333334)(9.215358931552588, 60.41613066666667)(9.31552587646077, 60.33716666666668)(9.41569282136895, 60.4153875)(9.51585976627713, 60.37372216666667)(9.61602671118531, 60.24925016666664)(9.71619365609349, 60.44772683333333)(9.81636060100167, 60.41916683333332)(9.916527545909851, 60.28459416666667)(10.016694490818029, 60.64863883333333)(10.11686143572621, 60.76521508333333)(10.21702838063439, 61.0895905)(10.31719532554257, 60.964347166666656)(10.41736227045075, 61.195130250000005)(10.51752921535893, 61.02308841666666)(10.617696160267112, 61.21909108333336)(10.717863105175292, 60.84979491666668)(10.818030050083472, 60.995164333333335)(10.918196994991652, 60.78358141666665)(11.018363939899833, 60.77994983333331)(11.118530884808013, 60.75591250000002)(11.218697829716193, 61.18048166666666)(11.318864774624373, 60.86851741666669)(11.419031719532555, 60.91355608333338)(11.519198664440735, 61.05471925)(11.619365609348915, 60.827139916666646)(11.719532554257095, 60.86003758333334)(11.819699499165276, 60.99945725000001)(11.919866444073456, 60.89096791666665)(12.020033388981636, 60.94379316666666)(12.120200333889816, 60.93660208333334)(12.220367278797998, 60.841594999999984)(12.320534223706177, 60.78538208333332)(12.420701168614357, 61.02982208333332)(12.520868113522537, 61.04433866666667)(12.621035058430719, 60.94567516666668)(12.721202003338899, 61.15881391666669)(12.821368948247079, 60.937197)(12.921535893155259, 60.691007333333346)(13.021702838063439, 61.02273641666669)(13.12186978297162, 60.95710424999999)(13.2220367278798, 60.89361250000002)(13.32220367278798, 60.95298916666664)(13.42237061769616, 61.08974341666667)(13.522537562604342, 60.99898916666666)(13.622704507512521, 60.8261738333333)(13.722871452420701, 60.98121341666667)(13.823038397328881, 60.77967008333336)(13.923205342237063, 60.8235338333333)(14.023372287145243, 61.11858283333332)(14.123539232053423, 61.200104416666626)(14.223706176961603, 60.852565833333365)(14.323873121869784, 61.04004074999998)(14.424040066777964, 60.95868633333331)(14.524207011686144, 60.92924091666665)(14.624373956594324, 61.22128899999999)(14.724540901502506, 60.73802933333331)(14.824707846410686, 60.96690066666667)(14.924874791318866, 61.07944841666666)(15.025041736227045, 60.90268625)(15.125208681135225, 61.042527666666665)(15.225375626043405, 60.883032916666664)(15.325542570951589, 61.00814458333333)(15.425709515859769, 61.19163624999999)(15.525876460767948, 60.90828125000001)(15.626043405676128, 61.21082641666667)(15.726210350584308, 60.78140991666667)(15.826377295492488, 61.08484508333334)(15.926544240400668, 60.84901666666667)(16.026711185308848, 60.82262849999998)(16.126878130217026, 61.05716600000002)(16.22704507512521, 60.689658999999956)(16.32721202003339, 60.99616083333333)(16.42737896494157, 60.9036225)(16.52754590984975, 61.096762250000005)(16.62771285475793, 61.067989666666655)(16.72787979966611, 60.88105416666665)(16.82804674457429, 60.93068108333334)(16.92821368948247, 60.8957945833333)(17.028380634390654, 61.05015191666668)(17.128547579298832, 61.16173875)(17.228714524207014, 60.94413966666669)(17.328881469115192, 60.97779424999997)(17.429048414023374, 60.88866383333332)(17.529215358931552, 60.996436416666675)(17.629382303839733, 60.98030183333332)(17.72954924874791, 60.94239483333333)(17.829716193656097, 60.89054066666667)(17.929883138564275, 60.9689400833333)(18.030050083472457, 61.17646850000001)(18.130217028380635, 61.01235625000003)(18.230383973288816, 61.02599708333332)(18.330550918196995, 60.882275416666694)(18.430717863105176, 60.84084241666669)(18.530884808013354, 61.030748250000016)(18.63105175292154, 61.22533208333338)(18.731218697829718, 61.155554333333335)(18.8313856427379, 61.0709295)(18.931552587646078, 60.96715916666669)(19.03171953255426, 61.22600908333337)(19.131886477462437, 60.63148791666668)(19.23205342237062, 61.043559583333305)(19.332220367278797, 60.888094250000044)(19.43238731218698, 61.342957083333346)(19.53255425709516, 60.83488641666667)(19.63272120200334, 60.93961191666665)(19.73288814691152, 61.133206083333334)(19.833055091819702, 61.04247158333333)(19.93322203672788, 61.01959908333333)(20.033388981636058, 61.30606141666668)(20.13355592654424, 61.11330791666668)(20.23372287145242, 61.35175158333332)(20.333889816360603, 61.21818616666668)(20.43405676126878, 61.29849783333335)(20.534223706176963, 61.48611991666669)(20.63439065108514, 61.3598745)(20.734557595993323, 61.324728000000015)(20.8347245409015, 61.52751641666668)(20.934891485809683, 61.19001375000002)(21.03505843071786, 61.49009224999999)(21.135225375626046, 61.28558966666666)(21.235392320534224, 61.3931483333333)(21.335559265442406, 61.25419191666667)(21.435726210350584, 61.25824025000004)(21.535893155258766, 61.4735619166667)(21.636060100166944, 61.43869558333335)(21.736227045075125, 61.25114066666669)(21.836393989983303, 61.34022575000004)(21.93656093489149, 61.29973974999999)(22.036727879799667, 61.369954249999985)(22.13689482470785, 61.384537749999986)(22.237061769616027, 61.58045424999998)(22.33722871452421, 61.36048491666666)(22.437395659432386, 61.366196249999994)(22.537562604340568, 61.37972516666667)(22.637729549248746, 61.149506833333334)(22.737896494156928, 61.31521675000001)(22.83806343906511, 61.35717333333334)(22.93823038397329, 61.38604775000002)(23.03839732888147, 61.47958908333332)(23.13856427378965, 61.08311074999998)(23.23873121869783, 61.2924913333333)(23.33889816360601, 61.23666975000002)(23.43906510851419, 61.370382666666664)(23.53923205342237, 61.143997999999996)(23.639398998330552, 61.39009208333336)(23.739565943238734, 61.40644291666667)(23.839732888146912, 61.27509641666669)(23.939899833055094, 61.56962124999999)(24.040066777963272, 61.47033208333334)(24.140233722871454, 61.308146916666686)(24.24040066777963, 61.3004465)(24.340567612687813, 61.58761566666669)(24.440734557595995, 61.282227666666685)(24.540901502504177, 61.362681083333335)(24.641068447412355, 61.512863833333356)(24.741235392320537, 61.19368083333333)(24.841402337228715, 61.14600766666664)(24.941569282136896, 61.61427741666664)(25.041736227045075, 61.52165291666665)(25.141903171953256, 61.21052583333333)(25.242070116861438, 61.35980866666664)(25.34223706176962, 61.530613833333334)(25.442404006677798, 61.3649198333333)(25.54257095158598, 61.60506699999998)(25.642737896494157, 61.53290249999999)(25.74290484140234, 61.50602816666663)(25.843071786310517, 61.15455224999998)(25.9432387312187, 61.62108324999999)(26.043405676126877, 61.253302166666664)(26.143572621035062, 61.70309908333329)(26.24373956594324, 61.693084)(26.34390651085142, 61.51294058333333)(26.4440734557596, 61.34507266666669)(26.544240400667782, 61.25377983333335)(26.64440734557596, 61.57712841666669)(26.744574290484138, 61.299001833333335)(26.84474123539232, 61.42871649999995)(26.9449081803005, 61.62739533333333)(27.045075125208683, 61.52557349999999)(27.14524207011686, 61.380849999999995)(27.245409015025043, 61.43880233333333)(27.34557595993322, 61.85000066666668)(27.445742904841403, 61.81530141666667)(27.54590984974958, 61.51875316666669)(27.646076794657763, 61.42199791666666)(27.746243739565944, 61.44813600000001)(27.846410684474126, 61.42576674999997)(27.946577629382304, 61.65102583333333)(28.046744574290486, 61.38394708333335)(28.146911519198664, 61.35779433333333)(28.247078464106846, 61.741047)(28.347245409015024, 61.490535333333355)(28.447412353923205, 61.82743774999997)(28.547579298831387, 61.84448716666666)(28.64774624373957, 61.56499191666667)(28.747913188647747, 62.03253533333334)(28.84808013355593, 61.55611683333336)(28.948247078464107, 61.9182935)(29.04841402337229, 61.47733616666668)(29.148580968280466, 61.557540666666675)(29.248747913188648, 61.74994916666664)(29.348914858096826, 61.89189066666669)(29.44908180300501, 61.58875016666668)(29.54924874791319, 61.54537400000002)(29.64941569282137, 61.897542333333334)(29.74958263772955, 61.81869416666665)(29.84974958263773, 61.61930808333332)(29.94991652754591, 61.50146533333336)(30.05008347245409, 61.67316674999997)(30.15025041736227, 61.84909525)(30.25041736227045, 61.30754666666669)(30.35058430717863, 61.58731025000001)(30.45075125208681, 61.30751699999999)(30.55091819699499, 61.41162150000001)(30.651085141903177, 61.77350783333332)(30.751252086811355, 61.62727308333334)(30.851419031719537, 61.23680758333334)(30.951585976627715, 61.683231999999975)(31.051752921535897, 61.25283925)(31.151919866444075, 61.69802341666669)(31.252086811352257, 61.78307549999998)(31.352253756260435, 61.77882333333333)(31.452420701168617, 61.67873608333336)(31.552587646076795, 61.423492749999994)(31.652754590984976, 61.35286016666666)(31.752921535893154, 61.39640333333334)(31.853088480801336, 61.61474099999999)(31.953255425709514, 61.71657308333333)(32.053422370617696, 61.39479583333333)(32.15358931552588, 61.57190958333331)(32.25375626043405, 61.5165405)(32.35392320534224, 61.76733366666667)(32.45409015025042, 61.66829458333332)(32.554257095158604, 61.50992341666667)(32.65442404006678, 61.480962000000005)(32.75459098497496, 61.48546408333333)(32.85475792988314, 61.59574925000001)(32.95492487479132, 61.607284166666666)(33.0550918196995, 61.393620916666684)(33.15525876460768, 61.606887833333325)(33.25542570951586, 61.466594166666674)(33.35559265442404, 61.59081041666668)(33.45575959933222, 61.620574249999976)(33.5559265442404, 61.58214841666667)(33.65609348914858, 61.62198375000003)(33.756260434056756, 61.58765683333331)(33.85642737896494, 61.77643741666668)(33.95659432387313, 61.450419499999974)(34.05676126878131, 61.203029499999985)(34.15692821368948, 61.76669291666667)(34.257095158597664, 61.54540991666669)(34.357262103505846, 61.50400366666664)(34.45742904841403, 61.57567808333337)(34.5575959933222, 61.53216058333333)(34.657762938230384, 61.64867174999998)(34.757929883138566, 61.58487975)(34.85809682804675, 61.67661516666668)(34.95826377295492, 61.69606466666667)(35.058430717863104, 61.702585750000026)(35.158597662771285, 61.785048833333335)(35.25876460767947, 61.46685791666668)(35.35893155258764, 61.91199183333335)(35.45909849749582, 61.43084291666667)(35.559265442404005, 61.699487666666705)(35.659432387312194, 61.69932524999998)(35.75959933222037, 61.897216000000036)(35.85976627712855, 61.659449166666676)(35.95993322203673, 61.67231725)(36.06010016694491, 61.65982508333333)(36.16026711185309, 61.640859000000006)(36.26043405676127, 61.47971641666667)(36.36060100166945, 61.65524225000001)(36.46076794657763, 61.696680916666665)(36.56093489148581, 61.57928983333333)(36.66110183639399, 61.87003525000001)(36.76126878130217, 61.75618925000002)(36.86143572621035, 61.67555208333335)(36.96160267111853, 61.654785333333315)(37.06176961602671, 61.556310166666634)(37.16193656093489, 61.577183583333344)(37.26210350584308, 61.60321066666667)(37.362270450751254, 61.81627866666669)(37.462437395659435, 61.35552016666665)(37.56260434056762, 61.802906833333324)(37.6627712854758, 61.480051916666675)(37.76293823038397, 61.67654416666666)(37.863105175292155, 61.62284799999996)(37.96327212020034, 61.69437191666668)(38.06343906510852, 61.614939083333326)(38.16360601001669, 61.657831249999994)(38.263772954924875, 61.49304250000001)(38.363939899833056, 61.775303750000006)(38.46410684474124, 61.49519899999999)(38.56427378964941, 61.79237800000001)(38.664440734557594, 61.602411416666676)(38.764607679465776, 61.565806416666675)(38.86477462437396, 61.73516833333336)(38.96494156928214, 61.74035083333333)(39.06510851419032, 61.560714083333345)(39.1652754590985, 61.88214575000003)(39.26544240400668, 61.63548333333333)(39.36560934891486, 61.771738000000006)(39.46577629382304, 61.54447375000003)(39.56594323873122, 61.77936808333336)(39.666110183639404, 61.70485933333332)(39.76627712854758, 61.405787583333314)(39.86644407345576, 61.74842791666664)(39.96661101836394, 61.532704250000016)(40.066777963272116, 61.89468891666663)(40.1669449081803, 61.72465441666666)(40.26711185308848, 61.70606991666666)(40.36727879799666, 61.695347583333316)(40.46744574290484, 61.513911333333354)(40.567612687813025, 61.93411141666665)(40.667779632721206, 61.49984325000005)(40.76794657762939, 61.60420133333335)(40.86811352253756, 62.00969941666665)(40.968280467445744, 61.633045916666696)(41.068447412353926, 61.770807250000004)(41.16861435726211, 61.61435966666667)(41.26878130217028, 61.55798341666668)(41.368948247078464, 61.80717900000002)(41.469115191986646, 61.718596666666684)(41.56928213689483, 61.716319250000005)(41.669449081803, 61.981154083333344)(41.769616026711184, 61.54389974999997)(41.869782971619365, 61.866642583333324)(41.96994991652755, 61.430949500000004)(42.07011686143572, 61.828866833333336)(42.1702838063439, 62.18060625000001)(42.27045075125209, 61.43257691666667)(42.370617696160274, 61.51089516666666)(42.47078464106845, 61.70638958333332)(42.57095158597663, 62.05361358333333)(42.67111853088481, 61.68316083333331)(42.77128547579299, 61.55980016666669)(42.87145242070117, 61.904824499999954)(42.97161936560935, 61.76059466666667)(43.07178631051753, 61.78499266666668)(43.17195325542571, 61.69599400000002)(43.27212020033389, 61.97305691666666)(43.37228714524207, 61.59453375000001)(43.47245409015025, 61.5904685)(43.57262103505843, 61.87355008333334)(43.67278797996661, 61.66939774999999)(43.77295492487479, 61.866932916666656)(43.87312186978298, 61.782073333333344)(43.97328881469116, 61.540669666666666)(44.073455759599334, 61.706110166666654)(44.173622704507515, 61.528905499999986)(44.2737896494157, 61.72950758333334)(44.37395659432388, 61.61952116666664)(44.47412353923205, 61.917963166666645)(44.574290484140235, 61.877440666666686)(44.67445742904842, 61.509481083333334)(44.7746243739566, 61.92065883333335)(44.87479131886477, 61.44632016666664)(44.974958263772955, 61.64944408333331)(45.075125208681136, 61.86713574999999)(45.17529215358932, 61.89352791666665)(45.27545909849749, 62.07966516666667)(45.375626043405674, 61.673111333333324)(45.475792988313856, 61.85286908333335)(45.57595993322204, 62.06362241666666)(45.67612687813022, 61.572703000000025)(45.7762938230384, 61.78447441666665)(45.87646076794658, 61.68374091666665)(45.976627712854764, 61.730402333333345)(46.07679465776294, 61.62381983333333)(46.17696160267112, 61.887257166666686)(46.2771285475793, 61.9343101666667)(46.37729549248748, 61.53286766666665)(46.47746243739566, 61.74630150000003)(46.57762938230384, 62.02570516666667)(46.67779632721202, 61.7687425)(46.7779632721202, 61.46372008333333)(46.87813021702838, 62.03506333333332)(46.97829716193656, 61.92095366666669)(47.07846410684474, 61.73186183333333)(47.17863105175292, 61.87447033333335)(47.278797996661105, 61.938317916666634)(47.378964941569286, 61.91874091666664)(47.47913188647747, 61.89146833333333)(47.57929883138564, 61.964883916666665)(47.679465776293824, 61.71888199999999)(47.779632721202006, 61.959166333333336)(47.87979966611019, 61.72896316666664)(47.97996661101836, 62.12710391666668)(48.080133555926544, 61.984089416666684)(48.180300500834726, 61.617548416666665)(48.28046744574291, 62.067050750000035)(48.38063439065108, 61.74981708333333)(48.48080133555926, 61.76605658333332)(48.580968280467445, 61.82574383333331)(48.68113522537563, 61.764368500000025)(48.7813021702838, 61.90880741666665)(48.88146911519199, 61.657481000000026)(48.98163606010017, 62.11769441666667)(49.081803005008354, 61.60123191666664)(49.18196994991653, 61.64368700000001)(49.28213689482471, 62.28719441666669)(49.38230383973289, 61.64799983333333)(49.48247078464107, 62.13684933333333)(49.58263772954925, 61.81796700000003)(49.68280467445743, 61.92013458333335)(49.78297161936561, 61.842386333333344)(49.88313856427379, 61.53159158333334)(49.98330550918197, 62.22936291666667)(50.08347245409015, 61.70774325000004)(50.18363939899833, 61.78212916666669)(50.28380634390651, 62.23482074999997)(50.38397328881469, 61.784296750000024)(50.484140233722876, 61.837142000000014)(50.58430717863106, 61.739175833333334)(50.68447412353924, 61.87357058333332)(50.784641068447414, 61.87806575)(50.884808013355595, 61.73561575)(50.98497495826378, 62.21994908333332)(51.08514190317196, 61.60091650000004)(51.18530884808013, 61.54906658333332)(51.285475792988315, 62.30517974999999)(51.3856427378965, 61.751860583333375)(51.48580968280468, 61.95909033333333)(51.58597662771285, 61.93246883333333)(51.686143572621035, 61.988992916666675)(51.786310517529216, 61.752140749999995)(51.8864774624374, 61.69964541666666)(51.98664440734557, 62.14552183333337)(52.086811352253754, 61.92377108333337)(52.18697829716194, 62.12459691666666)(52.287145242070125, 61.79213399999999)(52.3873121869783, 61.791015250000015)(52.48747913188648, 61.93137525000001)(52.58764607679466, 61.82186791666668)(52.68781302170284, 62.21235041666668)(52.78797996661102, 61.850167916666656)(52.8881469115192, 61.61434933333332)(52.98831385642738, 62.10032000000002)(53.088480801335564, 61.98020375000002)(53.18864774624374, 61.82971608333334)(53.28881469115192, 61.96628766666668)(53.3889816360601, 61.82462991666667)(53.489148580968276, 62.08432874999996)(53.58931552587646, 61.86589000000001)(53.68948247078464, 61.99963233333335)(53.78964941569283, 61.55037483333332)(53.889816360601, 61.616205999999984)(53.989983305509185, 62.14438716666667)(54.090150250417366, 62.09283349999998)(54.19031719532555, 61.84919608333333)(54.29048414023372, 61.47191349999999)(54.390651085141904, 61.868662916666686)(54.490818030050086, 61.821425583333344)(54.59098497495827, 61.924076333333325)(54.69115191986644, 62.30334274999998)(54.791318864774624, 61.70664466666667)(54.891485809682806, 61.61206608333335)(54.99165275459099, 62.14132016666666)(55.09181969949916, 62.07598750000001)(55.19198664440734, 61.61523433333332)(55.292153589315525, 62.215441916666656)(55.39232053422371, 62.08371366666668)(55.49248747913189, 61.807957833333326)(55.59265442404007, 61.70070341666668)(55.69282136894825, 62.15571441666666)(55.79298831385643, 61.827401916666645)(55.89315525876461, 62.19985299999998)(55.99332220367279, 62.0472558333333)(56.09348914858097, 61.686761916666676)(56.19365609348915, 61.787877)(56.29382303839733, 62.206062916666674)(56.39398998330551, 62.06733116666667)(56.49415692821369, 61.99412874999998)(56.59432387312187, 61.800379916666664)(56.69449081803005, 62.066236500000024)(56.79465776293823, 61.83323133333332)(56.89482470784641, 61.97709491666668)(56.99499165275459, 62.089385166666645)(57.095158597662774, 62.073520416666675)(57.195325542570956, 62.11131158333332)(57.29549248747914, 61.91701708333333)(57.39565943238732, 61.96328191666667)(57.495826377295494, 61.76019724999999)(57.595993322203675, 62.08104816666666)(57.69616026711186, 62.205412)(57.79632721202004, 61.88396608333333)(57.89649415692821, 61.770284083333316)(57.996661101836395, 61.94094216666665)(58.09682804674458, 62.109236666666646)(58.19699499165276, 61.806924333333335)(58.29716193656093, 61.93247950000002)(58.397328881469114, 62.15529725000002)(58.497495826377296, 61.75043758333332)(58.59766277128548, 61.80088666666665)(58.69782971619365, 62.16149749999999)(58.79799666110184, 62.018106083333315)(58.89816360601002, 61.68694583333335)(58.9983305509182, 62.101744000000025)(59.09849749582638, 62.25498316666666)(59.19866444073456, 62.075087333333315)(59.29883138564274, 61.99443425000001)(59.398998330550924, 62.09195866666664)(59.4991652754591, 61.80208275000003)(59.59933222036728, 61.92220458333332)(59.69949916527546, 62.184217666666655)(59.79966611018364, 61.86635275000001)(59.89983305509182, 62.20994424999999)(60.0, 61.96823016666667)
            };
            \addplot[color=blue, mark=none,name path=A] coordinates { %% MAX value
            (0.0, 0.0)(0.1001669449081803, 62.37411)(0.2003338898163606, 67.08173000000001)(0.3005008347245409, 68.20661)(0.4006677796327212, 73.45439999999999)(0.5008347245409015, 64.15877)(0.6010016694490818, 67.33076)(0.7011686143572621, 78.27251000000001)(0.8013355592654424, 68.01802)(0.9015025041736228, 66.44331)(1.001669449081803, 63.792559999999995)(1.1018363939899833, 62.93502)(1.2020033388981637, 63.276199999999996)(1.3021702838063438, 63.72238)(1.4023372287145242, 64.84725)(1.5025041736227045, 63.49105)(1.6026711185308848, 64.1191)(1.7028380634390652, 62.8221)(1.8030050083472455, 63.73519)(1.9031719532554257, 63.927440000000004)(2.003338898163606, 63.92501)(2.1035058430717863, 64.29671)(2.2036727879799667, 64.32356999999999)(2.303839732888147, 69.19599)(2.4040066777963274, 64.18563)(2.5041736227045073, 63.46542)(2.6043405676126876, 63.23287)(2.704507512520868, 63.20968)(2.8046744574290483, 63.49227)(2.9048414023372287, 65.84394)(3.005008347245409, 64.81307)(3.1051752921535893, 63.18832)(3.2053422370617697, 63.475789999999996)(3.30550918196995, 63.60091)(3.4056761268781304, 63.256069999999994)(3.5058430717863107, 63.94942)(3.606010016694491, 63.31526)(3.7061769616026714, 63.18344)(3.8063439065108513, 62.87948)(3.906510851419032, 63.908519999999996)(4.006677796327212, 63.025349999999996)(4.106844741235393, 63.71566)(4.207011686143573, 63.962849999999996)(4.3071786310517535, 64.43892)(4.407345575959933, 66.8437)(4.507512520868114, 63.83529)(4.607679465776294, 63.29573)(4.707846410684475, 65.02119)(4.808013355592655, 64.17830000000001)(4.908180300500835, 67.62311)(5.0083472454090145, 63.33907)(5.108514190317195, 64.99434)(5.208681135225375, 62.92586)(5.308848080133556, 64.29732)(5.409015025041736, 63.04977)(5.509181969949917, 63.666219999999996)(5.609348914858097, 63.49105)(5.709515859766277, 64.02205000000001)(5.809682804674457, 63.8243)(5.909849749582638, 63.183429999999994)(6.010016694490818, 63.66194)(6.110183639398999, 63.40194)(6.210350584307179, 64.17769)(6.3105175292153595, 63.83406)(6.410684474123539, 63.52218)(6.510851419031719, 64.0428)(6.6110183639399, 62.98018)(6.71118530884808, 63.550250000000005)(6.811352253756261, 63.025349999999996)(6.911519198664441, 65.48201)(7.011686143572621, 63.6119)(7.111853088480801, 63.487989999999996)(7.212020033388982, 64.13374)(7.312186978297162, 64.35285999999999)(7.412353923205343, 65.86165)(7.512520868113523, 65.68709)(7.612687813021703, 64.85029)(7.712854757929884, 63.48921)(7.813021702838064, 68.69245000000001)(7.913188647746244, 63.93477)(8.013355592654424, 63.48922)(8.113522537562606, 64.1661)(8.213689482470786, 62.468709999999994)(8.313856427378965, 63.38912)(8.414023372287145, 63.80599)(8.514190317195327, 63.71199)(8.614357262103507, 62.99056)(8.714524207011687, 64.39376)(8.814691151919867, 63.67782)(8.914858096828048, 64.88081)(9.015025041736228, 63.38728)(9.115191986644408, 72.18731)(9.215358931552588, 64.53597)(9.31552587646077, 63.754720000000006)(9.41569282136895, 63.45077)(9.51585976627713, 64.22652000000001)(9.61602671118531, 64.29732)(9.71619365609349, 63.990919999999996)(9.81636060100167, 63.478229999999996)(9.916527545909851, 63.4709)(10.016694490818029, 63.06563)(10.11686143572621, 64.43404)(10.21702838063439, 65.20125)(10.31719532554257, 63.9714)(10.41736227045075, 65.12007)(10.51752921535893, 63.974450000000004)(10.617696160267112, 64.1014)(10.717863105175292, 64.46639)(10.818030050083472, 64.27291)(10.918196994991652, 64.29854)(11.018363939899833, 63.205400000000004)(11.118530884808013, 65.04073)(11.218697829716193, 65.44416)(11.318864774624373, 63.68697)(11.419031719532555, 64.9095)(11.519198664440735, 64.43647999999999)(11.619365609348915, 63.59969)(11.719532554257095, 64.0544)(11.819699499165276, 64.05378999999999)(11.919866444073456, 63.56979)(12.020033388981636, 65.2983)(12.120200333889816, 64.81307)(12.220367278797998, 64.77338999999999)(12.320534223706177, 63.74374)(12.420701168614357, 65.72432)(12.520868113522537, 63.99337)(12.621035058430719, 64.13863)(12.721202003338899, 63.98299)(12.821368948247079, 64.14901)(12.921535893155259, 64.31014)(13.021702838063439, 63.45443)(13.12186978297162, 64.33699)(13.2220367278798, 63.67476)(13.32220367278798, 64.19112)(13.42237061769616, 64.05135)(13.522537562604342, 64.83565)(13.622704507512521, 63.84566)(13.722871452420701, 64.69405)(13.823038397328881, 65.21223)(13.923205342237063, 64.07699)(14.023372287145243, 64.05196000000001)(14.123539232053423, 64.41146)(14.223706176961603, 64.05989)(14.323873121869784, 66.87604999999999)(14.424040066777964, 63.941489999999995)(14.524207011686144, 63.8896)(14.624373956594324, 64.84419)(14.724540901502506, 64.40108000000001)(14.824707846410686, 64.71908)(14.924874791318866, 64.06539000000001)(15.025041736227045, 64.04035999999999)(15.125208681135225, 64.36934)(15.225375626043405, 64.08553)(15.325542570951589, 63.890820000000005)(15.425709515859769, 65.82869)(15.525876460767948, 64.02144)(15.626043405676128, 65.33614)(15.726210350584308, 64.05624)(15.826377295492488, 64.467)(15.926544240400668, 64.26558)(16.026711185308848, 64.60371)(16.126878130217026, 64.57075999999999)(16.22704507512521, 63.77181)(16.32721202003339, 64.00008)(16.42737896494157, 64.33760000000001)(16.52754590984975, 64.71114)(16.62771285475793, 64.56953)(16.72787979966611, 64.56099)(16.82804674457429, 64.75202999999999)(16.92821368948247, 66.29683)(17.028380634390654, 65.37092)(17.128547579298832, 65.28425)(17.228714524207014, 64.11788)(17.328881469115192, 66.70332)(17.429048414023374, 65.13656)(17.529215358931552, 63.995810000000006)(17.629382303839733, 65.81953)(17.72954924874791, 63.85482)(17.829716193656097, 63.4703)(17.929883138564275, 65.5052)(18.030050083472457, 63.82552)(18.130217028380635, 64.69771)(18.230383973288816, 64.44686)(18.330550918196995, 65.25801)(18.430717863105176, 63.50143)(18.530884808013354, 65.22323)(18.63105175292154, 64.71236)(18.731218697829718, 65.55404)(18.8313856427379, 64.24239)(18.931552587646078, 64.80452)(19.03171953255426, 63.78646)(19.131886477462437, 64.8973)(19.23205342237062, 65.38985)(19.332220367278797, 64.07149000000001)(19.43238731218698, 64.29427)(19.53255425709516, 64.23141)(19.63272120200334, 63.50753)(19.73288814691152, 64.3254)(19.833055091819702, 63.86398)(19.93322203672788, 63.41781)(20.033388981636058, 65.86653)(20.13355592654424, 63.95736)(20.23372287145242, 64.85884)(20.333889816360603, 64.96443000000001)(20.43405676126878, 64.8088)(20.534223706176963, 64.59456)(20.63439065108514, 64.21431)(20.734557595993323, 64.56527)(20.8347245409015, 64.07088)(20.934891485809683, 64.96748)(21.03505843071786, 64.15389)(21.135225375626046, 64.63485)(21.235392320534224, 67.40339)(21.335559265442406, 65.33736)(21.435726210350584, 64.89363)(21.535893155258766, 64.49202)(21.636060100166944, 64.85762)(21.736227045075125, 65.30989)(21.836393989983303, 65.38253)(21.93656093489149, 64.80025)(22.036727879799667, 65.30989)(22.13689482470785, 64.80452)(22.237061769616027, 64.61043000000001)(22.33722871452421, 65.5937)(22.437395659432386, 64.71786)(22.537562604340568, 66.13936)(22.637729549248746, 64.69588)(22.737896494156928, 65.17072999999999)(22.83806343906511, 63.76693)(22.93823038397329, 65.26534)(23.03839732888147, 65.17806)(23.13856427378965, 63.80904)(23.23873121869783, 64.09407)(23.33889816360601, 66.00935)(23.43906510851419, 66.05391)(23.53923205342237, 65.16279)(23.639398998330552, 63.98055)(23.739565943238734, 64.44319)(23.839732888146912, 64.83931)(23.939899833055094, 66.55074)(24.040066777963272, 65.72187)(24.140233722871454, 64.32356999999999)(24.24040066777963, 64.89364)(24.340567612687813, 64.86068)(24.440734557595995, 64.91744)(24.540901502504177, 65.16341)(24.641068447412355, 65.5107)(24.741235392320537, 64.0721)(24.841402337228715, 64.69649000000001)(24.941569282136896, 65.9062)(25.041736227045075, 65.82869)(25.141903171953256, 63.98666)(25.242070116861438, 65.25252)(25.34223706176962, 64.72883999999999)(25.442404006677798, 64.49508)(25.54257095158598, 65.22567)(25.642737896494157, 64.86556)(25.74290484140234, 65.63094000000001)(25.843071786310517, 65.03402)(25.9432387312187, 64.7563)(26.043405676126877, 64.55305)(26.143572621035062, 65.28975)(26.24373956594324, 65.79756)(26.34390651085142, 65.33675)(26.4440734557596, 64.73006)(26.544240400667782, 64.64095)(26.64440734557596, 65.95259)(26.744574290484138, 64.07943)(26.84474123539232, 65.47469000000001)(26.9449081803005, 72.68109)(27.045075125208683, 65.12129)(27.14524207011686, 64.77523)(27.245409015025043, 70.36664)(27.34557595993322, 85.32632)(27.445742904841403, 79.98026)(27.54590984974958, 69.00984)(27.646076794657763, 65.14571000000001)(27.746243739565944, 65.49116)(27.846410684474126, 65.42586)(27.946577629382304, 65.31905)(28.046744574290486, 63.86885)(28.146911519198664, 64.71297)(28.247078464106846, 66.96821)(28.347245409015024, 70.79267)(28.447412353923205, 73.93109)(28.547579298831387, 75.62359)(28.64774624373957, 73.33417)(28.747913188647747, 77.98686000000001)(28.84808013355593, 74.96074)(28.948247078464107, 71.74419999999999)(29.04841402337229, 72.25140999999999)(29.148580968280466, 72.76226)(29.248747913188648, 74.26678)(29.348914858096826, 74.55914)(29.44908180300501, 71.91021)(29.54924874791319, 74.88934)(29.64941569282137, 73.0473)(29.74958263772955, 78.72783)(29.84974958263773, 74.7636)(29.94991652754591, 65.20185)(30.05008347245409, 65.19819)(30.15025041736227, 64.81978)(30.25041736227045, 64.94246000000001)(30.35058430717863, 64.74043)(30.45075125208681, 64.51888)(30.55091819699499, 64.88569)(30.651085141903177, 64.98945)(30.751252086811355, 64.09224)(30.851419031719537, 64.66414)(30.951585976627715, 66.43721000000001)(31.051752921535897, 64.7325)(31.151919866444075, 64.99679)(31.252086811352257, 66.28462)(31.352253756260435, 73.96404000000001)(31.452420701168617, 69.76666)(31.552587646076795, 65.90682)(31.652754590984976, 64.22407)(31.752921535893154, 65.2635)(31.853088480801336, 65.32087)(31.953255425709514, 67.73909)(32.053422370617696, 64.52132)(32.15358931552588, 64.67818)(32.25375626043405, 65.07369)(32.35392320534224, 64.88508999999999)(32.45409015025042, 65.19391999999999)(32.554257095158604, 64.58052)(32.65442404006678, 65.11153)(32.75459098497496, 65.48812)(32.85475792988314, 64.38765000000001)(32.95492487479132, 65.77253999999999)(33.0550918196995, 65.32149000000001)(33.15525876460768, 65.82808)(33.25542570951586, 65.25312)(33.35559265442404, 65.222)(33.45575959933222, 64.42183)(33.5559265442404, 65.14509000000001)(33.65609348914858, 65.35749)(33.756260434056756, 66.72713)(33.85642737896494, 65.45332)(33.95659432387313, 64.99494999999999)(34.05676126878131, 64.62324)(34.15692821368948, 64.94491)(34.257095158597664, 64.73188999999999)(34.357262103505846, 65.59797999999999)(34.45742904841403, 64.90644999999999)(34.5575959933222, 64.01229)(34.657762938230384, 65.43929)(34.757929883138566, 64.92659)(34.85809682804675, 66.85957)(34.95826377295492, 65.17744)(35.058430717863104, 66.77351)(35.158597662771285, 64.58541)(35.25876460767947, 64.25641999999999)(35.35893155258764, 66.77595)(35.45909849749582, 64.40108000000001)(35.559265442404005, 65.70235)(35.659432387312194, 66.05696)(35.75959933222037, 66.55378999999999)(35.85976627712855, 64.30587)(35.95993322203673, 64.52009000000001)(36.06010016694491, 66.05024)(36.16026711185309, 65.50093000000001)(36.26043405676127, 65.82381)(36.36060100166945, 65.8055)(36.46076794657763, 64.96748)(36.56093489148581, 65.77986)(36.66110183639399, 65.08588999999999)(36.76126878130217, 65.70600999999999)(36.86143572621035, 65.13228)(36.96160267111853, 65.77131)(37.06176961602671, 65.02852)(37.16193656093489, 65.1103)(37.26210350584308, 65.63033)(37.362270450751254, 65.90376)(37.462437395659435, 65.15914000000001)(37.56260434056762, 64.95161999999999)(37.6627712854758, 64.8973)(37.76293823038397, 66.19307)(37.863105175292155, 65.23909)(37.96327212020034, 65.39046)(38.06343906510852, 64.71419)(38.16360601001669, 65.3166)(38.263772954924875, 65.51558)(38.363939899833056, 64.86922)(38.46410684474124, 65.41243)(38.56427378964941, 65.37214)(38.664440734557594, 64.47249000000001)(38.764607679465776, 65.94038)(38.86477462437396, 64.84908)(38.96494156928214, 65.86409)(39.06510851419032, 65.15364)(39.1652754590985, 66.69660999999999)(39.26544240400668, 67.05794)(39.36560934891486, 66.67157999999999)(39.46577629382304, 65.68464)(39.56594323873122, 65.83052)(39.666110183639404, 64.64034)(39.76627712854758, 65.15486)(39.86644407345576, 65.46674999999999)(39.96661101836394, 64.50422999999999)(40.066777963272116, 66.95296)(40.1669449081803, 65.24091999999999)(40.26711185308848, 66.23945)(40.36727879799666, 65.3929)(40.46744574290484, 66.25288)(40.567612687813025, 65.11702)(40.667779632721206, 65.56136000000001)(40.76794657762939, 64.95161999999999)(40.86811352253756, 65.20186)(40.968280467445744, 65.29097)(41.068447412353926, 65.3575)(41.16861435726211, 64.91255)(41.26878130217028, 65.10909)(41.368948247078464, 66.01118)(41.469115191986646, 65.68098)(41.56928213689483, 64.59517)(41.669449081803, 65.07307)(41.769616026711184, 65.73957)(41.869782971619365, 64.80330000000001)(41.96994991652755, 65.04256)(42.07011686143572, 65.72919999999999)(42.1702838063439, 65.12862)(42.27045075125209, 64.60615)(42.370617696160274, 65.55098)(42.47078464106845, 66.34566000000001)(42.57095158597663, 65.21101)(42.67111853088481, 65.18233000000001)(42.77128547579299, 64.65132)(42.87145242070117, 65.48506)(42.97161936560935, 64.97847)(43.07178631051753, 65.96785)(43.17195325542571, 66.30476)(43.27212020033389, 65.34956)(43.37228714524207, 66.9731)(43.47245409015025, 65.05476)(43.57262103505843, 66.61726)(43.67278797996661, 65.24032)(43.77295492487479, 65.60408)(43.87312186978298, 65.1097)(43.97328881469116, 65.21529)(44.073455759599334, 66.12837)(44.173622704507515, 65.03768)(44.2737896494157, 67.25691)(44.37395659432388, 65.29707)(44.47412353923205, 65.24886000000001)(44.574290484140235, 65.67122)(44.67445742904842, 64.70931)(44.7746243739566, 66.74971000000001)(44.87479131886477, 64.32661999999999)(44.974958263772955, 64.67757)(45.075125208681136, 65.21894999999999)(45.17529215358932, 72.5694)(45.27545909849749, 72.44305)(45.375626043405674, 65.62056)(45.475792988313856, 65.70478)(45.57595993322204, 66.11739)(45.67612687813022, 64.83626)(45.7762938230384, 65.60408)(45.87646076794658, 64.66719)(45.976627712854764, 65.6761)(46.07679465776294, 64.7441)(46.17696160267112, 65.43502000000001)(46.2771285475793, 66.90169)(46.37729549248748, 66.08503)(46.47746243739566, 65.25069)(46.57762938230384, 65.55037)(46.67779632721202, 65.26106999999999)(46.7779632721202, 64.29488)(46.87813021702838, 64.91438)(46.97829716193656, 65.6755)(47.07846410684474, 65.74568000000001)(47.17863105175292, 66.201)(47.278797996661105, 68.89997)(47.378964941569286, 65.87386000000001)(47.47913188647747, 64.85822999999999)(47.57929883138564, 66.08443)(47.679465776293824, 65.22444)(47.779632721202006, 65.45699)(47.87979966611019, 67.24653)(47.97996661101836, 65.31966)(48.080133555926544, 65.1927)(48.180300500834726, 65.2043)(48.28046744574291, 65.11764000000001)(48.38063439065108, 66.25167)(48.48080133555926, 64.45296)(48.580968280467445, 65.86165)(48.68113522537563, 65.41365)(48.7813021702838, 64.97359)(48.88146911519199, 65.49605)(48.98163606010017, 65.51558)(49.081803005008354, 65.29342)(49.18196994991653, 64.6971)(49.28213689482471, 65.34834000000001)(49.38230383973289, 65.1451)(49.48247078464107, 65.30074)(49.58263772954925, 66.5843)(49.68280467445743, 65.95442)(49.78297161936561, 65.22017)(49.88313856427379, 64.87716)(49.98330550918197, 65.32576)(50.08347245409015, 65.80183)(50.18363939899833, 64.92049)(50.28380634390651, 65.8116)(50.38397328881469, 65.61079000000001)(50.484140233722876, 65.50093000000001)(50.58430717863106, 65.78535)(50.68447412353924, 65.15059)(50.784641068447414, 65.00715)(50.884808013355595, 65.39595)(50.98497495826378, 65.41304)(51.08514190317196, 65.26044999999999)(51.18530884808013, 64.76668000000001)(51.285475792988315, 65.30806)(51.3856427378965, 64.91988)(51.48580968280468, 65.67488)(51.58597662771285, 65.43074)(51.686143572621035, 65.65413000000001)(51.786310517529216, 66.09846)(51.8864774624374, 65.16951)(51.98664440734557, 67.06648)(52.086811352253754, 64.84969000000001)(52.18697829716194, 65.24091999999999)(52.287145242070125, 65.86104)(52.3873121869783, 65.32271)(52.48747913188648, 66.25227)(52.58764607679466, 65.54366)(52.68781302170284, 66.40364)(52.78797996661102, 65.44478)(52.8881469115192, 64.86434)(52.98831385642738, 65.30074)(53.088480801335564, 65.96174)(53.18864774624374, 64.84908)(53.28881469115192, 66.30964)(53.3889816360601, 68.18831)(53.489148580968276, 64.98641)(53.58931552587646, 64.75447)(53.68948247078464, 66.61543)(53.78964941569283, 65.05171)(53.889816360601, 65.91719)(53.989983305509185, 66.1656)(54.090150250417366, 66.11617)(54.19031719532555, 65.92085)(54.29048414023372, 64.75264)(54.390651085141904, 65.58455000000001)(54.490818030050086, 64.82711)(54.59098497495827, 66.94868)(54.69115191986644, 64.79781)(54.791318864774624, 64.94246000000001)(54.891485809682806, 65.53573)(54.99165275459099, 66.12105)(55.09181969949916, 68.15778)(55.19198664440734, 65.94588)(55.292153589315525, 65.68890999999999)(55.39232053422371, 65.46003)(55.49248747913189, 65.02425)(55.59265442404007, 66.12654)(55.69282136894825, 65.82258999999999)(55.79298831385643, 64.79293)(55.89315525876461, 66.89619)(55.99332220367279, 65.88728)(56.09348914858097, 65.34651)(56.19365609348915, 64.95101)(56.29382303839733, 66.04903)(56.39398998330551, 64.87105)(56.49415692821369, 64.93514)(56.59432387312187, 65.28609)(56.69449081803005, 66.72834)(56.79465776293823, 65.22689)(56.89482470784641, 65.48994)(56.99499165275459, 65.5638)(57.095158597662774, 66.68867)(57.195325542570956, 65.46369999999999)(57.29549248747914, 66.58369)(57.39565943238732, 65.20735)(57.495826377295494, 66.00019999999999)(57.595993322203675, 65.62299999999999)(57.69616026711186, 66.05879)(57.79632721202004, 66.32735)(57.89649415692821, 65.37215)(57.996661101836395, 65.99348)(58.09682804674458, 68.37812)(58.19699499165276, 66.40303)(58.29716193656093, 68.07234)(58.397328881469114, 65.65779)(58.497495826377296, 65.70235)(58.59766277128548, 65.18233000000001)(58.69782971619365, 66.3304)(58.79799666110184, 65.41487000000001)(58.89816360601002, 64.68794)(58.9983305509182, 64.82466)(59.09849749582638, 65.69929)(59.19866444073456, 65.64192)(59.29883138564274, 65.15303)(59.398998330550924, 66.23091)(59.4991652754591, 65.06452999999999)(59.59933222036728, 65.68769999999999)(59.69949916527546, 65.80671)(59.79966611018364, 64.3138)(59.89983305509182, 65.98188999999999)(60.0, 66.53059)
            };
            \addplot[color=blue, mark=none,name path=B] coordinates { %% MIN value
            (0.0, 0.0)(0.1001669449081803, 54.10142)(0.2003338898163606, 59.94491)(0.3005008347245409, 55.776830000000004)(0.4006677796327212, 55.622420000000005)(0.5008347245409015, 56.45737)(0.6010016694490818, 56.42381)(0.7011686143572621, 55.197010000000006)(0.8013355592654424, 57.03538)(0.9015025041736228, 57.2905)(1.001669449081803, 57.02134)(1.1018363939899833, 56.03868)(1.2020033388981637, 56.13571999999999)(1.3021702838063438, 56.39512)(1.4023372287145242, 56.62705)(1.5025041736227045, 56.64597)(1.6026711185308848, 56.96885)(1.7028380634390652, 57.189189999999996)(1.8030050083472455, 56.89073)(1.9031719532554257, 56.14487)(2.003338898163606, 57.076269999999994)(2.1035058430717863, 56.14488)(2.2036727879799667, 57.30637)(2.303839732888147, 56.754619999999996)(2.4040066777963274, 55.47288)(2.5041736227045073, 56.259620000000005)(2.6043405676126876, 57.06102)(2.704507512520868, 55.73106)(2.8046744574290483, 56.69663)(2.9048414023372287, 56.15647)(3.005008347245409, 56.505590000000005)(3.1051752921535893, 55.70359)(3.2053422370617697, 45.392340000000004)(3.30550918196995, 56.75279)(3.4056761268781304, 56.65757)(3.5058430717863107, 56.80466)(3.606010016694491, 56.75339)(3.7061769616026714, 55.32274)(3.8063439065108513, 55.488749999999996)(3.906510851419032, 57.67869)(4.006677796327212, 57.173930000000006)(4.106844741235393, 55.25742)(4.207011686143573, 57.32651)(4.3071786310517535, 57.769009999999994)(4.407345575959933, 56.89622)(4.507512520868114, 56.274879999999996)(4.607679465776294, 56.603849999999994)(4.707846410684475, 57.011570000000006)(4.808013355592655, 56.56358)(4.908180300500835, 56.35727)(5.0083472454090145, 57.13121)(5.108514190317195, 57.266690000000004)(5.208681135225375, 55.7512)(5.308848080133556, 56.18272)(5.409015025041736, 57.26914)(5.509181969949917, 56.22117)(5.609348914858097, 56.431740000000005)(5.709515859766277, 56.53733)(5.809682804674457, 56.5709)(5.909849749582638, 57.02439)(6.010016694490818, 56.75584)(6.110183639398999, 56.875460000000004)(6.210350584307179, 56.61485)(6.3105175292153595, 57.24412)(6.410684474123539, 57.45224999999999)(6.510851419031719, 57.543800000000005)(6.6110183639399, 56.45188)(6.71118530884808, 57.04759)(6.811352253756261, 56.559309999999996)(6.911519198664441, 56.40244)(7.011686143572621, 56.97068)(7.111853088480801, 57.30332)(7.212020033388982, 57.028059999999996)(7.312186978297162, 57.0842)(7.412353923205343, 57.562110000000004)(7.512520868113523, 57.360699999999994)(7.612687813021703, 57.17026)(7.712854757929884, 56.01609)(7.813021702838064, 56.46226)(7.913188647746244, 56.78269)(8.013355592654424, 57.03843)(8.113522537562606, 57.130590000000005)(8.213689482470786, 56.1345)(8.313856427378965, 56.65818)(8.414023372287145, 55.82566)(8.514190317195327, 57.20078)(8.614357262103507, 57.092749999999995)(8.714524207011687, 56.79307)(8.814691151919867, 56.57761)(8.914858096828048, 57.12876)(9.015025041736228, 57.81785)(9.115191986644408, 57.20689)(9.215358931552588, 56.333479999999994)(9.31552587646077, 56.55259)(9.41569282136895, 57.1135)(9.51585976627713, 56.515969999999996)(9.61602671118531, 56.49583)(9.71619365609349, 57.3082)(9.81636060100167, 57.63474000000001)(9.916527545909851, 57.03049)(10.016694490818029, 55.695660000000004)(10.11686143572621, 56.96946)(10.21702838063439, 58.26096)(10.31719532554257, 56.887679999999996)(10.41736227045075, 56.99509)(10.51752921535893, 57.20444)(10.617696160267112, 58.16269)(10.717863105175292, 56.715559999999996)(10.818030050083472, 57.28928)(10.918196994991652, 57.31004)(11.018363939899833, 57.713469999999994)(11.118530884808013, 57.40098)(11.218697829716193, 58.001560000000005)(11.318864774624373, 58.18894)(11.419031719532555, 57.5261)(11.519198664440735, 57.2551)(11.619365609348915, 57.399750000000004)(11.719532554257095, 57.51694)(11.819699499165276, 56.38901)(11.919866444073456, 57.91245)(12.020033388981636, 57.24534)(12.120200333889816, 57.19529)(12.220367278797998, 57.78672)(12.320534223706177, 57.131809999999994)(12.420701168614357, 57.369240000000005)(12.520868113522537, 57.57675)(12.621035058430719, 58.36045)(12.721202003338899, 57.87339)(12.821368948247079, 57.16905)(12.921535893155259, 57.78306)(13.021702838063439, 57.026219999999995)(13.12186978297162, 57.387550000000005)(13.2220367278798, 57.97288)(13.32220367278798, 56.64597)(13.42237061769616, 57.93809)(13.522537562604342, 57.83248999999999)(13.622704507512521, 57.49008)(13.722871452420701, 57.51634)(13.823038397328881, 57.14095999999999)(13.923205342237063, 57.392430000000004)(14.023372287145243, 57.46262)(14.123539232053423, 57.08054)(14.223706176961603, 56.98228)(14.323873121869784, 57.640840000000004)(14.424040066777964, 58.31895)(14.524207011686144, 57.41867)(14.624373956594324, 58.182219999999994)(14.724540901502506, 57.01523)(14.824707846410686, 57.628029999999995)(14.924874791318866, 58.0864)(15.025041736227045, 57.071389999999994)(15.125208681135225, 55.793929999999996)(15.225375626043405, 57.54929)(15.325542570951589, 55.63157)(15.425709515859769, 58.17429)(15.525876460767948, 57.014630000000004)(15.626043405676128, 57.12632)(15.726210350584308, 56.41648)(15.826377295492488, 56.450050000000005)(15.926544240400668, 57.081160000000004)(16.026711185308848, 57.47971)(16.126878130217026, 56.97923)(16.22704507512521, 56.438449999999996)(16.32721202003339, 57.20078)(16.42737896494157, 57.543800000000005)(16.52754590984975, 57.47605)(16.62771285475793, 57.6677)(16.72787979966611, 57.454679999999996)(16.82804674457429, 57.131809999999994)(16.92821368948247, 56.4763)(17.028380634390654, 56.571509999999996)(17.128547579298832, 58.00705)(17.228714524207014, 56.526959999999995)(17.328881469115192, 56.67344)(17.429048414023374, 57.14219)(17.529215358931552, 58.02537)(17.629382303839733, 57.86302)(17.72954924874791, 57.51023)(17.829716193656097, 57.61582)(17.929883138564275, 57.075050000000005)(18.030050083472457, 58.13645)(18.130217028380635, 58.111430000000006)(18.230383973288816, 56.84616)(18.330550918196995, 56.68748)(18.430717863105176, 56.004490000000004)(18.530884808013354, 57.667080000000006)(18.63105175292154, 57.83615)(18.731218697829718, 58.16941)(18.8313856427379, 56.79429)(18.931552587646078, 57.343599999999995)(19.03171953255426, 58.27256)(19.131886477462437, 56.59348)(19.23205342237062, 57.98386)(19.332220367278797, 56.327369999999995)(19.43238731218698, 57.97288)(19.53255425709516, 57.69883)(19.63272120200334, 58.38058)(19.73288814691152, 58.047940000000004)(19.833055091819702, 56.807100000000005)(19.93322203672788, 56.25352)(20.033388981636058, 56.61851)(20.13355592654424, 56.86814)(20.23372287145242, 55.37461999999999)(20.333889816360603, 57.173930000000006)(20.43405676126878, 56.364599999999996)(20.534223706176963, 57.272800000000004)(20.63439065108514, 58.18101)(20.734557595993323, 57.532199999999996)(20.8347245409015, 58.45017)(20.934891485809683, 57.700050000000005)(21.03505843071786, 58.571020000000004)(21.135225375626046, 58.26462)(21.235392320534224, 57.494969999999995)(21.335559265442406, 56.36399)(21.435726210350584, 57.555389999999996)(21.535893155258766, 58.03207999999999)(21.636060100166944, 57.90208)(21.736227045075125, 56.54405)(21.836393989983303, 57.65304999999999)(21.93656093489149, 58.033910000000006)(22.036727879799667, 58.708349999999996)(22.13689482470785, 57.76353)(22.237061769616027, 57.72507)(22.33722871452421, 57.77879)(22.437395659432386, 57.72018)(22.537562604340568, 57.928309999999996)(22.637729549248746, 56.472030000000004)(22.737896494156928, 56.84251)(22.83806343906511, 58.381809999999994)(22.93823038397329, 57.69394)(23.03839732888147, 57.84226)(23.13856427378965, 58.30979)(23.23873121869783, 58.83896)(23.33889816360601, 57.08786)(23.43906510851419, 57.387550000000005)(23.53923205342237, 57.562110000000004)(23.639398998330552, 45.786629999999995)(23.739565943238734, 57.14951)(23.839732888146912, 57.37351)(23.939899833055094, 58.024139999999996)(24.040066777963272, 58.3464)(24.140233722871454, 57.865460000000006)(24.24040066777963, 58.10655)(24.340567612687813, 56.47813)(24.440734557595995, 57.886810000000004)(24.540901502504177, 58.22678)(24.641068447412355, 58.24082)(24.741235392320537, 57.945409999999995)(24.841402337228715, 57.77024)(24.941569282136896, 58.14498999999999)(25.041736227045075, 58.08518)(25.141903171953256, 58.51974)(25.242070116861438, 57.92587)(25.34223706176962, 58.257909999999995)(25.442404006677798, 58.016819999999996)(25.54257095158598, 56.534890000000004)(25.642737896494157, 57.82089)(25.74290484140234, 58.00584)(25.843071786310517, 56.87852)(25.9432387312187, 56.99326)(26.043405676126877, 57.227639999999994)(26.143572621035062, 58.26828)(26.24373956594324, 58.422090000000004)(26.34390651085142, 58.067479999999996)(26.4440734557596, 58.40622)(26.544240400667782, 58.120580000000004)(26.64440734557596, 58.09372)(26.744574290484138, 56.89805)(26.84474123539232, 58.29086)(26.9449081803005, 58.27377)(27.045075125208683, 57.84714)(27.14524207011686, 58.654019999999996)(27.245409015025043, 57.72568)(27.34557595993322, 58.15659)(27.445742904841403, 56.623999999999995)(27.54590984974958, 57.900850000000005)(27.646076794657763, 57.97898)(27.746243739565944, 57.37473)(27.846410684474126, 58.31589)(27.946577629382304, 58.18588)(28.046744574290486, 57.47421)(28.146911519198664, 58.04184)(28.247078464106846, 58.63266)(28.347245409015024, 58.228609999999996)(28.447412353923205, 57.94663)(28.547579298831387, 58.210300000000004)(28.64774624373957, 57.9216)(28.747913188647747, 58.26523)(28.84808013355593, 57.98142)(28.948247078464107, 57.59262)(29.04841402337229, 58.11142)(29.148580968280466, 57.213589999999996)(29.248747913188648, 57.701260000000005)(29.348914858096826, 57.61216)(29.44908180300501, 58.44956)(29.54924874791319, 57.61888)(29.64941569282137, 58.253029999999995)(29.74958263772955, 58.34335)(29.84974958263773, 58.374480000000005)(29.94991652754591, 57.889860000000006)(30.05008347245409, 57.812960000000004)(30.15025041736227, 58.76022)(30.25041736227045, 57.945409999999995)(30.35058430717863, 57.308820000000004)(30.45075125208681, 57.1782)(30.55091819699499, 57.08909)(30.651085141903177, 57.90024)(30.751252086811355, 58.86947)(30.851419031719537, 57.61338000000001)(30.951585976627715, 58.44894)(31.051752921535897, 57.66831)(31.151919866444075, 57.84226)(31.252086811352257, 58.48984)(31.352253756260435, 58.50815)(31.452420701168617, 58.18772)(31.552587646076795, 57.68784)(31.652754590984976, 57.37046)(31.752921535893154, 57.10801)(31.853088480801336, 58.25302)(31.953255425709514, 57.64207)(32.053422370617696, 58.35678)(32.15358931552588, 58.162079999999996)(32.25375626043405, 58.3458)(32.35392320534224, 58.74619)(32.45409015025042, 58.046119999999995)(32.554257095158604, 57.63168)(32.65442404006678, 57.638400000000004)(32.75459098497496, 58.38547)(32.85475792988314, 58.25058)(32.95492487479132, 57.85019)(33.0550918196995, 57.603)(33.15525876460768, 58.574070000000006)(33.25542570951586, 57.096410000000006)(33.35559265442404, 58.73276)(33.45575959933222, 58.16391)(33.5559265442404, 58.60214)(33.65609348914858, 57.82517)(33.756260434056756, 57.9094)(33.85642737896494, 57.716519999999996)(33.95659432387313, 57.60911)(34.05676126878131, 57.34299)(34.15692821368948, 58.60825)(34.257095158597664, 57.49375)(34.357262103505846, 57.73484)(34.45742904841403, 57.23313)(34.5575959933222, 57.70677)(34.657762938230384, 57.9503)(34.757929883138566, 57.78184)(34.85809682804675, 57.97165)(34.95826377295492, 56.86142)(35.058430717863104, 58.113859999999995)(35.158597662771285, 58.14744)(35.25876460767947, 57.1544)(35.35893155258764, 58.40318)(35.45909849749582, 58.03696000000001)(35.559265442404005, 57.759859999999996)(35.659432387312194, 58.63571)(35.75959933222037, 57.62802)(35.85976627712855, 59.440160000000006)(35.95993322203673, 57.07078)(36.06010016694491, 57.80259)(36.16026711185309, 56.5355)(36.26043405676127, 57.74521)(36.36060100166945, 58.45078)(36.46076794657763, 56.82176)(36.56093489148581, 58.28293000000001)(36.66110183639399, 57.916109999999996)(36.76126878130217, 58.53195)(36.86143572621035, 57.9802)(36.96160267111853, 58.347629999999995)(37.06176961602671, 57.74888)(37.16193656093489, 57.428439999999995)(37.26210350584308, 57.44798)(37.362270450751254, 57.58164)(37.462437395659435, 57.98873999999999)(37.56260434056762, 58.48679)(37.6627712854758, 57.69455)(37.76293823038397, 58.09311)(37.863105175292155, 58.080290000000005)(37.96327212020034, 57.74276999999999)(38.06343906510852, 57.97898)(38.16360601001669, 57.92466)(38.263772954924875, 57.524879999999996)(38.363939899833056, 57.51817)(38.46410684474124, 57.32895)(38.56427378964941, 57.548069999999996)(38.664440734557594, 58.37021)(38.764607679465776, 58.58994)(38.86477462437396, 58.231049999999996)(38.96494156928214, 57.42356)(39.06510851419032, 56.855320000000006)(39.1652754590985, 58.660740000000004)(39.26544240400668, 58.515480000000004)(39.36560934891486, 58.33848)(39.46577629382304, 57.368629999999996)(39.56594323873122, 57.58775)(39.666110183639404, 58.06564)(39.76627712854758, 57.92465)(39.86644407345576, 58.23227)(39.96661101836394, 58.62228)(40.066777963272116, 57.89109)(40.1669449081803, 58.11936)(40.26711185308848, 57.587140000000005)(40.36727879799666, 57.69089)(40.46744574290484, 57.94785)(40.567612687813025, 58.67538)(40.667779632721206, 58.66868)(40.76794657762939, 57.93076)(40.86811352253756, 57.74705)(40.968280467445744, 57.93748)(41.068447412353926, 57.9802)(41.16861435726211, 57.55845)(41.26878130217028, 58.261570000000006)(41.368948247078464, 57.86362)(41.469115191986646, 57.607879999999994)(41.56928213689483, 57.711639999999996)(41.669449081803, 57.6854)(41.769616026711184, 58.40622)(41.869782971619365, 58.14194)(41.96994991652755, 57.52854)(42.07011686143572, 57.985079999999996)(42.1702838063439, 58.784029999999994)(42.27045075125209, 57.47483)(42.370617696160274, 57.94968)(42.47078464106845, 58.699799999999996)(42.57095158597663, 58.722390000000004)(42.67111853088481, 58.202369999999995)(42.77128547579299, 58.64426)(42.87145242070117, 58.41659)(42.97161936560935, 58.22983)(43.07178631051753, 57.5145)(43.17195325542571, 58.627770000000005)(43.27212020033389, 57.65793000000001)(43.37228714524207, 58.257909999999995)(43.47245409015025, 57.186130000000006)(43.57262103505843, 58.462990000000005)(43.67278797996661, 57.89597)(43.77295492487479, 58.55576)(43.87312186978298, 58.415369999999996)(43.97328881469116, 57.52548)(44.073455759599334, 57.8624)(44.173622704507515, 57.116550000000004)(44.2737896494157, 57.47483)(44.37395659432388, 57.62985)(44.47412353923205, 56.72349)(44.574290484140235, 57.63962)(44.67445742904842, 57.936249999999994)(44.7746243739566, 58.80417)(44.87479131886477, 58.23532)(44.974958263772955, 57.62802)(45.075125208681136, 58.281099999999995)(45.17529215358932, 58.43369)(45.27545909849749, 58.297579999999996)(45.375626043405674, 57.651219999999995)(45.475792988313856, 58.08945)(45.57595993322204, 57.60483)(45.67612687813022, 57.76353)(45.7762938230384, 58.11203)(45.87646076794658, 58.08274)(45.976627712854764, 58.07602)(46.07679465776294, 58.43674)(46.17696160267112, 58.19443)(46.2771285475793, 57.78366)(46.37729549248748, 58.55454)(46.47746243739566, 58.24998)(46.57762938230384, 58.173069999999996)(46.67779632721202, 59.00924)(46.7779632721202, 57.196509999999996)(46.87813021702838, 57.95151)(46.97829716193656, 58.75717)(47.07846410684474, 57.86606)(47.17863105175292, 58.0272)(47.278797996661105, 58.4752)(47.378964941569286, 58.39829)(47.47913188647747, 58.441010000000006)(47.57929883138564, 57.900850000000005)(47.679465776293824, 58.113859999999995)(47.779632721202006, 59.20517)(47.87979966611019, 57.90025)(47.97996661101836, 58.56492)(48.080133555926544, 58.23532)(48.180300500834726, 56.73447)(48.28046744574291, 58.20297)(48.38063439065108, 57.93565)(48.48080133555926, 58.43369)(48.580968280467445, 58.5405)(48.68113522537563, 58.492290000000004)(48.7813021702838, 58.5173)(48.88146911519199, 57.442479999999996)(48.98163606010017, 58.7114)(49.081803005008354, 58.063810000000004)(49.18196994991653, 58.015600000000006)(49.28213689482471, 58.26707)(49.38230383973289, 58.87314)(49.48247078464107, 58.709559999999996)(49.58263772954925, 58.16696999999999)(49.68280467445743, 59.44321000000001)(49.78297161936561, 56.68016)(49.88313856427379, 58.11875)(49.98330550918197, 59.05869)(50.08347245409015, 58.60703)(50.18363939899833, 57.50718)(50.28380634390651, 57.86363)(50.38397328881469, 58.46665)(50.484140233722876, 57.97287)(50.58430717863106, 57.60544)(50.68447412353924, 58.098)(50.784641068447414, 58.04307)(50.884808013355595, 58.00767)(50.98497495826378, 57.71226)(51.08514190317196, 57.77756)(51.18530884808013, 58.24204)(51.285475792988315, 57.681740000000005)(51.3856427378965, 58.343360000000004)(51.48580968280468, 58.384859999999996)(51.58597662771285, 57.9216)(51.686143572621035, 58.938449999999996)(51.786310517529216, 58.22495)(51.8864774624374, 58.00828)(51.98664440734557, 58.57895)(52.086811352253754, 57.93564)(52.18697829716194, 59.130700000000004)(52.287145242070125, 58.87069)(52.3873121869783, 56.90659)(52.48747913188648, 57.67563)(52.58764607679466, 57.922219999999996)(52.68781302170284, 59.52255)(52.78797996661102, 58.08823)(52.8881469115192, 58.73642)(52.98831385642738, 58.61252)(53.088480801335564, 58.4929)(53.18864774624374, 58.23532)(53.28881469115192, 57.21848)(53.3889816360601, 58.01865)(53.489148580968276, 58.9537)(53.58931552587646, 56.3054)(53.68948247078464, 58.68088)(53.78964941569283, 57.7855)(53.889816360601, 58.493500000000004)(53.989983305509185, 57.922830000000005)(54.090150250417366, 58.77792)(54.19031719532555, 58.19626)(54.29048414023372, 58.11813)(54.390651085141904, 58.248749999999994)(54.490818030050086, 57.71165)(54.59098497495827, 59.04038)(54.69115191986644, 57.01218)(54.791318864774624, 58.67783)(54.891485809682806, 58.57285)(54.99165275459099, 58.01804)(55.09181969949916, 59.18869)(55.19198664440734, 57.31675)(55.292153589315525, 58.805389999999996)(55.39232053422371, 58.49655)(55.49248747913189, 57.34971)(55.59265442404007, 58.58323)(55.69282136894825, 59.33823)(55.79298831385643, 58.05588)(55.89315525876461, 58.994600000000005)(55.99332220367279, 58.07297)(56.09348914858097, 57.480320000000006)(56.19365609348915, 56.963359999999994)(56.29382303839733, 58.20725)(56.39398998330551, 58.80051)(56.49415692821369, 58.2512)(56.59432387312187, 58.72483)(56.69449081803005, 57.32468)(56.79465776293823, 59.384)(56.89482470784641, 58.768159999999995)(56.99499165275459, 58.18345)(57.095158597662774, 58.397059999999996)(57.195325542570956, 59.1655)(57.29549248747914, 57.62742)(57.39565943238732, 58.35373)(57.495826377295494, 58.36106)(57.595993322203675, 58.17429)(57.69616026711186, 59.034890000000004)(57.79632721202004, 57.941739999999996)(57.89649415692821, 58.275)(57.996661101836395, 57.227019999999996)(58.09682804674458, 58.76755)(58.19699499165276, 56.53184)(58.29716193656093, 57.978370000000005)(58.397328881469114, 58.08518)(58.497495826377296, 57.91245)(58.59766277128548, 58.69187)(58.69782971619365, 57.745819999999995)(58.79799666110184, 58.24387)(58.89816360601002, 58.024139999999996)(58.9983305509182, 58.21396)(59.09849749582638, 59.01962)(59.19866444073456, 58.34153)(59.29883138564274, 58.536229999999996)(59.398998330550924, 58.37693)(59.4991652754591, 57.90329)(59.59933222036728, 58.00217)(59.69949916527546, 58.85178)(59.79966611018364, 58.32199)(59.89983305509182, 59.00436)(60.0, 57.72873)
            };
            \addplot [pattern=north east lines,pattern color=red] 
            fill between [
                of=A and B,soft clip={domain=0:800},
            ];
            \end{axis}
    \end{tikzpicture}
    \caption{Test case: BinaryTrees}      
\end{subfigure}
\begin{subfigure}[b]{0.49\linewidth}
    \begin{tikzpicture}
        \pgfplotsset{%
        width=1\linewidth,
        % height=1\textheight
            }
        \begin{axis}[ymax=120,
        xlabel={Time (Seconds)},
        ylabel={Energy Consumption (Joules)},
        ]
        \addplot[color=blue, mark=none,] coordinates { %% AVG value
        (0.0, 0.0)(0.1001669449081803, 53.5840155)(0.2003338898163606, 55.72025533333331)(0.3005008347245409, 54.29596274999999)(0.4006677796327212, 54.17243349999999)(0.5008347245409015, 54.39149749999999)(0.6010016694490818, 54.52888275000001)(0.7011686143572621, 54.35835600000003)(0.8013355592654424, 54.56871291666665)(0.9015025041736228, 55.19686458333333)(1.001669449081803, 54.6910175)(1.1018363939899833, 54.36337608333337)(1.2020033388981637, 54.62761141666666)(1.3021702838063438, 54.82617908333332)(1.4023372287145242, 54.41876583333336)(1.5025041736227045, 54.600440333333346)(1.6026711185308848, 54.90785433333334)(1.7028380634390652, 54.643892916666644)(1.8030050083472455, 54.61003375000001)(1.9031719532554257, 54.58265916666667)(2.003338898163606, 54.83299958333337)(2.1035058430717863, 54.629946833333314)(2.2036727879799667, 54.77681708333332)(2.303839732888147, 54.86027725000003)(2.4040066777963274, 54.83449991666666)(2.5041736227045073, 54.807533)(2.6043405676126876, 54.86174166666669)(2.704507512520868, 54.675000999999995)(2.8046744574290483, 54.90246258333331)(2.9048414023372287, 54.833904416666634)(3.005008347245409, 54.89016941666669)(3.1051752921535893, 54.70429766666665)(3.2053422370617697, 54.79643425000003)(3.30550918196995, 54.84716033333336)(3.4056761268781304, 54.84124924999996)(3.5058430717863107, 55.099244583333345)(3.606010016694491, 54.849326250000004)(3.7061769616026714, 54.54544366666666)(3.8063439065108513, 54.88116608333333)(3.906510851419032, 54.83131125000002)(4.006677796327212, 54.90941041666665)(4.106844741235393, 54.70105200000002)(4.207011686143573, 55.31852858333332)(4.3071786310517535, 54.582298083333335)(4.407345575959933, 54.89266666666666)(4.507512520868114, 54.696871166666675)(4.607679465776294, 54.90259483333331)(4.707846410684475, 54.962277000000014)(4.808013355592655, 54.708447166666666)(4.908180300500835, 54.99969683333335)(5.0083472454090145, 55.022594583333316)(5.108514190317195, 54.892168000000005)(5.208681135225375, 54.91504583333336)(5.308848080133556, 55.061824833333354)(5.409015025041736, 54.91238566666666)(5.509181969949917, 55.01425391666665)(5.609348914858097, 54.93634741666664)(5.709515859766277, 55.00115075000001)(5.809682804674457, 54.79202466666667)(5.909849749582638, 54.732862)(6.010016694490818, 54.902212583333345)(6.110183639398999, 54.83361558333332)(6.210350584307179, 54.893770916666675)(6.3105175292153595, 55.048208749999986)(6.410684474123539, 54.87101424999999)(6.510851419031719, 55.016236833333295)(6.6110183639399, 54.99762108333333)(6.71118530884808, 55.160646000000035)(6.811352253756261, 54.807481833333334)(6.911519198664441, 54.979666333333334)(7.011686143572621, 55.02417658333332)(7.111853088480801, 55.125154333333306)(7.212020033388982, 55.1449646666667)(7.312186978297162, 54.83642291666665)(7.412353923205343, 54.831112250000004)(7.512520868113523, 55.021044)(7.612687813021703, 54.99194991666669)(7.712854757929884, 54.980165000000035)(7.813021702838064, 54.97436725000001)(7.913188647746244, 55.005927416666665)(8.013355592654424, 54.94434774999999)(8.113522537562606, 54.8955605)(8.213689482470786, 55.02828625000003)(8.313856427378965, 54.96857808333333)(8.414023372287145, 54.85219558333335)(8.514190317195327, 55.0554774166667)(8.614357262103507, 54.96483516666667)(8.714524207011687, 54.83927591666666)(8.814691151919867, 54.92810283333336)(8.914858096828048, 55.019405333333324)(9.015025041736228, 55.28858566666667)(9.115191986644408, 54.91099183333332)(9.215358931552588, 55.084209333333334)(9.31552587646077, 54.977500249999956)(9.41569282136895, 55.1323661666667)(9.51585976627713, 54.88728025000001)(9.61602671118531, 55.24982824999998)(9.71619365609349, 55.16230933333333)(9.81636060100167, 54.97804358333336)(9.916527545909851, 55.196239916666656)(10.016694490818029, 55.02261483333335)(10.11686143572621, 55.01549916666666)(10.21702838063439, 55.115698416666646)(10.31719532554257, 55.120154416666665)(10.41736227045075, 55.11674108333332)(10.51752921535893, 54.99466624999997)(10.617696160267112, 55.138830666666664)(10.717863105175292, 55.034390166666654)(10.818030050083472, 55.02306716666664)(10.918196994991652, 55.09052683333332)(11.018363939899833, 55.08407699999999)(11.118530884808013, 54.93240066666669)(11.218697829716193, 55.04428741666666)(11.318864774624373, 54.98851175)(11.419031719532555, 55.054673833333325)(11.519198664440735, 54.777432499999996)(11.619365609348915, 55.09144758333333)(11.719532554257095, 54.98673108333335)(11.819699499165276, 55.083680916666665)(11.919866444073456, 55.06432183333333)(12.020033388981636, 55.00582091666668)(12.120200333889816, 55.09072483333333)(12.220367278797998, 55.07700749999999)(12.320534223706177, 55.12447741666666)(12.420701168614357, 55.03979633333331)(12.520868113522537, 54.976014500000005)(12.621035058430719, 55.075568249999975)(12.721202003338899, 54.95737341666665)(12.821368948247079, 55.05534500000002)(12.921535893155259, 54.96135158333332)(13.021702838063439, 55.04045274999999)(13.12186978297162, 55.05490216666665)(13.2220367278798, 54.98446824999998)(13.32220367278798, 55.12063224999999)(13.42237061769616, 55.192023166666665)(13.522537562604342, 55.07675333333331)(13.622704507512521, 54.99672541666666)(13.722871452420701, 55.10124874999998)(13.823038397328881, 55.070100666666704)(13.923205342237063, 55.173656)(14.023372287145243, 55.21348675)(14.123539232053423, 55.29680508333333)(14.223706176961603, 55.03575783333333)(14.323873121869784, 55.05864108333331)(14.424040066777964, 55.10410700000001)(14.524207011686144, 55.07101083333334)(14.624373956594324, 55.045874)(14.724540901502506, 55.04800066666664)(14.824707846410686, 55.00701074999997)(14.924874791318866, 55.22235166666669)(15.025041736227045, 55.233333583333305)(15.125208681135225, 55.157681)(15.225375626043405, 54.991115)(15.325542570951589, 55.26854150000001)(15.425709515859769, 55.010020833333336)(15.525876460767948, 55.203191833333335)(15.626043405676128, 55.02764549999999)(15.726210350584308, 55.01284958333333)(15.826377295492488, 55.203639750000015)(15.926544240400668, 55.145554666666676)(16.026711185308848, 55.19422041666663)(16.126878130217026, 55.18234366666666)(16.22704507512521, 55.24908008333329)(16.32721202003339, 55.23560691666667)(16.42737896494157, 55.109417)(16.52754590984975, 55.12059716666665)(16.62771285475793, 55.26871308333331)(16.72787979966611, 55.16902266666668)(16.82804674457429, 55.178122416666696)(16.92821368948247, 55.19134091666667)(17.028380634390654, 55.13810841666668)(17.128547579298832, 55.143531)(17.228714524207014, 55.16240041666667)(17.328881469115192, 55.34205683333331)(17.429048414023374, 55.106884)(17.529215358931552, 55.154496166666654)(17.629382303839733, 55.10891883333334)(17.72954924874791, 55.154048916666646)(17.829716193656097, 55.05569591666665)(17.929883138564275, 55.248541416666676)(18.030050083472457, 55.26461408333335)(18.130217028380635, 54.96902058333331)(18.230383973288816, 55.12925874999999)(18.330550918196995, 55.184296833333356)(18.430717863105176, 55.03684641666668)(18.530884808013354, 54.883231666666674)(18.63105175292154, 55.20223591666668)(18.731218697829718, 55.27437983333334)(18.8313856427379, 55.258515249999974)(18.931552587646078, 55.12009833333338)(19.03171953255426, 55.29961141666668)(19.131886477462437, 55.13652675)(19.23205342237062, 55.24638966666665)(19.332220367278797, 55.177863)(19.43238731218698, 55.188218666666664)(19.53255425709516, 55.147858750000005)(19.63272120200334, 55.14310291666668)(19.73288814691152, 55.10800808333333)(19.833055091819702, 55.090755583333355)(19.93322203672788, 55.095603000000004)(20.033388981636058, 55.19572549999999)(20.13355592654424, 55.44799858333332)(20.23372287145242, 55.23467591666667)(20.333889816360603, 55.19797874999999)(20.43405676126878, 55.418676249999976)(20.534223706176963, 55.34154324999999)(20.63439065108514, 55.28960816666668)(20.734557595993323, 55.380000166666626)(20.8347245409015, 55.14674041666669)(20.934891485809683, 55.25951191666664)(21.03505843071786, 55.39900291666667)(21.135225375626046, 55.12491925000002)(21.235392320534224, 55.27579341666664)(21.335559265442406, 55.13783416666666)(21.435726210350584, 55.14786416666668)(21.535893155258766, 55.26862091666667)(21.636060100166944, 55.04994941666665)(21.736227045075125, 55.19732733333333)(21.836393989983303, 55.31033425000002)(21.93656093489149, 55.066656916666645)(22.036727879799667, 55.14451716666667)(22.13689482470785, 55.18697199999998)(22.237061769616027, 55.286561250000005)(22.33722871452421, 55.25364783333332)(22.437395659432386, 55.20420908333331)(22.537562604340568, 55.242433166666686)(22.637729549248746, 54.992468750000015)(22.737896494156928, 55.20451466666669)(22.83806343906511, 55.20031341666666)(22.93823038397329, 55.12455358333331)(23.03839732888147, 55.253586750000004)(23.13856427378965, 55.24243766666669)(23.23873121869783, 55.19120383333336)(23.33889816360601, 55.315471583333355)(23.43906510851419, 55.13387166666667)(23.53923205342237, 55.16093599999999)(23.639398998330552, 55.23414149999999)(23.739565943238734, 55.319519750000026)(23.839732888146912, 55.33522625000002)(23.939899833055094, 55.52624566666665)(24.040066777963272, 55.269282916666675)(24.140233722871454, 55.07584774999999)(24.24040066777963, 55.287873166666664)(24.340567612687813, 55.287985166666665)(24.440734557595995, 55.26367241666666)(24.540901502504177, 55.41514683333331)(24.641068447412355, 55.27703425000002)(24.741235392320537, 55.25246741666668)(24.841402337228715, 55.30407283333337)(24.941569282136896, 55.35139549999999)(25.041736227045075, 55.26467516666667)(25.141903171953256, 55.19678358333333)(25.242070116861438, 55.35075441666667)(25.34223706176962, 55.259893999999996)(25.442404006677798, 55.156098833333346)(25.54257095158598, 55.20762683333332)(25.642737896494157, 55.38262491666667)(25.74290484140234, 55.25892258333331)(25.843071786310517, 55.43783066666665)(25.9432387312187, 55.22175183333333)(26.043405676126877, 55.427999583333346)(26.143572621035062, 55.195323583333334)(26.24373956594324, 55.33722041666668)(26.34390651085142, 55.32041983333332)(26.4440734557596, 55.46670108333334)(26.544240400667782, 55.351882916666646)(26.64440734557596, 55.37789533333332)(26.744574290484138, 55.69514441666664)(26.84474123539232, 55.285437)(26.9449081803005, 55.40856483333334)(27.045075125208683, 55.27883466666666)(27.14524207011686, 55.33076025000002)(27.245409015025043, 55.29639783333332)(27.34557595993322, 55.522059333333324)(27.445742904841403, 55.35286525)(27.54590984974958, 55.38902383333335)(27.646076794657763, 55.36504666666666)(27.746243739565944, 55.305440666666655)(27.846410684474126, 55.49256008333335)(27.946577629382304, 55.37742099999998)(28.046744574290486, 55.47694483333334)(28.146911519198664, 55.36374974999999)(28.247078464106846, 55.3997145)(28.347245409015024, 55.54385424999999)(28.447412353923205, 55.59563683333333)(28.547579298831387, 55.67833925000002)(28.64774624373957, 55.49269199999999)(28.747913188647747, 55.635777333333294)(28.84808013355593, 55.573705583333314)(28.948247078464107, 55.647323416666694)(29.04841402337229, 55.46973716666665)(29.148580968280466, 55.52367666666668)(29.248747913188648, 55.49784416666667)(29.348914858096826, 55.34185316666667)(29.44908180300501, 55.49552966666665)(29.54924874791319, 55.419759749999955)(29.64941569282137, 55.67406725000001)(29.74958263772955, 55.53949458333332)(29.84974958263773, 55.534134250000015)(29.94991652754591, 55.525274416666676)(30.05008347245409, 55.6259045)(30.15025041736227, 55.62749233333336)(30.25041736227045, 55.48200583333331)(30.35058430717863, 55.439509916666665)(30.45075125208681, 55.49382033333335)(30.55091819699499, 55.497339999999994)(30.651085141903177, 55.59463025000002)(30.751252086811355, 55.57311508333329)(30.851419031719537, 55.57918258333334)(30.951585976627715, 55.396449583333315)(31.051752921535897, 55.261470583333306)(31.151919866444075, 55.45642108333335)(31.252086811352257, 55.55861941666665)(31.352253756260435, 55.48885125)(31.452420701168617, 55.63252808333335)(31.552587646076795, 55.57440658333333)(31.652754590984976, 55.531352083333346)(31.752921535893154, 55.65750116666668)(31.853088480801336, 55.57468225000002)(31.953255425709514, 55.554976916666654)(32.053422370617696, 55.41686591666669)(32.15358931552588, 55.61867766666668)(32.25375626043405, 55.49085525000001)(32.35392320534224, 55.536036583333335)(32.45409015025042, 55.685501000000016)(32.554257095158604, 55.477208666666655)(32.65442404006678, 55.49150625000001)(32.75459098497496, 55.558350083333345)(32.85475792988314, 55.74629683333337)(32.95492487479132, 55.530421749999995)(33.0550918196995, 55.67867966666668)(33.15525876460768, 55.53616333333335)(33.25542570951586, 55.561085999999996)(33.35559265442404, 55.54453591666668)(33.45575959933222, 55.689233999999985)(33.5559265442404, 55.38882016666668)(33.65609348914858, 55.56900525)(33.756260434056756, 55.34695541666668)(33.85642737896494, 55.36376016666665)(33.95659432387313, 55.412461)(34.05676126878131, 55.422851666666666)(34.15692821368948, 55.457846166666656)(34.257095158597664, 55.44313050000001)(34.357262103505846, 55.50616950000001)(34.45742904841403, 55.491801416666675)(34.5575959933222, 55.551564916666656)(34.657762938230384, 55.436127416666686)(34.757929883138566, 55.446757666666635)(34.85809682804675, 55.62989816666666)(34.95826377295492, 55.376048333333316)(35.058430717863104, 55.43701183333336)(35.158597662771285, 55.52839725)(35.25876460767947, 55.382660083333306)(35.35893155258764, 55.7499695)(35.45909849749582, 55.50376408333333)(35.559265442404005, 55.41111275)(35.659432387312194, 55.36957341666668)(35.75959933222037, 55.58948825)(35.85976627712855, 55.55503841666666)(35.95993322203673, 55.51328099999999)(36.06010016694491, 55.43742958333333)(36.16026711185309, 55.52816283333332)(36.26043405676127, 55.47601825)(36.36060100166945, 55.48827624999997)(36.46076794657763, 55.43464224999998)(36.56093489148581, 55.32286141666666)(36.66110183639399, 55.63207975)(36.76126878130217, 55.355230833333344)(36.86143572621035, 55.504979916666684)(36.96160267111853, 55.456619833333335)(37.06176961602671, 55.622751666666694)(37.16193656093489, 55.650965416666665)(37.26210350584308, 55.27900766666666)(37.362270450751254, 55.62643983333337)(37.462437395659435, 55.473801)(37.56260434056762, 55.55872608333337)(37.6627712854758, 55.42083758333334)(37.76293823038397, 55.40981141666669)(37.863105175292155, 55.67762225000001)(37.96327212020034, 55.499049250000034)(38.06343906510852, 55.60588566666669)(38.16360601001669, 55.530614416666644)(38.263772954924875, 55.42505983333334)(38.363939899833056, 55.72952700000001)(38.46410684474124, 55.53319866666668)(38.56427378964941, 55.50173449999999)(38.664440734557594, 55.468816583333336)(38.764607679465776, 55.5888064166667)(38.86477462437396, 55.565348583333325)(38.96494156928214, 55.43099516666667)(39.06510851419032, 55.380712499999994)(39.1652754590985, 55.49583441666668)(39.26544240400668, 55.50074316666664)(39.36560934891486, 55.26559016666665)(39.46577629382304, 55.34500208333335)(39.56594323873122, 55.62917608333334)(39.666110183639404, 55.39609833333334)(39.76627712854758, 55.65019708333331)(39.86644407345576, 55.61334741666665)(39.96661101836394, 55.51870750000002)(40.066777963272116, 55.62722716666668)(40.1669449081803, 55.442170583333315)(40.26711185308848, 55.53766375)(40.36727879799666, 55.492732416666655)(40.46744574290484, 55.42899591666667)(40.567612687813025, 55.65687066666667)(40.667779632721206, 55.40264425000003)(40.76794657762939, 55.42378291666666)(40.86811352253756, 55.48443108333332)(40.968280467445744, 55.58909625)(41.068447412353926, 55.64793916666667)(41.16861435726211, 55.55763275000001)(41.26878130217028, 55.482177916666686)(41.368948247078464, 55.57501733333332)(41.469115191986646, 55.61553466666666)(41.56928213689483, 55.605320833333316)(41.669449081803, 55.58682300000001)(41.769616026711184, 55.41503499999997)(41.869782971619365, 55.824960333333344)(41.96994991652755, 55.35978316666665)(42.07011686143572, 55.52882916666666)(42.1702838063439, 55.45162966666666)(42.27045075125209, 55.47128825)(42.370617696160274, 55.41922591666666)(42.47078464106845, 55.504949333333336)(42.57095158597663, 55.60473666666667)(42.67111853088481, 55.42601533333334)(42.77128547579299, 55.50650058333334)(42.87145242070117, 55.64676916666667)(42.97161936560935, 55.64617391666666)(43.07178631051753, 55.46976249999999)(43.17195325542571, 55.41828000000002)(43.27212020033389, 55.770746416666675)(43.37228714524207, 55.47177183333334)(43.47245409015025, 55.57415749999999)(43.57262103505843, 55.67730191666669)(43.67278797996661, 55.44777966666667)(43.77295492487479, 55.540115666666665)(43.87312186978298, 55.45068925)(43.97328881469116, 55.7000738333333)(44.073455759599334, 55.52427633333331)(44.173622704507515, 55.44134041666668)(44.2737896494157, 55.606257583333345)(44.37395659432388, 55.5584765)(44.47412353923205, 55.52881441666666)(44.574290484140235, 55.77720591666667)(44.67445742904842, 55.42975383333333)(44.7746243739566, 55.48668966666671)(44.87479131886477, 55.56452950000001)(44.974958263772955, 55.46246858333333)(45.075125208681136, 55.500443)(45.17529215358932, 55.625859750000025)(45.27545909849749, 55.67430658333331)(45.375626043405674, 55.47057108333333)(45.475792988313856, 55.52172366666669)(45.57595993322204, 55.683908666666646)(45.67612687813022, 55.68906658333333)(45.7762938230384, 55.45993083333334)(45.87646076794658, 55.54520716666668)(45.976627712854764, 55.700317083333296)(46.07679465776294, 55.62450666666667)(46.17696160267112, 55.67832933333331)(46.2771285475793, 55.657897999999975)(46.37729549248748, 55.46742266666662)(46.47746243739566, 55.54836558333333)(46.57762938230384, 55.71373508333336)(46.67779632721202, 55.54516566666664)(46.7779632721202, 55.34466125000002)(46.87813021702838, 55.74337225000002)(46.97829716193656, 55.60557083333332)(47.07846410684474, 55.58929483333331)(47.17863105175292, 55.697361750000034)(47.278797996661105, 55.581120916666684)(47.378964941569286, 55.565562)(47.47913188647747, 55.50229416666668)(47.57929883138564, 55.87046791666667)(47.679465776293824, 55.33244416666667)(47.779632721202006, 55.65985574999998)(47.87979966611019, 55.620127333333336)(47.97996661101836, 55.682027)(48.080133555926544, 55.581583583333334)(48.180300500834726, 55.60217341666666)(48.28046744574291, 56.096591666666654)(48.38063439065108, 55.41635708333332)(48.48080133555926, 55.690047833333324)(48.580968280467445, 55.56910233333333)(48.68113522537563, 55.59599825)(48.7813021702838, 55.863957000000006)(48.88146911519199, 55.55644241666668)(48.98163606010017, 55.819009999999984)(49.081803005008354, 55.56810008333334)(49.18196994991653, 55.55436233333334)(49.28213689482471, 55.72526533333333)(49.38230383973289, 55.54298916666668)(49.48247078464107, 55.651743833333356)(49.58263772954925, 55.80596849999998)(49.68280467445743, 55.643950999999994)(49.78297161936561, 55.71040383333334)(49.88313856427379, 55.554697749999995)(49.98330550918197, 55.78559808333332)(50.08347245409015, 55.588831916666656)(50.18363939899833, 55.56001316666663)(50.28380634390651, 55.919264500000004)(50.38397328881469, 55.598063666666675)(50.484140233722876, 55.62485749999998)(50.58430717863106, 55.63770533333336)(50.68447412353924, 55.641657833333355)(50.784641068447414, 55.72609950000001)(50.884808013355595, 55.65508458333333)(50.98497495826378, 56.01664116666665)(51.08514190317196, 55.65769966666663)(51.18530884808013, 55.65180441666668)(51.285475792988315, 55.916590083333354)(51.3856427378965, 55.73988308333335)(51.48580968280468, 55.63273649999999)(51.58597662771285, 55.51991283333334)(51.686143572621035, 55.85598675000002)(51.786310517529216, 55.48445175000001)(51.8864774624374, 55.53924066666664)(51.98664440734557, 55.80347616666665)(52.086811352253754, 55.694666250000026)(52.18697829716194, 55.79895599999999)(52.287145242070125, 55.62337208333333)(52.3873121869783, 55.56428016666668)(52.48747913188648, 55.711196999999956)(52.58764607679466, 55.64325883333335)(52.68781302170284, 55.89432783333333)(52.78797996661102, 55.56114125000002)(52.8881469115192, 55.64324008333333)(52.98831385642738, 55.84335758333336)(53.088480801335564, 55.48816450000002)(53.18864774624374, 55.60597275)(53.28881469115192, 55.708760166666664)(53.3889816360601, 55.691614416666674)(53.489148580968276, 55.74018349999999)(53.58931552587646, 55.70059166666662)(53.68948247078464, 55.745259499999996)(53.78964941569283, 55.651072000000006)(53.889816360601, 55.58071441666669)(53.989983305509185, 56.050723500000004)(54.090150250417366, 55.67744983333333)(54.19031719532555, 55.82863325)(54.29048414023372, 55.59213241666666)(54.390651085141904, 55.62775174999999)(54.490818030050086, 55.840982333333365)(54.59098497495827, 55.76278683333333)(54.69115191986644, 55.99932774999999)(54.791318864774624, 55.57777358333335)(54.891485809682806, 55.480144083333336)(54.99165275459099, 55.94827216666668)(55.09181969949916, 55.57033274999999)(55.19198664440734, 55.50592566666664)(55.292153589315525, 56.1212605)(55.39232053422371, 55.78855875)(55.49248747913189, 55.56741316666666)(55.59265442404007, 55.87422149999999)(55.69282136894825, 55.86302066666665)(55.79298831385643, 55.386755500000014)(55.89315525876461, 55.83078483333332)(55.99332220367279, 55.79729075)(56.09348914858097, 55.58448883333333)(56.19365609348915, 55.76762358333335)(56.29382303839733, 55.759058083333336)(56.39398998330551, 55.844893999999975)(56.49415692821369, 55.71015874999999)(56.59432387312187, 55.6795145)(56.69449081803005, 55.96688275)(56.79465776293823, 55.56100983333333)(56.89482470784641, 55.896442916666665)(56.99499165275459, 55.72305766666668)(57.095158597662774, 55.822377083333336)(57.195325542570956, 55.91392449999998)(57.29549248747914, 55.69148249999999)(57.39565943238732, 55.98936858333334)(57.495826377295494, 55.65285233333335)(57.595993322203675, 55.48272758333335)(57.69616026711186, 56.24267966666667)(57.79632721202004, 55.62384966666666)(57.89649415692821, 55.783132)(57.996661101836395, 55.84379974999998)(58.09682804674458, 55.74991391666667)(58.19699499165276, 55.81395958333331)(58.29716193656093, 55.72827066666667)(58.397328881469114, 56.062809)(58.497495826377296, 55.629445583333336)(58.59766277128548, 55.628494083333315)(58.69782971619365, 56.03533358333331)(58.79799666110184, 55.8049520833333)(58.89816360601002, 55.75642266666666)(58.9983305509182, 55.71738158333332)(59.09849749582638, 55.63153608333334)(59.19866444073456, 55.95233608333335)(59.29883138564274, 55.715784333333325)(59.398998330550924, 56.109042666666646)(59.4991652754591, 55.62317483333333)(59.59933222036728, 55.63756791666665)(59.69949916527546, 56.117155499999996)(59.79966611018364, 55.689163499999985)(59.89983305509182, 55.83932933333333)(60.0, 55.64555841666668)
        };
        \addplot[color=blue, mark=none,name path=A] coordinates { %% MAX value
        (0.0, 0.0)(0.1001669449081803, 58.008269999999996)(0.2003338898163606, 58.30796)(0.3005008347245409, 65.54305000000001)(0.4006677796327212, 71.62946)(0.5008347245409015, 56.78207999999999)(0.6010016694490818, 64.37299999999999)(0.7011686143572621, 61.10519)(0.8013355592654424, 59.125209999999996)(0.9015025041736228, 57.13548)(1.001669449081803, 56.735699999999994)(1.1018363939899833, 56.8657)(1.2020033388981637, 56.54771)(1.3021702838063438, 56.606300000000005)(1.4023372287145242, 56.5117)(1.5025041736227045, 56.387190000000004)(1.6026711185308848, 61.47873)(1.7028380634390652, 56.38962)(1.8030050083472455, 56.35179)(1.9031719532554257, 56.065529999999995)(2.003338898163606, 57.07383)(2.1035058430717863, 56.61851)(2.2036727879799667, 56.506809999999994)(2.303839732888147, 56.30906)(2.4040066777963274, 56.67161)(2.5041736227045073, 56.401830000000004)(2.6043405676126876, 56.75644)(2.704507512520868, 56.58127999999999)(2.8046744574290483, 56.56602)(2.9048414023372287, 56.63621)(3.005008347245409, 56.31821000000001)(3.1051752921535893, 56.57273)(3.2053422370617697, 57.10984)(3.30550918196995, 56.96763)(3.4056761268781304, 56.89805)(3.5058430717863107, 57.139129999999994)(3.606010016694491, 56.785129999999995)(3.7061769616026714, 56.47385)(3.8063439065108513, 56.654520000000005)(3.906510851419032, 57.67442)(4.006677796327212, 58.381800000000005)(4.106844741235393, 56.77537)(4.207011686143573, 57.210539999999995)(4.3071786310517535, 57.333220000000004)(4.407345575959933, 56.38414)(4.507512520868114, 56.39634)(4.607679465776294, 57.18675)(4.707846410684475, 56.69358)(4.808013355592655, 56.67344)(4.908180300500835, 57.003029999999995)(5.0083472454090145, 57.2136)(5.108514190317195, 57.00059)(5.208681135225375, 56.564800000000005)(5.308848080133556, 58.08579)(5.409015025041736, 56.70456)(5.509181969949917, 56.76499)(5.609348914858097, 56.818090000000005)(5.709515859766277, 57.14402)(5.809682804674457, 56.796730000000004)(5.909849749582638, 56.34385)(6.010016694490818, 57.224579999999996)(6.110183639398999, 56.86387)(6.210350584307179, 56.84189)(6.3105175292153595, 56.85105)(6.410684474123539, 56.84983)(6.510851419031719, 56.92551)(6.6110183639399, 56.83762)(6.71118530884808, 65.90681000000001)(6.811352253756261, 56.825419999999994)(6.911519198664441, 56.96763)(7.011686143572621, 57.10373)(7.111853088480801, 66.85896)(7.212020033388982, 57.07017)(7.312186978297162, 56.70945)(7.412353923205343, 56.77415)(7.512520868113523, 56.65818)(7.612687813021703, 56.65879)(7.712854757929884, 58.17856)(7.813021702838064, 56.927960000000006)(7.913188647746244, 56.91209)(8.013355592654424, 56.94932)(8.113522537562606, 57.342389999999995)(8.213689482470786, 56.64353)(8.313856427378965, 56.83274)(8.414023372287145, 56.80771)(8.514190317195327, 57.4083)(8.614357262103507, 57.37473)(8.714524207011687, 56.12168)(8.814691151919867, 57.1251)(8.914858096828048, 57.488260000000004)(9.015025041736228, 71.73199)(9.115191986644408, 56.76438)(9.215358931552588, 56.636810000000004)(9.31552587646077, 56.48484)(9.41569282136895, 56.86569)(9.51585976627713, 58.54844)(9.61602671118531, 64.0074)(9.71619365609349, 57.573100000000004)(9.81636060100167, 57.33262)(9.916527545909851, 57.08908)(10.016694490818029, 57.44797)(10.11686143572621, 58.09311)(10.21702838063439, 57.17942)(10.31719532554257, 57.80381)(10.41736227045075, 57.4083)(10.51752921535893, 57.45103)(10.617696160267112, 57.298429999999996)(10.717863105175292, 57.137919999999994)(10.818030050083472, 57.661590000000004)(10.918196994991652, 57.35215)(11.018363939899833, 57.11289)(11.118530884808013, 57.23862)(11.218697829716193, 57.95395)(11.318864774624373, 57.15501)(11.419031719532555, 57.140969999999996)(11.519198664440735, 56.49949)(11.619365609348915, 57.68418)(11.719532554257095, 57.47849000000001)(11.819699499165276, 57.06955)(11.919866444073456, 56.83518)(12.020033388981636, 56.900490000000005)(12.120200333889816, 57.28257)(12.220367278797998, 56.766819999999996)(12.320534223706177, 57.15439)(12.420701168614357, 58.68576)(12.520868113522537, 58.60397)(12.621035058430719, 57.54196999999999)(12.721202003338899, 56.94748)(12.821368948247079, 57.101910000000004)(12.921535893155259, 56.87241)(13.021702838063439, 57.1721)(13.12186978297162, 57.21604)(13.2220367278798, 57.315529999999995)(13.32220367278798, 57.47177)(13.42237061769616, 56.82481)(13.522537562604342, 56.95481)(13.622704507512521, 56.60813)(13.722871452420701, 57.24656)(13.823038397328881, 57.40281)(13.923205342237063, 56.852270000000004)(14.023372287145243, 57.40403)(14.123539232053423, 57.44126)(14.223706176961603, 57.480320000000006)(14.323873121869784, 57.40707999999999)(14.424040066777964, 56.81382)(14.524207011686144, 56.32127)(14.624373956594324, 56.945660000000004)(14.724540901502506, 56.65879)(14.824707846410686, 57.03172000000001)(14.924874791318866, 57.68235)(15.025041736227045, 58.215180000000004)(15.125208681135225, 57.19162)(15.225375626043405, 57.27401999999999)(15.325542570951589, 57.51877)(15.425709515859769, 57.39548)(15.525876460767948, 56.978)(15.626043405676128, 57.00547)(15.726210350584308, 56.90842)(15.826377295492488, 57.24473)(15.926544240400668, 57.004859999999994)(16.026711185308848, 57.101290000000006)(16.126878130217026, 57.2667)(16.22704507512521, 57.11289)(16.32721202003339, 57.13303)(16.42737896494157, 57.108619999999995)(16.52754590984975, 57.13303)(16.62771285475793, 57.36069)(16.72787979966611, 57.37107)(16.82804674457429, 56.626450000000006)(16.92821368948247, 56.96274)(17.028380634390654, 57.299049999999994)(17.128547579298832, 56.93345)(17.228714524207014, 57.29356)(17.328881469115192, 56.937110000000004)(17.429048414023374, 56.7125)(17.529215358931552, 57.11776999999999)(17.629382303839733, 56.810159999999996)(17.72954924874791, 57.12022)(17.829716193656097, 57.332010000000004)(17.929883138564275, 57.53647)(18.030050083472457, 57.219699999999996)(18.130217028380635, 56.96824)(18.230383973288816, 57.03111)(18.330550918196995, 56.83213000000001)(18.430717863105176, 57.47971)(18.530884808013354, 56.85349000000001)(18.63105175292154, 56.91269)(18.731218697829718, 57.4144)(18.8313856427379, 57.043929999999996)(18.931552587646078, 57.21726)(19.03171953255426, 57.31308)(19.131886477462437, 56.68686)(19.23205342237062, 57.20078)(19.332220367278797, 57.25693)(19.43238731218698, 57.08298)(19.53255425709516, 57.03111)(19.63272120200334, 57.900850000000005)(19.73288814691152, 57.02073)(19.833055091819702, 57.32651)(19.93322203672788, 57.44553)(20.033388981636058, 57.21543)(20.13355592654424, 57.47788)(20.23372287145242, 57.376560000000005)(20.333889816360603, 57.18979)(20.43405676126878, 57.08603)(20.534223706176963, 57.29965)(20.63439065108514, 56.92063)(20.734557595993323, 56.989599999999996)(20.8347245409015, 56.93772)(20.934891485809683, 57.20017)(21.03505843071786, 57.47971)(21.135225375626046, 57.39426)(21.235392320534224, 56.83762)(21.335559265442406, 57.39792)(21.435726210350584, 57.387550000000005)(21.535893155258766, 57.02439)(21.636060100166944, 57.388769999999994)(21.736227045075125, 57.884370000000004)(21.836393989983303, 57.1605)(21.93656093489149, 57.08238)(22.036727879799667, 56.73873999999999)(22.13689482470785, 56.99509)(22.237061769616027, 57.17575)(22.33722871452421, 56.67711)(22.437395659432386, 57.187349999999995)(22.537562604340568, 57.227639999999994)(22.637729549248746, 56.830290000000005)(22.737896494156928, 56.73325)(22.83806343906511, 57.05491000000001)(22.93823038397329, 56.83213000000001)(23.03839732888147, 56.8419)(23.13856427378965, 57.196509999999996)(23.23873121869783, 56.86082)(23.33889816360601, 56.89194)(23.43906510851419, 56.66978)(23.53923205342237, 57.034760000000006)(23.639398998330552, 57.465669999999996)(23.739565943238734, 56.83884)(23.839732888146912, 57.11289)(23.939899833055094, 69.68487999999999)(24.040066777963272, 57.2368)(24.140233722871454, 57.380829999999996)(24.24040066777963, 57.86973)(24.340567612687813, 57.22215)(24.440734557595995, 56.89072)(24.540901502504177, 57.4437)(24.641068447412355, 56.92551)(24.741235392320537, 57.20322)(24.841402337228715, 57.1251)(24.941569282136896, 57.11716)(25.041736227045075, 56.7772)(25.141903171953256, 57.14402)(25.242070116861438, 57.02439)(25.34223706176962, 57.207499999999996)(25.442404006677798, 57.11228)(25.54257095158598, 57.25571000000001)(25.642737896494157, 57.49679999999999)(25.74290484140234, 57.02744)(25.843071786310517, 57.097629999999995)(25.9432387312187, 57.43272)(26.043405676126877, 57.56577)(26.143572621035062, 57.13425)(26.24373956594324, 57.04392)(26.34390651085142, 57.25022)(26.4440734557596, 57.02439)(26.544240400667782, 56.964569999999995)(26.64440734557596, 65.35688999999999)(26.744574290484138, 78.95121999999999)(26.84474123539232, 58.470310000000005)(26.9449081803005, 57.72202)(27.045075125208683, 57.358869999999996)(27.14524207011686, 57.405249999999995)(27.245409015025043, 57.376569999999994)(27.34557595993322, 81.83085)(27.445742904841403, 61.73751)(27.54590984974958, 57.769009999999994)(27.646076794657763, 62.84469)(27.746243739565944, 58.93967)(27.846410684474126, 72.57794)(27.946577629382304, 73.10711)(28.046744574290486, 71.00262)(28.146911519198664, 69.33943000000001)(28.247078464106846, 71.45672)(28.347245409015024, 73.6784)(28.447412353923205, 73.60638)(28.547579298831387, 74.57013)(28.64774624373957, 70.45697)(28.747913188647747, 69.71357)(28.84808013355593, 71.768)(28.948247078464107, 73.50384)(29.04841402337229, 72.89410000000001)(29.148580968280466, 73.05829)(29.248747913188648, 71.61114)(29.348914858096826, 76.31572)(29.44908180300501, 74.87652)(29.54924874791319, 74.61224)(29.64941569282137, 72.05059)(29.74958263772955, 74.33575)(29.84974958263773, 72.52667)(29.94991652754591, 73.82916)(30.05008347245409, 72.02374)(30.15025041736227, 74.84966)(30.25041736227045, 75.11272)(30.35058430717863, 73.54291)(30.45075125208681, 61.335899999999995)(30.55091819699499, 72.20379)(30.651085141903177, 75.45452)(30.751252086811355, 75.92937)(30.851419031719537, 72.68353)(30.951585976627715, 71.76862)(31.051752921535897, 64.31746000000001)(31.151919866444075, 67.62922)(31.252086811352257, 61.99752)(31.352253756260435, 71.41828000000001)(31.452420701168617, 72.74517)(31.552587646076795, 72.34235)(31.652754590984976, 70.74566)(31.752921535893154, 72.69147000000001)(31.853088480801336, 71.58917)(31.953255425709514, 73.065)(32.053422370617696, 66.39143)(32.15358931552588, 77.7299)(32.25375626043405, 76.54644)(32.35392320534224, 71.59528)(32.45409015025042, 73.28594)(32.554257095158604, 73.15594)(32.65442404006678, 72.12262)(32.75459098497496, 71.08258000000001)(32.85475792988314, 75.19146)(32.95492487479132, 74.65435000000001)(33.0550918196995, 75.88115)(33.15525876460768, 75.04192)(33.25542570951586, 74.89544000000001)(33.35559265442404, 74.21978)(33.45575959933222, 67.29414)(33.5559265442404, 68.3177)(33.65609348914858, 63.18587)(33.756260434056756, 64.45785)(33.85642737896494, 58.50082)(33.95659432387313, 58.07297)(34.05676126878131, 57.25693)(34.15692821368948, 57.49192)(34.257095158597664, 57.14036)(34.357262103505846, 57.37351)(34.45742904841403, 57.63474000000001)(34.5575959933222, 57.40524)(34.657762938230384, 57.02439)(34.757929883138566, 59.20945)(34.85809682804675, 57.45407)(34.95826377295492, 57.19468)(35.058430717863104, 57.61277)(35.158597662771285, 57.95945)(35.25876460767947, 57.95823)(35.35893155258764, 64.53963)(35.45909849749582, 78.79862)(35.559265442404005, 57.219699999999996)(35.659432387312194, 57.092749999999995)(35.75959933222037, 57.58042)(35.85976627712855, 59.2363)(35.95993322203673, 57.43943)(36.06010016694491, 57.30332)(36.16026711185309, 58.04795)(36.26043405676127, 57.86912)(36.36060100166945, 57.92893)(36.46076794657763, 57.48276)(36.56093489148581, 57.70555)(36.66110183639399, 58.973240000000004)(36.76126878130217, 57.688449999999996)(36.86143572621035, 57.40769)(36.96160267111853, 57.477270000000004)(37.06176961602671, 57.565160000000006)(37.16193656093489, 58.130340000000004)(37.26210350584308, 57.69638)(37.362270450751254, 57.81785)(37.462437395659435, 57.43515000000001)(37.56260434056762, 57.58103)(37.6627712854758, 57.89475)(37.76293823038397, 58.3934)(37.863105175292155, 58.37509)(37.96327212020034, 57.70554)(38.06343906510852, 57.0604)(38.16360601001669, 57.14402)(38.263772954924875, 57.12815)(38.363939899833056, 58.01133)(38.46410684474124, 57.83005)(38.56427378964941, 57.83493)(38.664440734557594, 57.6854)(38.764607679465776, 58.32931000000001)(38.86477462437396, 57.53221)(38.96494156928214, 57.889860000000006)(39.06510851419032, 57.756809999999994)(39.1652754590985, 57.54013)(39.26544240400668, 57.252660000000006)(39.36560934891486, 57.0366)(39.46577629382304, 57.17148)(39.56594323873122, 57.88071000000001)(39.666110183639404, 57.576150000000005)(39.76627712854758, 57.9155)(39.86644407345576, 57.48276)(39.96661101836394, 58.14805)(40.066777963272116, 57.900850000000005)(40.1669449081803, 57.58347)(40.26711185308848, 58.0333)(40.36727879799666, 58.16696999999999)(40.46744574290484, 58.30063)(40.567612687813025, 57.64328)(40.667779632721206, 57.621320000000004)(40.76794657762939, 57.051860000000005)(40.86811352253756, 57.392430000000004)(40.968280467445744, 57.50352)(41.068447412353926, 58.20175)(41.16861435726211, 58.265840000000004)(41.26878130217028, 57.07566)(41.368948247078464, 58.11081)(41.469115191986646, 57.94297)(41.56928213689483, 58.231049999999996)(41.669449081803, 57.458960000000005)(41.769616026711184, 57.11351)(41.869782971619365, 57.96067)(41.96994991652755, 57.96128)(42.07011686143572, 58.0394)(42.1702838063439, 57.9332)(42.27045075125209, 57.585300000000004)(42.370617696160274, 57.53586)(42.47078464106845, 57.67442)(42.57095158597663, 57.94052)(42.67111853088481, 57.35398)(42.77128547579299, 57.03965)(42.87145242070117, 58.56491)(42.97161936560935, 57.14341)(43.07178631051753, 57.64023)(43.17195325542571, 58.92868)(43.27212020033389, 57.968599999999995)(43.37228714524207, 58.52524)(43.47245409015025, 57.85996)(43.57262103505843, 57.576150000000005)(43.67278797996661, 57.59568)(43.77295492487479, 58.46176)(43.87312186978298, 57.60423)(43.97328881469116, 58.02109)(44.073455759599334, 57.45591)(44.173622704507515, 57.53342)(44.2737896494157, 58.10349000000001)(44.37395659432388, 57.77451)(44.47412353923205, 57.91489)(44.574290484140235, 58.13279)(44.67445742904842, 57.12449)(44.7746243739566, 57.31675)(44.87479131886477, 57.778780000000005)(44.974958263772955, 57.49131)(45.075125208681136, 58.503879999999995)(45.17529215358932, 69.54145)(45.27545909849749, 57.55845)(45.375626043405674, 57.49192)(45.475792988313856, 57.6738)(45.57595993322204, 58.067479999999996)(45.67612687813022, 58.342749999999995)(45.7762938230384, 57.92099999999999)(45.87646076794658, 57.19956)(45.976627712854764, 57.80686)(46.07679465776294, 57.98814)(46.17696160267112, 57.81052)(46.2771285475793, 57.632909999999995)(46.37729549248748, 57.54196999999999)(46.47746243739566, 57.561499999999995)(46.57762938230384, 57.89353)(46.67779632721202, 57.934419999999996)(46.7779632721202, 57.41318)(46.87813021702838, 58.351290000000006)(46.97829716193656, 58.68638)(47.07846410684474, 57.94908)(47.17863105175292, 58.0742)(47.278797996661105, 58.047940000000004)(47.378964941569286, 57.902069999999995)(47.47913188647747, 57.107400000000005)(47.57929883138564, 58.54294)(47.679465776293824, 59.00925)(47.779632721202006, 58.35496)(47.87979966611019, 58.03147)(47.97996661101836, 57.7269)(48.080133555926544, 57.71897)(48.180300500834726, 57.67502999999999)(48.28046744574291, 58.13157)(48.38063439065108, 58.043060000000004)(48.48080133555926, 58.17124)(48.580968280467445, 57.811130000000006)(48.68113522537563, 57.29416)(48.7813021702838, 58.348240000000004)(48.88146911519199, 58.51243)(48.98163606010017, 59.058080000000004)(49.081803005008354, 57.94358)(49.18196994991653, 57.90024)(49.28213689482471, 58.335429999999995)(49.38230383973289, 58.16696999999999)(49.48247078464107, 58.236549999999994)(49.58263772954925, 58.37265000000001)(49.68280467445743, 58.755340000000004)(49.78297161936561, 59.14109)(49.88313856427379, 58.07785)(49.98330550918197, 57.692119999999996)(50.08347245409015, 57.40952)(50.18363939899833, 57.72934)(50.28380634390651, 58.85544)(50.38397328881469, 57.64695)(50.484140233722876, 57.860569999999996)(50.58430717863106, 57.860569999999996)(50.68447412353924, 58.33298)(50.784641068447414, 58.46909)(50.884808013355595, 57.99362000000001)(50.98497495826378, 58.72422)(51.08514190317196, 57.55051)(51.18530884808013, 59.03914999999999)(51.285475792988315, 57.812960000000004)(51.3856427378965, 57.71957)(51.48580968280468, 57.495580000000004)(51.58597662771285, 57.20505)(51.686143572621035, 57.665870000000005)(51.786310517529216, 57.32529)(51.8864774624374, 57.03843)(51.98664440734557, 58.26706)(52.086811352253754, 58.246919999999996)(52.18697829716194, 57.82273)(52.287145242070125, 57.963719999999995)(52.3873121869783, 57.73911)(52.48747913188648, 57.81418000000001)(52.58764607679466, 57.65793000000001)(52.68781302170284, 58.29453)(52.78797996661102, 57.24595)(52.8881469115192, 57.58225)(52.98831385642738, 57.68479)(53.088480801335564, 57.6207)(53.18864774624374, 57.889869999999995)(53.28881469115192, 58.23349)(53.3889816360601, 57.74216)(53.489148580968276, 58.04123)(53.58931552587646, 58.7291)(53.68948247078464, 58.66684)(53.78964941569283, 57.90817)(53.889816360601, 57.544399999999996)(53.989983305509185, 58.601530000000004)(54.090150250417366, 57.935030000000005)(54.19031719532555, 57.8917)(54.29048414023372, 57.853849999999994)(54.390651085141904, 57.7562)(54.490818030050086, 58.16879)(54.59098497495827, 58.187110000000004)(54.69115191986644, 58.64365)(54.791318864774624, 57.66343)(54.891485809682806, 57.219699999999996)(54.99165275459099, 58.448339999999995)(55.09181969949916, 57.71653)(55.19198664440734, 57.46384)(55.292153589315525, 58.570409999999995)(55.39232053422371, 58.774879999999996)(55.49248747913189, 58.38791)(55.59265442404007, 57.783049999999996)(55.69282136894825, 58.21151999999999)(55.79298831385643, 57.96127)(55.89315525876461, 58.84262)(55.99332220367279, 58.30856)(56.09348914858097, 57.87094)(56.19365609348915, 58.293910000000004)(56.29382303839733, 60.876310000000004)(56.39398998330551, 58.61923)(56.49415692821369, 57.65611)(56.59432387312187, 58.06686)(56.69449081803005, 58.15476)(56.79465776293823, 58.090669999999996)(56.89482470784641, 58.22739)(56.99499165275459, 58.32016)(57.095158597662774, 58.04856)(57.195325542570956, 57.320409999999995)(57.29549248747914, 58.19749)(57.39565943238732, 58.24937)(57.495826377295494, 57.64023)(57.595993322203675, 57.651219999999995)(57.69616026711186, 58.210300000000004)(57.79632721202004, 57.812349999999995)(57.89649415692821, 59.17282)(57.996661101836395, 58.6528)(58.09682804674458, 58.11631)(58.19699499165276, 57.86484)(58.29716193656093, 57.96799)(58.397328881469114, 58.276219999999995)(58.497495826377296, 58.403169999999996)(58.59766277128548, 57.51328)(58.69782971619365, 58.55087)(58.79799666110184, 57.379619999999996)(58.89816360601002, 58.08579)(58.9983305509182, 57.89353)(59.09849749582638, 58.11448)(59.19866444073456, 58.21091)(59.29883138564274, 58.11813)(59.398998330550924, 58.34518)(59.4991652754591, 57.38205)(59.59933222036728, 58.083349999999996)(59.69949916527546, 58.30064)(59.79966611018364, 57.71103)(59.89983305509182, 58.67295)(60.0, 57.80625)
        };
        \addplot[color=blue, mark=none,name path=B] coordinates { %% MIN value
        (0.0, 0.0)(0.1001669449081803, 50.29345)(0.2003338898163606, 52.592020000000005)(0.3005008347245409, 51.73692)(0.4006677796327212, 51.09545)(0.5008347245409015, 52.60485)(0.6010016694490818, 52.759879999999995)(0.7011686143572621, 52.18248)(0.8013355592654424, 51.75158)(0.9015025041736228, 52.914899999999996)(1.001669449081803, 52.77513)(1.1018363939899833, 52.58837)(1.2020033388981637, 51.368280000000006)(1.3021702838063438, 53.04674)(1.4023372287145242, 52.07994)(1.5025041736227045, 52.56273)(1.6026711185308848, 53.50023)(1.7028380634390652, 52.59142)(1.8030050083472455, 52.79284)(1.9031719532554257, 52.02562999999999)(2.003338898163606, 53.224349999999994)(2.1035058430717863, 52.57127)(2.2036727879799667, 53.25792)(2.303839732888147, 52.59264)(2.4040066777963274, 53.22068)(2.5041736227045073, 53.10777)(2.6043405676126876, 52.47789)(2.704507512520868, 52.85142999999999)(2.8046744574290483, 53.14927)(2.9048414023372287, 52.776360000000004)(3.005008347245409, 52.51635)(3.1051752921535893, 51.615469999999995)(3.2053422370617697, 53.00463)(3.30550918196995, 52.7727)(3.4056761268781304, 52.7965)(3.5058430717863107, 53.397690000000004)(3.606010016694491, 52.68175)(3.7061769616026714, 51.899280000000005)(3.8063439065108513, 52.886829999999996)(3.906510851419032, 52.96801000000001)(4.006677796327212, 53.397690000000004)(4.106844741235393, 51.822379999999995)(4.207011686143573, 53.89207)(4.3071786310517535, 52.65612)(4.407345575959933, 52.85142999999999)(4.507512520868114, 52.55236)(4.607679465776294, 52.38817)(4.707846410684475, 52.554790000000004)(4.808013355592655, 52.41686)(4.908180300500835, 51.45128)(5.0083472454090145, 52.40648)(5.108514190317195, 52.33995)(5.208681135225375, 52.37413)(5.308848080133556, 52.933209999999995)(5.409015025041736, 52.795880000000004)(5.509181969949917, 53.0034)(5.609348914858097, 53.15294)(5.709515859766277, 52.519400000000005)(5.809682804674457, 52.51452)(5.909849749582638, 52.85937)(6.010016694490818, 53.323229999999995)(6.110183639398999, 52.33629)(6.210350584307179, 52.55663)(6.3105175292153595, 53.229240000000004)(6.410684474123539, 52.91429)(6.510851419031719, 52.82823)(6.6110183639399, 53.060159999999996)(6.71118530884808, 53.058949999999996)(6.811352253756261, 52.202619999999996)(6.911519198664441, 52.88988)(7.011686143572621, 52.98021)(7.111853088480801, 52.56761)(7.212020033388982, 52.949690000000004)(7.312186978297162, 53.380599999999994)(7.412353923205343, 52.296009999999995)(7.512520868113523, 53.38243)(7.612687813021703, 53.24755)(7.712854757929884, 52.583479999999994)(7.813021702838064, 52.87584)(7.913188647746244, 52.32226)(8.013355592654424, 52.56884)(8.113522537562606, 52.81481)(8.213689482470786, 52.7025)(8.313856427378965, 52.00975)(8.414023372287145, 52.385729999999995)(8.514190317195327, 52.491929999999996)(8.614357262103507, 52.62864999999999)(8.714524207011687, 53.2799)(8.814691151919867, 52.96983)(8.914858096828048, 53.10412)(9.015025041736228, 53.66503)(9.115191986644408, 52.307610000000004)(9.215358931552588, 53.16332)(9.31552587646077, 52.993030000000005)(9.41569282136895, 52.91491)(9.51585976627713, 51.744260000000004)(9.61602671118531, 53.23595)(9.71619365609349, 53.080920000000006)(9.81636060100167, 52.928940000000004)(9.916527545909851, 52.328359999999996)(10.016694490818029, 53.54052)(10.11686143572621, 52.61461)(10.21702838063439, 53.14134)(10.31719532554257, 53.02477)(10.41736227045075, 53.384879999999995)(10.51752921535893, 53.14989)(10.617696160267112, 52.642689999999995)(10.717863105175292, 52.82152)(10.818030050083472, 52.74155999999999)(10.918196994991652, 52.993030000000005)(11.018363939899833, 53.282939999999996)(11.118530884808013, 53.05955)(11.218697829716193, 53.01012)(11.318864774624373, 52.35583)(11.419031719532555, 52.93992)(11.519198664440735, 52.626819999999995)(11.619365609348915, 53.118759999999995)(11.719532554257095, 53.02049)(11.819699499165276, 53.24877)(11.919866444073456, 53.00767999999999)(12.020033388981636, 52.947860000000006)(12.120200333889816, 52.98571)(12.220367278797998, 53.22069)(12.320534223706177, 53.301869999999994)(12.420701168614357, 53.01012)(12.520868113522537, 53.35618)(12.621035058430719, 53.33238)(12.721202003338899, 52.560900000000004)(12.821368948247079, 53.047959999999996)(12.921535893155259, 53.06749)(13.021702838063439, 52.52672)(13.12186978297162, 53.122420000000005)(13.2220367278798, 53.11327)(13.32220367278798, 52.836780000000005)(13.42237061769616, 53.42821)(13.522537562604342, 53.17736)(13.622704507512521, 53.021710000000006)(13.722871452420701, 52.90147)(13.823038397328881, 52.319810000000004)(13.923205342237063, 53.42455)(14.023372287145243, 53.43675)(14.123539232053423, 53.425149999999995)(14.223706176961603, 52.79527)(14.323873121869784, 53.65831)(14.424040066777964, 53.355579999999996)(14.524207011686144, 52.91308)(14.624373956594324, 53.314069999999994)(14.724540901502506, 53.547219999999996)(14.824707846410686, 52.97533)(14.924874791318866, 53.25792)(15.025041736227045, 53.24693)(15.125208681135225, 52.99059)(15.225375626043405, 52.65183999999999)(15.325542570951589, 53.70103)(15.425709515859769, 53.22618)(15.525876460767948, 53.29576)(15.626043405676128, 52.22887)(15.726210350584308, 53.33178)(15.826377295492488, 53.45872)(15.926544240400668, 53.458119999999994)(16.026711185308848, 52.820910000000005)(16.126878130217026, 53.046119999999995)(16.22704507512521, 52.2185)(16.32721202003339, 53.594229999999996)(16.42737896494157, 53.110220000000005)(16.52754590984975, 52.786730000000006)(16.62771285475793, 53.00707)(16.72787979966611, 53.400740000000006)(16.82804674457429, 52.962509999999995)(16.92821368948247, 53.301869999999994)(17.028380634390654, 53.32628)(17.128547579298832, 52.45531)(17.228714524207014, 52.37902)(17.328881469115192, 53.515480000000004)(17.429048414023374, 52.59875)(17.529215358931552, 53.301249999999996)(17.629382303839733, 53.312239999999996)(17.72954924874791, 53.28844)(17.829716193656097, 53.37084)(17.929883138564275, 53.498999999999995)(18.030050083472457, 53.46727)(18.130217028380635, 52.89842)(18.230383973288816, 53.409279999999995)(18.330550918196995, 52.68908)(18.430717863105176, 53.6461)(18.530884808013354, 52.21239)(18.63105175292154, 53.31041)(18.731218697829718, 53.055899999999994)(18.8313856427379, 51.97375)(18.931552587646078, 52.91124)(19.03171953255426, 52.97533)(19.131886477462437, 52.732409999999994)(19.23205342237062, 53.5338)(19.332220367278797, 53.2805)(19.43238731218698, 52.732409999999994)(19.53255425709516, 52.81237)(19.63272120200334, 53.26464)(19.73288814691152, 53.11632)(19.833055091819702, 53.35863)(19.93322203672788, 53.68822)(20.033388981636058, 53.16637)(20.13355592654424, 53.5338)(20.23372287145242, 53.5637)(20.333889816360603, 53.39586)(20.43405676126878, 53.445910000000005)(20.534223706176963, 52.88377)(20.63439065108514, 53.5338)(20.734557595993323, 53.53929)(20.8347245409015, 52.88195)(20.934891485809683, 52.970439999999996)(21.03505843071786, 54.07579)(21.135225375626046, 53.54112)(21.235392320534224, 53.54356)(21.335559265442406, 53.03026)(21.435726210350584, 52.3839)(21.535893155258766, 53.2451)(21.636060100166944, 52.96433999999999)(21.736227045075125, 52.697010000000006)(21.836393989983303, 53.03331)(21.93656093489149, 53.28539)(22.036727879799667, 53.418440000000004)(22.13689482470785, 53.0028)(22.237061769616027, 52.66649)(22.33722871452421, 52.958239999999996)(22.437395659432386, 52.299670000000006)(22.537562604340568, 53.31041)(22.637729549248746, 51.851060000000004)(22.737896494156928, 52.66894)(22.83806343906511, 52.79222)(22.93823038397329, 52.05919)(23.03839732888147, 53.07054)(23.13856427378965, 53.12975)(23.23873121869783, 53.000350000000005)(23.33889816360601, 53.26158)(23.43906510851419, 52.95091)(23.53923205342237, 52.96861)(23.639398998330552, 53.13829)(23.739565943238734, 53.2805)(23.839732888146912, 53.38793)(23.939899833055094, 53.44652)(24.040066777963272, 53.42821)(24.140233722871454, 52.511469999999996)(24.24040066777963, 53.13097)(24.340567612687813, 52.97777)(24.440734557595995, 52.86059)(24.540901502504177, 52.23681)(24.641068447412355, 53.03086999999999)(24.741235392320537, 52.60301)(24.841402337228715, 53.56676)(24.941569282136896, 51.270619999999994)(25.041736227045075, 52.895979999999994)(25.141903171953256, 52.939930000000004)(25.242070116861438, 53.17614)(25.34223706176962, 52.599349999999994)(25.442404006677798, 53.1151)(25.54257095158598, 53.08702)(25.642737896494157, 53.40319)(25.74290484140234, 53.27745)(25.843071786310517, 53.4514)(25.9432387312187, 52.70678)(26.043405676126877, 53.34764)(26.143572621035062, 53.234719999999996)(26.24373956594324, 52.79283)(26.34390651085142, 53.035759999999996)(26.4440734557596, 53.594840000000005)(26.544240400667782, 53.09984)(26.64440734557596, 52.77208)(26.744574290484138, 53.65526)(26.84474123539232, 52.8435)(26.9449081803005, 53.24083)(27.045075125208683, 51.99389)(27.14524207011686, 53.21825)(27.245409015025043, 53.44042)(27.34557595993322, 53.24266)(27.445742904841403, 53.425760000000004)(27.54590984974958, 53.682109999999994)(27.646076794657763, 52.53648)(27.746243739565944, 52.9674)(27.846410684474126, 53.56554)(27.946577629382304, 52.82701)(28.046744574290486, 53.53624)(28.146911519198664, 53.193830000000005)(28.247078464106846, 51.98901)(28.347245409015024, 53.17674)(28.447412353923205, 53.30675)(28.547579298831387, 52.15135)(28.64774624373957, 52.86181)(28.747913188647747, 53.81028)(28.84808013355593, 52.944199999999995)(28.948247078464107, 53.29027)(29.04841402337229, 53.20909)(29.148580968280466, 52.90147)(29.248747913188648, 53.27562)(29.348914858096826, 52.61522)(29.44908180300501, 52.9088)(29.54924874791319, 52.45653)(29.64941569282137, 52.741569999999996)(29.74958263772955, 53.40989999999999)(29.84974958263773, 53.22923)(29.94991652754591, 53.23961)(30.05008347245409, 53.736439999999995)(30.15025041736227, 54.05748)(30.25041736227045, 52.981429999999996)(30.35058430717863, 52.82335)(30.45075125208681, 52.76903)(30.55091819699499, 53.15355)(30.651085141903177, 52.62681)(30.751252086811355, 53.17674)(30.851419031719537, 53.44896)(30.951585976627715, 53.386709999999994)(31.051752921535897, 52.82946)(31.151919866444075, 53.478260000000006)(31.252086811352257, 53.108380000000004)(31.352253756260435, 53.328720000000004)(31.452420701168617, 53.811499999999995)(31.552587646076795, 52.80138)(31.652754590984976, 52.69274)(31.752921535893154, 53.65831)(31.853088480801336, 52.64146)(31.953255425709514, 53.71508)(32.053422370617696, 53.20482)(32.15358931552588, 53.18711999999999)(32.25375626043405, 52.99669)(32.35392320534224, 53.33727)(32.45409015025042, 53.669900000000005)(32.554257095158604, 53.03148)(32.65442404006678, 52.351549999999996)(32.75459098497496, 53.48802)(32.85475792988314, 53.29638)(32.95492487479132, 52.797720000000005)(33.0550918196995, 52.70982)(33.15525876460768, 53.224349999999994)(33.25542570951586, 52.55663)(33.35559265442404, 52.3247)(33.45575959933222, 53.81700000000001)(33.5559265442404, 52.978379999999994)(33.65609348914858, 53.0272)(33.756260434056756, 53.30858)(33.85642737896494, 53.14622)(33.95659432387313, 53.34031)(34.05676126878131, 52.78734)(34.15692821368948, 53.29638)(34.257095158597664, 53.2036)(34.357262103505846, 53.50023)(34.45742904841403, 53.41112)(34.5575959933222, 53.48253)(34.657762938230384, 52.977160000000005)(34.757929883138566, 52.412580000000005)(34.85809682804675, 53.51549)(34.95826377295492, 53.4038)(35.058430717863104, 52.77208)(35.158597662771285, 53.5869)(35.25876460767947, 53.53136)(35.35893155258764, 53.57531)(35.45909849749582, 52.64941)(35.559265442404005, 53.155989999999996)(35.659432387312194, 53.257310000000004)(35.75959933222037, 53.87255)(35.85976627712855, 53.71996)(35.95993322203673, 53.47642999999999)(36.06010016694491, 52.78489999999999)(36.16026711185309, 53.18224)(36.26043405676127, 52.251450000000006)(36.36060100166945, 53.894510000000004)(36.46076794657763, 53.22679)(36.56093489148581, 53.33849)(36.66110183639399, 53.172470000000004)(36.76126878130217, 53.17369)(36.86143572621035, 53.50084)(36.96160267111853, 52.42479)(37.06176961602671, 53.52280999999999)(37.16193656093489, 53.38732)(37.26210350584308, 52.50292)(37.362270450751254, 53.31774)(37.462437395659435, 53.323840000000004)(37.56260434056762, 53.50450000000001)(37.6627712854758, 53.12791)(37.76293823038397, 52.742180000000005)(37.863105175292155, 53.58446)(37.96327212020034, 53.45873)(38.06343906510852, 53.77794)(38.16360601001669, 53.61802)(38.263772954924875, 53.49657)(38.363939899833056, 53.57896)(38.46410684474124, 53.2744)(38.56427378964941, 53.53197)(38.664440734557594, 52.59447)(38.764607679465776, 52.983869999999996)(38.86477462437396, 53.677229999999994)(38.96494156928214, 53.38304)(39.06510851419032, 51.91454)(39.1652754590985, 53.111439999999995)(39.26544240400668, 52.78734)(39.36560934891486, 53.45384)(39.46577629382304, 52.38145)(39.56594323873122, 53.693099999999994)(39.666110183639404, 52.81237)(39.76627712854758, 53.06505)(39.86644407345576, 53.29454)(39.96661101836394, 52.88805)(40.066777963272116, 53.53624)(40.1669449081803, 53.018660000000004)(40.26711185308848, 53.007059999999996)(40.36727879799666, 53.47215)(40.46744574290484, 52.8502)(40.567612687813025, 53.278059999999996)(40.667779632721206, 53.12975)(40.76794657762939, 53.246320000000004)(40.86811352253756, 53.30736)(40.968280467445744, 53.609489999999994)(41.068447412353926, 53.541740000000004)(41.16861435726211, 53.39036)(41.26878130217028, 53.84263)(41.368948247078464, 53.1682)(41.469115191986646, 53.020500000000006)(41.56928213689483, 53.71813)(41.669449081803, 53.31347)(41.769616026711184, 53.46971)(41.869782971619365, 53.8347)(41.96994991652755, 52.41442)(42.07011686143572, 53.58933999999999)(42.1702838063439, 53.16332)(42.27045075125209, 53.13402)(42.370617696160274, 53.21092)(42.47078464106845, 52.97899)(42.57095158597663, 53.895739999999996)(42.67111853088481, 53.21458)(42.77128547579299, 53.156)(42.87145242070117, 52.87157)(42.97161936560935, 53.413560000000004)(43.07178631051753, 52.801989999999996)(43.17195325542571, 53.39891)(43.27212020033389, 52.78002)(43.37228714524207, 53.62047)(43.47245409015025, 53.104110000000006)(43.57262103505843, 53.59057)(43.67278797996661, 53.42821)(43.77295492487479, 53.643660000000004)(43.87312186978298, 52.76903)(43.97328881469116, 53.959830000000004)(44.073455759599334, 53.682109999999994)(44.173622704507515, 52.480940000000004)(44.2737896494157, 53.31346)(44.37395659432388, 53.06139)(44.47412353923205, 52.86058)(44.574290484140235, 52.79101)(44.67445742904842, 53.66746)(44.7746243739566, 53.27196)(44.87479131886477, 52.13243)(44.974958263772955, 53.23839)(45.075125208681136, 52.86912)(45.17529215358932, 52.8264)(45.27545909849749, 53.634510000000006)(45.375626043405674, 52.87889)(45.475792988313856, 53.50634)(45.57595993322204, 53.23961)(45.67612687813022, 53.462990000000005)(45.7762938230384, 52.1123)(45.87646076794658, 53.643660000000004)(45.976627712854764, 54.07335)(46.07679465776294, 53.041850000000004)(46.17696160267112, 53.55028)(46.2771285475793, 52.990579999999994)(46.37729549248748, 52.62804)(46.47746243739566, 53.38487)(46.57762938230384, 53.05773000000001)(46.67779632721202, 53.385490000000004)(46.7779632721202, 52.39061)(46.87813021702838, 53.416)(46.97829716193656, 53.542959999999994)(47.07846410684474, 52.90453)(47.17863105175292, 53.03575000000001)(47.278797996661105, 52.999739999999996)(47.378964941569286, 52.9796)(47.47913188647747, 53.70286)(47.57929883138564, 53.76146)(47.679465776293824, 52.836169999999996)(47.779632721202006, 53.0736)(47.87979966611019, 53.29577)(47.97996661101836, 53.17614)(48.080133555926544, 53.53135)(48.180300500834726, 53.41722)(48.28046744574291, 53.80968)(48.38063439065108, 53.60704)(48.48080133555926, 53.2744)(48.580968280467445, 53.64183)(48.68113522537563, 53.587509999999995)(48.7813021702838, 53.68334)(48.88146911519199, 53.07177)(48.98163606010017, 53.59911)(49.081803005008354, 53.31225)(49.18196994991653, 53.46666)(49.28213689482471, 53.33605)(49.38230383973289, 53.46727)(49.48247078464107, 53.20788)(49.58263772954925, 53.9348)(49.68280467445743, 53.4044)(49.78297161936561, 53.18407)(49.88313856427379, 52.40282)(49.98330550918197, 53.09252)(50.08347245409015, 53.56065)(50.18363939899833, 53.29332)(50.28380634390651, 53.639390000000006)(50.38397328881469, 53.48863)(50.484140233722876, 53.63571999999999)(50.58430717863106, 53.12853)(50.68447412353924, 53.60948)(50.784641068447414, 53.296369999999996)(50.884808013355595, 52.68846)(50.98497495826378, 53.80052)(51.08514190317196, 53.70897)(51.18530884808013, 53.259150000000005)(51.285475792988315, 53.43308999999999)(51.3856427378965, 54.048930000000006)(51.48580968280468, 53.38182)(51.58597662771285, 53.05955)(51.686143572621035, 53.715070000000004)(51.786310517529216, 53.31163)(51.8864774624374, 53.72972)(51.98664440734557, 53.987899999999996)(52.086811352253754, 52.99119)(52.18697829716194, 53.05346)(52.287145242070125, 53.759629999999994)(52.3873121869783, 53.614369999999994)(52.48747913188648, 52.49926000000001)(52.58764607679466, 53.42455)(52.68781302170284, 53.708349999999996)(52.78797996661102, 53.30919)(52.8881469115192, 53.324439999999996)(52.98831385642738, 54.0172)(53.088480801335564, 53.19627)(53.18864774624374, 52.560289999999995)(53.28881469115192, 53.7108)(53.3889816360601, 53.461169999999996)(53.489148580968276, 52.73607)(53.58931552587646, 53.059560000000005)(53.68948247078464, 53.14256)(53.78964941569283, 52.93016)(53.889816360601, 53.62963)(53.989983305509185, 54.21983)(54.090150250417366, 53.0266)(54.19031719532555, 53.60948)(54.29048414023372, 52.91246)(54.390651085141904, 53.11571)(54.490818030050086, 53.58873)(54.59098497495827, 52.81236)(54.69115191986644, 53.126689999999996)(54.791318864774624, 53.76939)(54.891485809682806, 53.31468)(54.99165275459099, 53.53319)(55.09181969949916, 53.7639)(55.19198664440734, 53.369609999999994)(55.292153589315525, 53.17308)(55.39232053422371, 52.61339)(55.49248747913189, 53.86339)(55.59265442404007, 53.92015)(55.69282136894825, 53.20787)(55.79298831385643, 53.569199999999995)(55.89315525876461, 52.59447)(55.99332220367279, 53.53441)(56.09348914858097, 53.19628)(56.19365609348915, 53.61009)(56.29382303839733, 53.07665)(56.39398998330551, 53.67905999999999)(56.49415692821369, 53.574690000000004)(56.59432387312187, 53.590559999999996)(56.69449081803005, 53.729110000000006)(56.79465776293823, 53.71629)(56.89482470784641, 52.764759999999995)(56.99499165275459, 53.23777)(57.095158597662774, 53.67846)(57.195325542570956, 53.583239999999996)(57.29549248747914, 52.87461999999999)(57.39565943238732, 53.81822)(57.495826377295494, 53.036370000000005)(57.595993322203675, 53.273179999999996)(57.69616026711186, 53.74253)(57.79632721202004, 53.72239999999999)(57.89649415692821, 52.858140000000006)(57.996661101836395, 53.74253)(58.09682804674458, 52.825179999999996)(58.19699499165276, 52.876459999999994)(58.29716193656093, 53.18162)(58.397328881469114, 53.78038)(58.497495826377296, 53.53807)(58.59766277128548, 53.395250000000004)(58.69782971619365, 53.81761)(58.79799666110184, 54.0117)(58.89816360601002, 53.35924)(58.9983305509182, 53.25304)(59.09849749582638, 53.84996)(59.19866444073456, 53.58629)(59.29883138564274, 53.89452)(59.398998330550924, 53.81455)(59.4991652754591, 53.21030999999999)(59.59933222036728, 52.88439)(59.69949916527546, 54.194199999999995)(59.79966611018364, 53.959210000000006)(59.89983305509182, 53.7877)(60.0, 53.270739999999996)
        };
        \addplot [pattern=north east lines,pattern color=red] 
        fill between [
            of=A and B,soft clip={domain=0:800},
        ];
        \end{axis}
\end{tikzpicture}
\caption{Test case: FannkuchRedux}
\end{subfigure}
\begin{subfigure}[b]{0.49\linewidth}
    \begin{tikzpicture}
        \pgfplotsset{%
        width=1\linewidth,
        % height=1\textheight
            }
        \begin{axis}[ymax=120,
        xlabel={Time (Seconds)},
        ylabel={Energy Consumption (Joules)},
        ]
        \addplot[color=blue, mark=none,] coordinates { %% AVG value
        (0.0, 0.0)(0.1001669449081803, 41.881535166666666)(0.2003338898163606, 44.46737491666669)(0.3005008347245409, 41.73337891666666)(0.4006677796327212, 41.69603983333333)(0.5008347245409015, 41.48155216666667)(0.6010016694490818, 41.64539608333333)(0.7011686143572621, 41.72418741666667)(0.8013355592654424, 41.717529083333346)(0.9015025041736228, 42.397515500000004)(1.001669449081803, 41.89016174999998)(1.1018363939899833, 41.23500125000001)(1.2020033388981637, 41.66399699999999)(1.3021702838063438, 42.0488628333333)(1.4023372287145242, 41.67816708333332)(1.5025041736227045, 42.34113400000001)(1.6026711185308848, 41.955982999999996)(1.7028380634390652, 41.77586291666666)(1.8030050083472455, 41.99256391666668)(1.9031719532554257, 41.721842249999995)(2.003338898163606, 41.90805958333332)(2.1035058430717863, 41.510535083333345)(2.2036727879799667, 41.89515083333333)(2.303839732888147, 41.9052935)(2.4040066777963274, 41.73178566666666)(2.5041736227045073, 41.57078625000001)(2.6043405676126876, 41.58492583333332)(2.704507512520868, 41.32442833333333)(2.8046744574290483, 41.782419333333316)(2.9048414023372287, 41.78312691666666)(3.005008347245409, 41.59574908333332)(3.1051752921535893, 41.71572358333333)(3.2053422370617697, 41.92462100000001)(3.30550918196995, 41.91482025)(3.4056761268781304, 41.88635166666667)(3.5058430717863107, 42.07804758333335)(3.606010016694491, 42.27862466666666)(3.7061769616026714, 41.88838116666667)(3.8063439065108513, 42.04770375000001)(3.906510851419032, 42.162225333333296)(4.006677796327212, 42.15119808333332)(4.106844741235393, 42.00819824999997)(4.207011686143573, 42.3818438333333)(4.3071786310517535, 42.09002108333333)(4.407345575959933, 42.088276499999985)(4.507512520868114, 42.08457358333336)(4.607679465776294, 42.14644199999997)(4.707846410684475, 41.89630116666667)(4.808013355592655, 41.92368025)(4.908180300500835, 42.063913166666666)(5.0083472454090145, 42.426873083333334)(5.108514190317195, 42.07187833333331)(5.208681135225375, 42.14474833333334)(5.308848080133556, 42.190662999999994)(5.409015025041736, 41.99111825000001)(5.509181969949917, 41.908131833333314)(5.609348914858097, 41.88442966666666)(5.709515859766277, 42.08243200000001)(5.809682804674457, 41.75007599999999)(5.909849749582638, 42.08743133333334)(6.010016694490818, 42.065892749999996)(6.110183639398999, 41.86274083333334)(6.210350584307179, 42.166650666666655)(6.3105175292153595, 42.34981641666668)(6.410684474123539, 42.21641933333334)(6.510851419031719, 41.876077499999994)(6.6110183639399, 41.96806808333333)(6.71118530884808, 41.65745033333334)(6.811352253756261, 41.70561791666666)(6.911519198664441, 41.82089733333333)(7.011686143572621, 42.22764908333334)(7.111853088480801, 41.865055583333316)(7.212020033388982, 42.02407800000001)(7.312186978297162, 41.709263750000005)(7.412353923205343, 41.596476583333356)(7.512520868113523, 41.91339058333334)(7.612687813021703, 41.866435)(7.712854757929884, 41.696395916666674)(7.813021702838064, 41.774434083333325)(7.913188647746244, 42.036208166666675)(8.013355592654424, 41.70985925)(8.113522537562606, 41.493108833333366)(8.213689482470786, 41.865778333333346)(8.313856427378965, 41.54591925)(8.414023372287145, 41.78408808333336)(8.514190317195327, 42.111677666666665)(8.614357262103507, 42.033660083333324)(8.714524207011687, 41.81489033333334)(8.814691151919867, 41.514003083333336)(8.914858096828048, 41.86404391666667)(9.015025041736228, 41.77895083333333)(9.115191986644408, 41.73567641666667)(9.215358931552588, 41.84109)(9.31552587646077, 41.72367866666666)(9.41569282136895, 42.004073999999996)(9.51585976627713, 41.800862166666654)(9.61602671118531, 41.97718725)(9.71619365609349, 41.86208033333335)(9.81636060100167, 41.62264033333334)(9.916527545909851, 41.74726350000001)(10.016694490818029, 41.85417658333332)(10.11686143572621, 41.85566683333333)(10.21702838063439, 41.61671491666667)(10.31719532554257, 41.833735000000004)(10.41736227045075, 41.92051124999998)(10.51752921535893, 41.54059933333331)(10.617696160267112, 41.660659333333356)(10.717863105175292, 41.86194841666667)(10.818030050083472, 41.91936166666666)(10.918196994991652, 41.76104224999998)(11.018363939899833, 41.88861541666667)(11.118530884808013, 41.53186083333333)(11.218697829716193, 41.45277991666668)(11.318864774624373, 41.42977483333333)(11.419031719532555, 41.960642000000014)(11.519198664440735, 42.01746524999998)(11.619365609348915, 42.01590391666666)(11.719532554257095, 41.979471750000016)(11.819699499165276, 42.06625208333332)(11.919866444073456, 41.82889283333332)(12.020033388981636, 41.742213166666666)(12.120200333889816, 41.5634615)(12.220367278797998, 41.81552141666667)(12.320534223706177, 41.7192275)(12.420701168614357, 41.74203525000001)(12.520868113522537, 41.65402783333335)(12.621035058430719, 41.66119866666666)(12.721202003338899, 41.81253591666665)(12.821368948247079, 41.85684133333333)(12.921535893155259, 41.732375999999974)(13.021702838063439, 42.100971166666675)(13.12186978297162, 42.186776916666666)(13.2220367278798, 41.81285591666666)(13.32220367278798, 41.81564808333335)(13.42237061769616, 41.63489325)(13.522537562604342, 41.79242916666665)(13.622704507512521, 41.84061658333332)(13.722871452420701, 41.83725966666666)(13.823038397328881, 42.195764416666655)(13.923205342237063, 41.99023808333332)(14.023372287145243, 41.99149008333334)(14.123539232053423, 41.87817333333334)(14.223706176961603, 41.52837150000001)(14.323873121869784, 41.94029183333334)(14.424040066777964, 41.96863275000001)(14.524207011686144, 42.03219541666667)(14.624373956594324, 41.65530424999999)(14.724540901502506, 41.64070650000001)(14.824707846410686, 41.66945366666667)(14.924874791318866, 41.54367591666666)(15.025041736227045, 41.95259091666666)(15.125208681135225, 41.84677075000002)(15.225375626043405, 41.917541166666666)(15.325542570951589, 42.050861333333344)(15.425709515859769, 41.85530591666669)(15.525876460767948, 41.80946350000001)(15.626043405676128, 42.21421683333332)(15.726210350584308, 42.10993875)(15.826377295492488, 41.95130816666667)(15.926544240400668, 42.01118433333334)(16.026711185308848, 41.75650050000001)(16.126878130217026, 41.88210999999999)(16.22704507512521, 41.9312080833333)(16.32721202003339, 42.04116191666665)(16.42737896494157, 42.09987225)(16.52754590984975, 41.83759591666665)(16.62771285475793, 41.92235758333333)(16.72787979966611, 41.85495966666669)(16.82804674457429, 41.91470291666667)(16.92821368948247, 41.67396533333335)(17.028380634390654, 41.98660758333332)(17.128547579298832, 41.943852250000006)(17.228714524207014, 41.48570733333334)(17.328881469115192, 41.7966925)(17.429048414023374, 41.7918389166667)(17.529215358931552, 41.77178924999997)(17.629382303839733, 41.598526250000006)(17.72954924874791, 42.12402825000001)(17.829716193656097, 41.77167725000002)(17.929883138564275, 41.608323)(18.030050083472457, 41.522685416666675)(18.130217028380635, 41.67701141666666)(18.230383973288816, 41.919951833333315)(18.330550918196995, 41.989221666666666)(18.430717863105176, 42.22942483333332)(18.530884808013354, 41.98187191666666)(18.63105175292154, 41.79114266666669)(18.731218697829718, 42.17679224999999)(18.8313856427379, 42.046762249999986)(18.931552587646078, 41.98695299999998)(19.03171953255426, 42.164209)(19.131886477462437, 41.736587500000006)(19.23205342237062, 42.126117916666665)(19.332220367278797, 41.98282774999998)(19.43238731218698, 41.87178016666667)(19.53255425709516, 42.09568191666665)(19.63272120200334, 42.01339733333335)(19.73288814691152, 41.944197166666655)(19.833055091819702, 42.09712691666668)(19.93322203672788, 41.75758816666666)(20.033388981636058, 41.609410916666654)(20.13355592654424, 41.68139658333333)(20.23372287145242, 41.446356166666654)(20.333889816360603, 41.985141999999996)(20.43405676126878, 42.26146866666665)(20.534223706176963, 41.698775750000024)(20.63439065108514, 41.82216908333335)(20.734557595993323, 42.15274399999999)(20.8347245409015, 42.18701591666669)(20.934891485809683, 42.0718676666667)(21.03505843071786, 41.75697391666667)(21.135225375626046, 41.76455583333335)(21.235392320534224, 42.11232991666665)(21.335559265442406, 41.94641025)(21.435726210350584, 41.84656733333334)(21.535893155258766, 41.78599033333333)(21.636060100166944, 41.58683808333332)(21.736227045075125, 41.56036983333334)(21.836393989983303, 41.900527083333344)(21.93656093489149, 42.015135249999986)(22.036727879799667, 42.114812)(22.13689482470785, 41.96194341666669)(22.237061769616027, 42.00938391666668)(22.33722871452421, 41.76011683333331)(22.437395659432386, 41.84950666666667)(22.537562604340568, 41.81193008333333)(22.637729549248746, 41.587595916666665)(22.737896494156928, 41.76163225000002)(22.83806343906511, 41.697764083333354)(22.93823038397329, 41.839854333333356)(23.03839732888147, 42.03446808333334)(23.13856427378965, 42.248706750000004)(23.23873121869783, 41.755467499999995)(23.33889816360601, 41.702621750000034)(23.43906510851419, 42.13714516666668)(23.53923205342237, 42.405190333333344)(23.639398998330552, 41.90561366666667)(23.739565943238734, 42.107736250000016)(23.839732888146912, 41.72878483333332)(23.939899833055094, 41.74500058333334)(24.040066777963272, 41.871240499999985)(24.140233722871454, 41.956949916666666)(24.24040066777963, 41.98480116666668)(24.340567612687813, 41.987451500000006)(24.440734557595995, 41.786915999999984)(24.540901502504177, 41.47185333333336)(24.641068447412355, 41.59094775000002)(24.741235392320537, 41.684209333333335)(24.841402337228715, 41.876317166666674)(24.941569282136896, 41.78248600000001)(25.041736227045075, 41.79703233333333)(25.141903171953256, 41.79602983333332)(25.242070116861438, 41.952193666666666)(25.34223706176962, 41.905649833333335)(25.442404006677798, 41.906401750000015)(25.54257095158598, 42.21109916666668)(25.642737896494157, 41.87547266666668)(25.74290484140234, 42.17241824999998)(25.843071786310517, 41.984729999999985)(25.9432387312187, 41.804936916666676)(26.043405676126877, 41.80148733333333)(26.143572621035062, 41.89468333333334)(26.24373956594324, 41.731115666666675)(26.34390651085142, 41.51893075)(26.4440734557596, 41.849523166666664)(26.544240400667782, 42.27399049999999)(26.64440734557596, 41.80315116666667)(26.744574290484138, 41.425207)(26.84474123539232, 41.916483333333325)(26.9449081803005, 42.15724600000001)(27.045075125208683, 42.08579425000002)(27.14524207011686, 42.05528666666668)(27.245409015025043, 42.293882416666676)(27.34557595993322, 41.79515058333333)(27.445742904841403, 41.80368533333331)(27.54590984974958, 41.90908308333333)(27.646076794657763, 42.182788833333326)(27.746243739565944, 42.319842583333326)(27.846410684474126, 42.304838583333336)(27.946577629382304, 42.127679333333326)(28.046744574290486, 42.02904741666666)(28.146911519198664, 42.040964)(28.247078464106846, 41.90279041666667)(28.347245409015024, 41.87694774999999)(28.447412353923205, 41.84762549999997)(28.547579298831387, 42.071501583333315)(28.64774624373957, 41.865137749999995)(28.747913188647747, 41.89296341666666)(28.84808013355593, 41.97239708333334)(28.948247078464107, 41.69151833333334)(29.04841402337229, 41.803786666666674)(29.148580968280466, 41.735545166666654)(29.248747913188648, 41.90939791666667)(29.348914858096826, 42.00493250000002)(29.44908180300501, 42.362679499999985)(29.54924874791319, 42.104994500000004)(29.64941569282137, 42.441913416666665)(29.74958263772955, 42.149108249999976)(29.84974958263773, 42.00753174999999)(29.94991652754591, 42.021061499999995)(30.05008347245409, 42.09126641666667)(30.15025041736227, 41.76799033333335)(30.25041736227045, 41.63217666666667)(30.35058430717863, 41.674072666666675)(30.45075125208681, 41.70216366666664)(30.55091819699499, 41.732742750000014)(30.651085141903177, 41.796701000000006)(30.751252086811355, 42.00201941666666)(30.851419031719537, 42.10757841666665)(30.951585976627715, 41.903538)(31.051752921535897, 41.92486525000002)(31.151919866444075, 42.111948250000005)(31.252086811352257, 41.925729166666656)(31.352253756260435, 42.19884641666663)(31.452420701168617, 42.076028666666645)(31.552587646076795, 41.92287166666667)(31.652754590984976, 41.821701166666685)(31.752921535893154, 41.82477216666667)(31.853088480801336, 42.02016641666667)(31.953255425709514, 41.91637100000001)(32.053422370617696, 41.52150033333334)(32.15358931552588, 41.912210749999986)(32.25375626043405, 41.64734424999998)(32.35392320534224, 42.1688015)(32.45409015025042, 42.114450166666664)(32.554257095158604, 42.07548383333333)(32.65442404006678, 41.843953583333345)(32.75459098497496, 41.90815691666667)(32.85475792988314, 42.24867583333332)(32.95492487479132, 41.977518166666684)(33.0550918196995, 42.16236800000001)(33.15525876460768, 41.99138258333335)(33.25542570951586, 41.68736358333336)(33.35559265442404, 41.64856424999998)(33.45575959933222, 41.56226658333334)(33.5559265442404, 41.73629275)(33.65609348914858, 41.68268824999999)(33.756260434056756, 41.71000699999999)(33.85642737896494, 41.81153825)(33.95659432387313, 41.759201250000004)(34.05676126878131, 41.94915174999998)(34.15692821368948, 41.984166)(34.257095158597664, 41.99156624999998)(34.357262103505846, 41.956084250000025)(34.45742904841403, 41.93881691666666)(34.5575959933222, 41.94694491666668)(34.657762938230384, 42.24986641666666)(34.757929883138566, 41.824136500000016)(34.85809682804675, 42.21531058333332)(34.95826377295492, 42.229531749999985)(35.058430717863104, 42.14816724999999)(35.158597662771285, 42.22252291666665)(35.25876460767947, 42.11339674999999)(35.35893155258764, 41.9422245)(35.45909849749582, 41.76821891666665)(35.559265442404005, 41.732239250000035)(35.659432387312194, 42.06488408333332)(35.75959933222037, 41.994938333333344)(35.85976627712855, 42.024352749999984)(35.95993322203673, 42.04811008333331)(36.06010016694491, 42.002847166666655)(36.16026711185309, 42.02251641666667)(36.26043405676127, 42.03663016666665)(36.36060100166945, 41.82928474999997)(36.46076794657763, 41.82655274999998)(36.56093489148581, 41.96867808333333)(36.66110183639399, 42.1773875)(36.76126878130217, 42.035770583333345)(36.86143572621035, 42.123850583333336)(36.96160267111853, 42.27564791666666)(37.06176961602671, 42.60398074999999)(37.16193656093489, 42.579104750000006)(37.26210350584308, 42.423938166666666)(37.362270450751254, 41.90579183333334)(37.462437395659435, 42.29563758333331)(37.56260434056762, 42.17407658333333)(37.6627712854758, 42.141325666666674)(37.76293823038397, 42.134169416666666)(37.863105175292155, 42.18594233333333)(37.96327212020034, 42.42950266666665)(38.06343906510852, 42.339984583333354)(38.16360601001669, 42.43243708333332)(38.263772954924875, 42.00913483333332)(38.363939899833056, 42.06707133333334)(38.46410684474124, 42.315682249999995)(38.56427378964941, 42.08695391666665)(38.664440734557594, 42.28164066666668)(38.764607679465776, 42.099837166666674)(38.86477462437396, 42.234094083333346)(38.96494156928214, 42.349312583333344)(39.06510851419032, 41.96495491666666)(39.1652754590985, 42.44622116666669)(39.26544240400668, 41.949085750000016)(39.36560934891486, 42.18524591666667)(39.46577629382304, 42.044463249999986)(39.56594323873122, 42.493212166666666)(39.666110183639404, 42.07623233333334)(39.76627712854758, 41.99746649999999)(39.86644407345576, 42.02027866666666)(39.96661101836394, 41.96908458333334)(40.066777963272116, 42.20076941666667)(40.1669449081803, 42.46880383333335)(40.26711185308848, 42.15998741666668)(40.36727879799666, 42.04052141666667)(40.46744574290484, 42.07057608333335)(40.567612687813025, 42.02786658333333)(40.667779632721206, 42.12897649999998)(40.76794657762939, 41.99397183333334)(40.86811352253756, 42.12833083333332)(40.968280467445744, 41.96970525)(41.068447412353926, 41.99767999999999)(41.16861435726211, 42.20075366666665)(41.26878130217028, 42.22277683333332)(41.368948247078464, 42.049687083333346)(41.469115191986646, 41.894205083333325)(41.56928213689483, 42.07839925)(41.669449081803, 42.09745675000001)(41.769616026711184, 42.15915716666668)(41.869782971619365, 42.04344741666667)(41.96994991652755, 42.35721075000001)(42.07011686143572, 42.26493733333333)(42.1702838063439, 41.930200749999976)(42.27045075125209, 42.51790191666667)(42.370617696160274, 41.98118000000001)(42.47078464106845, 42.36766908333331)(42.57095158597663, 42.02121399999998)(42.67111853088481, 42.38250491666668)(42.77128547579299, 42.13599583333334)(42.87145242070117, 42.20384066666667)(42.97161936560935, 41.92365016666667)(43.07178631051753, 42.247028250000014)(43.17195325542571, 42.23755283333333)(43.27212020033389, 42.29379116666666)(43.37228714524207, 42.00868166666666)(43.47245409015025, 42.10327533333331)(43.57262103505843, 42.3400045)(43.67278797996661, 42.16879199999999)(43.77295492487479, 41.82466599999999)(43.87312186978298, 42.25559841666667)(43.97328881469116, 42.295515666666674)(44.073455759599334, 42.152581749999996)(44.173622704507515, 41.94049991666667)(44.2737896494157, 41.92405225)(44.37395659432388, 42.18860691666667)(44.47412353923205, 42.260512250000005)(44.574290484140235, 42.25148358333335)(44.67445742904842, 41.94034300000001)(44.7746243739566, 42.08498983333334)(44.87479131886477, 42.04023758333334)(44.974958263772955, 41.951770916666646)(45.075125208681136, 42.10377999999998)(45.17529215358932, 42.497392916666655)(45.27545909849749, 42.398176583333324)(45.375626043405674, 42.10310266666667)(45.475792988313856, 42.22072199999999)(45.57595993322204, 42.282338166666655)(45.67612687813022, 42.22361458333332)(45.7762938230384, 42.39705833333334)(45.87646076794658, 41.9765055)(45.976627712854764, 42.387439666666666)(46.07679465776294, 42.46931758333334)(46.17696160267112, 42.30384666666667)(46.2771285475793, 42.52824741666666)(46.37729549248748, 42.246570083333346)(46.47746243739566, 42.248727750000015)(46.57762938230384, 42.312075916666664)(46.67779632721202, 42.40738183333334)(46.7779632721202, 42.20717333333335)(46.87813021702838, 42.21530516666667)(46.97829716193656, 42.52990500000003)(47.07846410684474, 42.11945008333331)(47.17863105175292, 42.21126666666666)(47.278797996661105, 42.10496441666666)(47.378964941569286, 42.09903849999998)(47.47913188647747, 42.11308199999999)(47.57929883138564, 42.50535399999999)(47.679465776293824, 42.344140000000024)(47.779632721202006, 42.24309166666665)(47.87979966611019, 41.997039249999986)(47.97996661101836, 42.475558083333325)(48.080133555926544, 42.38885816666667)(48.180300500834726, 42.38406666666666)(48.28046744574291, 42.32571833333332)(48.38063439065108, 42.39447358333337)(48.48080133555926, 42.15746466666668)(48.580968280467445, 42.134001583333315)(48.68113522537563, 42.106072999999995)(48.7813021702838, 41.89365100000001)(48.88146911519199, 41.59857691666666)(48.98163606010017, 41.947667166666676)(49.081803005008354, 41.859150666666665)(49.18196994991653, 41.884404249999996)(49.28213689482471, 42.0849595)(49.38230383973289, 42.25574141666666)(49.48247078464107, 42.25000325)(49.58263772954925, 42.274173749999974)(49.68280467445743, 42.21550425)(49.78297161936561, 42.35223150000002)(49.88313856427379, 42.10827033333333)(49.98330550918197, 42.07015950000003)(50.08347245409015, 41.94872508333332)(50.18363939899833, 42.06747316666668)(50.28380634390651, 42.07055091666665)(50.38397328881469, 42.16526174999998)(50.484140233722876, 42.28542449999998)(50.58430717863106, 41.930388499999985)(50.68447412353924, 42.10927233333335)(50.784641068447414, 41.82393858333332)(50.884808013355595, 42.28740791666667)(50.98497495826378, 42.42061191666668)(51.08514190317196, 42.45713641666668)(51.18530884808013, 42.1337475)(51.285475792988315, 42.548820666666685)(51.3856427378965, 42.17655841666668)(51.48580968280468, 42.27805949999999)(51.58597662771285, 42.27961591666665)(51.686143572621035, 42.42574383333332)(51.786310517529216, 42.212594083333336)(51.8864774624374, 42.42362300000002)(51.98664440734557, 42.32598725)(52.086811352253754, 42.36243008333333)(52.18697829716194, 42.19362799999999)(52.287145242070125, 42.120756833333346)(52.3873121869783, 42.29004275)(52.48747913188648, 42.033629916666676)(52.58764607679466, 41.86756799999999)(52.68781302170284, 42.05071441666668)(52.78797996661102, 42.175663333333354)(52.8881469115192, 42.03768833333334)(52.98831385642738, 42.24157058333332)(53.088480801335564, 42.06940166666668)(53.18864774624374, 42.20439008333335)(53.28881469115192, 42.19404441666666)(53.3889816360601, 42.43084591666668)(53.489148580968276, 42.56137816666665)(53.58931552587646, 42.310306499999996)(53.68948247078464, 42.43097783333334)(53.78964941569283, 42.19833258333333)(53.889816360601, 42.26344108333335)(53.989983305509185, 42.56421199999999)(54.090150250417366, 42.57704975000001)(54.19031719532555, 42.49660066666668)(54.29048414023372, 42.340681416666705)(54.390651085141904, 42.32830150000002)(54.490818030050086, 42.37829408333335)(54.59098497495827, 42.779248083333364)(54.69115191986644, 42.55173600000001)(54.791318864774624, 42.24515149999999)(54.891485809682806, 42.23147399999997)(54.99165275459099, 42.28066883333334)(55.09181969949916, 42.156151833333325)(55.19198664440734, 42.08090175)(55.292153589315525, 42.30447241666665)(55.39232053422371, 42.421476916666656)(55.49248747913189, 42.129667916666655)(55.59265442404007, 42.17960491666666)(55.69282136894825, 42.214298416666644)(55.79298831385643, 42.1316769166667)(55.89315525876461, 42.047153916666666)(55.99332220367279, 42.126407500000006)(56.09348914858097, 41.787938499999996)(56.19365609348915, 42.092136499999995)(56.29382303839733, 42.28994641666666)(56.39398998330551, 42.33067658333332)(56.49415692821369, 42.334206500000015)(56.59432387312187, 42.19510808333334)(56.69449081803005, 42.136600416666674)(56.79465776293823, 42.25547675000003)(56.89482470784641, 42.52287608333335)(56.99499165275459, 42.55362716666666)(57.095158597662774, 42.472537416666675)(57.195325542570956, 42.27398025000001)(57.29549248747914, 42.48109300000001)(57.39565943238732, 41.994505749999995)(57.495826377295494, 41.89802974999999)(57.595993322203675, 42.3906745)(57.69616026711186, 42.248899666666674)(57.79632721202004, 42.36990674999999)(57.89649415692821, 42.27449425000001)(57.996661101836395, 42.302209250000004)(58.09682804674458, 42.252506333333365)(58.19699499165276, 42.22731908333333)(58.29716193656093, 42.26320733333331)(58.397328881469114, 42.50271941666665)(58.497495826377296, 42.403888333333335)(58.59766277128548, 42.19145575)(58.69782971619365, 42.22705991666667)(58.79799666110184, 42.16820116666668)(58.89816360601002, 42.353854833333344)(58.9983305509182, 42.52324283333333)(59.09849749582638, 42.477979083333324)(59.19866444073456, 42.30578516666667)(59.29883138564274, 42.296562750000014)(59.398998330550924, 42.09732991666668)(59.4991652754591, 42.05373549999999)(59.59933222036728, 42.014678249999996)(59.69949916527546, 42.160109333333324)(59.79966611018364, 42.13889916666667)(59.89983305509182, 42.42456408333335)(60.0, 42.12976958333333)
        };
        \addplot[color=blue, mark=none,name path=A] coordinates { %% MAX value
        (0.0, 0.0)(0.1001669449081803, 46.81933)(0.2003338898163606, 50.4143)(0.3005008347245409, 48.27381)(0.4006677796327212, 55.92882)(0.5008347245409015, 46.04114)(0.6010016694490818, 58.008889999999994)(0.7011686143572621, 53.52769000000001)(0.8013355592654424, 54.89488)(0.9015025041736228, 53.51488)(1.001669449081803, 45.74695)(1.1018363939899833, 46.486689999999996)(1.2020033388981637, 46.25171)(1.3021702838063438, 46.18457)(1.4023372287145242, 46.289550000000006)(1.5025041736227045, 45.94959)(1.6026711185308848, 46.433600000000006)(1.7028380634390652, 46.3457)(1.8030050083472455, 46.53857000000001)(1.9031719532554257, 46.08936)(2.003338898163606, 46.33594000000001)(2.1035058430717863, 46.08204)(2.2036727879799667, 45.88611)(2.303839732888147, 46.22791)(2.4040066777963274, 46.929809999999996)(2.5041736227045073, 45.851940000000006)(2.6043405676126876, 45.63282)(2.704507512520868, 46.07105)(2.8046744574290483, 46.11804)(2.9048414023372287, 45.95202999999999)(3.005008347245409, 46.05823)(3.1051752921535893, 46.603269999999995)(3.2053422370617697, 46.18091)(3.30550918196995, 46.11377)(3.4056761268781304, 46.53308)(3.5058430717863107, 46.48059)(3.606010016694491, 46.31763)(3.7061769616026714, 46.508669999999995)(3.8063439065108513, 46.003299999999996)(3.906510851419032, 47.02319)(4.006677796327212, 45.97278)(4.106844741235393, 46.13574)(4.207011686143573, 46.31397)(4.3071786310517535, 45.862309999999994)(4.407345575959933, 46.51294)(4.507512520868114, 46.212039999999995)(4.607679465776294, 51.46105)(4.707846410684475, 47.29297)(4.808013355592655, 45.68103)(4.908180300500835, 46.49768)(5.0083472454090145, 54.29247)(5.108514190317195, 46.81568)(5.208681135225375, 46.546510000000005)(5.308848080133556, 46.29444)(5.409015025041736, 45.964850000000006)(5.509181969949917, 46.2096)(5.609348914858097, 46.57947)(5.709515859766277, 46.30298)(5.809682804674457, 46.15223)(5.909849749582638, 45.91236000000001)(6.010016694490818, 47.115970000000004)(6.110183639398999, 46.41712)(6.210350584307179, 46.75159)(6.3105175292153595, 45.961800000000004)(6.410684474123539, 47.39306)(6.510851419031719, 46.22363)(6.6110183639399, 46.268190000000004)(6.71118530884808, 45.99964)(6.811352253756261, 45.496100000000006)(6.911519198664441, 46.028330000000004)(7.011686143572621, 45.77136)(7.111853088480801, 46.568490000000004)(7.212020033388982, 45.8025)(7.312186978297162, 46.47083000000001)(7.412353923205343, 45.676759999999994)(7.512520868113523, 46.23096)(7.612687813021703, 46.09669)(7.712854757929884, 46.27308)(7.813021702838064, 46.09485)(7.913188647746244, 46.69238)(8.013355592654424, 46.05151)(8.113522537562606, 46.509280000000004)(8.213689482470786, 46.02344000000001)(8.313856427378965, 46.600840000000005)(8.414023372287145, 46.160160000000005)(8.514190317195327, 46.34082)(8.614357262103507, 47.07019)(8.714524207011687, 46.770509999999994)(8.814691151919867, 46.57214)(8.914858096828048, 46.1272)(9.015025041736228, 46.5166)(9.115191986644408, 46.456179999999996)(9.215358931552588, 46.25049)(9.31552587646077, 46.48365)(9.41569282136895, 46.71558)(9.51585976627713, 46.874880000000005)(9.61602671118531, 46.25537)(9.71619365609349, 46.932249999999996)(9.81636060100167, 46.747319999999995)(9.916527545909851, 46.52942)(10.016694490818029, 46.320690000000006)(10.11686143572621, 46.75891)(10.21702838063439, 46.508669999999995)(10.31719532554257, 46.05701)(10.41736227045075, 46.29139000000001)(10.51752921535893, 46.76441)(10.617696160267112, 47.00489)(10.717863105175292, 46.982299999999995)(10.818030050083472, 46.00574)(10.918196994991652, 46.8053)(11.018363939899833, 45.63221)(11.118530884808013, 46.17054)(11.218697829716193, 46.061899999999994)(11.318864774624373, 46.246219999999994)(11.419031719532555, 46.692989999999995)(11.519198664440735, 47.28748)(11.619365609348915, 46.122930000000004)(11.719532554257095, 45.966680000000004)(11.819699499165276, 46.33045)(11.919866444073456, 46.099129999999995)(12.020033388981636, 46.185790000000004)(12.120200333889816, 46.00636)(12.220367278797998, 46.19007)(12.320534223706177, 46.00696)(12.420701168614357, 47.23315)(12.520868113522537, 46.525150000000004)(12.621035058430719, 45.85865)(12.721202003338899, 46.36462)(12.821368948247079, 46.77906)(12.921535893155259, 46.86084)(13.021702838063439, 53.73582)(13.12186978297162, 46.15527)(13.2220367278798, 46.05213)(13.32220367278798, 46.404300000000006)(13.42237061769616, 46.04419)(13.522537562604342, 49.904650000000004)(13.622704507512521, 45.99048)(13.722871452420701, 45.9386)(13.823038397328881, 49.99804)(13.923205342237063, 46.29444)(14.023372287145243, 46.59351)(14.123539232053423, 45.91907)(14.223706176961603, 45.92395)(14.323873121869784, 46.23768)(14.424040066777964, 46.422)(14.524207011686144, 46.10157)(14.624373956594324, 46.64295)(14.724540901502506, 46.20044)(14.824707846410686, 46.03687)(14.924874791318866, 46.38416)(15.025041736227045, 46.356089999999995)(15.125208681135225, 45.8031)(15.225375626043405, 46.88526)(15.325542570951589, 46.519040000000004)(15.425709515859769, 46.28284)(15.525876460767948, 46.560550000000006)(15.626043405676128, 46.03137)(15.726210350584308, 45.88977)(15.826377295492488, 46.41528)(15.926544240400668, 46.925540000000005)(16.026711185308848, 45.62976)(16.126878130217026, 45.81958)(16.22704507512521, 46.18824)(16.32721202003339, 46.180310000000006)(16.42737896494157, 45.909909999999996)(16.52754590984975, 46.769290000000005)(16.62771285475793, 45.90504)(16.72787979966611, 45.82264000000001)(16.82804674457429, 47.35644)(16.92821368948247, 46.04419)(17.028380634390654, 45.61084)(17.128547579298832, 47.103159999999995)(17.228714524207014, 45.58399)(17.328881469115192, 45.95264)(17.429048414023374, 45.950199999999995)(17.529215358931552, 46.4574)(17.629382303839733, 45.97828)(17.72954924874791, 46.70337)(17.829716193656097, 45.95509)(17.929883138564275, 46.1388)(18.030050083472457, 45.99719)(18.130217028380635, 45.88184)(18.230383973288816, 45.47657)(18.330550918196995, 45.98682)(18.430717863105176, 46.05274)(18.530884808013354, 46.34082)(18.63105175292154, 46.353030000000004)(18.731218697829718, 45.954480000000004)(18.8313856427379, 46.33167)(18.931552587646078, 46.018559999999994)(19.03171953255426, 46.72657)(19.131886477462437, 46.56727)(19.23205342237062, 45.88184)(19.332220367278797, 46.17969000000001)(19.43238731218698, 46.31335)(19.53255425709516, 45.99719999999999)(19.63272120200334, 46.10584)(19.73288814691152, 46.2273)(19.833055091819702, 46.05518)(19.93322203672788, 45.9148)(20.033388981636058, 46.2096)(20.13355592654424, 46.713139999999996)(20.23372287145242, 46.57215)(20.333889816360603, 46.846810000000005)(20.43405676126878, 46.92066)(20.534223706176963, 46.16749)(20.63439065108514, 46.033210000000004)(20.734557595993323, 46.58313)(20.8347245409015, 46.545899999999996)(20.934891485809683, 46.38904)(21.03505843071786, 47.18066)(21.135225375626046, 46.08631)(21.235392320534224, 46.085089999999994)(21.335559265442406, 46.84985)(21.435726210350584, 46.458009999999994)(21.535893155258766, 46.06372)(21.636060100166944, 46.19799999999999)(21.736227045075125, 45.81775999999999)(21.836393989983303, 46.43298)(21.93656093489149, 46.598389999999995)(22.036727879799667, 46.39454)(22.13689482470785, 46.56239)(22.237061769616027, 46.815059999999995)(22.33722871452421, 46.31519)(22.437395659432386, 46.434819999999995)(22.537562604340568, 46.311530000000005)(22.637729549248746, 46.434819999999995)(22.737896494156928, 46.57702999999999)(22.83806343906511, 45.999030000000005)(22.93823038397329, 46.68506)(23.03839732888147, 46.35547)(23.13856427378965, 46.703979999999994)(23.23873121869783, 46.184580000000004)(23.33889816360601, 46.23768)(23.43906510851419, 46.054570000000005)(23.53923205342237, 46.33595)(23.639398998330552, 46.40308)(23.739565943238734, 47.668949999999995)(23.839732888146912, 46.42077999999999)(23.939899833055094, 46.26575)(24.040066777963272, 46.35424999999999)(24.140233722871454, 46.45557)(24.24040066777963, 46.540400000000005)(24.340567612687813, 45.765269999999994)(24.440734557595995, 46.04174999999999)(24.540901502504177, 46.689949999999996)(24.641068447412355, 46.26392)(24.741235392320537, 46.60877)(24.841402337228715, 45.736580000000004)(24.941569282136896, 46.81995)(25.041736227045075, 46.97498)(25.141903171953256, 46.133309999999994)(25.242070116861438, 46.828500000000005)(25.34223706176962, 46.07471)(25.442404006677798, 46.043580000000006)(25.54257095158598, 47.163579999999996)(25.642737896494157, 45.83484)(25.74290484140234, 46.357299999999995)(25.843071786310517, 46.172979999999995)(25.9432387312187, 46.406130000000005)(26.043405676126877, 49.54089)(26.143572621035062, 65.0157)(26.24373956594324, 49.754509999999996)(26.34390651085142, 46.04786)(26.4440734557596, 46.06251)(26.544240400667782, 70.46857)(26.64440734557596, 52.299670000000006)(26.744574290484138, 46.58252)(26.84474123539232, 46.82056)(26.9449081803005, 46.331059999999994)(27.045075125208683, 52.739729999999994)(27.14524207011686, 58.425140000000006)(27.245409015025043, 58.015600000000006)(27.34557595993322, 59.09714)(27.445742904841403, 55.900740000000006)(27.54590984974958, 57.9979)(27.646076794657763, 59.290620000000004)(27.746243739565944, 58.37449)(27.846410684474126, 53.322010000000006)(27.946577629382304, 53.549670000000006)(28.046744574290486, 56.42869)(28.146911519198664, 56.67893)(28.247078464106846, 54.2888)(28.347245409015024, 57.35642)(28.447412353923205, 57.88621)(28.547579298831387, 59.68063)(28.64774624373957, 52.269769999999994)(28.747913188647747, 46.37012)(28.84808013355593, 46.411629999999995)(28.948247078464107, 45.84156)(29.04841402337229, 49.80639)(29.148580968280466, 59.9083)(29.248747913188648, 45.5846)(29.348914858096826, 45.89893)(29.44908180300501, 46.821169999999995)(29.54924874791319, 59.847260000000006)(29.64941569282137, 46.65271)(29.74958263772955, 46.646609999999995)(29.84974958263773, 46.46351)(29.94991652754591, 46.45252)(30.05008347245409, 46.81689)(30.15025041736227, 46.018559999999994)(30.25041736227045, 46.077149999999996)(30.35058430717863, 47.25696)(30.45075125208681, 45.62672)(30.55091819699499, 46.02649)(30.651085141903177, 46.21509)(30.751252086811355, 47.01709)(30.851419031719537, 46.199830000000006)(30.951585976627715, 46.578860000000006)(31.051752921535897, 46.72473)(31.151919866444075, 46.143679999999996)(31.252086811352257, 45.9917)(31.352253756260435, 46.233399999999996)(31.452420701168617, 46.27247)(31.552587646076795, 45.977050000000006)(31.652754590984976, 45.9386)(31.752921535893154, 46.088750000000005)(31.853088480801336, 46.46167)(31.953255425709514, 45.76771)(32.053422370617696, 46.617309999999996)(32.15358931552588, 46.40675)(32.25375626043405, 46.411629999999995)(32.35392320534224, 46.45496)(32.45409015025042, 46.563610000000004)(32.554257095158604, 46.59778)(32.65442404006678, 46.7876)(32.75459098497496, 46.69849)(32.85475792988314, 46.02772)(32.95492487479132, 46.445800000000006)(33.0550918196995, 46.874880000000005)(33.15525876460768, 46.91333)(33.25542570951586, 47.11658)(33.35559265442404, 46.73267)(33.45575959933222, 46.04481)(33.5559265442404, 45.9856)(33.65609348914858, 46.51294)(33.756260434056756, 46.70765)(33.85642737896494, 46.724740000000004)(33.95659432387313, 45.732299999999995)(34.05676126878131, 46.78821)(34.15692821368948, 46.18274)(34.257095158597664, 46.307869999999994)(34.357262103505846, 46.35853)(34.45742904841403, 46.73327)(34.5575959933222, 46.71008)(34.657762938230384, 46.30786)(34.757929883138566, 47.027469999999994)(34.85809682804675, 46.31519)(34.95826377295492, 46.96155)(35.058430717863104, 46.94385)(35.158597662771285, 46.254160000000006)(35.25876460767947, 47.08972)(35.35893155258764, 46.97132)(35.45909849749582, 45.93799)(35.559265442404005, 46.20472)(35.659432387312194, 46.53675)(35.75959933222037, 46.38476)(35.85976627712855, 46.4574)(35.95993322203673, 47.15381)(36.06010016694491, 47.23559)(36.16026711185309, 46.490359999999995)(36.26043405676127, 46.49341)(36.36060100166945, 46.12354)(36.46076794657763, 46.46473)(36.56093489148581, 46.32556)(36.66110183639399, 46.261469999999996)(36.76126878130217, 46.406130000000005)(36.86143572621035, 46.42139)(36.96160267111853, 46.4928)(37.06176961602671, 46.02771)(37.16193656093489, 47.0824)(37.26210350584308, 46.73877)(37.362270450751254, 46.85962)(37.462437395659435, 47.85754)(37.56260434056762, 46.55262)(37.6627712854758, 46.747319999999995)(37.76293823038397, 46.323730000000005)(37.863105175292155, 46.70154)(37.96327212020034, 46.74427)(38.06343906510852, 46.743050000000004)(38.16360601001669, 46.16382)(38.263772954924875, 46.41712)(38.363939899833056, 46.412839999999996)(38.46410684474124, 47.44007)(38.56427378964941, 46.35669)(38.664440734557594, 47.11719000000001)(38.764607679465776, 46.640499999999996)(38.86477462437396, 46.56116)(38.96494156928214, 46.20228)(39.06510851419032, 46.48242)(39.1652754590985, 46.36524)(39.26544240400668, 45.81654)(39.36560934891486, 46.32068)(39.46577629382304, 46.57459)(39.56594323873122, 46.134519999999995)(39.666110183639404, 47.350339999999996)(39.76627712854758, 46.468990000000005)(39.86644407345576, 46.27552)(39.96661101836394, 47.16785)(40.066777963272116, 46.81934)(40.1669449081803, 47.966800000000006)(40.26711185308848, 46.91516)(40.36727879799666, 47.170899999999996)(40.46744574290484, 46.67407)(40.567612687813025, 46.16992)(40.667779632721206, 46.57398)(40.76794657762939, 46.45313)(40.86811352253756, 47.66407)(40.968280467445744, 46.25171)(41.068447412353926, 46.032590000000006)(41.16861435726211, 46.33838)(41.26878130217028, 46.088139999999996)(41.368948247078464, 46.30847)(41.469115191986646, 46.39331)(41.56928213689483, 46.29505)(41.669449081803, 46.60021999999999)(41.769616026711184, 46.442139999999995)(41.869782971619365, 46.58863)(41.96994991652755, 46.20838)(42.07011686143572, 46.67102)(42.1702838063439, 46.43177)(42.27045075125209, 46.66431)(42.370617696160274, 46.679570000000005)(42.47078464106845, 47.164190000000005)(42.57095158597663, 46.636230000000005)(42.67111853088481, 46.22546)(42.77128547579299, 46.32801)(42.87145242070117, 46.09303)(42.97161936560935, 46.18824)(43.07178631051753, 46.85596)(43.17195325542571, 46.21143)(43.27212020033389, 46.332890000000006)(43.37228714524207, 46.73755)(43.47245409015025, 46.389649999999996)(43.57262103505843, 46.5166)(43.67278797996661, 46.83826)(43.77295492487479, 47.17517)(43.87312186978298, 46.45863)(43.97328881469116, 46.5044)(44.073455759599334, 46.65577)(44.173622704507515, 46.445800000000006)(44.2737896494157, 46.79737)(44.37395659432388, 46.96521)(44.47412353923205, 46.77601)(44.574290484140235, 47.29114)(44.67445742904842, 46.75342)(44.7746243739566, 47.838010000000004)(44.87479131886477, 46.44824)(44.974958263772955, 46.436029999999995)(45.075125208681136, 46.07471)(45.17529215358932, 47.07507)(45.27545909849749, 47.48767)(45.375626043405674, 46.24561)(45.475792988313856, 46.73144)(45.57595993322204, 46.47388)(45.67612687813022, 46.56482)(45.7762938230384, 46.7345)(45.87646076794658, 46.844359999999995)(45.976627712854764, 46.761970000000005)(46.07679465776294, 46.78027)(46.17696160267112, 47.499880000000005)(46.2771285475793, 46.616099999999996)(46.37729549248748, 46.304809999999996)(46.47746243739566, 46.56421)(46.57762938230384, 46.433600000000006)(46.67779632721202, 46.501340000000006)(46.7779632721202, 46.92615)(46.87813021702838, 45.93799)(46.97829716193656, 47.35523)(47.07846410684474, 46.92921)(47.17863105175292, 46.43359)(47.278797996661105, 47.79529)(47.378964941569286, 46.15711)(47.47913188647747, 46.07532)(47.57929883138564, 46.178470000000004)(47.679465776293824, 47.57678)(47.779632721202006, 46.42505)(47.87979966611019, 47.228880000000004)(47.97996661101836, 46.74487)(48.080133555926544, 47.88501)(48.180300500834726, 46.619749999999996)(48.28046744574291, 46.652100000000004)(48.38063439065108, 47.83435)(48.48080133555926, 46.681400000000004)(48.580968280467445, 46.650890000000004)(48.68113522537563, 46.14795)(48.7813021702838, 46.5813)(48.88146911519199, 46.028330000000004)(48.98163606010017, 46.70948)(49.081803005008354, 47.04822)(49.18196994991653, 46.097910000000006)(49.28213689482471, 46.84375)(49.38230383973289, 47.473029999999994)(49.48247078464107, 46.18945)(49.58263772954925, 46.72473)(49.68280467445743, 46.13819)(49.78297161936561, 46.3628)(49.88313856427379, 46.69482000000001)(49.98330550918197, 46.55811)(50.08347245409015, 45.98499)(50.18363939899833, 46.111329999999995)(50.28380634390651, 46.617309999999996)(50.38397328881469, 47.62134)(50.484140233722876, 47.15137)(50.58430717863106, 46.20227)(50.68447412353924, 46.56544)(50.784641068447414, 46.3982)(50.884808013355595, 46.85657)(50.98497495826378, 46.82422)(51.08514190317196, 46.67896)(51.18530884808013, 46.689949999999996)(51.285475792988315, 46.81629)(51.3856427378965, 46.613040000000005)(51.48580968280468, 46.73023)(51.58597662771285, 46.57947)(51.686143572621035, 47.18128)(51.786310517529216, 46.160160000000005)(51.8864774624374, 46.52393)(51.98664440734557, 47.003659999999996)(52.086811352253754, 46.41223)(52.18697829716194, 46.339600000000004)(52.287145242070125, 46.80897)(52.3873121869783, 46.246219999999994)(52.48747913188648, 46.57459)(52.58764607679466, 47.38941)(52.68781302170284, 46.241339999999994)(52.78797996661102, 46.73327999999999)(52.8881469115192, 47.08728)(52.98831385642738, 47.01404)(53.088480801335564, 46.508669999999995)(53.18864774624374, 45.99353)(53.28881469115192, 46.37683)(53.3889816360601, 47.28382)(53.489148580968276, 47.26611)(53.58931552587646, 46.899300000000004)(53.68948247078464, 47.04517)(53.78964941569283, 46.55566)(53.889816360601, 46.68993999999999)(53.989983305509185, 46.623419999999996)(54.090150250417366, 47.49866)(54.19031719532555, 47.12939)(54.29048414023372, 47.1593)(54.390651085141904, 46.47022)(54.490818030050086, 46.28895)(54.59098497495827, 48.72851)(54.69115191986644, 46.562380000000005)(54.791318864774624, 46.67407)(54.891485809682806, 46.54712)(54.99165275459099, 46.808350000000004)(55.09181969949916, 46.89381)(55.19198664440734, 46.562380000000005)(55.292153589315525, 46.89075)(55.39232053422371, 46.49707)(55.49248747913189, 46.59351)(55.59265442404007, 46.86145)(55.69282136894825, 46.32618)(55.79298831385643, 46.82239)(55.89315525876461, 47.65674)(55.99332220367279, 47.24719)(56.09348914858097, 46.41285)(56.19365609348915, 46.22118999999999)(56.29382303839733, 46.24561)(56.39398998330551, 46.99146)(56.49415692821369, 46.57336)(56.59432387312187, 47.60119)(56.69449081803005, 46.79615)(56.79465776293823, 46.797979999999995)(56.89482470784641, 46.337770000000006)(56.99499165275459, 46.824830000000006)(57.095158597662774, 46.896860000000004)(57.195325542570956, 47.037240000000004)(57.29549248747914, 47.35401)(57.39565943238732, 46.378659999999996)(57.495826377295494, 45.91541)(57.595993322203675, 47.180049999999994)(57.69616026711186, 46.43298)(57.79632721202004, 47.781859999999995)(57.89649415692821, 47.198969999999996)(57.996661101836395, 47.13245)(58.09682804674458, 47.825810000000004)(58.19699499165276, 46.445800000000006)(58.29716193656093, 47.6842)(58.397328881469114, 46.791869999999996)(58.497495826377296, 46.74549)(58.59766277128548, 46.586800000000004)(58.69782971619365, 46.948730000000005)(58.79799666110184, 47.18615)(58.89816360601002, 46.8407)(58.9983305509182, 47.23559)(59.09849749582638, 47.521240000000006)(59.19866444073456, 46.635020000000004)(59.29883138564274, 46.96948)(59.398998330550924, 47.327760000000005)(59.4991652754591, 46.61182)(59.59933222036728, 46.405519999999996)(59.69949916527546, 47.087289999999996)(59.79966611018364, 47.01709)(59.89983305509182, 46.84497)(60.0, 46.599000000000004)
        };
        \addplot[color=blue, mark=none,name path=B] coordinates { %% MIN value
        (0.0, 0.0)(0.1001669449081803, 38.05654)(0.2003338898163606, 38.51736)(0.3005008347245409, 37.61954)(0.4006677796327212, 37.912499999999994)(0.5008347245409015, 37.1984)(0.6010016694490818, 38.78652)(0.7011686143572621, 38.32204)(0.8013355592654424, 38.01259)(0.9015025041736228, 38.78103)(1.001669449081803, 37.92898)(1.1018363939899833, 37.67385)(1.2020033388981637, 38.53262)(1.3021702838063438, 38.71816)(1.4023372287145242, 37.76541)(1.5025041736227045, 38.48256)(1.6026711185308848, 38.79202)(1.7028380634390652, 38.54665)(1.8030050083472455, 38.68765)(1.9031719532554257, 38.775529999999996)(2.003338898163606, 38.97634)(2.1035058430717863, 37.77151)(2.2036727879799667, 38.51919)(2.303839732888147, 38.91713)(2.4040066777963274, 38.68704)(2.5041736227045073, 38.792010000000005)(2.6043405676126876, 38.591210000000004)(2.704507512520868, 38.13406)(2.8046744574290483, 38.73586)(2.9048414023372287, 38.54543)(3.005008347245409, 38.492329999999995)(3.1051752921535893, 38.156639999999996)(3.2053422370617697, 38.46976)(3.30550918196995, 38.023579999999995)(3.4056761268781304, 38.43191)(3.5058430717863107, 38.26651)(3.606010016694491, 38.363550000000004)(3.7061769616026714, 38.1896)(3.8063439065108513, 38.384299999999996)(3.906510851419032, 38.54787)(4.006677796327212, 37.730000000000004)(4.106844741235393, 38.331810000000004)(4.207011686143573, 38.42947)(4.3071786310517535, 38.29336)(4.407345575959933, 38.21829)(4.507512520868114, 38.499660000000006)(4.607679465776294, 38.66933)(4.707846410684475, 38.323260000000005)(4.808013355592655, 38.2543)(4.908180300500835, 38.1835)(5.0083472454090145, 38.78958)(5.108514190317195, 38.201809999999995)(5.208681135225375, 38.63271)(5.308848080133556, 38.49355)(5.409015025041736, 37.98452)(5.509181969949917, 38.62295)(5.609348914858097, 38.5961)(5.709515859766277, 38.366600000000005)(5.809682804674457, 38.43679)(5.909849749582638, 38.3013)(6.010016694490818, 38.92324)(6.110183639398999, 37.81424)(6.210350584307179, 38.22622)(6.3105175292153595, 38.12551)(6.410684474123539, 38.68154)(6.510851419031719, 38.19509)(6.6110183639399, 38.450829999999996)(6.71118530884808, 38.365990000000004)(6.811352253756261, 38.59426)(6.911519198664441, 38.515519999999995)(7.011686143572621, 38.40627)(7.111853088480801, 38.30557)(7.212020033388982, 38.84511)(7.312186978297162, 38.42825)(7.412353923205343, 37.48769)(7.512520868113523, 37.98574)(7.612687813021703, 38.272)(7.712854757929884, 37.786770000000004)(7.813021702838064, 38.01504)(7.913188647746244, 38.55825)(8.013355592654424, 37.97414)(8.113522537562606, 38.22073)(8.213689482470786, 38.53445)(8.313856427378965, 37.60732)(8.414023372287145, 38.09316)(8.514190317195327, 38.90493)(8.614357262103507, 38.55276)(8.714524207011687, 38.50454)(8.814691151919867, 38.51613)(8.914858096828048, 38.861599999999996)(9.015025041736228, 38.62722)(9.115191986644408, 38.02236)(9.215358931552588, 38.491119999999995)(9.31552587646077, 37.428490000000004)(9.41569282136895, 38.78042)(9.51585976627713, 37.18802)(9.61602671118531, 38.11941)(9.71619365609349, 38.37026)(9.81636060100167, 38.47524)(9.916527545909851, 37.534079999999996)(10.016694490818029, 38.300070000000005)(10.11686143572621, 37.9009)(10.21702838063439, 37.46389)(10.31719532554257, 38.37698)(10.41736227045075, 38.58022)(10.51752921535893, 38.30556)(10.617696160267112, 38.336079999999995)(10.717863105175292, 38.85183)(10.818030050083472, 37.32839)(10.918196994991652, 37.55301)(11.018363939899833, 38.52896)(11.118530884808013, 38.25857)(11.218697829716193, 38.36782)(11.318864774624373, 37.92837)(11.419031719532555, 38.19449)(11.519198664440735, 38.48867)(11.619365609348915, 38.29275)(11.719532554257095, 38.37698)(11.819699499165276, 38.05044)(11.919866444073456, 38.2012)(12.020033388981636, 38.14627)(12.120200333889816, 38.353790000000004)(12.220367278797998, 38.39712)(12.320534223706177, 38.15969)(12.420701168614357, 37.720240000000004)(12.520868113522537, 38.651019999999995)(12.621035058430719, 38.66567)(12.721202003338899, 38.44473)(12.821368948247079, 37.67263)(12.921535893155259, 38.21523)(13.021702838063439, 38.49355)(13.12186978297162, 38.18838)(13.2220367278798, 38.12307)(13.32220367278798, 38.70596)(13.42237061769616, 38.14504)(13.522537562604342, 38.74197)(13.622704507512521, 38.589380000000006)(13.722871452420701, 38.278710000000004)(13.823038397328881, 38.31777)(13.923205342237063, 38.50271)(14.023372287145243, 38.51614)(14.123539232053423, 38.10232)(14.223706176961603, 38.456320000000005)(14.323873121869784, 38.09439)(14.424040066777964, 38.638819999999996)(14.524207011686144, 38.721830000000004)(14.624373956594324, 38.08768)(14.724540901502506, 38.644310000000004)(14.824707846410686, 38.53139)(14.924874791318866, 37.777)(15.025041736227045, 38.26284)(15.125208681135225, 38.41299)(15.225375626043405, 38.342800000000004)(15.325542570951589, 38.22256)(15.425709515859769, 38.50088)(15.525876460767948, 38.535669999999996)(15.626043405676128, 37.99062)(15.726210350584308, 38.14993)(15.826377295492488, 38.09499)(15.926544240400668, 37.41995)(16.026711185308848, 38.06691)(16.126878130217026, 38.53506)(16.22704507512521, 38.29824)(16.32721202003339, 38.38064)(16.42737896494157, 38.52895)(16.52754590984975, 38.4722)(16.62771285475793, 38.23599)(16.72787979966611, 38.46181)(16.82804674457429, 38.47463)(16.92821368948247, 38.54116)(17.028380634390654, 38.71877)(17.128547579298832, 38.05959)(17.228714524207014, 38.52041)(17.328881469115192, 38.02725)(17.429048414023374, 38.75966)(17.529215358931552, 38.253080000000004)(17.629382303839733, 38.45205)(17.72954924874791, 38.58571)(17.829716193656097, 38.44412)(17.929883138564275, 38.73586)(18.030050083472457, 38.4545)(18.130217028380635, 38.179829999999995)(18.230383973288816, 38.25796)(18.330550918196995, 37.35159)(18.430717863105176, 38.35378)(18.530884808013354, 38.50454)(18.63105175292154, 38.30495)(18.731218697829718, 38.479510000000005)(18.8313856427379, 38.68582)(18.931552587646078, 37.839870000000005)(19.03171953255426, 38.2427)(19.131886477462437, 38.47402)(19.23205342237062, 38.47524)(19.332220367278797, 38.90737)(19.43238731218698, 38.148089999999996)(19.53255425709516, 38.80117)(19.63272120200334, 38.77614)(19.73288814691152, 38.133449999999996)(19.833055091819702, 38.45144)(19.93322203672788, 38.848169999999996)(20.033388981636058, 38.50637)(20.13355592654424, 38.14199)(20.23372287145242, 38.62967)(20.333889816360603, 38.59182)(20.43405676126878, 38.87624)(20.534223706176963, 38.50699)(20.63439065108514, 38.52956)(20.734557595993323, 38.54543)(20.8347245409015, 38.589380000000006)(20.934891485809683, 38.16945)(21.03505843071786, 37.96255)(21.135225375626046, 38.56008)(21.235392320534224, 38.32144)(21.335559265442406, 38.32327)(21.435726210350584, 38.47525)(21.535893155258766, 37.54079)(21.636060100166944, 38.491119999999995)(21.736227045075125, 38.351350000000004)(21.836393989983303, 38.52713)(21.93656093489149, 38.50027)(22.036727879799667, 38.57961)(22.13689482470785, 38.24087)(22.237061769616027, 38.15054)(22.33722871452421, 38.09682)(22.437395659432386, 38.73769)(22.537562604340568, 38.33059)(22.637729549248746, 38.54849)(22.737896494156928, 39.04592)(22.83806343906511, 38.02847)(22.93823038397329, 38.66933)(23.03839732888147, 38.430080000000004)(23.13856427378965, 38.65346)(23.23873121869783, 38.72122)(23.33889816360601, 37.97292)(23.43906510851419, 38.87014)(23.53923205342237, 38.19753)(23.639398998330552, 38.29091)(23.739565943238734, 38.49782)(23.839732888146912, 38.36111)(23.939899833055094, 37.63724)(24.040066777963272, 38.75783)(24.140233722871454, 38.41237)(24.24040066777963, 38.22988)(24.340567612687813, 38.58083)(24.440734557595995, 38.35439)(24.540901502504177, 38.35562)(24.641068447412355, 38.75967)(24.741235392320537, 38.45449)(24.841402337228715, 38.496)(24.941569282136896, 38.26956)(25.041736227045075, 38.37332)(25.141903171953256, 38.340360000000004)(25.242070116861438, 38.17373)(25.34223706176962, 38.196310000000004)(25.442404006677798, 38.725480000000005)(25.54257095158598, 37.89175)(25.642737896494157, 38.17678)(25.74290484140234, 38.83657)(25.843071786310517, 38.13161)(25.9432387312187, 38.49416)(26.043405676126877, 38.2134)(26.143572621035062, 38.57717)(26.24373956594324, 38.45205)(26.34390651085142, 37.96194)(26.4440734557596, 38.2549)(26.544240400667782, 38.96902)(26.64440734557596, 38.75356)(26.744574290484138, 38.944610000000004)(26.84474123539232, 38.33425)(26.9449081803005, 38.62417)(27.045075125208683, 38.34585)(27.14524207011686, 39.0154)(27.245409015025043, 38.198750000000004)(27.34557595993322, 38.135279999999995)(27.445742904841403, 38.66079)(27.54590984974958, 38.15848)(27.646076794657763, 37.58291)(27.746243739565944, 38.849999999999994)(27.846410684474126, 38.48501)(27.946577629382304, 38.31045)(28.046744574290486, 38.49783)(28.146911519198664, 38.48561)(28.247078464106846, 38.175560000000004)(28.347245409015024, 38.33975)(28.447412353923205, 38.715109999999996)(28.547579298831387, 38.78958)(28.64774624373957, 38.63271)(28.747913188647747, 38.92202)(28.84808013355593, 38.80179)(28.948247078464107, 38.58267)(29.04841402337229, 38.93849)(29.148580968280466, 38.665060000000004)(29.248747913188648, 38.366600000000005)(29.348914858096826, 38.68581)(29.44908180300501, 38.96658)(29.54924874791319, 38.3898)(29.64941569282137, 38.41421)(29.74958263772955, 38.381859999999996)(29.84974958263773, 38.20791)(29.94991652754591, 38.120020000000004)(30.05008347245409, 38.23415)(30.15025041736227, 38.99405)(30.25041736227045, 38.69619)(30.35058430717863, 37.98758)(30.45075125208681, 38.44473)(30.55091819699499, 38.45877)(30.651085141903177, 38.11819)(30.751252086811355, 38.71267)(30.851419031719537, 38.47585)(30.951585976627715, 38.38369)(31.051752921535897, 38.21768)(31.151919866444075, 38.32204)(31.252086811352257, 38.5729)(31.352253756260435, 38.12612)(31.452420701168617, 38.82742)(31.552587646076795, 38.91897)(31.652754590984976, 38.867090000000005)(31.752921535893154, 38.46792)(31.853088480801336, 38.459379999999996)(31.953255425709514, 38.49356)(32.053422370617696, 38.402)(32.15358931552588, 38.76577)(32.25375626043405, 38.17922)(32.35392320534224, 38.50515)(32.45409015025042, 38.62356)(32.554257095158604, 38.76821)(32.65442404006678, 38.464870000000005)(32.75459098497496, 38.43191)(32.85475792988314, 38.81277)(32.95492487479132, 38.29885)(33.0550918196995, 38.23904)(33.15525876460768, 38.24148)(33.25542570951586, 38.8439)(33.35559265442404, 38.09926)(33.45575959933222, 38.24698)(33.5559265442404, 38.35989)(33.65609348914858, 38.667500000000004)(33.756260434056756, 38.65041)(33.85642737896494, 38.23538)(33.95659432387313, 38.535669999999996)(34.05676126878131, 38.62966)(34.15692821368948, 39.13809)(34.257095158597664, 38.26468)(34.357262103505846, 38.44534)(34.45742904841403, 38.15114)(34.5575959933222, 38.29825)(34.657762938230384, 38.88723)(34.757929883138566, 38.725480000000005)(34.85809682804675, 38.6907)(34.95826377295492, 38.723040000000005)(35.058430717863104, 38.86526)(35.158597662771285, 38.963519999999995)(35.25876460767947, 38.42031)(35.35893155258764, 38.57045)(35.45909849749582, 38.069359999999996)(35.559265442404005, 38.36782)(35.659432387312194, 38.496)(35.75959933222037, 38.16702)(35.85976627712855, 38.19509)(35.95993322203673, 38.82803)(36.06010016694491, 39.01052)(36.16026711185309, 38.25186)(36.26043405676127, 38.38674)(36.36060100166945, 38.18838)(36.46076794657763, 38.73891)(36.56093489148581, 38.97634)(36.66110183639399, 38.60769)(36.76126878130217, 37.98147)(36.86143572621035, 38.16824)(36.96160267111853, 38.95803)(37.06176961602671, 38.31838)(37.16193656093489, 37.6354)(37.26210350584308, 38.68764)(37.362270450751254, 38.74014)(37.462437395659435, 38.30557)(37.56260434056762, 38.36904)(37.6627712854758, 38.99282)(37.76293823038397, 38.83169)(37.863105175292155, 38.66262)(37.96327212020034, 38.21645)(38.06343906510852, 38.673)(38.16360601001669, 39.32547)(38.263772954924875, 38.40139)(38.363939899833056, 39.06606)(38.46410684474124, 38.849999999999994)(38.56427378964941, 38.430080000000004)(38.664440734557594, 38.768820000000005)(38.764607679465776, 38.44534)(38.86477462437396, 38.23843)(38.96494156928214, 38.99404)(39.06510851419032, 38.07546)(39.1652754590985, 37.96072)(39.26544240400668, 39.014790000000005)(39.36560934891486, 38.87624)(39.46577629382304, 38.7853)(39.56594323873122, 38.85366)(39.666110183639404, 38.35623)(39.76627712854758, 38.36233)(39.86644407345576, 38.848169999999996)(39.96661101836394, 38.58023)(40.066777963272116, 38.48928)(40.1669449081803, 38.32143)(40.26711185308848, 38.87624)(40.36727879799666, 38.24148)(40.46744574290484, 38.67544)(40.567612687813025, 38.31655)(40.667779632721206, 38.83047)(40.76794657762939, 38.70107)(40.86811352253756, 38.78347)(40.968280467445744, 38.70596)(41.068447412353926, 37.82706)(41.16861435726211, 38.86282)(41.26878130217028, 38.70534)(41.368948247078464, 38.50393)(41.469115191986646, 38.294579999999996)(41.56928213689483, 38.291529999999995)(41.669449081803, 38.990990000000004)(41.769616026711184, 38.43618)(41.869782971619365, 38.861599999999996)(41.96994991652755, 39.057520000000004)(42.07011686143572, 39.03311)(42.1702838063439, 38.5967)(42.27045075125209, 39.27663)(42.370617696160274, 38.19814)(42.47078464106845, 38.28725)(42.57095158597663, 38.74197)(42.67111853088481, 38.8622)(42.77128547579299, 38.73465)(42.87145242070117, 38.88417)(42.97161936560935, 38.33182)(43.07178631051753, 38.89516)(43.17195325542571, 38.803619999999995)(43.27212020033389, 39.09352)(43.37228714524207, 38.53445)(43.47245409015025, 38.7267)(43.57262103505843, 38.2189)(43.67278797996661, 38.55825)(43.77295492487479, 38.39102)(43.87312186978298, 38.873799999999996)(43.97328881469116, 39.13442)(44.073455759599334, 37.60611)(44.173622704507515, 38.45998)(44.2737896494157, 38.07607)(44.37395659432388, 38.57595)(44.47412353923205, 38.39102)(44.574290484140235, 37.07388)(44.67445742904842, 38.56008)(44.7746243739566, 37.299099999999996)(44.87479131886477, 38.34096)(44.974958263772955, 38.43374)(45.075125208681136, 38.5497)(45.17529215358932, 38.82741)(45.27545909849749, 38.64736)(45.375626043405674, 38.62844)(45.475792988313856, 38.58999)(45.57595993322204, 38.70779)(45.67612687813022, 39.27236)(45.7762938230384, 38.7853)(45.87646076794658, 38.40017)(45.976627712854764, 38.45327)(46.07679465776294, 37.38577)(46.17696160267112, 38.52957)(46.2771285475793, 38.86892)(46.37729549248748, 38.402)(46.47746243739566, 38.77249)(46.57762938230384, 37.51455)(46.67779632721202, 38.62844)(46.7779632721202, 38.88113)(46.87813021702838, 38.40811)(46.97829716193656, 38.9031)(47.07846410684474, 38.35866)(47.17863105175292, 38.56069)(47.278797996661105, 38.47708)(47.378964941569286, 38.65285)(47.47913188647747, 38.79812)(47.57929883138564, 38.81459)(47.679465776293824, 39.19607)(47.779632721202006, 38.50882)(47.87979966611019, 38.368430000000004)(47.97996661101836, 38.42641)(48.080133555926544, 38.63393)(48.180300500834726, 38.44351)(48.28046744574291, 38.676050000000004)(48.38063439065108, 38.7792)(48.48080133555926, 38.49844)(48.580968280467445, 38.09133)(48.68113522537563, 38.36416)(48.7813021702838, 38.98061)(48.88146911519199, 38.32387)(48.98163606010017, 38.5314)(49.081803005008354, 38.86953)(49.18196994991653, 38.3727)(49.28213689482471, 38.78652)(49.38230383973289, 38.809110000000004)(49.48247078464107, 38.4435)(49.58263772954925, 39.217420000000004)(49.68280467445743, 38.86221)(49.78297161936561, 38.588770000000004)(49.88313856427379, 38.58755)(49.98330550918197, 38.58999)(50.08347245409015, 38.77676)(50.18363939899833, 38.27505)(50.28380634390651, 38.64309)(50.38397328881469, 38.754780000000004)(50.484140233722876, 38.75966)(50.58430717863106, 38.39589)(50.68447412353924, 38.81094)(50.784641068447414, 38.26712)(50.884808013355595, 38.61014)(50.98497495826378, 38.3019)(51.08514190317196, 38.535669999999996)(51.18530884808013, 38.8793)(51.285475792988315, 38.55276)(51.3856427378965, 38.82314)(51.48580968280468, 38.94094)(51.58597662771285, 38.90554)(51.686143572621035, 38.42825)(51.786310517529216, 39.05934)(51.8864774624374, 38.56069)(51.98664440734557, 39.22048)(52.086811352253754, 39.05569)(52.18697829716194, 39.24856)(52.287145242070125, 38.79995)(52.3873121869783, 38.803619999999995)(52.48747913188648, 38.45144)(52.58764607679466, 38.40932)(52.68781302170284, 38.38064)(52.78797996661102, 38.93056)(52.8881469115192, 38.933)(52.98831385642738, 38.84328)(53.088480801335564, 38.89761)(53.18864774624374, 38.72976)(53.28881469115192, 38.39346)(53.3889816360601, 38.89577)(53.489148580968276, 38.63149)(53.58931552587646, 38.74562)(53.68948247078464, 38.535669999999996)(53.78964941569283, 39.11916)(53.889816360601, 38.76455)(53.989983305509185, 38.80788)(54.090150250417366, 39.10269)(54.19031719532555, 38.81521)(54.29048414023372, 38.58754)(54.390651085141904, 38.848169999999996)(54.490818030050086, 38.425799999999995)(54.59098497495827, 38.06937)(54.69115191986644, 38.5674)(54.791318864774624, 38.25002)(54.891485809682806, 38.84084)(54.99165275459099, 38.6492)(55.09181969949916, 38.43557)(55.19198664440734, 39.04165)(55.292153589315525, 38.55398)(55.39232053422371, 39.12099)(55.49248747913189, 38.36965)(55.59265442404007, 39.057520000000004)(55.69282136894825, 38.80239)(55.79298831385643, 38.72609)(55.89315525876461, 38.30495)(55.99332220367279, 38.306779999999996)(56.09348914858097, 38.86404)(56.19365609348915, 38.50515)(56.29382303839733, 38.99831)(56.39398998330551, 38.65041)(56.49415692821369, 38.4136)(56.59432387312187, 38.24331)(56.69449081803005, 38.53262)(56.79465776293823, 38.62173)(56.89482470784641, 39.057520000000004)(56.99499165275459, 38.64492)(57.095158597662774, 38.97939)(57.195325542570956, 39.18569)(57.29549248747914, 38.67239)(57.39565943238732, 38.88174)(57.495826377295494, 38.73769)(57.595993322203675, 39.21926)(57.69616026711186, 38.95925)(57.79632721202004, 38.26712)(57.89649415692821, 38.96291)(57.996661101836395, 38.55093)(58.09682804674458, 38.60769)(58.19699499165276, 39.214380000000006)(58.29716193656093, 38.338519999999995)(58.397328881469114, 39.296769999999995)(58.497495826377296, 39.1393)(58.59766277128548, 38.61318)(58.69782971619365, 39.20217)(58.79799666110184, 39.2156)(58.89816360601002, 38.97817)(58.9983305509182, 39.32302)(59.09849749582638, 38.70107)(59.19866444073456, 38.86648)(59.29883138564274, 38.6437)(59.398998330550924, 38.90982)(59.4991652754591, 39.1039)(59.59933222036728, 38.602199999999996)(59.69949916527546, 38.93667)(59.79966611018364, 38.56497)(59.89983305509182, 38.67116)(60.0, 38.58755)
        };
        \addplot [pattern=north east lines,pattern color=red] 
        fill between [
            of=A and B,soft clip={domain=0:800},
        ];
        \end{axis}
\end{tikzpicture}
\caption{Test case: Fasta}
\end{subfigure}
\begin{subfigure}[b]{0.49\linewidth}
    \begin{tikzpicture}
        \pgfplotsset{%
        width=1\linewidth,
        % height=1\textheight
            }
        \begin{axis}[ymax=120,
        xlabel={Time (Seconds)},
        ylabel={Energy Consumption (Joules)},
        ]
        \addplot[color=blue, mark=none,] coordinates { %% AVG value
        (0.0, 0.0)(0.1001669449081803, 31.36127358333334)(0.2003338898163606, 33.578369333333335)(0.3005008347245409, 31.87076291666666)(0.4006677796327212, 32.59785291666665)(0.5008347245409015, 31.385452416666656)(0.6010016694490818, 31.572286916666673)(0.7011686143572621, 31.28779774999999)(0.8013355592654424, 31.50951208333333)(0.9015025041736228, 32.09380008333333)(1.001669449081803, 31.38264075)(1.1018363939899833, 31.275004916666656)(1.2020033388981637, 31.432190916666666)(1.3021702838063438, 31.456599500000006)(1.4023372287145242, 31.423839500000007)(1.5025041736227045, 31.487610916666654)(1.6026711185308848, 31.452957500000014)(1.7028380634390652, 31.435380166666672)(1.8030050083472455, 31.426768833333327)(1.9031719532554257, 31.52563049999999)(2.003338898163606, 31.592941416666676)(2.1035058430717863, 31.43728766666666)(2.2036727879799667, 31.5366725)(2.303839732888147, 31.51636858333333)(2.4040066777963274, 31.92403008333333)(2.5041736227045073, 31.20535041666667)(2.6043405676126876, 31.534693833333332)(2.704507512520868, 31.46809983333332)(2.8046744574290483, 31.46593249999999)(2.9048414023372287, 31.533356666666673)(3.005008347245409, 31.50684750000001)(3.1051752921535893, 31.421366583333345)(3.2053422370617697, 31.534272166666653)(3.30550918196995, 31.543457999999998)(3.4056761268781304, 31.440475333333335)(3.5058430717863107, 31.51595708333334)(3.606010016694491, 31.482829499999998)(3.7061769616026714, 31.46643133333334)(3.8063439065108513, 31.547343666666656)(3.906510851419032, 31.550258166666666)(4.006677796327212, 31.486853083333347)(4.106844741235393, 31.50844324999998)(4.207011686143573, 31.491959416666692)(4.3071786310517535, 31.658467416666667)(4.407345575959933, 31.550385749999993)(4.507512520868114, 31.550283250000014)(4.607679465776294, 31.486629250000007)(4.707846410684475, 31.585536166666675)(4.808013355592655, 31.46619233333333)(4.908180300500835, 31.605454)(5.0083472454090145, 31.71438583333334)(5.108514190317195, 31.59557183333334)(5.208681135225375, 31.538981083333326)(5.308848080133556, 31.58186433333334)(5.409015025041736, 31.522645083333327)(5.509181969949917, 31.522567916666667)(5.609348914858097, 31.580343083333343)(5.709515859766277, 31.60047491666668)(5.809682804674457, 31.43634591666666)(5.909849749582638, 31.72507316666667)(6.010016694490818, 31.55253666666666)(6.110183639398999, 31.47032183333332)(6.210350584307179, 31.612915500000003)(6.3105175292153595, 31.554601583333334)(6.410684474123539, 31.488124666666646)(6.510851419031719, 31.582458999999997)(6.6110183639399, 31.596939666666664)(6.71118530884808, 31.52966925000001)(6.811352253756261, 31.57105000000001)(6.911519198664441, 31.592173833333323)(7.011686143572621, 31.542252166666653)(7.111853088480801, 31.588760333333337)(7.212020033388982, 31.583334250000004)(7.312186978297162, 31.72009775000001)(7.412353923205343, 31.52160275)(7.512520868113523, 31.614863500000002)(7.612687813021703, 31.558471666666673)(7.712854757929884, 31.622310000000017)(7.813021702838064, 31.522756666666666)(7.913188647746244, 31.59945225000001)(8.013355592654424, 31.63217200000001)(8.113522537562606, 31.56983466666666)(8.213689482470786, 31.585714416666665)(8.313856427378965, 31.606227250000014)(8.414023372287145, 31.52997349999999)(8.514190317195327, 31.570018166666678)(8.614357262103507, 31.74329641666666)(8.714524207011687, 31.581080750000005)(8.814691151919867, 31.56304475000001)(8.914858096828048, 31.625428166666662)(9.015025041736228, 31.62208083333334)(9.115191986644408, 31.56244433333334)(9.215358931552588, 31.66485175000002)(9.31552587646077, 31.614613749999997)(9.41569282136895, 31.64160716666667)(9.51585976627713, 31.533737916666656)(9.61602671118531, 31.609141500000003)(9.71619365609349, 31.657486499999997)(9.81636060100167, 31.822540166666666)(9.916527545909851, 31.739731)(10.016694490818029, 31.651773999999996)(10.11686143572621, 31.62884050000001)(10.21702838063439, 31.74779741666666)(10.31719532554257, 31.720153999999987)(10.41736227045075, 31.793665666666662)(10.51752921535893, 31.63355066666667)(10.617696160267112, 31.639521833333337)(10.717863105175292, 31.660502166666667)(10.818030050083472, 31.597702416666657)(10.918196994991652, 31.68413366666666)(11.018363939899833, 31.640248666666665)(11.118530884808013, 31.581762833333325)(11.218697829716193, 31.61011816666667)(11.318864774624373, 31.610870166666675)(11.419031719532555, 31.625102083333328)(11.519198664440735, 31.598343666666672)(11.619365609348915, 31.660293916666678)(11.719532554257095, 31.677739666666657)(11.819699499165276, 31.558065166666676)(11.919866444073456, 31.724254250000005)(12.020033388981636, 31.600849666666655)(12.120200333889816, 31.582607166666666)(12.220367278797998, 31.65936850000001)(12.320534223706177, 31.63166799999999)(12.420701168614357, 31.606618416666674)(12.520868113522537, 31.644471000000017)(12.621035058430719, 31.63164275)(12.721202003338899, 31.64993408333333)(12.821368948247079, 31.618881083333324)(12.921535893155259, 31.81032341666667)(13.021702838063439, 31.593795999999994)(13.12186978297162, 31.723470250000002)(13.2220367278798, 31.545232749999986)(13.32220367278798, 31.62289458333334)(13.42237061769616, 31.69077075)(13.522537562604342, 31.58060283333333)(13.622704507512521, 31.629074250000013)(13.722871452420701, 31.652466083333323)(13.823038397328881, 31.743362583333344)(13.923205342237063, 31.655960500000003)(14.023372287145243, 31.658828749999987)(14.123539232053423, 31.738932583333344)(14.223706176961603, 31.55237916666667)(14.323873121869784, 31.684158166666656)(14.424040066777964, 31.637955666666667)(14.524207011686144, 31.637985583333325)(14.624373956594324, 31.739948916666673)(14.724540901502506, 31.75076358333335)(14.824707846410686, 31.639592583333332)(14.924874791318866, 31.749033500000014)(15.025041736227045, 31.70512008333333)(15.125208681135225, 31.738586250000008)(15.225375626043405, 31.657969083333334)(15.325542570951589, 31.74616558333333)(15.425709515859769, 31.624852750000006)(15.525876460767948, 31.671107166666662)(15.626043405676128, 31.759720249999997)(15.726210350584308, 31.655044666666665)(15.826377295492488, 31.707193916666657)(15.926544240400668, 31.730225499999996)(16.026711185308848, 31.654098083333338)(16.126878130217026, 31.66571075000001)(16.22704507512521, 31.699036000000003)(16.32721202003339, 31.686590000000002)(16.42737896494157, 31.76074749999999)(16.52754590984975, 31.659958)(16.62771285475793, 31.70455433333333)(16.72787979966611, 31.694458166666674)(16.82804674457429, 31.629145416666656)(16.92821368948247, 31.660269416666676)(17.028380634390654, 31.73032616666667)(17.128547579298832, 31.730443)(17.228714524207014, 31.674525583333324)(17.328881469115192, 31.73765591666665)(17.429048414023374, 31.623312166666675)(17.529215358931552, 31.672124333333333)(17.629382303839733, 31.68908641666667)(17.72954924874791, 31.760803000000003)(17.829716193656097, 31.683029916666676)(17.929883138564275, 31.690287500000004)(18.030050083472457, 31.619457250000018)(18.130217028380635, 31.714593999999998)(18.230383973288816, 31.797750166666667)(18.330550918196995, 31.72278841666666)(18.430717863105176, 31.674672333333337)(18.530884808013354, 31.705236666666675)(18.63105175292154, 31.705108749999987)(18.731218697829718, 31.69874091666666)(18.8313856427379, 31.780054250000003)(18.931552587646078, 31.702632250000015)(19.03171953255426, 31.7514545)(19.131886477462437, 31.933145916666653)(19.23205342237062, 31.68818199999999)(19.332220367278797, 31.833811)(19.43238731218698, 31.775767416666657)(19.53255425709516, 31.77686033333335)(19.63272120200334, 31.71494541666667)(19.73288814691152, 31.76508125)(19.833055091819702, 31.706858749999977)(19.93322203672788, 31.74451683333332)(20.033388981636058, 31.836913916666667)(20.13355592654424, 31.755594583333334)(20.23372287145242, 31.712733583333343)(20.333889816360603, 31.87389558333333)(20.43405676126878, 31.86365283333334)(20.534223706176963, 31.876505333333327)(20.63439065108514, 31.800888000000004)(20.734557595993323, 31.75318383333332)(20.8347245409015, 31.883310833333343)(20.934891485809683, 31.77230849999999)(21.03505843071786, 31.861225916666672)(21.135225375626046, 31.757263916666655)(21.235392320534224, 31.844380333333334)(21.335559265442406, 31.75929266666666)(21.435726210350584, 31.747594249999985)(21.535893155258766, 31.756429416666673)(21.636060100166944, 31.752507166666657)(21.736227045075125, 31.787881999999993)(21.836393989983303, 31.791382583333323)(21.93656093489149, 31.809605416666656)(22.036727879799667, 31.71057191666666)(22.13689482470785, 31.81742316666665)(22.237061769616027, 31.74443049999999)(22.33722871452421, 31.877787749999996)(22.437395659432386, 31.775181750000005)(22.537562604340568, 31.950978333333328)(22.637729549248746, 31.69439683333334)(22.737896494156928, 31.831461916666658)(22.83806343906511, 31.70299766666666)(22.93823038397329, 31.77269491666668)(23.03839732888147, 31.76212050000002)(23.13856427378965, 31.84607958333332)(23.23873121869783, 31.72411091666667)(23.33889816360601, 31.831009333333323)(23.43906510851419, 31.777023333333325)(23.53923205342237, 31.825271583333336)(23.639398998330552, 31.76985683333333)(23.739565943238734, 31.985661333333336)(23.839732888146912, 31.756520249999998)(23.939899833055094, 31.85625133333333)(24.040066777963272, 31.81558316666666)(24.140233722871454, 31.805048750000008)(24.24040066777963, 31.799366416666665)(24.340567612687813, 31.851623750000012)(24.440734557595995, 31.797999666666676)(24.540901502504177, 31.783787083333326)(24.641068447412355, 31.814793500000007)(24.741235392320537, 31.801122249999995)(24.841402337228715, 31.810231916666666)(24.941569282136896, 31.904199916666663)(25.041736227045075, 31.944676249999972)(25.141903171953256, 31.81085191666666)(25.242070116861438, 31.829442749999995)(25.34223706176962, 31.828210583333327)(25.442404006677798, 31.84890274999999)(25.54257095158598, 31.79972308333334)(25.642737896494157, 31.840072999999986)(25.74290484140234, 31.789158250000014)(25.843071786310517, 31.860835250000004)(25.9432387312187, 31.79426566666667)(26.043405676126877, 31.895710583333326)(26.143572621035062, 31.946950249999993)(26.24373956594324, 32.02170708333333)(26.34390651085142, 32.016484250000005)(26.4440734557596, 31.93241341666666)(26.544240400667782, 31.976613166666656)(26.64440734557596, 32.04021108333333)(26.744574290484138, 31.84677575)(26.84474123539232, 31.853861916666663)(26.9449081803005, 31.94823141666667)(27.045075125208683, 32.08431408333333)(27.14524207011686, 32.065179916666665)(27.245409015025043, 31.952432583333334)(27.34557595993322, 31.83199075000001)(27.445742904841403, 31.86195391666667)(27.54590984974958, 31.962813583333336)(27.646076794657763, 31.850193833333343)(27.746243739565944, 31.831324666666664)(27.846410684474126, 31.891123083333316)(27.946577629382304, 31.951629333333337)(28.046744574290486, 32.020431)(28.146911519198664, 32.10620999999999)(28.247078464106846, 32.118555083333334)(28.347245409015024, 32.14027341666667)(28.447412353923205, 32.099404750000005)(28.547579298831387, 32.177107916666664)(28.64774624373957, 32.165465416666656)(28.747913188647747, 32.17270800000002)(28.84808013355593, 32.07073375000002)(28.948247078464107, 32.17228666666667)(29.04841402337229, 31.943404083333323)(29.148580968280466, 31.922068333333335)(29.248747913188648, 31.821654499999998)(29.348914858096826, 31.96099383333335)(29.44908180300501, 31.80846633333333)(29.54924874791319, 31.87897683333334)(29.64941569282137, 31.869822000000006)(29.74958263772955, 32.005437249999986)(29.84974958263773, 31.920287249999994)(29.94991652754591, 31.959671)(30.05008347245409, 32.06125783333333)(30.15025041736227, 32.056497666666665)(30.25041736227045, 31.885010083333324)(30.35058430717863, 31.979765916666672)(30.45075125208681, 31.916213333333335)(30.55091819699499, 31.83791150000002)(30.651085141903177, 31.881332000000008)(30.751252086811355, 31.98554908333332)(30.851419031719537, 31.917190333333338)(30.951585976627715, 32.00036550000001)(31.051752921535897, 31.84278858333334)(31.151919866444075, 31.913263750000002)(31.252086811352257, 31.926752750000013)(31.352253756260435, 31.857105999999998)(31.452420701168617, 31.92643674999999)(31.552587646076795, 32.01795458333332)(31.652754590984976, 31.864007916666655)(31.752921535893154, 31.89967325)(31.853088480801336, 31.944849)(31.953255425709514, 32.050791000000004)(32.053422370617696, 31.884170250000007)(32.15358931552588, 31.99343816666666)(32.25375626043405, 31.982487749999997)(32.35392320534224, 31.89413933333333)(32.45409015025042, 31.86324025000001)(32.554257095158604, 31.961674666666674)(32.65442404006678, 31.941182166666653)(32.75459098497496, 31.908370583333337)(32.85475792988314, 31.959410750000018)(32.95492487479132, 31.990874833333326)(33.0550918196995, 31.99410966666668)(33.15525876460768, 31.899016583333342)(33.25542570951586, 31.988794833333333)(33.35559265442404, 31.91861474999998)(33.45575959933222, 31.961506666666665)(33.5559265442404, 31.948831666666663)(33.65609348914858, 31.922942583333324)(33.756260434056756, 31.898122083333337)(33.85642737896494, 31.849761833333336)(33.95659432387313, 32.09703474999999)(34.05676126878131, 31.948938666666674)(34.15692821368948, 31.93278974999999)(34.257095158597664, 31.92172725)(34.357262103505846, 31.933359499999984)(34.45742904841403, 31.919016)(34.5575959933222, 31.967696666666676)(34.657762938230384, 31.91731216666666)(34.757929883138566, 31.924529749999994)(34.85809682804675, 32.0786785)(34.95826377295492, 31.955983583333325)(35.058430717863104, 31.96157274999999)(35.158597662771285, 32.04576491666666)(35.25876460767947, 31.930323083333327)(35.35893155258764, 31.919896249999997)(35.45909849749582, 32.014948583333336)(35.559265442404005, 31.906798583333334)(35.659432387312194, 32.03249058333333)(35.75959933222037, 32.06730050000001)(35.85976627712855, 32.15854291666667)(35.95993322203673, 31.946878666666677)(36.06010016694491, 31.94445741666667)(36.16026711185309, 31.985723)(36.26043405676127, 31.98030491666666)(36.36060100166945, 31.983566500000006)(36.46076794657763, 31.933074)(36.56093489148581, 31.944157333333333)(36.66110183639399, 31.93533300000001)(36.76126878130217, 32.14575124999999)(36.86143572621035, 31.98771641666667)(36.96160267111853, 31.954787833333327)(37.06176961602671, 32.01872675)(37.16193656093489, 31.982177749999984)(37.26210350584308, 31.977553583333336)(37.362270450751254, 31.963134166666674)(37.462437395659435, 31.994277250000007)(37.56260434056762, 31.94980333333333)(37.6627712854758, 32.036386749999984)(37.76293823038397, 32.00513641666667)(37.863105175292155, 31.999679000000008)(37.96327212020034, 31.955982333333324)(38.06343906510852, 31.964726999999993)(38.16360601001669, 32.108092333333325)(38.263772954924875, 31.931711666666676)(38.363939899833056, 31.933013500000012)(38.46410684474124, 32.10290924999999)(38.56427378964941, 32.02684441666667)(38.664440734557594, 31.980097666666655)(38.764607679465776, 32.00365075000001)(38.86477462437396, 32.07371483333334)(38.96494156928214, 31.893529083333323)(39.06510851419032, 32.027470416666674)(39.1652754590985, 32.01346308333334)(39.26544240400668, 31.972294333333327)(39.36560934891486, 32.19355141666665)(39.46577629382304, 31.945307166666662)(39.56594323873122, 32.060764583333324)(39.666110183639404, 32.043924583333336)(39.76627712854758, 31.998875333333338)(39.86644407345576, 32.01976466666667)(39.96661101836394, 32.0394585)(40.066777963272116, 32.00691141666667)(40.1669449081803, 32.080093083333324)(40.26711185308848, 31.996306916666676)(40.36727879799666, 31.98175458333333)(40.46744574290484, 32.01517691666667)(40.567612687813025, 32.05137608333334)(40.667779632721206, 31.938094583333335)(40.76794657762939, 32.06258033333332)(40.86811352253756, 32.02087349999999)(40.968280467445744, 32.10552883333334)(41.068447412353926, 32.10883041666666)(41.16861435726211, 31.973687583333334)(41.26878130217028, 31.992090666666652)(41.368948247078464, 31.999348916666673)(41.469115191986646, 32.002980083333334)(41.56928213689483, 32.12598541666668)(41.669449081803, 31.88301541666664)(41.769616026711184, 32.03867066666667)(41.869782971619365, 32.06859283333334)(41.96994991652755, 32.000929666666664)(42.07011686143572, 31.99825075)(42.1702838063439, 32.019245166666664)(42.27045075125209, 31.952971750000007)(42.370617696160274, 32.02019166666668)(42.47078464106845, 31.947189499999993)(42.57095158597663, 32.140415250000004)(42.67111853088481, 32.05076049999999)(42.77128547579299, 31.931548333333332)(42.87145242070117, 32.064839583333345)(42.97161936560935, 32.01756158333333)(43.07178631051753, 32.010019833333324)(43.17195325542571, 31.897765749999984)(43.27212020033389, 32.108524666666675)(43.37228714524207, 32.06517483333334)(43.47245409015025, 31.991200583333345)(43.57262103505843, 31.974664583333343)(43.67278797996661, 31.96494508333334)(43.77295492487479, 32.02169708333335)(43.87312186978298, 32.12096608333334)(43.97328881469116, 31.941670249999998)(44.073455759599334, 32.09176566666667)(44.173622704507515, 32.04632499999998)(44.2737896494157, 31.997242083333326)(44.37395659432388, 32.02033941666666)(44.47412353923205, 32.01915958333334)(44.574290484140235, 31.992070249999994)(44.67445742904842, 32.009388499999986)(44.7746243739566, 31.990931000000007)(44.87479131886477, 32.17948275)(44.974958263772955, 31.997217250000006)(45.075125208681136, 31.966333916666663)(45.17529215358932, 32.106714000000004)(45.27545909849749, 32.109673916666665)(45.375626043405674, 32.015701)(45.475792988313856, 31.99905383333332)(45.57595993322204, 32.09818375)(45.67612687813022, 32.05909691666666)(45.7762938230384, 31.969278416666658)(45.87646076794658, 32.10719216666666)(45.976627712854764, 32.05162491666667)(46.07679465776294, 32.00100708333333)(46.17696160267112, 32.09456275)(46.2771285475793, 32.15578608333333)(46.37729549248748, 31.989369333333354)(46.47746243739566, 31.989159916666658)(46.57762938230384, 32.08361275)(46.67779632721202, 32.079273250000014)(46.7779632721202, 31.994552833333326)(46.87813021702838, 32.14852775000001)(46.97829716193656, 32.02682916666667)(47.07846410684474, 32.07962999999998)(47.17863105175292, 32.062153833333355)(47.278797996661105, 32.12180458333333)(47.378964941569286, 32.03082258333332)(47.47913188647747, 32.1511325)(47.57929883138564, 32.00138224999999)(47.679465776293824, 32.07852625)(47.779632721202006, 32.09013808333335)(47.87979966611019, 32.023401583333325)(47.97996661101836, 32.08304724999999)(48.080133555926544, 32.100209)(48.180300500834726, 32.087803166666674)(48.28046744574291, 32.03374708333333)(48.38063439065108, 32.081872249999996)(48.48080133555926, 32.05447325)(48.580968280467445, 32.102925333333324)(48.68113522537563, 31.979043833333353)(48.7813021702838, 32.04027733333334)(48.88146911519199, 32.15210375)(48.98163606010017, 32.02125558333334)(49.081803005008354, 32.05983324999999)(49.18196994991653, 32.06901041666666)(49.28213689482471, 32.05497141666665)(49.38230383973289, 32.074940083333324)(49.48247078464107, 32.11490816666667)(49.58263772954925, 32.0865525)(49.68280467445743, 32.03812566666665)(49.78297161936561, 32.167754083333335)(49.88313856427379, 32.0815065)(49.98330550918197, 32.21412508333333)(50.08347245409015, 31.99715133333335)(50.18363939899833, 32.11391616666668)(50.28380634390651, 32.129160250000005)(50.38397328881469, 32.15496741666665)(50.484140233722876, 32.11268533333333)(50.58430717863106, 32.08446725)(50.68447412353924, 32.02376708333334)(50.784641068447414, 32.00808091666666)(50.884808013355595, 32.16024225)(50.98497495826378, 32.26714458333333)(51.08514190317196, 32.173928499999995)(51.18530884808013, 32.04731149999999)(51.285475792988315, 32.174285250000004)(51.3856427378965, 32.13613791666667)(51.48580968280468, 32.13729808333334)(51.58597662771285, 32.10627608333333)(51.686143572621035, 32.16095925000001)(51.786310517529216, 32.13053266666667)(51.8864774624374, 32.040954000000006)(51.98664440734557, 32.097091083333325)(52.086811352253754, 32.05864891666668)(52.18697829716194, 32.151371499999996)(52.287145242070125, 32.18211249999999)(52.3873121869783, 32.230319916666666)(52.48747913188648, 32.07240225000001)(52.58764607679466, 31.97149124999999)(52.68781302170284, 32.187993166666644)(52.78797996661102, 32.20659191666668)(52.8881469115192, 32.106953416666656)(52.98831385642738, 32.122257416666685)(53.088480801335564, 32.12591)(53.18864774624374, 32.03088808333333)(53.28881469115192, 32.19790033333333)(53.3889816360601, 32.075240750000006)(53.489148580968276, 32.17222991666667)(53.58931552587646, 32.05520508333333)(53.68948247078464, 32.31880558333333)(53.78964941569283, 32.033945166666655)(53.889816360601, 32.04262250000001)(53.989983305509185, 32.087401250000006)(54.090150250417366, 32.23257816666666)(54.19031719532555, 32.11215191666666)(54.29048414023372, 32.21167416666665)(54.390651085141904, 32.05754)(54.490818030050086, 32.098062)(54.59098497495827, 32.12853416666667)(54.69115191986644, 32.26206833333334)(54.791318864774624, 32.06719866666666)(54.891485809682806, 32.207905666666655)(54.99165275459099, 32.127211249999995)(55.09181969949916, 32.15291816666666)(55.19198664440734, 32.12733841666668)(55.292153589315525, 32.212248833333334)(55.39232053422371, 32.188190666666664)(55.49248747913189, 32.18317100000001)(55.59265442404007, 32.040444833333346)(55.69282136894825, 32.145618750000004)(55.79298831385643, 32.14020716666666)(55.89315525876461, 32.21383066666667)(55.99332220367279, 32.124586916666665)(56.09348914858097, 32.21106891666666)(56.19365609348915, 32.075403)(56.29382303839733, 32.189289166666676)(56.39398998330551, 32.210427750000015)(56.49415692821369, 32.19847033333334)(56.59432387312187, 32.075077250000014)(56.69449081803005, 32.297291250000015)(56.79465776293823, 32.091094500000004)(56.89482470784641, 32.123375666666654)(56.99499165275459, 32.14119383333334)(57.095158597662774, 32.21820999999999)(57.195325542570956, 32.07689841666668)(57.29549248747914, 32.25097025)(57.39565943238732, 32.17645683333331)(57.495826377295494, 32.11214116666666)(57.595993322203675, 32.07812858333334)(57.69616026711186, 32.278416416666666)(57.79632721202004, 32.17669574999999)(57.89649415692821, 32.24967291666666)(57.996661101836395, 32.135965750000004)(58.09682804674458, 32.078093333333335)(58.19699499165276, 32.122197250000006)(58.29716193656093, 32.38202225000001)(58.397328881469114, 32.19955350000001)(58.497495826377296, 32.23630166666667)(58.59766277128548, 32.02870633333333)(58.69782971619365, 32.205203250000004)(58.79799666110184, 32.252913583333346)(58.89816360601002, 32.24022833333334)(58.9983305509182, 32.176995166666664)(59.09849749582638, 32.156814000000004)(59.19866444073456, 32.099105500000015)(59.29883138564274, 32.26596458333332)(59.398998330550924, 32.25129566666668)(59.4991652754591, 32.1914155)(59.59933222036728, 32.23056966666665)(59.69949916527546, 32.19982258333333)(59.79966611018364, 32.148822999999986)(59.89983305509182, 32.217116916666654)(60.0, 32.169676833333334)
        };
        \addplot[color=blue, mark=none,name path=A] coordinates { %% MAX value
        (0.0, 0.0)(0.1001669449081803, 36.79311)(0.2003338898163606, 36.16019)(0.3005008347245409, 41.76625)(0.4006677796327212, 48.3916)(0.5008347245409015, 33.65043)(0.6010016694490818, 44.37855)(0.7011686143572621, 34.42313)(0.8013355592654424, 35.569359999999996)(0.9015025041736228, 34.22111)(1.001669449081803, 32.951570000000004)(1.1018363939899833, 32.40959)(1.2020033388981637, 32.39372)(1.3021702838063438, 32.91434)(1.4023372287145242, 33.243939999999995)(1.5025041736227045, 32.57926)(1.6026711185308848, 32.459630000000004)(1.7028380634390652, 32.57316)(1.8030050083472455, 32.695840000000004)(1.9031719532554257, 37.48404)(2.003338898163606, 37.38637)(2.1035058430717863, 34.46342)(2.2036727879799667, 32.45353)(2.303839732888147, 33.07181)(2.4040066777963274, 44.261970000000005)(2.5041736227045073, 33.06877)(2.6043405676126876, 32.68852)(2.704507512520868, 32.77152)(2.8046744574290483, 32.97538)(2.9048414023372287, 32.79533)(3.005008347245409, 32.78739)(3.1051752921535893, 32.49259)(3.2053422370617697, 32.51884)(3.30550918196995, 33.59245)(3.4056761268781304, 34.5452)(3.5058430717863107, 32.54142)(3.606010016694491, 32.89482)(3.7061769616026714, 32.835)(3.8063439065108513, 32.51029)(3.906510851419032, 32.32902)(4.006677796327212, 32.52739)(4.106844741235393, 40.32094)(4.207011686143573, 32.51335)(4.3071786310517535, 33.45695)(4.407345575959933, 43.44593)(4.507512520868114, 32.86551)(4.607679465776294, 32.66532)(4.707846410684475, 32.96806)(4.808013355592655, 32.83073)(4.908180300500835, 32.62076)(5.0083472454090145, 33.091339999999995)(5.108514190317195, 32.66838)(5.208681135225375, 32.9998)(5.308848080133556, 32.552409999999995)(5.409015025041736, 32.87773)(5.509181969949917, 32.30766)(5.609348914858097, 32.8411)(5.709515859766277, 32.78067)(5.809682804674457, 32.65739)(5.909849749582638, 33.228680000000004)(6.010016694490818, 33.37088)(6.110183639398999, 32.31132)(6.210350584307179, 32.73673)(6.3105175292153595, 32.35465)(6.410684474123539, 32.96073)(6.510851419031719, 32.69767)(6.6110183639399, 32.83561)(6.71118530884808, 32.43217)(6.811352253756261, 32.39921)(6.911519198664441, 32.45048)(7.011686143572621, 33.55094)(7.111853088480801, 32.51029)(7.212020033388982, 33.04313)(7.312186978297162, 33.24882)(7.412353923205343, 32.89237)(7.512520868113523, 32.93143)(7.612687813021703, 33.055949999999996)(7.712854757929884, 32.8179)(7.813021702838064, 33.35929)(7.913188647746244, 32.74527)(8.013355592654424, 32.71048)(8.113522537562606, 32.584140000000005)(8.213689482470786, 33.58512)(8.313856427378965, 33.06998)(8.414023372287145, 32.6226)(8.514190317195327, 32.41997)(8.614357262103507, 32.49137)(8.714524207011687, 33.127970000000005)(8.814691151919867, 32.532869999999996)(8.914858096828048, 32.603680000000004)(9.015025041736228, 32.90641)(9.115191986644408, 33.12309)(9.215358931552588, 32.54753)(9.31552587646077, 32.5286)(9.41569282136895, 32.77396)(9.51585976627713, 32.55363)(9.61602671118531, 32.51212)(9.71619365609349, 33.33549)(9.81636060100167, 33.61869)(9.916527545909851, 33.16214)(10.016694490818029, 33.2067)(10.11686143572621, 33.185340000000004)(10.21702838063439, 33.43376)(10.31719532554257, 33.2598)(10.41736227045075, 33.34708)(10.51752921535893, 33.14811)(10.617696160267112, 33.187780000000004)(10.717863105175292, 32.54508)(10.818030050083472, 32.54569)(10.918196994991652, 33.463660000000004)(11.018363939899833, 32.81486)(11.118530884808013, 32.293620000000004)(11.218697829716193, 32.42057)(11.318864774624373, 32.66715)(11.419031719532555, 32.86857)(11.519198664440735, 32.6165)(11.619365609348915, 33.25919)(11.719532554257095, 33.08768)(11.819699499165276, 32.49686)(11.919866444073456, 33.88785)(12.020033388981636, 32.73551)(12.120200333889816, 32.70621)(12.220367278797998, 32.640299999999996)(12.320534223706177, 32.45536)(12.420701168614357, 32.71842)(12.520868113522537, 32.71354)(12.621035058430719, 32.47855)(12.721202003338899, 32.74467)(12.821368948247079, 32.94303)(12.921535893155259, 33.378209999999996)(13.021702838063439, 32.4279)(13.12186978297162, 32.93387)(13.2220367278798, 32.56889)(13.32220367278798, 32.95096)(13.42237061769616, 32.497479999999996)(13.522537562604342, 32.41386)(13.622704507512521, 32.50846)(13.722871452420701, 32.43217)(13.823038397328881, 32.63602)(13.923205342237063, 32.82463)(14.023372287145243, 32.695840000000004)(14.123539232053423, 32.51457)(14.223706176961603, 32.4755)(14.323873121869784, 32.53531)(14.424040066777964, 32.43034)(14.524207011686144, 32.67631)(14.624373956594324, 32.55485)(14.724540901502506, 33.42521)(14.824707846410686, 32.42241)(14.924874791318866, 32.7233)(15.025041736227045, 32.5103)(15.125208681135225, 32.59818)(15.225375626043405, 32.76786)(15.325542570951589, 32.954629999999995)(15.425709515859769, 32.32597)(15.525876460767948, 32.45597)(15.626043405676128, 32.68973)(15.726210350584308, 32.53837)(15.826377295492488, 32.81485)(15.926544240400668, 32.96134)(16.026711185308848, 32.8173)(16.126878130217026, 32.488929999999996)(16.22704507512521, 32.67814)(16.32721202003339, 32.64396)(16.42737896494157, 33.07304)(16.52754590984975, 32.56156)(16.62771285475793, 32.60306)(16.72787979966611, 32.57499)(16.82804674457429, 32.70194)(16.92821368948247, 32.53226)(17.028380634390654, 33.24637)(17.128547579298832, 32.66532)(17.228714524207014, 32.56157)(17.328881469115192, 32.45536)(17.429048414023374, 32.52678)(17.529215358931552, 32.97904)(17.629382303839733, 32.69706)(17.72954924874791, 33.1591)(17.829716193656097, 32.56217)(17.929883138564275, 32.43461)(18.030050083472457, 32.51456)(18.130217028380635, 32.50663)(18.230383973288816, 32.53654)(18.330550918196995, 32.68119)(18.430717863105176, 33.62113)(18.530884808013354, 32.45353)(18.63105175292154, 32.66288)(18.731218697829718, 32.49931)(18.8313856427379, 32.88444)(18.931552587646078, 32.93021)(19.03171953255426, 32.80326)(19.131886477462437, 32.96378)(19.23205342237062, 32.74101)(19.332220367278797, 32.888099999999994)(19.43238731218698, 32.83561)(19.53255425709516, 32.90702)(19.63272120200334, 32.61162)(19.73288814691152, 32.64457)(19.833055091819702, 32.59025)(19.93322203672788, 32.55546)(20.033388981636058, 32.75505)(20.13355592654424, 32.65677)(20.23372287145242, 32.78861)(20.333889816360603, 33.209140000000005)(20.43405676126878, 33.12858)(20.534223706176963, 32.87834)(20.63439065108514, 33.44779)(20.734557595993323, 32.66776)(20.8347245409015, 32.70072)(20.934891485809683, 32.73551)(21.03505843071786, 33.405680000000004)(21.135225375626046, 32.58903)(21.235392320534224, 33.042519999999996)(21.335559265442406, 32.798379999999995)(21.435726210350584, 32.98575)(21.535893155258766, 32.560950000000005)(21.636060100166944, 32.92228)(21.736227045075125, 32.48954)(21.836393989983303, 33.223189999999995)(21.93656093489149, 32.73673)(22.036727879799667, 32.54752)(22.13689482470785, 32.67264)(22.237061769616027, 32.86979)(22.33722871452421, 32.77031)(22.437395659432386, 32.75138)(22.537562604340568, 33.81827)(22.637729549248746, 32.70378)(22.737896494156928, 33.17924)(22.83806343906511, 32.39249)(22.93823038397329, 32.580490000000005)(23.03839732888147, 33.096230000000006)(23.13856427378965, 32.75504)(23.23873121869783, 32.51822)(23.33889816360601, 33.28483)(23.43906510851419, 32.69523)(23.53923205342237, 32.56889)(23.639398998330552, 33.24149)(23.739565943238734, 32.8057)(23.839732888146912, 32.53837)(23.939899833055094, 32.61771)(24.040066777963272, 32.88932)(24.140233722871454, 32.895430000000005)(24.24040066777963, 32.70805)(24.340567612687813, 33.12675)(24.440734557595995, 33.12186)(24.540901502504177, 32.72209)(24.641068447412355, 32.69096)(24.741235392320537, 32.99003)(24.841402337228715, 32.77702)(24.941569282136896, 32.70804)(25.041736227045075, 32.99552)(25.141903171953256, 32.994910000000004)(25.242070116861438, 32.68302)(25.34223706176962, 32.6348)(25.442404006677798, 32.99308)(25.54257095158598, 33.282379999999996)(25.642737896494157, 33.80424)(25.74290484140234, 32.56522)(25.843071786310517, 33.12186)(25.9432387312187, 35.61454)(26.043405676126877, 34.25712)(26.143572621035062, 46.34937)(26.24373956594324, 47.81543)(26.34390651085142, 44.68311)(26.4440734557596, 46.94874)(26.544240400667782, 47.067139999999995)(26.64440734557596, 47.692750000000004)(26.744574290484138, 36.50869)(26.84474123539232, 33.06205)(26.9449081803005, 43.958009999999994)(27.045075125208683, 57.98752999999999)(27.14524207011686, 61.44393)(27.245409015025043, 39.69045)(27.34557595993322, 32.6702)(27.445742904841403, 32.93631)(27.54590984974958, 33.38187)(27.646076794657763, 33.31351)(27.746243739565944, 32.7587)(27.846410684474126, 32.92411)(27.946577629382304, 38.25003)(28.046744574290486, 46.30542)(28.146911519198664, 49.00744)(28.247078464106846, 48.88598)(28.347245409015024, 47.683600000000006)(28.447412353923205, 47.144040000000004)(28.547579298831387, 48.14197)(28.64774624373957, 47.137330000000006)(28.747913188647747, 48.12854)(28.84808013355593, 48.43249)(28.948247078464107, 46.13575)(29.04841402337229, 44.124030000000005)(29.148580968280466, 32.8411)(29.248747913188648, 32.911899999999996)(29.348914858096826, 33.65958)(29.44908180300501, 32.8649)(29.54924874791319, 32.86307)(29.64941569282137, 32.70683)(29.74958263772955, 32.82463)(29.84974958263773, 32.91617)(29.94991652754591, 40.49245)(30.05008347245409, 43.88599)(30.15025041736227, 47.44739)(30.25041736227045, 34.11734)(30.35058430717863, 34.38956)(30.45075125208681, 33.3129)(30.55091819699499, 33.04374)(30.651085141903177, 33.13041)(30.751252086811355, 33.40385)(30.851419031719537, 32.83805)(30.951585976627715, 32.886269999999996)(31.051752921535897, 33.21586)(31.151919866444075, 32.626870000000004)(31.252086811352257, 33.11576)(31.352253756260435, 32.60734)(31.452420701168617, 33.25064)(31.552587646076795, 33.01932)(31.652754590984976, 32.69523)(31.752921535893154, 32.88444)(31.853088480801336, 33.14994)(31.953255425709514, 33.51493)(32.053422370617696, 32.7642)(32.15358931552588, 33.123689999999996)(32.25375626043405, 32.80632)(32.35392320534224, 32.957679999999996)(32.45409015025042, 32.64152)(32.554257095158604, 33.068149999999996)(32.65442404006678, 32.7758)(32.75459098497496, 32.80448)(32.85475792988314, 32.80204)(32.95492487479132, 32.84232)(33.0550918196995, 32.90153)(33.15525876460768, 32.95036)(33.25542570951586, 33.71146)(33.35559265442404, 33.04069)(33.45575959933222, 33.67545)(33.5559265442404, 33.46305)(33.65609348914858, 32.56828)(33.756260434056756, 32.820969999999996)(33.85642737896494, 33.10111)(33.95659432387313, 33.23966)(34.05676126878131, 32.90397)(34.15692821368948, 33.11332)(34.257095158597664, 32.79777)(34.357262103505846, 32.76603)(34.45742904841403, 33.25004)(34.5575959933222, 32.88077)(34.657762938230384, 32.82279)(34.757929883138566, 32.78983)(34.85809682804675, 33.11881)(34.95826377295492, 33.330600000000004)(35.058430717863104, 33.09317)(35.158597662771285, 33.14689)(35.25876460767947, 33.12247)(35.35893155258764, 32.86002)(35.45909849749582, 33.251870000000004)(35.559265442404005, 33.330600000000004)(35.659432387312194, 35.64627)(35.75959933222037, 33.62235)(35.85976627712855, 33.35563)(35.95993322203673, 32.92777)(36.06010016694491, 32.91556)(36.16026711185309, 33.03946)(36.26043405676127, 33.32634)(36.36060100166945, 32.92472)(36.46076794657763, 32.91923)(36.56093489148581, 32.98148)(36.66110183639399, 33.20915)(36.76126878130217, 33.28117)(36.86143572621035, 33.52897)(36.96160267111853, 33.06754)(37.06176961602671, 33.283)(37.16193656093489, 33.22867)(37.26210350584308, 33.30619)(37.362270450751254, 32.88687)(37.462437395659435, 33.53874)(37.56260434056762, 33.25126)(37.6627712854758, 33.14689)(37.76293823038397, 33.087689999999995)(37.863105175292155, 33.22013)(37.96327212020034, 32.72696)(38.06343906510852, 33.1597)(38.16360601001669, 33.58329)(38.263772954924875, 33.3831)(38.363939899833056, 32.932649999999995)(38.46410684474124, 33.59671)(38.56427378964941, 33.22441)(38.664440734557594, 32.86979)(38.764607679465776, 33.25553)(38.86477462437396, 33.23478)(38.96494156928214, 32.75566)(39.06510851419032, 33.01932)(39.1652754590985, 36.29264)(39.26544240400668, 33.20792)(39.36560934891486, 33.320840000000004)(39.46577629382304, 32.93083)(39.56594323873122, 33.03581)(39.666110183639404, 33.37027)(39.76627712854758, 32.838660000000004)(39.86644407345576, 33.16825)(39.96661101836394, 32.986360000000005)(40.066777963272116, 32.809979999999996)(40.1669449081803, 33.63944)(40.26711185308848, 33.134679999999996)(40.36727879799666, 33.28116)(40.46744574290484, 33.21464)(40.567612687813025, 33.13895)(40.667779632721206, 33.08646)(40.76794657762939, 33.17496)(40.86811352253756, 33.903729999999996)(40.968280467445744, 33.24699)(41.068447412353926, 33.61137)(41.16861435726211, 33.04862)(41.26878130217028, 33.127970000000005)(41.368948247078464, 33.223189999999995)(41.469115191986646, 32.90214)(41.56928213689483, 33.90616)(41.669449081803, 32.94059)(41.769616026711184, 32.955239999999996)(41.869782971619365, 33.12552)(41.96994991652755, 33.38126)(42.07011686143572, 33.04069)(42.1702838063439, 33.19694)(42.27045075125209, 32.77153)(42.370617696160274, 33.20854)(42.47078464106845, 33.14994)(42.57095158597663, 33.13407)(42.67111853088481, 33.209140000000005)(42.77128547579299, 32.79411)(42.87145242070117, 32.93449)(42.97161936560935, 33.18717)(43.07178631051753, 32.97965)(43.17195325542571, 33.134679999999996)(43.27212020033389, 33.38553)(43.37228714524207, 33.13895)(43.47245409015025, 32.97294)(43.57262103505843, 33.27079)(43.67278797996661, 33.43131)(43.77295492487479, 32.9412)(43.87312186978298, 33.17496)(43.97328881469116, 33.08403)(44.073455759599334, 33.18472)(44.173622704507515, 33.05106)(44.2737896494157, 33.14749)(44.37395659432388, 32.8942)(44.47412353923205, 32.95707)(44.574290484140235, 33.40934)(44.67445742904842, 33.05961)(44.7746243739566, 32.88017)(44.87479131886477, 33.54423)(44.974958263772955, 32.90397)(45.075125208681136, 33.12064)(45.17529215358932, 33.168859999999995)(45.27545909849749, 33.235389999999995)(45.375626043405674, 33.36052)(45.475792988313856, 32.75627)(45.57595993322204, 33.165800000000004)(45.67612687813022, 33.1182)(45.7762938230384, 32.80753)(45.87646076794658, 33.61503)(45.976627712854764, 33.02055)(46.07679465776294, 33.26896)(46.17696160267112, 33.928129999999996)(46.2771285475793, 33.12919)(46.37729549248748, 33.15665)(46.47746243739566, 32.83927)(46.57762938230384, 33.11271)(46.67779632721202, 33.84147)(46.7779632721202, 32.88688)(46.87813021702838, 33.30131)(46.97829716193656, 33.02177)(47.07846410684474, 34.79606)(47.17863105175292, 33.08952)(47.278797996661105, 33.44657)(47.378964941569286, 32.975989999999996)(47.47913188647747, 33.07304)(47.57929883138564, 33.16093)(47.679465776293824, 33.32572)(47.779632721202006, 33.17191)(47.87979966611019, 33.05167)(47.97996661101836, 33.19938)(48.080133555926544, 33.03336)(48.180300500834726, 32.72757)(48.28046744574291, 32.91495)(48.38063439065108, 33.55643)(48.48080133555926, 32.86796)(48.580968280467445, 33.35318)(48.68113522537563, 33.22562)(48.7813021702838, 33.098060000000004)(48.88146911519199, 33.33426)(48.98163606010017, 33.32755)(49.081803005008354, 33.44413)(49.18196994991653, 32.86552)(49.28213689482471, 33.20731)(49.38230383973289, 33.21036)(49.48247078464107, 33.00528)(49.58263772954925, 33.41422)(49.68280467445743, 33.095620000000004)(49.78297161936561, 33.1713)(49.88313856427379, 33.14139)(49.98330550918197, 33.21708)(50.08347245409015, 33.01993)(50.18363939899833, 33.24699)(50.28380634390651, 33.31901)(50.38397328881469, 33.6193)(50.484140233722876, 33.17558)(50.58430717863106, 33.19083)(50.68447412353924, 32.77885)(50.784641068447414, 32.95219)(50.884808013355595, 33.217690000000005)(50.98497495826378, 34.14542)(51.08514190317196, 33.81339)(51.18530884808013, 33.27384)(51.285475792988315, 33.19205)(51.3856427378965, 33.20915)(51.48580968280468, 33.56559)(51.58597662771285, 33.07852)(51.686143572621035, 33.03275)(51.786310517529216, 33.27689)(51.8864774624374, 32.783120000000004)(51.98664440734557, 33.249430000000004)(52.086811352253754, 33.17924)(52.18697829716194, 33.62052)(52.287145242070125, 33.17618)(52.3873121869783, 33.34037)(52.48747913188648, 33.03825)(52.58764607679466, 33.24699)(52.68781302170284, 33.49113)(52.78797996661102, 33.36844)(52.8881469115192, 32.96866)(52.98831385642738, 33.30619)(53.088480801335564, 33.31778)(53.18864774624374, 33.07059)(53.28881469115192, 33.4008)(53.3889816360601, 32.96561)(53.489148580968276, 34.72159)(53.58931552587646, 33.10966)(53.68948247078464, 33.477090000000004)(53.78964941569283, 33.057159999999996)(53.889816360601, 33.474650000000004)(53.989983305509185, 33.48441)(54.090150250417366, 33.84635)(54.19031719532555, 33.436189999999996)(54.29048414023372, 33.15299)(54.390651085141904, 33.55766)(54.490818030050086, 33.41544)(54.59098497495827, 33.254310000000004)(54.69115191986644, 33.5253)(54.791318864774624, 32.798379999999995)(54.891485809682806, 33.13956)(54.99165275459099, 32.93754)(55.09181969949916, 33.26835)(55.19198664440734, 32.91556)(55.292153589315525, 33.254310000000004)(55.39232053422371, 33.51982)(55.49248747913189, 33.3367)(55.59265442404007, 32.94853)(55.69282136894825, 33.74626)(55.79298831385643, 32.96928)(55.89315525876461, 33.235389999999995)(55.99332220367279, 33.2537)(56.09348914858097, 33.25797)(56.19365609348915, 33.2067)(56.29382303839733, 33.79264)(56.39398998330551, 33.62723)(56.49415692821369, 33.11454)(56.59432387312187, 33.11332)(56.69449081803005, 33.63944)(56.79465776293823, 33.27812)(56.89482470784641, 33.25858)(56.99499165275459, 33.04374)(57.095158597662774, 33.566810000000004)(57.195325542570956, 33.044349999999994)(57.29549248747914, 33.39958)(57.39565943238732, 33.07853)(57.495826377295494, 33.257979999999996)(57.595993322203675, 33.2061)(57.69616026711186, 33.22379)(57.79632721202004, 33.05472)(57.89649415692821, 33.32938)(57.996661101836395, 33.17558)(58.09682804674458, 32.99064)(58.19699499165276, 34.03617)(58.29716193656093, 33.36722)(58.397328881469114, 33.16337)(58.497495826377296, 33.913489999999996)(58.59766277128548, 33.7493)(58.69782971619365, 33.37638)(58.79799666110184, 33.38981)(58.89816360601002, 33.29398)(58.9983305509182, 33.0767)(59.09849749582638, 33.344030000000004)(59.19866444073456, 33.26347)(59.29883138564274, 33.83964)(59.398998330550924, 33.11027)(59.4991652754591, 33.22806)(59.59933222036728, 33.14628)(59.69949916527546, 33.19571)(59.79966611018364, 33.15177)(59.89983305509182, 32.822179999999996)(60.0, 33.185340000000004)
        };
        \addplot[color=blue, mark=none,name path=B] coordinates { %% MIN value
        (0.0, 0.0)(0.1001669449081803, 29.580000000000002)(0.2003338898163606, 32.29667)(0.3005008347245409, 30.16228)(0.4006677796327212, 30.53032)(0.5008347245409015, 30.61271)(0.6010016694490818, 30.670090000000002)(0.7011686143572621, 30.323410000000003)(0.8013355592654424, 30.396040000000003)(0.9015025041736228, 31.00578)(1.001669449081803, 30.167769999999997)(1.1018363939899833, 29.979779999999998)(1.2020033388981637, 30.848309999999998)(1.3021702838063438, 30.389940000000003)(1.4023372287145242, 29.73686)(1.5025041736227045, 30.70304)(1.6026711185308848, 30.06646)(1.7028380634390652, 30.598679999999998)(1.8030050083472455, 29.87358)(1.9031719532554257, 30.57853)(2.003338898163606, 30.16044)(2.1035058430717863, 30.521169999999998)(2.2036727879799667, 30.438760000000002)(2.303839732888147, 30.83061)(2.4040066777963274, 30.33745)(2.5041736227045073, 23.07795)(2.6043405676126876, 30.42839)(2.704507512520868, 29.938889999999997)(2.8046744574290483, 30.79094)(2.9048414023372287, 30.79582)(3.005008347245409, 30.712809999999998)(3.1051752921535893, 30.35881)(3.2053422370617697, 30.36857)(3.30550918196995, 30.74821)(3.4056761268781304, 30.492469999999997)(3.5058430717863107, 30.313650000000003)(3.606010016694491, 30.66704)(3.7061769616026714, 30.4528)(3.8063439065108513, 30.85442)(3.906510851419032, 30.516280000000002)(4.006677796327212, 30.613929999999996)(4.106844741235393, 30.227590000000003)(4.207011686143573, 30.29106)(4.3071786310517535, 30.96855)(4.407345575959933, 30.61881)(4.507512520868114, 30.58463)(4.607679465776294, 30.64995)(4.707846410684475, 30.511400000000002)(4.808013355592655, 30.010910000000003)(4.908180300500835, 30.75065)(5.0083472454090145, 30.770179999999996)(5.108514190317195, 30.48149)(5.208681135225375, 30.65177)(5.308848080133556, 30.70243)(5.409015025041736, 30.45402)(5.509181969949917, 30.739060000000002)(5.609348914858097, 30.87456)(5.709515859766277, 30.77324)(5.809682804674457, 30.40886)(5.909849749582638, 30.624920000000003)(6.010016694490818, 30.93131)(6.110183639398999, 30.732339999999997)(6.210350584307179, 30.624920000000003)(6.3105175292153595, 30.665809999999997)(6.410684474123539, 30.19341)(6.510851419031719, 30.76957)(6.6110183639399, 30.86174)(6.71118530884808, 30.76836)(6.811352253756261, 30.634079999999997)(6.911519198664441, 30.88737)(7.011686143572621, 30.66094)(7.111853088480801, 30.62003)(7.212020033388982, 30.721349999999997)(7.312186978297162, 30.96855)(7.412353923205343, 30.18608)(7.512520868113523, 30.66215)(7.612687813021703, 30.884320000000002)(7.712854757929884, 30.87517)(7.813021702838064, 30.15801)(7.913188647746244, 30.78056)(8.013355592654424, 30.71891)(8.113522537562606, 30.70976)(8.213689482470786, 30.79094)(8.313856427378965, 30.70304)(8.414023372287145, 30.5828)(8.514190317195327, 30.58769)(8.614357262103507, 31.15532)(8.714524207011687, 30.443649999999998)(8.814691151919867, 30.672530000000002)(8.914858096828048, 30.748820000000002)(9.015025041736228, 30.78239)(9.115191986644408, 30.77324)(9.215358931552588, 30.72991)(9.31552587646077, 30.70304)(9.41569282136895, 30.702440000000003)(9.51585976627713, 30.460120000000003)(9.61602671118531, 30.832440000000002)(9.71619365609349, 30.77995)(9.81636060100167, 30.96366)(9.916527545909851, 30.809859999999997)(10.016694490818029, 30.91422)(10.11686143572621, 30.63164)(10.21702838063439, 30.90202)(10.31719532554257, 30.81962)(10.41736227045075, 30.838549999999998)(10.51752921535893, 30.330119999999997)(10.617696160267112, 30.79215)(10.717863105175292, 30.96916)(10.818030050083472, 30.670090000000002)(10.918196994991652, 30.882489999999997)(11.018363939899833, 30.93253)(11.118530884808013, 30.72197)(11.218697829716193, 30.91972)(11.318864774624373, 30.740280000000002)(11.419031719532555, 30.903239999999997)(11.519198664440735, 30.66642)(11.619365609348915, 30.613929999999996)(11.719532554257095, 31.021649999999998)(11.819699499165276, 30.670700000000004)(11.919866444073456, 30.83428)(12.020033388981636, 30.57671)(12.120200333889816, 30.72685)(12.220367278797998, 31.05705)(12.320534223706177, 30.916670000000003)(12.420701168614357, 30.95268)(12.520868113522537, 30.59135)(12.621035058430719, 30.85319)(12.721202003338899, 30.86723)(12.821368948247079, 30.68168)(12.921535893155259, 30.77019)(13.021702838063439, 30.94109)(13.12186978297162, 30.79338)(13.2220367278798, 30.83916)(13.32220367278798, 30.928870000000003)(13.42237061769616, 30.93131)(13.522537562604342, 30.47233)(13.622704507512521, 30.97282)(13.722871452420701, 30.804969999999997)(13.823038397328881, 30.676190000000002)(13.923205342237063, 30.873330000000003)(14.023372287145243, 30.75371)(14.123539232053423, 30.907510000000002)(14.223706176961603, 30.77202)(14.323873121869784, 30.9362)(14.424040066777964, 30.925829999999998)(14.524207011686144, 30.435720000000003)(14.624373956594324, 31.04118)(14.724540901502506, 30.908739999999998)(14.824707846410686, 30.80193)(14.924874791318866, 31.02348)(15.025041736227045, 30.905079999999998)(15.125208681135225, 30.82207)(15.225375626043405, 30.604170000000003)(15.325542570951589, 31.002119999999998)(15.425709515859769, 30.97893)(15.525876460767948, 30.532140000000002)(15.626043405676128, 31.01371)(15.726210350584308, 31.03081)(15.826377295492488, 30.71647)(15.926544240400668, 30.9362)(16.026711185308848, 30.954510000000003)(16.126878130217026, 30.81962)(16.22704507512521, 30.908739999999998)(16.32721202003339, 30.7653)(16.42737896494157, 30.96794)(16.52754590984975, 30.74333)(16.62771285475793, 30.986250000000002)(16.72787979966611, 30.83916)(16.82804674457429, 30.642010000000003)(16.92821368948247, 30.82756)(17.028380634390654, 30.87883)(17.128547579298832, 30.76347)(17.228714524207014, 30.80558)(17.328881469115192, 31.02042)(17.429048414023374, 30.929479999999998)(17.529215358931552, 30.32524)(17.629382303839733, 30.94535)(17.72954924874791, 30.892870000000002)(17.829716193656097, 30.84282)(17.929883138564275, 30.719530000000002)(18.030050083472457, 30.30754)(18.130217028380635, 30.97343)(18.230383973288816, 30.92155)(18.330550918196995, 30.53092)(18.430717863105176, 30.9893)(18.530884808013354, 30.714640000000003)(18.63105175292154, 30.754920000000002)(18.731218697829718, 30.75554)(18.8313856427379, 30.969769999999997)(18.931552587646078, 30.7537)(19.03171953255426, 30.80741)(19.131886477462437, 31.13396)(19.23205342237062, 30.79888)(19.332220367278797, 30.70732)(19.43238731218698, 31.082679999999996)(19.53255425709516, 30.902630000000002)(19.63272120200334, 30.82939)(19.73288814691152, 30.95146)(19.833055091819702, 30.94719)(19.93322203672788, 30.93254)(20.033388981636058, 31.15959)(20.13355592654424, 30.63041)(20.23372287145242, 30.207440000000002)(20.333889816360603, 30.96244)(20.43405676126878, 30.896530000000002)(20.534223706176963, 30.728070000000002)(20.63439065108514, 31.01127)(20.734557595993323, 30.660330000000002)(20.8347245409015, 30.95268)(20.934891485809683, 30.85929)(21.03505843071786, 30.913)(21.135225375626046, 31.091839999999998)(21.235392320534224, 30.09452)(21.335559265442406, 31.0125)(21.435726210350584, 30.60112)(21.535893155258766, 30.91057)(21.636060100166944, 30.991749999999996)(21.736227045075125, 30.99601)(21.836393989983303, 30.988690000000002)(21.93656093489149, 30.67802)(22.036727879799667, 30.92705)(22.13689482470785, 31.01982)(22.237061769616027, 30.97953)(22.33722871452421, 30.905070000000002)(22.437395659432386, 31.19621)(22.537562604340568, 30.81047)(22.637729549248746, 30.82939)(22.737896494156928, 30.9539)(22.83806343906511, 30.914830000000002)(22.93823038397329, 30.68657)(23.03839732888147, 31.1132)(23.13856427378965, 30.844649999999998)(23.23873121869783, 30.045090000000002)(23.33889816360601, 31.09306)(23.43906510851419, 31.04179)(23.53923205342237, 31.13518)(23.639398998330552, 30.84953)(23.739565943238734, 30.77567)(23.839732888146912, 30.7653)(23.939899833055094, 30.876389999999997)(24.040066777963272, 31.00334)(24.140233722871454, 30.99907)(24.24040066777963, 30.56877)(24.340567612687813, 30.986250000000002)(24.440734557595995, 30.98502)(24.540901502504177, 30.391759999999998)(24.641068447412355, 31.021649999999998)(24.741235392320537, 30.91911)(24.841402337228715, 31.032030000000002)(24.941569282136896, 30.76164)(25.041736227045075, 30.79643)(25.141903171953256, 31.08696)(25.242070116861438, 30.83732)(25.34223706176962, 30.97526)(25.442404006677798, 30.972209999999997)(25.54257095158598, 30.64018)(25.642737896494157, 30.97954)(25.74290484140234, 30.99357)(25.843071786310517, 31.09734)(25.9432387312187, 30.84342)(26.043405676126877, 31.059489999999997)(26.143572621035062, 30.960009999999997)(26.24373956594324, 30.95329)(26.34390651085142, 30.45097)(26.4440734557596, 31.06437)(26.544240400667782, 31.06864)(26.64440734557596, 30.892249999999997)(26.744574290484138, 31.03935)(26.84474123539232, 31.142490000000002)(26.9449081803005, 30.65666)(27.045075125208683, 30.93009)(27.14524207011686, 30.74333)(27.245409015025043, 31.000899999999998)(27.34557595993322, 30.78911)(27.445742904841403, 31.0479)(27.54590984974958, 31.163249999999998)(27.646076794657763, 30.563889999999997)(27.746243739565944, 30.895310000000002)(27.846410684474126, 30.98564)(27.946577629382304, 31.027140000000003)(28.046744574290486, 30.848309999999998)(28.146911519198664, 31.05217)(28.247078464106846, 30.65849)(28.347245409015024, 30.955119999999997)(28.447412353923205, 30.938029999999998)(28.547579298831387, 30.57853)(28.64774624373957, 30.858069999999998)(28.747913188647747, 31.0662)(28.84808013355593, 30.624319999999997)(28.948247078464107, 30.584030000000002)(29.04841402337229, 31.01493)(29.148580968280466, 31.110760000000003)(29.248747913188648, 31.02592)(29.348914858096826, 30.32097)(29.44908180300501, 30.56571)(29.54924874791319, 30.966109999999997)(29.64941569282137, 31.0778)(29.74958263772955, 30.76653)(29.84974958263773, 31.09123)(29.94991652754591, 30.88066)(30.05008347245409, 31.080849999999998)(30.15025041736227, 30.92155)(30.25041736227045, 31.11869)(30.35058430717863, 30.83427)(30.45075125208681, 30.546190000000003)(30.55091819699499, 30.92399)(30.651085141903177, 30.981969999999997)(30.751252086811355, 30.92277)(30.851419031719537, 31.00761)(30.951585976627715, 30.615160000000003)(31.051752921535897, 30.775070000000003)(31.151919866444075, 31.1132)(31.252086811352257, 30.847699999999996)(31.352253756260435, 30.98319)(31.452420701168617, 31.05827)(31.552587646076795, 31.128459999999997)(31.652754590984976, 30.97648)(31.752921535893154, 30.9008)(31.853088480801336, 31.04789)(31.953255425709514, 30.876990000000003)(32.053422370617696, 31.016149999999996)(32.15358931552588, 31.04118)(32.25375626043405, 31.19743)(32.35392320534224, 30.99906)(32.45409015025042, 31.098560000000003)(32.554257095158604, 30.94657)(32.65442404006678, 30.22209)(32.75459098497496, 31.263350000000003)(32.85475792988314, 31.18095)(32.95492487479132, 30.792160000000003)(33.0550918196995, 31.10954)(33.15525876460768, 30.94291)(33.25542570951586, 30.85808)(33.35559265442404, 31.05582)(33.45575959933222, 31.18522)(33.5559265442404, 31.00029)(33.65609348914858, 30.659100000000002)(33.756260434056756, 31.10222)(33.85642737896494, 30.70183)(33.95659432387313, 30.848309999999998)(34.05676126878131, 30.746389999999998)(34.15692821368948, 31.21452)(34.257095158597664, 31.055220000000002)(34.357262103505846, 30.73112)(34.45742904841403, 31.25725)(34.5575959933222, 31.211470000000002)(34.657762938230384, 31.085120000000003)(34.757929883138566, 30.91239)(34.85809682804675, 31.16386)(34.95826377295492, 31.18461)(35.058430717863104, 30.890430000000002)(35.158597662771285, 30.95817)(35.25876460767947, 31.128459999999997)(35.35893155258764, 30.886770000000002)(35.45909849749582, 30.86417)(35.559265442404005, 31.07597)(35.659432387312194, 31.24931)(35.75959933222037, 30.89164)(35.85976627712855, 31.11748)(35.95993322203673, 30.824509999999997)(36.06010016694491, 30.90752)(36.16026711185309, 31.10709)(36.26043405676127, 31.2426)(36.36060100166945, 31.16631)(36.46076794657763, 31.05583)(36.56093489148581, 31.073529999999998)(36.66110183639399, 31.12479)(36.76126878130217, 31.03141)(36.86143572621035, 30.97526)(36.96160267111853, 31.11015)(37.06176961602671, 30.828780000000002)(37.16193656093489, 30.97282)(37.26210350584308, 31.0424)(37.362270450751254, 30.99724)(37.462437395659435, 31.21452)(37.56260434056762, 31.08024)(37.6627712854758, 31.13151)(37.76293823038397, 31.058880000000002)(37.863105175292155, 30.83976)(37.96327212020034, 31.310959999999998)(38.06343906510852, 30.94596)(38.16360601001669, 31.21086)(38.263772954924875, 30.89164)(38.363939899833056, 31.00273)(38.46410684474124, 30.920939999999998)(38.56427378964941, 30.760420000000003)(38.664440734557594, 30.94963)(38.764607679465776, 30.436320000000002)(38.86477462437396, 30.81474)(38.96494156928214, 31.07658)(39.06510851419032, 30.75737)(39.1652754590985, 31.038739999999997)(39.26544240400668, 31.24076)(39.36560934891486, 31.229779999999998)(39.46577629382304, 31.015549999999998)(39.56594323873122, 30.76225)(39.666110183639404, 31.23466)(39.76627712854758, 31.14433)(39.86644407345576, 31.16142)(39.96661101836394, 31.317059999999998)(40.066777963272116, 31.05583)(40.1669449081803, 31.0601)(40.26711185308848, 31.116860000000003)(40.36727879799666, 31.02836)(40.46744574290484, 30.995400000000004)(40.567612687813025, 31.215739999999997)(40.667779632721206, 31.071089999999998)(40.76794657762939, 31.151650000000004)(40.86811352253756, 30.9478)(40.968280467445744, 31.15409)(41.068447412353926, 31.19865)(41.16861435726211, 31.20353)(41.26878130217028, 31.25603)(41.368948247078464, 31.17362)(41.469115191986646, 30.92155)(41.56928213689483, 30.94169)(41.669449081803, 30.82389)(41.769616026711184, 30.8831)(41.869782971619365, 31.0778)(41.96994991652755, 31.1132)(42.07011686143572, 31.070479999999996)(42.1702838063439, 30.98747)(42.27045075125209, 31.1248)(42.370617696160274, 31.23405)(42.47078464106845, 30.46929)(42.57095158597663, 31.08574)(42.67111853088481, 31.30851)(42.77128547579299, 31.139450000000004)(42.87145242070117, 31.25114)(42.97161936560935, 31.18217)(43.07178631051753, 31.35185)(43.17195325542571, 30.77995)(43.27212020033389, 31.408610000000003)(43.37228714524207, 31.293259999999997)(43.47245409015025, 31.1834)(43.57262103505843, 31.15409)(43.67278797996661, 31.28105)(43.77295492487479, 30.94536)(43.87312186978298, 31.169359999999998)(43.97328881469116, 31.04667)(44.073455759599334, 31.02409)(44.173622704507515, 31.2194)(44.2737896494157, 31.16203)(44.37395659432388, 31.233439999999998)(44.47412353923205, 31.16752)(44.574290484140235, 31.28226)(44.67445742904842, 31.13823)(44.7746243739566, 31.14922)(44.87479131886477, 31.31522)(44.974958263772955, 31.286540000000002)(45.075125208681136, 31.10161)(45.17529215358932, 31.11564)(45.27545909849749, 31.360999999999997)(45.375626043405674, 31.24687)(45.475792988313856, 31.08513)(45.57595993322204, 31.17912)(45.67612687813022, 31.15959)(45.7762938230384, 30.938029999999998)(45.87646076794658, 31.10039)(45.976627712854764, 31.35612)(46.07679465776294, 30.60906)(46.17696160267112, 30.92399)(46.2771285475793, 31.207810000000002)(46.37729549248748, 31.25541)(46.47746243739566, 30.679249999999996)(46.57762938230384, 31.24748)(46.67779632721202, 31.229170000000003)(46.7779632721202, 31.28593)(46.87813021702838, 31.16997)(46.97829716193656, 31.12297)(47.07846410684474, 30.97343)(47.17863105175292, 31.24687)(47.278797996661105, 31.46293)(47.378964941569286, 30.848309999999998)(47.47913188647747, 31.052780000000002)(47.57929883138564, 31.0833)(47.679465776293824, 31.08451)(47.779632721202006, 31.05766)(47.87979966611019, 30.877)(47.97996661101836, 31.050330000000002)(48.080133555926544, 31.29021)(48.180300500834726, 31.239539999999998)(48.28046744574291, 31.10954)(48.38063439065108, 30.876990000000003)(48.48080133555926, 31.17851)(48.580968280467445, 31.071089999999998)(48.68113522537563, 31.09061)(48.7813021702838, 31.36101)(48.88146911519199, 31.08024)(48.98163606010017, 31.24077)(49.081803005008354, 30.77019)(49.18196994991653, 31.19377)(49.28213689482471, 31.14799)(49.38230383973289, 31.31217)(49.48247078464107, 31.22916)(49.58263772954925, 30.983200000000004)(49.68280467445743, 31.15776)(49.78297161936561, 31.010659999999998)(49.88313856427379, 31.09489)(49.98330550918197, 30.89409)(50.08347245409015, 31.19804)(50.18363939899833, 31.18096)(50.28380634390651, 31.12114)(50.38397328881469, 31.3372)(50.484140233722876, 31.10771)(50.58430717863106, 31.19377)(50.68447412353924, 31.25785)(50.784641068447414, 31.31583)(50.884808013355595, 31.21025)(50.98497495826378, 31.562420000000003)(51.08514190317196, 31.31218)(51.18530884808013, 31.17851)(51.285475792988315, 31.29325)(51.3856427378965, 31.278000000000002)(51.48580968280468, 31.296300000000002)(51.58597662771285, 30.92765)(51.686143572621035, 31.248089999999998)(51.786310517529216, 31.326819999999998)(51.8864774624374, 30.73295)(51.98664440734557, 31.23466)(52.086811352253754, 31.09001)(52.18697829716194, 31.14555)(52.287145242070125, 31.23161)(52.3873121869783, 31.45439)(52.48747913188648, 31.19743)(52.58764607679466, 31.09673)(52.68781302170284, 31.235879999999998)(52.78797996661102, 31.486130000000003)(52.8881469115192, 31.1895)(52.98831385642738, 31.35123)(53.088480801335564, 31.12052)(53.18864774624374, 31.16814)(53.28881469115192, 31.16996)(53.3889816360601, 31.28166)(53.489148580968276, 31.15593)(53.58931552587646, 31.18766)(53.68948247078464, 31.267010000000003)(53.78964941569283, 30.96367)(53.889816360601, 31.33293)(53.989983305509185, 31.127850000000002)(54.090150250417366, 31.36284)(54.19031719532555, 30.83672)(54.29048414023372, 31.111369999999997)(54.390651085141904, 30.845869999999998)(54.490818030050086, 31.250529999999998)(54.59098497495827, 30.98198)(54.69115191986644, 31.27677)(54.791318864774624, 31.14189)(54.891485809682806, 31.160199999999996)(54.99165275459099, 30.992350000000002)(55.09181969949916, 31.25358)(55.19198664440734, 31.32072)(55.292153589315525, 31.25785)(55.39232053422371, 31.317050000000002)(55.49248747913189, 30.9539)(55.59265442404007, 30.983800000000002)(55.69282136894825, 31.42998)(55.79298831385643, 31.122359999999997)(55.89315525876461, 31.229170000000003)(55.99332220367279, 31.26518)(56.09348914858097, 31.27738)(56.19365609348915, 31.01799)(56.29382303839733, 31.033849999999997)(56.39398998330551, 31.13518)(56.49415692821369, 31.11869)(56.59432387312187, 30.58586)(56.69449081803005, 31.06681)(56.79465776293823, 31.23283)(56.89482470784641, 31.09245)(56.99499165275459, 31.31034)(57.095158597662774, 31.26823)(57.195325542570956, 31.196820000000002)(57.29549248747914, 30.9185)(57.39565943238732, 31.337809999999998)(57.495826377295494, 31.06071)(57.595993322203675, 31.24077)(57.69616026711186, 31.37627)(57.79632721202004, 31.23405)(57.89649415692821, 31.21452)(57.996661101836395, 31.33842)(58.09682804674458, 31.18523)(58.19699499165276, 31.241370000000003)(58.29716193656093, 31.545939999999998)(58.397328881469114, 30.95818)(58.497495826377296, 30.96123)(58.59766277128548, 31.16996)(58.69782971619365, 31.05338)(58.79799666110184, 31.42814)(58.89816360601002, 31.23527)(58.9983305509182, 31.149209999999997)(59.09849749582638, 31.06437)(59.19866444073456, 31.21452)(59.29883138564274, 31.033849999999997)(59.398998330550924, 31.49772)(59.4991652754591, 31.48978)(59.59933222036728, 31.16447)(59.69949916527546, 31.304850000000002)(59.79966611018364, 31.229170000000003)(59.89983305509182, 31.48247)(60.0, 31.271279999999997)
        };
        \addplot [pattern=north east lines,pattern color=red] 
        fill between [
            of=A and B,soft clip={domain=0:800},
        ];
        \end{axis}
\end{tikzpicture}
\caption{Test case: Nbody}
\end{subfigure}
\begin{subfigure}[b]{0.49\linewidth}
    \begin{tikzpicture}
        \pgfplotsset{%
        width=1\linewidth,
        % height=1\textheight
            }
        \begin{axis}[ymax=120,
        xlabel={Time (Seconds)},
        ylabel={Energy Consumption (Joules)},
        ]
        \addplot[color=blue, mark=none,] coordinates { %% AVG value
        (0.0, 0.0)(0.1001669449081803, 6.289904065040652)(0.2003338898163606, 5.276476016260164)(0.3005008347245409, 4.359337967479674)(0.4006677796327212, 5.02925918699187)(0.5008347245409015, 4.355626666666667)(0.6010016694490818, 4.722114227642278)(0.7011686143572621, 5.166488780487808)(0.8013355592654424, 4.507693089430896)(0.9015025041736228, 4.95206138211382)(1.001669449081803, 5.076336178861786)(1.1018363939899833, 3.8781932520325197)(1.2020033388981637, 3.8946436585365842)(1.3021702838063438, 3.8502411382113837)(1.4023372287145242, 3.8076462601626013)(1.5025041736227045, 3.9652106504065037)(1.6026711185308848, 4.125519756097563)(1.7028380634390652, 4.03147056910569)(1.8030050083472455, 3.917196666666665)(1.9031719532554257, 4.026275447154473)(2.003338898163606, 3.811874390243904)(2.1035058430717863, 4.115664634146341)(2.2036727879799667, 4.012793414634147)(2.303839732888147, 3.7195321951219524)(2.4040066777963274, 4.101170081300812)(2.5041736227045073, 4.237912682926829)(2.6043405676126876, 4.060598780487805)(2.704507512520868, 3.746378536585364)(2.8046744574290483, 3.896622845528454)(2.9048414023372287, 3.954794878048782)(3.005008347245409, 4.02339300813008)(3.1051752921535893, 4.253732195121949)(3.2053422370617697, 3.9953561788617886)(3.30550918196995, 4.1309824390243906)(3.4056761268781304, 4.23935731707317)(3.5058430717863107, 4.070052032520326)(3.606010016694491, 4.297295447154469)(3.7061769616026714, 4.075833170731708)(3.8063439065108513, 3.902518617886178)(3.906510851419032, 4.068672601626016)(4.006677796327212, 4.034755772357722)(4.106844741235393, 4.120378130081301)(4.207011686143573, 3.9729321138211375)(4.3071786310517535, 4.07109382113821)(4.407345575959933, 3.9425586991869914)(4.507512520868114, 4.13743406504065)(4.607679465776294, 3.767476829268292)(4.707846410684475, 3.954432601626016)(4.808013355592655, 4.1617781300813)(4.908180300500835, 3.950601951219513)(5.0083472454090145, 4.150350569105693)(5.108514190317195, 4.513160894308944)(5.208681135225375, 4.126482357723575)(5.308848080133556, 3.670133333333334)(5.409015025041736, 3.992721219512195)(5.509181969949917, 3.9360530894308927)(5.609348914858097, 3.6971180487804896)(5.709515859766277, 3.8833395121951213)(5.809682804674457, 3.6276924390243925)(5.909849749582638, 3.7952008943089433)(6.010016694490818, 3.7234276422764214)(6.110183639398999, 3.7091514634146328)(6.210350584307179, 3.9266193495934947)(6.3105175292153595, 3.6848121138211374)(6.410684474123539, 3.620278455284554)(6.510851419031719, 3.8552338211382122)(6.6110183639399, 3.9977129268292666)(6.71118530884808, 4.243277235772357)(6.811352253756261, 4.160666666666666)(6.911519198664441, 4.098783577235771)(7.011686143572621, 3.8902324390243925)(7.111853088480801, 3.9384835772357727)(7.212020033388982, 3.963241138211381)(7.312186978297162, 3.796966747967478)(7.412353923205343, 3.902086829268292)(7.512520868113523, 3.646732439024389)(7.612687813021703, 3.892038536585365)(7.712854757929884, 3.975010650406504)(7.813021702838064, 4.137433495934959)(7.913188647746244, 3.962198699186993)(8.013355592654424, 4.072562764227641)(8.113522537562606, 3.8397224390243885)(8.213689482470786, 3.5617591869918703)(8.313856427378965, 3.846137642276423)(8.414023372287145, 3.908820894308946)(8.514190317195327, 3.863128455284552)(8.614357262103507, 3.9315910569105688)(8.714524207011687, 4.047895853658535)(8.814691151919867, 4.189090650406505)(8.914858096828048, 4.0736144715447145)(9.015025041736228, 3.9983931707317075)(9.115191986644408, 3.854032357723576)(9.215358931552588, 4.025417398373984)(9.31552587646077, 4.157688780487805)(9.41569282136895, 4.226202195121952)(9.51585976627713, 3.922005284552846)(9.61602671118531, 3.94366)(9.71619365609349, 3.899207967479674)(9.81636060100167, 4.2256070731707345)(9.916527545909851, 4.04484487804878)(10.016694490818029, 4.1508064227642265)(10.11686143572621, 3.9975144715447164)(10.21702838063439, 4.1699365853658525)(10.31719532554257, 4.262664227642277)(10.41736227045075, 3.994095365853658)(10.51752921535893, 4.075475853658537)(10.617696160267112, 4.052545121951222)(10.717863105175292, 3.9949391869918696)(10.818030050083472, 3.792620894308943)(10.918196994991652, 3.93253463414634)(11.018363939899833, 3.9741280487804866)(11.118530884808013, 4.093369105691057)(11.218697829716193, 3.8413737398373975)(11.318864774624373, 3.7539603252032534)(11.419031719532555, 3.770499349593496)(11.519198664440735, 3.927081300813007)(11.619365609348915, 3.859371951219513)(11.719532554257095, 3.929135609756097)(11.819699499165276, 3.9123985365853664)(11.919866444073456, 3.8504494308943085)(12.020033388981636, 3.680023658536586)(12.120200333889816, 3.9552018699187)(12.220367278797998, 3.9363155284552844)(12.320534223706177, 3.749087642276424)(12.420701168614357, 4.026066991869919)(12.520868113522537, 3.743276504065039)(12.621035058430719, 3.6525476422764225)(12.721202003338899, 3.8514973170731714)(12.821368948247079, 3.8172932520325196)(12.921535893155259, 3.9767073983739842)(13.021702838063439, 3.9365142276422778)(13.12186978297162, 3.8602012195121955)(13.2220367278798, 4.030711463414633)(13.32220367278798, 4.083643658536584)(13.42237061769616, 3.789970894308943)(13.522537562604342, 3.7287127642276423)(13.622704507512521, 3.5047283739837396)(13.722871452420701, 3.7794014634146347)(13.823038397328881, 3.8652322764227627)(13.923205342237063, 3.9717013821138227)(14.023372287145243, 4.009751219512195)(14.123539232053423, 4.007116504065042)(14.223706176961603, 4.0917121138211385)(14.323873121869784, 3.792773902439026)(14.424040066777964, 4.081485284552846)(14.524207011686144, 3.8590690243902444)(14.624373956594324, 4.028563495934961)(14.724540901502506, 4.118587479674795)(14.824707846410686, 3.6635089430894294)(14.924874791318866, 3.819337154471544)(15.025041736227045, 3.9600650406504077)(15.125208681135225, 3.8625128455284536)(15.225375626043405, 3.740860569105691)(15.325542570951589, 3.7007944715447167)(15.425709515859769, 3.825282032520326)(15.525876460767948, 3.9906369918699194)(15.626043405676128, 3.805839756097562)(15.726210350584308, 3.712431544715447)(15.826377295492488, 4.052580081300814)(15.926544240400668, 3.7363249593495946)(16.026711185308848, 3.963066666666667)(16.126878130217026, 3.8688252032520336)(16.22704507512521, 3.953629268292682)(16.32721202003339, 3.9154199999999992)(16.42737896494157, 4.039256666666666)(16.52754590984975, 4.289356422764228)(16.62771285475793, 4.12466569105691)(16.72787979966611, 4.109858699186992)(16.82804674457429, 4.168069674796749)(16.92821368948247, 4.219329674796749)(17.028380634390654, 4.113699349593497)(17.128547579298832, 3.9640895121951205)(17.228714524207014, 4.137622520325203)(17.328881469115192, 4.123038373983739)(17.429048414023374, 4.087588536585367)(17.529215358931552, 4.292095040650406)(17.629382303839733, 4.209092520325204)(17.72954924874791, 3.8754248780487797)(17.829716193656097, 3.9971273983739857)(17.929883138564275, 3.942845609756096)(18.030050083472457, 3.971052195121952)(18.130217028380635, 4.004595609756097)(18.230383973288816, 4.114969674796749)(18.330550918196995, 3.9032925203252016)(18.430717863105176, 3.9210965040650407)(18.530884808013354, 4.243927317073171)(18.63105175292154, 4.4505431707317085)(18.731218697829718, 3.921320325203253)(18.8313856427379, 4.157405853658537)(18.931552587646078, 4.1263582926829265)(19.03171953255426, 4.197838455284553)(19.131886477462437, 4.0213384552845515)(19.23205342237062, 4.053314308943089)(19.332220367278797, 4.084164634146343)(19.43238731218698, 4.070096341463414)(19.53255425709516, 4.516893008130083)(19.63272120200334, 4.044719918699187)(19.73288814691152, 3.89327837398374)(19.833055091819702, 4.0554982113821145)(19.93322203672788, 4.149823902439024)(20.033388981636058, 4.437810487804876)(20.13355592654424, 4.098673902439025)(20.23372287145242, 4.270420162601628)(20.333889816360603, 4.341677642276423)(20.43405676126878, 4.386674552845529)(20.534223706176963, 4.413654796747968)(20.63439065108514, 4.047181056910569)(20.734557595993323, 4.007082276422766)(20.8347245409015, 4.327445609756098)(20.934891485809683, 4.413624552845526)(21.03505843071786, 4.370904796747966)(21.135225375626046, 4.278359918699185)(21.235392320534224, 4.3439406504065055)(21.335559265442406, 4.375816991869919)(21.435726210350584, 4.072483821138211)(21.535893155258766, 4.21720130081301)(21.636060100166944, 4.15681544715447)(21.736227045075125, 3.9608391869918704)(21.836393989983303, 4.306391463414633)(21.93656093489149, 4.202968373983739)(22.036727879799667, 4.146177886178861)(22.13689482470785, 4.1425547154471545)(22.237061769616027, 4.131915772357724)(22.33722871452421, 4.111316910569107)(22.437395659432386, 4.230316422764228)(22.537562604340568, 4.269656341463414)(22.637729549248746, 4.346292032520326)(22.737896494156928, 4.498185447154472)(22.83806343906511, 4.279670000000002)(22.93823038397329, 3.8864114634146354)(23.03839732888147, 3.940959999999999)(23.13856427378965, 4.0861742276422754)(23.23873121869783, 4.165783089430896)(23.33889816360601, 4.250367642276424)(23.43906510851419, 4.322940487804879)(23.53923205342237, 4.255926260162603)(23.639398998330552, 4.477106178861791)(23.739565943238734, 4.31622601626016)(23.839732888146912, 4.326140813008128)(23.939899833055094, 4.744965203252032)(24.040066777963272, 4.477686016260166)(24.140233722871454, 4.339321219512195)(24.24040066777963, 4.526573739837399)(24.340567612687813, 4.1921859349593475)(24.440734557595995, 4.0202418699187)(24.540901502504177, 3.9511473170731715)(24.641068447412355, 4.343876260162601)(24.741235392320537, 4.125088211382116)(24.841402337228715, 4.088222439024388)(24.941569282136896, 4.455689349593498)(25.041736227045075, 4.221309512195122)(25.141903171953256, 4.121113414634148)(25.242070116861438, 4.285827886178862)(25.34223706176962, 4.20026487804878)(25.442404006677798, 4.277248373983739)(25.54257095158598, 4.635856585365855)(25.642737896494157, 4.245877154471545)(25.74290484140234, 4.177354715447154)(25.843071786310517, 4.340655365853656)(25.9432387312187, 4.455297398373983)(26.043405676126877, 4.231229105691056)(26.143572621035062, 4.063159349593494)(26.24373956594324, 4.441089349593496)(26.34390651085142, 4.289664715447154)(26.4440734557596, 4.327847723577237)(26.544240400667782, 4.165698130081301)(26.64440734557596, 4.231323089430895)(26.744574290484138, 4.221285121951218)(26.84474123539232, 4.1097945528455275)(26.9449081803005, 3.9395110569105682)(27.045075125208683, 3.98679617886179)(27.14524207011686, 4.098322032520326)(27.245409015025043, 4.225289837398373)(27.34557595993322, 4.29516642276423)(27.445742904841403, 4.49634886178862)(27.54590984974958, 4.364954959349595)(27.646076794657763, 4.916930406504066)(27.746243739565944, 5.400595528455283)(27.846410684474126, 4.511265528455285)(27.946577629382304, 4.64542317073171)(28.046744574290486, 4.360107154471544)(28.146911519198664, 4.2669816260162605)(28.247078464106846, 4.392400975609757)(28.347245409015024, 4.216699593495936)(28.447412353923205, 4.367020000000001)(28.547579298831387, 4.267839918699188)(28.64774624373957, 4.7883492682926825)(28.747913188647747, 5.493244715447154)(28.84808013355593, 5.495303658536584)(28.948247078464107, 5.504245447154474)(29.04841402337229, 5.619591626016259)(29.148580968280466, 5.498211138211382)(29.248747913188648, 5.385871951219511)(29.348914858096826, 5.348174552845527)(29.44908180300501, 5.5169645528455264)(29.54924874791319, 5.323402926829271)(29.64941569282137, 5.179469999999998)(29.74958263772955, 4.473518130081299)(29.84974958263773, 4.501683495934959)(29.94991652754591, 4.232826260162601)(30.05008347245409, 4.229681382113822)(30.15025041736227, 4.354266747967478)(30.25041736227045, 4.498656260162601)(30.35058430717863, 4.425767235772358)(30.45075125208681, 4.272474390243903)(30.55091819699499, 4.534478780487806)(30.651085141903177, 5.399766585365854)(30.751252086811355, 5.081159024390242)(30.851419031719537, 4.459817073170732)(30.951585976627715, 4.494468699186991)(31.051752921535897, 4.510337560975611)(31.151919866444075, 4.51056593495935)(31.252086811352257, 4.44210268292683)(31.352253756260435, 4.340868455284552)(31.452420701168617, 4.575278048780486)(31.552587646076795, 4.616588130081302)(31.652754590984976, 4.614663089430893)(31.752921535893154, 4.168819186991871)(31.853088480801336, 4.259736747967483)(31.953255425709514, 4.606351626016259)(32.053422370617696, 4.3280316260162595)(32.15358931552588, 4.283704390243905)(32.25375626043405, 4.43089674796748)(32.35392320534224, 4.484882682926829)(32.45409015025042, 4.405674959349595)(32.554257095158604, 4.284442601626019)(32.65442404006678, 4.615303577235771)(32.75459098497496, 4.298293252032522)(32.85475792988314, 4.331757967479676)(32.95492487479132, 4.205649349593497)(33.0550918196995, 4.268177154471545)(33.15525876460768, 4.438058373983742)(33.25542570951586, 4.553488943089433)(33.35559265442404, 4.117922439024391)(33.45575959933222, 4.433209674796748)(33.5559265442404, 4.047885853658538)(33.65609348914858, 4.119813252032521)(33.756260434056756, 4.448999837398374)(33.85642737896494, 4.29623406504065)(33.95659432387313, 4.2624658536585365)(34.05676126878131, 4.431403902439025)(34.15692821368948, 4.4102796747967465)(34.257095158597664, 4.627629349593494)(34.357262103505846, 4.743173658536585)(34.45742904841403, 4.634045447154472)(34.5575959933222, 4.2097718699187)(34.657762938230384, 4.495907886178861)(34.757929883138566, 4.610871544715448)(34.85809682804675, 4.108807398373984)(34.95826377295492, 4.5590756097561)(35.058430717863104, 4.5191112195121965)(35.158597662771285, 4.393964146341462)(35.25876460767947, 4.300024552845527)(35.35893155258764, 4.293633902439025)(35.45909849749582, 4.165549674796746)(35.559265442404005, 4.360906097560978)(35.659432387312194, 4.438653333333335)(35.75959933222037, 4.411203333333335)(35.85976627712855, 4.414368780487804)(35.95993322203673, 4.344173333333335)(36.06010016694491, 4.3445359349593495)(36.16026711185309, 4.451877642276423)(36.26043405676127, 4.343528780487802)(36.36060100166945, 4.713787560975611)(36.46076794657763, 4.416512195121951)(36.56093489148581, 4.460597317073172)(36.66110183639399, 4.449242601626018)(36.76126878130217, 4.409789105691059)(36.86143572621035, 4.315606097560975)(36.96160267111853, 4.45497430894309)(37.06176961602671, 4.186752520325203)(37.16193656093489, 4.116518211382113)(37.26210350584308, 4.443844390243903)(37.362270450751254, 4.281511219512198)(37.462437395659435, 4.4208488617886195)(37.56260434056762, 4.472213170731708)(37.6627712854758, 4.144951544715448)(37.76293823038397, 4.256302601626018)(37.863105175292155, 4.595811626016262)(37.96327212020034, 4.446970406504062)(38.06343906510852, 4.390257479674796)(38.16360601001669, 4.814987073170735)(38.263772954924875, 4.274270325203252)(38.363939899833056, 4.40727300813008)(38.46410684474124, 4.514640325203254)(38.56427378964941, 4.302600000000002)(38.664440734557594, 4.5334715447154466)(38.764607679465776, 4.377038455284551)(38.86477462437396, 4.540745691056912)(38.96494156928214, 4.605289349593499)(39.06510851419032, 4.492939999999998)(39.1652754590985, 4.370716260162602)(39.26544240400668, 4.407302601626016)(39.36560934891486, 4.4170885365853625)(39.46577629382304, 4.335559024390244)(39.56594323873122, 4.503444878048782)(39.666110183639404, 4.413753821138211)(39.76627712854758, 4.588362926829267)(39.86644407345576, 4.436575121951219)(39.96661101836394, 4.26861349593496)(40.066777963272116, 4.651100650406503)(40.1669449081803, 4.5814109756097565)(40.26711185308848, 4.190579268292684)(40.36727879799666, 4.401277804878051)(40.46744574290484, 4.315799512195124)(40.567612687813025, 4.049379349593498)(40.667779632721206, 4.689860731707317)(40.76794657762939, 4.662240081300816)(40.86811352253756, 4.537739105691055)(40.968280467445744, 4.445804146341464)(41.068447412353926, 4.424272601626016)(41.16861435726211, 4.367908292682925)(41.26878130217028, 4.393983983739837)(41.368948247078464, 4.5197960162601625)(41.469115191986646, 4.357978130081301)(41.56928213689483, 4.2920307317073165)(41.669449081803, 4.109878861788617)(41.769616026711184, 4.41646756097561)(41.869782971619365, 4.282186260162603)(41.96994991652755, 4.6476611382113795)(42.07011686143572, 4.370289593495936)(42.1702838063439, 4.506159674796747)(42.27045075125209, 4.647542032520326)(42.370617696160274, 4.239471463414634)(42.47078464106845, 4.326368292682926)(42.57095158597663, 4.367079756097561)(42.67111853088481, 4.3641266666666665)(42.77128547579299, 4.3577748780487795)(42.87145242070117, 4.284120975609757)(42.97161936560935, 4.296219349593497)(43.07178631051753, 4.58877487804878)(43.17195325542571, 4.2473260975609755)(43.27212020033389, 4.102629430894309)(43.37228714524207, 4.178123008130078)(43.47245409015025, 4.387364715447156)(43.57262103505843, 4.550228455284553)(43.67278797996661, 4.641046991869918)(43.77295492487479, 4.672427967479674)(43.87312186978298, 4.48230593495935)(43.97328881469116, 4.352505203252032)(44.073455759599334, 4.354311382113822)(44.173622704507515, 4.614325121951221)(44.2737896494157, 4.530955609756096)(44.37395659432388, 4.558714146341463)(44.47412353923205, 4.6929513821138205)(44.574290484140235, 4.704289512195121)(44.67445742904842, 4.699775040650406)(44.7746243739566, 4.629573739837397)(44.87479131886477, 4.417028699186994)(44.974958263772955, 4.592957967479674)(45.075125208681136, 4.335282032520326)(45.17529215358932, 4.251315609756096)(45.27545909849749, 4.610147235772357)(45.375626043405674, 4.588006097560974)(45.475792988313856, 4.569982845528457)(45.57595993322204, 4.729002032520329)(45.67612687813022, 4.336720162601626)(45.7762938230384, 4.761057317073171)(45.87646076794658, 4.640813170731706)(45.976627712854764, 4.609592439024389)(46.07679465776294, 4.732509756097562)(46.17696160267112, 4.693784878048777)(46.2771285475793, 4.649750487804877)(46.37729549248748, 4.465464552845532)(46.47746243739566, 4.659198455284554)(46.57762938230384, 4.464660569105691)(46.67779632721202, 4.457738292682926)(46.7779632721202, 4.312971788617887)(46.87813021702838, 4.5196760162601635)(46.97829716193656, 4.655586178861788)(47.07846410684474, 4.469275528455288)(47.17863105175292, 4.4545328455284565)(47.278797996661105, 4.751014146341463)(47.378964941569286, 4.446096910569106)(47.47913188647747, 4.170725365853657)(47.57929883138564, 4.360503902439024)(47.679465776293824, 4.461037804878051)(47.779632721202006, 4.685329430894309)(47.87979966611019, 4.6058603252032535)(47.97996661101836, 4.411326422764229)(48.080133555926544, 4.529502276422763)(48.180300500834726, 4.489391788617886)(48.28046744574291, 4.41243325203252)(48.38063439065108, 4.330920406504065)(48.48080133555926, 4.351596829268292)(48.580968280467445, 4.373797642276423)(48.68113522537563, 4.848139512195124)(48.7813021702838, 4.700051869918699)(48.88146911519199, 4.574538292682926)(48.98163606010017, 4.8461893495935)(49.081803005008354, 4.527561300813006)(49.18196994991653, 4.366413414634148)(49.28213689482471, 4.452657073170732)(49.38230383973289, 4.302094552845528)(49.48247078464107, 4.5168234959349585)(49.58263772954925, 4.494850731707318)(49.68280467445743, 4.716645691056911)(49.78297161936561, 4.284235447154471)(49.88313856427379, 4.579142601626018)(49.98330550918197, 4.369471544715449)(50.08347245409015, 4.687006666666666)(50.18363939899833, 4.718426991869919)(50.28380634390651, 4.437239105691056)(50.38397328881469, 4.427926097560977)(50.484140233722876, 4.609244227642276)(50.58430717863106, 4.641399024390245)(50.68447412353924, 4.525601544715447)(50.784641068447414, 4.378695284552845)(50.884808013355595, 4.358514065040651)(50.98497495826378, 4.351736666666668)(51.08514190317196, 4.827426341463415)(51.18530884808013, 4.912092032520325)(51.285475792988315, 4.457029268292683)(51.3856427378965, 4.223239268292682)(51.48580968280468, 4.375827317073171)(51.58597662771285, 4.416001626016258)(51.686143572621035, 4.371370975609757)(51.786310517529216, 4.4452982926829305)(51.8864774624374, 4.5243504878048775)(51.98664440734557, 4.340110162601624)(52.086811352253754, 4.812594796747968)(52.18697829716194, 5.023746422764228)(52.287145242070125, 4.645189674796747)(52.3873121869783, 4.6556213821138215)(52.48747913188648, 4.514341626016259)(52.58764607679466, 4.411986991869919)(52.68781302170284, 4.178208048780489)(52.78797996661102, 4.467330162601625)(52.8881469115192, 4.212546585365855)(52.98831385642738, 4.148117398373984)(53.088480801335564, 4.472192439024387)(53.18864774624374, 4.415093658536585)(53.28881469115192, 4.479646585365852)(53.3889816360601, 4.375663414634146)(53.489148580968276, 4.560416666666666)(53.58931552587646, 4.526906341463414)(53.68948247078464, 4.354519349593495)(53.78964941569283, 4.568082682926829)(53.889816360601, 4.1385456097560995)(53.989983305509185, 4.1975056097560985)(54.090150250417366, 4.201034390243904)(54.19031719532555, 4.519234715447154)(54.29048414023372, 4.42623845528455)(54.390651085141904, 4.445590487804879)(54.490818030050086, 4.5194634146341475)(54.59098497495827, 4.376398048780489)(54.69115191986644, 3.9901756097560965)(54.791318864774624, 4.3544304065040675)(54.891485809682806, 4.5259537398374)(54.99165275459099, 4.640143577235772)(55.09181969949916, 4.630060813008132)(55.19198664440734, 4.337712845528453)(55.292153589315525, 4.593091788617885)(55.39232053422371, 4.364722357723579)(55.49248747913189, 4.366254878048781)(55.59265442404007, 4.601002276422765)(55.69282136894825, 4.344793577235772)(55.79298831385643, 4.256397235772358)(55.89315525876461, 4.477964552845532)(55.99332220367279, 4.406140975609753)(56.09348914858097, 4.436262195121951)(56.19365609348915, 4.5074354471544735)(56.29382303839733, 4.368389268292681)(56.39398998330551, 4.583911300813008)(56.49415692821369, 4.433155609756098)(56.59432387312187, 4.291063170731708)(56.69449081803005, 4.427022845528454)(56.79465776293823, 4.286562032520327)(56.89482470784641, 4.315729918699189)(56.99499165275459, 4.468625691056912)(57.095158597662774, 4.409937154471545)(57.195325542570956, 4.34485430894309)(57.29549248747914, 4.759488780487807)(57.39565943238732, 4.3828740650406495)(57.495826377295494, 4.408786178861788)(57.595993322203675, 4.478534715447155)(57.69616026711186, 4.528182032520326)(57.79632721202004, 4.2954646341463425)(57.89649415692821, 4.45719243902439)(57.996661101836395, 4.2712091056910575)(58.09682804674458, 4.23826569105691)(58.19699499165276, 4.411455447154476)(58.29716193656093, 4.406231707317074)(58.397328881469114, 4.288135121951222)(58.497495826377296, 4.583946829268292)(58.59766277128548, 4.287043577235772)(58.69782971619365, 4.41298455284553)(58.79799666110184, 4.286915203252034)(58.89816360601002, 4.405634959349592)(58.9983305509182, 4.561468211382116)(59.09849749582638, 4.648216991869918)(59.19866444073456, 4.56699617886179)(59.29883138564274, 4.660061707317074)(59.398998330550924, 4.668929512195122)(59.4991652754591, 4.571615528455285)(59.59933222036728, 4.840924390243902)(59.69949916527546, 4.413932032520325)(59.79966611018364, 4.653328455284554)(59.89983305509182, 4.396564715447153)(60.0, 4.976278278688525)
        };
        \addplot[color=blue, mark=none,name path=A] coordinates { %% MAX value
        (0.0, 0.0)(0.1001669449081803, 18.70967)(0.2003338898163606, 16.0046)(0.3005008347245409, 8.815290000000001)(0.4006677796327212, 31.874920000000003)(0.5008347245409015, 10.27524)(0.6010016694490818, 23.88849)(0.7011686143572621, 33.32633)(0.8013355592654424, 33.981849999999994)(0.9015025041736228, 12.27536)(1.001669449081803, 10.384500000000001)(1.1018363939899833, 7.7014000000000005)(1.2020033388981637, 7.317489999999999)(1.3021702838063438, 8.77866)(1.4023372287145242, 7.83201)(1.5025041736227045, 8.992289999999999)(1.6026711185308848, 9.50071)(1.7028380634390652, 8.51011)(1.8030050083472455, 9.31272)(1.9031719532554257, 7.49327)(2.003338898163606, 8.565660000000001)(2.1035058430717863, 8.70847)(2.2036727879799667, 9.20225)(2.303839732888147, 8.06089)(2.4040066777963274, 9.94932)(2.5041736227045073, 15.01522)(2.6043405676126876, 14.89376)(2.704507512520868, 7.53722)(2.8046744574290483, 7.56162)(2.9048414023372287, 7.86008)(3.005008347245409, 8.46372)(3.1051752921535893, 11.17429)(3.2053422370617697, 9.67893)(3.30550918196995, 10.801369999999999)(3.4056761268781304, 9.51963)(3.5058430717863107, 9.224219999999999)(3.606010016694491, 8.40879)(3.7061769616026714, 8.08042)(3.8063439065108513, 9.02403)(3.906510851419032, 8.55833)(4.006677796327212, 7.27659)(4.106844741235393, 8.4509)(4.207011686143573, 9.038070000000001)(4.3071786310517535, 9.11741)(4.407345575959933, 7.91258)(4.507512520868114, 8.55528)(4.607679465776294, 9.212010000000001)(4.707846410684475, 8.73105)(4.808013355592655, 8.63584)(4.908180300500835, 8.40879)(5.0083472454090145, 8.13718)(5.108514190317195, 9.65329)(5.208681135225375, 7.38523)(5.308848080133556, 7.2888)(5.409015025041736, 7.92418)(5.509181969949917, 9.28647)(5.609348914858097, 8.46556)(5.709515859766277, 9.20651)(5.809682804674457, 7.92723)(5.909849749582638, 8.12864)(6.010016694490818, 8.18357)(6.110183639398999, 8.23545)(6.210350584307179, 7.44688)(6.3105175292153595, 8.523530000000001)(6.410684474123539, 7.04222)(6.510851419031719, 8.63218)(6.6110183639399, 8.64927)(6.71118530884808, 7.88145)(6.811352253756261, 8.847629999999999)(6.911519198664441, 8.66331)(7.011686143572621, 8.45762)(7.111853088480801, 9.13205)(7.212020033388982, 8.83786)(7.312186978297162, 7.64341)(7.412353923205343, 7.23203)(7.512520868113523, 7.06847)(7.612687813021703, 8.70969)(7.712854757929884, 9.1406)(7.813021702838064, 10.08481)(7.913188647746244, 9.3585)(8.013355592654424, 7.5451500000000005)(8.113522537562606, 7.22838)(8.213689482470786, 7.32847)(8.313856427378965, 8.42588)(8.414023372287145, 8.80491)(8.514190317195327, 7.79783)(8.614357262103507, 7.55369)(8.714524207011687, 8.787199999999999)(8.814691151919867, 9.19065)(8.914858096828048, 9.147920000000001)(9.015025041736228, 7.33275)(9.115191986644408, 8.787199999999999)(9.215358931552588, 9.061869999999999)(9.31552587646077, 9.1113)(9.41569282136895, 9.26877)(9.51585976627713, 8.4094)(9.61602671118531, 8.27634)(9.71619365609349, 7.94431)(9.81636060100167, 9.233369999999999)(9.916527545909851, 9.82175)(10.016694490818029, 8.954450000000001)(10.11686143572621, 9.99143)(10.21702838063439, 7.92357)(10.31719532554257, 9.41465)(10.41736227045075, 8.78111)(10.51752921535893, 9.00449)(10.617696160267112, 10.59019)(10.717863105175292, 8.186630000000001)(10.818030050083472, 8.68955)(10.918196994991652, 8.5864)(11.018363939899833, 8.029770000000001)(11.118530884808013, 8.8867)(11.218697829716193, 8.22508)(11.318864774624373, 9.48423)(11.419031719532555, 8.39415)(11.519198664440735, 17.6971)(11.619365609348915, 7.95103)(11.719532554257095, 8.0615)(11.819699499165276, 12.31686)(11.919866444073456, 13.00656)(12.020033388981636, 8.042589999999999)(12.120200333889816, 7.656230000000001)(12.220367278797998, 7.75694)(12.320534223706177, 7.924790000000001)(12.420701168614357, 8.07249)(12.520868113522537, 8.84092)(12.621035058430719, 8.18175)(12.721202003338899, 7.02757)(12.821368948247079, 7.2326500000000005)(12.921535893155259, 8.46251)(13.021702838063439, 9.25107)(13.12186978297162, 8.80125)(13.2220367278798, 8.32883)(13.32220367278798, 8.39231)(13.42237061769616, 8.40574)(13.522537562604342, 7.404159999999999)(13.622704507512521, 9.43174)(13.722871452420701, 8.55894)(13.823038397328881, 9.06125)(13.923205342237063, 7.965680000000001)(14.023372287145243, 8.95383)(14.123539232053423, 8.25438)(14.223706176961603, 8.92393)(14.323873121869784, 7.78806)(14.424040066777964, 7.645849999999999)(14.524207011686144, 9.2352)(14.624373956594324, 9.49644)(14.724540901502506, 9.42991)(14.824707846410686, 9.00815)(14.924874791318866, 8.39536)(15.025041736227045, 11.340300000000001)(15.125208681135225, 10.58774)(15.225375626043405, 7.38707)(15.325542570951589, 9.23521)(15.425709515859769, 8.746310000000001)(15.525876460767948, 16.70223)(15.626043405676128, 8.9929)(15.726210350584308, 7.08739)(15.826377295492488, 8.91233)(15.926544240400668, 7.39012)(16.026711185308848, 7.185040000000001)(16.126878130217026, 8.7805)(16.22704507512521, 7.44872)(16.32721202003339, 7.76182)(16.42737896494157, 8.899510000000001)(16.52754590984975, 9.15892)(16.62771285475793, 8.55161)(16.72787979966611, 7.70322)(16.82804674457429, 8.28)(16.92821368948247, 8.89158)(17.028380634390654, 9.89438)(17.128547579298832, 8.5333)(17.228714524207014, 8.76036)(17.328881469115192, 7.8338399999999995)(17.429048414023374, 7.95896)(17.529215358931552, 8.46006)(17.629382303839733, 7.971780000000001)(17.72954924874791, 9.06614)(17.829716193656097, 8.74082)(17.929883138564275, 7.43284)(18.030050083472457, 8.49058)(18.130217028380635, 7.46153)(18.230383973288816, 8.25743)(18.330550918196995, 7.85398)(18.430717863105176, 7.170999999999999)(18.530884808013354, 7.69224)(18.63105175292154, 8.14757)(18.731218697829718, 7.959580000000001)(18.8313856427379, 8.94285)(18.931552587646078, 8.38621)(19.03171953255426, 8.80552)(19.131886477462437, 9.39329)(19.23205342237062, 8.6157)(19.332220367278797, 7.19542)(19.43238731218698, 7.74229)(19.53255425709516, 9.040510000000001)(19.63272120200334, 8.90928)(19.73288814691152, 8.517430000000001)(19.833055091819702, 10.07321)(19.93322203672788, 9.05698)(20.033388981636058, 9.72776)(20.13355592654424, 7.503640000000001)(20.23372287145242, 9.54343)(20.333889816360603, 8.39903)(20.43405676126878, 9.087499999999999)(20.534223706176963, 8.72923)(20.63439065108514, 8.56809)(20.734557595993323, 8.94285)(20.8347245409015, 9.65696)(20.934891485809683, 8.59678)(21.03505843071786, 8.87571)(21.135225375626046, 8.39353)(21.235392320534224, 8.26475)(21.335559265442406, 8.731670000000001)(21.435726210350584, 7.5927500000000006)(21.535893155258766, 9.25108)(21.636060100166944, 9.34141)(21.736227045075125, 7.424300000000001)(21.836393989983303, 7.53721)(21.93656093489149, 9.35056)(22.036727879799667, 7.45238)(22.13689482470785, 9.23826)(22.237061769616027, 9.23703)(22.33722871452421, 7.9998499999999995)(22.437395659432386, 8.523539999999999)(22.537562604340568, 7.92112)(22.637729549248746, 7.6653899999999995)(22.737896494156928, 9.692960000000001)(22.83806343906511, 9.17112)(22.93823038397329, 7.54088)(23.03839732888147, 7.48838)(23.13856427378965, 8.02916)(23.23873121869783, 9.232759999999999)(23.33889816360601, 8.87083)(23.43906510851419, 8.82444)(23.53923205342237, 7.4358900000000006)(23.639398998330552, 9.69113)(23.739565943238734, 7.65501)(23.839732888146912, 9.41648)(23.939899833055094, 9.25169)(24.040066777963272, 8.24034)(24.140233722871454, 9.54831)(24.24040066777963, 8.95017)(24.340567612687813, 9.13083)(24.440734557595995, 7.65135)(24.540901502504177, 9.27121)(24.641068447412355, 9.82297)(24.741235392320537, 9.69663)(24.841402337228715, 7.88145)(24.941569282136896, 9.35361)(25.041736227045075, 7.7667)(25.141903171953256, 8.139619999999999)(25.242070116861438, 7.68675)(25.34223706176962, 7.6751499999999995)(25.442404006677798, 9.46592)(25.54257095158598, 9.41465)(25.642737896494157, 7.78806)(25.74290484140234, 8.64317)(25.843071786310517, 9.5532)(25.9432387312187, 9.67038)(26.043405676126877, 8.85313)(26.143572621035062, 8.446019999999999)(26.24373956594324, 8.106060000000001)(26.34390651085142, 9.40183)(26.4440734557596, 9.53001)(26.544240400667782, 9.64658)(26.64440734557596, 8.82261)(26.744574290484138, 9.53305)(26.84474123539232, 8.26353)(26.9449081803005, 7.38218)(27.045075125208683, 7.7587600000000005)(27.14524207011686, 8.71275)(27.245409015025043, 7.563459999999999)(27.34557595993322, 7.468240000000001)(27.445742904841403, 8.84885)(27.54590984974958, 9.36398)(27.646076794657763, 36.80045)(27.746243739565944, 37.91006)(27.846410684474126, 24.47747)(27.946577629382304, 9.65025)(28.046744574290486, 9.23216)(28.146911519198664, 9.66428)(28.247078464106846, 7.48533)(28.347245409015024, 8.76951)(28.447412353923205, 9.696019999999999)(28.547579298831387, 8.68406)(28.64774624373957, 26.895069999999997)(28.747913188647747, 31.353070000000002)(28.84808013355593, 33.17496)(28.948247078464107, 31.86394)(29.04841402337229, 31.071089999999998)(29.148580968280466, 31.07475)(29.248747913188648, 32.798379999999995)(29.348914858096826, 32.560950000000005)(29.44908180300501, 32.2686)(29.54924874791319, 33.035199999999996)(29.64941569282137, 31.8676)(29.74958263772955, 12.609829999999999)(29.84974958263773, 9.69968)(29.94991652754591, 9.45311)(30.05008347245409, 8.18052)(30.15025041736227, 9.37315)(30.25041736227045, 8.345320000000001)(30.35058430717863, 9.22239)(30.45075125208681, 7.88999)(30.55091819699499, 8.048680000000001)(30.651085141903177, 30.505290000000002)(30.751252086811355, 29.020310000000002)(30.851419031719537, 20.474800000000002)(30.951585976627715, 7.578110000000001)(31.051752921535897, 8.61753)(31.151919866444075, 9.689910000000001)(31.252086811352257, 9.3115)(31.352253756260435, 8.46738)(31.452420701168617, 9.74119)(31.552587646076795, 9.27)(31.652754590984976, 8.79759)(31.752921535893154, 7.653790000000001)(31.853088480801336, 9.0283)(31.953255425709514, 8.58214)(32.053422370617696, 9.33774)(32.15358931552588, 9.68869)(32.25375626043405, 8.01511)(32.35392320534224, 9.98594)(32.45409015025042, 9.84067)(32.554257095158604, 7.90159)(32.65442404006678, 8.51926)(32.75459098497496, 8.89585)(32.85475792988314, 10.91977)(32.95492487479132, 9.09605)(33.0550918196995, 7.76793)(33.15525876460768, 9.47812)(33.25542570951586, 8.42099)(33.35559265442404, 7.908309999999999)(33.45575959933222, 8.42344)(33.5559265442404, 9.218110000000001)(33.65609348914858, 7.67027)(33.756260434056756, 10.133030000000002)(33.85642737896494, 9.40428)(33.95659432387313, 7.86253)(34.05676126878131, 9.81321)(34.15692821368948, 26.039360000000002)(34.257095158597664, 31.732100000000003)(34.357262103505846, 31.190099999999997)(34.45742904841403, 16.74374)(34.5575959933222, 8.14695)(34.657762938230384, 8.16648)(34.757929883138566, 8.98191)(34.85809682804675, 9.24559)(34.95826377295492, 9.63682)(35.058430717863104, 10.42783)(35.158597662771285, 8.274519999999999)(35.25876460767947, 9.93833)(35.35893155258764, 8.11765)(35.45909849749582, 8.33616)(35.559265442404005, 8.84397)(35.659432387312194, 7.93088)(35.75959933222037, 9.01731)(35.85976627712855, 9.076509999999999)(35.95993322203673, 9.382909999999999)(36.06010016694491, 9.28647)(36.16026711185309, 7.8137)(36.26043405676127, 8.269020000000001)(36.36060100166945, 9.31639)(36.46076794657763, 8.84519)(36.56093489148581, 9.88096)(36.66110183639399, 9.2822)(36.76126878130217, 7.88573)(36.86143572621035, 8.86289)(36.96160267111853, 8.8519)(37.06176961602671, 8.99351)(37.16193656093489, 8.35141)(37.26210350584308, 8.900120000000001)(37.362270450751254, 9.50193)(37.462437395659435, 8.25437)(37.56260434056762, 9.71616)(37.6627712854758, 9.79612)(37.76293823038397, 7.69346)(37.863105175292155, 8.52964)(37.96327212020034, 7.51341)(38.06343906510852, 9.11741)(38.16360601001669, 9.915140000000001)(38.263772954924875, 7.98521)(38.363939899833056, 7.78135)(38.46410684474124, 9.406099999999999)(38.56427378964941, 10.36802)(38.664440734557594, 8.21836)(38.764607679465776, 8.11948)(38.86477462437396, 8.21775)(38.96494156928214, 8.88181)(39.06510851419032, 8.73289)(39.1652754590985, 8.10484)(39.26544240400668, 9.5239)(39.36560934891486, 7.96995)(39.46577629382304, 9.81321)(39.56594323873122, 7.72764)(39.666110183639404, 9.64109)(39.76627712854758, 9.464080000000001)(39.86644407345576, 7.91013)(39.96661101836394, 8.02306)(40.066777963272116, 8.81346)(40.1669449081803, 8.57969)(40.26711185308848, 8.7097)(40.36727879799666, 9.40122)(40.46744574290484, 9.66611)(40.567612687813025, 7.45787)(40.667779632721206, 9.22483)(40.76794657762939, 9.25534)(40.86811352253756, 10.76352)(40.968280467445744, 9.78146)(41.068447412353926, 8.87083)(41.16861435726211, 10.344819999999999)(41.26878130217028, 8.2031)(41.368948247078464, 8.5211)(41.469115191986646, 8.9929)(41.56928213689483, 9.86264)(41.669449081803, 7.65073)(41.769616026711184, 8.92149)(41.869782971619365, 7.7581500000000005)(41.96994991652755, 12.00925)(42.07011686143572, 8.9697)(42.1702838063439, 8.0853)(42.27045075125209, 8.16465)(42.370617696160274, 8.042580000000001)(42.47078464106845, 7.882059999999999)(42.57095158597663, 9.741800000000001)(42.67111853088481, 9.58616)(42.77128547579299, 8.77073)(42.87145242070117, 8.207980000000001)(42.97161936560935, 7.83201)(43.07178631051753, 8.2501)(43.17195325542571, 8.808580000000001)(43.27212020033389, 7.95897)(43.37228714524207, 9.6478)(43.47245409015025, 8.72679)(43.57262103505843, 9.868749999999999)(43.67278797996661, 9.70701)(43.77295492487479, 9.926120000000001)(43.87312186978298, 9.27548)(43.97328881469116, 8.038920000000001)(44.073455759599334, 9.20347)(44.173622704507515, 8.262920000000001)(44.2737896494157, 8.23301)(44.37395659432388, 9.39756)(44.47412353923205, 9.87546)(44.574290484140235, 8.30748)(44.67445742904842, 9.46287)(44.7746243739566, 8.0493)(44.87479131886477, 9.91086)(44.974958263772955, 9.307830000000001)(45.075125208681136, 9.3292)(45.17529215358932, 8.63829)(45.27545909849749, 29.84489)(45.375626043405674, 28.337940000000003)(45.475792988313856, 27.715390000000003)(45.57595993322204, 28.759079999999997)(45.67612687813022, 27.72637)(45.7762938230384, 28.26164)(45.87646076794658, 26.9793)(45.976627712854764, 27.56829)(46.07679465776294, 28.77312)(46.17696160267112, 28.182299999999998)(46.2771285475793, 15.2789)(46.37729549248748, 23.08343)(46.47746243739566, 16.69735)(46.57762938230384, 20.376530000000002)(46.67779632721202, 9.23277)(46.7779632721202, 9.987770000000001)(46.87813021702838, 8.84397)(46.97829716193656, 9.46836)(47.07846410684474, 9.77842)(47.17863105175292, 9.9188)(47.278797996661105, 9.71311)(47.378964941569286, 8.04991)(47.47913188647747, 8.17503)(47.57929883138564, 8.00047)(47.679465776293824, 9.093)(47.779632721202006, 8.18052)(47.87979966611019, 9.5178)(47.97996661101836, 8.05296)(48.080133555926544, 8.91356)(48.180300500834726, 8.21592)(48.28046744574291, 10.34726)(48.38063439065108, 8.99595)(48.48080133555926, 8.07432)(48.580968280467445, 8.424660000000001)(48.68113522537563, 9.72653)(48.7813021702838, 9.6594)(48.88146911519199, 9.10337)(48.98163606010017, 9.761330000000001)(49.081803005008354, 9.8242)(49.18196994991653, 9.412199999999999)(49.28213689482471, 8.49119)(49.38230383973289, 8.03037)(49.48247078464107, 9.10032)(49.58263772954925, 8.84214)(49.68280467445743, 9.64292)(49.78297161936561, 9.62644)(49.88313856427379, 9.67954)(49.98330550918197, 9.68015)(50.08347245409015, 9.37009)(50.18363939899833, 9.81809)(50.28380634390651, 8.52598)(50.38397328881469, 9.91208)(50.484140233722876, 9.71616)(50.58430717863106, 10.03415)(50.68447412353924, 9.41221)(50.784641068447414, 8.18174)(50.884808013355595, 9.72105)(50.98497495826378, 10.07199)(51.08514190317196, 9.40061)(51.18530884808013, 9.62461)(51.285475792988315, 9.304170000000001)(51.3856427378965, 9.1821)(51.48580968280468, 9.29136)(51.58597662771285, 9.98959)(51.686143572621035, 8.06699)(51.786310517529216, 8.24338)(51.8864774624374, 7.9900899999999995)(51.98664440734557, 8.55467)(52.086811352253754, 9.36766)(52.18697829716194, 9.6948)(52.287145242070125, 8.25864)(52.3873121869783, 8.663920000000001)(52.48747913188648, 8.30137)(52.58764607679466, 9.699069999999999)(52.68781302170284, 8.16954)(52.78797996661102, 9.35727)(52.8881469115192, 9.6832)(52.98831385642738, 8.327)(53.088480801335564, 20.49861)(53.18864774624374, 8.2501)(53.28881469115192, 9.7717)(53.3889816360601, 8.821390000000001)(53.489148580968276, 9.03501)(53.58931552587646, 7.964449999999999)(53.68948247078464, 9.6356)(53.78964941569283, 9.692969999999999)(53.889816360601, 9.63865)(53.989983305509185, 9.076509999999999)(54.090150250417366, 9.18821)(54.19031719532555, 8.22752)(54.29048414023372, 8.23362)(54.390651085141904, 10.03904)(54.490818030050086, 9.50437)(54.59098497495827, 8.132909999999999)(54.69115191986644, 7.743510000000001)(54.791318864774624, 9.205910000000001)(54.891485809682806, 8.42954)(54.99165275459099, 8.46983)(55.09181969949916, 9.47568)(55.19198664440734, 9.68259)(55.292153589315525, 10.469949999999999)(55.39232053422371, 7.82346)(55.49248747913189, 9.88035)(55.59265442404007, 8.89768)(55.69282136894825, 8.352030000000001)(55.79298831385643, 9.26633)(55.89315525876461, 9.296240000000001)(55.99332220367279, 8.24033)(56.09348914858097, 8.22446)(56.19365609348915, 10.176369999999999)(56.29382303839733, 9.69114)(56.39398998330551, 8.69321)(56.49415692821369, 9.00327)(56.59432387312187, 8.970320000000001)(56.69449081803005, 7.949199999999999)(56.79465776293823, 9.74973)(56.89482470784641, 7.73374)(56.99499165275459, 9.31211)(57.095158597662774, 8.18235)(57.195325542570956, 8.9929)(57.29549248747914, 8.884260000000001)(57.39565943238732, 9.83945)(57.495826377295494, 9.868749999999999)(57.595993322203675, 9.291970000000001)(57.69616026711186, 10.081150000000001)(57.79632721202004, 10.02439)(57.89649415692821, 8.85617)(57.996661101836395, 9.55442)(58.09682804674458, 7.82896)(58.19699499165276, 9.230319999999999)(58.29716193656093, 7.99497)(58.397328881469114, 8.41855)(58.497495826377296, 9.15342)(58.59766277128548, 8.64744)(58.69782971619365, 7.8845)(58.79799666110184, 8.06882)(58.89816360601002, 7.98582)(58.9983305509182, 9.8773)(59.09849749582638, 8.89585)(59.19866444073456, 8.18662)(59.29883138564274, 9.19737)(59.398998330550924, 8.37827)(59.4991652754591, 9.76011)(59.59933222036728, 9.4946)(59.69949916527546, 10.19407)(59.79966611018364, 10.1135)(59.89983305509182, 9.92123)(60.0, 10.095799999999999)
        };
        \addplot[color=blue, mark=none,name path=B] coordinates { %% MIN value
        (0.0, 0.0)(0.1001669449081803, 2.15515)(0.2003338898163606, 2.18383)(0.3005008347245409, 2.20214)(0.4006677796327212, 2.17223)(0.5008347245409015, 2.22472)(0.6010016694490818, 2.1582)(0.7011686143572621, 2.17284)(0.8013355592654424, 2.1429400000000003)(0.9015025041736228, 2.35168)(1.001669449081803, 2.21069)(1.1018363939899833, 2.10021)(1.2020033388981637, 2.1514800000000003)(1.3021702838063438, 2.1283)(1.4023372287145242, 2.1759)(1.5025041736227045, 2.14416)(1.6026711185308848, 2.22228)(1.7028380634390652, 2.20397)(1.8030050083472455, 2.19543)(1.9031719532554257, 2.22046)(2.003338898163606, 2.13867)(2.1035058430717863, 2.18505)(2.2036727879799667, 2.17956)(2.303839732888147, 2.22412)(2.4040066777963274, 2.07824)(2.5041736227045073, 1.4782699999999998)(2.6043405676126876, 2.19543)(2.704507512520868, 2.18078)(2.8046744574290483, 2.17224)(2.9048414023372287, 2.1881)(3.005008347245409, 2.19238)(3.1051752921535893, 2.21863)(3.2053422370617697, 2.22533)(3.30550918196995, 2.20885)(3.4056761268781304, 2.18566)(3.5058430717863107, 2.18871)(3.606010016694491, 2.15637)(3.7061769616026714, 2.19726)(3.8063439065108513, 2.21008)(3.906510851419032, 2.2351)(4.006677796327212, 2.2479199999999997)(4.106844741235393, 2.20825)(4.207011686143573, 2.21129)(4.3071786310517535, 2.20397)(4.407345575959933, 2.21679)(4.507512520868114, 2.22595)(4.607679465776294, 2.19299)(4.707846410684475, 2.21191)(4.808013355592655, 2.16186)(4.908180300500835, 2.16857)(5.0083472454090145, 2.14905)(5.108514190317195, 2.18444)(5.208681135225375, 2.23876)(5.308848080133556, 2.17468)(5.409015025041736, 2.135)(5.509181969949917, 2.19848)(5.609348914858097, 2.2479199999999997)(5.709515859766277, 2.21985)(5.809682804674457, 2.21679)(5.909849749582638, 2.17284)(6.010016694490818, 2.21923)(6.110183639398999, 2.18872)(6.210350584307179, 2.25219)(6.3105175292153595, 2.20336)(6.410684474123539, 2.19787)(6.510851419031719, 2.17041)(6.6110183639399, 2.22168)(6.71118530884808, 2.20275)(6.811352253756261, 2.1148700000000002)(6.911519198664441, 2.22351)(7.011686143572621, 2.21069)(7.111853088480801, 2.09839)(7.212020033388982, 2.18322)(7.312186978297162, 2.18078)(7.412353923205343, 2.21252)(7.512520868113523, 2.20275)(7.612687813021703, 2.22717)(7.712854757929884, 2.16186)(7.813021702838064, 2.19482)(7.913188647746244, 2.1997)(8.013355592654424, 2.14721)(8.113522537562606, 2.21008)(8.213689482470786, 2.1997)(8.313856427378965, 2.23266)(8.414023372287145, 2.18689)(8.514190317195327, 2.182)(8.614357262103507, 2.23693)(8.714524207011687, 2.21373)(8.814691151919867, 2.15637)(8.914858096828048, 2.19604)(9.015025041736228, 2.22472)(9.115191986644408, 2.23633)(9.215358931552588, 2.2442599999999997)(9.31552587646077, 2.2058)(9.41569282136895, 2.25768)(9.51585976627713, 2.18811)(9.61602671118531, 2.21374)(9.71619365609349, 2.1453800000000003)(9.81636060100167, 2.21557)(9.916527545909851, 2.1936)(10.016694490818029, 2.26439)(10.11686143572621, 2.15637)(10.21702838063439, 2.2412)(10.31719532554257, 2.2479199999999997)(10.41736227045075, 2.18811)(10.51752921535893, 2.13805)(10.617696160267112, 2.27172)(10.717863105175292, 2.22655)(10.818030050083472, 2.17407)(10.918196994991652, 2.1521)(11.018363939899833, 2.21862)(11.118530884808013, 2.22839)(11.218697829716193, 2.21252)(11.318864774624373, 2.2296)(11.419031719532555, 2.19787)(11.519198664440735, 2.14721)(11.619365609348915, 2.18932)(11.719532554257095, 2.22778)(11.819699499165276, 2.1276800000000002)(11.919866444073456, 2.20642)(12.020033388981636, 2.20092)(12.120200333889816, 2.27111)(12.220367278797998, 2.22167)(12.320534223706177, 2.20825)(12.420701168614357, 2.18566)(12.520868113522537, 2.22961)(12.621035058430719, 2.26196)(12.721202003338899, 2.19298)(12.821368948247079, 2.17346)(12.921535893155259, 2.21252)(13.021702838063439, 2.20275)(13.12186978297162, 2.15331)(13.2220367278798, 2.30956)(13.32220367278798, 2.2448699999999997)(13.42237061769616, 2.17834)(13.522537562604342, 2.14965)(13.622704507512521, 2.2174)(13.722871452420701, 2.18688)(13.823038397328881, 2.1875)(13.923205342237063, 2.21191)(14.023372287145243, 2.20947)(14.123539232053423, 2.3046800000000003)(14.223706176961603, 2.23633)(14.323873121869784, 2.22045)(14.424040066777964, 2.28454)(14.524207011686144, 2.19116)(14.624373956594324, 2.20764)(14.724540901502506, 2.14355)(14.824707846410686, 2.19909)(14.924874791318866, 2.23205)(15.025041736227045, 2.18811)(15.125208681135225, 2.20092)(15.225375626043405, 2.21618)(15.325542570951589, 2.10205)(15.425709515859769, 2.1459900000000003)(15.525876460767948, 2.25036)(15.626043405676128, 2.20276)(15.726210350584308, 2.18505)(15.826377295492488, 2.3077300000000003)(15.926544240400668, 2.27966)(16.026711185308848, 2.22961)(16.126878130217026, 2.2692799999999997)(16.22704507512521, 2.23022)(16.32721202003339, 2.2540199999999997)(16.42737896494157, 2.22411)(16.52754590984975, 2.2869800000000002)(16.62771285475793, 2.1875)(16.72787979966611, 2.20703)(16.82804674457429, 2.20459)(16.92821368948247, 2.23571)(17.028380634390654, 2.21313)(17.128547579298832, 2.31628)(17.228714524207014, 2.21984)(17.328881469115192, 2.16857)(17.429048414023374, 2.2589)(17.529215358931552, 2.27356)(17.629382303839733, 2.23388)(17.72954924874791, 2.22839)(17.829716193656097, 2.2058)(17.929883138564275, 2.1936)(18.030050083472457, 2.19177)(18.130217028380635, 2.21008)(18.230383973288816, 2.22168)(18.330550918196995, 2.2753799999999997)(18.430717863105176, 2.23266)(18.530884808013354, 2.27356)(18.63105175292154, 2.23021)(18.731218697829718, 2.19177)(18.8313856427379, 2.2052)(18.931552587646078, 2.16674)(19.03171953255426, 2.27477)(19.131886477462437, 2.23327)(19.23205342237062, 2.23876)(19.332220367278797, 2.29491)(19.43238731218698, 2.25768)(19.53255425709516, 2.27111)(19.63272120200334, 2.24853)(19.73288814691152, 2.25341)(19.833055091819702, 2.2601299999999998)(19.93322203672788, 2.2985800000000003)(20.033388981636058, 2.276)(20.13355592654424, 2.20336)(20.23372287145242, 2.31322)(20.333889816360603, 2.25524)(20.43405676126878, 2.31506)(20.534223706176963, 2.21984)(20.63439065108514, 2.23754)(20.734557595993323, 2.23327)(20.8347245409015, 2.26806)(20.934891485809683, 2.17895)(21.03505843071786, 2.25464)(21.135225375626046, 2.23022)(21.235392320534224, 2.23938)(21.335559265442406, 2.20641)(21.435726210350584, 2.20825)(21.535893155258766, 2.16919)(21.636060100166944, 2.23266)(21.736227045075125, 2.24609)(21.836393989983303, 2.26196)(21.93656093489149, 2.26135)(22.036727879799667, 2.19177)(22.13689482470785, 2.2882)(22.237061769616027, 2.28576)(22.33722871452421, 2.23144)(22.437395659432386, 2.23754)(22.537562604340568, 2.16247)(22.637729549248746, 2.2589)(22.737896494156928, 2.29492)(22.83806343906511, 2.32788)(22.93823038397329, 2.24914)(23.03839732888147, 2.25951)(23.13856427378965, 2.28271)(23.23873121869783, 2.276)(23.33889816360601, 2.33031)(23.43906510851419, 2.2662299999999997)(23.53923205342237, 2.24975)(23.639398998330552, 2.31567)(23.739565943238734, 2.2570699999999997)(23.839732888146912, 2.30651)(23.939899833055094, 2.28271)(24.040066777963272, 2.22167)(24.140233722871454, 2.25158)(24.24040066777963, 2.19543)(24.340567612687813, 2.25951)(24.440734557595995, 2.27539)(24.540901502504177, 2.18688)(24.641068447412355, 2.27966)(24.741235392320537, 2.27966)(24.841402337228715, 2.26134)(24.941569282136896, 2.2778300000000002)(25.041736227045075, 2.32726)(25.141903171953256, 2.21924)(25.242070116861438, 2.2930800000000002)(25.34223706176962, 2.23755)(25.442404006677798, 2.23205)(25.54257095158598, 2.2821)(25.642737896494157, 2.22167)(25.74290484140234, 2.22778)(25.843071786310517, 2.28271)(25.9432387312187, 2.2052)(26.043405676126877, 2.27965)(26.143572621035062, 2.2985800000000003)(26.24373956594324, 2.31506)(26.34390651085142, 2.18994)(26.4440734557596, 2.2540299999999998)(26.544240400667782, 2.3107800000000003)(26.64440734557596, 2.23693)(26.744574290484138, 2.24914)(26.84474123539232, 2.2753799999999997)(26.9449081803005, 2.22533)(27.045075125208683, 2.2351)(27.14524207011686, 2.15027)(27.245409015025043, 2.27416)(27.34557595993322, 2.19909)(27.445742904841403, 2.31323)(27.54590984974958, 2.21557)(27.646076794657763, 2.2174)(27.746243739565944, 2.2662299999999997)(27.846410684474126, 2.19848)(27.946577629382304, 2.22716)(28.046744574290486, 2.23633)(28.146911519198664, 2.1148700000000002)(28.247078464106846, 2.26867)(28.347245409015024, 2.24059)(28.447412353923205, 2.18567)(28.547579298831387, 2.22716)(28.64774624373957, 2.2418199999999997)(28.747913188647747, 2.30041)(28.84808013355593, 2.3016300000000003)(28.948247078464107, 2.27599)(29.04841402337229, 2.26135)(29.148580968280466, 2.2601299999999998)(29.248747913188648, 2.25524)(29.348914858096826, 2.2448699999999997)(29.44908180300501, 2.2894200000000002)(29.54924874791319, 2.31872)(29.64941569282137, 2.22778)(29.74958263772955, 2.25768)(29.84974958263773, 2.3297)(29.94991652754591, 2.23144)(30.05008347245409, 2.2467)(30.15025041736227, 2.33825)(30.25041736227045, 2.23388)(30.35058430717863, 2.2961400000000003)(30.45075125208681, 2.2534199999999998)(30.55091819699499, 2.27661)(30.651085141903177, 2.3052900000000003)(30.751252086811355, 2.29675)(30.851419031719537, 2.24181)(30.951585976627715, 2.3083400000000003)(31.051752921535897, 2.2345)(31.151919866444075, 2.27355)(31.252086811352257, 2.27721)(31.352253756260435, 2.21557)(31.452420701168617, 2.16247)(31.552587646076795, 2.2473099999999997)(31.652754590984976, 2.229)(31.752921535893154, 2.30102)(31.853088480801336, 2.18566)(31.953255425709514, 2.2985800000000003)(32.053422370617696, 2.24975)(32.15358931552588, 2.2882)(32.25375626043405, 2.2839300000000002)(32.35392320534224, 2.25036)(32.45409015025042, 2.29735)(32.554257095158604, 2.26074)(32.65442404006678, 2.2509699999999997)(32.75459098497496, 2.32848)(32.85475792988314, 2.26745)(32.95492487479132, 2.27111)(33.0550918196995, 2.2863700000000002)(33.15525876460768, 2.2473099999999997)(33.25542570951586, 2.23876)(33.35559265442404, 2.29675)(33.45575959933222, 2.2937)(33.5559265442404, 2.31811)(33.65609348914858, 2.25036)(33.756260434056756, 2.35168)(33.85642737896494, 2.30896)(33.95659432387313, 2.23449)(34.05676126878131, 2.3016300000000003)(34.15692821368948, 2.31261)(34.257095158597664, 2.3016300000000003)(34.357262103505846, 2.24242)(34.45742904841403, 2.27416)(34.5575959933222, 2.32056)(34.657762938230384, 2.2985700000000002)(34.757929883138566, 2.3016300000000003)(34.85809682804675, 2.2991900000000003)(34.95826377295492, 2.26989)(35.058430717863104, 2.05566)(35.158597662771285, 2.2985800000000003)(35.25876460767947, 2.26989)(35.35893155258764, 2.33764)(35.45909849749582, 2.32116)(35.559265442404005, 2.24608)(35.659432387312194, 2.31689)(35.75959933222037, 2.26378)(35.85976627712855, 2.32482)(35.95993322203673, 2.26989)(36.06010016694491, 2.2662299999999997)(36.16026711185309, 2.26256)(36.26043405676127, 2.32239)(36.36060100166945, 2.34618)(36.46076794657763, 2.29736)(36.56093489148581, 2.25158)(36.66110183639399, 2.35167)(36.76126878130217, 2.2448699999999997)(36.86143572621035, 2.29064)(36.96160267111853, 2.21496)(37.06176961602671, 2.2644)(37.16193656093489, 2.26379)(37.26210350584308, 2.33093)(37.362270450751254, 2.28882)(37.462437395659435, 2.26684)(37.56260434056762, 2.34497)(37.6627712854758, 2.25341)(37.76293823038397, 2.30102)(37.863105175292155, 2.34985)(37.96327212020034, 2.25586)(38.06343906510852, 2.23266)(38.16360601001669, 2.2601299999999998)(38.263772954924875, 2.21435)(38.363939899833056, 2.2692799999999997)(38.46410684474124, 2.3236)(38.56427378964941, 2.23693)(38.664440734557594, 2.3529)(38.764607679465776, 2.24364)(38.86477462437396, 2.3645)(38.96494156928214, 2.33887)(39.06510851419032, 2.31628)(39.1652754590985, 2.31445)(39.26544240400668, 2.2991900000000003)(39.36560934891486, 2.29125)(39.46577629382304, 2.34191)(39.56594323873122, 2.27844)(39.666110183639404, 2.3529)(39.76627712854758, 2.22168)(39.86644407345576, 2.40111)(39.96661101836394, 2.2900400000000003)(40.066777963272116, 2.36388)(40.1669449081803, 2.30956)(40.26711185308848, 2.2839300000000002)(40.36727879799666, 2.3706)(40.46744574290484, 2.31933)(40.567612687813025, 2.3297)(40.667779632721206, 2.3645)(40.76794657762939, 2.36998)(40.86811352253756, 2.2821)(40.968280467445744, 2.32848)(41.068447412353926, 2.24975)(41.16861435726211, 2.36023)(41.26878130217028, 2.35656)(41.368948247078464, 2.27844)(41.469115191986646, 2.31505)(41.56928213689483, 2.27172)(41.669449081803, 2.29186)(41.769616026711184, 2.30712)(41.869782971619365, 2.26135)(41.96994991652755, 2.30713)(42.07011686143572, 2.21619)(42.1702838063439, 2.28149)(42.27045075125209, 2.33825)(42.370617696160274, 2.3529)(42.47078464106845, 2.2570699999999997)(42.57095158597663, 2.37732)(42.67111853088481, 2.2467)(42.77128547579299, 2.30407)(42.87145242070117, 2.30956)(42.97161936560935, 2.2869800000000002)(43.07178631051753, 2.27355)(43.17195325542571, 2.27661)(43.27212020033389, 2.31384)(43.37228714524207, 2.25281)(43.47245409015025, 2.30591)(43.57262103505843, 2.29736)(43.67278797996661, 2.29431)(43.77295492487479, 2.32422)(43.87312186978298, 2.31871)(43.97328881469116, 2.2991900000000003)(44.073455759599334, 2.38342)(44.173622704507515, 2.35534)(44.2737896494157, 2.29186)(44.37395659432388, 2.30407)(44.47412353923205, 2.2955200000000002)(44.574290484140235, 2.2863700000000002)(44.67445742904842, 2.33581)(44.7746243739566, 2.3175)(44.87479131886477, 2.3107800000000003)(44.974958263772955, 2.36815)(45.075125208681136, 2.3022400000000003)(45.17529215358932, 2.31445)(45.27545909849749, 2.2644)(45.375626043405674, 2.32666)(45.475792988313856, 2.38037)(45.57595993322204, 2.23571)(45.67612687813022, 2.34496)(45.7762938230384, 2.26745)(45.87646076794658, 2.41088)(45.976627712854764, 2.2930900000000003)(46.07679465776294, 2.32361)(46.17696160267112, 2.38952)(46.2771285475793, 2.32909)(46.37729549248748, 2.31262)(46.47746243739566, 2.17895)(46.57762938230384, 2.27905)(46.67779632721202, 2.33093)(46.7779632721202, 2.3107800000000003)(46.87813021702838, 2.34375)(46.97829716193656, 2.27234)(47.07846410684474, 2.19299)(47.17863105175292, 2.3645)(47.278797996661105, 2.31323)(47.378964941569286, 2.2412099999999997)(47.47913188647747, 2.33215)(47.57929883138564, 2.31445)(47.679465776293824, 2.31261)(47.779632721202006, 2.28271)(47.87979966611019, 2.2772200000000002)(47.97996661101836, 2.34374)(48.080133555926544, 2.2991900000000003)(48.180300500834726, 2.3059)(48.28046744574291, 2.31689)(48.38063439065108, 2.3291)(48.48080133555926, 2.35717)(48.580968280467445, 2.30407)(48.68113522537563, 2.26195)(48.7813021702838, 2.34375)(48.88146911519199, 2.32421)(48.98163606010017, 2.2863700000000002)(49.081803005008354, 2.2808800000000002)(49.18196994991653, 2.37427)(49.28213689482471, 2.33398)(49.38230383973289, 2.37487)(49.48247078464107, 2.33825)(49.58263772954925, 2.35717)(49.68280467445743, 2.35961)(49.78297161936561, 2.29736)(49.88313856427379, 2.3706)(49.98330550918197, 2.27966)(50.08347245409015, 2.33215)(50.18363939899833, 2.29797)(50.28380634390651, 2.32482)(50.38397328881469, 2.29369)(50.484140233722876, 2.31567)(50.58430717863106, 2.22045)(50.68447412353924, 2.32787)(50.784641068447414, 2.3052900000000003)(50.884808013355595, 2.2631699999999997)(50.98497495826378, 2.32238)(51.08514190317196, 2.29369)(51.18530884808013, 2.42981)(51.285475792988315, 2.30895)(51.3856427378965, 2.35473)(51.48580968280468, 2.3046800000000003)(51.58597662771285, 2.29675)(51.686143572621035, 2.31933)(51.786310517529216, 2.38403)(51.8864774624374, 2.39135)(51.98664440734557, 2.32055)(52.086811352253754, 2.38952)(52.18697829716194, 2.37487)(52.287145242070125, 2.38403)(52.3873121869783, 2.2955300000000003)(52.48747913188648, 2.32482)(52.58764607679466, 2.30956)(52.68781302170284, 2.29736)(52.78797996661102, 2.28882)(52.8881469115192, 2.359)(52.98831385642738, 2.30651)(53.088480801335564, 2.4359)(53.18864774624374, 2.31079)(53.28881469115192, 2.3046800000000003)(53.3889816360601, 2.33337)(53.489148580968276, 2.24304)(53.58931552587646, 2.2808800000000002)(53.68948247078464, 2.31994)(53.78964941569283, 2.37426)(53.889816360601, 2.2747800000000002)(53.989983305509185, 2.35962)(54.090150250417366, 2.41882)(54.19031719532555, 2.39013)(54.29048414023372, 2.2839300000000002)(54.390651085141904, 2.3883)(54.490818030050086, 2.27172)(54.59098497495827, 2.31323)(54.69115191986644, 2.30346)(54.791318864774624, 2.37975)(54.891485809682806, 2.25586)(54.99165275459099, 2.2924800000000003)(55.09181969949916, 2.30957)(55.19198664440734, 2.32421)(55.292153589315525, 2.38891)(55.39232053422371, 2.3077300000000003)(55.49248747913189, 2.29369)(55.59265442404007, 2.36693)(55.69282136894825, 2.39501)(55.79298831385643, 2.36816)(55.89315525876461, 2.276)(55.99332220367279, 2.28515)(56.09348914858097, 2.33703)(56.19365609348915, 2.36694)(56.29382303839733, 2.34069)(56.39398998330551, 2.33642)(56.49415692821369, 2.2747800000000002)(56.59432387312187, 2.33581)(56.69449081803005, 2.32726)(56.79465776293823, 2.22716)(56.89482470784641, 2.29065)(56.99499165275459, 2.21741)(57.095158597662774, 2.40539)(57.195325542570956, 2.31933)(57.29549248747914, 2.38585)(57.39565943238732, 2.31994)(57.495826377295494, 2.3291)(57.595993322203675, 2.33215)(57.69616026711186, 2.39074)(57.79632721202004, 2.25464)(57.89649415692821, 2.2955300000000003)(57.996661101836395, 2.4163799999999998)(58.09682804674458, 2.35228)(58.19699499165276, 2.37792)(58.29716193656093, 2.33459)(58.397328881469114, 2.2961400000000003)(58.497495826377296, 2.37731)(58.59766277128548, 2.26562)(58.69782971619365, 2.34436)(58.79799666110184, 2.2894200000000002)(58.89816360601002, 2.35656)(58.9983305509182, 2.34558)(59.09849749582638, 2.36205)(59.19866444073456, 2.43286)(59.29883138564274, 2.42492)(59.398998330550924, 2.35961)(59.4991652754591, 2.32787)(59.59933222036728, 2.4432400000000003)(59.69949916527546, 2.39013)(59.79966611018364, 2.30346)(59.89983305509182, 2.36816)(60.0, 2.36023)
        };
        \addplot [pattern=north east lines,pattern color=red] 
        fill between [
            of=A and B,soft clip={domain=0:800},
        ];
        \end{axis}
\end{tikzpicture}
\caption{Test case: TestCaseIdle}
\end{subfigure}
\caption{The energy consumption of all the test cases on the DUT: workstation with measuring instrument: RAPL. The lines represent the minimum, maximum and average energy consumption}\label{fig:time_series_Workstation_RAPL}
\end{figure}


\subsection{Comparing the DUTs}
After looking at the different test cases, this subsection will compare the difference between the DUTs. Therefore the test case and the measuring instrument will be kept consistent. 

\paragraph{Expectation:}
We expect that the workstation has a higher energy consumption than the laptops since the CPU has a higher TDP. Regarding the laptops, it is expected they perform similarly since the hardware specifications are very similar.

\paragraph{Results:}
%Så vælg test case. vælg forksellige DUTs.
As can be seen in \cref{fig:time_series_Fankuch_IntelPowerGadget} the Surface Book and Surface Pro 4 look similar as expected. However, the Surface Pro 4 consumes $~5-10$ more joules than the Surface Book. Although, they are similar in regards to variance except that the Surface Pro 4 has lower valleys. When looking at \cref{fig:time_series_Fankuch_IntelPowerGadgetWorkstation} which is the workstation it is clear that it consumes notably more energy than the other systems as expected. It also has more variance with more peaks and valleys. Interestingly the workstation shows a trend with a slight climb in the average energy consumption as the test case goes on.\todo{why?}

\begin{figure}[H]  
    \centering 
    \begin{subfigure}[b]{0.49\linewidth}
        \begin{tikzpicture}
            \pgfplotsset{%
        width=1\linewidth,
        % height=1\textheight
        }
        \begin{axis}[ymax=120,
            xlabel={Time (Seconds)},
            ylabel={Energy Consumption (Joules)},
            ]
            \addplot[color=blue, mark=none,] coordinates { %% AVG value
            (0.0, 48.21988333333334)(0.09983361064891848, 51.91779166666666)(0.19966722129783696, 58.734224999999974)(0.2995008319467554, 53.84733333333332)(0.3993344425956739, 57.82893333333333)(0.49916805324459235, 57.81869166666668)(0.5990016638935108, 56.08766666666665)(0.6988352745424293, 58.84439999999999)(0.7986688851913478, 56.59066666666669)(0.8985024958402663, 56.609708333333366)(0.9983361064891847, 57.09276666666668)(1.0981697171381033, 57.04582500000001)(1.1980033277870217, 57.54740833333332)(1.2978369384359403, 57.39043333333333)(1.3976705490848587, 55.72083333333332)(1.4975041597337773, 56.696199999999976)(1.5973377703826956, 56.926075)(1.697171381031614, 56.95915833333336)(1.7970049916805326, 57.21712500000001)(1.896838602329451, 56.436700000000044)(1.9966722129783694, 56.53028333333334)(2.096505823627288, 56.70840000000001)(2.1963394342762066, 55.951991666666686)(2.296173044925125, 58.00165833333334)(2.3960066555740434, 55.90116666666669)(2.4958402662229617, 57.50826666666669)(2.5956738768718806, 57.47493333333335)(2.695507487520799, 57.03269166666666)(2.7953410981697173, 57.32719166666664)(2.8951747088186357, 56.26819166666667)(2.9950083194675545, 57.525849999999984)(3.094841930116473, 57.337075000000006)(3.1946755407653913, 57.20451666666668)(3.2945091514143097, 56.71186666666667)(3.394342762063228, 57.071366666666634)(3.494176372712147, 57.104558333333344)(3.5940099833610653, 57.48334166666666)(3.6938435940099836, 56.857066666666675)(3.793677204658902, 56.84204166666665)(3.8935108153078204, 57.207375)(3.9933444259567388, 56.99001666666664)(4.093178036605657, 58.48667500000001)(4.193011647254576, 57.254708333333355)(4.292845257903495, 57.24130833333333)(4.392678868552413, 57.526533333333326)(4.492512479201332, 56.675324999999994)(4.59234608985025, 58.207150000000006)(4.692179700499168, 57.404458333333345)(4.792013311148087, 56.95541666666666)(4.891846921797005, 57.89444999999998)(4.9916805324459235, 57.826049999999974)(5.091514143094843, 57.834125)(5.191347753743761, 57.60068333333334)(5.2911813643926795, 57.318308333333334)(5.391014975041598, 58.03585833333334)(5.490848585690516, 58.28424999999999)(5.590682196339435, 57.06525)(5.690515806988353, 57.74603333333336)(5.790349417637271, 57.96253333333334)(5.89018302828619, 57.37312499999999)(5.990016638935109, 58.321924999999965)(6.089850249584027, 57.129150000000045)(6.189683860232946, 58.54914166666665)(6.289517470881864, 57.872541666666656)(6.389351081530783, 57.41381666666668)(6.489184692179701, 57.85493333333333)(6.589018302828619, 57.06841666666664)(6.688851913477538, 58.14766666666667)(6.788685524126456, 57.95741666666665)(6.888519134775375, 57.51019999999999)(6.988352745424294, 58.41750833333335)(7.088186356073212, 57.31336666666665)(7.1880199667221305, 57.62010833333331)(7.287853577371049, 57.64659166666666)(7.387687188019967, 58.216375000000006)(7.487520798668886, 57.82960000000002)(7.587354409317804, 58.02724166666665)(7.687188019966722, 57.405525000000004)(7.787021630615641, 57.73911666666664)(7.886855241264559, 56.852549999999994)(7.9866888519134775, 58.04321666666665)(8.086522462562396, 58.836016666666666)(8.186356073211314, 57.45552500000001)(8.286189683860234, 57.11128333333332)(8.386023294509153, 57.762625000000035)(8.485856905158071, 57.87297500000001)(8.58569051580699, 57.36938333333332)(8.685524126455908, 57.83747500000001)(8.785357737104826, 57.536958333333295)(8.885191347753745, 58.259850000000014)(8.985024958402663, 57.72662500000002)(9.084858569051582, 58.803916666666645)(9.1846921797005, 58.06444166666666)(9.284525790349418, 58.13981666666666)(9.384359400998337, 58.82125000000003)(9.484193011647255, 58.00562499999999)(9.584026622296173, 58.586183333333324)(9.683860232945092, 58.635458333333325)(9.78369384359401, 57.528300000000016)(9.883527454242929, 58.71210833333332)(9.983361064891847, 58.406249999999986)(10.083194675540767, 58.20718333333333)(10.183028286189685, 58.734100000000005)(10.282861896838604, 57.91448333333332)(10.382695507487522, 58.32216666666667)(10.48252911813644, 57.69990833333333)(10.582362728785359, 57.13064166666667)(10.682196339434277, 58.572083333333325)(10.782029950083196, 58.16030833333332)(10.881863560732114, 57.54412499999998)(10.981697171381033, 58.45291666666666)(11.081530782029951, 58.462683333333345)(11.18136439267887, 58.79325000000001)(11.281198003327788, 58.52914166666666)(11.381031613976706, 58.612766666666644)(11.480865224625624, 58.45905000000002)(11.580698835274543, 58.75544166666668)(11.680532445923461, 57.63391666666668)(11.78036605657238, 59.22149166666666)(11.8801996672213, 58.48041666666667)(11.980033277870218, 57.67416666666671)(12.079866888519136, 59.807425)(12.179700499168055, 58.93529999999999)(12.279534109816973, 57.627050000000004)(12.379367720465892, 58.30583333333333)(12.47920133111481, 57.834983333333334)(12.579034941763728, 58.57301666666668)(12.678868552412647, 58.77347500000001)(12.778702163061565, 57.459858333333344)(12.878535773710484, 59.0463583333333)(12.978369384359402, 58.12898333333335)(13.07820299500832, 57.94775000000002)(13.178036605657239, 57.87256666666667)(13.277870216306157, 57.62191666666667)(13.377703826955075, 58.28202499999997)(13.477537437603994, 58.20480833333332)(13.577371048252912, 57.8483)(13.67720465890183, 57.88710000000001)(13.77703826955075, 58.440883333333325)(13.876871880199669, 57.50694166666666)(13.976705490848587, 58.15410833333332)(14.076539101497506, 58.139066666666665)(14.176372712146424, 57.39117499999999)(14.276206322795343, 57.9326)(14.376039933444261, 57.42876666666668)(14.47587354409318, 57.85422499999999)(14.575707154742098, 58.275174999999976)(14.675540765391016, 57.16953333333333)(14.775374376039935, 58.17819999999998)(14.875207986688853, 58.14019999999996)(14.975041597337771, 58.04575)(15.07487520798669, 57.954975)(15.174708818635608, 58.13080833333335)(15.274542429284526, 57.26194166666671)(15.374376039933445, 57.91935)(15.474209650582363, 58.31122499999997)(15.574043261231282, 58.232216666666666)(15.6738768718802, 57.857816666666665)(15.773710482529118, 57.98219166666666)(15.873544093178037, 58.08437499999999)(15.973377703826955, 58.13303333333333)(16.073211314475873, 58.58229166666667)(16.173044925124792, 58.79347500000001)(16.27287853577371, 57.256124999999976)(16.37271214642263, 57.59774166666664)(16.47254575707155, 58.29585000000001)(16.57237936772047, 57.114800000000024)(16.672212978369387, 58.01096666666663)(16.772046589018306, 57.88048333333332)(16.871880199667224, 57.936374999999984)(16.971713810316142, 57.830574999999996)(17.07154742096506, 58.16402500000001)(17.17138103161398, 57.298791666666645)(17.271214642262898, 57.942458333333335)(17.371048252911816, 57.792125000000006)(17.470881863560734, 58.254625000000004)(17.570715474209653, 57.663008333333345)(17.67054908485857, 58.230741666666674)(17.77038269550749, 58.33254999999998)(17.870216306156408, 57.688791666666674)(17.970049916805326, 57.82919166666669)(18.069883527454245, 57.64069166666666)(18.169717138103163, 57.63102499999997)(18.26955074875208, 57.88454166666666)(18.369384359401, 57.784833333333346)(18.469217970049918, 57.68211666666666)(18.569051580698837, 58.19019166666666)(18.668885191347755, 57.83133333333333)(18.768718801996673, 58.40570833333335)(18.86855241264559, 58.39685833333333)(18.96838602329451, 57.46141666666664)(19.06821963394343, 57.86581666666668)(19.168053244592347, 58.39303333333332)(19.267886855241265, 58.051649999999995)(19.367720465890184, 58.1679083333333)(19.467554076539102, 58.32088333333332)(19.56738768718802, 57.940791666666634)(19.66722129783694, 57.68915833333332)(19.767054908485857, 57.935116666666694)(19.866888519134775, 58.63459999999999)(19.966722129783694, 57.47228333333331)(20.066555740432612, 58.77204999999996)(20.166389351081534, 58.91198333333334)(20.266222961730453, 58.06535833333334)(20.36605657237937, 57.8071166666667)(20.46589018302829, 57.92220833333331)(20.565723793677208, 58.02731666666667)(20.665557404326126, 57.138933333333334)(20.765391014975044, 57.70775833333334)(20.865224625623963, 57.83314166666665)(20.96505823627288, 58.15463333333334)(21.0648918469218, 57.81801666666668)(21.164725457570718, 57.849366666666675)(21.264559068219636, 58.17216666666668)(21.364392678868555, 58.192275000000016)(21.464226289517473, 57.94262500000002)(21.56405990016639, 58.142650000000025)(21.66389351081531, 58.23838333333337)(21.76372712146423, 58.20271666666668)(21.863560732113147, 58.13859166666666)(21.963394342762065, 58.17633333333334)(22.063227953410983, 58.25283333333334)(22.163061564059902, 58.589150000000004)(22.26289517470882, 58.06528333333335)(22.36272878535774, 58.22290833333332)(22.462562396006657, 58.16661666666668)(22.562396006655575, 58.27689999999998)(22.662229617304494, 57.90454999999998)(22.762063227953412, 57.837024999999976)(22.86189683860233, 57.85294999999997)(22.96173044925125, 58.02257499999999)(23.061564059900167, 57.98859166666666)(23.161397670549086, 58.18199999999998)(23.261231281198004, 57.79010833333332)(23.361064891846922, 57.41535833333334)(23.46089850249584, 57.73982499999997)(23.56073211314476, 58.628141666666664)(23.660565723793678, 57.69488333333333)(23.7603993344426, 58.36234999999999)(23.860232945091518, 58.105299999999986)(23.960066555740436, 57.928899999999985)(24.059900166389355, 58.493683333333294)(24.159733777038273, 58.905674999999995)(24.25956738768719, 57.53740833333337)(24.35940099833611, 58.40520000000003)(24.459234608985028, 57.82364166666665)(24.559068219633946, 57.53975833333335)(24.658901830282865, 58.62568333333332)(24.758735440931783, 57.52001666666665)(24.8585690515807, 58.55249166666668)(24.95840266222962, 58.903400000000026)(25.05823627287854, 57.54786666666667)(25.158069883527457, 58.87505833333334)(25.257903494176375, 58.10345833333333)(25.357737104825294, 58.19779166666669)(25.457570715474212, 59.32973333333334)(25.55740432612313, 58.034800000000004)(25.65723793677205, 58.761291666666686)(25.757071547420967, 58.706791666666646)(25.856905158069885, 57.754300000000015)(25.956738768718804, 58.86574166666665)(26.056572379367722, 58.380083333333324)(26.15640599001664, 57.85956666666665)(26.25623960066556, 58.938958333333325)(26.356073211314477, 58.03998333333332)(26.455906821963396, 58.507483333333326)(26.555740432612314, 58.37959166666664)(26.655574043261232, 58.28213333333331)(26.75540765391015, 58.11746666666665)(26.85524126455907, 58.349299999999985)(26.955074875207988, 58.54689166666667)(27.054908485856906, 58.10606666666664)(27.154742096505824, 58.27427500000001)(27.254575707154743, 58.00487499999999)(27.35440931780366, 58.65178333333335)(27.45424292845258, 58.442625)(27.5540765391015, 57.75548333333333)(27.653910149750416, 58.8531)(27.753743760399338, 58.17071666666666)(27.853577371048253, 58.22205000000001)(27.953410981697175, 58.743991666666666)(28.05324459234609, 59.017208333333336)(28.15307820299501, 59.34572499999999)(28.252911813643927, 57.59946666666667)(28.35274542429285, 58.191100000000006)(28.452579034941763, 58.91911666666671)(28.552412645590685, 58.02106666666667)(28.6522462562396, 58.88274166666668)(28.752079866888522, 58.07051666666668)(28.851913477537437, 58.257191666666685)(28.95174708818636, 58.639999999999986)(29.051580698835274, 58.60205833333335)(29.151414309484196, 58.036358333333325)(29.25124792013311, 59.07980833333332)(29.351081530782032, 57.92594999999998)(29.450915141430947, 58.273699999999984)(29.55074875207987, 58.839458333333305)(29.650582362728784, 58.217450000000014)(29.750415973377706, 59.220666666666666)(29.85024958402662, 58.146483333333315)(29.950083194675543, 58.22658333333333)(30.049916805324457, 58.56862499999999)(30.14975041597338, 58.16846666666665)(30.249584026622294, 57.695600000000006)(30.349417637271216, 58.401824999999995)(30.44925124792013, 57.686091666666655)(30.549084858569053, 58.004175000000004)(30.648918469217968, 58.58541666666667)(30.74875207986689, 57.55280833333332)(30.848585690515804, 59.0824)(30.948419301164726, 58.141166666666656)(31.04825291181364, 57.80878333333336)(31.148086522462563, 58.19255833333333)(31.247920133111478, 58.345483333333306)(31.3477537437604, 58.221458333333324)(31.447587354409315, 58.457741666666685)(31.547420965058237, 57.865333333333325)(31.64725457570715, 58.42226666666665)(31.747088186356073, 58.266625)(31.84692179700499, 57.36766666666664)(31.94675540765391, 58.080083333333356)(32.046589018302825, 57.58853333333331)(32.14642262895175, 59.261874999999954)(32.24625623960066, 57.99745833333331)(32.346089850249584, 57.57714166666666)(32.4459234608985, 57.87827499999997)(32.54575707154742, 58.929208333333314)(32.645590682196335, 57.74645833333334)(32.74542429284526, 58.38113333333332)(32.84525790349417, 58.314674999999994)(32.9450915141431, 57.93735000000001)(33.044925124792016, 58.61523333333333)(33.14475873544094, 57.561316666666684)(33.24459234608985, 57.71270833333331)(33.344425956738775, 57.996758333333304)(33.44425956738769, 58.700483333333324)(33.54409317803661, 57.83857499999999)(33.643926788685526, 57.854983333333365)(33.74376039933445, 58.43061666666666)(33.84359400998336, 58.418983333333365)(33.943427620632285, 58.20513333333333)(34.0432612312812, 59.02405000000002)(34.14309484193012, 57.77397499999997)(34.24292845257904, 58.747366666666686)(34.34276206322796, 58.1079333333333)(34.44259567387687, 58.424766666666635)(34.542429284525795, 57.83110000000002)(34.64226289517471, 58.31913333333335)(34.74209650582363, 58.630633333333314)(34.84193011647255, 58.01506666666666)(34.94176372712147, 58.635566666666676)(35.04159733777038, 57.50981666666669)(35.141430948419305, 58.80890000000001)(35.24126455906822, 58.340016666666656)(35.34109816971714, 58.508725)(35.44093178036606, 58.706908333333324)(35.54076539101498, 58.52671666666668)(35.640599001663894, 58.211525)(35.740432612312816, 58.269574999999996)(35.84026622296173, 57.93952499999999)(35.94009983361065, 58.652100000000004)(36.03993344425957, 58.03631666666669)(36.13976705490849, 59.80693333333332)(36.239600665557404, 58.88819166666665)(36.339434276206326, 57.754774999999995)(36.43926788685524, 58.57686666666666)(36.53910149750416, 58.99773333333333)(36.63893510815308, 58.133708333333324)(36.738768718802, 58.7722416666667)(36.838602329450914, 59.28005000000001)(36.938435940099836, 57.86103333333333)(37.03826955074875, 59.176150000000014)(37.13810316139767, 58.67609166666662)(37.23793677204659, 58.87825)(37.33777038269551, 59.59644166666667)(37.437603993344425, 58.41046666666667)(37.53743760399335, 58.58268333333331)(37.63727121464226, 58.66234166666667)(37.73710482529118, 58.80891666666668)(37.8369384359401, 59.1417666666667)(37.93677204658902, 59.29799166666667)(38.036605657237935, 59.289933333333344)(38.13643926788686, 59.70965)(38.23627287853577, 58.77356666666669)(38.336106489184694, 58.965850000000025)(38.43594009983361, 58.96815833333332)(38.53577371048253, 58.76604166666667)(38.635607321131445, 59.35532500000001)(38.73544093178037, 58.754100000000015)(38.83527454242928, 59.774591666666666)(38.935108153078204, 59.43879999999999)(39.03494176372712, 58.709583333333335)(39.13477537437604, 58.86879999999999)(39.234608985024956, 59.80196666666668)(39.33444259567388, 59.370799999999996)(39.43427620632279, 58.91553333333332)(39.534109816971714, 58.65395833333334)(39.63394342762063, 58.83245000000002)(39.73377703826955, 59.043391666666636)(39.833610648918466, 59.2092)(39.93344425956739, 59.04994166666664)(40.0332778702163, 59.699983333333336)(40.133111480865225, 60.8489916666667)(40.232945091514146, 58.753975)(40.33277870216307, 59.21152499999998)(40.43261231281198, 58.92220000000001)(40.532445923460905, 57.81406666666668)(40.63227953410982, 59.27180833333333)(40.73211314475874, 59.509916666666676)(40.83194675540766, 57.99486666666668)(40.93178036605658, 59.73048333333332)(41.03161397670549, 59.560350000000014)(41.131447587354415, 58.79998333333333)(41.23128119800333, 59.81038333333333)(41.33111480865225, 58.672383333333315)(41.43094841930117, 58.89354166666668)(41.53078202995009, 59.17903333333331)(41.630615640599004, 59.55133333333335)(41.730449251247926, 58.724258333333324)(41.83028286189684, 59.90230833333332)(41.93011647254576, 58.88338333333332)(42.02995008319468, 59.52902499999998)(42.1297836938436, 59.284300000000016)(42.229617304492514, 58.937075000000014)(42.329450915141436, 59.932550000000006)(42.42928452579035, 59.60292500000001)(42.52911813643927, 58.77144166666666)(42.62895174708819, 59.93267500000001)(42.72878535773711, 59.209066666666665)(42.828618968386024, 59.61660833333333)(42.928452579034946, 59.87521666666667)(43.02828618968386, 58.55789166666664)(43.12811980033278, 59.856525)(43.2279534109817, 59.80508333333331)(43.32778702163062, 59.56744166666668)(43.427620632279535, 60.243708333333366)(43.52745424292846, 59.49425)(43.62728785357737, 59.23664166666667)(43.72712146422629, 60.10470833333332)(43.82695507487521, 59.530066666666684)(43.92678868552413, 60.254283333333305)(44.026622296173045, 60.44260833333336)(44.12645590682197, 59.90587500000002)(44.22628951747088, 60.139224999999996)(44.326123128119804, 59.621400000000015)(44.42595673876872, 59.37199999999998)(44.52579034941764, 60.482883333333334)(44.625623960066555, 59.85184166666666)(44.72545757071548, 59.67564166666664)(44.82529118136439, 60.32070833333334)(44.925124792013314, 59.77105833333334)(45.02495840266223, 59.82191666666666)(45.12479201331115, 60.08697499999999)(45.224625623960065, 59.35990833333332)(45.32445923460899, 59.592000000000006)(45.4242928452579, 59.809483333333354)(45.524126455906824, 59.71126666666668)(45.62396006655574, 60.44449166666664)(45.72379367720466, 60.84959999999999)(45.823627287853576, 59.79336666666666)(45.9234608985025, 60.07230833333332)(46.02329450915141, 60.410300000000014)(46.123128119800334, 59.83058333333336)(46.22296173044925, 60.77741666666665)(46.32279534109817, 60.119708333333314)(46.422628951747086, 60.23826666666667)(46.52246256239601, 60.13477499999999)(46.62229617304492, 59.69924166666666)(46.722129783693845, 60.4723)(46.82196339434276, 60.2182666666667)(46.92179700499168, 60.05324166666666)(47.021630615640596, 60.19575833333334)(47.12146422628952, 60.08200833333334)(47.22129783693843, 60.719775)(47.321131447587355, 60.59821666666673)(47.42096505823627, 60.682008333333314)(47.5207986688852, 60.817474999999995)(47.620632279534114, 59.53203333333335)(47.720465890183036, 60.42025833333333)(47.82029950083195, 60.52396666666669)(47.92013311148087, 61.211750000000016)(48.01996672212979, 60.52882500000002)(48.11980033277871, 60.7373666666667)(48.219633943427624, 60.28704166666667)(48.319467554076546, 60.33181666666669)(48.41930116472546, 59.99950833333337)(48.51913477537438, 60.44073333333333)(48.6189683860233, 60.612200000000016)(48.71880199667222, 59.83152499999997)(48.818635607321134, 60.57335833333333)(48.918469217970056, 60.59350000000002)(49.01830282861897, 60.50696666666666)(49.11813643926789, 61.268175)(49.21797004991681, 60.65858333333336)(49.31780366056573, 60.674475)(49.417637271214645, 60.304716666666664)(49.51747088186357, 60.91398333333332)(49.61730449251248, 61.36495833333332)(49.7171381031614, 60.51624166666665)(49.81697171381032, 60.69317500000001)(49.91680532445924, 60.99286666666667)(50.016638935108155, 60.82545)(50.11647254575708, 61.43866666666666)(50.21630615640599, 61.38371666666667)(50.31613976705491, 61.19751666666668)(50.41597337770383, 61.006150000000005)(50.51580698835275, 61.00678333333336)(50.615640599001665, 60.530725000000025)(50.71547420965059, 61.31093333333333)(50.8153078202995, 61.07538333333335)(50.915141430948424, 61.27456666666665)(51.01497504159734, 61.10125000000003)(51.11480865224626, 61.81923333333332)(51.214642262895175, 60.65160833333333)(51.3144758735441, 61.36236666666666)(51.41430948419301, 61.12884999999997)(51.514143094841934, 60.73637499999996)(51.61397670549085, 61.08379166666665)(51.71381031613977, 60.98598333333332)(51.813643926788686, 60.85883333333335)(51.91347753743761, 61.45158333333337)(52.01331114808652, 60.71275000000003)(52.113144758735444, 61.20731666666667)(52.21297836938436, 61.63860833333339)(52.31281198003328, 61.45618333333335)(52.412645590682196, 60.341533333333324)(52.51247920133112, 61.992741666666674)(52.61231281198003, 60.32594166666666)(52.712146422628955, 61.50972500000001)(52.81198003327787, 61.74897499999999)(52.91181364392679, 60.84354166666666)(53.011647254575706, 60.86894166666666)(53.11148086522463, 61.11336666666669)(53.21131447587354, 61.05315)(53.311148086522465, 61.46170833333334)(53.41098169717138, 61.48522499999998)(53.5108153078203, 60.719449999999945)(53.61064891846922, 61.356641666666654)(53.71048252911814, 61.87217500000001)(53.81031613976705, 60.91028333333331)(53.910149750415975, 61.17941666666665)(54.00998336106489, 61.402058333333336)(54.10981697171381, 61.18543333333335)(54.20965058236273, 61.40372499999998)(54.30948419301165, 61.64863333333335)(54.409317803660564, 61.62541666666664)(54.509151414309486, 61.598966666666676)(54.6089850249584, 61.321791666666655)(54.70881863560732, 61.452625000000005)(54.808652246256244, 61.57366666666666)(54.90848585690516, 60.835391666666666)(55.00831946755408, 61.30402500000001)(55.108153078203, 60.858183333333336)(55.20798668885192, 60.90859166666666)(55.30782029950083, 61.037049999999994)(55.407653910149754, 61.11860833333333)(55.507487520798676, 60.52362500000002)(55.60732113144759, 60.38001666666669)(55.707154742096506, 61.44250833333332)(55.80698835274543, 61.145775)(55.90682196339435, 60.89184166666667)(56.006655574043265, 61.578125)(56.10648918469218, 61.59771666666663)(56.2063227953411, 61.76641666666667)(56.30615640599002, 61.38554166666669)(56.40599001663894, 61.395975000000014)(56.50582362728785, 61.04043333333333)(56.605657237936775, 61.33874166666664)(56.7054908485857, 60.67652500000001)(56.80532445923461, 61.20141666666666)(56.90515806988353, 61.62620833333334)(57.00499168053245, 60.92076666666664)(57.10482529118137, 61.178275000000006)(57.204658901830285, 61.41579166666666)(57.3044925124792, 61.29492499999999)(57.40432612312812, 61.37351666666669)(57.504159733777044, 61.56623333333334)(57.60399334442596, 61.4048)(57.703826955074874, 61.58738333333331)(57.803660565723796, 61.57108333333336)(57.90349417637272, 61.24882499999999)(58.00332778702163, 61.0018)(58.10316139767055, 61.81615000000001)(58.20299500831947, 61.49299166666667)(58.30282861896839, 61.740908333333344)(58.402662229617306, 61.63536666666667)(58.50249584026622, 61.54935000000001)(58.60232945091514, 61.386266666666685)(58.702163061564065, 61.44199999999999)(58.80199667221298, 61.71741666666667)(58.901830282861894, 61.99364166666666)(59.001663893510816, 62.070116666666685)(59.10149750415974, 61.72813333333332)(59.20133111480865, 61.76678333333334)(59.30116472545757, 61.60629999999999)(59.40099833610649, 61.635541666666654)(59.50083194675541, 61.41445833333335)(59.60066555740433, 61.72413333333332)(59.70049916805324, 61.83704166666666)(59.80033277870216, 61.63337499999998)(59.900166389351085, 43.117584745762684)(60.0, 12.288)
            };
            \addplot[color=blue, mark=none,name path=A] coordinates { %% MAX value
            (0.0, 56.738)(0.09983361064891848, 61.783)(0.19966722129783696, 65.882)(0.2995008319467554, 63.516)(0.3993344425956739, 64.132)(0.49916805324459235, 65.318)(0.5990016638935108, 64.012)(0.6988352745424293, 64.388)(0.7986688851913478, 63.707)(0.8985024958402663, 65.122)(0.9983361064891847, 64.7)(1.0981697171381033, 64.038)(1.1980033277870217, 64.856)(1.2978369384359403, 71.844)(1.3976705490848587, 69.42)(1.4975041597337773, 67.445)(1.5973377703826956, 68.423)(1.697171381031614, 70.191)(1.7970049916805326, 80.858)(1.896838602329451, 69.922)(1.9966722129783694, 72.763)(2.096505823627288, 67.007)(2.1963394342762066, 65.122)(2.296173044925125, 65.956)(2.3960066555740434, 64.962)(2.4958402662229617, 66.554)(2.5956738768718806, 65.584)(2.695507487520799, 65.785)(2.7953410981697173, 65.868)(2.8951747088186357, 64.942)(2.9950083194675545, 64.563)(3.094841930116473, 65.287)(3.1946755407653913, 64.552)(3.2945091514143097, 64.866)(3.394342762063228, 66.038)(3.494176372712147, 64.953)(3.5940099833610653, 64.494)(3.6938435940099836, 64.692)(3.793677204658902, 65.23)(3.8935108153078204, 65.436)(3.9933444259567388, 65.017)(4.093178036605657, 64.823)(4.193011647254576, 65.339)(4.292845257903495, 65.494)(4.392678868552413, 64.511)(4.492512479201332, 64.338)(4.59234608985025, 66.254)(4.692179700499168, 64.682)(4.792013311148087, 64.866)(4.891846921797005, 65.357)(4.9916805324459235, 65.306)(5.091514143094843, 64.494)(5.191347753743761, 64.927)(5.2911813643926795, 64.631)(5.391014975041598, 65.398)(5.490848585690516, 65.661)(5.590682196339435, 66.445)(5.690515806988353, 65.482)(5.790349417637271, 65.473)(5.89018302828619, 66.174)(5.990016638935109, 64.892)(6.089850249584027, 65.848)(6.189683860232946, 64.226)(6.289517470881864, 65.307)(6.389351081530783, 64.19)(6.489184692179701, 65.166)(6.589018302828619, 64.805)(6.688851913477538, 65.516)(6.788685524126456, 65.129)(6.888519134775375, 65.348)(6.988352745424294, 65.149)(7.088186356073212, 65.146)(7.1880199667221305, 64.407)(7.287853577371049, 64.637)(7.387687188019967, 66.955)(7.487520798668886, 65.844)(7.587354409317804, 64.935)(7.687188019966722, 65.426)(7.787021630615641, 66.119)(7.886855241264559, 65.264)(7.9866888519134775, 66.624)(8.086522462562396, 64.677)(8.186356073211314, 66.821)(8.286189683860234, 65.136)(8.386023294509153, 64.63)(8.485856905158071, 64.715)(8.58569051580699, 64.946)(8.685524126455908, 65.518)(8.785357737104826, 65.519)(8.885191347753745, 65.621)(8.985024958402663, 89.52)(9.084858569051582, 108.562)(9.1846921797005, 105.888)(9.284525790349418, 72.984)(9.384359400998337, 83.31)(9.484193011647255, 83.359)(9.584026622296173, 100.586)(9.683860232945092, 99.391)(9.78369384359401, 98.319)(9.883527454242929, 89.877)(9.983361064891847, 96.568)(10.083194675540767, 96.31)(10.183028286189685, 94.341)(10.282861896838604, 80.751)(10.382695507487522, 74.745)(10.48252911813644, 77.82)(10.582362728785359, 76.17)(10.682196339434277, 65.13)(10.782029950083196, 74.333)(10.881863560732114, 83.361)(10.981697171381033, 81.438)(11.081530782029951, 93.644)(11.18136439267887, 108.015)(11.281198003327788, 104.619)(11.381031613976706, 81.651)(11.480865224625624, 80.787)(11.580698835274543, 75.686)(11.680532445923461, 87.758)(11.78036605657238, 95.297)(11.8801996672213, 96.331)(11.980033277870218, 90.639)(12.079866888519136, 97.523)(12.179700499168055, 87.958)(12.279534109816973, 91.001)(12.379367720465892, 83.449)(12.47920133111481, 76.707)(12.579034941763728, 76.571)(12.678868552412647, 73.614)(12.778702163061565, 68.665)(12.878535773710484, 72.491)(12.978369384359402, 84.652)(13.07820299500832, 86.063)(13.178036605657239, 78.63)(13.277870216306157, 76.874)(13.377703826955075, 81.622)(13.477537437603994, 77.323)(13.577371048252912, 78.193)(13.67720465890183, 75.372)(13.77703826955075, 73.033)(13.876871880199669, 75.473)(13.976705490848587, 70.103)(14.076539101497506, 67.916)(14.176372712146424, 68.011)(14.276206322795343, 67.242)(14.376039933444261, 70.763)(14.47587354409318, 67.411)(14.575707154742098, 77.857)(14.675540765391016, 70.219)(14.775374376039935, 64.762)(14.875207986688853, 64.875)(14.975041597337771, 65.045)(15.07487520798669, 65.183)(15.174708818635608, 65.084)(15.274542429284526, 66.881)(15.374376039933445, 65.068)(15.474209650582363, 67.429)(15.574043261231282, 64.697)(15.6738768718802, 65.904)(15.773710482529118, 66.698)(15.873544093178037, 65.202)(15.973377703826955, 65.06)(16.073211314475873, 65.485)(16.173044925124792, 67.505)(16.27287853577371, 66.019)(16.37271214642263, 65.424)(16.47254575707155, 67.103)(16.57237936772047, 67.746)(16.672212978369387, 67.648)(16.772046589018306, 66.463)(16.871880199667224, 64.585)(16.971713810316142, 69.838)(17.07154742096506, 67.973)(17.17138103161398, 64.875)(17.271214642262898, 64.889)(17.371048252911816, 67.344)(17.470881863560734, 65.152)(17.570715474209653, 65.018)(17.67054908485857, 66.251)(17.77038269550749, 65.88)(17.870216306156408, 67.29)(17.970049916805326, 65.685)(18.069883527454245, 66.477)(18.169717138103163, 65.285)(18.26955074875208, 65.25)(18.369384359401, 65.044)(18.469217970049918, 65.155)(18.569051580698837, 65.443)(18.668885191347755, 66.193)(18.768718801996673, 65.971)(18.86855241264559, 65.007)(18.96838602329451, 64.941)(19.06821963394343, 65.708)(19.168053244592347, 66.274)(19.267886855241265, 66.892)(19.367720465890184, 65.083)(19.467554076539102, 65.611)(19.56738768718802, 64.885)(19.66722129783694, 65.636)(19.767054908485857, 65.816)(19.866888519134775, 65.657)(19.966722129783694, 65.687)(20.066555740432612, 67.523)(20.166389351081534, 66.164)(20.266222961730453, 67.641)(20.36605657237937, 65.331)(20.46589018302829, 66.057)(20.565723793677208, 65.4)(20.665557404326126, 64.754)(20.765391014975044, 65.276)(20.865224625623963, 65.784)(20.96505823627288, 64.95)(21.0648918469218, 65.359)(21.164725457570718, 64.926)(21.264559068219636, 65.025)(21.364392678868555, 65.174)(21.464226289517473, 65.914)(21.56405990016639, 64.787)(21.66389351081531, 65.774)(21.76372712146423, 67.513)(21.863560732113147, 64.644)(21.963394342762065, 65.401)(22.063227953410983, 65.78)(22.163061564059902, 65.132)(22.26289517470882, 65.481)(22.36272878535774, 66.321)(22.462562396006657, 64.771)(22.562396006655575, 65.26)(22.662229617304494, 79.764)(22.762063227953412, 75.22)(22.86189683860233, 80.21)(22.96173044925125, 74.535)(23.061564059900167, 73.048)(23.161397670549086, 65.353)(23.261231281198004, 64.98)(23.361064891846922, 65.967)(23.46089850249584, 65.344)(23.56073211314476, 64.693)(23.660565723793678, 65.473)(23.7603993344426, 65.965)(23.860232945091518, 65.542)(23.960066555740436, 65.493)(24.059900166389355, 65.19)(24.159733777038273, 65.567)(24.25956738768719, 65.181)(24.35940099833611, 65.828)(24.459234608985028, 65.99)(24.559068219633946, 64.937)(24.658901830282865, 65.417)(24.758735440931783, 64.87)(24.8585690515807, 65.653)(24.95840266222962, 64.737)(25.05823627287854, 65.908)(25.158069883527457, 77.091)(25.257903494176375, 78.887)(25.357737104825294, 83.758)(25.457570715474212, 82.378)(25.55740432612313, 66.231)(25.65723793677205, 65.97)(25.757071547420967, 65.717)(25.856905158069885, 65.461)(25.956738768718804, 65.582)(26.056572379367722, 65.302)(26.15640599001664, 65.676)(26.25623960066556, 65.938)(26.356073211314477, 65.127)(26.455906821963396, 65.321)(26.555740432612314, 64.937)(26.655574043261232, 65.91)(26.75540765391015, 65.109)(26.85524126455907, 66.13)(26.955074875207988, 65.289)(27.054908485856906, 65.381)(27.154742096505824, 65.262)(27.254575707154743, 70.295)(27.35440931780366, 66.06)(27.45424292845258, 65.506)(27.5540765391015, 65.323)(27.653910149750416, 65.781)(27.753743760399338, 66.047)(27.853577371048253, 64.885)(27.953410981697175, 69.289)(28.05324459234609, 75.121)(28.15307820299501, 66.762)(28.252911813643927, 65.73)(28.35274542429285, 65.158)(28.452579034941763, 66.036)(28.552412645590685, 66.955)(28.6522462562396, 65.193)(28.752079866888522, 65.094)(28.851913477537437, 65.61)(28.95174708818636, 65.771)(29.051580698835274, 65.758)(29.151414309484196, 65.756)(29.25124792013311, 65.102)(29.351081530782032, 65.444)(29.450915141430947, 65.629)(29.55074875207987, 64.362)(29.650582362728784, 66.926)(29.750415973377706, 65.663)(29.85024958402662, 64.762)(29.950083194675543, 65.335)(30.049916805324457, 66.024)(30.14975041597338, 73.308)(30.249584026622294, 66.416)(30.349417637271216, 65.695)(30.44925124792013, 65.31)(30.549084858569053, 64.637)(30.648918469217968, 65.8)(30.74875207986689, 64.641)(30.848585690515804, 65.727)(30.948419301164726, 65.035)(31.04825291181364, 64.825)(31.148086522462563, 65.419)(31.247920133111478, 70.383)(31.3477537437604, 64.409)(31.447587354409315, 65.918)(31.547420965058237, 64.961)(31.64725457570715, 64.595)(31.747088186356073, 65.664)(31.84692179700499, 66.152)(31.94675540765391, 65.241)(32.046589018302825, 64.371)(32.14642262895175, 67.91)(32.24625623960066, 72.639)(32.346089850249584, 67.547)(32.4459234608985, 67.598)(32.54575707154742, 67.397)(32.645590682196335, 64.893)(32.74542429284526, 65.604)(32.84525790349417, 66.026)(32.9450915141431, 64.806)(33.044925124792016, 65.287)(33.14475873544094, 65.577)(33.24459234608985, 64.868)(33.344425956738775, 64.889)(33.44425956738769, 66.285)(33.54409317803661, 65.107)(33.643926788685526, 66.609)(33.74376039933445, 66.431)(33.84359400998336, 67.318)(33.943427620632285, 65.406)(34.0432612312812, 65.314)(34.14309484193012, 65.336)(34.24292845257904, 66.259)(34.34276206322796, 65.121)(34.44259567387687, 65.889)(34.542429284525795, 64.551)(34.64226289517471, 65.331)(34.74209650582363, 65.652)(34.84193011647255, 65.555)(34.94176372712147, 65.468)(35.04159733777038, 64.784)(35.141430948419305, 65.873)(35.24126455906822, 65.755)(35.34109816971714, 64.856)(35.44093178036606, 65.942)(35.54076539101498, 65.643)(35.640599001663894, 65.007)(35.740432612312816, 64.377)(35.84026622296173, 65.079)(35.94009983361065, 65.376)(36.03993344425957, 65.076)(36.13976705490849, 73.63)(36.239600665557404, 66.237)(36.339434276206326, 64.382)(36.43926788685524, 65.6)(36.53910149750416, 64.697)(36.63893510815308, 65.839)(36.738768718802, 64.493)(36.838602329450914, 64.592)(36.938435940099836, 64.731)(37.03826955074875, 65.115)(37.13810316139767, 64.643)(37.23793677204659, 65.573)(37.33777038269551, 64.933)(37.437603993344425, 65.265)(37.53743760399335, 65.548)(37.63727121464226, 64.964)(37.73710482529118, 65.5)(37.8369384359401, 65.351)(37.93677204658902, 66.453)(38.036605657237935, 65.152)(38.13643926788686, 65.202)(38.23627287853577, 65.639)(38.336106489184694, 66.806)(38.43594009983361, 65.106)(38.53577371048253, 64.594)(38.635607321131445, 64.902)(38.73544093178037, 66.051)(38.83527454242928, 65.03)(38.935108153078204, 65.472)(39.03494176372712, 65.869)(39.13477537437604, 64.484)(39.234608985024956, 65.769)(39.33444259567388, 65.564)(39.43427620632279, 65.424)(39.534109816971714, 65.14)(39.63394342762063, 65.273)(39.73377703826955, 64.882)(39.833610648918466, 65.843)(39.93344425956739, 72.072)(40.0332778702163, 70.01)(40.133111480865225, 72.592)(40.232945091514146, 65.107)(40.33277870216307, 65.568)(40.43261231281198, 65.152)(40.532445923460905, 64.965)(40.63227953410982, 65.391)(40.73211314475874, 65.491)(40.83194675540766, 65.812)(40.93178036605658, 65.318)(41.03161397670549, 65.537)(41.131447587354415, 64.926)(41.23128119800333, 65.143)(41.33111480865225, 66.699)(41.43094841930117, 65.295)(41.53078202995009, 65.597)(41.630615640599004, 65.839)(41.730449251247926, 65.346)(41.83028286189684, 68.995)(41.93011647254576, 71.324)(42.02995008319468, 69.611)(42.1297836938436, 74.126)(42.229617304492514, 69.569)(42.329450915141436, 77.405)(42.42928452579035, 76.715)(42.52911813643927, 75.312)(42.62895174708819, 76.838)(42.72878535773711, 76.111)(42.828618968386024, 76.292)(42.928452579034946, 74.021)(43.02828618968386, 65.278)(43.12811980033278, 64.96)(43.2279534109817, 75.142)(43.32778702163062, 73.635)(43.427620632279535, 73.318)(43.52745424292846, 73.313)(43.62728785357737, 66.745)(43.72712146422629, 65.918)(43.82695507487521, 80.408)(43.92678868552413, 84.394)(44.026622296173045, 80.819)(44.12645590682197, 65.09)(44.22628951747088, 65.418)(44.326123128119804, 65.588)(44.42595673876872, 66.098)(44.52579034941764, 64.822)(44.625623960066555, 65.187)(44.72545757071548, 65.829)(44.82529118136439, 65.902)(44.925124792013314, 64.577)(45.02495840266223, 65.733)(45.12479201331115, 66.6)(45.224625623960065, 65.048)(45.32445923460899, 65.003)(45.4242928452579, 66.142)(45.524126455906824, 65.064)(45.62396006655574, 65.26)(45.72379367720466, 65.13)(45.823627287853576, 65.567)(45.9234608985025, 65.931)(46.02329450915141, 66.586)(46.123128119800334, 65.78)(46.22296173044925, 66.684)(46.32279534109817, 65.857)(46.422628951747086, 65.203)(46.52246256239601, 65.441)(46.62229617304492, 65.381)(46.722129783693845, 65.853)(46.82196339434276, 65.734)(46.92179700499168, 66.404)(47.021630615640596, 65.076)(47.12146422628952, 65.462)(47.22129783693843, 65.09)(47.321131447587355, 65.777)(47.42096505823627, 66.529)(47.5207986688852, 67.142)(47.620632279534114, 65.738)(47.720465890183036, 64.818)(47.82029950083195, 66.133)(47.92013311148087, 65.028)(48.01996672212979, 66.068)(48.11980033277871, 66.407)(48.219633943427624, 66.636)(48.319467554076546, 66.57)(48.41930116472546, 65.196)(48.51913477537438, 64.379)(48.6189683860233, 66.74)(48.71880199667222, 66.555)(48.818635607321134, 64.985)(48.918469217970056, 65.597)(49.01830282861897, 65.627)(49.11813643926789, 65.286)(49.21797004991681, 64.876)(49.31780366056573, 66.625)(49.417637271214645, 65.028)(49.51747088186357, 66.59)(49.61730449251248, 94.633)(49.7171381031614, 65.663)(49.81697171381032, 66.109)(49.91680532445924, 66.445)(50.016638935108155, 65.774)(50.11647254575708, 93.974)(50.21630615640599, 87.279)(50.31613976705491, 83.202)(50.41597337770383, 78.545)(50.51580698835275, 76.827)(50.615640599001665, 78.235)(50.71547420965059, 78.397)(50.8153078202995, 77.257)(50.915141430948424, 76.288)(51.01497504159734, 77.134)(51.11480865224626, 84.077)(51.214642262895175, 79.013)(51.3144758735441, 77.006)(51.41430948419301, 65.926)(51.514143094841934, 66.294)(51.61397670549085, 77.076)(51.71381031613977, 79.035)(51.813643926788686, 75.929)(51.91347753743761, 66.804)(52.01331114808652, 65.003)(52.113144758735444, 67.742)(52.21297836938436, 66.737)(52.31281198003328, 65.068)(52.412645590682196, 66.065)(52.51247920133112, 67.041)(52.61231281198003, 66.478)(52.712146422628955, 67.9)(52.81198003327787, 73.461)(52.91181364392679, 72.783)(53.011647254575706, 81.055)(53.11148086522463, 76.695)(53.21131447587354, 82.771)(53.311148086522465, 77.779)(53.41098169717138, 76.588)(53.5108153078203, 65.599)(53.61064891846922, 65.558)(53.71048252911814, 66.362)(53.81031613976705, 71.432)(53.910149750415975, 66.917)(54.00998336106489, 65.516)(54.10981697171381, 65.528)(54.20965058236273, 65.638)(54.30948419301165, 81.784)(54.409317803660564, 103.717)(54.509151414309486, 88.646)(54.6089850249584, 97.709)(54.70881863560732, 85.685)(54.808652246256244, 96.317)(54.90848585690516, 94.559)(55.00831946755408, 95.33)(55.108153078203, 86.93)(55.20798668885192, 82.477)(55.30782029950083, 64.857)(55.407653910149754, 65.845)(55.507487520798676, 65.541)(55.60732113144759, 65.843)(55.707154742096506, 66.413)(55.80698835274543, 65.362)(55.90682196339435, 68.609)(56.006655574043265, 66.719)(56.10648918469218, 74.15)(56.2063227953411, 87.16)(56.30615640599002, 80.885)(56.40599001663894, 71.15)(56.50582362728785, 65.665)(56.605657237936775, 65.228)(56.7054908485857, 65.537)(56.80532445923461, 65.472)(56.90515806988353, 65.404)(57.00499168053245, 65.698)(57.10482529118137, 68.112)(57.204658901830285, 65.665)(57.3044925124792, 65.952)(57.40432612312812, 65.559)(57.504159733777044, 66.042)(57.60399334442596, 66.31)(57.703826955074874, 67.159)(57.803660565723796, 66.156)(57.90349417637272, 66.293)(58.00332778702163, 67.315)(58.10316139767055, 74.084)(58.20299500831947, 71.271)(58.30282861896839, 65.491)(58.402662229617306, 65.041)(58.50249584026622, 68.446)(58.60232945091514, 65.569)(58.702163061564065, 65.795)(58.80199667221298, 65.987)(58.901830282861894, 65.403)(59.001663893510816, 73.684)(59.10149750415974, 66.47)(59.20133111480865, 65.372)(59.30116472545757, 65.099)(59.40099833610649, 66.564)(59.50083194675541, 66.437)(59.60066555740433, 66.01)(59.70049916805324, 65.975)(59.80033277870216, 66.195)(59.900166389351085, 61.147)(60.0, 12.288)
            };
            \addplot[color=blue, mark=none,name path=B] coordinates { %% MIN value
            (0.0, 36.633)(0.09983361064891848, 38.757)(0.19966722129783696, 50.598)(0.2995008319467554, 36.788)(0.3993344425956739, 48.406)(0.49916805324459235, 42.813)(0.5990016638935108, 44.646)(0.6988352745424293, 45.203)(0.7986688851913478, 36.495)(0.8985024958402663, 43.428)(0.9983361064891847, 32.402)(1.0981697171381033, 47.113)(1.1980033277870217, 42.556)(1.2978369384359403, 38.387)(1.3976705490848587, 38.816)(1.4975041597337773, 34.681)(1.5973377703826956, 36.389)(1.697171381031614, 37.589)(1.7970049916805326, 36.634)(1.896838602329451, 40.764)(1.9966722129783694, 36.045)(2.096505823627288, 36.502)(2.1963394342762066, 37.435)(2.296173044925125, 42.823)(2.3960066555740434, 38.211)(2.4958402662229617, 39.23)(2.5956738768718806, 41.053)(2.695507487520799, 40.945)(2.7953410981697173, 43.922)(2.8951747088186357, 41.399)(2.9950083194675545, 38.624)(3.094841930116473, 36.546)(3.1946755407653913, 38.221)(3.2945091514143097, 37.724)(3.394342762063228, 40.425)(3.494176372712147, 42.847)(3.5940099833610653, 43.027)(3.6938435940099836, 43.271)(3.793677204658902, 37.142)(3.8935108153078204, 45.07)(3.9933444259567388, 38.237)(4.093178036605657, 36.064)(4.193011647254576, 43.657)(4.292845257903495, 44.05)(4.392678868552413, 44.447)(4.492512479201332, 44.402)(4.59234608985025, 47.174)(4.692179700499168, 44.971)(4.792013311148087, 38.858)(4.891846921797005, 45.85)(4.9916805324459235, 46.545)(5.091514143094843, 41.44)(5.191347753743761, 44.459)(5.2911813643926795, 46.777)(5.391014975041598, 40.617)(5.490848585690516, 46.724)(5.590682196339435, 45.288)(5.690515806988353, 48.532)(5.790349417637271, 46.298)(5.89018302828619, 43.07)(5.990016638935109, 46.907)(6.089850249584027, 38.824)(6.189683860232946, 48.571)(6.289517470881864, 46.129)(6.389351081530783, 40.208)(6.489184692179701, 39.112)(6.589018302828619, 40.696)(6.688851913477538, 46.581)(6.788685524126456, 47.804)(6.888519134775375, 41.608)(6.988352745424294, 47.076)(7.088186356073212, 42.607)(7.1880199667221305, 44.614)(7.287853577371049, 39.024)(7.387687188019967, 36.049)(7.487520798668886, 47.392)(7.587354409317804, 41.424)(7.687188019966722, 39.295)(7.787021630615641, 41.16)(7.886855241264559, 46.772)(7.9866888519134775, 44.696)(8.086522462562396, 45.724)(8.186356073211314, 45.177)(8.286189683860234, 45.669)(8.386023294509153, 41.992)(8.485856905158071, 39.771)(8.58569051580699, 44.005)(8.685524126455908, 45.324)(8.785357737104826, 35.476)(8.885191347753745, 37.674)(8.985024958402663, 43.987)(9.084858569051582, 48.857)(9.1846921797005, 43.082)(9.284525790349418, 44.998)(9.384359400998337, 43.804)(9.484193011647255, 35.943)(9.584026622296173, 47.704)(9.683860232945092, 44.873)(9.78369384359401, 43.828)(9.883527454242929, 45.896)(9.983361064891847, 46.866)(10.083194675540767, 43.549)(10.183028286189685, 45.59)(10.282861896838604, 38.51)(10.382695507487522, 46.5)(10.48252911813644, 44.403)(10.582362728785359, 41.638)(10.682196339434277, 48.522)(10.782029950083196, 46.63)(10.881863560732114, 45.266)(10.981697171381033, 45.708)(11.081530782029951, 39.697)(11.18136439267887, 39.023)(11.281198003327788, 40.717)(11.381031613976706, 48.068)(11.480865224625624, 48.836)(11.580698835274543, 42.369)(11.680532445923461, 45.136)(11.78036605657238, 45.841)(11.8801996672213, 45.368)(11.980033277870218, 37.49)(12.079866888519136, 49.73)(12.179700499168055, 48.597)(12.279534109816973, 43.254)(12.379367720465892, 44.908)(12.47920133111481, 43.315)(12.579034941763728, 41.961)(12.678868552412647, 50.821)(12.778702163061565, 44.853)(12.878535773710484, 45.051)(12.978369384359402, 45.547)(13.07820299500832, 42.742)(13.178036605657239, 38.504)(13.277870216306157, 42.864)(13.377703826955075, 47.507)(13.477537437603994, 39.228)(13.577371048252912, 43.527)(13.67720465890183, 43.327)(13.77703826955075, 41.55)(13.876871880199669, 43.181)(13.976705490848587, 42.032)(14.076539101497506, 45.804)(14.176372712146424, 45.183)(14.276206322795343, 48.623)(14.376039933444261, 49.192)(14.47587354409318, 43.307)(14.575707154742098, 43.516)(14.675540765391016, 43.789)(14.775374376039935, 46.052)(14.875207986688853, 42.339)(14.975041597337771, 41.734)(15.07487520798669, 43.78)(15.174708818635608, 48.69)(15.274542429284526, 42.942)(15.374376039933445, 47.406)(15.474209650582363, 42.432)(15.574043261231282, 45.846)(15.6738768718802, 44.84)(15.773710482529118, 46.657)(15.873544093178037, 35.487)(15.973377703826955, 43.65)(16.073211314475873, 45.728)(16.173044925124792, 40.362)(16.27287853577371, 42.087)(16.37271214642263, 42.132)(16.47254575707155, 43.038)(16.57237936772047, 35.943)(16.672212978369387, 47.061)(16.772046589018306, 44.056)(16.871880199667224, 46.777)(16.971713810316142, 47.879)(17.07154742096506, 45.411)(17.17138103161398, 45.272)(17.271214642262898, 44.186)(17.371048252911816, 41.588)(17.470881863560734, 49.01)(17.570715474209653, 44.103)(17.67054908485857, 45.774)(17.77038269550749, 47.24)(17.870216306156408, 40.773)(17.970049916805326, 40.72)(18.069883527454245, 41.444)(18.169717138103163, 41.547)(18.26955074875208, 47.442)(18.369384359401, 40.317)(18.469217970049918, 49.129)(18.569051580698837, 45.793)(18.668885191347755, 47.49)(18.768718801996673, 46.582)(18.86855241264559, 46.107)(18.96838602329451, 45.086)(19.06821963394343, 39.235)(19.168053244592347, 45.841)(19.267886855241265, 48.711)(19.367720465890184, 44.939)(19.467554076539102, 42.889)(19.56738768718802, 46.132)(19.66722129783694, 43.063)(19.767054908485857, 47.046)(19.866888519134775, 45.405)(19.966722129783694, 43.805)(20.066555740432612, 46.063)(20.166389351081534, 45.87)(20.266222961730453, 48.465)(20.36605657237937, 43.754)(20.46589018302829, 45.384)(20.565723793677208, 45.407)(20.665557404326126, 42.135)(20.765391014975044, 46.666)(20.865224625623963, 44.47)(20.96505823627288, 44.993)(21.0648918469218, 46.812)(21.164725457570718, 46.264)(21.264559068219636, 44.297)(21.364392678868555, 47.012)(21.464226289517473, 36.779)(21.56405990016639, 49.037)(21.66389351081531, 38.715)(21.76372712146423, 47.181)(21.863560732113147, 37.35)(21.963394342762065, 45.654)(22.063227953410983, 43.054)(22.163061564059902, 49.535)(22.26289517470882, 35.159)(22.36272878535774, 45.456)(22.462562396006657, 43.212)(22.562396006655575, 48.78)(22.662229617304494, 47.369)(22.762063227953412, 43.629)(22.86189683860233, 45.469)(22.96173044925125, 41.756)(23.061564059900167, 44.546)(23.161397670549086, 41.106)(23.261231281198004, 44.489)(23.361064891846922, 36.034)(23.46089850249584, 47.346)(23.56073211314476, 46.788)(23.660565723793678, 48.903)(23.7603993344426, 46.892)(23.860232945091518, 42.439)(23.960066555740436, 47.563)(24.059900166389355, 38.95)(24.159733777038273, 41.597)(24.25956738768719, 48.63)(24.35940099833611, 40.185)(24.459234608985028, 46.416)(24.559068219633946, 36.175)(24.658901830282865, 45.19)(24.758735440931783, 40.667)(24.8585690515807, 47.936)(24.95840266222962, 48.577)(25.05823627287854, 40.236)(25.158069883527457, 46.01)(25.257903494176375, 44.439)(25.357737104825294, 46.43)(25.457570715474212, 48.435)(25.55740432612313, 34.94)(25.65723793677205, 50.144)(25.757071547420967, 43.037)(25.856905158069885, 43.462)(25.956738768718804, 46.394)(26.056572379367722, 47.254)(26.15640599001664, 47.398)(26.25623960066556, 47.769)(26.356073211314477, 42.729)(26.455906821963396, 48.115)(26.555740432612314, 46.042)(26.655574043261232, 45.813)(26.75540765391015, 39.125)(26.85524126455907, 47.091)(26.955074875207988, 44.475)(27.054908485856906, 42.665)(27.154742096505824, 41.262)(27.254575707154743, 46.757)(27.35440931780366, 49.687)(27.45424292845258, 46.395)(27.5540765391015, 44.864)(27.653910149750416, 47.886)(27.753743760399338, 47.425)(27.853577371048253, 46.487)(27.953410981697175, 48.296)(28.05324459234609, 43.164)(28.15307820299501, 45.666)(28.252911813643927, 46.29)(28.35274542429285, 45.578)(28.452579034941763, 48.139)(28.552412645590685, 47.467)(28.6522462562396, 49.556)(28.752079866888522, 44.511)(28.851913477537437, 45.02)(28.95174708818636, 43.543)(29.051580698835274, 46.529)(29.151414309484196, 39.943)(29.25124792013311, 47.981)(29.351081530782032, 46.438)(29.450915141430947, 39.289)(29.55074875207987, 49.125)(29.650582362728784, 47.059)(29.750415973377706, 46.429)(29.85024958402662, 49.001)(29.950083194675543, 44.724)(30.049916805324457, 38.832)(30.14975041597338, 47.596)(30.249584026622294, 43.205)(30.349417637271216, 37.341)(30.44925124792013, 46.331)(30.549084858569053, 35.604)(30.648918469217968, 43.103)(30.74875207986689, 40.069)(30.848585690515804, 40.554)(30.948419301164726, 38.654)(31.04825291181364, 44.908)(31.148086522462563, 33.246)(31.247920133111478, 48.145)(31.3477537437604, 39.771)(31.447587354409315, 45.278)(31.547420965058237, 48.681)(31.64725457570715, 49.887)(31.747088186356073, 48.816)(31.84692179700499, 36.645)(31.94675540765391, 39.319)(32.046589018302825, 42.674)(32.14642262895175, 48.324)(32.24625623960066, 39.7)(32.346089850249584, 41.68)(32.4459234608985, 46.147)(32.54575707154742, 46.569)(32.645590682196335, 43.536)(32.74542429284526, 43.003)(32.84525790349417, 42.159)(32.9450915141431, 47.777)(33.044925124792016, 41.154)(33.14475873544094, 43.081)(33.24459234608985, 39.359)(33.344425956738775, 46.079)(33.44425956738769, 37.719)(33.54409317803661, 42.928)(33.643926788685526, 46.323)(33.74376039933445, 42.417)(33.84359400998336, 48.5)(33.943427620632285, 44.465)(34.0432612312812, 46.021)(34.14309484193012, 47.351)(34.24292845257904, 42.232)(34.34276206322796, 44.588)(34.44259567387687, 50.813)(34.542429284525795, 43.271)(34.64226289517471, 38.699)(34.74209650582363, 38.718)(34.84193011647255, 45.829)(34.94176372712147, 50.014)(35.04159733777038, 40.263)(35.141430948419305, 49.776)(35.24126455906822, 49.418)(35.34109816971714, 49.4)(35.44093178036606, 46.916)(35.54076539101498, 48.959)(35.640599001663894, 39.002)(35.740432612312816, 38.371)(35.84026622296173, 46.362)(35.94009983361065, 45.759)(36.03993344425957, 46.051)(36.13976705490849, 48.535)(36.239600665557404, 48.482)(36.339434276206326, 45.885)(36.43926788685524, 43.602)(36.53910149750416, 43.16)(36.63893510815308, 46.437)(36.738768718802, 47.33)(36.838602329450914, 43.313)(36.938435940099836, 36.912)(37.03826955074875, 45.98)(37.13810316139767, 48.565)(37.23793677204659, 46.461)(37.33777038269551, 49.449)(37.437603993344425, 40.859)(37.53743760399335, 46.355)(37.63727121464226, 47.875)(37.73710482529118, 41.118)(37.8369384359401, 46.041)(37.93677204658902, 49.457)(38.036605657237935, 48.925)(38.13643926788686, 50.583)(38.23627287853577, 46.831)(38.336106489184694, 34.631)(38.43594009983361, 39.492)(38.53577371048253, 48.3)(38.635607321131445, 43.71)(38.73544093178037, 45.778)(38.83527454242928, 48.681)(38.935108153078204, 49.074)(39.03494176372712, 43.492)(39.13477537437604, 43.508)(39.234608985024956, 49.537)(39.33444259567388, 50.983)(39.43427620632279, 37.276)(39.534109816971714, 48.946)(39.63394342762063, 41.39)(39.73377703826955, 45.439)(39.833610648918466, 48.112)(39.93344425956739, 46.65)(40.0332778702163, 48.942)(40.133111480865225, 48.253)(40.232945091514146, 46.78)(40.33277870216307, 44.977)(40.43261231281198, 46.694)(40.532445923460905, 35.3)(40.63227953410982, 47.113)(40.73211314475874, 48.597)(40.83194675540766, 47.052)(40.93178036605658, 51.201)(41.03161397670549, 48.4)(41.131447587354415, 47.359)(41.23128119800333, 42.142)(41.33111480865225, 43.182)(41.43094841930117, 48.763)(41.53078202995009, 46.904)(41.630615640599004, 46.019)(41.730449251247926, 45.311)(41.83028286189684, 49.778)(41.93011647254576, 44.62)(42.02995008319468, 45.951)(42.1297836938436, 47.537)(42.229617304492514, 45.778)(42.329450915141436, 45.263)(42.42928452579035, 48.228)(42.52911813643927, 43.795)(42.62895174708819, 47.668)(42.72878535773711, 47.667)(42.828618968386024, 45.165)(42.928452579034946, 48.127)(43.02828618968386, 45.569)(43.12811980033278, 49.658)(43.2279534109817, 50.167)(43.32778702163062, 49.126)(43.427620632279535, 49.271)(43.52745424292846, 38.287)(43.62728785357737, 43.685)(43.72712146422629, 46.665)(43.82695507487521, 41.243)(43.92678868552413, 47.3)(44.026622296173045, 48.525)(44.12645590682197, 47.118)(44.22628951747088, 45.361)(44.326123128119804, 42.088)(44.42595673876872, 44.498)(44.52579034941764, 45.397)(44.625623960066555, 49.879)(44.72545757071548, 45.985)(44.82529118136439, 48.38)(44.925124792013314, 49.853)(45.02495840266223, 49.72)(45.12479201331115, 45.451)(45.224625623960065, 44.742)(45.32445923460899, 49.692)(45.4242928452579, 46.573)(45.524126455906824, 47.376)(45.62396006655574, 49.916)(45.72379367720466, 51.38)(45.823627287853576, 44.865)(45.9234608985025, 47.298)(46.02329450915141, 46.216)(46.123128119800334, 49.752)(46.22296173044925, 48.513)(46.32279534109817, 48.181)(46.422628951747086, 49.616)(46.52246256239601, 46.475)(46.62229617304492, 47.012)(46.722129783693845, 46.987)(46.82196339434276, 45.368)(46.92179700499168, 47.823)(47.021630615640596, 50.505)(47.12146422628952, 49.986)(47.22129783693843, 46.22)(47.321131447587355, 49.681)(47.42096505823627, 48.05)(47.5207986688852, 51.063)(47.620632279534114, 39.223)(47.720465890183036, 49.378)(47.82029950083195, 44.562)(47.92013311148087, 50.643)(48.01996672212979, 48.083)(48.11980033277871, 47.246)(48.219633943427624, 49.676)(48.319467554076546, 44.089)(48.41930116472546, 49.723)(48.51913477537438, 48.685)(48.6189683860233, 50.164)(48.71880199667222, 43.797)(48.818635607321134, 49.052)(48.918469217970056, 50.535)(49.01830282861897, 48.569)(49.11813643926789, 48.535)(49.21797004991681, 49.35)(49.31780366056573, 45.343)(49.417637271214645, 43.771)(49.51747088186357, 49.634)(49.61730449251248, 48.953)(49.7171381031614, 48.753)(49.81697171381032, 48.993)(49.91680532445924, 46.132)(50.016638935108155, 48.887)(50.11647254575708, 49.803)(50.21630615640599, 49.234)(50.31613976705491, 45.231)(50.41597337770383, 47.303)(50.51580698835275, 45.696)(50.615640599001665, 49.572)(50.71547420965059, 50.02)(50.8153078202995, 49.053)(50.915141430948424, 45.558)(51.01497504159734, 48.91)(51.11480865224626, 49.17)(51.214642262895175, 48.003)(51.3144758735441, 48.37)(51.41430948419301, 42.737)(51.514143094841934, 49.805)(51.61397670549085, 45.193)(51.71381031613977, 50.05)(51.813643926788686, 39.535)(51.91347753743761, 49.002)(52.01331114808652, 49.368)(52.113144758735444, 47.567)(52.21297836938436, 42.404)(52.31281198003328, 49.964)(52.412645590682196, 42.471)(52.51247920133112, 52.14)(52.61231281198003, 29.535)(52.712146422628955, 42.654)(52.81198003327787, 51.055)(52.91181364392679, 39.326)(53.011647254575706, 48.432)(53.11148086522463, 49.686)(53.21131447587354, 46.345)(53.311148086522465, 48.565)(53.41098169717138, 44.277)(53.5108153078203, 48.641)(53.61064891846922, 50.827)(53.71048252911814, 50.242)(53.81031613976705, 46.435)(53.910149750415975, 47.777)(54.00998336106489, 47.16)(54.10981697171381, 48.174)(54.20965058236273, 49.15)(54.30948419301165, 51.909)(54.409317803660564, 53.352)(54.509151414309486, 48.854)(54.6089850249584, 49.371)(54.70881863560732, 48.98)(54.808652246256244, 46.271)(54.90848585690516, 47.88)(55.00831946755408, 48.909)(55.108153078203, 46.628)(55.20798668885192, 49.222)(55.30782029950083, 51.126)(55.407653910149754, 49.97)(55.507487520798676, 47.37)(55.60732113144759, 47.286)(55.707154742096506, 38.672)(55.80698835274543, 50.463)(55.90682196339435, 42.791)(56.006655574043265, 45.449)(56.10648918469218, 50.723)(56.2063227953411, 50.015)(56.30615640599002, 47.381)(56.40599001663894, 50.146)(56.50582362728785, 51.295)(56.605657237936775, 52.172)(56.7054908485857, 47.664)(56.80532445923461, 49.282)(56.90515806988353, 49.23)(57.00499168053245, 51.388)(57.10482529118137, 50.001)(57.204658901830285, 49.997)(57.3044925124792, 46.538)(57.40432612312812, 51.632)(57.504159733777044, 50.949)(57.60399334442596, 48.809)(57.703826955074874, 49.471)(57.803660565723796, 47.665)(57.90349417637272, 45.413)(58.00332778702163, 47.347)(58.10316139767055, 50.263)(58.20299500831947, 47.673)(58.30282861896839, 49.229)(58.402662229617306, 48.349)(58.50249584026622, 48.917)(58.60232945091514, 49.648)(58.702163061564065, 50.771)(58.80199667221298, 48.023)(58.901830282861894, 50.421)(59.001663893510816, 50.74)(59.10149750415974, 50.592)(59.20133111480865, 53.856)(59.30116472545757, 48.806)(59.40099833610649, 48.86)(59.50083194675541, 46.815)(59.60066555740433, 42.165)(59.70049916805324, 49.345)(59.80033277870216, 38.546)(59.900166389351085, 23.046)(60.0, 12.288)
            };
            \addplot [pattern=north east lines,pattern color=red] 
            fill between [
                of=A and B,soft clip={domain=0:800},
            ];
            \end{axis}
    \end{tikzpicture}
    \caption{DUT: Workstation}\label{fig:time_series_Fankuch_IntelPowerGadgetWorkstation}      
\end{subfigure}
\begin{subfigure}[b]{0.49\linewidth}
    \begin{tikzpicture}
        \pgfplotsset{%
        width=1\linewidth,
        % height=1\textheight
        }
        \begin{axis}[ymax=120,
            xlabel={Time (Seconds)},
            ylabel={Energy Consumption (Joules)},
            ]
            \addplot[color=blue, mark=none,] coordinates { %% AVG value
            (0.0, 17.128674796747966)(0.09966777408637874, 17.387512195121946)(0.19933554817275748, 17.370747967479673)(0.2990033222591362, 17.19963414634147)(0.39867109634551495, 17.185447154471547)(0.49833887043189373, 17.142227642276428)(0.5980066445182723, 16.974829268292677)(0.6976744186046512, 16.97786991869918)(0.7973421926910299, 16.978284552845533)(0.8970099667774087, 16.970373983739837)(0.9966777408637875, 17.014528455284555)(1.0963455149501662, 16.949292682926824)(1.1960132890365447, 16.93062601626016)(1.2956810631229236, 16.936162601626016)(1.3953488372093024, 16.922983739837402)(1.495016611295681, 16.878178861788623)(1.5946843853820598, 16.9389024390244)(1.6943521594684385, 16.917284552845526)(1.7940199335548175, 16.937813008130078)(1.893687707641196, 16.941747967479674)(1.993355481727575, 17.018040650406512)(2.0930232558139537, 17.10226829268293)(2.1926910299003324, 17.03450406504065)(2.292358803986711, 17.060536585365853)(2.3920265780730894, 17.05665040650407)(2.4916943521594686, 17.056235772357724)(2.5913621262458473, 17.07672357723578)(2.691029900332226, 17.125626016260163)(2.7906976744186047, 17.147650406504063)(2.8903654485049834, 17.181894308943093)(2.990033222591362, 17.27426829268293)(3.089700996677741, 17.30595934959349)(3.1893687707641196, 17.27987804878049)(3.2890365448504983, 17.32153658536586)(3.388704318936877, 17.356130081300815)(3.488372093023256, 17.35347967479675)(3.588039867109635, 17.3759105691057)(3.6877076411960132, 17.38952032520325)(3.787375415282392, 17.37138211382113)(3.887043189368771, 17.306536585365855)(3.98671096345515, 17.25934146341463)(4.086378737541528, 17.24744715447155)(4.186046511627907, 17.210048780487803)(4.285714285714286, 17.2119024390244)(4.385382059800665, 17.221008130081298)(4.485049833887043, 17.217113821138206)(4.584717607973422, 17.198178861788616)(4.6843853820598005, 17.160349593495933)(4.784053156146179, 17.174373983739834)(4.883720930232559, 16.985186991869917)(4.983388704318937, 17.067780487804875)(5.083056478405315, 17.03863414634146)(5.1827242524916945, 16.931804878048776)(5.282392026578074, 16.872341463414635)(5.382059800664452, 16.827520325203256)(5.48172757475083, 16.759414634146346)(5.5813953488372094, 16.670252032520327)(5.681063122923588, 16.576991869918704)(5.780730897009967, 16.598878048780485)(5.880398671096346, 16.592130081300812)(5.980066445182724, 16.60254471544715)(6.079734219269103, 16.63350406504065)(6.179401993355482, 16.626471544715447)(6.279069767441861, 16.661479674796745)(6.378737541528239, 16.693382113821137)(6.4784053156146175, 16.732560975609758)(6.578073089700997, 16.723918699186992)(6.677740863787376, 16.738113821138217)(6.777408637873754, 16.73581300813008)(6.877076411960133, 16.677674796747972)(6.976744186046512, 16.705780487804876)(7.076411960132891, 16.668609756097563)(7.17607973421927, 16.657219512195113)(7.275747508305648, 16.632886178861792)(7.3754152823920265, 16.641325203252034)(7.475083056478405, 16.60349593495935)(7.574750830564784, 16.59656097560975)(7.674418604651163, 16.588455284552847)(7.774086378737542, 16.566658536585372)(7.8737541528239205, 16.586869918699193)(7.9734219269103, 16.55537398373983)(8.073089700996677, 16.53850406504064)(8.172757475083056, 16.39751219512196)(8.272425249169435, 16.47158536585366)(8.372093023255815, 16.400333333333332)(8.471760797342194, 16.35720325203252)(8.571428571428571, 16.347406504065034)(8.67109634551495, 16.323634146341465)(8.77076411960133, 16.31221951219512)(8.870431893687707, 16.228593495934966)(8.970099667774086, 16.219804878048773)(9.069767441860465, 16.231682926829272)(9.169435215946844, 16.23080487804878)(9.269102990033224, 16.262837398373986)(9.368770764119601, 16.25840650406505)(9.46843853820598, 16.27274796747968)(9.568106312292358, 16.277089430894314)(9.667774086378738, 16.30827642276423)(9.767441860465118, 16.328926829268294)(9.867109634551495, 16.33971544715448)(9.966777408637874, 16.3539430894309)(10.066445182724253, 16.44246341463415)(10.16611295681063, 16.348170731707313)(10.26578073089701, 16.391617886178857)(10.365448504983389, 16.295495934959344)(10.465116279069768, 16.344536585365855)(10.564784053156147, 16.2989837398374)(10.664451827242525, 16.284325203252035)(10.764119601328904, 16.305056910569107)(10.863787375415281, 16.199146341463415)(10.96345514950166, 16.290089430894312)(11.063122923588042, 16.253235772357726)(11.162790697674419, 16.239601626016256)(11.262458471760798, 16.21117886178861)(11.362126245847175, 16.25333333333333)(11.461794019933555, 16.230162601626013)(11.561461794019934, 16.214260162601626)(11.661129568106311, 16.186113821138207)(11.760797342192692, 16.143512195121954)(11.860465116279071, 16.12030081300813)(11.960132890365449, 16.04508130081301)(12.059800664451828, 16.048487804878054)(12.159468438538205, 15.977983739837393)(12.259136212624584, 15.873512195121961)(12.358803986710964, 15.980130081300812)(12.458471760797343, 15.933829268292676)(12.558139534883722, 15.947268292682933)(12.6578073089701, 15.951642276422755)(12.757475083056478, 15.947308943089437)(12.857142857142858, 15.914195121951215)(12.956810631229235, 15.926943089430898)(13.056478405315616, 15.970910569105689)(13.156146179401993, 15.908747967479675)(13.255813953488373, 15.934577235772359)(13.355481727574752, 15.905886178861795)(13.455149501661129, 15.924544715447151)(13.554817275747508, 15.910455284552842)(13.654485049833887, 15.91190243902439)(13.754152823920267, 15.947715447154472)(13.853820598006646, 15.902495934959349)(13.953488372093023, 15.820162601626016)(14.053156146179402, 15.855146341463408)(14.152823920265782, 15.841032520325193)(14.252491694352159, 15.866991869918703)(14.35215946843854, 15.777447154471547)(14.451827242524917, 15.820097560975611)(14.551495016611296, 15.741902439024395)(14.651162790697676, 15.786398373983737)(14.750830564784053, 15.723390243902434)(14.850498338870432, 15.727333333333322)(14.95016611295681, 15.63120325203252)(15.049833887043192, 15.650495934959356)(15.149501661129568, 15.618650406504063)(15.249169435215947, 15.5269756097561)(15.348837209302326, 15.668048780487801)(15.448504983388705, 15.606813008130079)(15.548172757475085, 15.619203252032518)(15.64784053156146, 15.658252032520334)(15.747508305647841, 15.638699186991866)(15.84717607973422, 15.63737398373984)(15.9468438538206, 15.648040650406497)(16.04651162790698, 15.647186991869916)(16.146179401993354, 15.629398373983742)(16.245847176079735, 15.596065040650407)(16.345514950166113, 15.598666666666665)(16.445182724252494, 15.544609756097563)(16.54485049833887, 15.57682113821139)(16.64451827242525, 15.613138211382108)(16.74418604651163, 15.613333333333326)(16.843853820598007, 15.606219512195121)(16.943521594684388, 15.597211382113816)(17.043189368770765, 15.604065040650418)(17.142857142857142, 15.553569105691057)(17.24252491694352, 15.593186991869919)(17.3421926910299, 15.519626016260165)(17.44186046511628, 15.51166666666667)(17.54152823920266, 15.528243902439025)(17.641196013289036, 15.42456097560976)(17.740863787375414, 15.433504065040646)(17.840531561461795, 15.468869918699188)(17.940199335548172, 15.344821138211385)(18.039867109634553, 15.395284552845531)(18.13953488372093, 15.42947967479675)(18.239202657807308, 15.399658536585365)(18.33887043189369, 15.428804878048783)(18.438538205980066, 15.43789430894309)(18.538205980066447, 15.439308943089436)(18.63787375415282, 15.44939837398374)(18.737541528239202, 15.45134959349593)(18.837209302325583, 15.40125203252032)(18.93687707641196, 15.386373983739835)(19.03654485049834, 15.355918699186992)(19.136212624584715, 15.245089430894305)(19.235880398671096, 15.372065040650401)(19.335548172757477, 15.362544715447152)(19.435215946843854, 15.309341463414633)(19.534883720930235, 15.321195121951218)(19.634551495016613, 15.303040650406501)(19.73421926910299, 15.325674796747972)(19.833887043189367, 15.319398373983741)(19.93355481727575, 15.286065040650403)(20.033222591362126, 15.300861788617883)(20.132890365448507, 15.270333333333333)(20.232558139534884, 15.243739837398378)(20.33222591362126, 15.31014634146342)(20.431893687707642, 15.31720325203252)(20.53156146179402, 15.30855284552846)(20.6312292358804, 15.318154471544707)(20.730897009966778, 15.336130081300809)(20.830564784053156, 15.3269918699187)(20.930232558139537, 15.351634146341464)(21.029900332225914, 15.298650406504064)(21.129568106312295, 15.270504065040653)(21.22923588039867, 15.30738211382114)(21.32890365448505, 15.280512195121956)(21.42857142857143, 15.304658536585363)(21.528239202657808, 15.264788617886182)(21.62790697674419, 15.225560975609753)(21.727574750830563, 15.224829268292691)(21.827242524916944, 15.249609756097563)(21.92691029900332, 15.288617886178866)(22.026578073089702, 15.253260162601629)(22.126245847176083, 15.241959349593508)(22.225913621262457, 15.276268292682929)(22.325581395348838, 15.259105691056922)(22.425249169435215, 15.238390243902439)(22.524916943521596, 15.289162601626014)(22.624584717607974, 15.311715447154477)(22.72425249169435, 15.407105691056914)(22.823920265780732, 15.515739837398375)(22.92358803986711, 15.490284552845532)(23.02325581395349, 15.311097560975616)(23.122923588039868, 15.25035772357724)(23.222591362126245, 15.233999999999998)(23.322259136212622, 15.286439024390248)(23.421926910299003, 15.290707317073164)(23.521594684385384, 15.28358536585366)(23.62126245847176, 15.275113821138211)(23.720930232558143, 15.313861788617888)(23.820598006644516, 15.201723577235768)(23.920265780730897, 15.251536585365859)(24.01993355481728, 15.236764227642274)(24.119601328903656, 15.166219512195118)(24.219269102990037, 15.196569105691056)(24.31893687707641, 15.155756097560968)(24.41860465116279, 15.236894308943091)(24.51827242524917, 15.232284552845531)(24.61794019933555, 15.229512195121954)(24.717607973421927, 15.173138211382112)(24.817275747508305, 15.07504065040651)(24.916943521594686, 15.173626016260169)(25.016611295681063, 15.156536585365846)(25.116279069767444, 15.17361788617887)(25.21594684385382, 15.1349756097561)(25.3156146179402, 15.069284552845527)(25.41528239202658, 15.093121951219512)(25.514950166112957, 15.065463414634149)(25.614617940199338, 15.158422764227643)(25.714285714285715, 15.15454471544716)(25.813953488372093, 15.124089430894307)(25.91362126245847, 15.089073170731709)(26.01328903654485, 15.104308943089425)(26.112956810631232, 15.125235772357724)(26.21262458471761, 15.124422764227637)(26.312292358803987, 15.110260162601621)(26.411960132890364, 15.067203252032522)(26.511627906976745, 15.068910569105686)(26.611295681063122, 15.05218699186991)(26.710963455149503, 15.049333333333337)(26.81063122923588, 15.07465040650407)(26.910299003322258, 15.042040650406506)(27.00996677740864, 15.069073170731711)(27.109634551495017, 15.078975609756096)(27.209302325581397, 15.126886178861794)(27.308970099667775, 15.120325203252035)(27.408637873754152, 15.113081300813006)(27.508305647840533, 15.04353658536585)(27.60797342192691, 15.075536585365855)(27.70764119601329, 15.115300813008126)(27.80730897009967, 15.097113821138212)(27.906976744186046, 15.11022764227642)(28.006644518272424, 15.104487804878048)(28.106312292358805, 15.120512195121949)(28.205980066445186, 15.110780487804874)(28.305647840531563, 15.095471544715442)(28.40531561461794, 15.055284552845533)(28.504983388704318, 15.091333333333331)(28.6046511627907, 15.107837398373983)(28.70431893687708, 15.122495934959355)(28.803986710963457, 15.149504065040654)(28.903654485049834, 15.172951219512202)(29.003322259136212, 15.072951219512195)(29.102990033222593, 15.077926829268286)(29.20265780730897, 15.112699186991872)(29.30232558139535, 15.046617886178863)(29.40199335548173, 15.012674796747973)(29.501661129568106, 14.978455284552853)(29.601328903654487, 15.046292682926829)(29.700996677740864, 15.049203252032523)(29.800664451827245, 15.006504065040653)(29.90033222591362, 15.02177235772357)(30.0, 14.964357723577237)(30.099667774086384, 15.03559349593496)(30.19933554817276, 15.064528455284554)(30.299003322259136, 15.017357723577238)(30.398671096345517, 15.013170731707321)(30.498338870431894, 15.006902439024392)(30.598006644518275, 14.839601626016258)(30.697674418604652, 14.97189430894309)(30.79734219269103, 14.930821138211382)(30.89700996677741, 14.95253658536585)(30.996677740863788, 14.851308943089435)(31.09634551495017, 14.89309756097561)(31.196013289036546, 14.926040650406504)(31.29568106312292, 14.91152845528455)(31.395348837209305, 14.923203252032518)(31.495016611295682, 15.006569105691058)(31.594684385382063, 14.955512195121942)(31.69435215946844, 14.994390243902446)(31.794019933554814, 14.968487804878043)(31.8936877076412, 14.935715447154475)(31.993355481727576, 14.974739837398376)(32.09302325581396, 14.793333333333337)(32.19269102990033, 14.937252032520327)(32.29235880398671, 14.93315447154472)(32.39202657807309, 14.927813008130085)(32.49169435215947, 14.902089430894305)(32.59136212624585, 14.83421138211382)(32.691029900332225, 14.924918699186994)(32.7906976744186, 14.88750406504065)(32.89036544850499, 14.964260162601619)(32.990033222591364, 14.937723577235774)(33.08970099667774, 14.902634146341464)(33.18936877076412, 14.921739837398373)(33.2890365448505, 14.94820325203252)(33.38870431893688, 14.936463414634146)(33.48837209302326, 14.882999999999996)(33.588039867109636, 14.96141463414634)(33.68770764119601, 14.949292682926822)(33.78737541528239, 14.92102439024391)(33.887043189368775, 14.890195121951216)(33.98671096345515, 14.907601626016257)(34.08637873754153, 14.929723577235766)(34.18604651162791, 14.894707317073173)(34.285714285714285, 14.924943089430892)(34.38538205980067, 14.922536585365858)(34.48504983388704, 14.95469918699187)(34.584717607973424, 15.016373983739836)(34.6843853820598, 14.897951219512189)(34.78405315614618, 14.924008130081306)(34.88372093023256, 14.808544715447153)(34.98338870431893, 14.881390243902446)(35.08305647840532, 14.919617886178864)(35.182724252491695, 14.840626016260158)(35.28239202657807, 14.892650406504064)(35.38205980066445, 14.86561788617886)(35.48172757475083, 14.903487804878042)(35.58139534883721, 14.838560975609756)(35.68106312292359, 14.830252032520324)(35.78073089700997, 14.821569105691049)(35.880398671096344, 14.790934959349595)(35.98006644518272, 14.768186991869921)(36.079734219269106, 14.799260162601632)(36.179401993355484, 14.851780487804868)(36.27906976744186, 14.795333333333327)(36.37873754152824, 14.798203252032511)(36.478405315614616, 14.810154471544719)(36.578073089701, 14.785528455284556)(36.67774086378738, 14.841878048780488)(36.777408637873755, 14.813861788617887)(36.87707641196013, 14.802650406504059)(36.97674418604651, 14.795983739837403)(37.076411960132894, 14.774341463414638)(37.17607973421927, 14.766967479674797)(37.27574750830564, 14.810048780487797)(37.37541528239203, 14.793674796747966)(37.475083056478404, 14.770609756097558)(37.57475083056479, 14.79769105691056)(37.674418604651166, 14.85287804878048)(37.774086378737536, 14.800341463414636)(37.87375415282392, 14.799991869918705)(37.9734219269103, 14.786943089430892)(38.07308970099668, 14.786105691056907)(38.17275747508306, 14.819495934959345)(38.27242524916943, 14.830235772357721)(38.372093023255815, 14.761764227642272)(38.47176079734219, 14.816357723577234)(38.57142857142858, 14.805382113821144)(38.671096345514954, 14.824146341463413)(38.77076411960133, 14.802617886178865)(38.87043189368771, 14.800674796747964)(38.970099667774086, 14.678829268292686)(39.06976744186047, 14.784829268292683)(39.16943521594684, 14.792650406504066)(39.269102990033225, 14.77858536585366)(39.3687707641196, 14.797154471544719)(39.46843853820598, 14.796813008130087)(39.568106312292365, 14.810512195121952)(39.667774086378735, 14.817772357723573)(39.76744186046512, 14.756024390243896)(39.8671096345515, 14.743585365853662)(39.966777408637874, 14.731252032520327)(40.06644518272425, 14.704731707317068)(40.16611295681063, 14.732756097560971)(40.26578073089701, 14.757317073170732)(40.36544850498339, 14.75921138211382)(40.46511627906977, 14.758674796747968)(40.564784053156146, 14.753813008130084)(40.66445182724252, 14.777788617886179)(40.76411960132891, 14.759219512195118)(40.863787375415285, 14.754333333333326)(40.96345514950166, 14.689674796747969)(41.06312292358804, 14.748821138211378)(41.16279069767442, 14.74512195121951)(41.2624584717608, 14.797040650406498)(41.36212624584718, 14.732617886178865)(41.461794019933556, 14.760536585365859)(41.561461794019934, 14.755788617886182)(41.66112956810631, 14.805634146341456)(41.760797342192696, 14.768593495934956)(41.86046511627907, 14.71889430894309)(41.96013289036544, 14.741414634146341)(42.05980066445183, 14.765569105691059)(42.159468438538205, 14.745617886178863)(42.25913621262459, 14.738252032520327)(42.35880398671097, 14.758081300812998)(42.45847176079734, 14.695569105691055)(42.55813953488372, 14.721650406504063)(42.6578073089701, 14.74465040650407)(42.757475083056484, 14.704999999999998)(42.85714285714286, 14.724024390243907)(42.95681063122923, 14.704073170731707)(43.056478405315616, 14.719373983739835)(43.15614617940199, 14.747739837398372)(43.25581395348838, 14.720260162601626)(43.355481727574755, 14.733552845528457)(43.455149501661126, 14.706130081300815)(43.55481727574751, 14.724821138211386)(43.65448504983389, 14.669170731707316)(43.75415282392027, 14.66753658536586)(43.85382059800664, 14.723105691056919)(43.95348837209302, 14.71687804878049)(44.053156146179404, 14.657772357723578)(44.15282392026578, 14.688471544715457)(44.252491694352166, 14.692195121951217)(44.352159468438536, 14.696349593495931)(44.451827242524914, 14.705105691056914)(44.5514950166113, 14.732016260162602)(44.651162790697676, 14.741975609756087)(44.75083056478405, 14.726666666666663)(44.85049833887043, 14.71608130081301)(44.95016611295681, 14.651918699186988)(45.04983388704319, 14.663772357723584)(45.14950166112957, 14.719178861788622)(45.24916943521595, 14.707869918699188)(45.348837209302324, 14.724601626016252)(45.4485049833887, 14.676203252032515)(45.548172757475086, 14.703512195121954)(45.647840531561464, 14.70965853658537)(45.74750830564784, 14.72292682926829)(45.84717607973422, 14.695504065040645)(45.946843853820596, 14.712926829268284)(46.04651162790698, 14.697357723577234)(46.14617940199336, 14.694707317073169)(46.245847176079735, 14.734894308943094)(46.34551495016611, 14.694634146341462)(46.44518272425249, 14.72381300813008)(46.544850498338874, 14.721918699186983)(46.644518272425245, 14.727398373983744)(46.74418604651163, 14.697455284552845)(46.84385382059801, 14.676585365853654)(46.94352159468439, 14.706439024390246)(47.04318936877077, 14.685219512195125)(47.14285714285714, 14.677780487804878)(47.24252491694352, 14.698016260162602)(47.3421926910299, 14.69633333333333)(47.441860465116285, 14.684878048780487)(47.54152823920266, 14.715414634146327)(47.64119601328903, 14.7029837398374)(47.74086378737542, 14.671414634146341)(47.840531561461795, 14.699008130081296)(47.94019933554818, 14.69144715447154)(48.03986710963456, 14.667235772357719)(48.13953488372093, 14.671048780487805)(48.23920265780731, 14.719211382113823)(48.33887043189369, 14.703796747967482)(48.43853820598007, 14.630609756097556)(48.538205980066444, 14.623943089430893)(48.63787375415282, 14.748317073170732)(48.737541528239205, 14.687097560975603)(48.83720930232558, 14.696292682926831)(48.93687707641197, 14.64206504065041)(49.03654485049834, 14.58305691056911)(49.136212624584715, 14.654203252032524)(49.2358803986711, 14.666040650406519)(49.33554817275748, 14.604609756097569)(49.435215946843854, 14.665821138211383)(49.53488372093023, 14.664829268292683)(49.63455149501661, 14.670008130081305)(49.734219269102994, 14.714186991869926)(49.83388704318937, 14.644357723577235)(49.93355481727575, 14.70815447154472)(50.033222591362126, 14.667463414634147)(50.1328903654485, 14.693121951219519)(50.23255813953489, 14.70990243902438)(50.332225913621265, 14.640560975609755)(50.43189368770764, 14.651520325203263)(50.53156146179402, 14.662406504065045)(50.6312292358804, 14.698951219512193)(50.73089700996678, 14.661471544715445)(50.83056478405316, 14.676869918699184)(50.93023255813954, 14.709528455284552)(51.029900332225914, 14.656756097560974)(51.12956810631229, 14.703300813008127)(51.229235880398676, 14.669056910569104)(51.328903654485046, 14.665544715447153)(51.42857142857143, 14.64924390243902)(51.52823920265781, 14.709504065040651)(51.627906976744185, 14.702048780487806)(51.72757475083057, 14.669219512195124)(51.82724252491694, 14.677544715447155)(51.926910299003325, 14.599739837398378)(52.0265780730897, 14.639699186991876)(52.12624584717608, 14.653902439024392)(52.225913621262464, 14.668219512195122)(52.325581395348834, 14.628471544715447)(52.42524916943522, 14.622341463414624)(52.524916943521596, 14.657560975609753)(52.62458471760797, 14.697008130081304)(52.72425249169436, 14.635252032520324)(52.82392026578073, 14.687894308943084)(52.92358803986711, 14.682796747967485)(53.02325581395349, 14.64618699186992)(53.12292358803987, 14.569065040650408)(53.222591362126245, 14.590983739837396)(53.32225913621262, 14.569089430894307)(53.42192691029901, 14.591902439024397)(53.521594684385384, 14.589390243902452)(53.62126245847176, 14.552235772357713)(53.72093023255814, 14.594341463414633)(53.820598006644516, 14.613390243902439)(53.9202657807309, 14.596650406504063)(54.01993355481728, 14.586349593495939)(54.119601328903656, 14.562552845528465)(54.21926910299003, 14.532731707317073)(54.31893687707641, 14.605609756097557)(54.418604651162795, 14.587650406504062)(54.51827242524917, 14.589764227642279)(54.61794019933555, 14.618512195121948)(54.71760797342193, 14.562186991869922)(54.817275747508305, 14.572422764227642)(54.91694352159469, 14.54637398373984)(55.01661129568107, 14.521869918699196)(55.11627906976744, 14.55334959349593)(55.21594684385382, 14.557569105691059)(55.3156146179402, 14.600471544715443)(55.41528239202658, 14.523447154471537)(55.51495016611296, 14.624032520325196)(55.61461794019934, 14.696739837398372)(55.714285714285715, 14.802983739837398)(55.81395348837209, 14.735146341463405)(55.91362126245848, 14.632260162601623)(56.01328903654485, 14.472471544715445)(56.11295681063123, 14.536959349593497)(56.21262458471761, 14.570747967479676)(56.31229235880399, 14.515040650406508)(56.41196013289037, 14.516382113821141)(56.51162790697674, 14.503617886178857)(56.611295681063126, 14.513292682926831)(56.7109634551495, 14.454325203252035)(56.81063122923588, 14.564065040650403)(56.910299003322265, 14.510024390243904)(57.009966777408636, 14.480195121951224)(57.10963455149502, 14.528260162601628)(57.2093023255814, 14.530105691056907)(57.308970099667775, 14.505146341463403)(57.40863787375416, 14.395861788617895)(57.50830564784053, 14.447414634146341)(57.607973421926914, 14.490317073170733)(57.70764119601329, 14.440569105691052)(57.80730897009967, 14.473585365853657)(57.906976744186046, 14.442170731707318)(58.006644518272424, 14.458308943089435)(58.10631229235881, 14.420650406504064)(58.205980066445186, 14.46369918699187)(58.30564784053156, 14.415479674796751)(58.40531561461794, 14.44243902439024)(58.50498338870432, 14.422170731707316)(58.6046511627907, 14.383048780487805)(58.70431893687708, 14.433626016260169)(58.80398671096346, 14.452398373983739)(58.903654485049834, 14.445211382113811)(59.00332225913621, 14.395943089430892)(59.102990033222596, 14.391861788617883)(59.202657807308974, 14.430756097560977)(59.30232558139535, 14.429609756097559)(59.40199335548173, 14.406674796747975)(59.501661129568106, 14.4339674796748)(59.60132890365449, 14.4690731707317)(59.70099667774087, 14.358975609756095)(59.80066445182724, 13.237148760330578)(59.90033222591362, 8.750384615384613)(60.0, 12.736)
            };
            \addplot[color=blue, mark=none,name path=A] coordinates { %% MAX value
            (0.0, 18.584)(0.09966777408637874, 18.453)(0.19933554817275748, 18.378)(0.2990033222591362, 18.271)(0.39867109634551495, 18.42)(0.49833887043189373, 18.561)(0.5980066445182723, 18.316)(0.6976744186046512, 17.988)(0.7973421926910299, 18.46)(0.8970099667774087, 17.775)(0.9966777408637875, 18.007)(1.0963455149501662, 17.822)(1.1960132890365447, 17.649)(1.2956810631229236, 17.937)(1.3953488372093024, 18.023)(1.495016611295681, 18.013)(1.5946843853820598, 17.658)(1.6943521594684385, 17.935)(1.7940199335548175, 17.826)(1.893687707641196, 17.939)(1.993355481727575, 17.946)(2.0930232558139537, 17.875)(2.1926910299003324, 18.239)(2.292358803986711, 18.013)(2.3920265780730894, 17.949)(2.4916943521594686, 17.98)(2.5913621262458473, 18.098)(2.691029900332226, 18.49)(2.7906976744186047, 18.203)(2.8903654485049834, 18.726)(2.990033222591362, 18.316)(3.089700996677741, 18.284)(3.1893687707641196, 18.651)(3.2890365448504983, 18.359)(3.388704318936877, 18.46)(3.488372093023256, 18.667)(3.588039867109635, 18.432)(3.6877076411960132, 18.679)(3.787375415282392, 18.71)(3.887043189368771, 18.902)(3.98671096345515, 18.459)(4.086378737541528, 18.385)(4.186046511627907, 18.603)(4.285714285714286, 18.252)(4.385382059800665, 18.423)(4.485049833887043, 18.255)(4.584717607973422, 18.407)(4.6843853820598005, 18.471)(4.784053156146179, 18.853)(4.883720930232559, 18.369)(4.983388704318937, 18.647)(5.083056478405315, 18.482)(5.1827242524916945, 18.276)(5.282392026578074, 18.384)(5.382059800664452, 18.34)(5.48172757475083, 18.017)(5.5813953488372094, 18.196)(5.681063122923588, 18.588)(5.780730897009967, 17.959)(5.880398671096346, 17.92)(5.980066445182724, 18.326)(6.079734219269103, 18.233)(6.179401993355482, 18.391)(6.279069767441861, 18.412)(6.378737541528239, 18.835)(6.4784053156146175, 18.667)(6.578073089700997, 18.341)(6.677740863787376, 18.711)(6.777408637873754, 18.168)(6.877076411960133, 18.116)(6.976744186046512, 18.26)(7.076411960132891, 18.375)(7.17607973421927, 18.206)(7.275747508305648, 18.248)(7.3754152823920265, 18.418)(7.475083056478405, 18.495)(7.574750830564784, 18.469)(7.674418604651163, 18.344)(7.774086378737542, 18.148)(7.8737541528239205, 17.839)(7.9734219269103, 18.441)(8.073089700996677, 18.214)(8.172757475083056, 18.138)(8.272425249169435, 17.942)(8.372093023255815, 17.921)(8.471760797342194, 17.86)(8.571428571428571, 17.717)(8.67109634551495, 17.457)(8.77076411960133, 17.69)(8.870431893687707, 17.6)(8.970099667774086, 18.008)(9.069767441860465, 17.708)(9.169435215946844, 17.721)(9.269102990033224, 17.74)(9.368770764119601, 17.893)(9.46843853820598, 17.697)(9.568106312292358, 17.933)(9.667774086378738, 18.392)(9.767441860465118, 18.054)(9.867109634551495, 18.235)(9.966777408637874, 18.145)(10.066445182724253, 18.178)(10.16611295681063, 18.266)(10.26578073089701, 18.213)(10.365448504983389, 18.192)(10.465116279069768, 18.09)(10.564784053156147, 18.275)(10.664451827242525, 18.176)(10.764119601328904, 18.013)(10.863787375415281, 17.993)(10.96345514950166, 18.344)(11.063122923588042, 18.438)(11.162790697674419, 18.131)(11.262458471760798, 18.018)(11.362126245847175, 18.081)(11.461794019933555, 17.856)(11.561461794019934, 18.294)(11.661129568106311, 18.125)(11.760797342192692, 17.992)(11.860465116279071, 17.914)(11.960132890365449, 18.015)(12.059800664451828, 17.81)(12.159468438538205, 17.655)(12.259136212624584, 17.961)(12.358803986710964, 17.888)(12.458471760797343, 17.621)(12.558139534883722, 17.563)(12.6578073089701, 17.78)(12.757475083056478, 17.572)(12.857142857142858, 18.069)(12.956810631229235, 18.403)(13.056478405315616, 18.077)(13.156146179401993, 17.609)(13.255813953488373, 17.61)(13.355481727574752, 17.908)(13.455149501661129, 18.081)(13.554817275747508, 17.883)(13.654485049833887, 17.675)(13.754152823920267, 18.012)(13.853820598006646, 17.637)(13.953488372093023, 17.818)(14.053156146179402, 18.119)(14.152823920265782, 17.802)(14.252491694352159, 17.564)(14.35215946843854, 17.815)(14.451827242524917, 17.541)(14.551495016611296, 18.249)(14.651162790697676, 17.846)(14.750830564784053, 17.471)(14.850498338870432, 17.316)(14.95016611295681, 17.931)(15.049833887043192, 18.166)(15.149501661129568, 18.172)(15.249169435215947, 19.048)(15.348837209302326, 19.195)(15.448504983388705, 19.531)(15.548172757475085, 19.278)(15.64784053156146, 19.208)(15.747508305647841, 19.206)(15.84717607973422, 19.103)(15.9468438538206, 17.97)(16.04651162790698, 18.844)(16.146179401993354, 18.761)(16.245847176079735, 18.244)(16.345514950166113, 18.321)(16.445182724252494, 18.344)(16.54485049833887, 18.199)(16.64451827242525, 18.075)(16.74418604651163, 18.425)(16.843853820598007, 18.106)(16.943521594684388, 18.531)(17.043189368770765, 18.524)(17.142857142857142, 18.286)(17.24252491694352, 18.792)(17.3421926910299, 18.616)(17.44186046511628, 18.66)(17.54152823920266, 18.496)(17.641196013289036, 18.454)(17.740863787375414, 18.693)(17.840531561461795, 18.812)(17.940199335548172, 18.731)(18.039867109634553, 18.048)(18.13953488372093, 18.012)(18.239202657807308, 18.29)(18.33887043189369, 18.465)(18.438538205980066, 18.729)(18.538205980066447, 17.834)(18.63787375415282, 18.006)(18.737541528239202, 18.047)(18.837209302325583, 18.093)(18.93687707641196, 17.936)(19.03654485049834, 17.702)(19.136212624584715, 17.934)(19.235880398671096, 17.885)(19.335548172757477, 18.027)(19.435215946843854, 17.823)(19.534883720930235, 17.917)(19.634551495016613, 18.072)(19.73421926910299, 17.635)(19.833887043189367, 18.491)(19.93355481727575, 17.793)(20.033222591362126, 17.972)(20.132890365448507, 17.798)(20.232558139534884, 17.782)(20.33222591362126, 17.917)(20.431893687707642, 18.207)(20.53156146179402, 18.174)(20.6312292358804, 18.122)(20.730897009966778, 17.741)(20.830564784053156, 17.962)(20.930232558139537, 17.863)(21.029900332225914, 18.091)(21.129568106312295, 18.101)(21.22923588039867, 17.94)(21.32890365448505, 18.048)(21.42857142857143, 18.048)(21.528239202657808, 17.905)(21.62790697674419, 17.865)(21.727574750830563, 18.14)(21.827242524916944, 18.075)(21.92691029900332, 17.915)(22.026578073089702, 17.958)(22.126245847176083, 18.241)(22.225913621262457, 18.19)(22.325581395348838, 17.862)(22.425249169435215, 18.538)(22.524916943521596, 18.098)(22.624584717607974, 17.918)(22.72425249169435, 18.328)(22.823920265780732, 18.156)(22.92358803986711, 17.41)(23.02325581395349, 17.688)(23.122923588039868, 18.103)(23.222591362126245, 18.008)(23.322259136212622, 17.764)(23.421926910299003, 18.128)(23.521594684385384, 17.863)(23.62126245847176, 17.63)(23.720930232558143, 18.12)(23.820598006644516, 17.858)(23.920265780730897, 17.827)(24.01993355481728, 17.659)(24.119601328903656, 18.073)(24.219269102990037, 17.93)(24.31893687707641, 18.02)(24.41860465116279, 18.267)(24.51827242524917, 18.066)(24.61794019933555, 17.983)(24.717607973421927, 17.894)(24.817275747508305, 18.226)(24.916943521594686, 18.137)(25.016611295681063, 18.117)(25.116279069767444, 17.792)(25.21594684385382, 18.139)(25.3156146179402, 17.744)(25.41528239202658, 18.283)(25.514950166112957, 17.959)(25.614617940199338, 17.871)(25.714285714285715, 17.988)(25.813953488372093, 17.918)(25.91362126245847, 18.016)(26.01328903654485, 17.763)(26.112956810631232, 17.954)(26.21262458471761, 17.865)(26.312292358803987, 18.506)(26.411960132890364, 18.519)(26.511627906976745, 18.087)(26.611295681063122, 17.902)(26.710963455149503, 17.923)(26.81063122923588, 17.981)(26.910299003322258, 18.051)(27.00996677740864, 18.14)(27.109634551495017, 18.144)(27.209302325581397, 18.432)(27.308970099667775, 17.904)(27.408637873754152, 18.293)(27.508305647840533, 17.853)(27.60797342192691, 18.055)(27.70764119601329, 18.533)(27.80730897009967, 18.514)(27.906976744186046, 18.545)(28.006644518272424, 18.358)(28.106312292358805, 18.102)(28.205980066445186, 18.067)(28.305647840531563, 18.068)(28.40531561461794, 18.233)(28.504983388704318, 17.763)(28.6046511627907, 17.988)(28.70431893687708, 17.891)(28.803986710963457, 18.001)(28.903654485049834, 17.933)(29.003322259136212, 17.383)(29.102990033222593, 18.333)(29.20265780730897, 18.151)(29.30232558139535, 17.897)(29.40199335548173, 18.278)(29.501661129568106, 17.936)(29.601328903654487, 17.996)(29.700996677740864, 18.027)(29.800664451827245, 17.895)(29.90033222591362, 18.009)(30.0, 17.843)(30.099667774086384, 17.796)(30.19933554817276, 17.709)(30.299003322259136, 18.161)(30.398671096345517, 18.646)(30.498338870431894, 17.934)(30.598006644518275, 18.112)(30.697674418604652, 17.906)(30.79734219269103, 17.996)(30.89700996677741, 18.054)(30.996677740863788, 17.697)(31.09634551495017, 18.055)(31.196013289036546, 17.75)(31.29568106312292, 18.292)(31.395348837209305, 18.216)(31.495016611295682, 17.854)(31.594684385382063, 18.065)(31.69435215946844, 18.149)(31.794019933554814, 18.366)(31.8936877076412, 18.01)(31.993355481727576, 18.01)(32.09302325581396, 17.847)(32.19269102990033, 18.317)(32.29235880398671, 18.113)(32.39202657807309, 18.435)(32.49169435215947, 18.042)(32.59136212624585, 17.945)(32.691029900332225, 18.143)(32.7906976744186, 17.845)(32.89036544850499, 18.007)(32.990033222591364, 18.022)(33.08970099667774, 18.175)(33.18936877076412, 18.346)(33.2890365448505, 18.119)(33.38870431893688, 18.373)(33.48837209302326, 17.847)(33.588039867109636, 18.035)(33.68770764119601, 17.816)(33.78737541528239, 18.183)(33.887043189368775, 18.024)(33.98671096345515, 18.111)(34.08637873754153, 18.025)(34.18604651162791, 17.793)(34.285714285714285, 18.188)(34.38538205980067, 18.351)(34.48504983388704, 18.462)(34.584717607973424, 18.362)(34.6843853820598, 18.004)(34.78405315614618, 17.838)(34.88372093023256, 18.066)(34.98338870431893, 18.087)(35.08305647840532, 18.546)(35.182724252491695, 18.58)(35.28239202657807, 17.886)(35.38205980066445, 18.604)(35.48172757475083, 18.074)(35.58139534883721, 17.933)(35.68106312292359, 17.84)(35.78073089700997, 18.089)(35.880398671096344, 18.074)(35.98006644518272, 17.682)(36.079734219269106, 18.079)(36.179401993355484, 17.954)(36.27906976744186, 17.723)(36.37873754152824, 18.482)(36.478405315614616, 17.871)(36.578073089701, 17.992)(36.67774086378738, 17.844)(36.777408637873755, 18.035)(36.87707641196013, 17.821)(36.97674418604651, 17.758)(37.076411960132894, 17.916)(37.17607973421927, 18.099)(37.27574750830564, 18.528)(37.37541528239203, 18.342)(37.475083056478404, 18.131)(37.57475083056479, 18.169)(37.674418604651166, 17.883)(37.774086378737536, 17.992)(37.87375415282392, 17.842)(37.9734219269103, 18.124)(38.07308970099668, 17.734)(38.17275747508306, 18.059)(38.27242524916943, 17.869)(38.372093023255815, 18.272)(38.47176079734219, 17.948)(38.57142857142858, 17.794)(38.671096345514954, 18.098)(38.77076411960133, 17.841)(38.87043189368771, 18.14)(38.970099667774086, 18.068)(39.06976744186047, 18.189)(39.16943521594684, 17.943)(39.269102990033225, 18.064)(39.3687707641196, 18.49)(39.46843853820598, 17.735)(39.568106312292365, 18.192)(39.667774086378735, 17.961)(39.76744186046512, 18.542)(39.8671096345515, 17.87)(39.966777408637874, 18.115)(40.06644518272425, 18.12)(40.16611295681063, 17.885)(40.26578073089701, 17.978)(40.36544850498339, 18.25)(40.46511627906977, 18.169)(40.564784053156146, 18.083)(40.66445182724252, 18.138)(40.76411960132891, 17.997)(40.863787375415285, 17.947)(40.96345514950166, 18.12)(41.06312292358804, 18.031)(41.16279069767442, 18.274)(41.2624584717608, 18.396)(41.36212624584718, 18.398)(41.461794019933556, 18.011)(41.561461794019934, 17.911)(41.66112956810631, 18.082)(41.760797342192696, 17.759)(41.86046511627907, 18.116)(41.96013289036544, 17.902)(42.05980066445183, 18.306)(42.159468438538205, 17.95)(42.25913621262459, 18.644)(42.35880398671097, 18.358)(42.45847176079734, 18.09)(42.55813953488372, 17.996)(42.6578073089701, 18.047)(42.757475083056484, 18.158)(42.85714285714286, 17.808)(42.95681063122923, 18.167)(43.056478405315616, 17.962)(43.15614617940199, 17.575)(43.25581395348838, 18.074)(43.355481727574755, 18.277)(43.455149501661126, 17.774)(43.55481727574751, 17.923)(43.65448504983389, 17.953)(43.75415282392027, 18.216)(43.85382059800664, 17.874)(43.95348837209302, 18.047)(44.053156146179404, 17.743)(44.15282392026578, 18.061)(44.252491694352166, 17.753)(44.352159468438536, 18.244)(44.451827242524914, 18.006)(44.5514950166113, 18.114)(44.651162790697676, 18.248)(44.75083056478405, 17.722)(44.85049833887043, 18.231)(44.95016611295681, 17.876)(45.04983388704319, 17.854)(45.14950166112957, 17.892)(45.24916943521595, 18.111)(45.348837209302324, 18.481)(45.4485049833887, 17.994)(45.548172757475086, 18.167)(45.647840531561464, 17.89)(45.74750830564784, 18.086)(45.84717607973422, 17.802)(45.946843853820596, 18.085)(46.04651162790698, 18.129)(46.14617940199336, 17.911)(46.245847176079735, 17.988)(46.34551495016611, 18.225)(46.44518272425249, 17.85)(46.544850498338874, 18.008)(46.644518272425245, 17.968)(46.74418604651163, 17.776)(46.84385382059801, 17.684)(46.94352159468439, 17.673)(47.04318936877077, 17.807)(47.14285714285714, 17.852)(47.24252491694352, 18.039)(47.3421926910299, 17.927)(47.441860465116285, 18.021)(47.54152823920266, 18.13)(47.64119601328903, 17.794)(47.74086378737542, 18.038)(47.840531561461795, 17.865)(47.94019933554818, 18.139)(48.03986710963456, 17.899)(48.13953488372093, 18.293)(48.23920265780731, 17.937)(48.33887043189369, 18.276)(48.43853820598007, 18.112)(48.538205980066444, 17.896)(48.63787375415282, 17.997)(48.737541528239205, 17.832)(48.83720930232558, 17.841)(48.93687707641197, 17.978)(49.03654485049834, 18.143)(49.136212624584715, 18.027)(49.2358803986711, 17.773)(49.33554817275748, 18.394)(49.435215946843854, 17.833)(49.53488372093023, 17.986)(49.63455149501661, 18.103)(49.734219269102994, 18.103)(49.83388704318937, 19.977)(49.93355481727575, 20.176)(50.033222591362126, 20.057)(50.1328903654485, 19.835)(50.23255813953489, 19.265)(50.332225913621265, 18.18)(50.43189368770764, 18.174)(50.53156146179402, 18.131)(50.6312292358804, 17.957)(50.73089700996678, 18.343)(50.83056478405316, 18.131)(50.93023255813954, 18.061)(51.029900332225914, 18.109)(51.12956810631229, 18.094)(51.229235880398676, 18.024)(51.328903654485046, 18.228)(51.42857142857143, 18.147)(51.52823920265781, 18.147)(51.627906976744185, 18.03)(51.72757475083057, 18.448)(51.82724252491694, 18.086)(51.926910299003325, 18.011)(52.0265780730897, 17.835)(52.12624584717608, 18.185)(52.225913621262464, 18.08)(52.325581395348834, 18.395)(52.42524916943522, 18.112)(52.524916943521596, 18.216)(52.62458471760797, 18.375)(52.72425249169436, 18.672)(52.82392026578073, 18.23)(52.92358803986711, 17.844)(53.02325581395349, 18.003)(53.12292358803987, 18.076)(53.222591362126245, 18.021)(53.32225913621262, 18.558)(53.42192691029901, 18.028)(53.521594684385384, 17.934)(53.62126245847176, 17.907)(53.72093023255814, 18.658)(53.820598006644516, 18.131)(53.9202657807309, 18.104)(54.01993355481728, 18.17)(54.119601328903656, 17.981)(54.21926910299003, 18.109)(54.31893687707641, 18.309)(54.418604651162795, 18.102)(54.51827242524917, 17.851)(54.61794019933555, 17.93)(54.71760797342193, 18.294)(54.817275747508305, 18.084)(54.91694352159469, 18.179)(55.01661129568107, 17.917)(55.11627906976744, 18.027)(55.21594684385382, 17.946)(55.3156146179402, 18.882)(55.41528239202658, 18.333)(55.51495016611296, 17.963)(55.61461794019934, 18.266)(55.714285714285715, 19.226)(55.81395348837209, 18.162)(55.91362126245848, 17.958)(56.01328903654485, 18.206)(56.11295681063123, 17.92)(56.21262458471761, 17.999)(56.31229235880399, 18.206)(56.41196013289037, 17.908)(56.51162790697674, 18.115)(56.611295681063126, 18.051)(56.7109634551495, 18.143)(56.81063122923588, 17.807)(56.910299003322265, 18.042)(57.009966777408636, 18.279)(57.10963455149502, 18.52)(57.2093023255814, 17.753)(57.308970099667775, 18.265)(57.40863787375416, 17.857)(57.50830564784053, 18.055)(57.607973421926914, 18.103)(57.70764119601329, 18.071)(57.80730897009967, 17.8)(57.906976744186046, 17.708)(58.006644518272424, 17.732)(58.10631229235881, 17.487)(58.205980066445186, 17.462)(58.30564784053156, 17.863)(58.40531561461794, 17.55)(58.50498338870432, 17.432)(58.6046511627907, 17.464)(58.70431893687708, 17.819)(58.80398671096346, 17.142)(58.903654485049834, 17.293)(59.00332225913621, 17.03)(59.102990033222596, 17.296)(59.202657807308974, 17.056)(59.30232558139535, 17.55)(59.40199335548173, 17.348)(59.501661129568106, 17.053)(59.60132890365449, 17.11)(59.70099667774087, 16.912)(59.80066445182724, 16.065)(59.90033222591362, 23.094)(60.0, 12.736)
            };
            \addplot[color=blue, mark=none,name path=B] coordinates { %% MIN value
            (0.0, 12.965)(0.09966777408637874, 14.86)(0.19933554817275748, 15.312)(0.2990033222591362, 14.901)(0.39867109634551495, 15.139)(0.49833887043189373, 15.555)(0.5980066445182723, 8.446)(0.6976744186046512, 9.852)(0.7973421926910299, 13.255)(0.8970099667774087, 12.999)(0.9966777408637875, 12.985)(1.0963455149501662, 13.229)(1.1960132890365447, 13.221)(1.2956810631229236, 13.254)(1.3953488372093024, 12.736)(1.495016611295681, 13.222)(1.5946843853820598, 13.65)(1.6943521594684385, 13.215)(1.7940199335548175, 13.376)(1.893687707641196, 12.906)(1.993355481727575, 13.222)(2.0930232558139537, 13.72)(2.1926910299003324, 13.044)(2.292358803986711, 13.682)(2.3920265780730894, 12.963)(2.4916943521594686, 13.191)(2.5913621262458473, 13.312)(2.691029900332226, 13.006)(2.7906976744186047, 13.131)(2.8903654485049834, 13.256)(2.990033222591362, 13.461)(3.089700996677741, 13.412)(3.1893687707641196, 11.124)(3.2890365448504983, 12.995)(3.388704318936877, 13.349)(3.488372093023256, 12.889)(3.588039867109635, 13.446)(3.6877076411960132, 13.165)(3.787375415282392, 12.688)(3.887043189368771, 13.219)(3.98671096345515, 13.051)(4.086378737541528, 13.224)(4.186046511627907, 13.108)(4.285714285714286, 13.753)(4.385382059800665, 13.314)(4.485049833887043, 12.998)(4.584717607973422, 13.507)(4.6843853820598005, 13.298)(4.784053156146179, 13.488)(4.883720930232559, 0.0)(4.983388704318937, 12.973)(5.083056478405315, 13.232)(5.1827242524916945, 13.025)(5.282392026578074, 13.185)(5.382059800664452, 13.173)(5.48172757475083, 13.357)(5.5813953488372094, 13.284)(5.681063122923588, 13.449)(5.780730897009967, 13.496)(5.880398671096346, 13.165)(5.980066445182724, 13.229)(6.079734219269103, 13.314)(6.179401993355482, 12.843)(6.279069767441861, 13.248)(6.378737541528239, 13.379)(6.4784053156146175, 13.355)(6.578073089700997, 13.475)(6.677740863787376, 13.466)(6.777408637873754, 13.241)(6.877076411960133, 13.096)(6.976744186046512, 12.936)(7.076411960132891, 13.126)(7.17607973421927, 13.417)(7.275747508305648, 13.348)(7.3754152823920265, 13.23)(7.475083056478405, 13.116)(7.574750830564784, 13.389)(7.674418604651163, 13.325)(7.774086378737542, 13.267)(7.8737541528239205, 13.188)(7.9734219269103, 12.964)(8.073089700996677, 13.374)(8.172757475083056, 0.0)(8.272425249169435, 13.578)(8.372093023255815, 13.317)(8.471760797342194, 12.988)(8.571428571428571, 12.1)(8.67109634551495, 13.298)(8.77076411960133, 13.334)(8.870431893687707, 12.93)(8.970099667774086, 13.07)(9.069767441860465, 13.188)(9.169435215946844, 13.079)(9.269102990033224, 13.109)(9.368770764119601, 12.72)(9.46843853820598, 13.257)(9.568106312292358, 13.072)(9.667774086378738, 13.397)(9.767441860465118, 12.89)(9.867109634551495, 12.934)(9.966777408637874, 13.285)(10.066445182724253, 13.129)(10.16611295681063, 13.653)(10.26578073089701, 13.407)(10.365448504983389, 8.154)(10.465116279069768, 11.722)(10.564784053156147, 13.236)(10.664451827242525, 13.192)(10.764119601328904, 13.06)(10.863787375415281, 5.658)(10.96345514950166, 13.154)(11.063122923588042, 13.107)(11.162790697674419, 13.058)(11.262458471760798, 12.847)(11.362126245847175, 13.034)(11.461794019933555, 13.242)(11.561461794019934, 13.132)(11.661129568106311, 12.979)(11.760797342192692, 13.28)(11.860465116279071, 13.13)(11.960132890365449, 12.904)(12.059800664451828, 12.864)(12.159468438538205, 10.035)(12.259136212624584, 9.442)(12.358803986710964, 13.072)(12.458471760797343, 12.95)(12.558139534883722, 12.931)(12.6578073089701, 13.171)(12.757475083056478, 12.746)(12.857142857142858, 12.87)(12.956810631229235, 13.107)(13.056478405315616, 13.001)(13.156146179401993, 12.964)(13.255813953488373, 12.929)(13.355481727574752, 13.081)(13.455149501661129, 12.965)(13.554817275747508, 13.074)(13.654485049833887, 12.858)(13.754152823920267, 13.054)(13.853820598006646, 13.047)(13.953488372093023, 7.055)(14.053156146179402, 12.833)(14.152823920265782, 12.874)(14.252491694352159, 13.133)(14.35215946843854, 5.977)(14.451827242524917, 13.05)(14.551495016611296, 9.179)(14.651162790697676, 13.018)(14.750830564784053, 11.586)(14.850498338870432, 9.699)(14.95016611295681, 5.512)(15.049833887043192, 9.767)(15.149501661129568, 7.782)(15.249169435215947, 5.261)(15.348837209302326, 12.34)(15.448504983388705, 9.579)(15.548172757475085, 10.003)(15.64784053156146, 12.911)(15.747508305647841, 12.767)(15.84717607973422, 12.643)(15.9468438538206, 12.924)(16.04651162790698, 12.722)(16.146179401993354, 13.264)(16.245847176079735, 12.91)(16.345514950166113, 12.769)(16.445182724252494, 10.719)(16.54485049833887, 12.322)(16.64451827242525, 12.734)(16.74418604651163, 12.881)(16.843853820598007, 12.879)(16.943521594684388, 13.005)(17.043189368770765, 12.674)(17.142857142857142, 12.864)(17.24252491694352, 12.9)(17.3421926910299, 9.246)(17.44186046511628, 7.88)(17.54152823920266, 12.763)(17.641196013289036, 6.783)(17.740863787375414, 7.675)(17.840531561461795, 13.077)(17.940199335548172, 6.988)(18.039867109634553, 11.658)(18.13953488372093, 13.011)(18.239202657807308, 12.72)(18.33887043189369, 13.053)(18.438538205980066, 12.963)(18.538205980066447, 13.03)(18.63787375415282, 13.178)(18.737541528239202, 12.714)(18.837209302325583, 11.504)(18.93687707641196, 11.122)(19.03654485049834, 11.126)(19.136212624584715, 4.598)(19.235880398671096, 12.713)(19.335548172757477, 13.013)(19.435215946843854, 12.701)(19.534883720930235, 12.892)(19.634551495016613, 8.596)(19.73421926910299, 12.825)(19.833887043189367, 12.893)(19.93355481727575, 12.833)(20.033222591362126, 13.023)(20.132890365448507, 12.78)(20.232558139534884, 10.271)(20.33222591362126, 12.907)(20.431893687707642, 12.823)(20.53156146179402, 13.017)(20.6312292358804, 12.894)(20.730897009966778, 12.832)(20.830564784053156, 12.956)(20.930232558139537, 12.665)(21.029900332225914, 12.905)(21.129568106312295, 12.713)(21.22923588039867, 12.798)(21.32890365448505, 8.588)(21.42857142857143, 12.622)(21.528239202657808, 12.701)(21.62790697674419, 11.294)(21.727574750830563, 10.363)(21.827242524916944, 11.725)(21.92691029900332, 13.034)(22.026578073089702, 12.829)(22.126245847176083, 12.382)(22.225913621262457, 12.774)(22.325581395348838, 12.839)(22.425249169435215, 9.343)(22.524916943521596, 12.928)(22.624584717607974, 12.951)(22.72425249169435, 12.63)(22.823920265780732, 13.004)(22.92358803986711, 13.009)(23.02325581395349, 11.687)(23.122923588039868, 12.473)(23.222591362126245, 13.046)(23.322259136212622, 12.859)(23.421926910299003, 12.953)(23.521594684385384, 13.147)(23.62126245847176, 13.224)(23.720930232558143, 12.98)(23.820598006644516, 4.573)(23.920265780730897, 12.78)(24.01993355481728, 12.216)(24.119601328903656, 6.928)(24.219269102990037, 12.705)(24.31893687707641, 6.696)(24.41860465116279, 12.87)(24.51827242524917, 12.686)(24.61794019933555, 12.952)(24.717607973421927, 12.04)(24.817275747508305, 8.151)(24.916943521594686, 12.527)(25.016611295681063, 12.963)(25.116279069767444, 12.982)(25.21594684385382, 12.935)(25.3156146179402, 8.168)(25.41528239202658, 12.531)(25.514950166112957, 12.762)(25.614617940199338, 13.022)(25.714285714285715, 13.066)(25.813953488372093, 12.624)(25.91362126245847, 12.833)(26.01328903654485, 12.893)(26.112956810631232, 13.085)(26.21262458471761, 12.867)(26.312292358803987, 12.878)(26.411960132890364, 7.671)(26.511627906976745, 13.137)(26.611295681063122, 12.901)(26.710963455149503, 10.912)(26.81063122923588, 12.994)(26.910299003322258, 10.272)(27.00996677740864, 13.202)(27.109634551495017, 11.098)(27.209302325581397, 12.72)(27.308970099667775, 12.964)(27.408637873754152, 12.844)(27.508305647840533, 9.815)(27.60797342192691, 12.916)(27.70764119601329, 12.763)(27.80730897009967, 12.626)(27.906976744186046, 13.038)(28.006644518272424, 13.233)(28.106312292358805, 12.706)(28.205980066445186, 13.076)(28.305647840531563, 10.935)(28.40531561461794, 12.785)(28.504983388704318, 13.147)(28.6046511627907, 13.027)(28.70431893687708, 13.011)(28.803986710963457, 13.15)(28.903654485049834, 13.083)(29.003322259136212, 12.928)(29.102990033222593, 12.853)(29.20265780730897, 13.247)(29.30232558139535, 11.168)(29.40199335548173, 7.652)(29.501661129568106, 7.775)(29.601328903654487, 12.862)(29.700996677740864, 13.125)(29.800664451827245, 13.025)(29.90033222591362, 12.914)(30.0, 12.914)(30.099667774086384, 12.743)(30.19933554817276, 12.883)(30.299003322259136, 12.94)(30.398671096345517, 12.931)(30.498338870431894, 13.144)(30.598006644518275, 6.129)(30.697674418604652, 12.879)(30.79734219269103, 10.595)(30.89700996677741, 12.837)(30.996677740863788, 4.061)(31.09634551495017, 11.238)(31.196013289036546, 13.045)(31.29568106312292, 12.671)(31.395348837209305, 12.419)(31.495016611295682, 13.039)(31.594684385382063, 12.768)(31.69435215946844, 13.082)(31.794019933554814, 13.186)(31.8936877076412, 12.63)(31.993355481727576, 13.056)(32.09302325581396, 5.591)(32.19269102990033, 12.917)(32.29235880398671, 12.793)(32.39202657807309, 12.747)(32.49169435215947, 12.999)(32.59136212624585, 7.804)(32.691029900332225, 12.705)(32.7906976744186, 13.024)(32.89036544850499, 12.808)(32.990033222591364, 12.674)(33.08970099667774, 13.016)(33.18936877076412, 12.709)(33.2890365448505, 12.951)(33.38870431893688, 12.874)(33.48837209302326, 12.69)(33.588039867109636, 13.149)(33.68770764119601, 13.064)(33.78737541528239, 12.664)(33.887043189368775, 12.567)(33.98671096345515, 12.813)(34.08637873754153, 12.917)(34.18604651162791, 12.788)(34.285714285714285, 12.906)(34.38538205980067, 12.949)(34.48504983388704, 12.508)(34.584717607973424, 13.122)(34.6843853820598, 10.772)(34.78405315614618, 12.792)(34.88372093023256, 5.355)(34.98338870431893, 12.566)(35.08305647840532, 12.818)(35.182724252491695, 8.135)(35.28239202657807, 12.164)(35.38205980066445, 12.737)(35.48172757475083, 13.02)(35.58139534883721, 12.854)(35.68106312292359, 12.66)(35.78073089700997, 13.052)(35.880398671096344, 13.027)(35.98006644518272, 12.82)(36.079734219269106, 12.568)(36.179401993355484, 12.903)(36.27906976744186, 12.909)(36.37873754152824, 12.579)(36.478405315614616, 12.894)(36.578073089701, 12.995)(36.67774086378738, 13.0)(36.777408637873755, 12.916)(36.87707641196013, 12.774)(36.97674418604651, 12.766)(37.076411960132894, 12.639)(37.17607973421927, 12.985)(37.27574750830564, 12.565)(37.37541528239203, 12.797)(37.475083056478404, 12.501)(37.57475083056479, 12.931)(37.674418604651166, 13.006)(37.774086378737536, 12.516)(37.87375415282392, 12.739)(37.9734219269103, 12.565)(38.07308970099668, 12.855)(38.17275747508306, 12.952)(38.27242524916943, 12.178)(38.372093023255815, 7.28)(38.47176079734219, 12.404)(38.57142857142858, 12.768)(38.671096345514954, 12.959)(38.77076411960133, 12.935)(38.87043189368771, 11.21)(38.970099667774086, 4.473)(39.06976744186047, 12.332)(39.16943521594684, 12.717)(39.269102990033225, 12.972)(39.3687707641196, 12.731)(39.46843853820598, 12.724)(39.568106312292365, 12.605)(39.667774086378735, 12.419)(39.76744186046512, 12.935)(39.8671096345515, 12.649)(39.966777408637874, 10.886)(40.06644518272425, 10.177)(40.16611295681063, 12.836)(40.26578073089701, 12.817)(40.36544850498339, 12.76)(40.46511627906977, 12.805)(40.564784053156146, 12.852)(40.66445182724252, 12.788)(40.76411960132891, 12.52)(40.863787375415285, 11.783)(40.96345514950166, 6.933)(41.06312292358804, 12.671)(41.16279069767442, 12.866)(41.2624584717608, 12.866)(41.36212624584718, 12.637)(41.461794019933556, 12.707)(41.561461794019934, 12.618)(41.66112956810631, 12.69)(41.760797342192696, 12.927)(41.86046511627907, 7.843)(41.96013289036544, 12.634)(42.05980066445183, 12.617)(42.159468438538205, 12.652)(42.25913621262459, 12.713)(42.35880398671097, 12.713)(42.45847176079734, 12.77)(42.55813953488372, 12.728)(42.6578073089701, 12.852)(42.757475083056484, 12.619)(42.85714285714286, 12.647)(42.95681063122923, 12.696)(43.056478405315616, 12.867)(43.15614617940199, 13.006)(43.25581395348838, 12.306)(43.355481727574755, 12.84)(43.455149501661126, 12.954)(43.55481727574751, 12.661)(43.65448504983389, 5.349)(43.75415282392027, 11.78)(43.85382059800664, 12.833)(43.95348837209302, 12.743)(44.053156146179404, 12.851)(44.15282392026578, 12.938)(44.252491694352166, 12.642)(44.352159468438536, 12.847)(44.451827242524914, 12.744)(44.5514950166113, 12.605)(44.651162790697676, 12.831)(44.75083056478405, 12.601)(44.85049833887043, 12.94)(44.95016611295681, 12.414)(45.04983388704319, 12.959)(45.14950166112957, 12.615)(45.24916943521595, 12.927)(45.348837209302324, 12.53)(45.4485049833887, 12.927)(45.548172757475086, 12.808)(45.647840531561464, 12.677)(45.74750830564784, 13.02)(45.84717607973422, 12.539)(45.946843853820596, 13.043)(46.04651162790698, 12.734)(46.14617940199336, 12.967)(46.245847176079735, 12.635)(46.34551495016611, 12.942)(46.44518272425249, 12.616)(46.544850498338874, 12.761)(46.644518272425245, 12.858)(46.74418604651163, 12.879)(46.84385382059801, 12.944)(46.94352159468439, 13.127)(47.04318936877077, 12.823)(47.14285714285714, 12.816)(47.24252491694352, 12.438)(47.3421926910299, 12.933)(47.441860465116285, 13.06)(47.54152823920266, 13.001)(47.64119601328903, 12.645)(47.74086378737542, 12.662)(47.840531561461795, 12.852)(47.94019933554818, 12.733)(48.03986710963456, 13.069)(48.13953488372093, 12.723)(48.23920265780731, 12.899)(48.33887043189369, 12.849)(48.43853820598007, 7.396)(48.538205980066444, 11.492)(48.63787375415282, 12.694)(48.737541528239205, 13.182)(48.83720930232558, 12.728)(48.93687707641197, 8.069)(49.03654485049834, 10.791)(49.136212624584715, 12.029)(49.2358803986711, 13.008)(49.33554817275748, 5.148)(49.435215946843854, 12.764)(49.53488372093023, 12.788)(49.63455149501661, 12.667)(49.734219269102994, 13.057)(49.83388704318937, 6.575)(49.93355481727575, 12.837)(50.033222591362126, 13.011)(50.1328903654485, 12.156)(50.23255813953489, 12.732)(50.332225913621265, 12.725)(50.43189368770764, 12.916)(50.53156146179402, 13.066)(50.6312292358804, 12.397)(50.73089700996678, 12.965)(50.83056478405316, 12.935)(50.93023255813954, 12.652)(51.029900332225914, 12.467)(51.12956810631229, 12.496)(51.229235880398676, 12.9)(51.328903654485046, 12.585)(51.42857142857143, 13.031)(51.52823920265781, 13.016)(51.627906976744185, 12.785)(51.72757475083057, 12.892)(51.82724252491694, 12.676)(51.926910299003325, 7.775)(52.0265780730897, 12.74)(52.12624584717608, 12.832)(52.225913621262464, 12.821)(52.325581395348834, 12.919)(52.42524916943522, 13.001)(52.524916943521596, 12.837)(52.62458471760797, 12.825)(52.72425249169436, 12.085)(52.82392026578073, 12.72)(52.92358803986711, 13.011)(53.02325581395349, 12.974)(53.12292358803987, 9.132)(53.222591362126245, 12.524)(53.32225913621262, 9.491)(53.42192691029901, 12.697)(53.521594684385384, 13.0)(53.62126245847176, 12.875)(53.72093023255814, 12.926)(53.820598006644516, 13.148)(53.9202657807309, 13.012)(54.01993355481728, 12.804)(54.119601328903656, 12.967)(54.21926910299003, 12.68)(54.31893687707641, 12.8)(54.418604651162795, 12.671)(54.51827242524917, 12.949)(54.61794019933555, 12.716)(54.71760797342193, 12.749)(54.817275747508305, 12.743)(54.91694352159469, 12.534)(55.01661129568107, 12.957)(55.11627906976744, 12.772)(55.21594684385382, 12.95)(55.3156146179402, 12.818)(55.41528239202658, 12.972)(55.51495016611296, 12.414)(55.61461794019934, 13.153)(55.714285714285715, 13.163)(55.81395348837209, 12.497)(55.91362126245848, 12.954)(56.01328903654485, 4.869)(56.11295681063123, 12.755)(56.21262458471761, 13.066)(56.31229235880399, 12.611)(56.41196013289037, 12.9)(56.51162790697674, 12.504)(56.611295681063126, 11.905)(56.7109634551495, 6.324)(56.81063122923588, 12.94)(56.910299003322265, 12.922)(57.009966777408636, 12.901)(57.10963455149502, 13.044)(57.2093023255814, 12.796)(57.308970099667775, 12.187)(57.40863787375416, 7.516)(57.50830564784053, 9.947)(57.607973421926914, 12.907)(57.70764119601329, 12.755)(57.80730897009967, 12.856)(57.906976744186046, 12.849)(58.006644518272424, 12.709)(58.10631229235881, 11.048)(58.205980066445186, 11.365)(58.30564784053156, 8.079)(58.40531561461794, 12.967)(58.50498338870432, 12.767)(58.6046511627907, 7.328)(58.70431893687708, 12.315)(58.80398671096346, 12.763)(58.903654485049834, 12.671)(59.00332225913621, 11.188)(59.102990033222596, 9.73)(59.202657807308974, 12.898)(59.30232558139535, 12.914)(59.40199335548173, 12.847)(59.501661129568106, 12.93)(59.60132890365449, 12.723)(59.70099667774087, 11.834)(59.80066445182724, 8.135)(59.90033222591362, 0.0)(60.0, 12.736)
            };
            \addplot [pattern=north east lines,pattern color=red] 
            fill between [
                of=A and B,soft clip={domain=0:800},
            ];
            \end{axis}
\end{tikzpicture}
\caption{DUT: Surface 4 Pro}\label{fig:time_series_Fankuch_IntelPowerGadgetPro}
\end{subfigure}
\begin{subfigure}[b]{0.49\linewidth}
    \begin{tikzpicture}
        \pgfplotsset{%
        width=1\linewidth,
        % height=1\textheight
        }
        \begin{axis}[ymax=120,
            xlabel={Time (Seconds)},
            ylabel={Energy Consumption (Joules)},
            ]
            \addplot[color=blue, mark=none,] coordinates { %% AVG value
            (0.0, 4.961984126984127)(0.09950248756218906, 6.692499999999997)(0.19900497512437812, 6.472206349206349)(0.29850746268656714, 6.846976190476187)(0.39800995024875624, 7.630992063492064)(0.4975124378109453, 8.004460317460317)(0.5970149253731343, 8.18068253968254)(0.6965174129353234, 8.28701587301587)(0.7960199004975125, 8.32481746031746)(0.8955223880597015, 8.356650793650797)(0.9950248756218906, 8.370761904761903)(1.0945273631840795, 8.413896825396826)(1.1940298507462686, 8.448968253968248)(1.2935323383084576, 8.441817460317461)(1.3930348258706469, 8.428412698412703)(1.492537313432836, 8.428158730158733)(1.592039800995025, 8.421706349206351)(1.691542288557214, 8.406777777777775)(1.791044776119403, 8.438777777777778)(1.890547263681592, 8.429444444444446)(1.9900497512437811, 8.422015873015873)(2.0895522388059704, 8.390817460317463)(2.189054726368159, 8.336515873015871)(2.288557213930348, 8.449849206349203)(2.388059701492537, 8.467976190476193)(2.487562189054726, 8.440309523809523)(2.587064676616915, 8.470460317460317)(2.6865671641791047, 8.434801587301585)(2.7860696517412937, 8.473428571428574)(2.8855721393034828, 8.447349206349203)(2.985074626865672, 8.440380952380952)(3.084577114427861, 8.485492063492064)(3.18407960199005, 8.396682539682539)(3.283582089552239, 8.380896825396826)(3.383084577114428, 8.383087301587304)(3.482587064676617, 8.401484126984128)(3.582089552238806, 8.416095238095236)(3.6815920398009947, 8.420809523809524)(3.781094527363184, 8.407563492063488)(3.880597014925373, 8.405198412698415)(3.9800995024875623, 8.39929365079365)(4.079601990049751, 8.416896825396826)(4.179104477611941, 8.42943650793651)(4.27860696517413, 8.383730158730156)(4.378109452736318, 8.406452380952379)(4.477611940298508, 8.42350793650794)(4.577114427860696, 8.421682539682541)(4.676616915422886, 8.503753968253964)(4.776119402985074, 8.51092063492063)(4.875621890547264, 8.513079365079367)(4.975124378109452, 8.535579365079363)(5.074626865671642, 8.544222222222222)(5.17412935323383, 8.601309523809523)(5.27363184079602, 8.551753968253967)(5.373134328358209, 8.484023809523809)(5.472636815920398, 8.507746031746029)(5.5721393034825875, 8.529309523809522)(5.6716417910447765, 8.540436507936507)(5.7711442786069655, 8.589119047619048)(5.870646766169154, 8.585111111111113)(5.970149253731344, 8.608531746031746)(6.069651741293532, 8.573492063492065)(6.169154228855722, 8.63338888888889)(6.26865671641791, 8.61484126984127)(6.3681592039801, 8.708944444444446)(6.467661691542289, 8.695825396825395)(6.567164179104478, 8.664214285714293)(6.666666666666667, 8.600182539682537)(6.766169154228856, 8.580769841269843)(6.865671641791045, 8.650063492063493)(6.965174129353234, 8.643293650793654)(7.064676616915423, 8.610404761904764)(7.164179104477612, 8.64317460317461)(7.263681592039801, 8.640690476190477)(7.363184079601989, 8.582738095238096)(7.462686567164179, 8.569047619047621)(7.562189054726368, 8.628)(7.661691542288557, 8.64402380952381)(7.761194029850746, 8.680174603174597)(7.860696517412936, 8.671134920634922)(7.960199004975125, 8.669595238095235)(8.059701492537313, 8.674166666666668)(8.159203980099502, 8.662730158730158)(8.258706467661693, 8.664785714285715)(8.358208955223882, 8.62918253968254)(8.457711442786069, 8.607007936507934)(8.55721393034826, 8.630174603174607)(8.656716417910449, 8.61874603174603)(8.756218905472636, 8.634119047619045)(8.855721393034825, 8.63918253968254)(8.955223880597016, 8.667214285714282)(9.054726368159205, 8.687642857142857)(9.154228855721392, 8.708746031746035)(9.253731343283581, 8.71195238095238)(9.353233830845772, 8.663722222222223)(9.452736318407961, 8.683460317460314)(9.552238805970148, 8.715396825396823)(9.65174129353234, 8.685349206349207)(9.751243781094528, 8.713611111111113)(9.850746268656717, 8.708261904761903)(9.950248756218905, 8.714158730158728)(10.049751243781095, 8.726531746031744)(10.149253731343284, 8.814833333333336)(10.248756218905474, 8.964626984126987)(10.34825870646766, 8.946253968253968)(10.447761194029852, 8.966857142857142)(10.54726368159204, 8.947380952380954)(10.646766169154228, 8.969190476190478)(10.746268656716419, 8.961111111111112)(10.845771144278608, 8.9494126984127)(10.945273631840797, 8.972404761904766)(11.044776119402984, 8.971936507936503)(11.144278606965175, 8.972253968253966)(11.243781094527364, 8.987134920634922)(11.343283582089553, 8.980642857142854)(11.44278606965174, 8.963158730158726)(11.542288557213931, 8.99154761904762)(11.64179104477612, 8.96180158730159)(11.741293532338307, 8.959865079365079)(11.840796019900498, 8.929206349206348)(11.940298507462687, 8.960484126984127)(12.039800995024876, 9.004777777777777)(12.139303482587064, 9.008650793650796)(12.238805970149254, 9.036206349206353)(12.338308457711443, 9.012015873015878)(12.437810945273633, 8.976198412698412)(12.53731343283582, 8.980960317460317)(12.63681592039801, 9.019301587301591)(12.7363184079602, 9.027269841269844)(12.835820895522389, 8.993769841269838)(12.935323383084578, 9.0108253968254)(13.034825870646767, 9.035904761904765)(13.134328358208956, 9.021595238095237)(13.233830845771145, 9.042619047619047)(13.333333333333334, 9.026650793650793)(13.432835820895523, 8.979674603174605)(13.532338308457712, 8.984777777777778)(13.6318407960199, 8.951793650793654)(13.73134328358209, 8.969984126984127)(13.83084577114428, 8.966436507936509)(13.930348258706468, 8.998301587301587)(14.029850746268657, 8.974785714285714)(14.129353233830846, 8.994825396825398)(14.228855721393035, 9.025253968253965)(14.328358208955224, 9.039706349206346)(14.427860696517413, 8.998238095238095)(14.527363184079602, 9.029325396825392)(14.626865671641792, 9.065206349206349)(14.726368159203979, 9.081285714285713)(14.82587064676617, 9.0809126984127)(14.925373134328359, 9.09435714285714)(15.02487562189055, 9.070111111111112)(15.124378109452737, 9.043904761904763)(15.223880597014926, 9.086698412698414)(15.323383084577115, 9.073690476190475)(15.422885572139304, 9.045253968253963)(15.522388059701491, 9.064420634920637)(15.62189054726368, 9.079666666666666)(15.721393034825873, 9.080388888888887)(15.82089552238806, 9.037444444444441)(15.92039800995025, 9.024388888888888)(16.019900497512438, 9.056047619047618)(16.119402985074625, 9.009499999999997)(16.218905472636816, 9.066134920634923)(16.318407960199004, 9.032174603174605)(16.417910447761194, 9.070785714285716)(16.517412935323385, 9.047079365079366)(16.616915422885572, 9.04385714285715)(16.716417910447763, 9.063111111111107)(16.81592039800995, 9.07474603174603)(16.915422885572138, 9.081103174603177)(17.01492537313433, 9.099388888888893)(17.11442786069652, 9.064515873015871)(17.213930348258707, 9.085880952380958)(17.313432835820898, 9.074603174603174)(17.412935323383085, 9.067007936507938)(17.512437810945272, 9.075769841269839)(17.611940298507463, 9.085896825396826)(17.71144278606965, 9.09426190476191)(17.81094527363184, 9.077793650793655)(17.910447761194032, 9.101095238095239)(18.00995024875622, 9.114007936507942)(18.10945273631841, 9.116174603174601)(18.208955223880597, 9.104126984126987)(18.308457711442784, 9.130246031746035)(18.407960199004975, 9.124500000000003)(18.507462686567163, 9.081674603174603)(18.606965174129357, 9.069841269841268)(18.706467661691544, 9.098142857142857)(18.80597014925373, 9.121492063492063)(18.905472636815922, 9.138023809523807)(19.00497512437811, 9.142015873015872)(19.104477611940297, 9.102452380952379)(19.203980099502488, 9.131261904761907)(19.30348258706468, 9.105269841269845)(19.402985074626866, 9.07596825396825)(19.502487562189057, 9.066404761904758)(19.601990049751244, 9.049039682539682)(19.701492537313435, 9.057293650793651)(19.800995024875622, 9.080523809523815)(19.90049751243781, 9.078007936507937)(20.0, 9.046992063492064)(20.09950248756219, 9.064198412698413)(20.199004975124378, 9.055761904761905)(20.29850746268657, 9.046031746031746)(20.398009950248756, 9.0500873015873)(20.497512437810947, 9.033634920634926)(20.597014925373134, 9.063428571428569)(20.69651741293532, 9.087166666666665)(20.796019900497516, 9.097571428571428)(20.895522388059703, 9.061047619047619)(20.99502487562189, 9.042261904761904)(21.09452736318408, 9.016777777777776)(21.19402985074627, 9.066634920634922)(21.293532338308456, 9.070055555555554)(21.393034825870647, 9.036555555555555)(21.492537313432837, 9.06319841269841)(21.592039800995025, 9.015952380952385)(21.691542288557216, 9.046642857142855)(21.791044776119403, 9.03227777777778)(21.890547263681594, 9.063730158730161)(21.99004975124378, 9.066849206349207)(22.089552238805968, 9.037214285714287)(22.18905472636816, 9.050166666666664)(22.28855721393035, 9.05869841269841)(22.388059701492537, 9.051309523809525)(22.487562189054728, 9.040261904761907)(22.587064676616915, 9.014809523809518)(22.686567164179106, 9.076761904761906)(22.786069651741293, 9.07581746031746)(22.88557213930348, 8.995396825396826)(22.985074626865675, 9.058952380952382)(23.084577114427862, 9.153920634920633)(23.18407960199005, 9.095801587301588)(23.28358208955224, 9.117563492063491)(23.383084577114428, 9.157230158730162)(23.482587064676615, 9.162674603174606)(23.582089552238806, 9.132087301587305)(23.681592039800996, 9.131412698412696)(23.781094527363187, 9.146134920634921)(23.880597014925375, 9.15850793650794)(23.980099502487562, 9.13577777777778)(24.079601990049753, 9.144349206349203)(24.17910447761194, 9.158349206349204)(24.278606965174127, 9.159531746031744)(24.378109452736318, 9.128523809523804)(24.47761194029851, 9.138896825396827)(24.5771144278607, 9.117984126984126)(24.676616915422887, 9.121452380952384)(24.776119402985074, 9.13692857142857)(24.875621890547265, 9.153563492063494)(24.975124378109452, 9.151341269841266)(25.07462686567164, 9.139698412698413)(25.174129353233834, 9.15464285714286)(25.27363184079602, 9.152047619047616)(25.37313432835821, 9.154484126984128)(25.4726368159204, 9.172484126984127)(25.572139303482587, 9.14869047619048)(25.671641791044777, 9.123388888888888)(25.771144278606965, 9.158722222222217)(25.870646766169155, 9.169261904761903)(25.970149253731346, 9.14246825396825)(26.069651741293534, 9.148777777777775)(26.16915422885572, 9.163476190476194)(26.26865671641791, 9.1548492063492)(26.3681592039801, 9.124531746031744)(26.46766169154229, 9.132095238095237)(26.56716417910448, 9.105261904761905)(26.666666666666668, 9.11361904761905)(26.76616915422886, 9.11811111111111)(26.865671641791046, 9.13350793650794)(26.965174129353233, 9.15306349206349)(27.064676616915424, 9.175769841269844)(27.16417910447761, 9.15940476190476)(27.2636815920398, 9.157063492063491)(27.363184079601993, 9.113642857142857)(27.46268656716418, 9.120015873015875)(27.562189054726367, 9.12215873015873)(27.66169154228856, 9.109230158730158)(27.761194029850746, 9.147769841269838)(27.860696517412936, 9.15405555555556)(27.960199004975124, 9.159873015873018)(28.059701492537314, 9.190079365079363)(28.159203980099505, 9.2205)(28.258706467661693, 9.235642857142857)(28.35820895522388, 9.195698412698412)(28.45771144278607, 9.18349206349206)(28.557213930348258, 9.197214285714287)(28.65671641791045, 9.22853968253968)(28.75621890547264, 9.176380952380947)(28.855721393034827, 9.196809523809527)(28.955223880597018, 9.194674603174601)(29.054726368159205, 9.16054761904762)(29.154228855721392, 9.204317460317464)(29.253731343283583, 9.16491269841269)(29.35323383084577, 9.132055555555558)(29.452736318407958, 9.151071428571427)(29.552238805970152, 9.173833333333329)(29.65174129353234, 9.153246031746031)(29.75124378109453, 9.114182539682538)(29.850746268656717, 9.123388888888886)(29.950248756218905, 9.106619047619045)(30.0497512437811, 9.080920634920632)(30.149253731343286, 9.12219841269841)(30.248756218905474, 9.126682539682538)(30.348258706467664, 9.138261904761908)(30.44776119402985, 9.124904761904766)(30.547263681592042, 9.15509523809524)(30.64676616915423, 9.138912698412696)(30.746268656716417, 9.154920634920638)(30.845771144278608, 9.13322222222222)(30.945273631840795, 9.202031746031745)(31.044776119402982, 9.161103174603177)(31.144278606965173, 9.157539682539678)(31.24378109452736, 9.16034126984127)(31.343283582089555, 9.17339682539682)(31.442786069651746, 9.15815873015873)(31.542288557213933, 9.162976190476193)(31.64179104477612, 9.158428571428576)(31.74129353233831, 9.14506349206349)(31.8407960199005, 9.147404761904767)(31.94029850746269, 9.12836507936508)(32.039800995024876, 9.120722222222222)(32.13930348258707, 9.135198412698411)(32.23880597014925, 9.178246031746024)(32.33830845771144, 9.174198412698413)(32.43781094527363, 9.16314285714286)(32.537313432835816, 9.170333333333328)(32.63681592039801, 9.123746031746029)(32.7363184079602, 9.128936507936512)(32.83582089552239, 9.116619047619048)(32.93532338308458, 9.131674603174602)(33.03482587064677, 9.147706349206352)(33.13432835820896, 9.115603174603175)(33.233830845771145, 9.144880952380946)(33.333333333333336, 9.152777777777779)(33.43283582089553, 9.095507936507937)(33.53233830845771, 9.079365079365077)(33.6318407960199, 9.10756349206349)(33.73134328358209, 9.084277777777775)(33.830845771144276, 9.073793650793647)(33.930348258706466, 9.059460317460323)(34.02985074626866, 9.088317460317466)(34.12935323383084, 9.064984126984127)(34.22885572139304, 9.112182539682545)(34.32835820895523, 9.081174603174604)(34.42786069651741, 9.09998412698413)(34.527363184079604, 9.093960317460319)(34.626865671641795, 9.10761111111111)(34.72636815920398, 9.069825396825399)(34.82587064676617, 9.118642857142856)(34.92537313432836, 9.107801587301593)(35.024875621890544, 9.096944444444448)(35.124378109452735, 9.111595238095235)(35.223880597014926, 9.093547619047618)(35.32338308457712, 9.104277777777776)(35.4228855721393, 9.113992063492068)(35.52238805970149, 9.096793650793648)(35.62189054726368, 9.03707936507937)(35.72139303482587, 9.02065873015873)(35.820895522388064, 9.00756349206349)(35.920398009950254, 9.023039682539684)(36.01990049751244, 9.02664285714286)(36.11940298507463, 9.052222222222225)(36.21890547263682, 9.034769841269844)(36.318407960199, 9.024785714285716)(36.417910447761194, 9.053960317460321)(36.517412935323385, 9.04396031746032)(36.61691542288557, 9.053166666666664)(36.71641791044776, 9.057261904761903)(36.81592039800995, 9.019412698412703)(36.91542288557214, 9.045198412698415)(37.014925373134325, 9.029126984126982)(37.114427860696516, 9.022849206349205)(37.213930348258714, 9.030944444444442)(37.3134328358209, 9.017309523809526)(37.41293532338309, 9.019722222222223)(37.51243781094528, 9.031325396825396)(37.61194029850746, 9.023357142857144)(37.711442786069654, 9.018055555555554)(37.810945273631845, 9.016111111111107)(37.91044776119403, 9.048706349206352)(38.00995024875622, 9.043873015873015)(38.10945273631841, 9.06056349206349)(38.208955223880594, 9.046833333333332)(38.308457711442784, 9.089119047619047)(38.407960199004975, 9.076444444444448)(38.50746268656716, 9.06419841269841)(38.60696517412936, 9.085031746031747)(38.70646766169155, 9.083960317460317)(38.80597014925373, 9.073888888888883)(38.90547263681592, 9.045936507936508)(39.00497512437811, 9.020841269841268)(39.1044776119403, 9.084849206349206)(39.20398009950249, 9.076047619047618)(39.30348258706468, 9.04451587301587)(39.40298507462687, 9.02835714285714)(39.50248756218905, 9.039166666666668)(39.601990049751244, 9.051841269841272)(39.701492537313435, 9.073888888888888)(39.80099502487562, 9.077865079365083)(39.90049751243781, 9.047365079365083)(40.0, 9.015047619047623)(40.09950248756219, 9.061960317460317)(40.19900497512438, 9.071460317460316)(40.29850746268657, 9.038404761904765)(40.398009950248756, 9.054730158730163)(40.49751243781095, 9.0105873015873)(40.59701492537314, 9.004047619047617)(40.69651741293532, 9.017936507936508)(40.79601990049751, 9.002190476190474)(40.8955223880597, 9.018158730158733)(40.995024875621894, 8.977420634920636)(41.09452736318408, 8.978523809523812)(41.19402985074627, 8.961063492063493)(41.29353233830846, 8.984246031746032)(41.39303482587064, 9.022428571428573)(41.49253731343284, 9.030833333333334)(41.59203980099503, 9.040817460317466)(41.691542288557216, 9.040214285714287)(41.791044776119406, 9.044190476190472)(41.8905472636816, 9.048952380952382)(41.99004975124378, 9.041396825396822)(42.08955223880597, 9.029492063492063)(42.18905472636816, 9.052023809523808)(42.288557213930346, 9.02324603174603)(42.38805970149254, 9.020769841269841)(42.48756218905473, 9.007674603174602)(42.58706467661691, 9.01013492063492)(42.6865671641791, 9.013888888888888)(42.78606965174129, 9.006960317460317)(42.88557213930348, 9.010031746031743)(42.985074626865675, 9.005293650793652)(43.084577114427866, 9.02861111111111)(43.18407960199005, 9.055198412698411)(43.28358208955224, 9.026476190476192)(43.38308457711443, 9.024174603174604)(43.48258706467662, 9.00585714285714)(43.582089552238806, 9.028793650793649)(43.681592039801, 9.016849206349207)(43.78109452736319, 9.042111111111106)(43.88059701492537, 9.0115873015873)(43.98009950248756, 8.961523809523808)(44.07960199004975, 9.03650793650794)(44.179104477611936, 9.051388888888889)(44.27860696517413, 9.023753968253967)(44.37810945273632, 9.022063492063484)(44.47761194029851, 9.019730158730152)(44.5771144278607, 9.00696031746032)(44.67661691542289, 9.006206349206344)(44.776119402985074, 9.032936507936512)(44.875621890547265, 9.05975396825397)(44.975124378109456, 9.034912698412704)(45.07462686567165, 9.051396825396825)(45.17412935323383, 9.06370634920635)(45.27363184079602, 9.01655555555556)(45.37313432835821, 9.038365079365077)(45.472636815920396, 9.016031746031754)(45.57213930348259, 9.013158730158723)(45.67164179104478, 9.042420634920633)(45.77114427860696, 9.040912698412699)(45.87064676616916, 8.968563492063492)(45.97014925373135, 8.929452380952384)(46.069651741293534, 9.027015873015877)(46.169154228855724, 9.045833333333333)(46.268656716417915, 9.026809523809527)(46.3681592039801, 9.04180952380952)(46.46766169154229, 9.013349206349206)(46.56716417910448, 9.001301587301585)(46.666666666666664, 9.024682539682543)(46.766169154228855, 9.016928571428572)(46.865671641791046, 8.966563492063491)(46.96517412935323, 8.944222222222225)(47.06467661691542, 8.954341269841269)(47.16417910447761, 8.968380952380956)(47.2636815920398, 8.990761904761907)(47.36318407960199, 8.995206349206352)(47.462686567164184, 8.999285714285715)(47.562189054726375, 9.01131746031746)(47.66169154228856, 9.027484126984126)(47.76119402985075, 9.015642857142858)(47.86069651741294, 8.989809523809525)(47.960199004975124, 9.011896825396823)(48.059701492537314, 8.994595238095238)(48.159203980099505, 9.026539682539685)(48.25870646766169, 8.993071428571431)(48.35820895522388, 9.010309523809527)(48.45771144278607, 9.004277777777785)(48.557213930348254, 9.011174603174602)(48.656716417910445, 8.999817460317459)(48.756218905472636, 9.03134126984127)(48.85572139303483, 8.93852380952381)(48.95522388059702, 8.980246031746033)(49.05472636815921, 9.01978571428571)(49.1542288557214, 9.052285714285714)(49.25373134328358, 9.013507936507938)(49.353233830845774, 9.048468253968258)(49.452736318407965, 9.028079365079368)(49.55223880597015, 8.957960317460314)(49.65174129353234, 8.968833333333334)(49.75124378109453, 8.997515873015875)(49.850746268656714, 9.001253968253966)(49.950248756218905, 9.01771428571429)(50.049751243781095, 9.019087301587296)(50.14925373134328, 9.021380952380953)(50.24875621890548, 8.985992063492066)(50.34825870646767, 8.999888888888888)(50.44776119402985, 8.9825873015873)(50.54726368159204, 8.976444444444445)(50.64676616915423, 8.975341269841273)(50.74626865671642, 8.964936507936507)(50.84577114427861, 8.980238095238095)(50.9452736318408, 8.979682539682543)(51.04477611940298, 8.960809523809521)(51.14427860696517, 8.963071428571428)(51.243781094527364, 8.963888888888885)(51.343283582089555, 8.953888888888889)(51.44278606965174, 8.97396031746032)(51.54228855721393, 8.976515873015872)(51.64179104477612, 8.906531746031746)(51.74129353233831, 8.93804761904762)(51.8407960199005, 8.953388888888888)(51.94029850746269, 8.938761904761906)(52.039800995024876, 8.939611111111109)(52.13930348258707, 8.947793650793649)(52.23880597014926, 8.94043650793651)(52.33830845771144, 8.954293650793648)(52.43781094527363, 8.945063492063488)(52.53731343283582, 8.95236507936508)(52.63681592039801, 8.919825396825397)(52.7363184079602, 8.940134920634925)(52.83582089552239, 8.924206349206347)(52.93532338308458, 8.904515873015876)(53.03482587064676, 8.903309523809527)(53.13432835820896, 8.911365079365074)(53.23383084577115, 8.897277777777779)(53.333333333333336, 8.92299206349206)(53.43283582089553, 8.934333333333333)(53.53233830845772, 8.95644444444444)(53.6318407960199, 8.937952380952385)(53.73134328358209, 8.931936507936504)(53.83084577114428, 8.953428571428576)(53.930348258706466, 8.914761904761907)(54.02985074626866, 8.935944444444443)(54.12935323383085, 8.942841269841265)(54.22885572139303, 8.926198412698412)(54.32835820895522, 8.937317460317459)(54.42786069651741, 8.921547619047619)(54.5273631840796, 8.923920634920638)(54.626865671641795, 8.883341269841273)(54.726368159203986, 8.952952380952379)(54.82587064676617, 8.94216666666667)(54.92537313432836, 8.930119047619048)(55.02487562189055, 9.009682539682544)(55.124378109452735, 8.986952380952381)(55.223880597014926, 8.953944444444444)(55.32338308457712, 8.960349206349205)(55.42288557213931, 9.003611111111113)(55.52238805970149, 9.060682539682542)(55.62189054726368, 8.946690476190478)(55.72139303482587, 8.925555555555555)(55.82089552238806, 8.902301587301592)(55.92039800995025, 8.884690476190473)(56.01990049751244, 8.93884126984127)(56.11940298507463, 8.91279365079365)(56.21890547263682, 8.891293650793658)(56.31840796019901, 8.915031746031751)(56.417910447761194, 8.941357142857141)(56.517412935323385, 8.938238095238097)(56.616915422885576, 9.021404761904764)(56.71641791044776, 9.083968253968251)(56.81592039800995, 9.051785714285716)(56.91542288557214, 8.992333333333335)(57.01492537313433, 8.978634920634923)(57.114427860696516, 8.960087301587306)(57.21393034825871, 8.93049206349206)(57.3134328358209, 8.96030952380952)(57.41293532338308, 8.93786507936508)(57.51243781094528, 8.959619047619046)(57.61194029850747, 8.966349206349204)(57.711442786069654, 8.984420634920637)(57.810945273631845, 8.992492063492065)(57.910447761194035, 8.964103174603174)(58.00995024875622, 9.027222222222218)(58.10945273631841, 9.012436507936506)(58.2089552238806, 8.984380952380954)(58.308457711442784, 8.975817460317463)(58.407960199004975, 8.937666666666669)(58.507462686567166, 8.913714285714285)(58.60696517412935, 8.957000000000004)(58.70646766169154, 8.95624603174603)(58.80597014925373, 8.958126984126991)(58.905472636815915, 8.950269841269844)(59.00497512437811, 8.969817460317467)(59.104477611940304, 8.942103174603176)(59.20398009950249, 8.963555555555553)(59.30348258706468, 9.001650793650795)(59.40298507462687, 9.022277777777777)(59.50248756218906, 8.975277777777778)(59.601990049751244, 8.877841269841273)(59.701492537313435, 8.054642276422763)(59.800995024875625, 5.003577777777777)(59.90049751243781, 5.715833333333333)(60.0, 4.5255)
            };
            \addplot[color=blue, mark=none,name path=A] coordinates { %% MAX value
            (0.0, 7.37)(0.09950248756218906, 8.252)(0.19900497512437812, 8.485)(0.29850746268656714, 8.732)(0.39800995024875624, 8.77)(0.4975124378109453, 9.112)(0.5970149253731343, 9.314)(0.6965174129353234, 9.839)(0.7960199004975125, 9.803)(0.8955223880597015, 9.426)(0.9950248756218906, 9.783)(1.0945273631840795, 9.748)(1.1940298507462686, 9.613)(1.2935323383084576, 9.634)(1.3930348258706469, 9.758)(1.492537313432836, 9.955)(1.592039800995025, 9.45)(1.691542288557214, 9.447)(1.791044776119403, 9.426)(1.890547263681592, 9.866)(1.9900497512437811, 9.706)(2.0895522388059704, 9.733)(2.189054726368159, 9.495)(2.288557213930348, 9.659)(2.388059701492537, 9.68)(2.487562189054726, 9.564)(2.587064676616915, 9.771)(2.6865671641791047, 9.68)(2.7860696517412937, 9.633)(2.8855721393034828, 9.957)(2.985074626865672, 9.807)(3.084577114427861, 9.872)(3.18407960199005, 9.841)(3.283582089552239, 9.625)(3.383084577114428, 9.845)(3.482587064676617, 9.466)(3.582089552238806, 9.955)(3.6815920398009947, 9.957)(3.781094527363184, 9.591)(3.880597014925373, 9.568)(3.9800995024875623, 9.561)(4.079601990049751, 9.839)(4.179104477611941, 9.875)(4.27860696517413, 9.706)(4.378109452736318, 10.005)(4.477611940298508, 9.492)(4.577114427860696, 9.581)(4.676616915422886, 9.936)(4.776119402985074, 10.087)(4.875621890547264, 9.755)(4.975124378109452, 10.037)(5.074626865671642, 9.751)(5.17412935323383, 9.958)(5.27363184079602, 9.873)(5.373134328358209, 9.596)(5.472636815920398, 9.674)(5.5721393034825875, 9.629)(5.6716417910447765, 9.793)(5.7711442786069655, 9.902)(5.870646766169154, 9.7)(5.970149253731344, 9.839)(6.069651741293532, 9.8)(6.169154228855722, 10.061)(6.26865671641791, 10.049)(6.3681592039801, 9.923)(6.467661691542289, 9.903)(6.567164179104478, 10.053)(6.666666666666667, 9.781)(6.766169154228856, 9.807)(6.865671641791045, 9.659)(6.965174129353234, 9.869)(7.064676616915423, 9.757)(7.164179104477612, 9.62)(7.263681592039801, 9.657)(7.363184079601989, 9.907)(7.462686567164179, 9.984)(7.562189054726368, 9.927)(7.661691542288557, 9.832)(7.761194029850746, 10.022)(7.860696517412936, 9.871)(7.960199004975125, 9.814)(8.059701492537313, 9.826)(8.159203980099502, 9.971)(8.258706467661693, 9.996)(8.358208955223882, 9.843)(8.457711442786069, 9.542)(8.55721393034826, 9.802)(8.656716417910449, 9.663)(8.756218905472636, 9.904)(8.855721393034825, 9.885)(8.955223880597016, 9.738)(9.054726368159205, 10.041)(9.154228855721392, 9.787)(9.253731343283581, 9.738)(9.353233830845772, 9.654)(9.452736318407961, 9.9)(9.552238805970148, 9.908)(9.65174129353234, 10.054)(9.751243781094528, 9.885)(9.850746268656717, 9.711)(9.950248756218905, 10.143)(10.049751243781095, 9.991)(10.149253731343284, 9.892)(10.248756218905474, 10.34)(10.34825870646766, 10.269)(10.447761194029852, 10.256)(10.54726368159204, 9.9)(10.646766169154228, 9.958)(10.746268656716419, 9.848)(10.845771144278608, 9.918)(10.945273631840797, 9.861)(11.044776119402984, 9.845)(11.144278606965175, 9.929)(11.243781094527364, 9.894)(11.343283582089553, 9.854)(11.44278606965174, 9.807)(11.542288557213931, 9.819)(11.64179104477612, 9.848)(11.741293532338307, 9.928)(11.840796019900498, 10.048)(11.940298507462687, 9.888)(12.039800995024876, 9.869)(12.139303482587064, 9.845)(12.238805970149254, 10.068)(12.338308457711443, 10.265)(12.437810945273633, 9.779)(12.53731343283582, 9.895)(12.63681592039801, 10.103)(12.7363184079602, 9.895)(12.835820895522389, 9.846)(12.935323383084578, 10.096)(13.034825870646767, 10.051)(13.134328358208956, 9.873)(13.233830845771145, 9.968)(13.333333333333334, 9.763)(13.432835820895523, 9.884)(13.532338308457712, 9.875)(13.6318407960199, 9.91)(13.73134328358209, 9.966)(13.83084577114428, 9.725)(13.930348258706468, 9.981)(14.029850746268657, 9.912)(14.129353233830846, 10.006)(14.228855721393035, 9.905)(14.328358208955224, 9.966)(14.427860696517413, 9.766)(14.527363184079602, 9.801)(14.626865671641792, 9.628)(14.726368159203979, 10.074)(14.82587064676617, 9.886)(14.925373134328359, 10.054)(15.02487562189055, 9.993)(15.124378109452737, 9.772)(15.223880597014926, 9.885)(15.323383084577115, 10.097)(15.422885572139304, 9.856)(15.522388059701491, 9.752)(15.62189054726368, 9.986)(15.721393034825873, 10.067)(15.82089552238806, 9.918)(15.92039800995025, 9.941)(16.019900497512438, 9.988)(16.119402985074625, 9.703)(16.218905472636816, 9.866)(16.318407960199004, 9.828)(16.417910447761194, 9.938)(16.517412935323385, 9.718)(16.616915422885572, 9.815)(16.716417910447763, 10.021)(16.81592039800995, 9.881)(16.915422885572138, 10.019)(17.01492537313433, 9.82)(17.11442786069652, 9.812)(17.213930348258707, 9.962)(17.313432835820898, 9.96)(17.412935323383085, 9.802)(17.512437810945272, 9.951)(17.611940298507463, 9.945)(17.71144278606965, 9.942)(17.81094527363184, 9.837)(17.910447761194032, 10.033)(18.00995024875622, 9.914)(18.10945273631841, 9.919)(18.208955223880597, 9.834)(18.308457711442784, 10.024)(18.407960199004975, 9.989)(18.507462686567163, 9.786)(18.606965174129357, 9.939)(18.706467661691544, 9.858)(18.80597014925373, 10.069)(18.905472636815922, 10.05)(19.00497512437811, 9.803)(19.104477611940297, 9.997)(19.203980099502488, 10.016)(19.30348258706468, 9.732)(19.402985074626866, 9.879)(19.502487562189057, 9.658)(19.601990049751244, 10.034)(19.701492537313435, 9.949)(19.800995024875622, 9.868)(19.90049751243781, 10.016)(20.0, 9.776)(20.09950248756219, 10.322)(20.199004975124378, 9.824)(20.29850746268657, 9.944)(20.398009950248756, 9.82)(20.497512437810947, 9.615)(20.597014925373134, 9.789)(20.69651741293532, 9.955)(20.796019900497516, 10.015)(20.895522388059703, 9.866)(20.99502487562189, 10.29)(21.09452736318408, 10.026)(21.19402985074627, 9.666)(21.293532338308456, 9.759)(21.393034825870647, 10.055)(21.492537313432837, 9.942)(21.592039800995025, 9.991)(21.691542288557216, 9.9)(21.791044776119403, 9.796)(21.890547263681594, 10.189)(21.99004975124378, 10.008)(22.089552238805968, 9.954)(22.18905472636816, 9.857)(22.28855721393035, 9.859)(22.388059701492537, 10.05)(22.487562189054728, 10.019)(22.587064676616915, 9.748)(22.686567164179106, 10.019)(22.786069651741293, 10.109)(22.88557213930348, 9.921)(22.985074626865675, 9.798)(23.084577114427862, 9.941)(23.18407960199005, 10.055)(23.28358208955224, 10.136)(23.383084577114428, 9.962)(23.482587064676615, 10.059)(23.582089552238806, 9.743)(23.681592039800996, 9.731)(23.781094527363187, 9.979)(23.880597014925375, 9.778)(23.980099502487562, 10.056)(24.079601990049753, 10.039)(24.17910447761194, 9.89)(24.278606965174127, 10.013)(24.378109452736318, 9.927)(24.47761194029851, 9.924)(24.5771144278607, 9.979)(24.676616915422887, 9.997)(24.776119402985074, 9.848)(24.875621890547265, 9.851)(24.975124378109452, 9.881)(25.07462686567164, 9.914)(25.174129353233834, 10.099)(25.27363184079602, 9.969)(25.37313432835821, 10.101)(25.4726368159204, 10.117)(25.572139303482587, 9.973)(25.671641791044777, 10.02)(25.771144278606965, 9.947)(25.870646766169155, 10.043)(25.970149253731346, 9.787)(26.069651741293534, 9.922)(26.16915422885572, 10.142)(26.26865671641791, 10.066)(26.3681592039801, 9.869)(26.46766169154229, 9.917)(26.56716417910448, 10.11)(26.666666666666668, 10.044)(26.76616915422886, 10.542)(26.865671641791046, 10.512)(26.965174129353233, 10.584)(27.064676616915424, 10.545)(27.16417910447761, 10.474)(27.2636815920398, 10.767)(27.363184079601993, 10.524)(27.46268656716418, 10.474)(27.562189054726367, 10.649)(27.66169154228856, 10.263)(27.761194029850746, 9.964)(27.860696517412936, 9.941)(27.960199004975124, 10.118)(28.059701492537314, 10.037)(28.159203980099505, 10.601)(28.258706467661693, 10.036)(28.35820895522388, 9.997)(28.45771144278607, 10.023)(28.557213930348258, 10.686)(28.65671641791045, 10.7)(28.75621890547264, 10.021)(28.855721393034827, 10.606)(28.955223880597018, 10.599)(29.054726368159205, 10.508)(29.154228855721392, 10.562)(29.253731343283583, 10.435)(29.35323383084577, 9.968)(29.452736318407958, 10.02)(29.552238805970152, 10.153)(29.65174129353234, 9.982)(29.75124378109453, 10.328)(29.850746268656717, 10.049)(29.950248756218905, 9.959)(30.0497512437811, 9.911)(30.149253731343286, 10.389)(30.248756218905474, 10.156)(30.348258706467664, 10.175)(30.44776119402985, 10.024)(30.547263681592042, 10.275)(30.64676616915423, 9.905)(30.746268656716417, 9.963)(30.845771144278608, 9.935)(30.945273631840795, 9.985)(31.044776119402982, 9.967)(31.144278606965173, 10.127)(31.24378109452736, 9.917)(31.343283582089555, 9.99)(31.442786069651746, 9.874)(31.542288557213933, 10.036)(31.64179104477612, 10.025)(31.74129353233831, 9.889)(31.8407960199005, 10.042)(31.94029850746269, 9.915)(32.039800995024876, 9.86)(32.13930348258707, 10.074)(32.23880597014925, 9.945)(32.33830845771144, 9.999)(32.43781094527363, 9.995)(32.537313432835816, 9.987)(32.63681592039801, 10.019)(32.7363184079602, 10.017)(32.83582089552239, 9.948)(32.93532338308458, 9.942)(33.03482587064677, 10.039)(33.13432835820896, 10.04)(33.233830845771145, 10.107)(33.333333333333336, 9.988)(33.43283582089553, 9.835)(33.53233830845771, 9.856)(33.6318407960199, 10.043)(33.73134328358209, 9.793)(33.830845771144276, 9.956)(33.930348258706466, 9.902)(34.02985074626866, 10.162)(34.12935323383084, 9.804)(34.22885572139304, 10.038)(34.32835820895523, 9.869)(34.42786069651741, 9.702)(34.527363184079604, 9.962)(34.626865671641795, 9.889)(34.72636815920398, 9.864)(34.82587064676617, 10.098)(34.92537313432836, 9.873)(35.024875621890544, 9.835)(35.124378109452735, 10.097)(35.223880597014926, 9.891)(35.32338308457712, 10.092)(35.4228855721393, 10.077)(35.52238805970149, 10.128)(35.62189054726368, 9.91)(35.72139303482587, 10.176)(35.820895522388064, 9.902)(35.920398009950254, 9.743)(36.01990049751244, 9.957)(36.11940298507463, 9.992)(36.21890547263682, 9.998)(36.318407960199, 9.858)(36.417910447761194, 10.014)(36.517412935323385, 9.961)(36.61691542288557, 9.927)(36.71641791044776, 10.059)(36.81592039800995, 9.977)(36.91542288557214, 9.881)(37.014925373134325, 9.764)(37.114427860696516, 9.977)(37.213930348258714, 9.965)(37.3134328358209, 9.967)(37.41293532338309, 9.817)(37.51243781094528, 10.068)(37.61194029850746, 9.967)(37.711442786069654, 9.854)(37.810945273631845, 9.937)(37.91044776119403, 10.061)(38.00995024875622, 9.906)(38.10945273631841, 10.018)(38.208955223880594, 10.143)(38.308457711442784, 9.93)(38.407960199004975, 9.923)(38.50746268656716, 9.992)(38.60696517412936, 10.19)(38.70646766169155, 9.874)(38.80597014925373, 9.877)(38.90547263681592, 10.022)(39.00497512437811, 9.979)(39.1044776119403, 9.985)(39.20398009950249, 9.757)(39.30348258706468, 10.008)(39.40298507462687, 9.995)(39.50248756218905, 9.766)(39.601990049751244, 9.681)(39.701492537313435, 9.989)(39.80099502487562, 9.984)(39.90049751243781, 9.863)(40.0, 10.065)(40.09950248756219, 10.111)(40.19900497512438, 10.173)(40.29850746268657, 10.092)(40.398009950248756, 10.041)(40.49751243781095, 10.143)(40.59701492537314, 10.024)(40.69651741293532, 9.972)(40.79601990049751, 9.905)(40.8955223880597, 9.995)(40.995024875621894, 9.791)(41.09452736318408, 9.979)(41.19402985074627, 9.859)(41.29353233830846, 9.938)(41.39303482587064, 9.951)(41.49253731343284, 10.159)(41.59203980099503, 10.047)(41.691542288557216, 9.889)(41.791044776119406, 9.983)(41.8905472636816, 10.189)(41.99004975124378, 10.17)(42.08955223880597, 10.052)(42.18905472636816, 10.093)(42.288557213930346, 10.152)(42.38805970149254, 10.296)(42.48756218905473, 10.242)(42.58706467661691, 10.005)(42.6865671641791, 9.927)(42.78606965174129, 10.018)(42.88557213930348, 9.757)(42.985074626865675, 9.791)(43.084577114427866, 10.083)(43.18407960199005, 9.928)(43.28358208955224, 9.914)(43.38308457711443, 9.918)(43.48258706467662, 10.065)(43.582089552238806, 10.734)(43.681592039801, 10.661)(43.78109452736319, 10.711)(43.88059701492537, 10.615)(43.98009950248756, 10.132)(44.07960199004975, 10.503)(44.179104477611936, 11.053)(44.27860696517413, 10.458)(44.37810945273632, 10.482)(44.47761194029851, 9.91)(44.5771144278607, 9.695)(44.67661691542289, 9.852)(44.776119402985074, 9.876)(44.875621890547265, 9.955)(44.975124378109456, 9.815)(45.07462686567165, 9.922)(45.17412935323383, 10.045)(45.27363184079602, 9.695)(45.37313432835821, 9.857)(45.472636815920396, 9.694)(45.57213930348259, 10.197)(45.67164179104478, 10.011)(45.77114427860696, 9.922)(45.87064676616916, 10.016)(45.97014925373135, 9.908)(46.069651741293534, 10.214)(46.169154228855724, 10.251)(46.268656716417915, 10.043)(46.3681592039801, 9.895)(46.46766169154229, 9.897)(46.56716417910448, 10.258)(46.666666666666664, 10.222)(46.766169154228855, 9.978)(46.865671641791046, 10.138)(46.96517412935323, 9.955)(47.06467661691542, 9.922)(47.16417910447761, 9.882)(47.2636815920398, 10.147)(47.36318407960199, 10.043)(47.462686567164184, 9.909)(47.562189054726375, 9.857)(47.66169154228856, 10.078)(47.76119402985075, 9.841)(47.86069651741294, 9.914)(47.960199004975124, 9.824)(48.059701492537314, 9.935)(48.159203980099505, 9.927)(48.25870646766169, 9.874)(48.35820895522388, 10.01)(48.45771144278607, 9.959)(48.557213930348254, 10.07)(48.656716417910445, 10.09)(48.756218905472636, 10.537)(48.85572139303483, 10.368)(48.95522388059702, 10.324)(49.05472636815921, 10.355)(49.1542288557214, 10.067)(49.25373134328358, 9.808)(49.353233830845774, 9.926)(49.452736318407965, 9.946)(49.55223880597015, 9.759)(49.65174129353234, 9.804)(49.75124378109453, 9.903)(49.850746268656714, 10.339)(49.950248756218905, 10.284)(50.049751243781095, 10.285)(50.14925373134328, 10.236)(50.24875621890548, 9.906)(50.34825870646767, 9.907)(50.44776119402985, 9.954)(50.54726368159204, 9.69)(50.64676616915423, 10.019)(50.74626865671642, 9.847)(50.84577114427861, 9.797)(50.9452736318408, 9.749)(51.04477611940298, 9.753)(51.14427860696517, 10.056)(51.243781094527364, 9.798)(51.343283582089555, 9.839)(51.44278606965174, 9.956)(51.54228855721393, 10.116)(51.64179104477612, 10.005)(51.74129353233831, 9.837)(51.8407960199005, 9.921)(51.94029850746269, 9.892)(52.039800995024876, 9.904)(52.13930348258707, 9.941)(52.23880597014926, 9.924)(52.33830845771144, 9.834)(52.43781094527363, 10.127)(52.53731343283582, 10.071)(52.63681592039801, 9.856)(52.7363184079602, 10.018)(52.83582089552239, 9.977)(52.93532338308458, 9.928)(53.03482587064676, 9.75)(53.13432835820896, 9.723)(53.23383084577115, 9.661)(53.333333333333336, 9.901)(53.43283582089553, 9.893)(53.53233830845772, 9.924)(53.6318407960199, 10.023)(53.73134328358209, 10.226)(53.83084577114428, 10.016)(53.930348258706466, 9.91)(54.02985074626866, 10.076)(54.12935323383085, 10.195)(54.22885572139303, 9.889)(54.32835820895522, 9.898)(54.42786069651741, 9.905)(54.5273631840796, 9.846)(54.626865671641795, 9.81)(54.726368159203986, 9.964)(54.82587064676617, 10.017)(54.92537313432836, 9.955)(55.02487562189055, 10.045)(55.124378109452735, 10.139)(55.223880597014926, 10.118)(55.32338308457712, 10.074)(55.42288557213931, 10.755)(55.52238805970149, 10.109)(55.62189054726368, 10.02)(55.72139303482587, 10.135)(55.82089552238806, 9.969)(55.92039800995025, 9.709)(56.01990049751244, 10.086)(56.11940298507463, 10.016)(56.21890547263682, 10.214)(56.31840796019901, 9.957)(56.417910447761194, 10.1)(56.517412935323385, 9.933)(56.616915422885576, 10.115)(56.71641791044776, 10.425)(56.81592039800995, 10.124)(56.91542288557214, 10.358)(57.01492537313433, 9.869)(57.114427860696516, 9.889)(57.21393034825871, 9.765)(57.3134328358209, 9.929)(57.41293532338308, 9.808)(57.51243781094528, 9.966)(57.61194029850747, 9.829)(57.711442786069654, 9.975)(57.810945273631845, 9.892)(57.910447761194035, 10.002)(58.00995024875622, 10.055)(58.10945273631841, 10.055)(58.2089552238806, 9.873)(58.308457711442784, 9.875)(58.407960199004975, 10.008)(58.507462686567166, 9.776)(58.60696517412935, 9.729)(58.70646766169154, 10.003)(58.80597014925373, 9.927)(58.905472636815915, 10.198)(59.00497512437811, 9.956)(59.104477611940304, 9.829)(59.20398009950249, 10.063)(59.30348258706468, 9.805)(59.40298507462687, 10.943)(59.50248756218906, 10.512)(59.601990049751244, 10.511)(59.701492537313435, 10.537)(59.800995024875625, 10.339)(59.90049751243781, 9.478)(60.0, 6.735)
            };
            \addplot[color=blue, mark=none,name path=B] coordinates { %% MIN value
            (0.0, 2.374)(0.09950248756218906, 3.45)(0.19900497512437812, 4.809)(0.29850746268656714, 5.036)(0.39800995024875624, 5.768)(0.4975124378109453, 5.454)(0.5970149253731343, 5.879)(0.6965174129353234, 6.042)(0.7960199004975125, 6.569)(0.8955223880597015, 4.477)(0.9950248756218906, 4.327)(1.0945273631840795, 7.29)(1.1940298507462686, 7.36)(1.2935323383084576, 7.414)(1.3930348258706469, 7.228)(1.492537313432836, 7.422)(1.592039800995025, 7.334)(1.691542288557214, 7.093)(1.791044776119403, 7.388)(1.890547263681592, 7.307)(1.9900497512437811, 7.323)(2.0895522388059704, 2.176)(2.189054726368159, 3.886)(2.288557213930348, 3.59)(2.388059701492537, 3.518)(2.487562189054726, 3.533)(2.587064676616915, 3.565)(2.6865671641791047, 3.575)(2.7860696517412937, 3.578)(2.8855721393034828, 3.796)(2.985074626865672, 3.934)(3.084577114427861, 3.691)(3.18407960199005, 3.611)(3.283582089552239, 3.516)(3.383084577114428, 3.724)(3.482587064676617, 3.477)(3.582089552238806, 3.546)(3.6815920398009947, 3.573)(3.781094527363184, 3.533)(3.880597014925373, 3.699)(3.9800995024875623, 4.847)(4.079601990049751, 4.841)(4.179104477611941, 4.989)(4.27860696517413, 4.542)(4.378109452736318, 4.686)(4.477611940298508, 4.989)(4.577114427860696, 4.637)(4.676616915422886, 4.887)(4.776119402985074, 5.411)(4.875621890547264, 5.224)(4.975124378109452, 5.034)(5.074626865671642, 4.927)(5.17412935323383, 4.957)(5.27363184079602, 2.225)(5.373134328358209, 2.142)(5.472636815920398, 4.911)(5.5721393034825875, 4.934)(5.6716417910447765, 4.203)(5.7711442786069655, 4.93)(5.870646766169154, 4.94)(5.970149253731344, 5.005)(6.069651741293532, 5.019)(6.169154228855722, 4.932)(6.26865671641791, 4.865)(6.3681592039801, 5.063)(6.467661691542289, 5.352)(6.567164179104478, 5.065)(6.666666666666667, 4.861)(6.766169154228856, 4.834)(6.865671641791045, 4.86)(6.965174129353234, 4.995)(7.064676616915423, 4.982)(7.164179104477612, 4.92)(7.263681592039801, 4.954)(7.363184079601989, 3.561)(7.462686567164179, 2.212)(7.562189054726368, 4.708)(7.661691542288557, 4.911)(7.761194029850746, 4.941)(7.860696517412936, 4.907)(7.960199004975125, 4.948)(8.059701492537313, 5.031)(8.159203980099502, 4.602)(8.258706467661693, 3.567)(8.358208955223882, 3.548)(8.457711442786069, 3.596)(8.55721393034826, 3.582)(8.656716417910449, 3.56)(8.756218905472636, 3.509)(8.855721393034825, 3.589)(8.955223880597016, 3.625)(9.054726368159205, 3.534)(9.154228855721392, 3.664)(9.253731343283581, 3.6)(9.353233830845772, 3.475)(9.452736318407961, 3.548)(9.552238805970148, 3.598)(9.65174129353234, 3.488)(9.751243781094528, 3.585)(9.850746268656717, 3.49)(9.950248756218905, 3.573)(10.049751243781095, 3.652)(10.149253731343284, 3.664)(10.248756218905474, 3.555)(10.34825870646766, 3.512)(10.447761194029852, 3.55)(10.54726368159204, 3.569)(10.646766169154228, 3.557)(10.746268656716419, 3.916)(10.845771144278608, 4.923)(10.945273631840797, 5.508)(11.044776119402984, 5.298)(11.144278606965175, 4.99)(11.243781094527364, 5.011)(11.343283582089553, 5.016)(11.44278606965174, 4.893)(11.542288557213931, 5.018)(11.64179104477612, 4.932)(11.741293532338307, 4.957)(11.840796019900498, 2.41)(11.940298507462687, 3.862)(12.039800995024876, 4.948)(12.139303482587064, 5.046)(12.238805970149254, 5.005)(12.338308457711443, 4.984)(12.437810945273633, 4.947)(12.53731343283582, 4.873)(12.63681592039801, 5.003)(12.7363184079602, 5.019)(12.835820895522389, 4.961)(12.935323383084578, 5.1)(13.034825870646767, 4.841)(13.134328358208956, 5.09)(13.233830845771145, 4.952)(13.333333333333334, 4.986)(13.432835820895523, 4.967)(13.532338308457712, 4.924)(13.6318407960199, 3.629)(13.73134328358209, 3.546)(13.83084577114428, 3.554)(13.930348258706468, 3.59)(14.029850746268657, 3.615)(14.129353233830846, 5.042)(14.228855721393035, 5.063)(14.328358208955224, 4.911)(14.427860696517413, 5.048)(14.527363184079602, 6.658)(14.626865671641792, 7.368)(14.726368159203979, 7.448)(14.82587064676617, 7.372)(14.925373134328359, 7.495)(15.02487562189055, 7.31)(15.124378109452737, 7.11)(15.223880597014926, 7.391)(15.323383084577115, 7.395)(15.422885572139304, 7.057)(15.522388059701491, 7.448)(15.62189054726368, 7.346)(15.721393034825873, 7.251)(15.82089552238806, 5.202)(15.92039800995025, 2.257)(16.019900497512438, 4.105)(16.119402985074625, 4.198)(16.218905472636816, 4.246)(16.318407960199004, 4.185)(16.417910447761194, 4.185)(16.517412935323385, 4.18)(16.616915422885572, 4.225)(16.716417910447763, 4.805)(16.81592039800995, 4.345)(16.915422885572138, 4.113)(17.01492537313433, 4.184)(17.11442786069652, 4.164)(17.213930348258707, 4.236)(17.313432835820898, 4.165)(17.412935323383085, 4.161)(17.512437810945272, 4.296)(17.611940298507463, 4.212)(17.71144278606965, 4.133)(17.81094527363184, 4.176)(17.910447761194032, 4.101)(18.00995024875622, 4.221)(18.10945273631841, 4.193)(18.208955223880597, 4.177)(18.308457711442784, 6.736)(18.407960199004975, 7.343)(18.507462686567163, 3.617)(18.606965174129357, 2.434)(18.706467661691544, 7.298)(18.80597014925373, 7.336)(18.905472636815922, 7.344)(19.00497512437811, 7.566)(19.104477611940297, 7.379)(19.203980099502488, 7.464)(19.30348258706468, 7.29)(19.402985074626866, 4.783)(19.502487562189057, 3.615)(19.601990049751244, 3.627)(19.701492537313435, 3.54)(19.800995024875622, 4.081)(19.90049751243781, 3.995)(20.0, 5.559)(20.09950248756219, 2.902)(20.199004975124378, 4.718)(20.29850746268657, 4.741)(20.398009950248756, 4.363)(20.497512437810947, 5.293)(20.597014925373134, 5.983)(20.69651741293532, 5.771)(20.796019900497516, 5.868)(20.895522388059703, 5.821)(20.99502487562189, 4.215)(21.09452736318408, 4.884)(21.19402985074627, 5.884)(21.293532338308456, 5.818)(21.393034825870647, 5.895)(21.492537313432837, 6.004)(21.592039800995025, 5.838)(21.691542288557216, 5.862)(21.791044776119403, 5.718)(21.890547263681594, 5.867)(21.99004975124378, 5.85)(22.089552238805968, 5.845)(22.18905472636816, 6.101)(22.28855721393035, 6.863)(22.388059701492537, 6.965)(22.487562189054728, 6.923)(22.587064676616915, 6.778)(22.686567164179106, 6.913)(22.786069651741293, 7.48)(22.88557213930348, 4.69)(22.985074626865675, 4.859)(23.084577114427862, 7.545)(23.18407960199005, 2.694)(23.28358208955224, 2.774)(23.383084577114428, 7.481)(23.482587064676615, 7.669)(23.582089552238806, 7.266)(23.681592039800996, 7.486)(23.781094527363187, 7.475)(23.880597014925375, 7.384)(23.980099502487562, 7.344)(24.079601990049753, 7.506)(24.17910447761194, 7.409)(24.278606965174127, 7.378)(24.378109452736318, 7.528)(24.47761194029851, 7.672)(24.5771144278607, 7.293)(24.676616915422887, 7.548)(24.776119402985074, 7.474)(24.875621890547265, 7.643)(24.975124378109452, 7.536)(25.07462686567164, 7.797)(25.174129353233834, 7.803)(25.27363184079602, 7.473)(25.37313432835821, 7.539)(25.4726368159204, 7.562)(25.572139303482587, 7.418)(25.671641791044777, 7.482)(25.771144278606965, 7.409)(25.870646766169155, 7.454)(25.970149253731346, 7.456)(26.069651741293534, 7.49)(26.16915422885572, 7.301)(26.26865671641791, 7.34)(26.3681592039801, 7.494)(26.46766169154229, 7.98)(26.56716417910448, 4.726)(26.666666666666668, 4.495)(26.76616915422886, 3.304)(26.865671641791046, 2.646)(26.965174129353233, 7.572)(27.064676616915424, 6.315)(27.16417910447761, 3.8)(27.2636815920398, 3.557)(27.363184079601993, 3.602)(27.46268656716418, 3.558)(27.562189054726367, 3.624)(27.66169154228856, 3.578)(27.761194029850746, 5.319)(27.860696517412936, 6.819)(27.960199004975124, 6.965)(28.059701492537314, 6.687)(28.159203980099505, 6.93)(28.258706467661693, 6.812)(28.35820895522388, 6.95)(28.45771144278607, 7.498)(28.557213930348258, 7.557)(28.65671641791045, 7.51)(28.75621890547264, 7.383)(28.855721393034827, 7.46)(28.955223880597018, 7.58)(29.054726368159205, 7.362)(29.154228855721392, 4.073)(29.253731343283583, 3.787)(29.35323383084577, 3.55)(29.452736318407958, 3.575)(29.552238805970152, 3.665)(29.65174129353234, 4.136)(29.75124378109453, 3.743)(29.850746268656717, 3.559)(29.950248756218905, 3.584)(30.0497512437811, 3.563)(30.149253731343286, 3.534)(30.248756218905474, 4.405)(30.348258706467664, 4.905)(30.44776119402985, 4.962)(30.547263681592042, 4.904)(30.64676616915423, 5.057)(30.746268656716417, 4.918)(30.845771144278608, 4.954)(30.945273631840795, 4.917)(31.044776119402982, 4.861)(31.144278606965173, 5.197)(31.24378109452736, 4.958)(31.343283582089555, 4.884)(31.442786069651746, 4.99)(31.542288557213933, 4.907)(31.64179104477612, 4.955)(31.74129353233831, 4.899)(31.8407960199005, 4.958)(31.94029850746269, 4.95)(32.039800995024876, 4.941)(32.13930348258707, 5.123)(32.23880597014925, 6.785)(32.33830845771144, 7.302)(32.43781094527363, 7.189)(32.537313432835816, 6.898)(32.63681592039801, 4.184)(32.7363184079602, 3.575)(32.83582089552239, 3.504)(32.93532338308458, 4.152)(33.03482587064677, 3.994)(33.13432835820896, 2.28)(33.233830845771145, 5.051)(33.333333333333336, 4.991)(33.43283582089553, 2.474)(33.53233830845771, 2.475)(33.6318407960199, 5.007)(33.73134328358209, 2.997)(33.830845771144276, 2.997)(33.930348258706466, 3.263)(34.02985074626866, 3.436)(34.12935323383084, 3.629)(34.22885572139304, 3.62)(34.32835820895523, 3.383)(34.42786069651741, 4.344)(34.527363184079604, 4.553)(34.626865671641795, 4.554)(34.72636815920398, 4.515)(34.82587064676617, 4.56)(34.92537313432836, 4.478)(35.024875621890544, 4.549)(35.124378109452735, 4.621)(35.223880597014926, 4.459)(35.32338308457712, 4.642)(35.4228855721393, 4.519)(35.52238805970149, 4.54)(35.62189054726368, 3.231)(35.72139303482587, 2.347)(35.820895522388064, 2.971)(35.920398009950254, 3.378)(36.01990049751244, 3.325)(36.11940298507463, 4.499)(36.21890547263682, 3.604)(36.318407960199, 3.443)(36.417910447761194, 3.469)(36.517412935323385, 3.42)(36.61691542288557, 3.476)(36.71641791044776, 3.542)(36.81592039800995, 3.536)(36.91542288557214, 3.475)(37.014925373134325, 3.54)(37.114427860696516, 3.446)(37.213930348258714, 3.498)(37.3134328358209, 3.912)(37.41293532338309, 3.623)(37.51243781094528, 3.538)(37.61194029850746, 3.423)(37.711442786069654, 3.549)(37.810945273631845, 3.537)(37.91044776119403, 3.504)(38.00995024875622, 3.497)(38.10945273631841, 3.411)(38.208955223880594, 3.494)(38.308457711442784, 3.519)(38.407960199004975, 3.41)(38.50746268656716, 3.467)(38.60696517412936, 3.427)(38.70646766169155, 3.51)(38.80597014925373, 3.548)(38.90547263681592, 2.249)(39.00497512437811, 3.405)(39.1044776119403, 3.464)(39.20398009950249, 3.447)(39.30348258706468, 3.534)(39.40298507462687, 3.496)(39.50248756218905, 3.416)(39.601990049751244, 3.458)(39.701492537313435, 3.484)(39.80099502487562, 3.6)(39.90049751243781, 3.475)(40.0, 3.018)(40.09950248756219, 3.476)(40.19900497512438, 3.476)(40.29850746268657, 3.494)(40.398009950248756, 3.494)(40.49751243781095, 3.479)(40.59701492537314, 3.463)(40.69651741293532, 3.49)(40.79601990049751, 3.464)(40.8955223880597, 3.51)(40.995024875621894, 3.483)(41.09452736318408, 2.839)(41.19402985074627, 2.232)(41.29353233830846, 3.399)(41.39303482587064, 3.547)(41.49253731343284, 3.473)(41.59203980099503, 3.438)(41.691542288557216, 3.421)(41.791044776119406, 3.465)(41.8905472636816, 3.577)(41.99004975124378, 3.497)(42.08955223880597, 3.458)(42.18905472636816, 3.461)(42.288557213930346, 3.423)(42.38805970149254, 2.336)(42.48756218905473, 3.472)(42.58706467661691, 3.503)(42.6865671641791, 3.385)(42.78606965174129, 3.496)(42.88557213930348, 4.235)(42.985074626865675, 4.133)(43.084577114427866, 4.203)(43.18407960199005, 4.387)(43.28358208955224, 4.197)(43.38308457711443, 4.144)(43.48258706467662, 4.219)(43.582089552238806, 4.216)(43.681592039801, 4.295)(43.78109452736319, 4.871)(43.88059701492537, 4.61)(43.98009950248756, 2.506)(44.07960199004975, 4.858)(44.179104477611936, 4.968)(44.27860696517413, 4.841)(44.37810945273632, 4.849)(44.47761194029851, 4.944)(44.5771144278607, 4.871)(44.67661691542289, 4.929)(44.776119402985074, 4.908)(44.875621890547265, 4.917)(44.975124378109456, 4.95)(45.07462686567165, 4.904)(45.17412935323383, 5.028)(45.27363184079602, 4.878)(45.37313432835821, 5.007)(45.472636815920396, 4.953)(45.57213930348259, 4.953)(45.67164179104478, 5.004)(45.77114427860696, 4.905)(45.87064676616916, 4.544)(45.97014925373135, 2.293)(46.069651741293534, 4.947)(46.169154228855724, 5.005)(46.268656716417915, 4.892)(46.3681592039801, 5.007)(46.46766169154229, 4.915)(46.56716417910448, 3.071)(46.666666666666664, 4.272)(46.766169154228855, 4.29)(46.865671641791046, 4.17)(46.96517412935323, 2.87)(47.06467661691542, 2.507)(47.16417910447761, 4.229)(47.2636815920398, 4.165)(47.36318407960199, 4.17)(47.462686567164184, 4.183)(47.562189054726375, 4.212)(47.66169154228856, 4.234)(47.76119402985075, 4.197)(47.86069651741294, 4.147)(47.960199004975124, 4.201)(48.059701492537314, 4.273)(48.159203980099505, 4.096)(48.25870646766169, 4.261)(48.35820895522388, 4.201)(48.45771144278607, 4.229)(48.557213930348254, 4.214)(48.656716417910445, 4.253)(48.756218905472636, 4.152)(48.85572139303483, 2.176)(48.95522388059702, 2.825)(49.05472636815921, 4.933)(49.1542288557214, 4.967)(49.25373134328358, 4.875)(49.353233830845774, 5.094)(49.452736318407965, 4.902)(49.55223880597015, 3.705)(49.65174129353234, 3.502)(49.75124378109453, 3.507)(49.850746268656714, 3.658)(49.950248756218905, 3.484)(50.049751243781095, 3.542)(50.14925373134328, 3.491)(50.24875621890548, 3.523)(50.34825870646767, 3.455)(50.44776119402985, 3.519)(50.54726368159204, 3.493)(50.64676616915423, 3.557)(50.74626865671642, 3.466)(50.84577114427861, 3.645)(50.9452736318408, 3.514)(51.04477611940298, 3.533)(51.14427860696517, 3.524)(51.243781094527364, 3.513)(51.343283582089555, 3.481)(51.44278606965174, 3.507)(51.54228855721393, 3.479)(51.64179104477612, 3.472)(51.74129353233831, 2.27)(51.8407960199005, 3.62)(51.94029850746269, 3.499)(52.039800995024876, 3.516)(52.13930348258707, 3.601)(52.23880597014926, 3.529)(52.33830845771144, 3.481)(52.43781094527363, 3.506)(52.53731343283582, 3.448)(52.63681592039801, 3.601)(52.7363184079602, 3.47)(52.83582089552239, 3.592)(52.93532338308458, 2.948)(53.03482587064676, 2.949)(53.13432835820896, 3.525)(53.23383084577115, 3.46)(53.333333333333336, 3.564)(53.43283582089553, 3.365)(53.53233830845772, 3.471)(53.6318407960199, 3.421)(53.73134328358209, 3.506)(53.83084577114428, 3.647)(53.930348258706466, 3.516)(54.02985074626866, 3.418)(54.12935323383085, 3.508)(54.22885572139303, 3.446)(54.32835820895522, 3.464)(54.42786069651741, 3.494)(54.5273631840796, 3.482)(54.626865671641795, 2.906)(54.726368159203986, 3.481)(54.82587064676617, 3.522)(54.92537313432836, 3.496)(55.02487562189055, 3.439)(55.124378109452735, 3.468)(55.223880597014926, 3.506)(55.32338308457712, 3.49)(55.42288557213931, 3.536)(55.52238805970149, 3.769)(55.62189054726368, 2.839)(55.72139303482587, 3.531)(55.82089552238806, 3.559)(55.92039800995025, 3.531)(56.01990049751244, 3.465)(56.11940298507463, 3.488)(56.21890547263682, 2.332)(56.31840796019901, 3.553)(56.417910447761194, 3.57)(56.517412935323385, 3.428)(56.616915422885576, 4.092)(56.71641791044776, 3.986)(56.81592039800995, 3.815)(56.91542288557214, 3.571)(57.01492537313433, 3.516)(57.114427860696516, 3.472)(57.21393034825871, 3.507)(57.3134328358209, 3.464)(57.41293532338308, 3.544)(57.51243781094528, 3.56)(57.61194029850747, 3.468)(57.711442786069654, 3.536)(57.810945273631845, 3.604)(57.910447761194035, 3.657)(58.00995024875622, 3.931)(58.10945273631841, 3.624)(58.2089552238806, 3.473)(58.308457711442784, 3.461)(58.407960199004975, 3.523)(58.507462686567166, 3.527)(58.60696517412935, 3.52)(58.70646766169154, 3.555)(58.80597014925373, 3.595)(58.905472636815915, 3.484)(59.00497512437811, 3.561)(59.104477611940304, 3.502)(59.20398009950249, 3.511)(59.30348258706468, 3.474)(59.40298507462687, 3.477)(59.50248756218906, 3.576)(59.601990049751244, 1.986)(59.701492537313435, 3.028)(59.800995024875625, 1.546)(59.90049751243781, 2.431)(60.0, 2.316)
            };
            \addplot [pattern=north east lines,pattern color=red] 
            fill between [
                of=A and B,soft clip={domain=0:800},
            ];
            \end{axis}
\end{tikzpicture}
\caption{DUT: Surface Book}\label{fig:time_series_Fankuch_IntelPowerGadgetBook}
\end{subfigure}
\caption{FannkuchRedux, measured by IntelPowerGadget, with the lines representing the minimum, maximum and average energy consumption}\label{fig:time_series_Fankuch_IntelPowerGadget}
\end{figure}

Regarding energy consumption, we can see that the Surface Book uses the least, followed by the Surface Pro 4 and then the Workstation. If we look at the specifications for each DUT shown in \cref{tab:surfaceBook,tab:surfacePro}. Here the two laptops' CPUs have the same TDP ($15W$), however, following the trend seen in the graphs, the Surface Book's CPU's configurable TDP-down ($7.5W$) is lower than the Surface Pro 4's($9.5W$). However, the workstation's CPU's TDP is much higher at $65W$, which also shows in the energy consumption. %There seems to be a correlation between energy consumption and the TDP.

%så vælg DUt hvis forskellige measruing instruments. 

\subsection{Comparing the measuring instruments}
Finally, after comparing the different DUTs, this section will proceed with a comparison of the different measuring instruments.

\paragraph{Expectation:}
We expect the clamp measurements to have a higher energy consumption than the software-based measuring instruments, due to the clamp measurements measure the whole system. The software-based measuring approaches should be very similar. However, we expect that the Linux measurements are slightly lower than Windows due to Linux being a more lightweight OS.

\paragraph{Results:}
On \cref{fig:time_series_Fankuch_Workstation} the graphs for test case FannkuchRedux on the workstations can be observed. The workstation was chosen since this DUT is the only one with clamp measurements as well as software-based measuring instruments. When considering \cref{fig:time_series_Fankuch_WorkstationIntelPowerGadget,fig:time_series_Fankuch_WorkstationLHM}, which are the two software-based measuring instruments on Windows, there is a consistent average, although with a slight upwards trend throughout the runtime. The average is very similar as expected, although with Intel Power Gadget there is more variance and the peaks and valleys are further from the average. Notable both measuring instruments have a peak at $~10$ seconds. Looking at \cref{fig:time_series_Fankuch_WorkstationRAPL}, which is RAPL, the energy consumption is slightly lower, and the slight upwards trend is not present. When looking at the clamp measurements shown in \cref{fig:time_series_Fankuch_WorkstationClampW,fig:time_series_Fankuch_WorkstationClampL} there is a higher energy consumption as expected since the clamp measurements measure the system as a whole and not just the CPU. Furthermore, the average is not as straight as a line as the software-based measurements. For the clamp on Windows as shown in \cref{fig:time_series_Fankuch_WorkstationClampW} the average becomes more varied in the final $20$ seconds of the measurement.\todo{why?} In \cref{fig:time_series_Fankuch_WorkstationClampL} there is a pattern where there is a small drop followed by a sharp peak followed by a sharp drop, where it then slowly rises until the pattern starts again.\todo{Any ideas as to what causes this?} The pattern observed in \cref{subsec:comparing_test_case} regarding the initial measurement being $~0$ is also noticeable here, however only on LHM and RAPL. Furthermore, there is missing data.\todo{Whats going on here??}

\begin{figure}[H]  
    \centering 
    \begin{subfigure}[b]{0.49\linewidth}
        \begin{tikzpicture}
            \pgfplotsset{%
        width=1\linewidth,
        % height=1\textheight
        }
        \begin{axis}[ymax=120,
            xlabel={Time (Seconds)},
            ylabel={Energy Consumption (Joules)},
            ]
            \addplot[color=blue, mark=none,] coordinates { %% AVG value
            (0.0, 48.21988333333334)(0.09983361064891848, 51.91779166666666)(0.19966722129783696, 58.734224999999974)(0.2995008319467554, 53.84733333333332)(0.3993344425956739, 57.82893333333333)(0.49916805324459235, 57.81869166666668)(0.5990016638935108, 56.08766666666665)(0.6988352745424293, 58.84439999999999)(0.7986688851913478, 56.59066666666669)(0.8985024958402663, 56.609708333333366)(0.9983361064891847, 57.09276666666668)(1.0981697171381033, 57.04582500000001)(1.1980033277870217, 57.54740833333332)(1.2978369384359403, 57.39043333333333)(1.3976705490848587, 55.72083333333332)(1.4975041597337773, 56.696199999999976)(1.5973377703826956, 56.926075)(1.697171381031614, 56.95915833333336)(1.7970049916805326, 57.21712500000001)(1.896838602329451, 56.436700000000044)(1.9966722129783694, 56.53028333333334)(2.096505823627288, 56.70840000000001)(2.1963394342762066, 55.951991666666686)(2.296173044925125, 58.00165833333334)(2.3960066555740434, 55.90116666666669)(2.4958402662229617, 57.50826666666669)(2.5956738768718806, 57.47493333333335)(2.695507487520799, 57.03269166666666)(2.7953410981697173, 57.32719166666664)(2.8951747088186357, 56.26819166666667)(2.9950083194675545, 57.525849999999984)(3.094841930116473, 57.337075000000006)(3.1946755407653913, 57.20451666666668)(3.2945091514143097, 56.71186666666667)(3.394342762063228, 57.071366666666634)(3.494176372712147, 57.104558333333344)(3.5940099833610653, 57.48334166666666)(3.6938435940099836, 56.857066666666675)(3.793677204658902, 56.84204166666665)(3.8935108153078204, 57.207375)(3.9933444259567388, 56.99001666666664)(4.093178036605657, 58.48667500000001)(4.193011647254576, 57.254708333333355)(4.292845257903495, 57.24130833333333)(4.392678868552413, 57.526533333333326)(4.492512479201332, 56.675324999999994)(4.59234608985025, 58.207150000000006)(4.692179700499168, 57.404458333333345)(4.792013311148087, 56.95541666666666)(4.891846921797005, 57.89444999999998)(4.9916805324459235, 57.826049999999974)(5.091514143094843, 57.834125)(5.191347753743761, 57.60068333333334)(5.2911813643926795, 57.318308333333334)(5.391014975041598, 58.03585833333334)(5.490848585690516, 58.28424999999999)(5.590682196339435, 57.06525)(5.690515806988353, 57.74603333333336)(5.790349417637271, 57.96253333333334)(5.89018302828619, 57.37312499999999)(5.990016638935109, 58.321924999999965)(6.089850249584027, 57.129150000000045)(6.189683860232946, 58.54914166666665)(6.289517470881864, 57.872541666666656)(6.389351081530783, 57.41381666666668)(6.489184692179701, 57.85493333333333)(6.589018302828619, 57.06841666666664)(6.688851913477538, 58.14766666666667)(6.788685524126456, 57.95741666666665)(6.888519134775375, 57.51019999999999)(6.988352745424294, 58.41750833333335)(7.088186356073212, 57.31336666666665)(7.1880199667221305, 57.62010833333331)(7.287853577371049, 57.64659166666666)(7.387687188019967, 58.216375000000006)(7.487520798668886, 57.82960000000002)(7.587354409317804, 58.02724166666665)(7.687188019966722, 57.405525000000004)(7.787021630615641, 57.73911666666664)(7.886855241264559, 56.852549999999994)(7.9866888519134775, 58.04321666666665)(8.086522462562396, 58.836016666666666)(8.186356073211314, 57.45552500000001)(8.286189683860234, 57.11128333333332)(8.386023294509153, 57.762625000000035)(8.485856905158071, 57.87297500000001)(8.58569051580699, 57.36938333333332)(8.685524126455908, 57.83747500000001)(8.785357737104826, 57.536958333333295)(8.885191347753745, 58.259850000000014)(8.985024958402663, 57.72662500000002)(9.084858569051582, 58.803916666666645)(9.1846921797005, 58.06444166666666)(9.284525790349418, 58.13981666666666)(9.384359400998337, 58.82125000000003)(9.484193011647255, 58.00562499999999)(9.584026622296173, 58.586183333333324)(9.683860232945092, 58.635458333333325)(9.78369384359401, 57.528300000000016)(9.883527454242929, 58.71210833333332)(9.983361064891847, 58.406249999999986)(10.083194675540767, 58.20718333333333)(10.183028286189685, 58.734100000000005)(10.282861896838604, 57.91448333333332)(10.382695507487522, 58.32216666666667)(10.48252911813644, 57.69990833333333)(10.582362728785359, 57.13064166666667)(10.682196339434277, 58.572083333333325)(10.782029950083196, 58.16030833333332)(10.881863560732114, 57.54412499999998)(10.981697171381033, 58.45291666666666)(11.081530782029951, 58.462683333333345)(11.18136439267887, 58.79325000000001)(11.281198003327788, 58.52914166666666)(11.381031613976706, 58.612766666666644)(11.480865224625624, 58.45905000000002)(11.580698835274543, 58.75544166666668)(11.680532445923461, 57.63391666666668)(11.78036605657238, 59.22149166666666)(11.8801996672213, 58.48041666666667)(11.980033277870218, 57.67416666666671)(12.079866888519136, 59.807425)(12.179700499168055, 58.93529999999999)(12.279534109816973, 57.627050000000004)(12.379367720465892, 58.30583333333333)(12.47920133111481, 57.834983333333334)(12.579034941763728, 58.57301666666668)(12.678868552412647, 58.77347500000001)(12.778702163061565, 57.459858333333344)(12.878535773710484, 59.0463583333333)(12.978369384359402, 58.12898333333335)(13.07820299500832, 57.94775000000002)(13.178036605657239, 57.87256666666667)(13.277870216306157, 57.62191666666667)(13.377703826955075, 58.28202499999997)(13.477537437603994, 58.20480833333332)(13.577371048252912, 57.8483)(13.67720465890183, 57.88710000000001)(13.77703826955075, 58.440883333333325)(13.876871880199669, 57.50694166666666)(13.976705490848587, 58.15410833333332)(14.076539101497506, 58.139066666666665)(14.176372712146424, 57.39117499999999)(14.276206322795343, 57.9326)(14.376039933444261, 57.42876666666668)(14.47587354409318, 57.85422499999999)(14.575707154742098, 58.275174999999976)(14.675540765391016, 57.16953333333333)(14.775374376039935, 58.17819999999998)(14.875207986688853, 58.14019999999996)(14.975041597337771, 58.04575)(15.07487520798669, 57.954975)(15.174708818635608, 58.13080833333335)(15.274542429284526, 57.26194166666671)(15.374376039933445, 57.91935)(15.474209650582363, 58.31122499999997)(15.574043261231282, 58.232216666666666)(15.6738768718802, 57.857816666666665)(15.773710482529118, 57.98219166666666)(15.873544093178037, 58.08437499999999)(15.973377703826955, 58.13303333333333)(16.073211314475873, 58.58229166666667)(16.173044925124792, 58.79347500000001)(16.27287853577371, 57.256124999999976)(16.37271214642263, 57.59774166666664)(16.47254575707155, 58.29585000000001)(16.57237936772047, 57.114800000000024)(16.672212978369387, 58.01096666666663)(16.772046589018306, 57.88048333333332)(16.871880199667224, 57.936374999999984)(16.971713810316142, 57.830574999999996)(17.07154742096506, 58.16402500000001)(17.17138103161398, 57.298791666666645)(17.271214642262898, 57.942458333333335)(17.371048252911816, 57.792125000000006)(17.470881863560734, 58.254625000000004)(17.570715474209653, 57.663008333333345)(17.67054908485857, 58.230741666666674)(17.77038269550749, 58.33254999999998)(17.870216306156408, 57.688791666666674)(17.970049916805326, 57.82919166666669)(18.069883527454245, 57.64069166666666)(18.169717138103163, 57.63102499999997)(18.26955074875208, 57.88454166666666)(18.369384359401, 57.784833333333346)(18.469217970049918, 57.68211666666666)(18.569051580698837, 58.19019166666666)(18.668885191347755, 57.83133333333333)(18.768718801996673, 58.40570833333335)(18.86855241264559, 58.39685833333333)(18.96838602329451, 57.46141666666664)(19.06821963394343, 57.86581666666668)(19.168053244592347, 58.39303333333332)(19.267886855241265, 58.051649999999995)(19.367720465890184, 58.1679083333333)(19.467554076539102, 58.32088333333332)(19.56738768718802, 57.940791666666634)(19.66722129783694, 57.68915833333332)(19.767054908485857, 57.935116666666694)(19.866888519134775, 58.63459999999999)(19.966722129783694, 57.47228333333331)(20.066555740432612, 58.77204999999996)(20.166389351081534, 58.91198333333334)(20.266222961730453, 58.06535833333334)(20.36605657237937, 57.8071166666667)(20.46589018302829, 57.92220833333331)(20.565723793677208, 58.02731666666667)(20.665557404326126, 57.138933333333334)(20.765391014975044, 57.70775833333334)(20.865224625623963, 57.83314166666665)(20.96505823627288, 58.15463333333334)(21.0648918469218, 57.81801666666668)(21.164725457570718, 57.849366666666675)(21.264559068219636, 58.17216666666668)(21.364392678868555, 58.192275000000016)(21.464226289517473, 57.94262500000002)(21.56405990016639, 58.142650000000025)(21.66389351081531, 58.23838333333337)(21.76372712146423, 58.20271666666668)(21.863560732113147, 58.13859166666666)(21.963394342762065, 58.17633333333334)(22.063227953410983, 58.25283333333334)(22.163061564059902, 58.589150000000004)(22.26289517470882, 58.06528333333335)(22.36272878535774, 58.22290833333332)(22.462562396006657, 58.16661666666668)(22.562396006655575, 58.27689999999998)(22.662229617304494, 57.90454999999998)(22.762063227953412, 57.837024999999976)(22.86189683860233, 57.85294999999997)(22.96173044925125, 58.02257499999999)(23.061564059900167, 57.98859166666666)(23.161397670549086, 58.18199999999998)(23.261231281198004, 57.79010833333332)(23.361064891846922, 57.41535833333334)(23.46089850249584, 57.73982499999997)(23.56073211314476, 58.628141666666664)(23.660565723793678, 57.69488333333333)(23.7603993344426, 58.36234999999999)(23.860232945091518, 58.105299999999986)(23.960066555740436, 57.928899999999985)(24.059900166389355, 58.493683333333294)(24.159733777038273, 58.905674999999995)(24.25956738768719, 57.53740833333337)(24.35940099833611, 58.40520000000003)(24.459234608985028, 57.82364166666665)(24.559068219633946, 57.53975833333335)(24.658901830282865, 58.62568333333332)(24.758735440931783, 57.52001666666665)(24.8585690515807, 58.55249166666668)(24.95840266222962, 58.903400000000026)(25.05823627287854, 57.54786666666667)(25.158069883527457, 58.87505833333334)(25.257903494176375, 58.10345833333333)(25.357737104825294, 58.19779166666669)(25.457570715474212, 59.32973333333334)(25.55740432612313, 58.034800000000004)(25.65723793677205, 58.761291666666686)(25.757071547420967, 58.706791666666646)(25.856905158069885, 57.754300000000015)(25.956738768718804, 58.86574166666665)(26.056572379367722, 58.380083333333324)(26.15640599001664, 57.85956666666665)(26.25623960066556, 58.938958333333325)(26.356073211314477, 58.03998333333332)(26.455906821963396, 58.507483333333326)(26.555740432612314, 58.37959166666664)(26.655574043261232, 58.28213333333331)(26.75540765391015, 58.11746666666665)(26.85524126455907, 58.349299999999985)(26.955074875207988, 58.54689166666667)(27.054908485856906, 58.10606666666664)(27.154742096505824, 58.27427500000001)(27.254575707154743, 58.00487499999999)(27.35440931780366, 58.65178333333335)(27.45424292845258, 58.442625)(27.5540765391015, 57.75548333333333)(27.653910149750416, 58.8531)(27.753743760399338, 58.17071666666666)(27.853577371048253, 58.22205000000001)(27.953410981697175, 58.743991666666666)(28.05324459234609, 59.017208333333336)(28.15307820299501, 59.34572499999999)(28.252911813643927, 57.59946666666667)(28.35274542429285, 58.191100000000006)(28.452579034941763, 58.91911666666671)(28.552412645590685, 58.02106666666667)(28.6522462562396, 58.88274166666668)(28.752079866888522, 58.07051666666668)(28.851913477537437, 58.257191666666685)(28.95174708818636, 58.639999999999986)(29.051580698835274, 58.60205833333335)(29.151414309484196, 58.036358333333325)(29.25124792013311, 59.07980833333332)(29.351081530782032, 57.92594999999998)(29.450915141430947, 58.273699999999984)(29.55074875207987, 58.839458333333305)(29.650582362728784, 58.217450000000014)(29.750415973377706, 59.220666666666666)(29.85024958402662, 58.146483333333315)(29.950083194675543, 58.22658333333333)(30.049916805324457, 58.56862499999999)(30.14975041597338, 58.16846666666665)(30.249584026622294, 57.695600000000006)(30.349417637271216, 58.401824999999995)(30.44925124792013, 57.686091666666655)(30.549084858569053, 58.004175000000004)(30.648918469217968, 58.58541666666667)(30.74875207986689, 57.55280833333332)(30.848585690515804, 59.0824)(30.948419301164726, 58.141166666666656)(31.04825291181364, 57.80878333333336)(31.148086522462563, 58.19255833333333)(31.247920133111478, 58.345483333333306)(31.3477537437604, 58.221458333333324)(31.447587354409315, 58.457741666666685)(31.547420965058237, 57.865333333333325)(31.64725457570715, 58.42226666666665)(31.747088186356073, 58.266625)(31.84692179700499, 57.36766666666664)(31.94675540765391, 58.080083333333356)(32.046589018302825, 57.58853333333331)(32.14642262895175, 59.261874999999954)(32.24625623960066, 57.99745833333331)(32.346089850249584, 57.57714166666666)(32.4459234608985, 57.87827499999997)(32.54575707154742, 58.929208333333314)(32.645590682196335, 57.74645833333334)(32.74542429284526, 58.38113333333332)(32.84525790349417, 58.314674999999994)(32.9450915141431, 57.93735000000001)(33.044925124792016, 58.61523333333333)(33.14475873544094, 57.561316666666684)(33.24459234608985, 57.71270833333331)(33.344425956738775, 57.996758333333304)(33.44425956738769, 58.700483333333324)(33.54409317803661, 57.83857499999999)(33.643926788685526, 57.854983333333365)(33.74376039933445, 58.43061666666666)(33.84359400998336, 58.418983333333365)(33.943427620632285, 58.20513333333333)(34.0432612312812, 59.02405000000002)(34.14309484193012, 57.77397499999997)(34.24292845257904, 58.747366666666686)(34.34276206322796, 58.1079333333333)(34.44259567387687, 58.424766666666635)(34.542429284525795, 57.83110000000002)(34.64226289517471, 58.31913333333335)(34.74209650582363, 58.630633333333314)(34.84193011647255, 58.01506666666666)(34.94176372712147, 58.635566666666676)(35.04159733777038, 57.50981666666669)(35.141430948419305, 58.80890000000001)(35.24126455906822, 58.340016666666656)(35.34109816971714, 58.508725)(35.44093178036606, 58.706908333333324)(35.54076539101498, 58.52671666666668)(35.640599001663894, 58.211525)(35.740432612312816, 58.269574999999996)(35.84026622296173, 57.93952499999999)(35.94009983361065, 58.652100000000004)(36.03993344425957, 58.03631666666669)(36.13976705490849, 59.80693333333332)(36.239600665557404, 58.88819166666665)(36.339434276206326, 57.754774999999995)(36.43926788685524, 58.57686666666666)(36.53910149750416, 58.99773333333333)(36.63893510815308, 58.133708333333324)(36.738768718802, 58.7722416666667)(36.838602329450914, 59.28005000000001)(36.938435940099836, 57.86103333333333)(37.03826955074875, 59.176150000000014)(37.13810316139767, 58.67609166666662)(37.23793677204659, 58.87825)(37.33777038269551, 59.59644166666667)(37.437603993344425, 58.41046666666667)(37.53743760399335, 58.58268333333331)(37.63727121464226, 58.66234166666667)(37.73710482529118, 58.80891666666668)(37.8369384359401, 59.1417666666667)(37.93677204658902, 59.29799166666667)(38.036605657237935, 59.289933333333344)(38.13643926788686, 59.70965)(38.23627287853577, 58.77356666666669)(38.336106489184694, 58.965850000000025)(38.43594009983361, 58.96815833333332)(38.53577371048253, 58.76604166666667)(38.635607321131445, 59.35532500000001)(38.73544093178037, 58.754100000000015)(38.83527454242928, 59.774591666666666)(38.935108153078204, 59.43879999999999)(39.03494176372712, 58.709583333333335)(39.13477537437604, 58.86879999999999)(39.234608985024956, 59.80196666666668)(39.33444259567388, 59.370799999999996)(39.43427620632279, 58.91553333333332)(39.534109816971714, 58.65395833333334)(39.63394342762063, 58.83245000000002)(39.73377703826955, 59.043391666666636)(39.833610648918466, 59.2092)(39.93344425956739, 59.04994166666664)(40.0332778702163, 59.699983333333336)(40.133111480865225, 60.8489916666667)(40.232945091514146, 58.753975)(40.33277870216307, 59.21152499999998)(40.43261231281198, 58.92220000000001)(40.532445923460905, 57.81406666666668)(40.63227953410982, 59.27180833333333)(40.73211314475874, 59.509916666666676)(40.83194675540766, 57.99486666666668)(40.93178036605658, 59.73048333333332)(41.03161397670549, 59.560350000000014)(41.131447587354415, 58.79998333333333)(41.23128119800333, 59.81038333333333)(41.33111480865225, 58.672383333333315)(41.43094841930117, 58.89354166666668)(41.53078202995009, 59.17903333333331)(41.630615640599004, 59.55133333333335)(41.730449251247926, 58.724258333333324)(41.83028286189684, 59.90230833333332)(41.93011647254576, 58.88338333333332)(42.02995008319468, 59.52902499999998)(42.1297836938436, 59.284300000000016)(42.229617304492514, 58.937075000000014)(42.329450915141436, 59.932550000000006)(42.42928452579035, 59.60292500000001)(42.52911813643927, 58.77144166666666)(42.62895174708819, 59.93267500000001)(42.72878535773711, 59.209066666666665)(42.828618968386024, 59.61660833333333)(42.928452579034946, 59.87521666666667)(43.02828618968386, 58.55789166666664)(43.12811980033278, 59.856525)(43.2279534109817, 59.80508333333331)(43.32778702163062, 59.56744166666668)(43.427620632279535, 60.243708333333366)(43.52745424292846, 59.49425)(43.62728785357737, 59.23664166666667)(43.72712146422629, 60.10470833333332)(43.82695507487521, 59.530066666666684)(43.92678868552413, 60.254283333333305)(44.026622296173045, 60.44260833333336)(44.12645590682197, 59.90587500000002)(44.22628951747088, 60.139224999999996)(44.326123128119804, 59.621400000000015)(44.42595673876872, 59.37199999999998)(44.52579034941764, 60.482883333333334)(44.625623960066555, 59.85184166666666)(44.72545757071548, 59.67564166666664)(44.82529118136439, 60.32070833333334)(44.925124792013314, 59.77105833333334)(45.02495840266223, 59.82191666666666)(45.12479201331115, 60.08697499999999)(45.224625623960065, 59.35990833333332)(45.32445923460899, 59.592000000000006)(45.4242928452579, 59.809483333333354)(45.524126455906824, 59.71126666666668)(45.62396006655574, 60.44449166666664)(45.72379367720466, 60.84959999999999)(45.823627287853576, 59.79336666666666)(45.9234608985025, 60.07230833333332)(46.02329450915141, 60.410300000000014)(46.123128119800334, 59.83058333333336)(46.22296173044925, 60.77741666666665)(46.32279534109817, 60.119708333333314)(46.422628951747086, 60.23826666666667)(46.52246256239601, 60.13477499999999)(46.62229617304492, 59.69924166666666)(46.722129783693845, 60.4723)(46.82196339434276, 60.2182666666667)(46.92179700499168, 60.05324166666666)(47.021630615640596, 60.19575833333334)(47.12146422628952, 60.08200833333334)(47.22129783693843, 60.719775)(47.321131447587355, 60.59821666666673)(47.42096505823627, 60.682008333333314)(47.5207986688852, 60.817474999999995)(47.620632279534114, 59.53203333333335)(47.720465890183036, 60.42025833333333)(47.82029950083195, 60.52396666666669)(47.92013311148087, 61.211750000000016)(48.01996672212979, 60.52882500000002)(48.11980033277871, 60.7373666666667)(48.219633943427624, 60.28704166666667)(48.319467554076546, 60.33181666666669)(48.41930116472546, 59.99950833333337)(48.51913477537438, 60.44073333333333)(48.6189683860233, 60.612200000000016)(48.71880199667222, 59.83152499999997)(48.818635607321134, 60.57335833333333)(48.918469217970056, 60.59350000000002)(49.01830282861897, 60.50696666666666)(49.11813643926789, 61.268175)(49.21797004991681, 60.65858333333336)(49.31780366056573, 60.674475)(49.417637271214645, 60.304716666666664)(49.51747088186357, 60.91398333333332)(49.61730449251248, 61.36495833333332)(49.7171381031614, 60.51624166666665)(49.81697171381032, 60.69317500000001)(49.91680532445924, 60.99286666666667)(50.016638935108155, 60.82545)(50.11647254575708, 61.43866666666666)(50.21630615640599, 61.38371666666667)(50.31613976705491, 61.19751666666668)(50.41597337770383, 61.006150000000005)(50.51580698835275, 61.00678333333336)(50.615640599001665, 60.530725000000025)(50.71547420965059, 61.31093333333333)(50.8153078202995, 61.07538333333335)(50.915141430948424, 61.27456666666665)(51.01497504159734, 61.10125000000003)(51.11480865224626, 61.81923333333332)(51.214642262895175, 60.65160833333333)(51.3144758735441, 61.36236666666666)(51.41430948419301, 61.12884999999997)(51.514143094841934, 60.73637499999996)(51.61397670549085, 61.08379166666665)(51.71381031613977, 60.98598333333332)(51.813643926788686, 60.85883333333335)(51.91347753743761, 61.45158333333337)(52.01331114808652, 60.71275000000003)(52.113144758735444, 61.20731666666667)(52.21297836938436, 61.63860833333339)(52.31281198003328, 61.45618333333335)(52.412645590682196, 60.341533333333324)(52.51247920133112, 61.992741666666674)(52.61231281198003, 60.32594166666666)(52.712146422628955, 61.50972500000001)(52.81198003327787, 61.74897499999999)(52.91181364392679, 60.84354166666666)(53.011647254575706, 60.86894166666666)(53.11148086522463, 61.11336666666669)(53.21131447587354, 61.05315)(53.311148086522465, 61.46170833333334)(53.41098169717138, 61.48522499999998)(53.5108153078203, 60.719449999999945)(53.61064891846922, 61.356641666666654)(53.71048252911814, 61.87217500000001)(53.81031613976705, 60.91028333333331)(53.910149750415975, 61.17941666666665)(54.00998336106489, 61.402058333333336)(54.10981697171381, 61.18543333333335)(54.20965058236273, 61.40372499999998)(54.30948419301165, 61.64863333333335)(54.409317803660564, 61.62541666666664)(54.509151414309486, 61.598966666666676)(54.6089850249584, 61.321791666666655)(54.70881863560732, 61.452625000000005)(54.808652246256244, 61.57366666666666)(54.90848585690516, 60.835391666666666)(55.00831946755408, 61.30402500000001)(55.108153078203, 60.858183333333336)(55.20798668885192, 60.90859166666666)(55.30782029950083, 61.037049999999994)(55.407653910149754, 61.11860833333333)(55.507487520798676, 60.52362500000002)(55.60732113144759, 60.38001666666669)(55.707154742096506, 61.44250833333332)(55.80698835274543, 61.145775)(55.90682196339435, 60.89184166666667)(56.006655574043265, 61.578125)(56.10648918469218, 61.59771666666663)(56.2063227953411, 61.76641666666667)(56.30615640599002, 61.38554166666669)(56.40599001663894, 61.395975000000014)(56.50582362728785, 61.04043333333333)(56.605657237936775, 61.33874166666664)(56.7054908485857, 60.67652500000001)(56.80532445923461, 61.20141666666666)(56.90515806988353, 61.62620833333334)(57.00499168053245, 60.92076666666664)(57.10482529118137, 61.178275000000006)(57.204658901830285, 61.41579166666666)(57.3044925124792, 61.29492499999999)(57.40432612312812, 61.37351666666669)(57.504159733777044, 61.56623333333334)(57.60399334442596, 61.4048)(57.703826955074874, 61.58738333333331)(57.803660565723796, 61.57108333333336)(57.90349417637272, 61.24882499999999)(58.00332778702163, 61.0018)(58.10316139767055, 61.81615000000001)(58.20299500831947, 61.49299166666667)(58.30282861896839, 61.740908333333344)(58.402662229617306, 61.63536666666667)(58.50249584026622, 61.54935000000001)(58.60232945091514, 61.386266666666685)(58.702163061564065, 61.44199999999999)(58.80199667221298, 61.71741666666667)(58.901830282861894, 61.99364166666666)(59.001663893510816, 62.070116666666685)(59.10149750415974, 61.72813333333332)(59.20133111480865, 61.76678333333334)(59.30116472545757, 61.60629999999999)(59.40099833610649, 61.635541666666654)(59.50083194675541, 61.41445833333335)(59.60066555740433, 61.72413333333332)(59.70049916805324, 61.83704166666666)(59.80033277870216, 61.63337499999998)(59.900166389351085, 43.117584745762684)(60.0, 12.288)
            };
            \addplot[color=blue, mark=none,name path=A] coordinates { %% MAX value
            (0.0, 56.738)(0.09983361064891848, 61.783)(0.19966722129783696, 65.882)(0.2995008319467554, 63.516)(0.3993344425956739, 64.132)(0.49916805324459235, 65.318)(0.5990016638935108, 64.012)(0.6988352745424293, 64.388)(0.7986688851913478, 63.707)(0.8985024958402663, 65.122)(0.9983361064891847, 64.7)(1.0981697171381033, 64.038)(1.1980033277870217, 64.856)(1.2978369384359403, 71.844)(1.3976705490848587, 69.42)(1.4975041597337773, 67.445)(1.5973377703826956, 68.423)(1.697171381031614, 70.191)(1.7970049916805326, 80.858)(1.896838602329451, 69.922)(1.9966722129783694, 72.763)(2.096505823627288, 67.007)(2.1963394342762066, 65.122)(2.296173044925125, 65.956)(2.3960066555740434, 64.962)(2.4958402662229617, 66.554)(2.5956738768718806, 65.584)(2.695507487520799, 65.785)(2.7953410981697173, 65.868)(2.8951747088186357, 64.942)(2.9950083194675545, 64.563)(3.094841930116473, 65.287)(3.1946755407653913, 64.552)(3.2945091514143097, 64.866)(3.394342762063228, 66.038)(3.494176372712147, 64.953)(3.5940099833610653, 64.494)(3.6938435940099836, 64.692)(3.793677204658902, 65.23)(3.8935108153078204, 65.436)(3.9933444259567388, 65.017)(4.093178036605657, 64.823)(4.193011647254576, 65.339)(4.292845257903495, 65.494)(4.392678868552413, 64.511)(4.492512479201332, 64.338)(4.59234608985025, 66.254)(4.692179700499168, 64.682)(4.792013311148087, 64.866)(4.891846921797005, 65.357)(4.9916805324459235, 65.306)(5.091514143094843, 64.494)(5.191347753743761, 64.927)(5.2911813643926795, 64.631)(5.391014975041598, 65.398)(5.490848585690516, 65.661)(5.590682196339435, 66.445)(5.690515806988353, 65.482)(5.790349417637271, 65.473)(5.89018302828619, 66.174)(5.990016638935109, 64.892)(6.089850249584027, 65.848)(6.189683860232946, 64.226)(6.289517470881864, 65.307)(6.389351081530783, 64.19)(6.489184692179701, 65.166)(6.589018302828619, 64.805)(6.688851913477538, 65.516)(6.788685524126456, 65.129)(6.888519134775375, 65.348)(6.988352745424294, 65.149)(7.088186356073212, 65.146)(7.1880199667221305, 64.407)(7.287853577371049, 64.637)(7.387687188019967, 66.955)(7.487520798668886, 65.844)(7.587354409317804, 64.935)(7.687188019966722, 65.426)(7.787021630615641, 66.119)(7.886855241264559, 65.264)(7.9866888519134775, 66.624)(8.086522462562396, 64.677)(8.186356073211314, 66.821)(8.286189683860234, 65.136)(8.386023294509153, 64.63)(8.485856905158071, 64.715)(8.58569051580699, 64.946)(8.685524126455908, 65.518)(8.785357737104826, 65.519)(8.885191347753745, 65.621)(8.985024958402663, 89.52)(9.084858569051582, 108.562)(9.1846921797005, 105.888)(9.284525790349418, 72.984)(9.384359400998337, 83.31)(9.484193011647255, 83.359)(9.584026622296173, 100.586)(9.683860232945092, 99.391)(9.78369384359401, 98.319)(9.883527454242929, 89.877)(9.983361064891847, 96.568)(10.083194675540767, 96.31)(10.183028286189685, 94.341)(10.282861896838604, 80.751)(10.382695507487522, 74.745)(10.48252911813644, 77.82)(10.582362728785359, 76.17)(10.682196339434277, 65.13)(10.782029950083196, 74.333)(10.881863560732114, 83.361)(10.981697171381033, 81.438)(11.081530782029951, 93.644)(11.18136439267887, 108.015)(11.281198003327788, 104.619)(11.381031613976706, 81.651)(11.480865224625624, 80.787)(11.580698835274543, 75.686)(11.680532445923461, 87.758)(11.78036605657238, 95.297)(11.8801996672213, 96.331)(11.980033277870218, 90.639)(12.079866888519136, 97.523)(12.179700499168055, 87.958)(12.279534109816973, 91.001)(12.379367720465892, 83.449)(12.47920133111481, 76.707)(12.579034941763728, 76.571)(12.678868552412647, 73.614)(12.778702163061565, 68.665)(12.878535773710484, 72.491)(12.978369384359402, 84.652)(13.07820299500832, 86.063)(13.178036605657239, 78.63)(13.277870216306157, 76.874)(13.377703826955075, 81.622)(13.477537437603994, 77.323)(13.577371048252912, 78.193)(13.67720465890183, 75.372)(13.77703826955075, 73.033)(13.876871880199669, 75.473)(13.976705490848587, 70.103)(14.076539101497506, 67.916)(14.176372712146424, 68.011)(14.276206322795343, 67.242)(14.376039933444261, 70.763)(14.47587354409318, 67.411)(14.575707154742098, 77.857)(14.675540765391016, 70.219)(14.775374376039935, 64.762)(14.875207986688853, 64.875)(14.975041597337771, 65.045)(15.07487520798669, 65.183)(15.174708818635608, 65.084)(15.274542429284526, 66.881)(15.374376039933445, 65.068)(15.474209650582363, 67.429)(15.574043261231282, 64.697)(15.6738768718802, 65.904)(15.773710482529118, 66.698)(15.873544093178037, 65.202)(15.973377703826955, 65.06)(16.073211314475873, 65.485)(16.173044925124792, 67.505)(16.27287853577371, 66.019)(16.37271214642263, 65.424)(16.47254575707155, 67.103)(16.57237936772047, 67.746)(16.672212978369387, 67.648)(16.772046589018306, 66.463)(16.871880199667224, 64.585)(16.971713810316142, 69.838)(17.07154742096506, 67.973)(17.17138103161398, 64.875)(17.271214642262898, 64.889)(17.371048252911816, 67.344)(17.470881863560734, 65.152)(17.570715474209653, 65.018)(17.67054908485857, 66.251)(17.77038269550749, 65.88)(17.870216306156408, 67.29)(17.970049916805326, 65.685)(18.069883527454245, 66.477)(18.169717138103163, 65.285)(18.26955074875208, 65.25)(18.369384359401, 65.044)(18.469217970049918, 65.155)(18.569051580698837, 65.443)(18.668885191347755, 66.193)(18.768718801996673, 65.971)(18.86855241264559, 65.007)(18.96838602329451, 64.941)(19.06821963394343, 65.708)(19.168053244592347, 66.274)(19.267886855241265, 66.892)(19.367720465890184, 65.083)(19.467554076539102, 65.611)(19.56738768718802, 64.885)(19.66722129783694, 65.636)(19.767054908485857, 65.816)(19.866888519134775, 65.657)(19.966722129783694, 65.687)(20.066555740432612, 67.523)(20.166389351081534, 66.164)(20.266222961730453, 67.641)(20.36605657237937, 65.331)(20.46589018302829, 66.057)(20.565723793677208, 65.4)(20.665557404326126, 64.754)(20.765391014975044, 65.276)(20.865224625623963, 65.784)(20.96505823627288, 64.95)(21.0648918469218, 65.359)(21.164725457570718, 64.926)(21.264559068219636, 65.025)(21.364392678868555, 65.174)(21.464226289517473, 65.914)(21.56405990016639, 64.787)(21.66389351081531, 65.774)(21.76372712146423, 67.513)(21.863560732113147, 64.644)(21.963394342762065, 65.401)(22.063227953410983, 65.78)(22.163061564059902, 65.132)(22.26289517470882, 65.481)(22.36272878535774, 66.321)(22.462562396006657, 64.771)(22.562396006655575, 65.26)(22.662229617304494, 79.764)(22.762063227953412, 75.22)(22.86189683860233, 80.21)(22.96173044925125, 74.535)(23.061564059900167, 73.048)(23.161397670549086, 65.353)(23.261231281198004, 64.98)(23.361064891846922, 65.967)(23.46089850249584, 65.344)(23.56073211314476, 64.693)(23.660565723793678, 65.473)(23.7603993344426, 65.965)(23.860232945091518, 65.542)(23.960066555740436, 65.493)(24.059900166389355, 65.19)(24.159733777038273, 65.567)(24.25956738768719, 65.181)(24.35940099833611, 65.828)(24.459234608985028, 65.99)(24.559068219633946, 64.937)(24.658901830282865, 65.417)(24.758735440931783, 64.87)(24.8585690515807, 65.653)(24.95840266222962, 64.737)(25.05823627287854, 65.908)(25.158069883527457, 77.091)(25.257903494176375, 78.887)(25.357737104825294, 83.758)(25.457570715474212, 82.378)(25.55740432612313, 66.231)(25.65723793677205, 65.97)(25.757071547420967, 65.717)(25.856905158069885, 65.461)(25.956738768718804, 65.582)(26.056572379367722, 65.302)(26.15640599001664, 65.676)(26.25623960066556, 65.938)(26.356073211314477, 65.127)(26.455906821963396, 65.321)(26.555740432612314, 64.937)(26.655574043261232, 65.91)(26.75540765391015, 65.109)(26.85524126455907, 66.13)(26.955074875207988, 65.289)(27.054908485856906, 65.381)(27.154742096505824, 65.262)(27.254575707154743, 70.295)(27.35440931780366, 66.06)(27.45424292845258, 65.506)(27.5540765391015, 65.323)(27.653910149750416, 65.781)(27.753743760399338, 66.047)(27.853577371048253, 64.885)(27.953410981697175, 69.289)(28.05324459234609, 75.121)(28.15307820299501, 66.762)(28.252911813643927, 65.73)(28.35274542429285, 65.158)(28.452579034941763, 66.036)(28.552412645590685, 66.955)(28.6522462562396, 65.193)(28.752079866888522, 65.094)(28.851913477537437, 65.61)(28.95174708818636, 65.771)(29.051580698835274, 65.758)(29.151414309484196, 65.756)(29.25124792013311, 65.102)(29.351081530782032, 65.444)(29.450915141430947, 65.629)(29.55074875207987, 64.362)(29.650582362728784, 66.926)(29.750415973377706, 65.663)(29.85024958402662, 64.762)(29.950083194675543, 65.335)(30.049916805324457, 66.024)(30.14975041597338, 73.308)(30.249584026622294, 66.416)(30.349417637271216, 65.695)(30.44925124792013, 65.31)(30.549084858569053, 64.637)(30.648918469217968, 65.8)(30.74875207986689, 64.641)(30.848585690515804, 65.727)(30.948419301164726, 65.035)(31.04825291181364, 64.825)(31.148086522462563, 65.419)(31.247920133111478, 70.383)(31.3477537437604, 64.409)(31.447587354409315, 65.918)(31.547420965058237, 64.961)(31.64725457570715, 64.595)(31.747088186356073, 65.664)(31.84692179700499, 66.152)(31.94675540765391, 65.241)(32.046589018302825, 64.371)(32.14642262895175, 67.91)(32.24625623960066, 72.639)(32.346089850249584, 67.547)(32.4459234608985, 67.598)(32.54575707154742, 67.397)(32.645590682196335, 64.893)(32.74542429284526, 65.604)(32.84525790349417, 66.026)(32.9450915141431, 64.806)(33.044925124792016, 65.287)(33.14475873544094, 65.577)(33.24459234608985, 64.868)(33.344425956738775, 64.889)(33.44425956738769, 66.285)(33.54409317803661, 65.107)(33.643926788685526, 66.609)(33.74376039933445, 66.431)(33.84359400998336, 67.318)(33.943427620632285, 65.406)(34.0432612312812, 65.314)(34.14309484193012, 65.336)(34.24292845257904, 66.259)(34.34276206322796, 65.121)(34.44259567387687, 65.889)(34.542429284525795, 64.551)(34.64226289517471, 65.331)(34.74209650582363, 65.652)(34.84193011647255, 65.555)(34.94176372712147, 65.468)(35.04159733777038, 64.784)(35.141430948419305, 65.873)(35.24126455906822, 65.755)(35.34109816971714, 64.856)(35.44093178036606, 65.942)(35.54076539101498, 65.643)(35.640599001663894, 65.007)(35.740432612312816, 64.377)(35.84026622296173, 65.079)(35.94009983361065, 65.376)(36.03993344425957, 65.076)(36.13976705490849, 73.63)(36.239600665557404, 66.237)(36.339434276206326, 64.382)(36.43926788685524, 65.6)(36.53910149750416, 64.697)(36.63893510815308, 65.839)(36.738768718802, 64.493)(36.838602329450914, 64.592)(36.938435940099836, 64.731)(37.03826955074875, 65.115)(37.13810316139767, 64.643)(37.23793677204659, 65.573)(37.33777038269551, 64.933)(37.437603993344425, 65.265)(37.53743760399335, 65.548)(37.63727121464226, 64.964)(37.73710482529118, 65.5)(37.8369384359401, 65.351)(37.93677204658902, 66.453)(38.036605657237935, 65.152)(38.13643926788686, 65.202)(38.23627287853577, 65.639)(38.336106489184694, 66.806)(38.43594009983361, 65.106)(38.53577371048253, 64.594)(38.635607321131445, 64.902)(38.73544093178037, 66.051)(38.83527454242928, 65.03)(38.935108153078204, 65.472)(39.03494176372712, 65.869)(39.13477537437604, 64.484)(39.234608985024956, 65.769)(39.33444259567388, 65.564)(39.43427620632279, 65.424)(39.534109816971714, 65.14)(39.63394342762063, 65.273)(39.73377703826955, 64.882)(39.833610648918466, 65.843)(39.93344425956739, 72.072)(40.0332778702163, 70.01)(40.133111480865225, 72.592)(40.232945091514146, 65.107)(40.33277870216307, 65.568)(40.43261231281198, 65.152)(40.532445923460905, 64.965)(40.63227953410982, 65.391)(40.73211314475874, 65.491)(40.83194675540766, 65.812)(40.93178036605658, 65.318)(41.03161397670549, 65.537)(41.131447587354415, 64.926)(41.23128119800333, 65.143)(41.33111480865225, 66.699)(41.43094841930117, 65.295)(41.53078202995009, 65.597)(41.630615640599004, 65.839)(41.730449251247926, 65.346)(41.83028286189684, 68.995)(41.93011647254576, 71.324)(42.02995008319468, 69.611)(42.1297836938436, 74.126)(42.229617304492514, 69.569)(42.329450915141436, 77.405)(42.42928452579035, 76.715)(42.52911813643927, 75.312)(42.62895174708819, 76.838)(42.72878535773711, 76.111)(42.828618968386024, 76.292)(42.928452579034946, 74.021)(43.02828618968386, 65.278)(43.12811980033278, 64.96)(43.2279534109817, 75.142)(43.32778702163062, 73.635)(43.427620632279535, 73.318)(43.52745424292846, 73.313)(43.62728785357737, 66.745)(43.72712146422629, 65.918)(43.82695507487521, 80.408)(43.92678868552413, 84.394)(44.026622296173045, 80.819)(44.12645590682197, 65.09)(44.22628951747088, 65.418)(44.326123128119804, 65.588)(44.42595673876872, 66.098)(44.52579034941764, 64.822)(44.625623960066555, 65.187)(44.72545757071548, 65.829)(44.82529118136439, 65.902)(44.925124792013314, 64.577)(45.02495840266223, 65.733)(45.12479201331115, 66.6)(45.224625623960065, 65.048)(45.32445923460899, 65.003)(45.4242928452579, 66.142)(45.524126455906824, 65.064)(45.62396006655574, 65.26)(45.72379367720466, 65.13)(45.823627287853576, 65.567)(45.9234608985025, 65.931)(46.02329450915141, 66.586)(46.123128119800334, 65.78)(46.22296173044925, 66.684)(46.32279534109817, 65.857)(46.422628951747086, 65.203)(46.52246256239601, 65.441)(46.62229617304492, 65.381)(46.722129783693845, 65.853)(46.82196339434276, 65.734)(46.92179700499168, 66.404)(47.021630615640596, 65.076)(47.12146422628952, 65.462)(47.22129783693843, 65.09)(47.321131447587355, 65.777)(47.42096505823627, 66.529)(47.5207986688852, 67.142)(47.620632279534114, 65.738)(47.720465890183036, 64.818)(47.82029950083195, 66.133)(47.92013311148087, 65.028)(48.01996672212979, 66.068)(48.11980033277871, 66.407)(48.219633943427624, 66.636)(48.319467554076546, 66.57)(48.41930116472546, 65.196)(48.51913477537438, 64.379)(48.6189683860233, 66.74)(48.71880199667222, 66.555)(48.818635607321134, 64.985)(48.918469217970056, 65.597)(49.01830282861897, 65.627)(49.11813643926789, 65.286)(49.21797004991681, 64.876)(49.31780366056573, 66.625)(49.417637271214645, 65.028)(49.51747088186357, 66.59)(49.61730449251248, 94.633)(49.7171381031614, 65.663)(49.81697171381032, 66.109)(49.91680532445924, 66.445)(50.016638935108155, 65.774)(50.11647254575708, 93.974)(50.21630615640599, 87.279)(50.31613976705491, 83.202)(50.41597337770383, 78.545)(50.51580698835275, 76.827)(50.615640599001665, 78.235)(50.71547420965059, 78.397)(50.8153078202995, 77.257)(50.915141430948424, 76.288)(51.01497504159734, 77.134)(51.11480865224626, 84.077)(51.214642262895175, 79.013)(51.3144758735441, 77.006)(51.41430948419301, 65.926)(51.514143094841934, 66.294)(51.61397670549085, 77.076)(51.71381031613977, 79.035)(51.813643926788686, 75.929)(51.91347753743761, 66.804)(52.01331114808652, 65.003)(52.113144758735444, 67.742)(52.21297836938436, 66.737)(52.31281198003328, 65.068)(52.412645590682196, 66.065)(52.51247920133112, 67.041)(52.61231281198003, 66.478)(52.712146422628955, 67.9)(52.81198003327787, 73.461)(52.91181364392679, 72.783)(53.011647254575706, 81.055)(53.11148086522463, 76.695)(53.21131447587354, 82.771)(53.311148086522465, 77.779)(53.41098169717138, 76.588)(53.5108153078203, 65.599)(53.61064891846922, 65.558)(53.71048252911814, 66.362)(53.81031613976705, 71.432)(53.910149750415975, 66.917)(54.00998336106489, 65.516)(54.10981697171381, 65.528)(54.20965058236273, 65.638)(54.30948419301165, 81.784)(54.409317803660564, 103.717)(54.509151414309486, 88.646)(54.6089850249584, 97.709)(54.70881863560732, 85.685)(54.808652246256244, 96.317)(54.90848585690516, 94.559)(55.00831946755408, 95.33)(55.108153078203, 86.93)(55.20798668885192, 82.477)(55.30782029950083, 64.857)(55.407653910149754, 65.845)(55.507487520798676, 65.541)(55.60732113144759, 65.843)(55.707154742096506, 66.413)(55.80698835274543, 65.362)(55.90682196339435, 68.609)(56.006655574043265, 66.719)(56.10648918469218, 74.15)(56.2063227953411, 87.16)(56.30615640599002, 80.885)(56.40599001663894, 71.15)(56.50582362728785, 65.665)(56.605657237936775, 65.228)(56.7054908485857, 65.537)(56.80532445923461, 65.472)(56.90515806988353, 65.404)(57.00499168053245, 65.698)(57.10482529118137, 68.112)(57.204658901830285, 65.665)(57.3044925124792, 65.952)(57.40432612312812, 65.559)(57.504159733777044, 66.042)(57.60399334442596, 66.31)(57.703826955074874, 67.159)(57.803660565723796, 66.156)(57.90349417637272, 66.293)(58.00332778702163, 67.315)(58.10316139767055, 74.084)(58.20299500831947, 71.271)(58.30282861896839, 65.491)(58.402662229617306, 65.041)(58.50249584026622, 68.446)(58.60232945091514, 65.569)(58.702163061564065, 65.795)(58.80199667221298, 65.987)(58.901830282861894, 65.403)(59.001663893510816, 73.684)(59.10149750415974, 66.47)(59.20133111480865, 65.372)(59.30116472545757, 65.099)(59.40099833610649, 66.564)(59.50083194675541, 66.437)(59.60066555740433, 66.01)(59.70049916805324, 65.975)(59.80033277870216, 66.195)(59.900166389351085, 61.147)(60.0, 12.288)
            };
            \addplot[color=blue, mark=none,name path=B] coordinates { %% MIN value
            (0.0, 36.633)(0.09983361064891848, 38.757)(0.19966722129783696, 50.598)(0.2995008319467554, 36.788)(0.3993344425956739, 48.406)(0.49916805324459235, 42.813)(0.5990016638935108, 44.646)(0.6988352745424293, 45.203)(0.7986688851913478, 36.495)(0.8985024958402663, 43.428)(0.9983361064891847, 32.402)(1.0981697171381033, 47.113)(1.1980033277870217, 42.556)(1.2978369384359403, 38.387)(1.3976705490848587, 38.816)(1.4975041597337773, 34.681)(1.5973377703826956, 36.389)(1.697171381031614, 37.589)(1.7970049916805326, 36.634)(1.896838602329451, 40.764)(1.9966722129783694, 36.045)(2.096505823627288, 36.502)(2.1963394342762066, 37.435)(2.296173044925125, 42.823)(2.3960066555740434, 38.211)(2.4958402662229617, 39.23)(2.5956738768718806, 41.053)(2.695507487520799, 40.945)(2.7953410981697173, 43.922)(2.8951747088186357, 41.399)(2.9950083194675545, 38.624)(3.094841930116473, 36.546)(3.1946755407653913, 38.221)(3.2945091514143097, 37.724)(3.394342762063228, 40.425)(3.494176372712147, 42.847)(3.5940099833610653, 43.027)(3.6938435940099836, 43.271)(3.793677204658902, 37.142)(3.8935108153078204, 45.07)(3.9933444259567388, 38.237)(4.093178036605657, 36.064)(4.193011647254576, 43.657)(4.292845257903495, 44.05)(4.392678868552413, 44.447)(4.492512479201332, 44.402)(4.59234608985025, 47.174)(4.692179700499168, 44.971)(4.792013311148087, 38.858)(4.891846921797005, 45.85)(4.9916805324459235, 46.545)(5.091514143094843, 41.44)(5.191347753743761, 44.459)(5.2911813643926795, 46.777)(5.391014975041598, 40.617)(5.490848585690516, 46.724)(5.590682196339435, 45.288)(5.690515806988353, 48.532)(5.790349417637271, 46.298)(5.89018302828619, 43.07)(5.990016638935109, 46.907)(6.089850249584027, 38.824)(6.189683860232946, 48.571)(6.289517470881864, 46.129)(6.389351081530783, 40.208)(6.489184692179701, 39.112)(6.589018302828619, 40.696)(6.688851913477538, 46.581)(6.788685524126456, 47.804)(6.888519134775375, 41.608)(6.988352745424294, 47.076)(7.088186356073212, 42.607)(7.1880199667221305, 44.614)(7.287853577371049, 39.024)(7.387687188019967, 36.049)(7.487520798668886, 47.392)(7.587354409317804, 41.424)(7.687188019966722, 39.295)(7.787021630615641, 41.16)(7.886855241264559, 46.772)(7.9866888519134775, 44.696)(8.086522462562396, 45.724)(8.186356073211314, 45.177)(8.286189683860234, 45.669)(8.386023294509153, 41.992)(8.485856905158071, 39.771)(8.58569051580699, 44.005)(8.685524126455908, 45.324)(8.785357737104826, 35.476)(8.885191347753745, 37.674)(8.985024958402663, 43.987)(9.084858569051582, 48.857)(9.1846921797005, 43.082)(9.284525790349418, 44.998)(9.384359400998337, 43.804)(9.484193011647255, 35.943)(9.584026622296173, 47.704)(9.683860232945092, 44.873)(9.78369384359401, 43.828)(9.883527454242929, 45.896)(9.983361064891847, 46.866)(10.083194675540767, 43.549)(10.183028286189685, 45.59)(10.282861896838604, 38.51)(10.382695507487522, 46.5)(10.48252911813644, 44.403)(10.582362728785359, 41.638)(10.682196339434277, 48.522)(10.782029950083196, 46.63)(10.881863560732114, 45.266)(10.981697171381033, 45.708)(11.081530782029951, 39.697)(11.18136439267887, 39.023)(11.281198003327788, 40.717)(11.381031613976706, 48.068)(11.480865224625624, 48.836)(11.580698835274543, 42.369)(11.680532445923461, 45.136)(11.78036605657238, 45.841)(11.8801996672213, 45.368)(11.980033277870218, 37.49)(12.079866888519136, 49.73)(12.179700499168055, 48.597)(12.279534109816973, 43.254)(12.379367720465892, 44.908)(12.47920133111481, 43.315)(12.579034941763728, 41.961)(12.678868552412647, 50.821)(12.778702163061565, 44.853)(12.878535773710484, 45.051)(12.978369384359402, 45.547)(13.07820299500832, 42.742)(13.178036605657239, 38.504)(13.277870216306157, 42.864)(13.377703826955075, 47.507)(13.477537437603994, 39.228)(13.577371048252912, 43.527)(13.67720465890183, 43.327)(13.77703826955075, 41.55)(13.876871880199669, 43.181)(13.976705490848587, 42.032)(14.076539101497506, 45.804)(14.176372712146424, 45.183)(14.276206322795343, 48.623)(14.376039933444261, 49.192)(14.47587354409318, 43.307)(14.575707154742098, 43.516)(14.675540765391016, 43.789)(14.775374376039935, 46.052)(14.875207986688853, 42.339)(14.975041597337771, 41.734)(15.07487520798669, 43.78)(15.174708818635608, 48.69)(15.274542429284526, 42.942)(15.374376039933445, 47.406)(15.474209650582363, 42.432)(15.574043261231282, 45.846)(15.6738768718802, 44.84)(15.773710482529118, 46.657)(15.873544093178037, 35.487)(15.973377703826955, 43.65)(16.073211314475873, 45.728)(16.173044925124792, 40.362)(16.27287853577371, 42.087)(16.37271214642263, 42.132)(16.47254575707155, 43.038)(16.57237936772047, 35.943)(16.672212978369387, 47.061)(16.772046589018306, 44.056)(16.871880199667224, 46.777)(16.971713810316142, 47.879)(17.07154742096506, 45.411)(17.17138103161398, 45.272)(17.271214642262898, 44.186)(17.371048252911816, 41.588)(17.470881863560734, 49.01)(17.570715474209653, 44.103)(17.67054908485857, 45.774)(17.77038269550749, 47.24)(17.870216306156408, 40.773)(17.970049916805326, 40.72)(18.069883527454245, 41.444)(18.169717138103163, 41.547)(18.26955074875208, 47.442)(18.369384359401, 40.317)(18.469217970049918, 49.129)(18.569051580698837, 45.793)(18.668885191347755, 47.49)(18.768718801996673, 46.582)(18.86855241264559, 46.107)(18.96838602329451, 45.086)(19.06821963394343, 39.235)(19.168053244592347, 45.841)(19.267886855241265, 48.711)(19.367720465890184, 44.939)(19.467554076539102, 42.889)(19.56738768718802, 46.132)(19.66722129783694, 43.063)(19.767054908485857, 47.046)(19.866888519134775, 45.405)(19.966722129783694, 43.805)(20.066555740432612, 46.063)(20.166389351081534, 45.87)(20.266222961730453, 48.465)(20.36605657237937, 43.754)(20.46589018302829, 45.384)(20.565723793677208, 45.407)(20.665557404326126, 42.135)(20.765391014975044, 46.666)(20.865224625623963, 44.47)(20.96505823627288, 44.993)(21.0648918469218, 46.812)(21.164725457570718, 46.264)(21.264559068219636, 44.297)(21.364392678868555, 47.012)(21.464226289517473, 36.779)(21.56405990016639, 49.037)(21.66389351081531, 38.715)(21.76372712146423, 47.181)(21.863560732113147, 37.35)(21.963394342762065, 45.654)(22.063227953410983, 43.054)(22.163061564059902, 49.535)(22.26289517470882, 35.159)(22.36272878535774, 45.456)(22.462562396006657, 43.212)(22.562396006655575, 48.78)(22.662229617304494, 47.369)(22.762063227953412, 43.629)(22.86189683860233, 45.469)(22.96173044925125, 41.756)(23.061564059900167, 44.546)(23.161397670549086, 41.106)(23.261231281198004, 44.489)(23.361064891846922, 36.034)(23.46089850249584, 47.346)(23.56073211314476, 46.788)(23.660565723793678, 48.903)(23.7603993344426, 46.892)(23.860232945091518, 42.439)(23.960066555740436, 47.563)(24.059900166389355, 38.95)(24.159733777038273, 41.597)(24.25956738768719, 48.63)(24.35940099833611, 40.185)(24.459234608985028, 46.416)(24.559068219633946, 36.175)(24.658901830282865, 45.19)(24.758735440931783, 40.667)(24.8585690515807, 47.936)(24.95840266222962, 48.577)(25.05823627287854, 40.236)(25.158069883527457, 46.01)(25.257903494176375, 44.439)(25.357737104825294, 46.43)(25.457570715474212, 48.435)(25.55740432612313, 34.94)(25.65723793677205, 50.144)(25.757071547420967, 43.037)(25.856905158069885, 43.462)(25.956738768718804, 46.394)(26.056572379367722, 47.254)(26.15640599001664, 47.398)(26.25623960066556, 47.769)(26.356073211314477, 42.729)(26.455906821963396, 48.115)(26.555740432612314, 46.042)(26.655574043261232, 45.813)(26.75540765391015, 39.125)(26.85524126455907, 47.091)(26.955074875207988, 44.475)(27.054908485856906, 42.665)(27.154742096505824, 41.262)(27.254575707154743, 46.757)(27.35440931780366, 49.687)(27.45424292845258, 46.395)(27.5540765391015, 44.864)(27.653910149750416, 47.886)(27.753743760399338, 47.425)(27.853577371048253, 46.487)(27.953410981697175, 48.296)(28.05324459234609, 43.164)(28.15307820299501, 45.666)(28.252911813643927, 46.29)(28.35274542429285, 45.578)(28.452579034941763, 48.139)(28.552412645590685, 47.467)(28.6522462562396, 49.556)(28.752079866888522, 44.511)(28.851913477537437, 45.02)(28.95174708818636, 43.543)(29.051580698835274, 46.529)(29.151414309484196, 39.943)(29.25124792013311, 47.981)(29.351081530782032, 46.438)(29.450915141430947, 39.289)(29.55074875207987, 49.125)(29.650582362728784, 47.059)(29.750415973377706, 46.429)(29.85024958402662, 49.001)(29.950083194675543, 44.724)(30.049916805324457, 38.832)(30.14975041597338, 47.596)(30.249584026622294, 43.205)(30.349417637271216, 37.341)(30.44925124792013, 46.331)(30.549084858569053, 35.604)(30.648918469217968, 43.103)(30.74875207986689, 40.069)(30.848585690515804, 40.554)(30.948419301164726, 38.654)(31.04825291181364, 44.908)(31.148086522462563, 33.246)(31.247920133111478, 48.145)(31.3477537437604, 39.771)(31.447587354409315, 45.278)(31.547420965058237, 48.681)(31.64725457570715, 49.887)(31.747088186356073, 48.816)(31.84692179700499, 36.645)(31.94675540765391, 39.319)(32.046589018302825, 42.674)(32.14642262895175, 48.324)(32.24625623960066, 39.7)(32.346089850249584, 41.68)(32.4459234608985, 46.147)(32.54575707154742, 46.569)(32.645590682196335, 43.536)(32.74542429284526, 43.003)(32.84525790349417, 42.159)(32.9450915141431, 47.777)(33.044925124792016, 41.154)(33.14475873544094, 43.081)(33.24459234608985, 39.359)(33.344425956738775, 46.079)(33.44425956738769, 37.719)(33.54409317803661, 42.928)(33.643926788685526, 46.323)(33.74376039933445, 42.417)(33.84359400998336, 48.5)(33.943427620632285, 44.465)(34.0432612312812, 46.021)(34.14309484193012, 47.351)(34.24292845257904, 42.232)(34.34276206322796, 44.588)(34.44259567387687, 50.813)(34.542429284525795, 43.271)(34.64226289517471, 38.699)(34.74209650582363, 38.718)(34.84193011647255, 45.829)(34.94176372712147, 50.014)(35.04159733777038, 40.263)(35.141430948419305, 49.776)(35.24126455906822, 49.418)(35.34109816971714, 49.4)(35.44093178036606, 46.916)(35.54076539101498, 48.959)(35.640599001663894, 39.002)(35.740432612312816, 38.371)(35.84026622296173, 46.362)(35.94009983361065, 45.759)(36.03993344425957, 46.051)(36.13976705490849, 48.535)(36.239600665557404, 48.482)(36.339434276206326, 45.885)(36.43926788685524, 43.602)(36.53910149750416, 43.16)(36.63893510815308, 46.437)(36.738768718802, 47.33)(36.838602329450914, 43.313)(36.938435940099836, 36.912)(37.03826955074875, 45.98)(37.13810316139767, 48.565)(37.23793677204659, 46.461)(37.33777038269551, 49.449)(37.437603993344425, 40.859)(37.53743760399335, 46.355)(37.63727121464226, 47.875)(37.73710482529118, 41.118)(37.8369384359401, 46.041)(37.93677204658902, 49.457)(38.036605657237935, 48.925)(38.13643926788686, 50.583)(38.23627287853577, 46.831)(38.336106489184694, 34.631)(38.43594009983361, 39.492)(38.53577371048253, 48.3)(38.635607321131445, 43.71)(38.73544093178037, 45.778)(38.83527454242928, 48.681)(38.935108153078204, 49.074)(39.03494176372712, 43.492)(39.13477537437604, 43.508)(39.234608985024956, 49.537)(39.33444259567388, 50.983)(39.43427620632279, 37.276)(39.534109816971714, 48.946)(39.63394342762063, 41.39)(39.73377703826955, 45.439)(39.833610648918466, 48.112)(39.93344425956739, 46.65)(40.0332778702163, 48.942)(40.133111480865225, 48.253)(40.232945091514146, 46.78)(40.33277870216307, 44.977)(40.43261231281198, 46.694)(40.532445923460905, 35.3)(40.63227953410982, 47.113)(40.73211314475874, 48.597)(40.83194675540766, 47.052)(40.93178036605658, 51.201)(41.03161397670549, 48.4)(41.131447587354415, 47.359)(41.23128119800333, 42.142)(41.33111480865225, 43.182)(41.43094841930117, 48.763)(41.53078202995009, 46.904)(41.630615640599004, 46.019)(41.730449251247926, 45.311)(41.83028286189684, 49.778)(41.93011647254576, 44.62)(42.02995008319468, 45.951)(42.1297836938436, 47.537)(42.229617304492514, 45.778)(42.329450915141436, 45.263)(42.42928452579035, 48.228)(42.52911813643927, 43.795)(42.62895174708819, 47.668)(42.72878535773711, 47.667)(42.828618968386024, 45.165)(42.928452579034946, 48.127)(43.02828618968386, 45.569)(43.12811980033278, 49.658)(43.2279534109817, 50.167)(43.32778702163062, 49.126)(43.427620632279535, 49.271)(43.52745424292846, 38.287)(43.62728785357737, 43.685)(43.72712146422629, 46.665)(43.82695507487521, 41.243)(43.92678868552413, 47.3)(44.026622296173045, 48.525)(44.12645590682197, 47.118)(44.22628951747088, 45.361)(44.326123128119804, 42.088)(44.42595673876872, 44.498)(44.52579034941764, 45.397)(44.625623960066555, 49.879)(44.72545757071548, 45.985)(44.82529118136439, 48.38)(44.925124792013314, 49.853)(45.02495840266223, 49.72)(45.12479201331115, 45.451)(45.224625623960065, 44.742)(45.32445923460899, 49.692)(45.4242928452579, 46.573)(45.524126455906824, 47.376)(45.62396006655574, 49.916)(45.72379367720466, 51.38)(45.823627287853576, 44.865)(45.9234608985025, 47.298)(46.02329450915141, 46.216)(46.123128119800334, 49.752)(46.22296173044925, 48.513)(46.32279534109817, 48.181)(46.422628951747086, 49.616)(46.52246256239601, 46.475)(46.62229617304492, 47.012)(46.722129783693845, 46.987)(46.82196339434276, 45.368)(46.92179700499168, 47.823)(47.021630615640596, 50.505)(47.12146422628952, 49.986)(47.22129783693843, 46.22)(47.321131447587355, 49.681)(47.42096505823627, 48.05)(47.5207986688852, 51.063)(47.620632279534114, 39.223)(47.720465890183036, 49.378)(47.82029950083195, 44.562)(47.92013311148087, 50.643)(48.01996672212979, 48.083)(48.11980033277871, 47.246)(48.219633943427624, 49.676)(48.319467554076546, 44.089)(48.41930116472546, 49.723)(48.51913477537438, 48.685)(48.6189683860233, 50.164)(48.71880199667222, 43.797)(48.818635607321134, 49.052)(48.918469217970056, 50.535)(49.01830282861897, 48.569)(49.11813643926789, 48.535)(49.21797004991681, 49.35)(49.31780366056573, 45.343)(49.417637271214645, 43.771)(49.51747088186357, 49.634)(49.61730449251248, 48.953)(49.7171381031614, 48.753)(49.81697171381032, 48.993)(49.91680532445924, 46.132)(50.016638935108155, 48.887)(50.11647254575708, 49.803)(50.21630615640599, 49.234)(50.31613976705491, 45.231)(50.41597337770383, 47.303)(50.51580698835275, 45.696)(50.615640599001665, 49.572)(50.71547420965059, 50.02)(50.8153078202995, 49.053)(50.915141430948424, 45.558)(51.01497504159734, 48.91)(51.11480865224626, 49.17)(51.214642262895175, 48.003)(51.3144758735441, 48.37)(51.41430948419301, 42.737)(51.514143094841934, 49.805)(51.61397670549085, 45.193)(51.71381031613977, 50.05)(51.813643926788686, 39.535)(51.91347753743761, 49.002)(52.01331114808652, 49.368)(52.113144758735444, 47.567)(52.21297836938436, 42.404)(52.31281198003328, 49.964)(52.412645590682196, 42.471)(52.51247920133112, 52.14)(52.61231281198003, 29.535)(52.712146422628955, 42.654)(52.81198003327787, 51.055)(52.91181364392679, 39.326)(53.011647254575706, 48.432)(53.11148086522463, 49.686)(53.21131447587354, 46.345)(53.311148086522465, 48.565)(53.41098169717138, 44.277)(53.5108153078203, 48.641)(53.61064891846922, 50.827)(53.71048252911814, 50.242)(53.81031613976705, 46.435)(53.910149750415975, 47.777)(54.00998336106489, 47.16)(54.10981697171381, 48.174)(54.20965058236273, 49.15)(54.30948419301165, 51.909)(54.409317803660564, 53.352)(54.509151414309486, 48.854)(54.6089850249584, 49.371)(54.70881863560732, 48.98)(54.808652246256244, 46.271)(54.90848585690516, 47.88)(55.00831946755408, 48.909)(55.108153078203, 46.628)(55.20798668885192, 49.222)(55.30782029950083, 51.126)(55.407653910149754, 49.97)(55.507487520798676, 47.37)(55.60732113144759, 47.286)(55.707154742096506, 38.672)(55.80698835274543, 50.463)(55.90682196339435, 42.791)(56.006655574043265, 45.449)(56.10648918469218, 50.723)(56.2063227953411, 50.015)(56.30615640599002, 47.381)(56.40599001663894, 50.146)(56.50582362728785, 51.295)(56.605657237936775, 52.172)(56.7054908485857, 47.664)(56.80532445923461, 49.282)(56.90515806988353, 49.23)(57.00499168053245, 51.388)(57.10482529118137, 50.001)(57.204658901830285, 49.997)(57.3044925124792, 46.538)(57.40432612312812, 51.632)(57.504159733777044, 50.949)(57.60399334442596, 48.809)(57.703826955074874, 49.471)(57.803660565723796, 47.665)(57.90349417637272, 45.413)(58.00332778702163, 47.347)(58.10316139767055, 50.263)(58.20299500831947, 47.673)(58.30282861896839, 49.229)(58.402662229617306, 48.349)(58.50249584026622, 48.917)(58.60232945091514, 49.648)(58.702163061564065, 50.771)(58.80199667221298, 48.023)(58.901830282861894, 50.421)(59.001663893510816, 50.74)(59.10149750415974, 50.592)(59.20133111480865, 53.856)(59.30116472545757, 48.806)(59.40099833610649, 48.86)(59.50083194675541, 46.815)(59.60066555740433, 42.165)(59.70049916805324, 49.345)(59.80033277870216, 38.546)(59.900166389351085, 23.046)(60.0, 12.288)
            };
            \addplot [pattern=north east lines,pattern color=red] 
            fill between [
                of=A and B,soft clip={domain=0:800},
            ];
            \end{axis}
    \end{tikzpicture}
    \caption{Measuring instrument: Intel Power Gadget}      
\end{subfigure}
\begin{subfigure}[b]{0.49\linewidth}
    \begin{tikzpicture}
        \pgfplotsset{%
        width=1\linewidth,
        % height=1\textheight
        }
        \begin{axis}[ymax=120,
            xlabel={Time (Seconds)},
            ylabel={Energy Consumption (Joules)},
            ]
            \addplot[color=blue, mark=none,] coordinates { %% AVG value
            (0.0, 32.27081175)(0.5042016806722689, 60.42327937499997)(1.0084033613445378, 60.4024382333333)(1.5126050420168067, 60.302899358333335)(2.0168067226890756, 60.55195880833331)(2.5210084033613445, 60.0532363166667)(3.0252100840336134, 60.32544241666666)(3.5294117647058822, 60.41187110833336)(4.033613445378151, 60.587539716666676)(4.53781512605042, 60.57326431666664)(5.042016806722689, 60.323387433333345)(5.546218487394958, 60.524838399999986)(6.050420168067227, 60.41082002499999)(6.554621848739496, 60.510211808333345)(7.0588235294117645, 60.35104204166667)(7.563025210084033, 60.06417598333334)(8.067226890756302, 60.68459924166668)(8.571428571428571, 60.58961446666665)(9.07563025210084, 60.69294124166666)(9.579831932773109, 60.56367823333335)(10.084033613445378, 60.47124108333334)(10.588235294117647, 60.23169019999998)(11.092436974789916, 59.97079601666662)(11.596638655462183, 59.938885658333334)(12.100840336134453, 60.47499790000001)(12.605042016806722, 59.70728824166666)(13.109243697478991, 59.773363649999965)(13.61344537815126, 59.6431445916667)(14.117647058823529, 59.796321325000015)(14.621848739495798, 59.78501387500001)(15.126050420168067, 60.17855269166666)(15.630252100840334, 60.12668800833335)(16.134453781512605, 60.76536466666664)(16.638655462184875, 60.68725993333333)(17.142857142857142, 60.532908799999966)(17.647058823529413, 60.424793233333354)(18.15126050420168, 60.471258608333336)(18.65546218487395, 60.4229636)(19.159663865546218, 60.28576485833333)(19.66386554621849, 60.4370787833333)(20.168067226890756, 60.84662073333331)(20.672268907563026, 60.57808199999999)(21.176470588235293, 60.325025025000016)(21.680672268907564, 60.620033275000004)(22.18487394957983, 60.34069542499999)(22.6890756302521, 60.14616618333337)(23.193277310924366, 60.263227275)(23.697478991596636, 60.03637134999999)(24.201680672268907, 60.18956634166666)(24.705882352941174, 60.14728498333334)(25.210084033613445, 60.04252120000001)(25.71428571428571, 60.17314940000002)(26.218487394957982, 59.81255598333332)(26.72268907563025, 59.66486713333333)(27.22689075630252, 59.80764945833334)(27.731092436974787, 59.63937795000001)(28.235294117647058, 60.10205771666665)(28.739495798319325, 59.65502909999999)(29.243697478991596, 59.64949170833333)(29.747899159663863, 59.40919734999998)(30.252100840336134, 59.432290724999994)(30.7563025210084, 59.41029688333332)(31.260504201680668, 59.242329183333304)(31.764705882352942, 59.328384608333344)(32.26890756302521, 59.53051000833336)(32.773109243697476, 59.62530433333332)(33.27731092436975, 59.917351291666655)(33.78151260504202, 59.97362810000003)(34.285714285714285, 60.14613748333333)(34.78991596638655, 60.16017080833332)(35.294117647058826, 60.32032638333337)(35.79831932773109, 60.46200551666667)(36.30252100840336, 60.45159421666663)(36.80672268907563, 60.37057385)(37.3109243697479, 60.55694072500001)(37.81512605042017, 60.601055825)(38.319327731092436, 60.61906021666667)(38.8235294117647, 60.249741416666694)(39.32773109243698, 60.42147497499995)(39.831932773109244, 60.712513124999994)(40.33613445378151, 60.707719483333335)(40.840336134453786, 60.61705624166666)(41.34453781512605, 60.71091888333332)(41.84873949579832, 60.72775523333336)(42.35294117647059, 60.83097946666668)(42.85714285714286, 60.63775451666668)(43.36134453781513, 60.82727932499998)(43.865546218487395, 60.754207766666646)(44.36974789915966, 60.87249010000001)(44.87394957983193, 60.84420319166666)(45.3781512605042, 60.66147667500001)(45.882352941176464, 60.7155622)(46.38655462184873, 61.13859094166669)(46.890756302521005, 60.69333435)(47.39495798319327, 60.81983091666667)(47.89915966386554, 60.932054091666686)(48.403361344537814, 61.030098475)(48.90756302521008, 61.021193050000015)(49.41176470588235, 60.992824466666676)(49.915966386554615, 61.13895932500001)(50.42016806722689, 61.242917141666666)(50.924369747899156, 60.979892158333314)(51.42857142857142, 61.06577360833332)(51.93277310924369, 61.363412683333344)(52.436974789915965, 61.54533122500003)(52.94117647058823, 61.42102846666668)(53.4453781512605, 61.68464905000001)(53.949579831932766, 61.79458894166667)(54.45378151260504, 61.77681088333333)(54.95798319327731, 61.894677141666655)(55.462184873949575, 61.87809869166666)(55.96638655462184, 62.02127138333332)(56.470588235294116, 62.21555184999999)(56.97478991596638, 62.145315283333325)(57.47899159663865, 62.415514891666675)(57.983193277310924, 62.08855897500002)(58.48739495798319, 62.296316566666675)(58.99159663865546, 62.163398008333346)(59.495798319327726, 62.24322587499999)(60.0, 56.52384533333332)
            };
            \addplot[color=blue, mark=none,name path=A] coordinates { %% MAX value
            (0.0, 52.686207)(0.5042016806722689, 72.501175)(1.0084033613445378, 64.24743)(1.5126050420168067, 64.57684)(2.0168067226890756, 64.104065)(2.5210084033613445, 69.482704)(3.0252100840336134, 64.20505)(3.5294117647058822, 63.82106)(4.033613445378151, 70.714836)(4.53781512605042, 73.65339)(5.042016806722689, 63.95129)(5.546218487394958, 64.140915)(6.050420168067227, 64.31793)(6.554621848739496, 63.77338)(7.0588235294117645, 67.82698)(7.563025210084033, 69.90889)(8.067226890756302, 64.22802)(8.571428571428571, 63.76927)(9.07563025210084, 91.00151)(9.579831932773109, 95.5939)(10.084033613445378, 87.97327)(10.588235294117647, 78.00645)(11.092436974789916, 80.89309)(11.596638655462183, 74.20021)(12.100840336134453, 73.34035)(12.605042016806722, 64.3058)(13.109243697478991, 64.46119)(13.61344537815126, 65.5405)(14.117647058823529, 63.595516)(14.621848739495798, 64.629654)(15.126050420168067, 64.594154)(15.630252100840334, 64.33024)(16.134453781512605, 63.579002)(16.638655462184875, 64.17427)(17.142857142857142, 64.21977)(17.647058823529413, 63.595997)(18.15126050420168, 63.674503)(18.65546218487395, 64.32061)(19.159663865546218, 64.563385)(19.66386554621849, 63.99271)(20.168067226890756, 67.74007)(20.672268907563026, 64.546425)(21.176470588235293, 64.138435)(21.680672268907564, 64.38781)(22.18487394957983, 64.07181)(22.6890756302521, 71.43621)(23.193277310924366, 71.526665)(23.697478991596636, 63.49447)(24.201680672268907, 63.858425)(24.705882352941174, 64.094154)(25.210084033613445, 64.24575)(25.71428571428571, 63.846485)(26.218487394957982, 63.881836)(26.72268907563025, 64.63618)(27.22689075630252, 63.947784)(27.731092436974787, 64.45911)(28.235294117647058, 64.18758)(28.739495798319325, 63.29134)(29.243697478991596, 64.19078)(29.747899159663863, 63.810226)(30.252100840336134, 66.2599)(30.7563025210084, 63.296127)(31.260504201680668, 63.1976)(31.764705882352942, 65.99598)(32.26890756302521, 64.07178)(32.773109243697476, 63.503185)(33.27731092436975, 63.43873)(33.78151260504202, 63.172848)(34.285714285714285, 63.684345)(34.78991596638655, 63.865993)(35.294117647058826, 63.360916)(35.79831932773109, 63.285664)(36.30252100840336, 63.76567)(36.80672268907563, 63.737595)(37.3109243697479, 63.932915)(37.81512605042017, 63.57512)(38.319327731092436, 63.96546)(38.8235294117647, 63.18456)(39.32773109243698, 63.813377)(39.831932773109244, 63.553288)(40.33613445378151, 63.521564)(40.840336134453786, 63.433525)(41.34453781512605, 63.954044)(41.84873949579832, 63.743885)(42.35294117647059, 64.020325)(42.85714285714286, 63.51649)(43.36134453781513, 64.31367)(43.865546218487395, 64.152985)(44.36974789915966, 67.658516)(44.87394957983193, 63.859512)(45.3781512605042, 63.962128)(45.882352941176464, 64.2111)(46.38655462184873, 63.956154)(46.890756302521005, 63.986282)(47.39495798319327, 64.7071)(47.89915966386554, 64.70991)(48.403361344537814, 64.17331)(48.90756302521008, 64.047066)(49.41176470588235, 64.08344)(49.915966386554615, 63.822342)(50.42016806722689, 63.85846)(50.924369747899156, 64.02338)(51.42857142857142, 64.41406)(51.93277310924369, 63.706997)(52.436974789915965, 64.278404)(52.94117647058823, 64.98663)(53.4453781512605, 79.84759)(53.949579831932766, 66.967316)(54.45378151260504, 64.047264)(54.95798319327731, 63.960697)(55.462184873949575, 64.04592)(55.96638655462184, 63.905434)(56.470588235294116, 64.09438)(56.97478991596638, 64.409996)(57.47899159663865, 64.4703)(57.983193277310924, 64.09872)(58.48739495798319, 64.37706)(58.99159663865546, 64.141884)(59.495798319327726, 64.413086)(60.0, 58.754143)
            };
            \addplot[color=blue, mark=none,name path=B] coordinates { %% MIN value
            (0.0, 18.880487)(0.5042016806722689, 55.909775)(1.0084033613445378, 55.18609)(1.5126050420168067, 55.51059)(2.0168067226890756, 56.507576)(2.5210084033613445, 54.07162)(3.0252100840336134, 53.662815)(3.5294117647058822, 56.19897)(4.033613445378151, 53.987392)(4.53781512605042, 55.40326)(5.042016806722689, 54.550476)(5.546218487394958, 55.70719)(6.050420168067227, 55.569252)(6.554621848739496, 56.268986)(7.0588235294117645, 55.68419)(7.563025210084033, 54.08372)(8.067226890756302, 55.080704)(8.571428571428571, 56.25152)(9.07563025210084, 56.045353)(9.579831932773109, 53.62409)(10.084033613445378, 56.10638)(10.588235294117647, 54.485226)(11.092436974789916, 54.828346)(11.596638655462183, 55.254593)(12.100840336134453, 55.20405)(12.605042016806722, 54.36835)(13.109243697478991, 55.51051)(13.61344537815126, 55.185314)(14.117647058823529, 54.337654)(14.621848739495798, 55.36398)(15.126050420168067, 55.513927)(15.630252100840334, 53.88789)(16.134453781512605, 55.91791)(16.638655462184875, 53.262306)(17.142857142857142, 55.510067)(17.647058823529413, 56.00794)(18.15126050420168, 54.093708)(18.65546218487395, 55.388123)(19.159663865546218, 55.20115)(19.66386554621849, 56.00674)(20.168067226890756, 56.480644)(20.672268907563026, 54.020405)(21.176470588235293, 54.86049)(21.680672268907564, 55.877018)(22.18487394957983, 56.284977)(22.6890756302521, 55.720707)(23.193277310924366, 55.796074)(23.697478991596636, 54.400867)(24.201680672268907, 54.46803)(24.705882352941174, 55.386154)(25.210084033613445, 53.731117)(25.71428571428571, 55.479115)(26.218487394957982, 54.069946)(26.72268907563025, 54.675995)(27.22689075630252, 55.55649)(27.731092436974787, 55.801945)(28.235294117647058, 55.462704)(28.739495798319325, 56.060684)(29.243697478991596, 55.198757)(29.747899159663863, 54.3034)(30.252100840336134, 55.782017)(30.7563025210084, 52.034554)(31.260504201680668, 55.161156)(31.764705882352942, 54.609142)(32.26890756302521, 51.908527)(32.773109243697476, 55.10897)(33.27731092436975, 52.324028)(33.78151260504202, 55.656445)(34.285714285714285, 55.467693)(34.78991596638655, 56.127636)(35.294117647058826, 55.957615)(35.79831932773109, 55.091194)(36.30252100840336, 56.9409)(36.80672268907563, 54.962486)(37.3109243697479, 53.285034)(37.81512605042017, 54.547253)(38.319327731092436, 56.79502)(38.8235294117647, 55.386562)(39.32773109243698, 55.279663)(39.831932773109244, 55.323483)(40.33613445378151, 56.15896)(40.840336134453786, 56.572273)(41.34453781512605, 56.40808)(41.84873949579832, 55.938454)(42.35294117647059, 55.841118)(42.85714285714286, 55.357807)(43.36134453781513, 57.33892)(43.865546218487395, 56.480556)(44.36974789915966, 57.331043)(44.87394957983193, 56.115612)(45.3781512605042, 55.869793)(45.882352941176464, 54.635506)(46.38655462184873, 56.710167)(46.890756302521005, 56.431973)(47.39495798319327, 56.09157)(47.89915966386554, 55.73779)(48.403361344537814, 55.669212)(48.90756302521008, 56.891476)(49.41176470588235, 55.33288)(49.915966386554615, 55.258003)(50.42016806722689, 56.622)(50.924369747899156, 54.357754)(51.42857142857142, 55.17483)(51.93277310924369, 55.443176)(52.436974789915965, 54.034843)(52.94117647058823, 57.455677)(53.4453781512605, 56.710964)(53.949579831932766, 53.742058)(54.45378151260504, 57.317833)(54.95798319327731, 57.932663)(55.462184873949575, 56.929684)(55.96638655462184, 57.941273)(56.470588235294116, 58.775578)(56.97478991596638, 56.877834)(57.47899159663865, 59.22039)(57.983193277310924, 58.27199)(58.48739495798319, 57.75512)(58.99159663865546, 57.37805)(59.495798319327726, 59.67281)(60.0, 53.47679)
            };
            \addplot [pattern=north east lines,pattern color=red] 
            fill between [
                of=A and B,soft clip={domain=0:800},
            ];
            \end{axis}
\end{tikzpicture}
\caption{Measuring instrument: LHM}
\end{subfigure}
\begin{subfigure}[b]{0.49\linewidth}
    \begin{tikzpicture}
        \pgfplotsset{%
        width=1\linewidth,
        % height=1\textheight
        }
\begin{axis}[ymax=120,
    xlabel={Time (Seconds)},
    ylabel={Energy Consumption (Joules)},
    ]
    \addplot[color=blue, mark=none,] coordinates { %% AVG value
    (0.0, 0.0)(0.1001669449081803, 53.5840155)(0.2003338898163606, 55.72025533333331)(0.3005008347245409, 54.29596274999999)(0.4006677796327212, 54.17243349999999)(0.5008347245409015, 54.39149749999999)(0.6010016694490818, 54.52888275000001)(0.7011686143572621, 54.35835600000003)(0.8013355592654424, 54.56871291666665)(0.9015025041736228, 55.19686458333333)(1.001669449081803, 54.6910175)(1.1018363939899833, 54.36337608333337)(1.2020033388981637, 54.62761141666666)(1.3021702838063438, 54.82617908333332)(1.4023372287145242, 54.41876583333336)(1.5025041736227045, 54.600440333333346)(1.6026711185308848, 54.90785433333334)(1.7028380634390652, 54.643892916666644)(1.8030050083472455, 54.61003375000001)(1.9031719532554257, 54.58265916666667)(2.003338898163606, 54.83299958333337)(2.1035058430717863, 54.629946833333314)(2.2036727879799667, 54.77681708333332)(2.303839732888147, 54.86027725000003)(2.4040066777963274, 54.83449991666666)(2.5041736227045073, 54.807533)(2.6043405676126876, 54.86174166666669)(2.704507512520868, 54.675000999999995)(2.8046744574290483, 54.90246258333331)(2.9048414023372287, 54.833904416666634)(3.005008347245409, 54.89016941666669)(3.1051752921535893, 54.70429766666665)(3.2053422370617697, 54.79643425000003)(3.30550918196995, 54.84716033333336)(3.4056761268781304, 54.84124924999996)(3.5058430717863107, 55.099244583333345)(3.606010016694491, 54.849326250000004)(3.7061769616026714, 54.54544366666666)(3.8063439065108513, 54.88116608333333)(3.906510851419032, 54.83131125000002)(4.006677796327212, 54.90941041666665)(4.106844741235393, 54.70105200000002)(4.207011686143573, 55.31852858333332)(4.3071786310517535, 54.582298083333335)(4.407345575959933, 54.89266666666666)(4.507512520868114, 54.696871166666675)(4.607679465776294, 54.90259483333331)(4.707846410684475, 54.962277000000014)(4.808013355592655, 54.708447166666666)(4.908180300500835, 54.99969683333335)(5.0083472454090145, 55.022594583333316)(5.108514190317195, 54.892168000000005)(5.208681135225375, 54.91504583333336)(5.308848080133556, 55.061824833333354)(5.409015025041736, 54.91238566666666)(5.509181969949917, 55.01425391666665)(5.609348914858097, 54.93634741666664)(5.709515859766277, 55.00115075000001)(5.809682804674457, 54.79202466666667)(5.909849749582638, 54.732862)(6.010016694490818, 54.902212583333345)(6.110183639398999, 54.83361558333332)(6.210350584307179, 54.893770916666675)(6.3105175292153595, 55.048208749999986)(6.410684474123539, 54.87101424999999)(6.510851419031719, 55.016236833333295)(6.6110183639399, 54.99762108333333)(6.71118530884808, 55.160646000000035)(6.811352253756261, 54.807481833333334)(6.911519198664441, 54.979666333333334)(7.011686143572621, 55.02417658333332)(7.111853088480801, 55.125154333333306)(7.212020033388982, 55.1449646666667)(7.312186978297162, 54.83642291666665)(7.412353923205343, 54.831112250000004)(7.512520868113523, 55.021044)(7.612687813021703, 54.99194991666669)(7.712854757929884, 54.980165000000035)(7.813021702838064, 54.97436725000001)(7.913188647746244, 55.005927416666665)(8.013355592654424, 54.94434774999999)(8.113522537562606, 54.8955605)(8.213689482470786, 55.02828625000003)(8.313856427378965, 54.96857808333333)(8.414023372287145, 54.85219558333335)(8.514190317195327, 55.0554774166667)(8.614357262103507, 54.96483516666667)(8.714524207011687, 54.83927591666666)(8.814691151919867, 54.92810283333336)(8.914858096828048, 55.019405333333324)(9.015025041736228, 55.28858566666667)(9.115191986644408, 54.91099183333332)(9.215358931552588, 55.084209333333334)(9.31552587646077, 54.977500249999956)(9.41569282136895, 55.1323661666667)(9.51585976627713, 54.88728025000001)(9.61602671118531, 55.24982824999998)(9.71619365609349, 55.16230933333333)(9.81636060100167, 54.97804358333336)(9.916527545909851, 55.196239916666656)(10.016694490818029, 55.02261483333335)(10.11686143572621, 55.01549916666666)(10.21702838063439, 55.115698416666646)(10.31719532554257, 55.120154416666665)(10.41736227045075, 55.11674108333332)(10.51752921535893, 54.99466624999997)(10.617696160267112, 55.138830666666664)(10.717863105175292, 55.034390166666654)(10.818030050083472, 55.02306716666664)(10.918196994991652, 55.09052683333332)(11.018363939899833, 55.08407699999999)(11.118530884808013, 54.93240066666669)(11.218697829716193, 55.04428741666666)(11.318864774624373, 54.98851175)(11.419031719532555, 55.054673833333325)(11.519198664440735, 54.777432499999996)(11.619365609348915, 55.09144758333333)(11.719532554257095, 54.98673108333335)(11.819699499165276, 55.083680916666665)(11.919866444073456, 55.06432183333333)(12.020033388981636, 55.00582091666668)(12.120200333889816, 55.09072483333333)(12.220367278797998, 55.07700749999999)(12.320534223706177, 55.12447741666666)(12.420701168614357, 55.03979633333331)(12.520868113522537, 54.976014500000005)(12.621035058430719, 55.075568249999975)(12.721202003338899, 54.95737341666665)(12.821368948247079, 55.05534500000002)(12.921535893155259, 54.96135158333332)(13.021702838063439, 55.04045274999999)(13.12186978297162, 55.05490216666665)(13.2220367278798, 54.98446824999998)(13.32220367278798, 55.12063224999999)(13.42237061769616, 55.192023166666665)(13.522537562604342, 55.07675333333331)(13.622704507512521, 54.99672541666666)(13.722871452420701, 55.10124874999998)(13.823038397328881, 55.070100666666704)(13.923205342237063, 55.173656)(14.023372287145243, 55.21348675)(14.123539232053423, 55.29680508333333)(14.223706176961603, 55.03575783333333)(14.323873121869784, 55.05864108333331)(14.424040066777964, 55.10410700000001)(14.524207011686144, 55.07101083333334)(14.624373956594324, 55.045874)(14.724540901502506, 55.04800066666664)(14.824707846410686, 55.00701074999997)(14.924874791318866, 55.22235166666669)(15.025041736227045, 55.233333583333305)(15.125208681135225, 55.157681)(15.225375626043405, 54.991115)(15.325542570951589, 55.26854150000001)(15.425709515859769, 55.010020833333336)(15.525876460767948, 55.203191833333335)(15.626043405676128, 55.02764549999999)(15.726210350584308, 55.01284958333333)(15.826377295492488, 55.203639750000015)(15.926544240400668, 55.145554666666676)(16.026711185308848, 55.19422041666663)(16.126878130217026, 55.18234366666666)(16.22704507512521, 55.24908008333329)(16.32721202003339, 55.23560691666667)(16.42737896494157, 55.109417)(16.52754590984975, 55.12059716666665)(16.62771285475793, 55.26871308333331)(16.72787979966611, 55.16902266666668)(16.82804674457429, 55.178122416666696)(16.92821368948247, 55.19134091666667)(17.028380634390654, 55.13810841666668)(17.128547579298832, 55.143531)(17.228714524207014, 55.16240041666667)(17.328881469115192, 55.34205683333331)(17.429048414023374, 55.106884)(17.529215358931552, 55.154496166666654)(17.629382303839733, 55.10891883333334)(17.72954924874791, 55.154048916666646)(17.829716193656097, 55.05569591666665)(17.929883138564275, 55.248541416666676)(18.030050083472457, 55.26461408333335)(18.130217028380635, 54.96902058333331)(18.230383973288816, 55.12925874999999)(18.330550918196995, 55.184296833333356)(18.430717863105176, 55.03684641666668)(18.530884808013354, 54.883231666666674)(18.63105175292154, 55.20223591666668)(18.731218697829718, 55.27437983333334)(18.8313856427379, 55.258515249999974)(18.931552587646078, 55.12009833333338)(19.03171953255426, 55.29961141666668)(19.131886477462437, 55.13652675)(19.23205342237062, 55.24638966666665)(19.332220367278797, 55.177863)(19.43238731218698, 55.188218666666664)(19.53255425709516, 55.147858750000005)(19.63272120200334, 55.14310291666668)(19.73288814691152, 55.10800808333333)(19.833055091819702, 55.090755583333355)(19.93322203672788, 55.095603000000004)(20.033388981636058, 55.19572549999999)(20.13355592654424, 55.44799858333332)(20.23372287145242, 55.23467591666667)(20.333889816360603, 55.19797874999999)(20.43405676126878, 55.418676249999976)(20.534223706176963, 55.34154324999999)(20.63439065108514, 55.28960816666668)(20.734557595993323, 55.380000166666626)(20.8347245409015, 55.14674041666669)(20.934891485809683, 55.25951191666664)(21.03505843071786, 55.39900291666667)(21.135225375626046, 55.12491925000002)(21.235392320534224, 55.27579341666664)(21.335559265442406, 55.13783416666666)(21.435726210350584, 55.14786416666668)(21.535893155258766, 55.26862091666667)(21.636060100166944, 55.04994941666665)(21.736227045075125, 55.19732733333333)(21.836393989983303, 55.31033425000002)(21.93656093489149, 55.066656916666645)(22.036727879799667, 55.14451716666667)(22.13689482470785, 55.18697199999998)(22.237061769616027, 55.286561250000005)(22.33722871452421, 55.25364783333332)(22.437395659432386, 55.20420908333331)(22.537562604340568, 55.242433166666686)(22.637729549248746, 54.992468750000015)(22.737896494156928, 55.20451466666669)(22.83806343906511, 55.20031341666666)(22.93823038397329, 55.12455358333331)(23.03839732888147, 55.253586750000004)(23.13856427378965, 55.24243766666669)(23.23873121869783, 55.19120383333336)(23.33889816360601, 55.315471583333355)(23.43906510851419, 55.13387166666667)(23.53923205342237, 55.16093599999999)(23.639398998330552, 55.23414149999999)(23.739565943238734, 55.319519750000026)(23.839732888146912, 55.33522625000002)(23.939899833055094, 55.52624566666665)(24.040066777963272, 55.269282916666675)(24.140233722871454, 55.07584774999999)(24.24040066777963, 55.287873166666664)(24.340567612687813, 55.287985166666665)(24.440734557595995, 55.26367241666666)(24.540901502504177, 55.41514683333331)(24.641068447412355, 55.27703425000002)(24.741235392320537, 55.25246741666668)(24.841402337228715, 55.30407283333337)(24.941569282136896, 55.35139549999999)(25.041736227045075, 55.26467516666667)(25.141903171953256, 55.19678358333333)(25.242070116861438, 55.35075441666667)(25.34223706176962, 55.259893999999996)(25.442404006677798, 55.156098833333346)(25.54257095158598, 55.20762683333332)(25.642737896494157, 55.38262491666667)(25.74290484140234, 55.25892258333331)(25.843071786310517, 55.43783066666665)(25.9432387312187, 55.22175183333333)(26.043405676126877, 55.427999583333346)(26.143572621035062, 55.195323583333334)(26.24373956594324, 55.33722041666668)(26.34390651085142, 55.32041983333332)(26.4440734557596, 55.46670108333334)(26.544240400667782, 55.351882916666646)(26.64440734557596, 55.37789533333332)(26.744574290484138, 55.69514441666664)(26.84474123539232, 55.285437)(26.9449081803005, 55.40856483333334)(27.045075125208683, 55.27883466666666)(27.14524207011686, 55.33076025000002)(27.245409015025043, 55.29639783333332)(27.34557595993322, 55.522059333333324)(27.445742904841403, 55.35286525)(27.54590984974958, 55.38902383333335)(27.646076794657763, 55.36504666666666)(27.746243739565944, 55.305440666666655)(27.846410684474126, 55.49256008333335)(27.946577629382304, 55.37742099999998)(28.046744574290486, 55.47694483333334)(28.146911519198664, 55.36374974999999)(28.247078464106846, 55.3997145)(28.347245409015024, 55.54385424999999)(28.447412353923205, 55.59563683333333)(28.547579298831387, 55.67833925000002)(28.64774624373957, 55.49269199999999)(28.747913188647747, 55.635777333333294)(28.84808013355593, 55.573705583333314)(28.948247078464107, 55.647323416666694)(29.04841402337229, 55.46973716666665)(29.148580968280466, 55.52367666666668)(29.248747913188648, 55.49784416666667)(29.348914858096826, 55.34185316666667)(29.44908180300501, 55.49552966666665)(29.54924874791319, 55.419759749999955)(29.64941569282137, 55.67406725000001)(29.74958263772955, 55.53949458333332)(29.84974958263773, 55.534134250000015)(29.94991652754591, 55.525274416666676)(30.05008347245409, 55.6259045)(30.15025041736227, 55.62749233333336)(30.25041736227045, 55.48200583333331)(30.35058430717863, 55.439509916666665)(30.45075125208681, 55.49382033333335)(30.55091819699499, 55.497339999999994)(30.651085141903177, 55.59463025000002)(30.751252086811355, 55.57311508333329)(30.851419031719537, 55.57918258333334)(30.951585976627715, 55.396449583333315)(31.051752921535897, 55.261470583333306)(31.151919866444075, 55.45642108333335)(31.252086811352257, 55.55861941666665)(31.352253756260435, 55.48885125)(31.452420701168617, 55.63252808333335)(31.552587646076795, 55.57440658333333)(31.652754590984976, 55.531352083333346)(31.752921535893154, 55.65750116666668)(31.853088480801336, 55.57468225000002)(31.953255425709514, 55.554976916666654)(32.053422370617696, 55.41686591666669)(32.15358931552588, 55.61867766666668)(32.25375626043405, 55.49085525000001)(32.35392320534224, 55.536036583333335)(32.45409015025042, 55.685501000000016)(32.554257095158604, 55.477208666666655)(32.65442404006678, 55.49150625000001)(32.75459098497496, 55.558350083333345)(32.85475792988314, 55.74629683333337)(32.95492487479132, 55.530421749999995)(33.0550918196995, 55.67867966666668)(33.15525876460768, 55.53616333333335)(33.25542570951586, 55.561085999999996)(33.35559265442404, 55.54453591666668)(33.45575959933222, 55.689233999999985)(33.5559265442404, 55.38882016666668)(33.65609348914858, 55.56900525)(33.756260434056756, 55.34695541666668)(33.85642737896494, 55.36376016666665)(33.95659432387313, 55.412461)(34.05676126878131, 55.422851666666666)(34.15692821368948, 55.457846166666656)(34.257095158597664, 55.44313050000001)(34.357262103505846, 55.50616950000001)(34.45742904841403, 55.491801416666675)(34.5575959933222, 55.551564916666656)(34.657762938230384, 55.436127416666686)(34.757929883138566, 55.446757666666635)(34.85809682804675, 55.62989816666666)(34.95826377295492, 55.376048333333316)(35.058430717863104, 55.43701183333336)(35.158597662771285, 55.52839725)(35.25876460767947, 55.382660083333306)(35.35893155258764, 55.7499695)(35.45909849749582, 55.50376408333333)(35.559265442404005, 55.41111275)(35.659432387312194, 55.36957341666668)(35.75959933222037, 55.58948825)(35.85976627712855, 55.55503841666666)(35.95993322203673, 55.51328099999999)(36.06010016694491, 55.43742958333333)(36.16026711185309, 55.52816283333332)(36.26043405676127, 55.47601825)(36.36060100166945, 55.48827624999997)(36.46076794657763, 55.43464224999998)(36.56093489148581, 55.32286141666666)(36.66110183639399, 55.63207975)(36.76126878130217, 55.355230833333344)(36.86143572621035, 55.504979916666684)(36.96160267111853, 55.456619833333335)(37.06176961602671, 55.622751666666694)(37.16193656093489, 55.650965416666665)(37.26210350584308, 55.27900766666666)(37.362270450751254, 55.62643983333337)(37.462437395659435, 55.473801)(37.56260434056762, 55.55872608333337)(37.6627712854758, 55.42083758333334)(37.76293823038397, 55.40981141666669)(37.863105175292155, 55.67762225000001)(37.96327212020034, 55.499049250000034)(38.06343906510852, 55.60588566666669)(38.16360601001669, 55.530614416666644)(38.263772954924875, 55.42505983333334)(38.363939899833056, 55.72952700000001)(38.46410684474124, 55.53319866666668)(38.56427378964941, 55.50173449999999)(38.664440734557594, 55.468816583333336)(38.764607679465776, 55.5888064166667)(38.86477462437396, 55.565348583333325)(38.96494156928214, 55.43099516666667)(39.06510851419032, 55.380712499999994)(39.1652754590985, 55.49583441666668)(39.26544240400668, 55.50074316666664)(39.36560934891486, 55.26559016666665)(39.46577629382304, 55.34500208333335)(39.56594323873122, 55.62917608333334)(39.666110183639404, 55.39609833333334)(39.76627712854758, 55.65019708333331)(39.86644407345576, 55.61334741666665)(39.96661101836394, 55.51870750000002)(40.066777963272116, 55.62722716666668)(40.1669449081803, 55.442170583333315)(40.26711185308848, 55.53766375)(40.36727879799666, 55.492732416666655)(40.46744574290484, 55.42899591666667)(40.567612687813025, 55.65687066666667)(40.667779632721206, 55.40264425000003)(40.76794657762939, 55.42378291666666)(40.86811352253756, 55.48443108333332)(40.968280467445744, 55.58909625)(41.068447412353926, 55.64793916666667)(41.16861435726211, 55.55763275000001)(41.26878130217028, 55.482177916666686)(41.368948247078464, 55.57501733333332)(41.469115191986646, 55.61553466666666)(41.56928213689483, 55.605320833333316)(41.669449081803, 55.58682300000001)(41.769616026711184, 55.41503499999997)(41.869782971619365, 55.824960333333344)(41.96994991652755, 55.35978316666665)(42.07011686143572, 55.52882916666666)(42.1702838063439, 55.45162966666666)(42.27045075125209, 55.47128825)(42.370617696160274, 55.41922591666666)(42.47078464106845, 55.504949333333336)(42.57095158597663, 55.60473666666667)(42.67111853088481, 55.42601533333334)(42.77128547579299, 55.50650058333334)(42.87145242070117, 55.64676916666667)(42.97161936560935, 55.64617391666666)(43.07178631051753, 55.46976249999999)(43.17195325542571, 55.41828000000002)(43.27212020033389, 55.770746416666675)(43.37228714524207, 55.47177183333334)(43.47245409015025, 55.57415749999999)(43.57262103505843, 55.67730191666669)(43.67278797996661, 55.44777966666667)(43.77295492487479, 55.540115666666665)(43.87312186978298, 55.45068925)(43.97328881469116, 55.7000738333333)(44.073455759599334, 55.52427633333331)(44.173622704507515, 55.44134041666668)(44.2737896494157, 55.606257583333345)(44.37395659432388, 55.5584765)(44.47412353923205, 55.52881441666666)(44.574290484140235, 55.77720591666667)(44.67445742904842, 55.42975383333333)(44.7746243739566, 55.48668966666671)(44.87479131886477, 55.56452950000001)(44.974958263772955, 55.46246858333333)(45.075125208681136, 55.500443)(45.17529215358932, 55.625859750000025)(45.27545909849749, 55.67430658333331)(45.375626043405674, 55.47057108333333)(45.475792988313856, 55.52172366666669)(45.57595993322204, 55.683908666666646)(45.67612687813022, 55.68906658333333)(45.7762938230384, 55.45993083333334)(45.87646076794658, 55.54520716666668)(45.976627712854764, 55.700317083333296)(46.07679465776294, 55.62450666666667)(46.17696160267112, 55.67832933333331)(46.2771285475793, 55.657897999999975)(46.37729549248748, 55.46742266666662)(46.47746243739566, 55.54836558333333)(46.57762938230384, 55.71373508333336)(46.67779632721202, 55.54516566666664)(46.7779632721202, 55.34466125000002)(46.87813021702838, 55.74337225000002)(46.97829716193656, 55.60557083333332)(47.07846410684474, 55.58929483333331)(47.17863105175292, 55.697361750000034)(47.278797996661105, 55.581120916666684)(47.378964941569286, 55.565562)(47.47913188647747, 55.50229416666668)(47.57929883138564, 55.87046791666667)(47.679465776293824, 55.33244416666667)(47.779632721202006, 55.65985574999998)(47.87979966611019, 55.620127333333336)(47.97996661101836, 55.682027)(48.080133555926544, 55.581583583333334)(48.180300500834726, 55.60217341666666)(48.28046744574291, 56.096591666666654)(48.38063439065108, 55.41635708333332)(48.48080133555926, 55.690047833333324)(48.580968280467445, 55.56910233333333)(48.68113522537563, 55.59599825)(48.7813021702838, 55.863957000000006)(48.88146911519199, 55.55644241666668)(48.98163606010017, 55.819009999999984)(49.081803005008354, 55.56810008333334)(49.18196994991653, 55.55436233333334)(49.28213689482471, 55.72526533333333)(49.38230383973289, 55.54298916666668)(49.48247078464107, 55.651743833333356)(49.58263772954925, 55.80596849999998)(49.68280467445743, 55.643950999999994)(49.78297161936561, 55.71040383333334)(49.88313856427379, 55.554697749999995)(49.98330550918197, 55.78559808333332)(50.08347245409015, 55.588831916666656)(50.18363939899833, 55.56001316666663)(50.28380634390651, 55.919264500000004)(50.38397328881469, 55.598063666666675)(50.484140233722876, 55.62485749999998)(50.58430717863106, 55.63770533333336)(50.68447412353924, 55.641657833333355)(50.784641068447414, 55.72609950000001)(50.884808013355595, 55.65508458333333)(50.98497495826378, 56.01664116666665)(51.08514190317196, 55.65769966666663)(51.18530884808013, 55.65180441666668)(51.285475792988315, 55.916590083333354)(51.3856427378965, 55.73988308333335)(51.48580968280468, 55.63273649999999)(51.58597662771285, 55.51991283333334)(51.686143572621035, 55.85598675000002)(51.786310517529216, 55.48445175000001)(51.8864774624374, 55.53924066666664)(51.98664440734557, 55.80347616666665)(52.086811352253754, 55.694666250000026)(52.18697829716194, 55.79895599999999)(52.287145242070125, 55.62337208333333)(52.3873121869783, 55.56428016666668)(52.48747913188648, 55.711196999999956)(52.58764607679466, 55.64325883333335)(52.68781302170284, 55.89432783333333)(52.78797996661102, 55.56114125000002)(52.8881469115192, 55.64324008333333)(52.98831385642738, 55.84335758333336)(53.088480801335564, 55.48816450000002)(53.18864774624374, 55.60597275)(53.28881469115192, 55.708760166666664)(53.3889816360601, 55.691614416666674)(53.489148580968276, 55.74018349999999)(53.58931552587646, 55.70059166666662)(53.68948247078464, 55.745259499999996)(53.78964941569283, 55.651072000000006)(53.889816360601, 55.58071441666669)(53.989983305509185, 56.050723500000004)(54.090150250417366, 55.67744983333333)(54.19031719532555, 55.82863325)(54.29048414023372, 55.59213241666666)(54.390651085141904, 55.62775174999999)(54.490818030050086, 55.840982333333365)(54.59098497495827, 55.76278683333333)(54.69115191986644, 55.99932774999999)(54.791318864774624, 55.57777358333335)(54.891485809682806, 55.480144083333336)(54.99165275459099, 55.94827216666668)(55.09181969949916, 55.57033274999999)(55.19198664440734, 55.50592566666664)(55.292153589315525, 56.1212605)(55.39232053422371, 55.78855875)(55.49248747913189, 55.56741316666666)(55.59265442404007, 55.87422149999999)(55.69282136894825, 55.86302066666665)(55.79298831385643, 55.386755500000014)(55.89315525876461, 55.83078483333332)(55.99332220367279, 55.79729075)(56.09348914858097, 55.58448883333333)(56.19365609348915, 55.76762358333335)(56.29382303839733, 55.759058083333336)(56.39398998330551, 55.844893999999975)(56.49415692821369, 55.71015874999999)(56.59432387312187, 55.6795145)(56.69449081803005, 55.96688275)(56.79465776293823, 55.56100983333333)(56.89482470784641, 55.896442916666665)(56.99499165275459, 55.72305766666668)(57.095158597662774, 55.822377083333336)(57.195325542570956, 55.91392449999998)(57.29549248747914, 55.69148249999999)(57.39565943238732, 55.98936858333334)(57.495826377295494, 55.65285233333335)(57.595993322203675, 55.48272758333335)(57.69616026711186, 56.24267966666667)(57.79632721202004, 55.62384966666666)(57.89649415692821, 55.783132)(57.996661101836395, 55.84379974999998)(58.09682804674458, 55.74991391666667)(58.19699499165276, 55.81395958333331)(58.29716193656093, 55.72827066666667)(58.397328881469114, 56.062809)(58.497495826377296, 55.629445583333336)(58.59766277128548, 55.628494083333315)(58.69782971619365, 56.03533358333331)(58.79799666110184, 55.8049520833333)(58.89816360601002, 55.75642266666666)(58.9983305509182, 55.71738158333332)(59.09849749582638, 55.63153608333334)(59.19866444073456, 55.95233608333335)(59.29883138564274, 55.715784333333325)(59.398998330550924, 56.109042666666646)(59.4991652754591, 55.62317483333333)(59.59933222036728, 55.63756791666665)(59.69949916527546, 56.117155499999996)(59.79966611018364, 55.689163499999985)(59.89983305509182, 55.83932933333333)(60.0, 55.64555841666668)
    };
    \addplot[color=blue, mark=none,name path=A] coordinates { %% MAX value
    (0.0, 0.0)(0.1001669449081803, 58.008269999999996)(0.2003338898163606, 58.30796)(0.3005008347245409, 65.54305000000001)(0.4006677796327212, 71.62946)(0.5008347245409015, 56.78207999999999)(0.6010016694490818, 64.37299999999999)(0.7011686143572621, 61.10519)(0.8013355592654424, 59.125209999999996)(0.9015025041736228, 57.13548)(1.001669449081803, 56.735699999999994)(1.1018363939899833, 56.8657)(1.2020033388981637, 56.54771)(1.3021702838063438, 56.606300000000005)(1.4023372287145242, 56.5117)(1.5025041736227045, 56.387190000000004)(1.6026711185308848, 61.47873)(1.7028380634390652, 56.38962)(1.8030050083472455, 56.35179)(1.9031719532554257, 56.065529999999995)(2.003338898163606, 57.07383)(2.1035058430717863, 56.61851)(2.2036727879799667, 56.506809999999994)(2.303839732888147, 56.30906)(2.4040066777963274, 56.67161)(2.5041736227045073, 56.401830000000004)(2.6043405676126876, 56.75644)(2.704507512520868, 56.58127999999999)(2.8046744574290483, 56.56602)(2.9048414023372287, 56.63621)(3.005008347245409, 56.31821000000001)(3.1051752921535893, 56.57273)(3.2053422370617697, 57.10984)(3.30550918196995, 56.96763)(3.4056761268781304, 56.89805)(3.5058430717863107, 57.139129999999994)(3.606010016694491, 56.785129999999995)(3.7061769616026714, 56.47385)(3.8063439065108513, 56.654520000000005)(3.906510851419032, 57.67442)(4.006677796327212, 58.381800000000005)(4.106844741235393, 56.77537)(4.207011686143573, 57.210539999999995)(4.3071786310517535, 57.333220000000004)(4.407345575959933, 56.38414)(4.507512520868114, 56.39634)(4.607679465776294, 57.18675)(4.707846410684475, 56.69358)(4.808013355592655, 56.67344)(4.908180300500835, 57.003029999999995)(5.0083472454090145, 57.2136)(5.108514190317195, 57.00059)(5.208681135225375, 56.564800000000005)(5.308848080133556, 58.08579)(5.409015025041736, 56.70456)(5.509181969949917, 56.76499)(5.609348914858097, 56.818090000000005)(5.709515859766277, 57.14402)(5.809682804674457, 56.796730000000004)(5.909849749582638, 56.34385)(6.010016694490818, 57.224579999999996)(6.110183639398999, 56.86387)(6.210350584307179, 56.84189)(6.3105175292153595, 56.85105)(6.410684474123539, 56.84983)(6.510851419031719, 56.92551)(6.6110183639399, 56.83762)(6.71118530884808, 65.90681000000001)(6.811352253756261, 56.825419999999994)(6.911519198664441, 56.96763)(7.011686143572621, 57.10373)(7.111853088480801, 66.85896)(7.212020033388982, 57.07017)(7.312186978297162, 56.70945)(7.412353923205343, 56.77415)(7.512520868113523, 56.65818)(7.612687813021703, 56.65879)(7.712854757929884, 58.17856)(7.813021702838064, 56.927960000000006)(7.913188647746244, 56.91209)(8.013355592654424, 56.94932)(8.113522537562606, 57.342389999999995)(8.213689482470786, 56.64353)(8.313856427378965, 56.83274)(8.414023372287145, 56.80771)(8.514190317195327, 57.4083)(8.614357262103507, 57.37473)(8.714524207011687, 56.12168)(8.814691151919867, 57.1251)(8.914858096828048, 57.488260000000004)(9.015025041736228, 71.73199)(9.115191986644408, 56.76438)(9.215358931552588, 56.636810000000004)(9.31552587646077, 56.48484)(9.41569282136895, 56.86569)(9.51585976627713, 58.54844)(9.61602671118531, 64.0074)(9.71619365609349, 57.573100000000004)(9.81636060100167, 57.33262)(9.916527545909851, 57.08908)(10.016694490818029, 57.44797)(10.11686143572621, 58.09311)(10.21702838063439, 57.17942)(10.31719532554257, 57.80381)(10.41736227045075, 57.4083)(10.51752921535893, 57.45103)(10.617696160267112, 57.298429999999996)(10.717863105175292, 57.137919999999994)(10.818030050083472, 57.661590000000004)(10.918196994991652, 57.35215)(11.018363939899833, 57.11289)(11.118530884808013, 57.23862)(11.218697829716193, 57.95395)(11.318864774624373, 57.15501)(11.419031719532555, 57.140969999999996)(11.519198664440735, 56.49949)(11.619365609348915, 57.68418)(11.719532554257095, 57.47849000000001)(11.819699499165276, 57.06955)(11.919866444073456, 56.83518)(12.020033388981636, 56.900490000000005)(12.120200333889816, 57.28257)(12.220367278797998, 56.766819999999996)(12.320534223706177, 57.15439)(12.420701168614357, 58.68576)(12.520868113522537, 58.60397)(12.621035058430719, 57.54196999999999)(12.721202003338899, 56.94748)(12.821368948247079, 57.101910000000004)(12.921535893155259, 56.87241)(13.021702838063439, 57.1721)(13.12186978297162, 57.21604)(13.2220367278798, 57.315529999999995)(13.32220367278798, 57.47177)(13.42237061769616, 56.82481)(13.522537562604342, 56.95481)(13.622704507512521, 56.60813)(13.722871452420701, 57.24656)(13.823038397328881, 57.40281)(13.923205342237063, 56.852270000000004)(14.023372287145243, 57.40403)(14.123539232053423, 57.44126)(14.223706176961603, 57.480320000000006)(14.323873121869784, 57.40707999999999)(14.424040066777964, 56.81382)(14.524207011686144, 56.32127)(14.624373956594324, 56.945660000000004)(14.724540901502506, 56.65879)(14.824707846410686, 57.03172000000001)(14.924874791318866, 57.68235)(15.025041736227045, 58.215180000000004)(15.125208681135225, 57.19162)(15.225375626043405, 57.27401999999999)(15.325542570951589, 57.51877)(15.425709515859769, 57.39548)(15.525876460767948, 56.978)(15.626043405676128, 57.00547)(15.726210350584308, 56.90842)(15.826377295492488, 57.24473)(15.926544240400668, 57.004859999999994)(16.026711185308848, 57.101290000000006)(16.126878130217026, 57.2667)(16.22704507512521, 57.11289)(16.32721202003339, 57.13303)(16.42737896494157, 57.108619999999995)(16.52754590984975, 57.13303)(16.62771285475793, 57.36069)(16.72787979966611, 57.37107)(16.82804674457429, 56.626450000000006)(16.92821368948247, 56.96274)(17.028380634390654, 57.299049999999994)(17.128547579298832, 56.93345)(17.228714524207014, 57.29356)(17.328881469115192, 56.937110000000004)(17.429048414023374, 56.7125)(17.529215358931552, 57.11776999999999)(17.629382303839733, 56.810159999999996)(17.72954924874791, 57.12022)(17.829716193656097, 57.332010000000004)(17.929883138564275, 57.53647)(18.030050083472457, 57.219699999999996)(18.130217028380635, 56.96824)(18.230383973288816, 57.03111)(18.330550918196995, 56.83213000000001)(18.430717863105176, 57.47971)(18.530884808013354, 56.85349000000001)(18.63105175292154, 56.91269)(18.731218697829718, 57.4144)(18.8313856427379, 57.043929999999996)(18.931552587646078, 57.21726)(19.03171953255426, 57.31308)(19.131886477462437, 56.68686)(19.23205342237062, 57.20078)(19.332220367278797, 57.25693)(19.43238731218698, 57.08298)(19.53255425709516, 57.03111)(19.63272120200334, 57.900850000000005)(19.73288814691152, 57.02073)(19.833055091819702, 57.32651)(19.93322203672788, 57.44553)(20.033388981636058, 57.21543)(20.13355592654424, 57.47788)(20.23372287145242, 57.376560000000005)(20.333889816360603, 57.18979)(20.43405676126878, 57.08603)(20.534223706176963, 57.29965)(20.63439065108514, 56.92063)(20.734557595993323, 56.989599999999996)(20.8347245409015, 56.93772)(20.934891485809683, 57.20017)(21.03505843071786, 57.47971)(21.135225375626046, 57.39426)(21.235392320534224, 56.83762)(21.335559265442406, 57.39792)(21.435726210350584, 57.387550000000005)(21.535893155258766, 57.02439)(21.636060100166944, 57.388769999999994)(21.736227045075125, 57.884370000000004)(21.836393989983303, 57.1605)(21.93656093489149, 57.08238)(22.036727879799667, 56.73873999999999)(22.13689482470785, 56.99509)(22.237061769616027, 57.17575)(22.33722871452421, 56.67711)(22.437395659432386, 57.187349999999995)(22.537562604340568, 57.227639999999994)(22.637729549248746, 56.830290000000005)(22.737896494156928, 56.73325)(22.83806343906511, 57.05491000000001)(22.93823038397329, 56.83213000000001)(23.03839732888147, 56.8419)(23.13856427378965, 57.196509999999996)(23.23873121869783, 56.86082)(23.33889816360601, 56.89194)(23.43906510851419, 56.66978)(23.53923205342237, 57.034760000000006)(23.639398998330552, 57.465669999999996)(23.739565943238734, 56.83884)(23.839732888146912, 57.11289)(23.939899833055094, 69.68487999999999)(24.040066777963272, 57.2368)(24.140233722871454, 57.380829999999996)(24.24040066777963, 57.86973)(24.340567612687813, 57.22215)(24.440734557595995, 56.89072)(24.540901502504177, 57.4437)(24.641068447412355, 56.92551)(24.741235392320537, 57.20322)(24.841402337228715, 57.1251)(24.941569282136896, 57.11716)(25.041736227045075, 56.7772)(25.141903171953256, 57.14402)(25.242070116861438, 57.02439)(25.34223706176962, 57.207499999999996)(25.442404006677798, 57.11228)(25.54257095158598, 57.25571000000001)(25.642737896494157, 57.49679999999999)(25.74290484140234, 57.02744)(25.843071786310517, 57.097629999999995)(25.9432387312187, 57.43272)(26.043405676126877, 57.56577)(26.143572621035062, 57.13425)(26.24373956594324, 57.04392)(26.34390651085142, 57.25022)(26.4440734557596, 57.02439)(26.544240400667782, 56.964569999999995)(26.64440734557596, 65.35688999999999)(26.744574290484138, 78.95121999999999)(26.84474123539232, 58.470310000000005)(26.9449081803005, 57.72202)(27.045075125208683, 57.358869999999996)(27.14524207011686, 57.405249999999995)(27.245409015025043, 57.376569999999994)(27.34557595993322, 81.83085)(27.445742904841403, 61.73751)(27.54590984974958, 57.769009999999994)(27.646076794657763, 62.84469)(27.746243739565944, 58.93967)(27.846410684474126, 72.57794)(27.946577629382304, 73.10711)(28.046744574290486, 71.00262)(28.146911519198664, 69.33943000000001)(28.247078464106846, 71.45672)(28.347245409015024, 73.6784)(28.447412353923205, 73.60638)(28.547579298831387, 74.57013)(28.64774624373957, 70.45697)(28.747913188647747, 69.71357)(28.84808013355593, 71.768)(28.948247078464107, 73.50384)(29.04841402337229, 72.89410000000001)(29.148580968280466, 73.05829)(29.248747913188648, 71.61114)(29.348914858096826, 76.31572)(29.44908180300501, 74.87652)(29.54924874791319, 74.61224)(29.64941569282137, 72.05059)(29.74958263772955, 74.33575)(29.84974958263773, 72.52667)(29.94991652754591, 73.82916)(30.05008347245409, 72.02374)(30.15025041736227, 74.84966)(30.25041736227045, 75.11272)(30.35058430717863, 73.54291)(30.45075125208681, 61.335899999999995)(30.55091819699499, 72.20379)(30.651085141903177, 75.45452)(30.751252086811355, 75.92937)(30.851419031719537, 72.68353)(30.951585976627715, 71.76862)(31.051752921535897, 64.31746000000001)(31.151919866444075, 67.62922)(31.252086811352257, 61.99752)(31.352253756260435, 71.41828000000001)(31.452420701168617, 72.74517)(31.552587646076795, 72.34235)(31.652754590984976, 70.74566)(31.752921535893154, 72.69147000000001)(31.853088480801336, 71.58917)(31.953255425709514, 73.065)(32.053422370617696, 66.39143)(32.15358931552588, 77.7299)(32.25375626043405, 76.54644)(32.35392320534224, 71.59528)(32.45409015025042, 73.28594)(32.554257095158604, 73.15594)(32.65442404006678, 72.12262)(32.75459098497496, 71.08258000000001)(32.85475792988314, 75.19146)(32.95492487479132, 74.65435000000001)(33.0550918196995, 75.88115)(33.15525876460768, 75.04192)(33.25542570951586, 74.89544000000001)(33.35559265442404, 74.21978)(33.45575959933222, 67.29414)(33.5559265442404, 68.3177)(33.65609348914858, 63.18587)(33.756260434056756, 64.45785)(33.85642737896494, 58.50082)(33.95659432387313, 58.07297)(34.05676126878131, 57.25693)(34.15692821368948, 57.49192)(34.257095158597664, 57.14036)(34.357262103505846, 57.37351)(34.45742904841403, 57.63474000000001)(34.5575959933222, 57.40524)(34.657762938230384, 57.02439)(34.757929883138566, 59.20945)(34.85809682804675, 57.45407)(34.95826377295492, 57.19468)(35.058430717863104, 57.61277)(35.158597662771285, 57.95945)(35.25876460767947, 57.95823)(35.35893155258764, 64.53963)(35.45909849749582, 78.79862)(35.559265442404005, 57.219699999999996)(35.659432387312194, 57.092749999999995)(35.75959933222037, 57.58042)(35.85976627712855, 59.2363)(35.95993322203673, 57.43943)(36.06010016694491, 57.30332)(36.16026711185309, 58.04795)(36.26043405676127, 57.86912)(36.36060100166945, 57.92893)(36.46076794657763, 57.48276)(36.56093489148581, 57.70555)(36.66110183639399, 58.973240000000004)(36.76126878130217, 57.688449999999996)(36.86143572621035, 57.40769)(36.96160267111853, 57.477270000000004)(37.06176961602671, 57.565160000000006)(37.16193656093489, 58.130340000000004)(37.26210350584308, 57.69638)(37.362270450751254, 57.81785)(37.462437395659435, 57.43515000000001)(37.56260434056762, 57.58103)(37.6627712854758, 57.89475)(37.76293823038397, 58.3934)(37.863105175292155, 58.37509)(37.96327212020034, 57.70554)(38.06343906510852, 57.0604)(38.16360601001669, 57.14402)(38.263772954924875, 57.12815)(38.363939899833056, 58.01133)(38.46410684474124, 57.83005)(38.56427378964941, 57.83493)(38.664440734557594, 57.6854)(38.764607679465776, 58.32931000000001)(38.86477462437396, 57.53221)(38.96494156928214, 57.889860000000006)(39.06510851419032, 57.756809999999994)(39.1652754590985, 57.54013)(39.26544240400668, 57.252660000000006)(39.36560934891486, 57.0366)(39.46577629382304, 57.17148)(39.56594323873122, 57.88071000000001)(39.666110183639404, 57.576150000000005)(39.76627712854758, 57.9155)(39.86644407345576, 57.48276)(39.96661101836394, 58.14805)(40.066777963272116, 57.900850000000005)(40.1669449081803, 57.58347)(40.26711185308848, 58.0333)(40.36727879799666, 58.16696999999999)(40.46744574290484, 58.30063)(40.567612687813025, 57.64328)(40.667779632721206, 57.621320000000004)(40.76794657762939, 57.051860000000005)(40.86811352253756, 57.392430000000004)(40.968280467445744, 57.50352)(41.068447412353926, 58.20175)(41.16861435726211, 58.265840000000004)(41.26878130217028, 57.07566)(41.368948247078464, 58.11081)(41.469115191986646, 57.94297)(41.56928213689483, 58.231049999999996)(41.669449081803, 57.458960000000005)(41.769616026711184, 57.11351)(41.869782971619365, 57.96067)(41.96994991652755, 57.96128)(42.07011686143572, 58.0394)(42.1702838063439, 57.9332)(42.27045075125209, 57.585300000000004)(42.370617696160274, 57.53586)(42.47078464106845, 57.67442)(42.57095158597663, 57.94052)(42.67111853088481, 57.35398)(42.77128547579299, 57.03965)(42.87145242070117, 58.56491)(42.97161936560935, 57.14341)(43.07178631051753, 57.64023)(43.17195325542571, 58.92868)(43.27212020033389, 57.968599999999995)(43.37228714524207, 58.52524)(43.47245409015025, 57.85996)(43.57262103505843, 57.576150000000005)(43.67278797996661, 57.59568)(43.77295492487479, 58.46176)(43.87312186978298, 57.60423)(43.97328881469116, 58.02109)(44.073455759599334, 57.45591)(44.173622704507515, 57.53342)(44.2737896494157, 58.10349000000001)(44.37395659432388, 57.77451)(44.47412353923205, 57.91489)(44.574290484140235, 58.13279)(44.67445742904842, 57.12449)(44.7746243739566, 57.31675)(44.87479131886477, 57.778780000000005)(44.974958263772955, 57.49131)(45.075125208681136, 58.503879999999995)(45.17529215358932, 69.54145)(45.27545909849749, 57.55845)(45.375626043405674, 57.49192)(45.475792988313856, 57.6738)(45.57595993322204, 58.067479999999996)(45.67612687813022, 58.342749999999995)(45.7762938230384, 57.92099999999999)(45.87646076794658, 57.19956)(45.976627712854764, 57.80686)(46.07679465776294, 57.98814)(46.17696160267112, 57.81052)(46.2771285475793, 57.632909999999995)(46.37729549248748, 57.54196999999999)(46.47746243739566, 57.561499999999995)(46.57762938230384, 57.89353)(46.67779632721202, 57.934419999999996)(46.7779632721202, 57.41318)(46.87813021702838, 58.351290000000006)(46.97829716193656, 58.68638)(47.07846410684474, 57.94908)(47.17863105175292, 58.0742)(47.278797996661105, 58.047940000000004)(47.378964941569286, 57.902069999999995)(47.47913188647747, 57.107400000000005)(47.57929883138564, 58.54294)(47.679465776293824, 59.00925)(47.779632721202006, 58.35496)(47.87979966611019, 58.03147)(47.97996661101836, 57.7269)(48.080133555926544, 57.71897)(48.180300500834726, 57.67502999999999)(48.28046744574291, 58.13157)(48.38063439065108, 58.043060000000004)(48.48080133555926, 58.17124)(48.580968280467445, 57.811130000000006)(48.68113522537563, 57.29416)(48.7813021702838, 58.348240000000004)(48.88146911519199, 58.51243)(48.98163606010017, 59.058080000000004)(49.081803005008354, 57.94358)(49.18196994991653, 57.90024)(49.28213689482471, 58.335429999999995)(49.38230383973289, 58.16696999999999)(49.48247078464107, 58.236549999999994)(49.58263772954925, 58.37265000000001)(49.68280467445743, 58.755340000000004)(49.78297161936561, 59.14109)(49.88313856427379, 58.07785)(49.98330550918197, 57.692119999999996)(50.08347245409015, 57.40952)(50.18363939899833, 57.72934)(50.28380634390651, 58.85544)(50.38397328881469, 57.64695)(50.484140233722876, 57.860569999999996)(50.58430717863106, 57.860569999999996)(50.68447412353924, 58.33298)(50.784641068447414, 58.46909)(50.884808013355595, 57.99362000000001)(50.98497495826378, 58.72422)(51.08514190317196, 57.55051)(51.18530884808013, 59.03914999999999)(51.285475792988315, 57.812960000000004)(51.3856427378965, 57.71957)(51.48580968280468, 57.495580000000004)(51.58597662771285, 57.20505)(51.686143572621035, 57.665870000000005)(51.786310517529216, 57.32529)(51.8864774624374, 57.03843)(51.98664440734557, 58.26706)(52.086811352253754, 58.246919999999996)(52.18697829716194, 57.82273)(52.287145242070125, 57.963719999999995)(52.3873121869783, 57.73911)(52.48747913188648, 57.81418000000001)(52.58764607679466, 57.65793000000001)(52.68781302170284, 58.29453)(52.78797996661102, 57.24595)(52.8881469115192, 57.58225)(52.98831385642738, 57.68479)(53.088480801335564, 57.6207)(53.18864774624374, 57.889869999999995)(53.28881469115192, 58.23349)(53.3889816360601, 57.74216)(53.489148580968276, 58.04123)(53.58931552587646, 58.7291)(53.68948247078464, 58.66684)(53.78964941569283, 57.90817)(53.889816360601, 57.544399999999996)(53.989983305509185, 58.601530000000004)(54.090150250417366, 57.935030000000005)(54.19031719532555, 57.8917)(54.29048414023372, 57.853849999999994)(54.390651085141904, 57.7562)(54.490818030050086, 58.16879)(54.59098497495827, 58.187110000000004)(54.69115191986644, 58.64365)(54.791318864774624, 57.66343)(54.891485809682806, 57.219699999999996)(54.99165275459099, 58.448339999999995)(55.09181969949916, 57.71653)(55.19198664440734, 57.46384)(55.292153589315525, 58.570409999999995)(55.39232053422371, 58.774879999999996)(55.49248747913189, 58.38791)(55.59265442404007, 57.783049999999996)(55.69282136894825, 58.21151999999999)(55.79298831385643, 57.96127)(55.89315525876461, 58.84262)(55.99332220367279, 58.30856)(56.09348914858097, 57.87094)(56.19365609348915, 58.293910000000004)(56.29382303839733, 60.876310000000004)(56.39398998330551, 58.61923)(56.49415692821369, 57.65611)(56.59432387312187, 58.06686)(56.69449081803005, 58.15476)(56.79465776293823, 58.090669999999996)(56.89482470784641, 58.22739)(56.99499165275459, 58.32016)(57.095158597662774, 58.04856)(57.195325542570956, 57.320409999999995)(57.29549248747914, 58.19749)(57.39565943238732, 58.24937)(57.495826377295494, 57.64023)(57.595993322203675, 57.651219999999995)(57.69616026711186, 58.210300000000004)(57.79632721202004, 57.812349999999995)(57.89649415692821, 59.17282)(57.996661101836395, 58.6528)(58.09682804674458, 58.11631)(58.19699499165276, 57.86484)(58.29716193656093, 57.96799)(58.397328881469114, 58.276219999999995)(58.497495826377296, 58.403169999999996)(58.59766277128548, 57.51328)(58.69782971619365, 58.55087)(58.79799666110184, 57.379619999999996)(58.89816360601002, 58.08579)(58.9983305509182, 57.89353)(59.09849749582638, 58.11448)(59.19866444073456, 58.21091)(59.29883138564274, 58.11813)(59.398998330550924, 58.34518)(59.4991652754591, 57.38205)(59.59933222036728, 58.083349999999996)(59.69949916527546, 58.30064)(59.79966611018364, 57.71103)(59.89983305509182, 58.67295)(60.0, 57.80625)
    };
    \addplot[color=blue, mark=none,name path=B] coordinates { %% MIN value
    (0.0, 0.0)(0.1001669449081803, 50.29345)(0.2003338898163606, 52.592020000000005)(0.3005008347245409, 51.73692)(0.4006677796327212, 51.09545)(0.5008347245409015, 52.60485)(0.6010016694490818, 52.759879999999995)(0.7011686143572621, 52.18248)(0.8013355592654424, 51.75158)(0.9015025041736228, 52.914899999999996)(1.001669449081803, 52.77513)(1.1018363939899833, 52.58837)(1.2020033388981637, 51.368280000000006)(1.3021702838063438, 53.04674)(1.4023372287145242, 52.07994)(1.5025041736227045, 52.56273)(1.6026711185308848, 53.50023)(1.7028380634390652, 52.59142)(1.8030050083472455, 52.79284)(1.9031719532554257, 52.02562999999999)(2.003338898163606, 53.224349999999994)(2.1035058430717863, 52.57127)(2.2036727879799667, 53.25792)(2.303839732888147, 52.59264)(2.4040066777963274, 53.22068)(2.5041736227045073, 53.10777)(2.6043405676126876, 52.47789)(2.704507512520868, 52.85142999999999)(2.8046744574290483, 53.14927)(2.9048414023372287, 52.776360000000004)(3.005008347245409, 52.51635)(3.1051752921535893, 51.615469999999995)(3.2053422370617697, 53.00463)(3.30550918196995, 52.7727)(3.4056761268781304, 52.7965)(3.5058430717863107, 53.397690000000004)(3.606010016694491, 52.68175)(3.7061769616026714, 51.899280000000005)(3.8063439065108513, 52.886829999999996)(3.906510851419032, 52.96801000000001)(4.006677796327212, 53.397690000000004)(4.106844741235393, 51.822379999999995)(4.207011686143573, 53.89207)(4.3071786310517535, 52.65612)(4.407345575959933, 52.85142999999999)(4.507512520868114, 52.55236)(4.607679465776294, 52.38817)(4.707846410684475, 52.554790000000004)(4.808013355592655, 52.41686)(4.908180300500835, 51.45128)(5.0083472454090145, 52.40648)(5.108514190317195, 52.33995)(5.208681135225375, 52.37413)(5.308848080133556, 52.933209999999995)(5.409015025041736, 52.795880000000004)(5.509181969949917, 53.0034)(5.609348914858097, 53.15294)(5.709515859766277, 52.519400000000005)(5.809682804674457, 52.51452)(5.909849749582638, 52.85937)(6.010016694490818, 53.323229999999995)(6.110183639398999, 52.33629)(6.210350584307179, 52.55663)(6.3105175292153595, 53.229240000000004)(6.410684474123539, 52.91429)(6.510851419031719, 52.82823)(6.6110183639399, 53.060159999999996)(6.71118530884808, 53.058949999999996)(6.811352253756261, 52.202619999999996)(6.911519198664441, 52.88988)(7.011686143572621, 52.98021)(7.111853088480801, 52.56761)(7.212020033388982, 52.949690000000004)(7.312186978297162, 53.380599999999994)(7.412353923205343, 52.296009999999995)(7.512520868113523, 53.38243)(7.612687813021703, 53.24755)(7.712854757929884, 52.583479999999994)(7.813021702838064, 52.87584)(7.913188647746244, 52.32226)(8.013355592654424, 52.56884)(8.113522537562606, 52.81481)(8.213689482470786, 52.7025)(8.313856427378965, 52.00975)(8.414023372287145, 52.385729999999995)(8.514190317195327, 52.491929999999996)(8.614357262103507, 52.62864999999999)(8.714524207011687, 53.2799)(8.814691151919867, 52.96983)(8.914858096828048, 53.10412)(9.015025041736228, 53.66503)(9.115191986644408, 52.307610000000004)(9.215358931552588, 53.16332)(9.31552587646077, 52.993030000000005)(9.41569282136895, 52.91491)(9.51585976627713, 51.744260000000004)(9.61602671118531, 53.23595)(9.71619365609349, 53.080920000000006)(9.81636060100167, 52.928940000000004)(9.916527545909851, 52.328359999999996)(10.016694490818029, 53.54052)(10.11686143572621, 52.61461)(10.21702838063439, 53.14134)(10.31719532554257, 53.02477)(10.41736227045075, 53.384879999999995)(10.51752921535893, 53.14989)(10.617696160267112, 52.642689999999995)(10.717863105175292, 52.82152)(10.818030050083472, 52.74155999999999)(10.918196994991652, 52.993030000000005)(11.018363939899833, 53.282939999999996)(11.118530884808013, 53.05955)(11.218697829716193, 53.01012)(11.318864774624373, 52.35583)(11.419031719532555, 52.93992)(11.519198664440735, 52.626819999999995)(11.619365609348915, 53.118759999999995)(11.719532554257095, 53.02049)(11.819699499165276, 53.24877)(11.919866444073456, 53.00767999999999)(12.020033388981636, 52.947860000000006)(12.120200333889816, 52.98571)(12.220367278797998, 53.22069)(12.320534223706177, 53.301869999999994)(12.420701168614357, 53.01012)(12.520868113522537, 53.35618)(12.621035058430719, 53.33238)(12.721202003338899, 52.560900000000004)(12.821368948247079, 53.047959999999996)(12.921535893155259, 53.06749)(13.021702838063439, 52.52672)(13.12186978297162, 53.122420000000005)(13.2220367278798, 53.11327)(13.32220367278798, 52.836780000000005)(13.42237061769616, 53.42821)(13.522537562604342, 53.17736)(13.622704507512521, 53.021710000000006)(13.722871452420701, 52.90147)(13.823038397328881, 52.319810000000004)(13.923205342237063, 53.42455)(14.023372287145243, 53.43675)(14.123539232053423, 53.425149999999995)(14.223706176961603, 52.79527)(14.323873121869784, 53.65831)(14.424040066777964, 53.355579999999996)(14.524207011686144, 52.91308)(14.624373956594324, 53.314069999999994)(14.724540901502506, 53.547219999999996)(14.824707846410686, 52.97533)(14.924874791318866, 53.25792)(15.025041736227045, 53.24693)(15.125208681135225, 52.99059)(15.225375626043405, 52.65183999999999)(15.325542570951589, 53.70103)(15.425709515859769, 53.22618)(15.525876460767948, 53.29576)(15.626043405676128, 52.22887)(15.726210350584308, 53.33178)(15.826377295492488, 53.45872)(15.926544240400668, 53.458119999999994)(16.026711185308848, 52.820910000000005)(16.126878130217026, 53.046119999999995)(16.22704507512521, 52.2185)(16.32721202003339, 53.594229999999996)(16.42737896494157, 53.110220000000005)(16.52754590984975, 52.786730000000006)(16.62771285475793, 53.00707)(16.72787979966611, 53.400740000000006)(16.82804674457429, 52.962509999999995)(16.92821368948247, 53.301869999999994)(17.028380634390654, 53.32628)(17.128547579298832, 52.45531)(17.228714524207014, 52.37902)(17.328881469115192, 53.515480000000004)(17.429048414023374, 52.59875)(17.529215358931552, 53.301249999999996)(17.629382303839733, 53.312239999999996)(17.72954924874791, 53.28844)(17.829716193656097, 53.37084)(17.929883138564275, 53.498999999999995)(18.030050083472457, 53.46727)(18.130217028380635, 52.89842)(18.230383973288816, 53.409279999999995)(18.330550918196995, 52.68908)(18.430717863105176, 53.6461)(18.530884808013354, 52.21239)(18.63105175292154, 53.31041)(18.731218697829718, 53.055899999999994)(18.8313856427379, 51.97375)(18.931552587646078, 52.91124)(19.03171953255426, 52.97533)(19.131886477462437, 52.732409999999994)(19.23205342237062, 53.5338)(19.332220367278797, 53.2805)(19.43238731218698, 52.732409999999994)(19.53255425709516, 52.81237)(19.63272120200334, 53.26464)(19.73288814691152, 53.11632)(19.833055091819702, 53.35863)(19.93322203672788, 53.68822)(20.033388981636058, 53.16637)(20.13355592654424, 53.5338)(20.23372287145242, 53.5637)(20.333889816360603, 53.39586)(20.43405676126878, 53.445910000000005)(20.534223706176963, 52.88377)(20.63439065108514, 53.5338)(20.734557595993323, 53.53929)(20.8347245409015, 52.88195)(20.934891485809683, 52.970439999999996)(21.03505843071786, 54.07579)(21.135225375626046, 53.54112)(21.235392320534224, 53.54356)(21.335559265442406, 53.03026)(21.435726210350584, 52.3839)(21.535893155258766, 53.2451)(21.636060100166944, 52.96433999999999)(21.736227045075125, 52.697010000000006)(21.836393989983303, 53.03331)(21.93656093489149, 53.28539)(22.036727879799667, 53.418440000000004)(22.13689482470785, 53.0028)(22.237061769616027, 52.66649)(22.33722871452421, 52.958239999999996)(22.437395659432386, 52.299670000000006)(22.537562604340568, 53.31041)(22.637729549248746, 51.851060000000004)(22.737896494156928, 52.66894)(22.83806343906511, 52.79222)(22.93823038397329, 52.05919)(23.03839732888147, 53.07054)(23.13856427378965, 53.12975)(23.23873121869783, 53.000350000000005)(23.33889816360601, 53.26158)(23.43906510851419, 52.95091)(23.53923205342237, 52.96861)(23.639398998330552, 53.13829)(23.739565943238734, 53.2805)(23.839732888146912, 53.38793)(23.939899833055094, 53.44652)(24.040066777963272, 53.42821)(24.140233722871454, 52.511469999999996)(24.24040066777963, 53.13097)(24.340567612687813, 52.97777)(24.440734557595995, 52.86059)(24.540901502504177, 52.23681)(24.641068447412355, 53.03086999999999)(24.741235392320537, 52.60301)(24.841402337228715, 53.56676)(24.941569282136896, 51.270619999999994)(25.041736227045075, 52.895979999999994)(25.141903171953256, 52.939930000000004)(25.242070116861438, 53.17614)(25.34223706176962, 52.599349999999994)(25.442404006677798, 53.1151)(25.54257095158598, 53.08702)(25.642737896494157, 53.40319)(25.74290484140234, 53.27745)(25.843071786310517, 53.4514)(25.9432387312187, 52.70678)(26.043405676126877, 53.34764)(26.143572621035062, 53.234719999999996)(26.24373956594324, 52.79283)(26.34390651085142, 53.035759999999996)(26.4440734557596, 53.594840000000005)(26.544240400667782, 53.09984)(26.64440734557596, 52.77208)(26.744574290484138, 53.65526)(26.84474123539232, 52.8435)(26.9449081803005, 53.24083)(27.045075125208683, 51.99389)(27.14524207011686, 53.21825)(27.245409015025043, 53.44042)(27.34557595993322, 53.24266)(27.445742904841403, 53.425760000000004)(27.54590984974958, 53.682109999999994)(27.646076794657763, 52.53648)(27.746243739565944, 52.9674)(27.846410684474126, 53.56554)(27.946577629382304, 52.82701)(28.046744574290486, 53.53624)(28.146911519198664, 53.193830000000005)(28.247078464106846, 51.98901)(28.347245409015024, 53.17674)(28.447412353923205, 53.30675)(28.547579298831387, 52.15135)(28.64774624373957, 52.86181)(28.747913188647747, 53.81028)(28.84808013355593, 52.944199999999995)(28.948247078464107, 53.29027)(29.04841402337229, 53.20909)(29.148580968280466, 52.90147)(29.248747913188648, 53.27562)(29.348914858096826, 52.61522)(29.44908180300501, 52.9088)(29.54924874791319, 52.45653)(29.64941569282137, 52.741569999999996)(29.74958263772955, 53.40989999999999)(29.84974958263773, 53.22923)(29.94991652754591, 53.23961)(30.05008347245409, 53.736439999999995)(30.15025041736227, 54.05748)(30.25041736227045, 52.981429999999996)(30.35058430717863, 52.82335)(30.45075125208681, 52.76903)(30.55091819699499, 53.15355)(30.651085141903177, 52.62681)(30.751252086811355, 53.17674)(30.851419031719537, 53.44896)(30.951585976627715, 53.386709999999994)(31.051752921535897, 52.82946)(31.151919866444075, 53.478260000000006)(31.252086811352257, 53.108380000000004)(31.352253756260435, 53.328720000000004)(31.452420701168617, 53.811499999999995)(31.552587646076795, 52.80138)(31.652754590984976, 52.69274)(31.752921535893154, 53.65831)(31.853088480801336, 52.64146)(31.953255425709514, 53.71508)(32.053422370617696, 53.20482)(32.15358931552588, 53.18711999999999)(32.25375626043405, 52.99669)(32.35392320534224, 53.33727)(32.45409015025042, 53.669900000000005)(32.554257095158604, 53.03148)(32.65442404006678, 52.351549999999996)(32.75459098497496, 53.48802)(32.85475792988314, 53.29638)(32.95492487479132, 52.797720000000005)(33.0550918196995, 52.70982)(33.15525876460768, 53.224349999999994)(33.25542570951586, 52.55663)(33.35559265442404, 52.3247)(33.45575959933222, 53.81700000000001)(33.5559265442404, 52.978379999999994)(33.65609348914858, 53.0272)(33.756260434056756, 53.30858)(33.85642737896494, 53.14622)(33.95659432387313, 53.34031)(34.05676126878131, 52.78734)(34.15692821368948, 53.29638)(34.257095158597664, 53.2036)(34.357262103505846, 53.50023)(34.45742904841403, 53.41112)(34.5575959933222, 53.48253)(34.657762938230384, 52.977160000000005)(34.757929883138566, 52.412580000000005)(34.85809682804675, 53.51549)(34.95826377295492, 53.4038)(35.058430717863104, 52.77208)(35.158597662771285, 53.5869)(35.25876460767947, 53.53136)(35.35893155258764, 53.57531)(35.45909849749582, 52.64941)(35.559265442404005, 53.155989999999996)(35.659432387312194, 53.257310000000004)(35.75959933222037, 53.87255)(35.85976627712855, 53.71996)(35.95993322203673, 53.47642999999999)(36.06010016694491, 52.78489999999999)(36.16026711185309, 53.18224)(36.26043405676127, 52.251450000000006)(36.36060100166945, 53.894510000000004)(36.46076794657763, 53.22679)(36.56093489148581, 53.33849)(36.66110183639399, 53.172470000000004)(36.76126878130217, 53.17369)(36.86143572621035, 53.50084)(36.96160267111853, 52.42479)(37.06176961602671, 53.52280999999999)(37.16193656093489, 53.38732)(37.26210350584308, 52.50292)(37.362270450751254, 53.31774)(37.462437395659435, 53.323840000000004)(37.56260434056762, 53.50450000000001)(37.6627712854758, 53.12791)(37.76293823038397, 52.742180000000005)(37.863105175292155, 53.58446)(37.96327212020034, 53.45873)(38.06343906510852, 53.77794)(38.16360601001669, 53.61802)(38.263772954924875, 53.49657)(38.363939899833056, 53.57896)(38.46410684474124, 53.2744)(38.56427378964941, 53.53197)(38.664440734557594, 52.59447)(38.764607679465776, 52.983869999999996)(38.86477462437396, 53.677229999999994)(38.96494156928214, 53.38304)(39.06510851419032, 51.91454)(39.1652754590985, 53.111439999999995)(39.26544240400668, 52.78734)(39.36560934891486, 53.45384)(39.46577629382304, 52.38145)(39.56594323873122, 53.693099999999994)(39.666110183639404, 52.81237)(39.76627712854758, 53.06505)(39.86644407345576, 53.29454)(39.96661101836394, 52.88805)(40.066777963272116, 53.53624)(40.1669449081803, 53.018660000000004)(40.26711185308848, 53.007059999999996)(40.36727879799666, 53.47215)(40.46744574290484, 52.8502)(40.567612687813025, 53.278059999999996)(40.667779632721206, 53.12975)(40.76794657762939, 53.246320000000004)(40.86811352253756, 53.30736)(40.968280467445744, 53.609489999999994)(41.068447412353926, 53.541740000000004)(41.16861435726211, 53.39036)(41.26878130217028, 53.84263)(41.368948247078464, 53.1682)(41.469115191986646, 53.020500000000006)(41.56928213689483, 53.71813)(41.669449081803, 53.31347)(41.769616026711184, 53.46971)(41.869782971619365, 53.8347)(41.96994991652755, 52.41442)(42.07011686143572, 53.58933999999999)(42.1702838063439, 53.16332)(42.27045075125209, 53.13402)(42.370617696160274, 53.21092)(42.47078464106845, 52.97899)(42.57095158597663, 53.895739999999996)(42.67111853088481, 53.21458)(42.77128547579299, 53.156)(42.87145242070117, 52.87157)(42.97161936560935, 53.413560000000004)(43.07178631051753, 52.801989999999996)(43.17195325542571, 53.39891)(43.27212020033389, 52.78002)(43.37228714524207, 53.62047)(43.47245409015025, 53.104110000000006)(43.57262103505843, 53.59057)(43.67278797996661, 53.42821)(43.77295492487479, 53.643660000000004)(43.87312186978298, 52.76903)(43.97328881469116, 53.959830000000004)(44.073455759599334, 53.682109999999994)(44.173622704507515, 52.480940000000004)(44.2737896494157, 53.31346)(44.37395659432388, 53.06139)(44.47412353923205, 52.86058)(44.574290484140235, 52.79101)(44.67445742904842, 53.66746)(44.7746243739566, 53.27196)(44.87479131886477, 52.13243)(44.974958263772955, 53.23839)(45.075125208681136, 52.86912)(45.17529215358932, 52.8264)(45.27545909849749, 53.634510000000006)(45.375626043405674, 52.87889)(45.475792988313856, 53.50634)(45.57595993322204, 53.23961)(45.67612687813022, 53.462990000000005)(45.7762938230384, 52.1123)(45.87646076794658, 53.643660000000004)(45.976627712854764, 54.07335)(46.07679465776294, 53.041850000000004)(46.17696160267112, 53.55028)(46.2771285475793, 52.990579999999994)(46.37729549248748, 52.62804)(46.47746243739566, 53.38487)(46.57762938230384, 53.05773000000001)(46.67779632721202, 53.385490000000004)(46.7779632721202, 52.39061)(46.87813021702838, 53.416)(46.97829716193656, 53.542959999999994)(47.07846410684474, 52.90453)(47.17863105175292, 53.03575000000001)(47.278797996661105, 52.999739999999996)(47.378964941569286, 52.9796)(47.47913188647747, 53.70286)(47.57929883138564, 53.76146)(47.679465776293824, 52.836169999999996)(47.779632721202006, 53.0736)(47.87979966611019, 53.29577)(47.97996661101836, 53.17614)(48.080133555926544, 53.53135)(48.180300500834726, 53.41722)(48.28046744574291, 53.80968)(48.38063439065108, 53.60704)(48.48080133555926, 53.2744)(48.580968280467445, 53.64183)(48.68113522537563, 53.587509999999995)(48.7813021702838, 53.68334)(48.88146911519199, 53.07177)(48.98163606010017, 53.59911)(49.081803005008354, 53.31225)(49.18196994991653, 53.46666)(49.28213689482471, 53.33605)(49.38230383973289, 53.46727)(49.48247078464107, 53.20788)(49.58263772954925, 53.9348)(49.68280467445743, 53.4044)(49.78297161936561, 53.18407)(49.88313856427379, 52.40282)(49.98330550918197, 53.09252)(50.08347245409015, 53.56065)(50.18363939899833, 53.29332)(50.28380634390651, 53.639390000000006)(50.38397328881469, 53.48863)(50.484140233722876, 53.63571999999999)(50.58430717863106, 53.12853)(50.68447412353924, 53.60948)(50.784641068447414, 53.296369999999996)(50.884808013355595, 52.68846)(50.98497495826378, 53.80052)(51.08514190317196, 53.70897)(51.18530884808013, 53.259150000000005)(51.285475792988315, 53.43308999999999)(51.3856427378965, 54.048930000000006)(51.48580968280468, 53.38182)(51.58597662771285, 53.05955)(51.686143572621035, 53.715070000000004)(51.786310517529216, 53.31163)(51.8864774624374, 53.72972)(51.98664440734557, 53.987899999999996)(52.086811352253754, 52.99119)(52.18697829716194, 53.05346)(52.287145242070125, 53.759629999999994)(52.3873121869783, 53.614369999999994)(52.48747913188648, 52.49926000000001)(52.58764607679466, 53.42455)(52.68781302170284, 53.708349999999996)(52.78797996661102, 53.30919)(52.8881469115192, 53.324439999999996)(52.98831385642738, 54.0172)(53.088480801335564, 53.19627)(53.18864774624374, 52.560289999999995)(53.28881469115192, 53.7108)(53.3889816360601, 53.461169999999996)(53.489148580968276, 52.73607)(53.58931552587646, 53.059560000000005)(53.68948247078464, 53.14256)(53.78964941569283, 52.93016)(53.889816360601, 53.62963)(53.989983305509185, 54.21983)(54.090150250417366, 53.0266)(54.19031719532555, 53.60948)(54.29048414023372, 52.91246)(54.390651085141904, 53.11571)(54.490818030050086, 53.58873)(54.59098497495827, 52.81236)(54.69115191986644, 53.126689999999996)(54.791318864774624, 53.76939)(54.891485809682806, 53.31468)(54.99165275459099, 53.53319)(55.09181969949916, 53.7639)(55.19198664440734, 53.369609999999994)(55.292153589315525, 53.17308)(55.39232053422371, 52.61339)(55.49248747913189, 53.86339)(55.59265442404007, 53.92015)(55.69282136894825, 53.20787)(55.79298831385643, 53.569199999999995)(55.89315525876461, 52.59447)(55.99332220367279, 53.53441)(56.09348914858097, 53.19628)(56.19365609348915, 53.61009)(56.29382303839733, 53.07665)(56.39398998330551, 53.67905999999999)(56.49415692821369, 53.574690000000004)(56.59432387312187, 53.590559999999996)(56.69449081803005, 53.729110000000006)(56.79465776293823, 53.71629)(56.89482470784641, 52.764759999999995)(56.99499165275459, 53.23777)(57.095158597662774, 53.67846)(57.195325542570956, 53.583239999999996)(57.29549248747914, 52.87461999999999)(57.39565943238732, 53.81822)(57.495826377295494, 53.036370000000005)(57.595993322203675, 53.273179999999996)(57.69616026711186, 53.74253)(57.79632721202004, 53.72239999999999)(57.89649415692821, 52.858140000000006)(57.996661101836395, 53.74253)(58.09682804674458, 52.825179999999996)(58.19699499165276, 52.876459999999994)(58.29716193656093, 53.18162)(58.397328881469114, 53.78038)(58.497495826377296, 53.53807)(58.59766277128548, 53.395250000000004)(58.69782971619365, 53.81761)(58.79799666110184, 54.0117)(58.89816360601002, 53.35924)(58.9983305509182, 53.25304)(59.09849749582638, 53.84996)(59.19866444073456, 53.58629)(59.29883138564274, 53.89452)(59.398998330550924, 53.81455)(59.4991652754591, 53.21030999999999)(59.59933222036728, 52.88439)(59.69949916527546, 54.194199999999995)(59.79966611018364, 53.959210000000006)(59.89983305509182, 53.7877)(60.0, 53.270739999999996)
    };
    \addplot [pattern=north east lines,pattern color=red] 
    fill between [
        of=A and B,soft clip={domain=0:800},
    ];
    \end{axis}
\end{tikzpicture}
\caption{Measuring instrument: RAPL}
\end{subfigure}
\begin{subfigure}[b]{0.49\linewidth}
    \begin{tikzpicture}
        \pgfplotsset{%
        width=1\linewidth,
        % height=1\textheight
        }
        \begin{axis}[ymax=120,
            xlabel={Time (Seconds)},
            ylabel={Energy Consumption (Joules)},
            ]
            \addplot[color=blue, mark=none,] coordinates { %% AVG value
            (0.0, 100.37889236963335)(0.07585335018963338, 100.95284488900829)(0.15170670037926676, 99.38019873163329)(0.22756005056890014, 100.06774462213329)(0.3034134007585335, 98.87162899073336)(0.3792667509481669, 99.67785048030838)(0.45512010113780027, 99.56084242255841)(0.5309734513274336, 99.7449368969417)(0.606826801517067, 99.86876449873338)(0.6826801517067005, 100.01279181698332)(0.7585335018963338, 100.147869974275)(0.8343868520859672, 99.96512701021668)(0.9102402022756005, 100.34249799132502)(0.986093552465234, 102.00805450040835)(1.0619469026548671, 100.92995972592502)(1.1378002528445006, 100.36306518315831)(1.213653603034134, 100.04565445972496)(1.2895069532237675, 100.81487036586671)(1.365360303413401, 100.11628446832499)(1.4412136536030342, 101.69691850024168)(1.5170670037926677, 101.31044698388334)(1.5929203539823011, 101.0117573987)(1.6687737041719344, 101.43234975628334)(1.7446270543615676, 101.80761972599994)(1.820480404551201, 101.3334094071167)(1.8963337547408345, 101.95310296035832)(1.972187104930468, 101.26157728348333)(2.0480404551201015, 101.8616435039)(2.1238938053097343, 103.85079909473339)(2.199747155499368, 101.69040725376665)(2.275600505689001, 102.29495487150834)(2.351453855878635, 102.59268100125834)(2.427307206068268, 100.74851934080836)(2.503160556257902, 102.08242080554167)(2.579013906447535, 100.84238606690002)(2.6548672566371687, 99.54830230374169)(2.730720606826802, 102.59854133085832)(2.806573957016435, 100.86156129285834)(2.8824273072060684, 101.14149314200833)(2.9582806573957017, 101.38978883542505)(3.0341340075853354, 101.1916099627084)(3.1099873577749686, 101.59966241869168)(3.1858407079646023, 101.20239734973335)(3.2616940581542355, 102.62255634061673)(3.3375474083438688, 100.97211814567491)(3.413400758533502, 100.87365768784166)(3.4892541087231352, 100.30772015881666)(3.565107458912769, 102.07840568850004)(3.640960809102402, 100.85159840817492)(3.716814159292036, 101.69094646052501)(3.792667509481669, 100.99941817916667)(3.8685208596713023, 99.80353915249174)(3.944374209860936, 101.54118304853331)(4.020227560050569, 102.12608433555)(4.096080910240203, 101.47864817771668)(4.171934260429836, 102.05046180576666)(4.2477876106194685, 100.39209386503335)(4.323640960809103, 101.58678732156667)(4.399494310998736, 101.98431482712503)(4.47534766118837, 100.93788455699169)(4.551201011378002, 101.14876848860838)(4.6270543615676365, 101.13229910549167)(4.70290771175727, 100.97058293125838)(4.778761061946904, 101.68371422809163)(4.854614412136536, 100.29466014919998)(4.9304677623261695, 100.52895264624165)(5.006321112515804, 100.76745570849165)(5.082174462705436, 100.12713216671665)(5.15802781289507, 100.44459291974165)(5.233881163084703, 100.84408143347498)(5.309734513274337, 100.68733058518335)(5.38558786346397, 101.25003730800003)(5.461441213653604, 100.94706704184166)(5.537294563843237, 100.71068009239164)(5.61314791403287, 101.75542462762503)(5.689001264222504, 100.41876455163332)(5.764854614412137, 101.0886892311083)(5.840707964601771, 103.46717915292507)(5.916561314791403, 102.16914838245833)(5.9924146649810375, 102.41332814112496)(6.068268015170671, 103.58567445145836)(6.144121365360304, 102.45327656024165)(6.219974715549937, 99.94676949729164)(6.29582806573957, 99.18279484961663)(6.371681415929205, 97.55673651600002)(6.447534766118837, 96.71006687999167)(6.523388116308471, 98.05094662172499)(6.599241466498104, 98.67230256260001)(6.6750948166877375, 100.96630548199997)(6.750948166877371, 101.65736019713331)(6.826801517067004, 102.11936383500006)(6.902654867256638, 101.54098320845002)(6.9785082174462705, 99.29412797055826)(7.054361567635905, 98.7285469696833)(7.130214917825538, 99.77504023663332)(7.206068268015172, 99.5427505993333)(7.281921618204804, 99.58946312961668)(7.357774968394438, 99.99197149142493)(7.433628318584072, 99.05404864100834)(7.509481668773706, 100.03326573450838)(7.585335018963338, 97.98191944610834)(7.661188369152971, 99.04914936685836)(7.737041719342605, 99.25395980027496)(7.812895069532239, 97.94405429924169)(7.888748419721872, 97.75493349825003)(7.964601769911505, 99.22170459339999)(8.040455120101138, 97.93405705215007)(8.116308470290772, 98.40691203615837)(8.192161820480406, 98.62251274515005)(8.268015170670038, 97.74308640513331)(8.343868520859672, 100.15367227253336)(8.419721871049305, 99.16894648503329)(8.495575221238937, 99.57124773267493)(8.571428571428573, 99.13401443474166)(8.647281921618205, 98.69008649283336)(8.72313527180784, 99.17913059273336)(8.798988621997472, 99.21305284207499)(8.874841972187106, 100.33342928900838)(8.95069532237674, 99.64062301041672)(9.026548672566372, 100.09436864403332)(9.102402022756005, 99.99775355080833)(9.178255372945639, 101.02954562374998)(9.254108723135273, 100.95359616559998)(9.329962073324905, 101.09978388249996)(9.40581542351454, 100.64766433253337)(9.481668773704172, 100.59908330702498)(9.557522123893808, 99.61118698040833)(9.63337547408344, 100.53944181620002)(9.709228824273072, 99.92570071171666)(9.785082174462707, 99.64217593202498)(9.860935524652339, 101.07149637917503)(9.936788874841973, 100.86979607093335)(10.012642225031607, 99.00312243391664)(10.08849557522124, 99.1947861435)(10.164348925410872, 99.76071511968334)(10.240202275600508, 98.17158801450002)(10.31605562579014, 99.7184600973917)(10.391908975979774, 97.87622153579169)(10.467762326169407, 99.43959133824166)(10.543615676359039, 98.37695924999167)(10.619469026548675, 98.53110985484996)(10.695322376738307, 99.35488823520002)(10.77117572692794, 98.88037926400834)(10.847029077117574, 98.794795083225)(10.922882427307208, 97.70202325659167)(10.99873577749684, 97.9955086634083)(11.074589127686474, 98.8389334311083)(11.150442477876107, 98.39531741804166)(11.22629582806574, 98.4667706837666)(11.302149178255375, 99.94070723306662)(11.378002528445007, 98.41413695179992)(11.453855878634641, 98.15503105892502)(11.529709228824274, 98.90436528013332)(11.605562579013906, 98.6701875716917)(11.681415929203542, 98.48510808379169)(11.757269279393174, 98.25607963550004)(11.833122629582807, 98.13877828116665)(11.90897597977244, 98.33211773832497)(11.984829329962075, 98.55816257579167)(12.060682680151707, 98.96019402494997)(12.136536030341341, 98.74023866935)(12.212389380530974, 98.93938192114999)(12.288242730720608, 97.81629987810835)(12.364096080910242, 98.68037585837502)(12.439949431099874, 98.91100116310832)(12.515802781289509, 99.60750211601666)(12.59165613147914, 99.70489886679171)(12.667509481668775, 99.22527701761669)(12.74336283185841, 98.8128449318)(12.819216182048041, 99.1687860308917)(12.895069532237674, 99.05717773799164)(12.970922882427308, 99.19818432790002)(13.046776232616942, 99.39455849838333)(13.122629582806574, 99.3636039966834)(13.198482932996209, 100.03373000100835)(13.274336283185841, 99.55640499237501)(13.350189633375475, 99.02421198903333)(13.42604298356511, 101.58459998470829)(13.501896333754742, 100.60426086912499)(13.577749683944376, 101.23890326912496)(13.653603034134008, 100.67138596125)(13.729456384323642, 99.45592977519166)(13.805309734513276, 98.97946260938328)(13.881163084702909, 96.97811819059997)(13.957016434892541, 96.70698251580833)(14.032869785082175, 97.63235425468338)(14.10872313527181, 98.18883599536667)(14.184576485461443, 100.42655941006663)(14.260429835651076, 101.59347065692496)(14.336283185840708, 101.94261899895831)(14.412136536030344, 100.88104425331667)(14.487989886219976, 100.61467248184165)(14.563843236409609, 99.45279364747502)(14.639696586599243, 98.33981174430836)(14.715549936788875, 98.7818307898167)(14.79140328697851, 98.91052180090001)(14.867256637168143, 97.81667185910834)(14.943109987357776, 99.58704671627498)(15.018963337547412, 95.96011315953332)(15.094816687737044, 96.46961547094165)(15.170670037926676, 95.98743091425006)(15.24652338811631, 96.06569436129166)(15.322376738305943, 96.31332029301667)(15.398230088495575, 96.15143225800831)(15.47408343868521, 94.97183025845835)(15.549936788874842, 94.65807016271665)(15.625790139064478, 95.16656288826668)(15.701643489254112, 95.01589407122508)(15.777496839443744, 94.41175057055003)(15.853350189633378, 95.79412003830831)(15.92920353982301, 95.83405699286666)(16.005056890012643, 96.27802194988334)(16.080910240202275, 97.04179696926666)(16.15676359039191, 98.86770159039163)(16.232616940581543, 96.87628667550003)(16.30847029077118, 96.98202626872502)(16.38432364096081, 97.84498496910001)(16.460176991150444, 97.40465487536663)(16.536030341340076, 98.83263625519172)(16.611883691529712, 97.77177593063331)(16.687737041719345, 99.02600971747498)(16.763590391908977, 98.88561251343337)(16.83944374209861, 99.12386208995832)(16.91529709228824, 99.10069797060001)(16.991150442477874, 100.05186084134995)(17.067003792667514, 99.27034555889996)(17.142857142857146, 99.51291010533336)(17.21871049304678, 99.83417956910829)(17.29456384323641, 98.87352412692503)(17.370417193426043, 99.8163815979333)(17.44627054361568, 100.06469978514164)(17.52212389380531, 98.10525960470835)(17.597977243994944, 97.72780073215836)(17.673830594184576, 97.16306171540835)(17.749683944374212, 95.85551901738334)(17.825537294563844, 97.43737989027504)(17.90139064475348, 97.27336466759171)(17.977243994943112, 96.52109276308333)(18.053097345132745, 97.19723453464167)(18.128950695322377, 95.16502004425834)(18.20480404551201, 98.27165929828332)(18.280657395701645, 97.71480330183334)(18.356510745891278, 95.96817029160836)(18.432364096080914, 98.40094842572496)(18.508217446270546, 94.60095326939168)(18.58407079646018, 95.30567174090002)(18.65992414664981, 94.02218009497491)(18.735777496839447, 95.3860694466667)(18.81163084702908, 95.17681051660001)(18.88748419721871, 95.01090285723333)(18.963337547408344, 95.52826062187498)(19.039190897597976, 96.70707048645833)(19.115044247787615, 94.43632051973329)(19.190897597977248, 94.96290413171671)(19.26675094816688, 94.74802464783333)(19.342604298356513, 93.87591875505832)(19.418457648546145, 94.361679337825)(19.494310998735777, 95.16215263633332)(19.570164348925413, 94.96257525630831)(19.646017699115045, 93.76106523854999)(19.721871049304678, 94.89007083972498)(19.797724399494314, 93.28453143218334)(19.873577749683946, 94.83701553968332)(19.94943109987358, 95.5396891830583)(20.025284450063214, 94.812542205225)(20.101137800252847, 95.55392900743333)(20.17699115044248, 95.24158970789999)(20.25284450063211, 94.57894546416667)(20.328697850821744, 94.870253680275)(20.40455120101138, 94.86580967455001)(20.480404551201016, 94.92734979715001)(20.556257901390648, 93.91642014680832)(20.63211125158028, 93.75520195635835)(20.707964601769913, 94.2571533675917)(20.78381795195955, 93.46730503419165)(20.85967130214918, 94.82419292155)(20.935524652338813, 94.74097988991666)(21.011378002528446, 96.00050548812494)(21.087231352718078, 96.58616878996659)(21.163084702907714, 97.12858394097498)(21.23893805309735, 94.89907697105829)(21.314791403286982, 96.29059675658335)(21.390644753476614, 95.72376549595833)(21.466498103666247, 95.55377246780834)(21.54235145385588, 97.5675567104)(21.61820480404551, 98.27483799064996)(21.694058154235147, 98.20867161630834)(21.76991150442478, 101.69715558615835)(21.845764854614416, 100.79908232660836)(21.921618204804048, 101.05046285355837)(21.99747155499368, 101.32394590426667)(22.073324905183316, 101.270576548575)(22.14917825537295, 102.35298718189165)(22.22503160556258, 101.92421470010837)(22.300884955752213, 101.25829003287502)(22.376738305941846, 101.47744164247504)(22.45259165613148, 102.33307232201668)(22.528445006321114, 99.33043218211665)(22.60429835651075, 98.20757973082502)(22.680151706700382, 97.92677215076665)(22.756005056890015, 98.08796861688333)(22.831858407079647, 97.4921477134666)(22.907711757269283, 98.23486549539997)(22.983565107458915, 97.72817113785837)(23.059418457648547, 97.89878823525832)(23.13527180783818, 98.78984147243334)(23.211125158027812, 98.07490009860837)(23.286978508217448, 97.67709390242499)(23.362831858407084, 98.71477985843333)(23.438685208596716, 98.18665559067497)(23.51453855878635, 99.25982744755834)(23.59039190897598, 99.95630285847503)(23.666245259165613, 99.44557102152501)(23.74209860935525, 99.48345118024166)(23.81795195954488, 99.63319315508333)(23.893805309734514, 100.38198106289163)(23.96965865992415, 99.1259711074666)(24.045512010113782, 99.54338898862503)(24.121365360303415, 99.37949732935833)(24.19721871049305, 99.67702111123332)(24.273072060682683, 100.16133733673331)(24.348925410872315, 100.10028016396672)(24.424778761061948, 100.64819876816665)(24.50063211125158, 100.38043241955833)(24.576485461441216, 100.72116439156659)(24.65233881163085, 100.58279277339999)(24.728192161820484, 100.49593836471665)(24.804045512010116, 100.95370973798337)(24.87989886219975, 99.58961006585837)(24.95575221238938, 100.59624311562501)(25.031605562579017, 99.56320632108333)(25.10745891276865, 99.88607204061664)(25.18331226295828, 99.67829926216668)(25.259165613147914, 97.714808774075)(25.33501896333755, 99.31518615689167)(25.410872313527186, 99.12884494505005)(25.48672566371682, 98.96638240595838)(25.56257901390645, 98.53268995585833)(25.638432364096083, 98.25537983325005)(25.714285714285715, 99.24621778286664)(25.790139064475348, 99.10223375836662)(25.865992414664984, 99.08013090643333)(25.941845764854616, 96.92317626847499)(26.017699115044252, 97.8875415200084)(26.093552465233884, 98.22952899479166)(26.169405815423517, 97.09095872545)(26.24525916561315, 98.03608595266664)(26.321112515802785, 97.6065695186)(26.396965865992417, 97.23697473124167)(26.47281921618205, 97.0460956358667)(26.548672566371682, 97.37133024064163)(26.624525916561314, 97.9303813458)(26.70037926675095, 98.60487764194173)(26.776232616940586, 97.56190189430836)(26.85208596713022, 99.01794973769165)(26.92793931731985, 97.42465023739169)(27.003792667509483, 99.46151546611665)(27.07964601769912, 99.80110428724996)(27.15549936788875, 99.13377147796668)(27.231352718078384, 99.5737510126417)(27.307206068268016, 99.69126373325837)(27.38305941845765, 100.7080444911917)(27.458912768647284, 100.86444465257499)(27.53476611883692, 99.72932409415833)(27.610619469026553, 99.00726187701663)(27.686472819216185, 99.95204568288337)(27.762326169405817, 99.99192903509999)(27.83817951959545, 100.27906528395835)(27.914032869785082, 100.38150515124165)(27.989886219974718, 99.74988608536661)(28.06573957016435, 100.96881660764164)(28.141592920353986, 100.830219504)(28.21744627054362, 99.05930992547498)(28.29329962073325, 101.76166254769164)(28.369152970922887, 99.95589435454168)(28.44500632111252, 101.6106772194333)(28.52085967130215, 101.41993161665837)(28.596713021491784, 98.63618095610839)(28.672566371681416, 100.72014337369997)(28.748419721871052, 97.94834603314166)(28.824273072060688, 96.40377994720001)(28.90012642225032, 96.88855970844162)(28.975979772439953, 97.91980434894164)(29.051833122629585, 96.774203224425)(29.127686472819217, 97.84453031008337)(29.203539823008853, 100.10168172097502)(29.279393173198486, 99.72606113129999)(29.355246523388118, 101.09917655327494)(29.43109987357775, 97.82896211708335)(29.506953223767386, 98.36789587351669)(29.58280657395702, 97.53219484175834)(29.658659924146654, 97.44814382459167)(29.734513274336287, 98.77649167796663)(29.81036662452592, 98.90466103380002)(29.88621997471555, 98.76229762424168)(29.962073324905184, 98.56929647027499)(30.037926675094823, 96.67690247425837)(30.113780025284452, 99.31522651108328)(30.189633375474088, 98.73237611473334)(30.265486725663717, 97.1850860250167)(30.341340075853353, 98.17973798164995)(30.417193426042985, 98.39815629794997)(30.49304677623262, 97.59053405189174)(30.568900126422257, 100.09719758934992)(30.644753476611886, 97.84975135312497)(30.72060682680152, 97.71474373441666)(30.79646017699115, 99.30215527730005)(30.872313527180786, 97.29946279255836)(30.94816687737042, 97.94778157307499)(31.024020227560055, 98.95184106224164)(31.099873577749683, 98.48520807584165)(31.17572692793932, 99.51024703908332)(31.251580278128955, 99.1457688013)(31.327433628318587, 98.47037321693338)(31.403286978508223, 98.5218984519417)(31.479140328697852, 98.82136188854166)(31.554993678887488, 99.14324558112499)(31.630847029077117, 99.84888239406665)(31.706700379266756, 99.23980487354169)(31.782553729456385, 99.04587600088333)(31.85840707964602, 99.41322001149166)(31.934260429835657, 99.24476232905836)(32.010113780025286, 99.04393199800829)(32.085967130214925, 99.90089411451662)(32.16182048040455, 100.71694433496667)(32.23767383059419, 98.91529487765835)(32.31352718078382, 99.50156286375831)(32.389380530973455, 98.98384357628335)(32.46523388116309, 99.99291347445002)(32.54108723135272, 100.22385667454174)(32.61694058154236, 99.5281819468583)(32.692793931731984, 99.27854517076668)(32.76864728192162, 98.91780262710832)(32.844500632111256, 98.52354953557497)(32.92035398230089, 98.94389395803333)(32.99620733249052, 99.63229672359998)(33.07206068268015, 98.89759603219997)(33.147914032869785, 99.95299700103334)(33.223767383059425, 99.27892148612504)(33.29962073324906, 99.05824739164167)(33.37547408343869, 99.93912491507498)(33.45132743362832, 99.16432792413327)(33.527180783817954, 99.47519478996666)(33.603034134007586, 98.43397392641671)(33.67888748419722, 99.14796182625831)(33.75474083438686, 97.53978679651664)(33.83059418457648, 98.27368968981669)(33.90644753476612, 99.74123374153336)(33.98230088495575, 97.88401492242498)(34.05815423514539, 98.15504577448333)(34.13400758533503, 98.14331136628334)(34.20986093552465, 97.95074422087492)(34.28571428571429, 97.93111695486668)(34.36156763590392, 97.82255886366667)(34.43742098609356, 97.41135295584999)(34.51327433628319, 97.22382204974996)(34.58912768647282, 96.96365657546669)(34.664981036662454, 97.55824248913333)(34.740834386852086, 98.36398763819169)(34.816687737041725, 98.101733631125)(34.89254108723136, 96.15881649925831)(34.96839443742099, 98.54248384618334)(35.04424778761062, 97.64870505714998)(35.120101137800255, 97.42762586448339)(35.19595448798989, 97.58029113890838)(35.27180783817953, 97.28495369618335)(35.34766118836915, 97.57987469277502)(35.42351453855879, 98.82252180368333)(35.499367888748424, 96.86169681586667)(35.575221238938056, 97.91037353804997)(35.65107458912769, 98.30442934170836)(35.72692793931732, 97.93557051085004)(35.80278128950696, 99.55395903511668)(35.878634639696585, 98.64264711779165)(35.954487989886225, 98.51939474485829)(36.03034134007585, 99.64480544180832)(36.10619469026549, 98.09204057258333)(36.18204804045513, 98.70286137939163)(36.257901390644754, 97.54946807491667)(36.333754740834394, 97.77218775708337)(36.40960809102402, 97.89990526472504)(36.48546144121366, 98.11670658977505)(36.56131479140329, 96.35058042914169)(36.63716814159292, 100.77396923159166)(36.713021491782555, 98.12555944408332)(36.78887484197219, 100.26465559201664)(36.86472819216183, 101.74201333344166)(36.94058154235145, 99.38006979555838)(37.01643489254109, 99.88574057811664)(37.092288242730724, 100.18038722294999)(37.16814159292036, 99.26796871242503)(37.24399494310999, 100.42030850722502)(37.31984829329962, 99.40416642308332)(37.395701643489254, 100.62977498216667)(37.47155499367889, 100.0489902651166)(37.547408343868526, 97.96965419796666)(37.62326169405816, 97.88277396687498)(37.69911504424779, 98.72561088731668)(37.77496839443742, 97.31728493863334)(37.85082174462706, 97.23191474209999)(37.92667509481669, 97.59445580776664)(38.00252844500633, 97.21521935351666)(38.07838179519595, 97.80482431300003)(38.15423514538559, 97.1308125511)(38.23008849557523, 96.91427886657495)(38.305941845764856, 97.79549997892498)(38.381795195954496, 98.50903348381665)(38.45764854614412, 98.3289151146417)(38.53350189633376, 97.64056197535001)(38.60935524652339, 99.0565948255084)(38.685208596713025, 97.51818231813334)(38.76106194690266, 99.27360158415833)(38.83691529709229, 98.13419269437502)(38.91276864728193, 98.64479121398328)(38.988621997471554, 99.82390393929164)(39.064475347661194, 99.53103627588335)(39.140328697850826, 99.66356658944167)(39.21618204804046, 100.38579714690835)(39.29203539823009, 99.84059396033335)(39.36788874841972, 100.751550890425)(39.443742098609356, 100.75109867605)(39.519595448798995, 100.56301061088331)(39.59544879898863, 100.00242697023329)(39.67130214917826, 100.04983023886659)(39.74715549936789, 100.63103535782503)(39.823008849557525, 100.69456204597496)(39.89886219974716, 99.49645406857499)(39.97471554993679, 101.0272917577833)(40.05056890012643, 98.2000333029333)(40.126422250316054, 100.06981047305833)(40.20227560050569, 98.392198222625)(40.278128950695326, 98.47004701190004)(40.35398230088496, 98.02103603059163)(40.4298356510746, 98.69046954515832)(40.50568900126422, 97.41598720981666)(40.58154235145386, 98.88916969396665)(40.65739570164349, 97.15601467719166)(40.73324905183313, 99.09356481148339)(40.80910240202276, 97.52187624439168)(40.88495575221239, 98.10861256539168)(40.96080910240203, 97.40119459000829)(41.036662452591656, 97.75937562362495)(41.112515802781296, 95.389480818625)(41.18836915297093, 96.32828037357504)(41.26422250316056, 96.06487760884995)(41.34007585335019, 95.51896728810001)(41.415929203539825, 97.02706685797497)(41.49178255372946, 96.4602649173167)(41.5676359039191, 95.04913157905834)(41.64348925410873, 95.39734228762498)(41.71934260429836, 95.32495909018331)(41.795195954487994, 95.30717400630002)(41.871049304677626, 96.25315755189163)(41.94690265486726, 95.0929345954)(42.02275600505689, 94.57263104829164)(42.09860935524653, 95.52373284408337)(42.174462705436156, 94.498766475825)(42.250316055625795, 95.6992329568084)(42.32616940581543, 95.85990847402495)(42.40202275600506, 95.08975430450833)(42.4778761061947, 94.33712351551665)(42.553729456384325, 95.32800237480835)(42.629582806573964, 93.87405401986669)(42.70543615676359, 96.171102247875)(42.78128950695323, 95.08032743004168)(42.85714285714286, 95.45883659490829)(42.932996207332494, 94.46167648217504)(43.00884955752213, 93.62755290550007)(43.08470290771176, 94.56935153440004)(43.1605562579014, 93.15602234015832)(43.23640960809102, 96.41914302566667)(43.31226295828066, 95.84755717195004)(43.388116308470295, 94.24659291194166)(43.46396965865993, 96.32267610815832)(43.53982300884956, 94.85913763415002)(43.61567635903919, 94.39763408305835)(43.69152970922883, 95.92359099492504)(43.767383059418464, 95.22382361874166)(43.843236409608096, 93.45747083904995)(43.91908975979773, 95.62056757729164)(43.99494310998736, 95.25486483857497)(44.07079646017699, 94.23768910450835)(44.14664981036663, 96.7378319744417)(44.22250316055626, 96.51738127190832)(44.2983565107459, 99.17557947736667)(44.37420986093552, 98.86824212816667)(44.45006321112516, 97.74758691056671)(44.5259165613148, 99.79047146798335)(44.60176991150443, 98.45783534040832)(44.677623261694066, 98.9627217456833)(44.75347661188369, 98.91691090200005)(44.82932996207333, 98.710038263425)(44.90518331226296, 98.20044469302493)(44.981036662452595, 98.03474596525001)(45.05689001264223, 101.22269798919164)(45.13274336283186, 100.1418311734834)(45.2085967130215, 100.586893975875)(45.284450063211125, 101.83410022080831)(45.360303413400764, 99.71999974122497)(45.4361567635904, 101.15748844507495)(45.51201011378003, 100.56094728497501)(45.58786346396966, 99.77337723927499)(45.663716814159294, 100.88854986114166)(45.739570164348926, 100.08903657820832)(45.815423514538566, 101.24686373798338)(45.8912768647282, 101.20187154433337)(45.96713021491783, 100.828875171575)(46.04298356510746, 101.23746821373337)(46.118836915297095, 101.389993329775)(46.19469026548673, 101.98378685001668)(46.27054361567636, 101.49610848620003)(46.346396965866, 101.76934732335829)(46.422250316055624, 101.61701762916668)(46.498103666245264, 102.12668807255838)(46.573957016434896, 101.92270900527501)(46.64981036662453, 101.85116142266662)(46.72566371681417, 101.21806971492501)(46.80151706700379, 101.42342977529168)(46.87737041719343, 102.69566517877307)(46.95322376738306, 102.44329151971431)(47.0290771175727, 101.86887013045374)(47.10493046776233, 102.47753609136973)(47.18078381795196, 102.60385121704205)(47.2566371681416, 101.86341495647903)(47.33249051833123, 102.47046045523531)(47.408343868520866, 102.69748355533898)(47.4841972187105, 102.15840118311014)(47.56005056890013, 102.57671625152544)(47.63590391908976, 101.72419165708473)(47.711757269279396, 102.2730408436694)(47.78761061946903, 102.26996372879661)(47.86346396965867, 101.9667056768559)(47.9393173198483, 101.83197920811861)(48.01517067003793, 101.9435613170254)(48.091024020227565, 101.16961455312709)(48.1668773704172, 102.36350725698308)(48.24273072060683, 101.44827640861016)(48.31858407079646, 101.89797363550845)(48.3944374209861, 102.06223317560166)(48.470290771175726, 100.70713113857624)(48.546144121365366, 101.09775221336443)(48.621997471555, 100.12575337753388)(48.69785082174463, 100.59708886063555)(48.77370417193427, 101.5301726529407)(48.849557522123895, 98.84441827553844)(48.925410872313535, 99.41839200968374)(49.00126422250316, 100.70204131747863)(49.0771175726928, 99.63369230968962)(49.15297092288243, 101.67510427978448)(49.228824273072064, 99.82675075829309)(49.3046776232617, 99.37935311113792)(49.38053097345133, 99.69214689718102)(49.45638432364097, 101.33584543722418)(49.53223767383059, 100.09369032614782)(49.60809102402023, 99.43471273614037)(49.683944374209865, 100.10301721283186)(49.7597977243995, 98.9284160302035)(49.83565107458913, 99.0673405786637)(49.91150442477876, 97.74289774014282)(49.9873577749684, 97.66616047341961)(50.063211125158034, 98.31535067763393)(50.139064475347666, 98.38808870619643)(50.2149178255373, 97.88956361513394)(50.29077117572693, 99.16974581649552)(50.36662452591656, 98.3195514613)(50.4424778761062, 97.67745559530908)(50.51833122629583, 99.46582576980913)(50.59418457648547, 97.0308031443091)(50.6700379266751, 98.83710834112037)(50.74589127686473, 99.00713863703702)(50.82174462705437, 97.16137257241662)(50.897597977244, 97.25844902671025)(50.97345132743364, 97.87457819738675)(51.04930467762326, 98.62869653536794)(51.1251580278129, 98.87261897020754)(51.20101137800253, 97.93424742461322)(51.276864728192166, 99.00211673412264)(51.352718078381805, 95.0879884036981)(51.42857142857143, 96.1623991111334)(51.50442477876107, 95.0598772510673)(51.580278128950695, 97.29759826663461)(51.656131479140335, 95.77195371617476)(51.73198482932997, 100.59952543113722)(51.8078381795196, 97.96860036627726)(51.88369152970923, 98.99936478087126)(51.959544879898864, 98.07536847961994)(52.035398230088504, 98.45540084836364)(52.111251580278136, 98.68484611449496)(52.18710493046777, 97.81484684348482)(52.2629582806574, 97.15291388521214)(52.33881163084703, 98.90619660558588)(52.414664981036665, 98.42940558760208)(52.4905183312263, 98.45419372879591)(52.56637168141593, 95.2489163205294)(52.64222503160557, 93.11722355988233)(52.7180783817952, 91.7943976037353)(52.793931731984834, 92.4264270487059)(52.86978508217447, 93.79676122241175)(52.9456384323641, 95.04984617508822)(53.02149178255374, 93.02282625100001)(53.097345132743364, 93.017508384)(53.173198482933, 93.2354166112353)(53.24905183312263, 92.27966833632355)(53.32490518331227, 93.16413979955881)(53.4007585335019, 92.49767545958822)(53.47661188369153, 93.4877684555294)(53.55246523388117, 95.11854364929411)(53.6283185840708, 92.04633846976468)(53.70417193426044, 93.75099408714705)(53.78002528445007, 92.62963734055883)(53.8558786346397, 90.98391117052941)(53.931731984829334, 94.87715988349998)(54.007585335018966, 93.88428595944119)(54.0834386852086, 92.94290020558823)(54.15929203539824, 94.57820173341176)(54.23514538558787, 94.52100951888235)(54.3109987357775, 94.21712587158824)(54.386852085967135, 94.60479558882355)(54.46270543615677, 94.28961216682353)(54.5385587863464, 92.62622233161767)(54.61441213653603, 96.55406674711764)(54.69026548672567, 95.68368777402941)(54.7661188369153, 95.56842493702942)(54.841972187104936, 95.31103696182353)(54.91782553729457, 94.37818738726473)(54.9936788874842, 96.75909028282354)(55.06953223767384, 97.99580103564706)(55.145385587863466, 96.94310680664705)(55.221238938053105, 94.66261422297059)(55.29709228824273, 96.39488876473528)(55.37294563843237, 96.12323706120587)(55.448798988622, 97.05615998035292)(55.524652338811634, 100.48866654720591)(55.600505689001274, 99.54307416508821)(55.6763590391909, 100.49063750188236)(55.75221238938054, 101.14660255291176)(55.828065739570164, 100.76873747441175)(55.9039190897598, 100.25228667797057)(55.979772439949436, 97.49260674508825)(56.05562579013907, 99.32776189144116)(56.1314791403287, 98.68726664344116)(56.20733249051833, 96.16570157294119)(56.28318584070797, 97.46614231879413)(56.359039190897604, 98.13792615173529)(56.43489254108724, 97.22521494558823)(56.51074589127687, 97.70108740251516)(56.5865992414665, 98.12507951900001)(56.662452591656134, 96.67223277978786)(56.73830594184577, 100.12168636248482)(56.8141592920354, 98.05223183769697)(56.89001264222504, 97.90754771733334)(56.96586599241467, 97.52394504015149)(57.0417193426043, 96.85774630090626)(57.11757269279394, 96.2053230509375)(57.19342604298357, 96.96781435859374)(57.26927939317321, 96.68459528203125)(57.34513274336283, 94.6449663058125)(57.42098609355247, 96.87988004040625)(57.496839443742104, 98.87075063256248)(57.572692793931736, 97.81322687753125)(57.648546144121376, 98.18794543984373)(57.724399494311, 97.34444767024999)(57.80025284450064, 99.81531816890622)(57.876106194690266, 99.44526634919353)(57.951959544879905, 94.45266954680646)(58.02781289506954, 96.7516965973871)(58.10366624525917, 97.7251959037742)(58.1795195954488, 94.5564307339)(58.255372945638435, 93.18116041703333)(58.331226295828074, 95.10590428666664)(58.407079646017706, 93.43108931176667)(58.48293299620734, 94.83683876593103)(58.55878634639697, 93.42242982682758)(58.6346396965866, 92.33896717334483)(58.710493046776236, 93.93292172789654)(58.78634639696587, 91.62135775982144)(58.8621997471555, 91.13996358271429)(58.93805309734514, 90.49177545418517)(59.01390644753477, 89.28607913151849)(59.089759797724405, 88.6326059425926)(59.16561314791404, 90.50093962981481)(59.24146649810367, 92.23489927570371)(59.31731984829331, 91.2695356339231)(59.393173198482934, 91.79985416780772)(59.46902654867257, 85.83851940292305)(59.5448798988622, 88.205101706)(59.62073324905184, 91.40304399423077)(59.69658659924148, 86.36445022238462)(59.7724399494311, 86.59121581623074)(59.84829329962074, 84.71436686046152)(59.92414664981037, 88.79269654815381)(60.0, 84.84407600046156)
            };
            \addplot[color=blue, mark=none,name path=A] coordinates { %% MAX value
            (0.0, 150.40541831000002)(0.07585335018963338, 146.820575048)(0.15170670037926676, 163.37797335000002)(0.22756005056890014, 134.472925215)(0.3034134007585335, 140.8545346)(0.3792667509481669, 143.163766765)(0.45512010113780027, 139.054843692)(0.5309734513274336, 146.90136408400002)(0.606826801517067, 136.828145278)(0.6826801517067005, 141.7558354)(0.7585335018963338, 146.66626405300002)(0.8343868520859672, 154.225127709)(0.9102402022756005, 157.081954459)(0.986093552465234, 155.109878778)(1.0619469026548671, 157.662289563)(1.1378002528445006, 156.354351408)(1.213653603034134, 149.716906311)(1.2895069532237675, 152.470715622)(1.365360303413401, 151.214086303)(1.4412136536030342, 158.585916808)(1.5170670037926677, 160.27114178000002)(1.5929203539823011, 157.805771015)(1.6687737041719344, 159.33169988999998)(1.7446270543615676, 152.452438127)(1.820480404551201, 157.447376448)(1.8963337547408345, 153.14553607300002)(1.972187104930468, 159.880357915)(2.0480404551201015, 158.108750995)(2.1238938053097343, 160.32534925800002)(2.199747155499368, 158.524737722)(2.275600505689001, 159.830924805)(2.351453855878635, 155.851137999)(2.427307206068268, 172.285727256)(2.503160556257902, 164.402564624)(2.579013906447535, 165.397223646)(2.6548672566371687, 170.531719888)(2.730720606826802, 169.982597961)(2.806573957016435, 172.617372526)(2.8824273072060684, 172.98218798699997)(2.9582806573957017, 170.538286972)(3.0341340075853354, 168.558837585)(3.1099873577749686, 169.86273965599997)(3.1858407079646023, 163.590117474)(3.2616940581542355, 166.780488283)(3.3375474083438688, 158.228144869)(3.413400758533502, 155.971250581)(3.4892541087231352, 163.794704769)(3.565107458912769, 163.452081641)(3.640960809102402, 161.127757331)(3.716814159292036, 160.384636549)(3.792667509481669, 151.806529123)(3.8685208596713023, 153.13695070699998)(3.944374209860936, 150.341812892)(4.020227560050569, 148.371932015)(4.096080910240203, 147.836638844)(4.171934260429836, 149.352522574)(4.2477876106194685, 151.449771287)(4.323640960809103, 134.690534097)(4.399494310998736, 131.44656333400002)(4.47534766118837, 126.44672220800001)(4.551201011378002, 120.253921086)(4.6270543615676365, 122.95972297)(4.70290771175727, 122.39236801300001)(4.778761061946904, 121.309365428)(4.854614412136536, 124.081045183)(4.9304677623261695, 120.931109638)(5.006321112515804, 119.862289238)(5.082174462705436, 121.073969971)(5.15802781289507, 119.08089195900001)(5.233881163084703, 123.964882753)(5.309734513274337, 121.57757479200001)(5.38558786346397, 120.751167705)(5.461441213653604, 123.131750712)(5.537294563843237, 125.05054828)(5.61314791403287, 121.76860463700001)(5.689001264222504, 122.491496551)(5.764854614412137, 121.310730425)(5.840707964601771, 119.46785040500001)(5.916561314791403, 130.87334443)(5.9924146649810375, 141.701278178)(6.068268015170671, 123.796569353)(6.144121365360304, 121.935427833)(6.219974715549937, 123.361155889)(6.29582806573957, 119.212044363)(6.371681415929205, 122.32980635000001)(6.447534766118837, 124.697018998)(6.523388116308471, 128.95762823)(6.599241466498104, 126.240627028)(6.6750948166877375, 132.792386356)(6.750948166877371, 127.590178014)(6.826801517067004, 124.138682896)(6.902654867256638, 131.242922675)(6.9785082174462705, 128.102381339)(7.054361567635905, 126.50704708199999)(7.130214917825538, 127.894920536)(7.206068268015172, 128.257987307)(7.281921618204804, 125.397266478)(7.357774968394438, 130.41941642199998)(7.433628318584072, 127.59651678899999)(7.509481668773706, 117.856718011)(7.585335018963338, 116.592216025)(7.661188369152971, 117.158092447)(7.737041719342605, 117.40281986299999)(7.812895069532239, 117.062602989)(7.888748419721872, 128.402135692)(7.964601769911505, 119.231883449)(8.040455120101138, 131.62495275499998)(8.116308470290772, 121.715350352)(8.192161820480406, 122.83969545299999)(8.268015170670038, 125.312162968)(8.343868520859672, 130.635999612)(8.419721871049305, 124.27336139399999)(8.495575221238937, 126.02135412799998)(8.571428571428573, 128.610193354)(8.647281921618205, 125.748866746)(8.72313527180784, 132.58653903700002)(8.798988621997472, 131.376594636)(8.874841972187106, 130.593126915)(8.95069532237674, 134.467375692)(9.026548672566372, 132.145296544)(9.102402022756005, 137.115355703)(9.178255372945639, 124.381254816)(9.254108723135273, 123.826249265)(9.329962073324905, 125.752042174)(9.40581542351454, 127.28188543)(9.481668773704172, 127.09094326900001)(9.557522123893808, 124.36590471000001)(9.63337547408344, 149.80707026099998)(9.709228824273072, 138.53914015200002)(9.785082174462707, 160.17886002400002)(9.860935524652339, 147.656711476)(9.936788874841973, 147.586612466)(10.012642225031607, 144.46488178299998)(10.08849557522124, 150.732302737)(10.164348925410872, 137.108724706)(10.240202275600508, 125.463235846)(10.31605562579014, 129.22349004)(10.391908975979774, 124.493289968)(10.467762326169407, 148.08309043399998)(10.543615676359039, 155.095155742)(10.619469026548675, 119.39815380200001)(10.695322376738307, 121.677105851)(10.77117572692794, 118.081645543)(10.847029077117574, 126.18815599)(10.922882427307208, 115.419802898)(10.99873577749684, 119.79355054099999)(11.074589127686474, 118.812462654)(11.150442477876107, 115.500304731)(11.22629582806574, 119.549634743)(11.302149178255375, 119.34949494199999)(11.378002528445007, 118.25472503)(11.453855878634641, 123.530377271)(11.529709228824274, 116.59693892)(11.605562579013906, 119.336032823)(11.681415929203542, 117.97536629199999)(11.757269279393174, 121.301462719)(11.833122629582807, 121.808255527)(11.90897597977244, 118.822658304)(11.984829329962075, 119.39858854600001)(12.060682680151707, 119.827941308)(12.136536030341341, 115.427386555)(12.212389380530974, 117.730361615)(12.288242730720608, 119.80625676000001)(12.364096080910242, 117.572483328)(12.439949431099874, 119.119693955)(12.515802781289509, 120.20225697200001)(12.59165613147914, 116.408380129)(12.667509481668775, 123.63534295299999)(12.74336283185841, 118.236715993)(12.819216182048041, 119.913526211)(12.895069532237674, 116.885072845)(12.970922882427308, 119.860782962)(13.046776232616942, 122.224672634)(13.122629582806574, 115.57645779399999)(13.198482932996209, 116.179932369)(13.274336283185841, 120.004462152)(13.350189633375475, 118.427112451)(13.42604298356511, 117.945736477)(13.501896333754742, 115.55171855399999)(13.577749683944376, 123.001637963)(13.653603034134008, 120.14623201299999)(13.729456384323642, 115.457285612)(13.805309734513276, 119.39439452900001)(13.881163084702909, 117.45768871799999)(13.957016434892541, 117.215664008)(14.032869785082175, 113.575725051)(14.10872313527181, 121.330934999)(14.184576485461443, 121.231779676)(14.260429835651076, 124.020605553)(14.336283185840708, 118.81153038800001)(14.412136536030344, 121.92849672199999)(14.487989886219976, 125.65823652700001)(14.563843236409609, 119.01989768199999)(14.639696586599243, 117.45428714299999)(14.715549936788875, 119.80196601399999)(14.79140328697851, 115.976879223)(14.867256637168143, 115.74944428500001)(14.943109987357776, 119.49186304999999)(15.018963337547412, 118.492025614)(15.094816687737044, 116.432931568)(15.170670037926676, 117.490860216)(15.24652338811631, 115.85752023799999)(15.322376738305943, 117.302818202)(15.398230088495575, 118.453392076)(15.47408343868521, 116.41486970800001)(15.549936788874842, 114.206068999)(15.625790139064478, 115.659827764)(15.701643489254112, 116.09764456900001)(15.777496839443744, 118.077501934)(15.853350189633378, 119.349462195)(15.92920353982301, 121.42553384899999)(16.005056890012643, 120.171443188)(16.080910240202275, 115.925278322)(16.15676359039191, 120.386397087)(16.232616940581543, 114.653405826)(16.30847029077118, 119.43885743099999)(16.38432364096081, 120.445377529)(16.460176991150444, 116.839885)(16.536030341340076, 121.658130656)(16.611883691529712, 114.96368538200001)(16.687737041719345, 118.795663377)(16.763590391908977, 116.003195892)(16.83944374209861, 118.111100151)(16.91529709228824, 115.73042009299999)(16.991150442477874, 115.81625189600001)(17.067003792667514, 118.059076918)(17.142857142857146, 116.057813539)(17.21871049304678, 116.30265661899999)(17.29456384323641, 115.348115233)(17.370417193426043, 120.74709596)(17.44627054361568, 116.83902625900001)(17.52212389380531, 116.335472854)(17.597977243994944, 120.08918677300001)(17.673830594184576, 121.288189368)(17.749683944374212, 124.820740799)(17.825537294563844, 134.399240244)(17.90139064475348, 132.300715184)(17.977243994943112, 126.505928487)(18.053097345132745, 126.270257295)(18.128950695322377, 130.07998587999998)(18.20480404551201, 131.872334896)(18.280657395701645, 126.014248532)(18.356510745891278, 164.336979793)(18.432364096080914, 123.020460122)(18.508217446270546, 124.74743247699999)(18.58407079646018, 121.45437684699999)(18.65992414664981, 116.866420325)(18.735777496839447, 117.194439894)(18.81163084702908, 123.276421641)(18.88748419721871, 123.042092902)(18.963337547408344, 118.618488246)(19.039190897597976, 144.360029542)(19.115044247787615, 123.37173397999999)(19.190897597977248, 129.77485943099998)(19.26675094816688, 124.079109993)(19.342604298356513, 131.827436268)(19.418457648546145, 118.72436523299999)(19.494310998735777, 116.168445944)(19.570164348925413, 117.632102493)(19.646017699115045, 119.781905954)(19.721871049304678, 118.143359267)(19.797724399494314, 118.076504071)(19.873577749683946, 114.999993942)(19.94943109987358, 126.952307699)(20.025284450063214, 116.932777907)(20.101137800252847, 122.436628357)(20.17699115044248, 116.86986144800001)(20.25284450063211, 119.17159115700001)(20.328697850821744, 123.40551099500001)(20.40455120101138, 119.88558123600001)(20.480404551201016, 126.755923515)(20.556257901390648, 118.058639574)(20.63211125158028, 119.423180401)(20.707964601769913, 117.12982417299999)(20.78381795195955, 115.988078978)(20.85967130214918, 118.66745197200001)(20.935524652338813, 115.55038267100001)(21.011378002528446, 118.08683754500001)(21.087231352718078, 119.587968069)(21.163084702907714, 118.06199832800002)(21.23893805309735, 122.36187916700001)(21.314791403286982, 117.797150647)(21.390644753476614, 121.022560552)(21.466498103666247, 118.362347686)(21.54235145385588, 117.598463414)(21.61820480404551, 118.967696344)(21.694058154235147, 120.11336002)(21.76991150442478, 153.530968566)(21.845764854614416, 116.56670434)(21.921618204804048, 115.627710068)(21.99747155499368, 125.05025095900001)(22.073324905183316, 128.62520143199998)(22.14917825537295, 136.785777649)(22.22503160556258, 134.667636775)(22.300884955752213, 133.80772047)(22.376738305941846, 124.06968619300001)(22.45259165613148, 127.114576965)(22.528445006321114, 121.82913298300001)(22.60429835651075, 124.84342425899999)(22.680151706700382, 122.518548102)(22.756005056890015, 126.20135158)(22.831858407079647, 123.95571146099999)(22.907711757269283, 122.185931247)(22.983565107458915, 121.546488857)(23.059418457648547, 126.699862147)(23.13527180783818, 124.75117042500001)(23.211125158027812, 125.740968204)(23.286978508217448, 134.37450823900002)(23.362831858407084, 131.373685988)(23.438685208596716, 129.374193068)(23.51453855878635, 130.906761909)(23.59039190897598, 128.753282626)(23.666245259165613, 136.82819781799998)(23.74209860935525, 139.376587011)(23.81795195954488, 124.347694533)(23.893805309734514, 126.89141988899999)(23.96965865992415, 127.957704191)(24.045512010113782, 124.19139758)(24.121365360303415, 127.955237353)(24.19721871049305, 133.700878713)(24.273072060682683, 144.30897087)(24.348925410872315, 147.902988551)(24.424778761061948, 147.054516451)(24.50063211125158, 147.251123543)(24.576485461441216, 149.012905727)(24.65233881163085, 151.430904973)(24.728192161820484, 146.136283436)(24.804045512010116, 127.039036666)(24.87989886219975, 129.28246593100002)(24.95575221238938, 126.017223981)(25.031605562579017, 136.0485142)(25.10745891276865, 141.174009855)(25.18331226295828, 146.824350874)(25.259165613147914, 120.96131297)(25.33501896333755, 121.28224813)(25.410872313527186, 130.1423043)(25.48672566371682, 140.106963578)(25.56257901390645, 148.094140639)(25.638432364096083, 157.358293842)(25.714285714285715, 141.287166216)(25.790139064475348, 139.15773994900002)(25.865992414664984, 149.81176099700002)(25.941845764854616, 143.704371187)(26.017699115044252, 159.28750689499998)(26.093552465233884, 125.547309257)(26.169405815423517, 118.04212016299999)(26.24525916561315, 118.122014868)(26.321112515802785, 118.63893807400001)(26.396965865992417, 116.408967827)(26.47281921618205, 125.818537065)(26.548672566371682, 119.941782825)(26.624525916561314, 120.513198915)(26.70037926675095, 117.27613447099999)(26.776232616940586, 119.49976271)(26.85208596713022, 123.38268687899999)(26.92793931731985, 120.63580116600001)(27.003792667509483, 122.368381664)(27.07964601769912, 117.432581842)(27.15549936788875, 118.652301225)(27.231352718078384, 121.122771027)(27.307206068268016, 125.752658663)(27.38305941845765, 126.53127293)(27.458912768647284, 126.921661715)(27.53476611883692, 123.158094063)(27.610619469026553, 127.469302019)(27.686472819216185, 121.68039112000001)(27.762326169405817, 119.939503577)(27.83817951959545, 131.645696248)(27.914032869785082, 118.42001161900001)(27.989886219974718, 126.845477017)(28.06573957016435, 129.827358318)(28.141592920353986, 131.872333087)(28.21744627054362, 129.69694926300002)(28.29329962073325, 131.38469460599998)(28.369152970922887, 125.266713861)(28.44500632111252, 130.77493582900001)(28.52085967130215, 135.636921025)(28.596713021491784, 118.952097679)(28.672566371681416, 125.457923421)(28.748419721871052, 126.84536873600001)(28.824273072060688, 129.279638339)(28.90012642225032, 132.272345927)(28.975979772439953, 127.259813446)(29.051833122629585, 128.56822046899998)(29.127686472819217, 129.277787663)(29.203539823008853, 126.49954878000001)(29.279393173198486, 131.548431105)(29.355246523388118, 125.62748899900001)(29.43109987357775, 125.043611884)(29.506953223767386, 130.35302658799998)(29.58280657395702, 132.011370555)(29.658659924146654, 129.35804427600002)(29.734513274336287, 133.06747574399998)(29.81036662452592, 148.308815352)(29.88621997471555, 144.557548212)(29.962073324905184, 148.208785452)(30.037926675094823, 115.762318046)(30.113780025284452, 118.32427210600001)(30.189633375474088, 115.63995773900001)(30.265486725663717, 116.256783619)(30.341340075853353, 115.5131799)(30.417193426042985, 119.39162993400001)(30.49304677623262, 113.897937004)(30.568900126422257, 122.705761957)(30.644753476611886, 117.446576701)(30.72060682680152, 122.977450324)(30.79646017699115, 121.98373524300001)(30.872313527180786, 121.438927172)(30.94816687737042, 120.404944525)(31.024020227560055, 118.60452895)(31.099873577749683, 117.32970658299999)(31.17572692793932, 114.93002020700001)(31.251580278128955, 117.57082030199999)(31.327433628318587, 120.390062432)(31.403286978508223, 115.90882817600001)(31.479140328697852, 118.151674219)(31.554993678887488, 118.180404326)(31.630847029077117, 119.89950457200001)(31.706700379266756, 122.06568110399999)(31.782553729456385, 117.99611279300001)(31.85840707964602, 116.54819458)(31.934260429835657, 125.074318889)(32.010113780025286, 119.99887058600001)(32.085967130214925, 120.952800201)(32.16182048040455, 116.38946562499999)(32.23767383059419, 119.19487029800001)(32.31352718078382, 119.178194995)(32.389380530973455, 120.70815354800001)(32.46523388116309, 116.63856005599999)(32.54108723135272, 118.648000989)(32.61694058154236, 114.698415803)(32.692793931731984, 117.098248958)(32.76864728192162, 119.73344756)(32.844500632111256, 118.65784675200001)(32.92035398230089, 118.508502459)(32.99620733249052, 120.414325119)(33.07206068268015, 122.652874777)(33.147914032869785, 122.746692749)(33.223767383059425, 118.167419101)(33.29962073324906, 115.393025671)(33.37547408343869, 121.24532093599998)(33.45132743362832, 129.094623739)(33.527180783817954, 119.45146141800001)(33.603034134007586, 161.673464116)(33.67888748419722, 119.60822608100001)(33.75474083438686, 117.92736773)(33.83059418457648, 118.768539656)(33.90644753476612, 121.51783034799999)(33.98230088495575, 119.364099572)(34.05815423514539, 116.605712999)(34.13400758533503, 116.566034759)(34.20986093552465, 115.864351346)(34.28571428571429, 119.942750847)(34.36156763590392, 122.786183168)(34.43742098609356, 117.472201272)(34.51327433628319, 122.62347014299999)(34.58912768647282, 117.217724478)(34.664981036662454, 114.196729832)(34.740834386852086, 118.584491413)(34.816687737041725, 117.468904519)(34.89254108723136, 114.590992074)(34.96839443742099, 119.688329382)(35.04424778761062, 118.93294517999999)(35.120101137800255, 125.797828245)(35.19595448798989, 124.681097903)(35.27180783817953, 119.534227711)(35.34766118836915, 116.739022457)(35.42351453855879, 119.730953189)(35.499367888748424, 117.111429297)(35.575221238938056, 123.2530601)(35.65107458912769, 122.748817362)(35.72692793931732, 131.676139058)(35.80278128950696, 120.352646104)(35.878634639696585, 124.18478589099999)(35.954487989886225, 126.42608281099999)(36.03034134007585, 131.381491359)(36.10619469026549, 122.342843109)(36.18204804045513, 123.840925721)(36.257901390644754, 129.75604196)(36.333754740834394, 125.913002366)(36.40960809102402, 127.979155457)(36.48546144121366, 127.40927314300001)(36.56131479140329, 145.743905045)(36.63716814159292, 143.224290518)(36.713021491782555, 140.098364771)(36.78887484197219, 146.263417325)(36.86472819216183, 141.07936160399998)(36.94058154235145, 145.73619311)(37.01643489254109, 138.482969719)(37.092288242730724, 135.556373987)(37.16814159292036, 140.184460417)(37.24399494310999, 138.233108926)(37.31984829329962, 142.218191523)(37.395701643489254, 153.665357523)(37.47155499367889, 156.71367371600002)(37.547408343868526, 152.676890962)(37.62326169405816, 142.779210299)(37.69911504424779, 161.02685512600002)(37.77496839443742, 156.76114209)(37.85082174462706, 158.51087783600002)(37.92667509481669, 151.185452305)(38.00252844500633, 158.762715539)(38.07838179519595, 157.456301652)(38.15423514538559, 156.227137313)(38.23008849557523, 155.48886358000001)(38.305941845764856, 160.92110807)(38.381795195954496, 158.98617671)(38.45764854614412, 150.680334788)(38.53350189633376, 155.905589927)(38.60935524652339, 156.25859637599999)(38.685208596713025, 158.74138979)(38.76106194690266, 154.911957214)(38.83691529709229, 158.93025628)(38.91276864728193, 160.28032534099998)(38.988621997471554, 159.443231254)(39.064475347661194, 156.424898707)(39.140328697850826, 152.639355567)(39.21618204804046, 164.780287966)(39.29203539823009, 164.672984707)(39.36788874841972, 164.920013358)(39.443742098609356, 163.419255516)(39.519595448798995, 172.608762294)(39.59544879898863, 172.10063051999998)(39.67130214917826, 169.171884685)(39.74715549936789, 162.786956856)(39.823008849557525, 166.186164537)(39.89886219974716, 157.778413217)(39.97471554993679, 165.586110429)(40.05056890012643, 159.906107442)(40.126422250316054, 165.754210307)(40.20227560050569, 162.318016376)(40.278128950695326, 166.976313634)(40.35398230088496, 162.460069654)(40.4298356510746, 165.30215514000002)(40.50568900126422, 161.26448171100003)(40.58154235145386, 168.98170821099998)(40.65739570164349, 162.19810713)(40.73324905183313, 173.41086811)(40.80910240202276, 165.87682773400002)(40.88495575221239, 173.77738159)(40.96080910240203, 171.765044783)(41.036662452591656, 170.264248381)(41.112515802781296, 160.423102255)(41.18836915297093, 162.57314929700001)(41.26422250316056, 163.96629010499998)(41.34007585335019, 163.635912063)(41.415929203539825, 156.815533653)(41.49178255372946, 161.229145533)(41.5676359039191, 163.027187081)(41.64348925410873, 151.254305606)(41.71934260429836, 161.554199357)(41.795195954487994, 147.783554537)(41.871049304677626, 155.818562967)(41.94690265486726, 144.31459182199998)(42.02275600505689, 139.309118083)(42.09860935524653, 136.302671239)(42.174462705436156, 129.89733228)(42.250316055625795, 138.46148075600001)(42.32616940581543, 121.524103954)(42.40202275600506, 121.310875928)(42.4778761061947, 123.013682821)(42.553729456384325, 121.54912702800002)(42.629582806573964, 122.469524716)(42.70543615676359, 121.66256602300001)(42.78128950695323, 120.536595265)(42.85714285714286, 134.051972562)(42.932996207332494, 132.771335525)(43.00884955752213, 118.67766190399999)(43.08470290771176, 121.930012523)(43.1605562579014, 120.444194885)(43.23640960809102, 122.90096794)(43.31226295828066, 118.917183704)(43.388116308470295, 122.912553473)(43.46396965865993, 119.617726739)(43.53982300884956, 122.39851461200001)(43.61567635903919, 123.22338999499999)(43.69152970922883, 123.195979782)(43.767383059418464, 124.778076881)(43.843236409608096, 122.603233665)(43.91908975979773, 122.79482596599999)(43.99494310998736, 121.250319518)(44.07079646017699, 123.255018119)(44.14664981036663, 119.026836272)(44.22250316055626, 121.56344842899999)(44.2983565107459, 125.050741795)(44.37420986093552, 120.20894528299999)(44.45006321112516, 127.66916143499999)(44.5259165613148, 115.777243853)(44.60176991150443, 119.025931845)(44.677623261694066, 117.956874168)(44.75347661188369, 115.82493062699999)(44.82932996207333, 124.03720296399999)(44.90518331226296, 116.7241183)(44.981036662452595, 121.43943921)(45.05689001264223, 117.428726403)(45.13274336283186, 117.131324409)(45.2085967130215, 128.738510943)(45.284450063211125, 120.74120243799999)(45.360303413400764, 117.848928199)(45.4361567635904, 121.63033396)(45.51201011378003, 119.69604515900001)(45.58786346396966, 118.33351437299999)(45.663716814159294, 115.14444515699999)(45.739570164348926, 115.197426059)(45.815423514538566, 117.795327853)(45.8912768647282, 120.81927377299999)(45.96713021491783, 118.715962499)(46.04298356510746, 125.92613475600001)(46.118836915297095, 118.11360349099999)(46.19469026548673, 119.367854295)(46.27054361567636, 116.948990523)(46.346396965866, 115.349381439)(46.422250316055624, 133.4799042)(46.498103666245264, 118.27890164899999)(46.573957016434896, 117.910249023)(46.64981036662453, 119.072608568)(46.72566371681417, 118.446244903)(46.80151706700379, 118.863811495)(46.87737041719343, 118.493503962)(46.95322376738306, 118.609477147)(47.0290771175727, 117.562841469)(47.10493046776233, 116.401896314)(47.18078381795196, 118.574339019)(47.2566371681416, 116.872961228)(47.33249051833123, 116.88816562299999)(47.408343868520866, 119.661130219)(47.4841972187105, 119.016192813)(47.56005056890013, 116.2199315)(47.63590391908976, 117.734828185)(47.711757269279396, 121.313761576)(47.78761061946903, 119.579653283)(47.86346396965867, 124.56045800300001)(47.9393173198483, 122.188901072)(48.01517067003793, 123.23996347299999)(48.091024020227565, 123.961118175)(48.1668773704172, 127.202139782)(48.24273072060683, 128.925439659)(48.31858407079646, 122.84546022200001)(48.3944374209861, 126.40132870400001)(48.470290771175726, 123.397369857)(48.546144121365366, 116.51402875800001)(48.621997471555, 119.52070774299999)(48.69785082174463, 115.67200765300001)(48.77370417193427, 128.987252238)(48.849557522123895, 119.568108416)(48.925410872313535, 115.437687895)(49.00126422250316, 116.023274819)(49.0771175726928, 119.31469138)(49.15297092288243, 126.222795785)(49.228824273072064, 117.411287757)(49.3046776232617, 119.092848265)(49.38053097345133, 119.24999720400001)(49.45638432364097, 119.777704282)(49.53223767383059, 119.287398557)(49.60809102402023, 119.108209715)(49.683944374209865, 129.584066191)(49.7597977243995, 123.011689725)(49.83565107458913, 124.709161853)(49.91150442477876, 131.26421520099998)(49.9873577749684, 122.64792313999999)(50.063211125158034, 127.821590457)(50.139064475347666, 123.440197528)(50.2149178255373, 126.519827389)(50.29077117572693, 126.814743144)(50.36662452591656, 119.771367653)(50.4424778761062, 124.503434456)(50.51833122629583, 126.18387572399999)(50.59418457648547, 126.024139155)(50.6700379266751, 126.43738304899999)(50.74589127686473, 129.80167624)(50.82174462705437, 127.0377655)(50.897597977244, 124.70711471599999)(50.97345132743364, 129.691176921)(51.04930467762326, 124.900055592)(51.1251580278129, 123.649701122)(51.20101137800253, 127.192594122)(51.276864728192166, 130.56546020500002)(51.352718078381805, 141.53971227899999)(51.42857142857143, 129.458214146)(51.50442477876107, 126.775550805)(51.580278128950695, 128.05751798)(51.656131479140335, 119.37097347)(51.73198482932997, 117.94570094899998)(51.8078381795196, 117.99716463899999)(51.88369152970923, 118.343541859)(51.959544879898864, 119.11340097)(52.035398230088504, 118.60318499499999)(52.111251580278136, 118.38944955400001)(52.18710493046777, 119.625402525)(52.2629582806574, 120.968366074)(52.33881163084703, 118.70075746399999)(52.414664981036665, 119.30278303)(52.4905183312263, 114.77078552)(52.56637168141593, 129.515256833)(52.64222503160557, 128.87502923899999)(52.7180783817952, 129.61039922199998)(52.793931731984834, 122.849240388)(52.86978508217447, 130.677210977)(52.9456384323641, 130.837104769)(53.02149178255374, 119.699392918)(53.097345132743364, 126.519316045)(53.173198482933, 129.849742365)(53.24905183312263, 129.79140164700001)(53.32490518331227, 123.03041772)(53.4007585335019, 128.857755102)(53.47661188369153, 129.713763624)(53.55246523388117, 124.526984751)(53.6283185840708, 129.029441848)(53.70417193426044, 128.157088)(53.78002528445007, 119.42763016800001)(53.8558786346397, 128.396707071)(53.931731984829334, 131.74385046199998)(54.007585335018966, 146.561635163)(54.0834386852086, 149.464950902)(54.15929203539824, 135.304939185)(54.23514538558787, 146.30661027)(54.3109987357775, 137.446477314)(54.386852085967135, 141.126452766)(54.46270543615677, 145.160262794)(54.5385587863464, 145.700666686)(54.61441213653603, 144.18374613999998)(54.69026548672567, 149.318032922)(54.7661188369153, 144.805044294)(54.841972187104936, 143.67455170099998)(54.91782553729457, 148.367743911)(54.9936788874842, 158.972289656)(55.06953223767384, 155.3025353)(55.145385587863466, 155.165249559)(55.221238938053105, 157.96773293799998)(55.29709228824273, 155.204261852)(55.37294563843237, 154.602738151)(55.448798988622, 157.673908576)(55.524652338811634, 157.348334059)(55.600505689001274, 153.822396999)(55.6763590391909, 156.70549792399999)(55.75221238938054, 153.302400583)(55.828065739570164, 156.852233222)(55.9039190897598, 157.349188873)(55.979772439949436, 160.00275072600002)(56.05562579013907, 162.637960854)(56.1314791403287, 157.608405005)(56.20733249051833, 162.983343427)(56.28318584070797, 152.38511839699999)(56.359039190897604, 157.918173772)(56.43489254108724, 159.25182808300002)(56.51074589127687, 160.250512802)(56.5865992414665, 164.419360633)(56.662452591656134, 172.043858571)(56.73830594184577, 165.766624583)(56.8141592920354, 171.72038705199998)(56.89001264222504, 173.72726208300003)(56.96586599241467, 170.55883434999998)(57.0417193426043, 169.436660396)(57.11757269279394, 171.02960885000002)(57.19342604298357, 171.852624311)(57.26927939317321, 166.224002534)(57.34513274336283, 165.023031198)(57.42098609355247, 163.232968823)(57.496839443742104, 166.18908172200003)(57.572692793931736, 163.61372766399998)(57.648546144121376, 163.340508663)(57.724399494311, 161.91206746)(57.80025284450064, 159.193627161)(57.876106194690266, 154.10739832599998)(57.951959544879905, 142.921268646)(58.02781289506954, 151.982900852)(58.10366624525917, 149.628462948)(58.1795195954488, 128.901465056)(58.255372945638435, 123.40957190799999)(58.331226295828074, 117.30789455)(58.407079646017706, 121.11711650800001)(58.48293299620734, 122.572990811)(58.55878634639697, 123.887275066)(58.6346396965866, 121.594295421)(58.710493046776236, 125.49285796999999)(58.78634639696587, 122.687738691)(58.8621997471555, 125.429102668)(58.93805309734514, 123.80588127899999)(59.01390644753477, 120.229107224)(59.089759797724405, 114.852514935)(59.16561314791404, 119.730307063)(59.24146649810367, 135.111706559)(59.31731984829331, 137.522881181)(59.393173198482934, 134.552106905)(59.46902654867257, 136.325022359)(59.5448798988622, 131.494302123)(59.62073324905184, 139.27938236699998)(59.69658659924148, 150.98153222000002)(59.7724399494311, 115.538198215)(59.84829329962074, 115.519552867)(59.92414664981037, 111.587258808)(60.0, 110.781374057)
            };
            \addplot[color=blue, mark=none,name path=B] coordinates { %% MIN value
            (0.0, 12.551251132)(0.07585335018963338, 28.5748761)(0.15170670037926676, 13.134133039)(0.22756005056890014, 42.90992131)(0.3034134007585335, 36.454246911)(0.3792667509481669, 40.320124203)(0.45512010113780027, 9.720807682)(0.5309734513274336, 35.166147682)(0.606826801517067, 52.51426919)(0.6826801517067005, 42.442615909)(0.7585335018963338, 44.59714413100001)(0.8343868520859672, 10.494375886)(0.9102402022756005, 29.137507945)(0.986093552465234, 53.00722814)(1.0619469026548671, 26.289047521)(1.1378002528445006, 45.247648232)(1.213653603034134, 21.701218383)(1.2895069532237675, 39.289461903)(1.365360303413401, 49.093003806000006)(1.4412136536030342, 51.618521073)(1.5170670037926677, 47.39649390300001)(1.5929203539823011, 27.010892181)(1.6687737041719344, 46.513323674000006)(1.7446270543615676, 48.057096679000004)(1.820480404551201, 52.326511882000005)(1.8963337547408345, 38.147403779)(1.972187104930468, 35.874696176)(2.0480404551201015, 52.378368636000005)(2.1238938053097343, 55.985368105999996)(2.199747155499368, 55.480530083000005)(2.275600505689001, 38.686560861000004)(2.351453855878635, 45.194013471)(2.427307206068268, 24.371687195)(2.503160556257902, 51.956181726)(2.579013906447535, 26.233266281)(2.6548672566371687, 22.824926212999998)(2.730720606826802, 48.684648179999996)(2.806573957016435, 35.891392293)(2.8824273072060684, 24.329068445)(2.9582806573957017, 35.913753549)(3.0341340075853354, 47.316983769)(3.1099873577749686, 43.037591255)(3.1858407079646023, 11.158466775)(3.2616940581542355, 49.244720337)(3.3375474083438688, 28.722916667)(3.413400758533502, 46.588763363)(3.4892541087231352, 11.878452401999999)(3.565107458912769, 17.368114363)(3.640960809102402, 32.838643122)(3.716814159292036, 43.881290338)(3.792667509481669, 49.055979633999996)(3.8685208596713023, 52.498580559)(3.944374209860936, 43.174388191000006)(4.020227560050569, 45.645294635999996)(4.096080910240203, 44.221868990999994)(4.171934260429836, 48.894504588)(4.2477876106194685, 40.35666216)(4.323640960809103, 42.503459768)(4.399494310998736, 44.097224106999995)(4.47534766118837, 46.646753777)(4.551201011378002, 34.582173311000005)(4.6270543615676365, 35.906908949)(4.70290771175727, 36.531292715)(4.778761061946904, 41.344154841999995)(4.854614412136536, 35.992516226)(4.9304677623261695, 39.564195552)(5.006321112515804, 34.979140291)(5.082174462705436, 41.799405619)(5.15802781289507, 36.651593583)(5.233881163084703, 34.400132569)(5.309734513274337, 31.202371031)(5.38558786346397, 26.935669505)(5.461441213653604, 54.792384759)(5.537294563843237, 14.285086528999999)(5.61314791403287, 24.307281265)(5.689001264222504, 35.82240903)(5.764854614412137, 12.154668509)(5.840707964601771, 45.559620181999996)(5.916561314791403, 24.658624675)(5.9924146649810375, 15.846907385)(6.068268015170671, 43.974787283000005)(6.144121365360304, 46.758341518)(6.219974715549937, 21.181699953000003)(6.29582806573957, 42.8107807)(6.371681415929205, 24.407091041)(6.447534766118837, 12.861235312999998)(6.523388116308471, 23.428480769000004)(6.599241466498104, 12.624884671)(6.6750948166877375, 38.102365893)(6.750948166877371, 13.070721329)(6.826801517067004, 50.519574520999996)(6.902654867256638, 46.791831484)(6.9785082174462705, 36.859140528)(7.054361567635905, 44.401887095)(7.130214917825538, 40.906304893999994)(7.206068268015172, 43.679384643)(7.281921618204804, 41.32656722)(7.357774968394438, 52.300697366)(7.433628318584072, 47.371530064)(7.509481668773706, 44.323838523)(7.585335018963338, 36.36564487)(7.661188369152971, 46.223697830999996)(7.737041719342605, 45.696159812000005)(7.812895069532239, 37.106718719)(7.888748419721872, 32.498464330000004)(7.964601769911505, 44.04603797)(8.040455120101138, 42.315899108)(8.116308470290772, 23.496261174)(8.192161820480406, 38.673825650000005)(8.268015170670038, 35.07136297)(8.343868520859672, 40.925606138)(8.419721871049305, 51.222290269000005)(8.495575221238937, 49.289043375999995)(8.571428571428573, 49.447044896)(8.647281921618205, 29.750832442)(8.72313527180784, 27.955003034999997)(8.798988621997472, 46.90134988)(8.874841972187106, 38.640592379)(8.95069532237674, 50.178566276)(9.026548672566372, 34.286719216)(9.102402022756005, 40.388964479)(9.178255372945639, 52.423558307)(9.254108723135273, 44.736948345)(9.329962073324905, 45.959755423)(9.40581542351454, 49.018344596999995)(9.481668773704172, 40.811111009)(9.557522123893808, 33.731626333)(9.63337547408344, 47.46095205899999)(9.709228824273072, 41.615385755)(9.785082174462707, 43.773568203)(9.860935524652339, 49.435684030999994)(9.936788874841973, 38.300944176)(10.012642225031607, 48.342402928999995)(10.08849557522124, 33.986676801)(10.164348925410872, 28.56220151)(10.240202275600508, 11.483473468)(10.31605562579014, 29.430450513)(10.391908975979774, 9.512947947)(10.467762326169407, 31.103714773)(10.543615676359039, 13.817806177000001)(10.619469026548675, 11.980078586000001)(10.695322376738307, 12.794051935999999)(10.77117572692794, 11.186245629)(10.847029077117574, 23.868492259)(10.922882427307208, 12.10585066)(10.99873577749684, 27.961123645)(11.074589127686474, 35.362622284)(11.150442477876107, 50.626410627)(11.22629582806574, 49.653275771)(11.302149178255375, 36.585151081)(11.378002528445007, 33.198421133)(11.453855878634641, 42.898310506)(11.529709228824274, 34.956667226)(11.605562579013906, 45.280500368999995)(11.681415929203542, 42.825979819000004)(11.757269279393174, 37.465779156)(11.833122629582807, 35.875299414000004)(11.90897597977244, 38.585999181)(11.984829329962075, 40.984432304)(12.060682680151707, 33.961636192)(12.136536030341341, 42.230802236)(12.212389380530974, 41.32094824)(12.288242730720608, 35.041997802)(12.364096080910242, 40.576461158)(12.439949431099874, 33.506094056)(12.515802781289509, 43.633280411)(12.59165613147914, 43.086728967)(12.667509481668775, 47.074511122000004)(12.74336283185841, 27.520754996)(12.819216182048041, 37.764066433)(12.895069532237674, 45.245131407)(12.970922882427308, 33.823589655999996)(13.046776232616942, 26.795893866)(13.122629582806574, 26.827400061)(13.198482932996209, 23.429303094999998)(13.274336283185841, 31.111369151999998)(13.350189633375475, 13.355989376)(13.42604298356511, 12.453548325)(13.501896333754742, 12.29113774)(13.577749683944376, 19.21772204)(13.653603034134008, 34.497365701)(13.729456384323642, 30.176417099)(13.805309734513276, 38.090284268)(13.881163084702909, 23.802322039)(13.957016434892541, 24.437363995000002)(14.032869785082175, 44.755526278)(14.10872313527181, 15.074853034)(14.184576485461443, 10.685617556)(14.260429835651076, 44.133461659000005)(14.336283185840708, 36.854576334)(14.412136536030344, 43.998648863)(14.487989886219976, 41.761782773)(14.563843236409609, 42.752216063999995)(14.639696586599243, 34.631326086)(14.715549936788875, 38.691888268999996)(14.79140328697851, 41.338799777000006)(14.867256637168143, 13.736319962000001)(14.943109987357776, 36.259954543999996)(15.018963337547412, 34.866143444)(15.094816687737044, 38.011845222999995)(15.170670037926676, 43.862304368)(15.24652338811631, 32.378342900999996)(15.322376738305943, 45.105091165999994)(15.398230088495575, 44.36334161799999)(15.47408343868521, 44.958582746)(15.549936788874842, 20.289576756000002)(15.625790139064478, 38.191278634)(15.701643489254112, 46.105441025)(15.777496839443744, 13.893584576)(15.853350189633378, 30.188505926999998)(15.92920353982301, 13.48499347)(16.005056890012643, 25.616843039)(16.080910240202275, 31.592448418)(16.15676359039191, 47.95416522)(16.232616940581543, 31.464305613999997)(16.30847029077118, 21.931578378999998)(16.38432364096081, 48.555662208)(16.460176991150444, 12.862375082)(16.536030341340076, 39.775374955000004)(16.611883691529712, 24.82183589)(16.687737041719345, 23.52891528)(16.763590391908977, 52.054206685000004)(16.83944374209861, 31.462743707)(16.91529709228824, 45.067184968999996)(16.991150442477874, 50.572075926000004)(17.067003792667514, 13.477700716000001)(17.142857142857146, 53.441183118)(17.21871049304678, 34.794947565)(17.29456384323641, 42.305945914)(17.370417193426043, 35.334479157000004)(17.44627054361568, 47.980518817000004)(17.52212389380531, 24.124269783000003)(17.597977243994944, 14.086618517)(17.673830594184576, 14.600662712000002)(17.749683944374212, 12.532033999000001)(17.825537294563844, 10.772821459)(17.90139064475348, 19.224009399)(17.977243994943112, 11.014463659)(18.053097345132745, 35.148940481)(18.128950695322377, 12.660526245)(18.20480404551201, 11.228197592)(18.280657395701645, 15.982627228999998)(18.356510745891278, 11.334375972)(18.432364096080914, 47.209266592999995)(18.508217446270546, 12.524949871)(18.58407079646018, 19.969704079)(18.65992414664981, 43.147468414)(18.735777496839447, 45.84617534)(18.81163084702908, 13.339872549)(18.88748419721871, 36.107739428)(18.963337547408344, 38.030019030000005)(19.039190897597976, 41.718123594)(19.115044247787615, 42.708811645999994)(19.190897597977248, 35.411137147999995)(19.26675094816688, 39.159296185)(19.342604298356513, 13.986847727)(19.418457648546145, 36.093640693000005)(19.494310998735777, 46.201351353)(19.570164348925413, 42.320245843)(19.646017699115045, 22.197489503)(19.721871049304678, 44.432886791)(19.797724399494314, 11.732297137)(19.873577749683946, 37.807044728)(19.94943109987358, 35.835674028)(20.025284450063214, 34.712955174)(20.101137800252847, 40.477866504)(20.17699115044248, 36.684418369)(20.25284450063211, 36.341355687000004)(20.328697850821744, 35.640749121)(20.40455120101138, 36.999223317)(20.480404551201016, 35.084942317999996)(20.556257901390648, 10.632877165)(20.63211125158028, 10.772493459000001)(20.707964601769913, 29.092013222)(20.78381795195955, 11.993870818000001)(20.85967130214918, 15.11930478)(20.935524652338813, 12.500262277000001)(21.011378002528446, 12.941554077)(21.087231352718078, 22.451970969)(21.163084702907714, 49.285483652)(21.23893805309735, 12.048648142000001)(21.314791403286982, 39.444543079)(21.390644753476614, 42.78427104800001)(21.466498103666247, 11.220688733)(21.54235145385588, 38.466338786)(21.61820480404551, 10.610043494000001)(21.694058154235147, 37.066105589)(21.76991150442478, 42.227312939)(21.845764854614416, 12.691636456)(21.921618204804048, 27.37183518)(21.99747155499368, 43.468339245)(22.073324905183316, 48.243028357)(22.14917825537295, 46.526690214000006)(22.22503160556258, 41.568553327000004)(22.300884955752213, 32.111822722)(22.376738305941846, 45.882593086)(22.45259165613148, 50.444916665)(22.528445006321114, 45.961856453)(22.60429835651075, 35.219542710999995)(22.680151706700382, 45.535529239)(22.756005056890015, 38.925223169)(22.831858407079647, 24.425467293)(22.907711757269283, 45.230481997)(22.983565107458915, 35.088358966)(23.059418457648547, 34.560850463)(23.13527180783818, 44.577032862)(23.211125158027812, 25.837511378000002)(23.286978508217448, 37.919547505)(23.362831858407084, 35.984414692)(23.438685208596716, 46.292720996999996)(23.51453855878635, 37.690494579)(23.59039190897598, 43.25290867)(23.666245259165613, 45.337913544)(23.74209860935525, 50.186278359)(23.81795195954488, 47.06412167)(23.893805309734514, 47.840211176)(23.96965865992415, 37.662905302)(24.045512010113782, 43.010463405)(24.121365360303415, 43.416323424999995)(24.19721871049305, 54.403151341999994)(24.273072060682683, 37.127828428)(24.348925410872315, 44.100318238999996)(24.424778761061948, 46.367682095999996)(24.50063211125158, 45.323516126)(24.576485461441216, 43.441534997)(24.65233881163085, 42.395908945)(24.728192161820484, 43.549589084000004)(24.804045512010116, 44.699367458)(24.87989886219975, 42.923580615)(24.95575221238938, 35.244423054)(25.031605562579017, 26.952690765999996)(25.10745891276865, 13.566612208999999)(25.18331226295828, 14.161355464)(25.259165613147914, 11.685806969999998)(25.33501896333755, 24.644126777)(25.410872313527186, 37.375001803)(25.48672566371682, 36.071216841)(25.56257901390645, 37.126285497)(25.638432364096083, 14.807824805000001)(25.714285714285715, 17.627984471)(25.790139064475348, 33.175306177)(25.865992414664984, 26.404323014000003)(25.941845764854616, 12.833225295999998)(26.017699115044252, 23.258761736000004)(26.093552465233884, 13.282811052000001)(26.169405815423517, 11.054963085)(26.24525916561315, 29.515259182)(26.321112515802785, 43.984210644)(26.396965865992417, 41.736755613)(26.47281921618205, 11.379949278999998)(26.548672566371682, 37.825801476)(26.624525916561314, 12.251308515999998)(26.70037926675095, 37.010465212999996)(26.776232616940586, 37.681536006)(26.85208596713022, 13.716000671)(26.92793931731985, 36.333182365)(27.003792667509483, 34.592798675)(27.07964601769912, 38.298434129)(27.15549936788875, 48.967851429999996)(27.231352718078384, 34.646727212)(27.307206068268016, 38.399182135)(27.38305941845765, 31.752844615)(27.458912768647284, 43.898707613)(27.53476611883692, 36.650005788)(27.610619469026553, 22.308372828)(27.686472819216185, 44.494092288999994)(27.762326169405817, 49.284542851)(27.83817951959545, 34.752081177)(27.914032869785082, 38.065956056000005)(27.989886219974718, 26.508859979)(28.06573957016435, 54.036432233)(28.141592920353986, 53.329613021)(28.21744627054362, 29.588217800000002)(28.29329962073325, 45.122174623)(28.369152970922887, 12.050538667)(28.44500632111252, 27.005719837)(28.52085967130215, 21.589693806)(28.596713021491784, 15.196185724)(28.672566371681416, 47.95902029)(28.748419721871052, 27.790420026)(28.824273072060688, 13.249304278)(28.90012642225032, 31.478905464)(28.975979772439953, 12.753546599)(29.051833122629585, 11.005868503)(29.127686472819217, 9.782238149)(29.203539823008853, 48.00760665600001)(29.279393173198486, 40.933496555)(29.355246523388118, 37.755142418999995)(29.43109987357775, 44.725286886999996)(29.506953223767386, 37.079467507000004)(29.58280657395702, 41.729334875)(29.658659924146654, 44.580575071000005)(29.734513274336287, 45.626763821000004)(29.81036662452592, 46.562040845)(29.88621997471555, 41.918914766)(29.962073324905184, 42.77951027500001)(30.037926675094823, 23.878806807)(30.113780025284452, 43.664066509)(30.189633375474088, 51.150370832)(30.265486725663717, 23.849814571)(30.341340075853353, 43.526574353)(30.417193426042985, 28.527455818999996)(30.49304677623262, 14.122760126)(30.568900126422257, 51.284611254000005)(30.644753476611886, 43.489609187999996)(30.72060682680152, 43.600844789)(30.79646017699115, 33.637185579)(30.872313527180786, 22.898758405000002)(30.94816687737042, 47.536148503)(31.024020227560055, 27.522134682)(31.099873577749683, 42.57097929300001)(31.17572692793932, 46.07712602)(31.251580278128955, 32.911984038)(31.327433628318587, 41.454163312)(31.403286978508223, 12.658906956)(31.479140328697852, 38.194634432)(31.554993678887488, 48.763325438)(31.630847029077117, 22.048769345)(31.706700379266756, 43.170176697)(31.782553729456385, 31.474677106)(31.85840707964602, 41.738345041)(31.934260429835657, 47.20233471900001)(32.010113780025286, 36.078058005)(32.085967130214925, 44.045697295000004)(32.16182048040455, 44.900596801999995)(32.23767383059419, 28.821780679)(32.31352718078382, 33.347402556999995)(32.389380530973455, 11.790555784)(32.46523388116309, 46.293033555)(32.54108723135272, 17.98439058)(32.61694058154236, 36.398096197)(32.692793931731984, 50.95831154)(32.76864728192162, 10.149795528)(32.844500632111256, 34.580672078)(32.92035398230089, 13.560414935)(32.99620733249052, 35.404987653)(33.07206068268015, 26.508841802)(33.147914032869785, 37.152596872000004)(33.223767383059425, 43.76016499000001)(33.29962073324906, 36.635078871000005)(33.37547408343869, 37.326703083)(33.45132743362832, 12.02418741)(33.527180783817954, 45.172967092)(33.603034134007586, 11.055711181000001)(33.67888748419722, 48.530812368)(33.75474083438686, 33.506500838)(33.83059418457648, 44.416163119)(33.90644753476612, 44.255406529999995)(33.98230088495575, 37.27226843)(34.05815423514539, 31.202967275)(34.13400758533503, 43.824100185999995)(34.20986093552465, 47.376055044000005)(34.28571428571429, 13.006627864)(34.36156763590392, 12.861947098)(34.43742098609356, 43.307232694)(34.51327433628319, 12.354308786999999)(34.58912768647282, 35.750752453)(34.664981036662454, 39.650155613)(34.740834386852086, 41.479790852)(34.816687737041725, 42.704180750000006)(34.89254108723136, 25.31061925)(34.96839443742099, 49.708431803)(35.04424778761062, 42.907436577)(35.120101137800255, 36.778929213)(35.19595448798989, 35.574552223000005)(35.27180783817953, 42.955244276)(35.34766118836915, 32.084475891)(35.42351453855879, 27.707939145)(35.499367888748424, 23.763397534)(35.575221238938056, 43.027715783999994)(35.65107458912769, 12.640916195)(35.72692793931732, 10.572489668)(35.80278128950696, 50.717510441)(35.878634639696585, 10.986146681)(35.954487989886225, 10.83732197)(36.03034134007585, 38.475340931000005)(36.10619469026549, 11.30638161)(36.18204804045513, 14.864846208)(36.257901390644754, 33.394392613)(36.333754740834394, 45.539798016999995)(36.40960809102402, 25.369941649999998)(36.48546144121366, 13.681364265)(36.56131479140329, 12.423689496999998)(36.63716814159292, 35.488839365)(36.713021491782555, 12.722710554)(36.78887484197219, 21.877257875999998)(36.86472819216183, 36.294629242)(36.94058154235145, 39.007099997)(37.01643489254109, 33.905691172000004)(37.092288242730724, 43.782286912000004)(37.16814159292036, 48.435066406000004)(37.24399494310999, 35.64344296)(37.31984829329962, 38.553613055999996)(37.395701643489254, 48.181876803)(37.47155499367889, 15.21437648)(37.547408343868526, 21.169517691)(37.62326169405816, 35.581083584)(37.69911504424779, 45.89767709)(37.77496839443742, 31.109565521999997)(37.85082174462706, 32.948704469)(37.92667509481669, 34.597220751)(38.00252844500633, 11.863354203)(38.07838179519595, 36.60590821)(38.15423514538559, 37.487885233)(38.23008849557523, 42.928960776)(38.305941845764856, 48.329389214)(38.381795195954496, 11.90153078)(38.45764854614412, 28.327399090999997)(38.53350189633376, 37.800341941)(38.60935524652339, 51.403457859)(38.685208596713025, 34.215809506)(38.76106194690266, 42.555345234)(38.83691529709229, 37.364495581)(38.91276864728193, 50.260478191000004)(38.988621997471554, 41.530307081000004)(39.064475347661194, 43.462748987)(39.140328697850826, 36.182430808)(39.21618204804046, 40.35500577)(39.29203539823009, 46.65512288400001)(39.36788874841972, 48.218070257)(39.443742098609356, 54.205686117)(39.519595448798995, 54.10890786)(39.59544879898863, 43.623871813)(39.67130214917826, 39.958415943)(39.74715549936789, 48.408139410000004)(39.823008849557525, 48.463116938999995)(39.89886219974716, 11.894422961)(39.97471554993679, 53.064654530000006)(40.05056890012643, 11.347198107)(40.126422250316054, 37.581385595)(40.20227560050569, 29.428836722)(40.278128950695326, 47.887037868)(40.35398230088496, 11.746602916)(40.4298356510746, 13.622727773000001)(40.50568900126422, 14.227465854000002)(40.58154235145386, 42.730677929)(40.65739570164349, 10.522631654000001)(40.73324905183313, 38.185439908)(40.80910240202276, 20.965923531)(40.88495575221239, 29.99511914)(40.96080910240203, 14.289692813999999)(41.036662452591656, 32.645438977)(41.112515802781296, 19.453377104)(41.18836915297093, 41.64810756199999)(41.26422250316056, 33.267090657)(41.34007585335019, 21.674511501999998)(41.415929203539825, 45.511605872000004)(41.49178255372946, 35.472651479)(41.5676359039191, 41.621590525)(41.64348925410873, 42.739124878999995)(41.71934260429836, 34.335757887)(41.795195954487994, 12.491904449)(41.871049304677626, 41.50890648)(41.94690265486726, 38.095403916)(42.02275600505689, 42.107427582)(42.09860935524653, 37.735940176)(42.174462705436156, 12.961435326)(42.250316055625795, 36.152029026)(42.32616940581543, 43.937166428000005)(42.40202275600506, 26.361394722)(42.4778761061947, 36.346508317)(42.553729456384325, 40.152624306)(42.629582806573964, 15.111653278999999)(42.70543615676359, 35.095926654)(42.78128950695323, 27.806368789999997)(42.85714285714286, 37.695250365)(42.932996207332494, 30.33796679)(43.00884955752213, 13.16308266)(43.08470290771176, 36.250804153)(43.1605562579014, 15.245388885)(43.23640960809102, 48.562782442)(43.31226295828066, 27.831940383)(43.388116308470295, 18.266684167)(43.46396965865993, 27.254892187)(43.53982300884956, 17.994470878999998)(43.61567635903919, 13.434356409)(43.69152970922883, 11.050688875)(43.767383059418464, 36.115192818000004)(43.843236409608096, 26.976977918)(43.91908975979773, 34.786052354999995)(43.99494310998736, 37.517670894)(44.07079646017699, 12.384979327)(44.14664981036663, 33.606480092)(44.22250316055626, 11.576574208999999)(44.2983565107459, 41.624265214)(44.37420986093552, 11.664472583)(44.45006321112516, 14.432987925)(44.5259165613148, 38.955104348999996)(44.60176991150443, 23.272756366)(44.677623261694066, 39.262199374)(44.75347661188369, 52.624112274000005)(44.82932996207333, 37.781114009)(44.90518331226296, 36.623936618)(44.981036662452595, 40.323271076)(45.05689001264223, 49.891694035)(45.13274336283186, 43.549066598)(45.2085967130215, 15.909069143)(45.284450063211125, 49.585780912)(45.360303413400764, 33.61412339)(45.4361567635904, 45.599394448000005)(45.51201011378003, 46.446864420000004)(45.58786346396966, 28.303819022)(45.663716814159294, 48.873256854999994)(45.739570164348926, 44.408964828)(45.815423514538566, 47.027714640999996)(45.8912768647282, 45.749852407000006)(45.96713021491783, 45.13941057700001)(46.04298356510746, 48.133554651)(46.118836915297095, 50.62967599)(46.19469026548673, 28.475176228)(46.27054361567636, 11.84167019)(46.346396965866, 48.261310218000006)(46.422250316055624, 44.295986352999996)(46.498103666245264, 48.133963733)(46.573957016434896, 51.38683377299999)(46.64981036662453, 50.25939756)(46.72566371681417, 32.582034239)(46.80151706700379, 44.467892501)(46.87737041719343, 48.786174826)(46.95322376738306, 42.661473451000006)(47.0290771175727, 45.425173910999995)(47.10493046776233, 55.841439183)(47.18078381795196, 54.284210255999994)(47.2566371681416, 26.646352254)(47.33249051833123, 41.404328206)(47.408343868520866, 53.397734684999996)(47.4841972187105, 45.137401595)(47.56005056890013, 46.724822728999996)(47.63590391908976, 34.374259038)(47.711757269279396, 15.769673339)(47.78761061946903, 46.121049925)(47.86346396965867, 45.46242529)(47.9393173198483, 12.046993784)(48.01517067003793, 11.514940467)(48.091024020227565, 9.977689308999999)(48.1668773704172, 12.80403643)(48.24273072060683, 12.12157345)(48.31858407079646, 13.677356199999998)(48.3944374209861, 38.924795892)(48.470290771175726, 11.547890063999999)(48.546144121365366, 52.967784442)(48.621997471555, 39.248482683)(48.69785082174463, 56.125958689)(48.77370417193427, 46.737465776)(48.849557522123895, 14.262337191)(48.925410872313535, 43.699229849)(49.00126422250316, 44.099770649999996)(49.0771175726928, 21.403245696000003)(49.15297092288243, 44.519102967)(49.228824273072064, 32.678155972999996)(49.3046776232617, 43.347243561)(49.38053097345133, 40.256457059999995)(49.45638432364097, 47.09276379000001)(49.53223767383059, 50.921425686999996)(49.60809102402023, 49.895767576)(49.683944374209865, 42.835817189)(49.7597977243995, 47.966950138)(49.83565107458913, 11.632855692)(49.91150442477876, 37.554818556)(49.9873577749684, 37.952329567)(50.063211125158034, 23.763356978)(50.139064475347666, 40.421413128)(50.2149178255373, 41.560213666)(50.29077117572693, 41.466103262000004)(50.36662452591656, 45.233746010000004)(50.4424778761062, 34.989859767999995)(50.51833122629583, 43.680897575)(50.59418457648547, 31.537016084)(50.6700379266751, 43.138658529000004)(50.74589127686473, 44.603688334999994)(50.82174462705437, 35.008033407)(50.897597977244, 28.015119498)(50.97345132743364, 44.180962272)(51.04930467762326, 29.997637904999998)(51.1251580278129, 10.310866548)(51.20101137800253, 38.625337965)(51.276864728192166, 25.506851175)(51.352718078381805, 9.269976217)(51.42857142857143, 22.681176487000002)(51.50442477876107, 11.346635656)(51.580278128950695, 35.619406315)(51.656131479140335, 11.862308733999999)(51.73198482932997, 44.216215306)(51.8078381795196, 13.336638479000001)(51.88369152970923, 29.190351566)(51.959544879898864, 47.349078953)(52.035398230088504, 48.942028887)(52.111251580278136, 46.062664604999995)(52.18710493046777, 48.833952759)(52.2629582806574, 37.143389128)(52.33881163084703, 34.558388515)(52.414664981036665, 33.709158383)(52.4905183312263, 44.525452157)(52.56637168141593, 54.322753688)(52.64222503160557, 37.675501454)(52.7180783817952, 43.625669115)(52.793931731984834, 51.737904041)(52.86978508217447, 43.142378611)(52.9456384323641, 45.78789663800001)(53.02149178255374, 41.006080611)(53.097345132743364, 49.298016946000004)(53.173198482933, 44.26371551)(53.24905183312263, 51.747798373)(53.32490518331227, 36.191024063)(53.4007585335019, 51.661726285)(53.47661188369153, 44.423573971)(53.55246523388117, 51.644146911)(53.6283185840708, 35.946754678000005)(53.70417193426044, 48.154317979)(53.78002528445007, 41.906044939)(53.8558786346397, 33.415359133)(53.931731984829334, 48.916440317)(54.007585335018966, 45.585166994000005)(54.0834386852086, 23.325737316)(54.15929203539824, 52.253649988)(54.23514538558787, 52.310844206)(54.3109987357775, 55.304726604)(54.386852085967135, 45.941954548000005)(54.46270543615677, 46.137300971)(54.5385587863464, 26.185613994999997)(54.61441213653603, 47.682767378)(54.69026548672567, 36.408652103)(54.7661188369153, 50.211146607)(54.841972187104936, 52.279219769)(54.91782553729457, 25.483195909)(54.9936788874842, 47.263828491)(55.06953223767384, 51.882386168)(55.145385587863466, 51.446405012999996)(55.221238938053105, 40.481573378)(55.29709228824273, 50.441673181000006)(55.37294563843237, 37.489898282)(55.448798988622, 53.458818205)(55.524652338811634, 59.597955855)(55.600505689001274, 58.569736794)(55.6763590391909, 43.745493057999994)(55.75221238938054, 50.365221143)(55.828065739570164, 47.686360052000005)(55.9039190897598, 49.002070865)(55.979772439949436, 55.866915283000004)(56.05562579013907, 54.433646196999995)(56.1314791403287, 50.97880254)(56.20733249051833, 12.123288094)(56.28318584070797, 42.175534249)(56.359039190897604, 50.953321718)(56.43489254108724, 56.359307106)(56.51074589127687, 55.034718873)(56.5865992414665, 44.492853052)(56.662452591656134, 33.497138356)(56.73830594184577, 49.790879825000005)(56.8141592920354, 51.504583719)(56.89001264222504, 52.924207456000005)(56.96586599241467, 41.229601426)(57.0417193426043, 36.293924949)(57.11757269279394, 50.889146726999996)(57.19342604298357, 49.669407992)(57.26927939317321, 52.647328565)(57.34513274336283, 27.668734263)(57.42098609355247, 52.02877457299999)(57.496839443742104, 56.014949343)(57.572692793931736, 52.632386301)(57.648546144121376, 42.800019512)(57.724399494311, 45.049363747)(57.80025284450064, 52.727519311)(57.876106194690266, 53.311284752)(57.951959544879905, 24.794461981)(58.02781289506954, 46.552883068)(58.10366624525917, 53.74533338)(58.1795195954488, 50.2767676)(58.255372945638435, 46.177287304)(58.331226295828074, 48.171846157999994)(58.407079646017706, 51.393089089)(58.48293299620734, 52.257495136)(58.55878634639697, 29.652288837)(58.6346396965866, 48.621160155)(58.710493046776236, 47.441838395)(58.78634639696587, 44.429116038)(58.8621997471555, 53.843045778000004)(58.93805309734514, 35.856274791)(59.01390644753477, 51.157010786)(59.089759797724405, 41.260317648)(59.16561314791404, 33.088270534)(59.24146649810367, 31.850046545)(59.31731984829331, 43.098939809)(59.393173198482934, 45.826160475)(59.46902654867257, 37.872406129)(59.5448798988622, 39.607319363)(59.62073324905184, 54.144004875)(59.69658659924148, 12.133735238)(59.7724399494311, 44.274649197)(59.84829329962074, 12.134621183)(59.92414664981037, 50.305574834)(60.0, 37.965215516)
            };
            \addplot [pattern=north east lines,pattern color=red] 
            fill between [
                of=A and B,soft clip={domain=0:800},
            ];
            \end{axis}
\end{tikzpicture}
\caption{Measuring instrument: Clamp(Win)}
\end{subfigure}
\begin{subfigure}[b]{0.49\linewidth}
    \begin{tikzpicture}
        \pgfplotsset{%
        width=1\linewidth,
        % height=1\textheight
        }
        \begin{axis}[ymax=120,
            xlabel={Time (Seconds)},
            ylabel={Energy Consumption (Joules)},
            ]
            \addplot[color=blue, mark=none,] coordinates { %% AVG value
            (0.0, 91.03252017031669)(0.07585335018963338, 91.01100711230833)(0.15170670037926676, 91.5315947597083)(0.22756005056890014, 92.12766283032502)(0.3034134007585335, 91.69815290518332)(0.3792667509481669, 91.47274526715002)(0.45512010113780027, 92.01123248335834)(0.5309734513274336, 92.47876174989165)(0.606826801517067, 92.71101672425002)(0.6826801517067005, 92.89491459456673)(0.7585335018963338, 93.26467189054165)(0.8343868520859672, 93.61000899234168)(0.9102402022756005, 93.0794723357)(0.986093552465234, 93.91352868179166)(1.0619469026548671, 93.73875797538334)(1.1378002528445006, 93.44598916153339)(1.213653603034134, 94.11862736025)(1.2895069532237675, 94.38702070711669)(1.365360303413401, 93.88861407547499)(1.4412136536030342, 93.50163790095831)(1.5170670037926677, 93.88050343696665)(1.5929203539823011, 93.84852763995832)(1.6687737041719344, 93.68457721546663)(1.7446270543615676, 94.01090671415837)(1.820480404551201, 93.57998855071669)(1.8963337547408345, 94.36299048565002)(1.972187104930468, 94.0982480520083)(2.0480404551201015, 93.88613651750003)(2.1238938053097343, 93.83091520293334)(2.199747155499368, 94.09733199239996)(2.275600505689001, 94.02762799384168)(2.351453855878635, 93.85224659532498)(2.427307206068268, 94.47898570802502)(2.503160556257902, 94.53275981451672)(2.579013906447535, 94.59452205613333)(2.6548672566371687, 94.58049900217499)(2.730720606826802, 94.3369178709417)(2.806573957016435, 94.37440372417494)(2.8824273072060684, 94.31737435193332)(2.9582806573957017, 94.26649630025834)(3.0341340075853354, 94.54704380024162)(3.1099873577749686, 94.9293906724833)(3.1858407079646023, 94.8716129663)(3.2616940581542355, 94.41751129232499)(3.3375474083438688, 94.57793705066666)(3.413400758533502, 94.87045719768328)(3.4892541087231352, 95.04831471602498)(3.565107458912769, 95.29659863176661)(3.640960809102402, 94.92477189403328)(3.716814159292036, 94.98104613695835)(3.792667509481669, 95.23378472496663)(3.8685208596713023, 94.91482530502505)(3.944374209860936, 94.89470017470003)(4.020227560050569, 94.44837089262501)(4.096080910240203, 94.9328037435333)(4.171934260429836, 95.11500463724168)(4.2477876106194685, 94.6584938027417)(4.323640960809103, 94.79326585143336)(4.399494310998736, 94.53352084878335)(4.47534766118837, 94.35825669540834)(4.551201011378002, 95.00357399076671)(4.6270543615676365, 94.80598612047497)(4.70290771175727, 94.50214321202498)(4.778761061946904, 95.33036346626668)(4.854614412136536, 94.43070916871665)(4.9304677623261695, 96.1499924601084)(5.006321112515804, 96.22386694442501)(5.082174462705436, 96.16340244315832)(5.15802781289507, 96.26370549853337)(5.233881163084703, 96.17840787455)(5.309734513274337, 96.2209387545417)(5.38558786346397, 96.31119777338334)(5.461441213653604, 96.65234113377502)(5.537294563843237, 96.39040875781664)(5.61314791403287, 96.55733651358331)(5.689001264222504, 96.40846096424168)(5.764854614412137, 96.53302927740836)(5.840707964601771, 96.64423682280001)(5.916561314791403, 96.58313331253335)(5.9924146649810375, 96.52052715485836)(6.068268015170671, 96.23393968790833)(6.144121365360304, 96.39075317580827)(6.219974715549937, 96.71790046775001)(6.29582806573957, 96.64286505126668)(6.371681415929205, 96.00046897340836)(6.447534766118837, 96.408875024775)(6.523388116308471, 96.17810856750835)(6.599241466498104, 95.54587908886671)(6.6750948166877375, 95.36939665061664)(6.750948166877371, 94.77098000061667)(6.826801517067004, 93.95901685267498)(6.902654867256638, 94.97399844629999)(6.9785082174462705, 94.74413818698334)(7.054361567635905, 96.21903444743339)(7.130214917825538, 96.58226611501665)(7.206068268015172, 98.66681735463332)(7.281921618204804, 99.20496587984998)(7.357774968394438, 99.77942331640834)(7.433628318584072, 99.78168117037504)(7.509481668773706, 93.03151274517501)(7.585335018963338, 93.55902542915835)(7.661188369152971, 93.93946932076666)(7.737041719342605, 94.04511836012499)(7.812895069532239, 93.52368276167496)(7.888748419721872, 93.43389642732497)(7.964601769911505, 92.58443046012502)(8.040455120101138, 93.51810729462497)(8.116308470290772, 93.56108148300001)(8.192161820480406, 93.40670945478334)(8.268015170670038, 92.96543293639996)(8.343868520859672, 93.41930693085827)(8.419721871049305, 93.64313120054162)(8.495575221238937, 93.81432471089998)(8.571428571428573, 93.99958427298333)(8.647281921618205, 93.50425144887501)(8.72313527180784, 94.24988073919167)(8.798988621997472, 94.43340657736664)(8.874841972187106, 94.45450950075832)(8.95069532237674, 94.16098770905838)(9.026548672566372, 94.37565165428335)(9.102402022756005, 94.43236157302499)(9.178255372945639, 94.36058666330003)(9.254108723135273, 94.09467302365003)(9.329962073324905, 94.49273780771668)(9.40581542351454, 94.94302031594997)(9.481668773704172, 94.82655486120838)(9.557522123893808, 94.76475074544169)(9.63337547408344, 94.94092170449166)(9.709228824273072, 95.01296795456666)(9.785082174462707, 95.00893608558333)(9.860935524652339, 95.09306923621672)(9.936788874841973, 95.04747593139996)(10.012642225031607, 95.00229744544164)(10.08849557522124, 95.04167233000837)(10.164348925410872, 95.17211828111667)(10.240202275600508, 95.23181320688336)(10.31605562579014, 95.04856506133332)(10.391908975979774, 94.85337470903329)(10.467762326169407, 94.91937139280002)(10.543615676359039, 95.07481826850001)(10.619469026548675, 95.27049870809996)(10.695322376738307, 95.18678575713336)(10.77117572692794, 95.32540974028333)(10.847029077117574, 94.63666260740835)(10.922882427307208, 95.00212710849163)(10.99873577749684, 95.13737254538334)(11.074589127686474, 95.29516070484163)(11.150442477876107, 95.21876944143334)(11.22629582806574, 94.6286947766917)(11.302149178255375, 95.39260689120832)(11.378002528445007, 95.56428866480836)(11.453855878634641, 95.53363968087498)(11.529709228824274, 95.51047508966667)(11.605562579013906, 95.66175131133335)(11.681415929203542, 95.56727048763334)(11.757269279393174, 95.54248443833333)(11.833122629582807, 95.87780918461665)(11.90897597977244, 95.95586742842502)(11.984829329962075, 95.64011023765835)(12.060682680151707, 95.60590258221664)(12.136536030341341, 95.70924132635831)(12.212389380530974, 95.63848663629163)(12.288242730720608, 95.82485805283332)(12.364096080910242, 95.99831974289164)(12.439949431099874, 96.39104120935002)(12.515802781289509, 96.20195473438335)(12.59165613147914, 96.35117124574997)(12.667509481668775, 95.90859701516666)(12.74336283185841, 95.63420698157498)(12.819216182048041, 95.70922234445834)(12.895069532237674, 95.85191262307497)(12.970922882427308, 95.67403370384164)(13.046776232616942, 95.452415776)(13.122629582806574, 95.11153331880836)(13.198482932996209, 96.00614960290831)(13.274336283185841, 96.87194735119996)(13.350189633375475, 96.40000200800003)(13.42604298356511, 96.81112381831662)(13.501896333754742, 97.06846373731663)(13.577749683944376, 97.02495529259164)(13.653603034134008, 97.09204924405834)(13.729456384323642, 97.30712608706664)(13.805309734513276, 97.13095988272498)(13.881163084702909, 96.78583270803338)(13.957016434892541, 96.58286101225832)(14.032869785082175, 95.88936396285833)(14.10872313527181, 94.89765980719994)(14.184576485461443, 94.59211537263333)(14.260429835651076, 94.761974028425)(14.336283185840708, 93.78032681299166)(14.412136536030344, 94.45730607362503)(14.487989886219976, 95.14025507314169)(14.563843236409609, 95.38810536789171)(14.639696586599243, 97.88707784026661)(14.715549936788875, 98.45493901580836)(14.79140328697851, 99.33644194720831)(14.867256637168143, 98.45714578699169)(14.943109987357776, 97.73604974834167)(15.018963337547412, 91.25360724513332)(15.094816687737044, 92.22376287291671)(15.170670037926676, 92.5606842258333)(15.24652338811631, 92.20957645877503)(15.322376738305943, 92.52384789120835)(15.398230088495575, 93.09185049185002)(15.47408343868521, 93.38666648722497)(15.549936788874842, 92.99756161626667)(15.625790139064478, 93.1702090604666)(15.701643489254112, 92.92588359727505)(15.777496839443744, 93.34659399093339)(15.853350189633378, 92.77906740634164)(15.92920353982301, 93.43477578807496)(16.005056890012643, 93.6781258287667)(16.080910240202275, 93.99715925886667)(16.15676359039191, 94.16308256384166)(16.232616940581543, 93.97836438515834)(16.30847029077118, 94.05234402765838)(16.38432364096081, 93.63262574161669)(16.460176991150444, 93.68841744835828)(16.536030341340076, 93.45013407921667)(16.611883691529712, 93.83118051885002)(16.687737041719345, 93.73053316950838)(16.763590391908977, 93.30968710720002)(16.83944374209861, 93.75306424139161)(16.91529709228824, 93.86937518001668)(16.991150442477874, 94.19438397035002)(17.067003792667514, 94.12608412184163)(17.142857142857146, 93.72203116008336)(17.21871049304678, 93.9728288286083)(17.29456384323641, 94.19050915802497)(17.370417193426043, 93.71012795375833)(17.44627054361568, 94.03431016826669)(17.52212389380531, 93.65170759651667)(17.597977243994944, 94.705195410925)(17.673830594184576, 94.68297678692501)(17.749683944374212, 94.75192853240833)(17.825537294563844, 94.85173959519165)(17.90139064475348, 94.57926123959169)(17.977243994943112, 94.57574854792499)(18.053097345132745, 94.2786896095333)(18.128950695322377, 94.12637647061668)(18.20480404551201, 94.33497486647495)(18.280657395701645, 94.58703713364166)(18.356510745891278, 94.64493134169166)(18.432364096080914, 94.47688628974169)(18.508217446270546, 94.36950779289162)(18.58407079646018, 94.17788559125837)(18.65992414664981, 94.66232089754165)(18.735777496839447, 94.50222030403333)(18.81163084702908, 94.64329044415831)(18.88748419721871, 94.64921101085835)(18.963337547408344, 94.83915900446664)(19.039190897597976, 94.76384861185836)(19.115044247787615, 94.76597306802495)(19.190897597977248, 94.74444777199169)(19.26675094816688, 94.68238058201669)(19.342604298356513, 94.65565224971665)(19.418457648546145, 94.96435683472497)(19.494310998735777, 94.58764651470003)(19.570164348925413, 94.43889874856669)(19.646017699115045, 94.72270356106664)(19.721871049304678, 94.68875133199165)(19.797724399494314, 94.66864508450004)(19.873577749683946, 94.62110725705834)(19.94943109987358, 94.84332058725)(20.025284450063214, 94.3948785506917)(20.101137800252847, 94.40346378592494)(20.17699115044248, 94.75240585957502)(20.25284450063211, 94.74509948454161)(20.328697850821744, 94.69259826160005)(20.40455120101138, 94.36460975085835)(20.480404551201016, 94.66592597858332)(20.556257901390648, 94.49058024206667)(20.63211125158028, 94.5042354209583)(20.707964601769913, 94.14249278821666)(20.78381795195955, 94.45443436062503)(20.85967130214918, 94.88850532607502)(20.935524652338813, 95.002438439525)(21.011378002528446, 94.33312181496666)(21.087231352718078, 94.35995731820829)(21.163084702907714, 94.40045119960003)(21.23893805309735, 94.65917987190832)(21.314791403286982, 94.68789181675832)(21.390644753476614, 94.79682902008331)(21.466498103666247, 94.59814118925831)(21.54235145385588, 93.96291440385002)(21.61820480404551, 93.84825403236665)(21.694058154235147, 92.9656753945333)(21.76991150442478, 92.5168416782167)(21.845764854614416, 90.99042147235838)(21.921618204804048, 91.40972604381668)(21.99747155499368, 91.40192024089167)(22.073324905183316, 92.66484570626663)(22.14917825537295, 95.18190906412501)(22.22503160556258, 94.70048662280841)(22.300884955752213, 98.45597322000836)(22.376738305941846, 98.26322975114998)(22.45259165613148, 99.57326806710832)(22.528445006321114, 92.22561629394168)(22.60429835651075, 92.03288315237494)(22.680151706700382, 91.68627377085)(22.756005056890015, 91.74034355378335)(22.831858407079647, 91.65941391367501)(22.907711757269283, 91.76279324376667)(22.983565107458915, 91.89353824392497)(23.059418457648547, 91.78640339026664)(23.13527180783818, 91.86276751434997)(23.211125158027812, 91.46855880149165)(23.286978508217448, 92.24084863413333)(23.362831858407084, 92.2471323201333)(23.438685208596716, 91.72944838233335)(23.51453855878635, 92.50284926823335)(23.59039190897598, 92.83752662309168)(23.666245259165613, 92.99269569976667)(23.74209860935525, 93.01614862099169)(23.81795195954488, 93.02094864111672)(23.893805309734514, 93.47092649905)(23.96965865992415, 93.59343493337498)(24.045512010113782, 93.38105540973326)(24.121365360303415, 93.26438169945003)(24.19721871049305, 93.46572266968333)(24.273072060682683, 93.64640537999168)(24.348925410872315, 93.52108128030001)(24.424778761061948, 93.667000424975)(24.50063211125158, 94.02805099747498)(24.576485461441216, 94.00998098483333)(24.65233881163085, 93.59473688137501)(24.728192161820484, 93.04155449640838)(24.804045512010116, 93.47367771006662)(24.87989886219975, 93.25021493905834)(24.95575221238938, 93.46387023219168)(25.031605562579017, 94.240493170725)(25.10745891276865, 93.46929776072503)(25.18331226295828, 94.98398216118326)(25.259165613147914, 95.08430782260835)(25.33501896333755, 94.8203767013083)(25.410872313527186, 94.94424156979166)(25.48672566371682, 94.65617051515)(25.56257901390645, 94.694828774875)(25.638432364096083, 95.01898863501668)(25.714285714285715, 94.93828699007501)(25.790139064475348, 94.78939455585832)(25.865992414664984, 94.27147151145839)(25.941845764854616, 94.75874180489167)(26.017699115044252, 94.5670876019667)(26.093552465233884, 94.46291901979166)(26.169405815423517, 94.78949629609164)(26.24525916561315, 94.30459394210001)(26.321112515802785, 94.66527414171672)(26.396965865992417, 94.86467844205004)(26.47281921618205, 95.22600409486672)(26.548672566371682, 95.66557089825831)(26.624525916561314, 95.43044737204995)(26.70037926675095, 95.41348766576671)(26.776232616940586, 95.07993862245)(26.85208596713022, 95.16668584117498)(26.92793931731985, 95.12773808065002)(27.003792667509483, 95.65950692164164)(27.07964601769912, 95.67007541857498)(27.15549936788875, 94.96536136506666)(27.231352718078384, 95.57318980137502)(27.307206068268016, 95.64951547448328)(27.38305941845765, 95.64395390807495)(27.458912768647284, 95.37754721114169)(27.53476611883692, 95.29350000094162)(27.610619469026553, 95.06271360529168)(27.686472819216185, 95.45784391111667)(27.762326169405817, 95.6713779177583)(27.83817951959545, 95.71327048006667)(27.914032869785082, 95.9995985121)(27.989886219974718, 95.87922093497498)(28.06573957016435, 95.77532043807496)(28.141592920353986, 95.82604287077497)(28.21744627054362, 95.92058291183336)(28.29329962073325, 96.04515795618335)(28.369152970922887, 96.10550035989999)(28.44500632111252, 95.81726891225827)(28.52085967130215, 95.75560289475835)(28.596713021491784, 96.06363915825837)(28.672566371681416, 95.57786938046668)(28.748419721871052, 95.55786706294995)(28.824273072060688, 95.16559649568333)(28.90012642225032, 95.03312103556665)(28.975979772439953, 94.76281211264168)(29.051833122629585, 94.58475881809159)(29.127686472819217, 93.12008132779997)(29.203539823008853, 92.36622666488333)(29.279393173198486, 93.27063862899166)(29.355246523388118, 92.57182345709167)(29.43109987357775, 94.75566327562495)(29.506953223767386, 94.83322362639161)(29.58280657395702, 95.6604570423083)(29.658659924146654, 98.33507388929999)(29.734513274336287, 98.448761728375)(29.81036662452592, 99.0275502961083)(29.88621997471555, 99.54369393267503)(29.962073324905184, 99.28361857346668)(30.037926675094823, 91.14214841100842)(30.113780025284452, 91.97331661936138)(30.189633375474088, 92.25309755557983)(30.265486725663717, 92.27921494447901)(30.341340075853353, 92.33555362226046)(30.417193426042985, 92.37793710960513)(30.49304677623262, 92.41325726091594)(30.568900126422257, 92.25823820002527)(30.644753476611886, 92.11151410012606)(30.72060682680152, 92.05819813656304)(30.79646017699115, 92.2975662790924)(30.872313527180786, 92.42580238518487)(30.94816687737042, 92.09175503589074)(31.024020227560055, 92.27471989913444)(31.099873577749683, 92.52571371026889)(31.17572692793932, 92.78872370026048)(31.251580278128955, 92.15637918759666)(31.327433628318587, 92.41812052927732)(31.403286978508223, 92.8491598751344)(31.479140328697852, 92.77859726763023)(31.554993678887488, 92.76831118370588)(31.630847029077117, 92.45030031494115)(31.706700379266756, 92.57913377337815)(31.782553729456385, 92.92009526799997)(31.85840707964602, 92.68824401104199)(31.934260429835657, 92.59432884578995)(32.010113780025286, 92.53604825631095)(32.085967130214925, 92.86864025680671)(32.16182048040455, 92.60239152379832)(32.23767383059419, 92.68457128665548)(32.31352718078382, 92.55398317470589)(32.389380530973455, 92.68545882913443)(32.46523388116309, 92.93930351727728)(32.54108723135272, 92.62854517457144)(32.61694058154236, 92.99004308046221)(32.692793931731984, 93.1192528430588)(32.76864728192162, 93.04470476849576)(32.844500632111256, 93.27463139631095)(32.92035398230089, 93.29702102062181)(32.99620733249052, 93.3338595274706)(33.07206068268015, 93.66707072056305)(33.147914032869785, 93.17912395860503)(33.223767383059425, 93.31488421908399)(33.29962073324906, 93.54142054774789)(33.37547408343869, 93.91846673585714)(33.45132743362832, 94.04954988142015)(33.527180783817954, 93.91354549612603)(33.603034134007586, 93.80996939948741)(33.67888748419722, 94.17754539011769)(33.75474083438686, 94.45713072234453)(33.83059418457648, 94.1430140564958)(33.90644753476612, 93.93781819996642)(33.98230088495575, 93.71060323251258)(34.05815423514539, 93.78742123088236)(34.13400758533503, 94.31160783005883)(34.20986093552465, 94.51024079585713)(34.28571428571429, 94.12470501319328)(34.36156763590392, 93.8648817370756)(34.43742098609356, 93.59373518411768)(34.51327433628319, 93.00551605219327)(34.58912768647282, 93.19458688484879)(34.664981036662454, 93.25479156568906)(34.740834386852086, 92.53764015612606)(34.816687737041725, 93.56009033878988)(34.89254108723136, 93.10506036673105)(34.96839443742099, 94.03013800279828)(35.04424778761062, 94.14837031939493)(35.120101137800255, 93.91539818443697)(35.19595448798989, 94.35906498516806)(35.27180783817953, 94.5677715489496)(35.34766118836915, 94.48122546673112)(35.42351453855879, 94.57639663454624)(35.499367888748424, 94.06015645007565)(35.575221238938056, 94.30115201417647)(35.65107458912769, 94.46613596229408)(35.72692793931732, 94.37462200172271)(35.80278128950696, 94.27290669360505)(35.878634639696585, 94.44381457407563)(35.954487989886225, 94.39081576679828)(36.03034134007585, 94.24480106321849)(36.10619469026549, 94.14476871140332)(36.18204804045513, 94.50862897608403)(36.257901390644754, 94.49513758515124)(36.333754740834394, 94.4669906084538)(36.40960809102402, 94.49919872587391)(36.48546144121366, 94.2392645699412)(36.56131479140329, 92.70712811737815)(36.63716814159292, 91.39277548321006)(36.713021491782555, 90.9926415594202)(36.78887484197219, 90.82452904531095)(36.86472819216183, 90.50403987398325)(36.94058154235145, 91.29859479055457)(37.01643489254109, 91.72792823869749)(37.092288242730724, 95.31785490310081)(37.16814159292036, 96.45996801222695)(37.24399494310999, 97.65673439441173)(37.31984829329962, 97.98586822571423)(37.395701643489254, 98.52131418786553)(37.47155499367889, 98.37063132573107)(37.547408343868526, 90.37266072883197)(37.62326169405816, 90.58232160317641)(37.69911504424779, 90.677455163832)(37.77496839443742, 90.92684128029413)(37.85082174462706, 90.80046216074795)(37.92667509481669, 90.70159936056302)(38.00252844500633, 91.24444475843698)(38.07838179519595, 91.36482381864705)(38.15423514538559, 91.32321781582351)(38.23008849557523, 91.1799204119916)(38.305941845764856, 90.980779148916)(38.381795195954496, 91.32934262917647)(38.45764854614412, 91.83126141624369)(38.53350189633376, 91.79072598395793)(38.60935524652339, 91.33400886689917)(38.685208596713025, 91.29201167584871)(38.76106194690266, 91.87572546774794)(38.83691529709229, 91.82519996405044)(38.91276864728193, 91.88705661610928)(38.988621997471554, 92.3474634142353)(39.064475347661194, 92.36864051202521)(39.140328697850826, 91.92519907802522)(39.21618204804046, 92.00729095563025)(39.29203539823009, 92.47964279149576)(39.36788874841972, 92.67877403863862)(39.443742098609356, 93.11979670752942)(39.519595448798995, 93.09406013319324)(39.59544879898863, 92.87773250765544)(39.67130214917826, 92.77344929798315)(39.74715549936789, 93.033322697605)(39.823008849557525, 93.17340352337817)(39.89886219974716, 93.22061860595797)(39.97471554993679, 93.33957991298318)(40.05056890012643, 92.9616446403866)(40.126422250316054, 92.47780731156307)(40.20227560050569, 93.2881895226555)(40.278128950695326, 93.47553127473108)(40.35398230088496, 93.58094928970588)(40.4298356510746, 93.95059652129412)(40.50568900126422, 93.06422220184034)(40.58154235145386, 93.66548330249577)(40.65739570164349, 94.06277322363867)(40.73324905183313, 94.08419757068062)(40.80910240202276, 94.05925551154625)(40.88495575221239, 94.07157832069747)(40.96080910240203, 93.90816106493278)(41.036662452591656, 93.84539296934452)(41.112515802781296, 93.96037162097475)(41.18836915297093, 93.76888325)(41.26422250316056, 93.49664550135294)(41.34007585335019, 93.54770044717642)(41.415929203539825, 93.93662147507563)(41.49178255372946, 94.2451973572773)(41.5676359039191, 94.07521583810926)(41.64348925410873, 93.77327438545377)(41.71934260429836, 93.87694665766382)(41.795195954487994, 93.85322221947895)(41.871049304677626, 93.97037799727731)(41.94690265486726, 94.02087050140331)(42.02275600505689, 93.51230356831088)(42.09860935524653, 93.58730039094117)(42.174462705436156, 93.83127008398316)(42.250316055625795, 93.30555750845764)(42.32616940581543, 93.42131423199152)(42.40202275600506, 93.07394263870336)(42.4778761061947, 94.18410236992372)(42.553729456384325, 94.54530725083896)(42.629582806573964, 94.33603456029064)(42.70543615676359, 94.60276152351287)(42.78128950695323, 94.19482175648717)(42.85714285714286, 94.26704054300853)(42.932996207332494, 94.25432333623083)(43.00884955752213, 94.20587884542739)(43.08470290771176, 94.26994760202567)(43.1605562579014, 94.84468236400858)(43.23640960809102, 94.57123529070083)(43.31226295828066, 94.64161095991452)(43.388116308470295, 94.57410465741881)(43.46396965865993, 94.58031394499996)(43.53982300884956, 94.45913004338462)(43.61567635903919, 94.47738817517093)(43.69152970922883, 94.3822689947522)(43.767383059418464, 94.5499654440427)(43.843236409608096, 94.83821522713674)(43.91908975979773, 94.62434989299997)(43.99494310998736, 94.83822342032475)(44.07079646017699, 94.26577091072416)(44.14664981036663, 94.0951682961293)(44.22250316055626, 94.01228981064654)(44.2983565107459, 94.48185392598275)(44.37420986093552, 94.4429624168362)(44.45006321112516, 94.15123079666375)(44.5259165613148, 93.91370701135347)(44.60176991150443, 93.73896715666379)(44.677623261694066, 94.78452568733623)(44.75347661188369, 95.50190917828452)(44.82932996207333, 96.0531356648362)(44.90518331226296, 96.00536222766382)(44.981036662452595, 96.91973260608617)(45.05689001264223, 90.99516578465654)(45.13274336283186, 91.1844760125757)(45.2085967130215, 91.68405994323231)(45.284450063211125, 91.36781217885861)(45.360303413400764, 90.33954940133334)(45.4361567635904, 91.4125731257374)(45.51201011378003, 90.94073567969694)(45.58786346396966, 91.59979996546468)(45.663716814159294, 91.39767919114144)(45.739570164348926, 92.4679725743839)(45.815423514538566, 92.31347992801011)(45.8912768647282, 92.23293424297982)(45.96713021491783, 91.98933170354549)(46.04298356510746, 92.10678542790912)(46.118836915297095, 91.99096946698992)(46.19469026548673, 92.37151696805053)(46.27054361567636, 92.20080114586868)(46.346396965866, 91.93191524887878)(46.422250316055624, 92.13024541583839)(46.498103666245264, 92.45888964657576)(46.573957016434896, 92.01625439697976)(46.64981036662453, 92.14529939886873)(46.72566371681417, 92.44464779810099)(46.80151706700379, 92.18038513403027)(46.87737041719343, 92.02248105272727)(46.95322376738306, 91.77638116134347)(47.0290771175727, 92.30035819698992)(47.10493046776233, 92.59281818135352)(47.18078381795196, 92.37254649827271)(47.2566371681416, 91.42036828632324)(47.33249051833123, 91.80013283336363)(47.408343868520866, 92.35839106965658)(47.4841972187105, 92.3380431379293)(47.56005056890013, 92.18105029701015)(47.63590391908976, 91.97197670401015)(47.711757269279396, 92.32241877478789)(47.78761061946903, 91.92014955867675)(47.86346396965867, 92.48190358502019)(47.9393173198483, 92.13288234680803)(48.01517067003793, 92.19451012269697)(48.091024020227565, 92.23484909969694)(48.1668773704172, 92.15408713679795)(48.24273072060683, 92.00583233730302)(48.31858407079646, 91.9967397420101)(48.3944374209861, 92.58492105285856)(48.470290771175726, 92.60412882732322)(48.546144121365366, 92.56092079805053)(48.621997471555, 92.00405265351513)(48.69785082174463, 91.86817174497983)(48.77370417193427, 92.11492213000001)(48.849557522123895, 92.67149299180805)(48.925410872313535, 92.58864825831319)(49.00126422250316, 92.44279961131315)(49.0771175726928, 92.39699224033336)(49.15297092288243, 92.66723250350508)(49.228824273072064, 92.66117071688889)(49.3046776232617, 92.3392397518788)(49.38053097345133, 92.1754613302828)(49.45638432364097, 92.77002124728283)(49.53223767383059, 92.45765629004036)(49.60809102402023, 92.3482824343131)(49.683944374209865, 92.61463143162625)(49.7597977243995, 92.98124945200001)(49.83565107458913, 92.52214743125256)(49.91150442477876, 92.73126267980804)(49.9873577749684, 92.33998315049496)(50.063211125158034, 93.0295975780707)(50.139064475347666, 93.07692176586872)(50.2149178255373, 92.91073112190912)(50.29077117572693, 93.20485956146938)(50.36662452591656, 92.84520826905154)(50.4424778761062, 93.09836294670102)(50.51833122629583, 92.64725558851045)(50.59418457648547, 92.92669296078128)(50.6700379266751, 92.96352683946874)(50.74589127686473, 92.77994786398956)(50.82174462705437, 92.81320144017711)(50.897597977244, 93.04733203453122)(50.97345132743364, 92.69723656886457)(51.04930467762326, 92.89527884398956)(51.1251580278129, 92.82268237620838)(51.20101137800253, 92.81252686360416)(51.276864728192166, 92.3075373829479)(51.352718078381805, 92.46678450746877)(51.42857142857143, 91.48757497397916)(51.50442477876107, 90.5879616579896)(51.580278128950695, 90.79563863808336)(51.656131479140335, 90.15442325191673)(51.73198482932997, 89.7665521682105)(51.8078381795196, 89.21662746804212)(51.88369152970923, 89.70499392517023)(51.959544879898864, 88.62777786873406)(52.035398230088504, 91.89229005124466)(52.111251580278136, 93.8747859101613)(52.18710493046777, 95.59279133683872)(52.2629582806574, 95.39282073232255)(52.33881163084703, 97.36836222407607)(52.414664981036665, 97.11612068118475)(52.4905183312263, 97.4848815793152)(52.56637168141593, 61.407006673)(52.64222503160557, 60.372863597000006)(52.7180783817952, 54.915182344)(52.793931731984834, 50.705249027)(52.86978508217447, 59.182684273)(52.9456384323641, 53.806003925)(53.02149178255374, 57.976047534)(53.097345132743364, 56.051000695)(53.173198482933, 56.331695507999996)(53.24905183312263, 57.639688088)(53.32490518331227, 50.567974109999994)(53.4007585335019, 48.364747571)(53.47661188369153, 60.83556798)(53.55246523388117, 60.198222462000004)(53.6283185840708, 60.816652074000004)(53.70417193426044, 53.941681208)(53.78002528445007, 49.132640081)(53.8558786346397, 54.001506749)(53.931731984829334, 53.62035854)(54.007585335018966, 55.641668894)(54.0834386852086, 62.206312610000005)(54.15929203539824, 63.325477135)(54.23514538558787, 36.54809556)(54.3109987357775, 54.316320152)(54.386852085967135, 61.661470704)(54.46270543615677, 59.578831127)(54.5385587863464, 56.949506544)(54.61441213653603, 55.489631113)(54.69026548672567, 57.420429475999995)(54.7661188369153, 57.18555144)(54.841972187104936, 57.035377405999995)(54.91782553729457, 56.685028861999996)(54.9936788874842, 60.656701674000004)(55.06953223767384, 46.333849461)(55.145385587863466, 48.265851591)(55.221238938053105, 55.727940264000004)(55.29709228824273, 59.613653971999994)(55.37294563843237, 56.657207511)(55.448798988622, 55.037852854)(55.524652338811634, 57.508890424)(55.600505689001274, 59.762057833)(55.6763590391909, 60.649796502)(55.75221238938054, 46.082852302)(55.828065739570164, 51.797457115)(55.9039190897598, 58.551343816)(55.979772439949436, 53.920243572000004)(56.05562579013907, 57.916775329)(56.1314791403287, 57.556940951)(56.20733249051833, 51.349796571)(56.28318584070797, 48.162284012)(56.359039190897604, 61.36949072)(56.43489254108724, 60.037157236)(56.51074589127687, 60.481663083)(56.5865992414665, 59.786735834)(56.662452591656134, 55.38663807)(56.73830594184577, 46.821031831999996)(56.8141592920354, 50.074133122)(56.89001264222504, 60.308360272)(56.96586599241467, 59.284612668)(57.0417193426043, 56.228817750999994)(57.11757269279394, 55.929678177)(57.19342604298357, 56.001832747)(57.26927939317321, 55.917592686)(57.34513274336283, 60.207046373)(57.42098609355247, 55.115267677000006)(57.496839443742104, 45.862569784)(57.572692793931736, 49.349789623999996)(57.648546144121376, 61.886972379999996)(57.724399494311, 62.979266667999994)(57.80025284450064, 61.621178012)(57.876106194690266, 42.883099757)(57.951959544879905, 51.306319384)(58.02781289506954, 59.040951645)(58.10366624525917, 58.687175902)(58.1795195954488, 60.227741783)(58.255372945638435, 61.511981992)(58.331226295828074, 45.253847641)(58.407079646017706, 51.968973872999996)(58.48293299620734, 57.938940093)(58.55878634639697, 54.621627129)(58.6346396965866, 58.437110992)(58.710493046776236, 49.732046413)(58.78634639696587, 55.647768883)(58.8621997471555, 61.700035696)(58.93805309734514, 59.017241547)(59.01390644753477, 58.152557126)(59.089759797724405, 60.134374011)(59.16561314791404, 46.892870374)(59.24146649810367, 47.705171667)(59.31731984829331, 52.797324032999995)(59.393173198482934, 57.802108386)(59.46902654867257, 60.271715699)(59.5448798988622, 58.903286078)(59.62073324905184, 54.142437201)(59.69658659924148, 51.33029341300001)(59.7724399494311, 43.95751060400001)(59.84829329962074, 61.167189992999994)(59.92414664981037, 60.329586746)(60.0, 60.734657711000004)
            };
            \addplot[color=blue, mark=none,name path=A] coordinates { %% MAX value
            (0.0, 114.800805424)(0.07585335018963338, 116.76394521)(0.15170670037926676, 116.23121251200001)(0.22756005056890014, 115.714902896)(0.3034134007585335, 115.690407895)(0.3792667509481669, 115.067683659)(0.45512010113780027, 118.587473354)(0.5309734513274336, 115.707735681)(0.606826801517067, 115.503071567)(0.6826801517067005, 118.168038943)(0.7585335018963338, 114.213785199)(0.8343868520859672, 114.34193961)(0.9102402022756005, 117.65613594999999)(0.986093552465234, 114.023980338)(1.0619469026548671, 113.95631120899999)(1.1378002528445006, 115.807559342)(1.213653603034134, 114.517411769)(1.2895069532237675, 116.854342929)(1.365360303413401, 116.35590429000001)(1.4412136536030342, 115.233479742)(1.5170670037926677, 115.968453637)(1.5929203539823011, 117.237212653)(1.6687737041719344, 115.502943881)(1.7446270543615676, 115.687032523)(1.820480404551201, 114.789752516)(1.8963337547408345, 115.06040676399999)(1.972187104930468, 115.595059276)(2.0480404551201015, 117.562836124)(2.1238938053097343, 116.706294583)(2.199747155499368, 117.412487264)(2.275600505689001, 116.120519335)(2.351453855878635, 117.849586675)(2.427307206068268, 117.35887335299999)(2.503160556257902, 115.63644907)(2.579013906447535, 117.375241672)(2.6548672566371687, 116.89273440599999)(2.730720606826802, 115.936738445)(2.806573957016435, 116.858495678)(2.8824273072060684, 116.693256031)(2.9582806573957017, 117.835306008)(3.0341340075853354, 117.316545431)(3.1099873577749686, 118.31210833)(3.1858407079646023, 119.519010614)(3.2616940581542355, 118.43882072)(3.3375474083438688, 118.934618452)(3.413400758533502, 118.351434768)(3.4892541087231352, 118.312492341)(3.565107458912769, 117.871608332)(3.640960809102402, 118.126791154)(3.716814159292036, 117.786921296)(3.792667509481669, 117.73701392199999)(3.8685208596713023, 117.982743091)(3.944374209860936, 119.47668116899999)(4.020227560050569, 118.789336882)(4.096080910240203, 117.72401206699999)(4.171934260429836, 118.50904624500001)(4.2477876106194685, 118.879966016)(4.323640960809103, 119.790609871)(4.399494310998736, 117.426519093)(4.47534766118837, 115.97948710300001)(4.551201011378002, 119.070859665)(4.6270543615676365, 118.345280656)(4.70290771175727, 116.95628904399999)(4.778761061946904, 116.265252762)(4.854614412136536, 119.720055472)(4.9304677623261695, 115.86972923299999)(5.006321112515804, 119.786293245)(5.082174462705436, 119.347159347)(5.15802781289507, 118.47926173399999)(5.233881163084703, 117.93456223199999)(5.309734513274337, 118.376995135)(5.38558786346397, 118.402830309)(5.461441213653604, 119.084646986)(5.537294563843237, 117.53064245099999)(5.61314791403287, 118.418866341)(5.689001264222504, 117.901184096)(5.764854614412137, 117.878442417)(5.840707964601771, 118.76332621099999)(5.916561314791403, 117.354344349)(5.9924146649810375, 118.563199358)(6.068268015170671, 116.672263907)(6.144121365360304, 116.627723541)(6.219974715549937, 116.33591792300001)(6.29582806573957, 117.15885648)(6.371681415929205, 118.157925465)(6.447534766118837, 118.594198589)(6.523388116308471, 117.08629067800001)(6.599241466498104, 117.71225080900001)(6.6750948166877375, 119.207029488)(6.750948166877371, 119.111155124)(6.826801517067004, 117.75810077099999)(6.902654867256638, 116.34515218000001)(6.9785082174462705, 117.505179331)(7.054361567635905, 116.90929561800002)(7.130214917825538, 120.27324568499999)(7.206068268015172, 119.062358336)(7.281921618204804, 118.654703117)(7.357774968394438, 118.66350985300001)(7.433628318584072, 119.316678279)(7.509481668773706, 115.649356881)(7.585335018963338, 115.569337367)(7.661188369152971, 114.40434463900002)(7.737041719342605, 114.379832128)(7.812895069532239, 114.98217566599999)(7.888748419721872, 115.355147749)(7.964601769911505, 115.920255486)(8.040455120101138, 114.778001187)(8.116308470290772, 114.888819314)(8.192161820480406, 115.55591979)(8.268015170670038, 114.785448234)(8.343868520859672, 116.33704813099999)(8.419721871049305, 116.80653446699999)(8.495575221238937, 116.246216043)(8.571428571428573, 115.982612478)(8.647281921618205, 115.576303176)(8.72313527180784, 115.743991386)(8.798988621997472, 115.440647944)(8.874841972187106, 116.469953997)(8.95069532237674, 118.22380921000001)(9.026548672566372, 115.71067816099999)(9.102402022756005, 116.658195459)(9.178255372945639, 115.790784508)(9.254108723135273, 116.11662609)(9.329962073324905, 116.88474878599999)(9.40581542351454, 115.805338673)(9.481668773704172, 115.59327418400001)(9.557522123893808, 115.418178106)(9.63337547408344, 115.404636636)(9.709228824273072, 116.498167809)(9.785082174462707, 114.844599407)(9.860935524652339, 115.818686751)(9.936788874841973, 116.81808779500001)(10.012642225031607, 115.726340051)(10.08849557522124, 115.020758001)(10.164348925410872, 114.295679194)(10.240202275600508, 115.06360006599999)(10.31605562579014, 115.49023532)(10.391908975979774, 115.697944212)(10.467762326169407, 117.212965389)(10.543615676359039, 118.354486714)(10.619469026548675, 117.046156644)(10.695322376738307, 117.983535916)(10.77117572692794, 116.16122866699999)(10.847029077117574, 116.028484768)(10.922882427307208, 115.154960559)(10.99873577749684, 117.185266933)(11.074589127686474, 117.289907185)(11.150442477876107, 115.717847386)(11.22629582806574, 116.607991516)(11.302149178255375, 117.59455306299999)(11.378002528445007, 117.737489031)(11.453855878634641, 117.368270182)(11.529709228824274, 117.910390265)(11.605562579013906, 117.299848781)(11.681415929203542, 116.44078075499999)(11.757269279393174, 116.28201471700001)(11.833122629582807, 114.80319845300001)(11.90897597977244, 116.923073682)(11.984829329962075, 118.47329324)(12.060682680151707, 118.222911988)(12.136536030341341, 118.048089311)(12.212389380530974, 118.251189931)(12.288242730720608, 117.776778751)(12.364096080910242, 117.2952153)(12.439949431099874, 116.523655834)(12.515802781289509, 117.84284853899999)(12.59165613147914, 116.688341653)(12.667509481668775, 117.23746117099999)(12.74336283185841, 116.485964529)(12.819216182048041, 117.9471876)(12.895069532237674, 118.135815164)(12.970922882427308, 117.443483629)(13.046776232616942, 116.460121108)(13.122629582806574, 117.651422797)(13.198482932996209, 119.403146796)(13.274336283185841, 118.95850612800001)(13.350189633375475, 118.818049112)(13.42604298356511, 118.76874247399999)(13.501896333754742, 119.125044292)(13.577749683944376, 118.72834896900001)(13.653603034134008, 118.76984752300001)(13.729456384323642, 119.78868050199999)(13.805309734513276, 119.127415837)(13.881163084702909, 118.688597306)(13.957016434892541, 118.81067882400001)(14.032869785082175, 119.785162445)(14.10872313527181, 119.507369673)(14.184576485461443, 118.89531419900001)(14.260429835651076, 118.52945264)(14.336283185840708, 117.870807418)(14.412136536030344, 119.27701843099999)(14.487989886219976, 117.545161238)(14.563843236409609, 118.580022034)(14.639696586599243, 119.622408863)(14.715549936788875, 118.578070186)(14.79140328697851, 119.325630958)(14.867256637168143, 118.670692716)(14.943109987357776, 119.477885722)(15.018963337547412, 117.482086284)(15.094816687737044, 117.38054702699999)(15.170670037926676, 117.07647925399999)(15.24652338811631, 115.413684918)(15.322376738305943, 115.911416782)(15.398230088495575, 113.57440522799999)(15.47408343868521, 117.39054645799999)(15.549936788874842, 115.070695664)(15.625790139064478, 114.700199384)(15.701643489254112, 115.02981944300001)(15.777496839443744, 115.140276086)(15.853350189633378, 114.46430944)(15.92920353982301, 114.442794896)(16.005056890012643, 117.802820139)(16.080910240202275, 116.457041482)(16.15676359039191, 115.529815465)(16.232616940581543, 115.143522378)(16.30847029077118, 117.009693121)(16.38432364096081, 119.252330242)(16.460176991150444, 118.03026300100001)(16.536030341340076, 117.21006881599999)(16.611883691529712, 117.629217083)(16.687737041719345, 118.167430915)(16.763590391908977, 118.19252497600002)(16.83944374209861, 117.717555685)(16.91529709228824, 117.69374825599999)(16.991150442477874, 117.66887642900001)(17.067003792667514, 116.37311562000001)(17.142857142857146, 116.186569777)(17.21871049304678, 116.94130109)(17.29456384323641, 118.651422507)(17.370417193426043, 117.493023411)(17.44627054361568, 117.52911798599999)(17.52212389380531, 118.62554365999999)(17.597977243994944, 118.429601068)(17.673830594184576, 116.76234954899999)(17.749683944374212, 120.147425017)(17.825537294563844, 119.01891183800001)(17.90139064475348, 118.595909087)(17.977243994943112, 118.075012965)(18.053097345132745, 117.542075744)(18.128950695322377, 118.118726548)(18.20480404551201, 117.79589675700001)(18.280657395701645, 117.491422951)(18.356510745891278, 118.51876717799999)(18.432364096080914, 119.42357186400001)(18.508217446270546, 115.98190075)(18.58407079646018, 117.10510976500001)(18.65992414664981, 121.57144211)(18.735777496839447, 118.422208483)(18.81163084702908, 117.93696368799999)(18.88748419721871, 119.101855113)(18.963337547408344, 118.108217165)(19.039190897597976, 117.900281681)(19.115044247787615, 117.86526004500001)(19.190897597977248, 118.11929959700001)(19.26675094816688, 119.28585932300001)(19.342604298356513, 118.522382656)(19.418457648546145, 118.173903996)(19.494310998735777, 118.588474034)(19.570164348925413, 118.46843909100001)(19.646017699115045, 116.92120015)(19.721871049304678, 118.428368389)(19.797724399494314, 116.790382652)(19.873577749683946, 118.56159199300001)(19.94943109987358, 119.70349932799999)(20.025284450063214, 119.83076922400001)(20.101137800252847, 118.89289072700001)(20.17699115044248, 119.0265918)(20.25284450063211, 117.775791175)(20.328697850821744, 119.286605506)(20.40455120101138, 116.15510989100001)(20.480404551201016, 117.989177273)(20.556257901390648, 117.592219928)(20.63211125158028, 118.778143468)(20.707964601769913, 118.181654048)(20.78381795195955, 118.79187572199999)(20.85967130214918, 117.736931908)(20.935524652338813, 118.54317594099999)(21.011378002528446, 117.32872413500002)(21.087231352718078, 115.158247163)(21.163084702907714, 115.269489166)(21.23893805309735, 117.164858341)(21.314791403286982, 115.564165335)(21.390644753476614, 115.919855982)(21.466498103666247, 116.075414071)(21.54235145385588, 116.27375644600001)(21.61820480404551, 115.72459805599999)(21.694058154235147, 117.11168864499999)(21.76991150442478, 117.00764357)(21.845764854614416, 117.456972512)(21.921618204804048, 116.07168489)(21.99747155499368, 117.83229556)(22.073324905183316, 119.29495578199999)(22.14917825537295, 118.545100705)(22.22503160556258, 117.13449701)(22.300884955752213, 118.32634791800001)(22.376738305941846, 119.843944785)(22.45259165613148, 119.227080478)(22.528445006321114, 110.280864064)(22.60429835651075, 99.51217805600001)(22.680151706700382, 99.79849891900001)(22.756005056890015, 99.78803020999999)(22.831858407079647, 100.13547263999999)(22.907711757269283, 101.844625448)(22.983565107458915, 101.67553609500001)(23.059418457648547, 100.145630753)(23.13527180783818, 116.231262601)(23.211125158027812, 115.168087531)(23.286978508217448, 116.63842841299999)(23.362831858407084, 116.588671027)(23.438685208596716, 115.27203149)(23.51453855878635, 116.39689325299999)(23.59039190897598, 115.77553938599999)(23.666245259165613, 118.370973042)(23.74209860935525, 115.679493981)(23.81795195954488, 116.30068960599999)(23.893805309734514, 117.182223562)(23.96965865992415, 116.730238242)(24.045512010113782, 116.61451080200001)(24.121365360303415, 116.97122983599999)(24.19721871049305, 117.759383844)(24.273072060682683, 116.613842595)(24.348925410872315, 115.73582117699999)(24.424778761061948, 116.097051991)(24.50063211125158, 116.143890539)(24.576485461441216, 115.431337353)(24.65233881163085, 116.252437036)(24.728192161820484, 117.241161854)(24.804045512010116, 115.502762805)(24.87989886219975, 116.849543557)(24.95575221238938, 117.11292025)(25.031605562579017, 117.005761686)(25.10745891276865, 116.22316683999999)(25.18331226295828, 118.27598695)(25.259165613147914, 119.788092518)(25.33501896333755, 118.78183742799999)(25.410872313527186, 117.106136617)(25.48672566371682, 117.407414643)(25.56257901390645, 118.496048935)(25.638432364096083, 119.078519343)(25.714285714285715, 118.10412686699999)(25.790139064475348, 118.04202986199999)(25.865992414664984, 119.024021859)(25.941845764854616, 117.92784044)(26.017699115044252, 117.803016273)(26.093552465233884, 117.621212023)(26.169405815423517, 116.524855376)(26.24525916561315, 117.585442289)(26.321112515802785, 117.34548350200001)(26.396965865992417, 118.396355374)(26.47281921618205, 117.21917000799999)(26.548672566371682, 117.176439511)(26.624525916561314, 117.24261378099999)(26.70037926675095, 117.42062473200001)(26.776232616940586, 118.93788304899999)(26.85208596713022, 117.235245854)(26.92793931731985, 117.822851484)(27.003792667509483, 118.276366284)(27.07964601769912, 119.380132326)(27.15549936788875, 119.679151812)(27.231352718078384, 118.783615743)(27.307206068268016, 118.581653989)(27.38305941845765, 119.638836747)(27.458912768647284, 118.758537811)(27.53476611883692, 118.250739095)(27.610619469026553, 119.06718353900001)(27.686472819216185, 118.19513112799999)(27.762326169405817, 117.345308443)(27.83817951959545, 117.912347689)(27.914032869785082, 118.603116486)(27.989886219974718, 118.363011224)(28.06573957016435, 116.80866832999999)(28.141592920353986, 120.07279408000001)(28.21744627054362, 118.012795003)(28.29329962073325, 118.24637893900001)(28.369152970922887, 118.73527817899999)(28.44500632111252, 118.01885732000001)(28.52085967130215, 118.459838571)(28.596713021491784, 118.418912398)(28.672566371681416, 117.486548387)(28.748419721871052, 118.324206778)(28.824273072060688, 118.192912732)(28.90012642225032, 118.571471426)(28.975979772439953, 117.712706302)(29.051833122629585, 119.110357299)(29.127686472819217, 117.525498468)(29.203539823008853, 118.806335619)(29.279393173198486, 118.604633294)(29.355246523388118, 119.003505187)(29.43109987357775, 119.166029283)(29.506953223767386, 118.389571294)(29.58280657395702, 119.225293365)(29.658659924146654, 119.509351267)(29.734513274336287, 119.112732707)(29.81036662452592, 119.65666258300001)(29.88621997471555, 118.382185109)(29.962073324905184, 117.867227413)(30.037926675094823, 116.65523795200001)(30.113780025284452, 117.275848492)(30.189633375474088, 118.45950736500001)(30.265486725663717, 119.16119592199999)(30.341340075853353, 117.307085758)(30.417193426042985, 117.439582322)(30.49304677623262, 116.20767966700001)(30.568900126422257, 117.80288060800001)(30.644753476611886, 119.856561045)(30.72060682680152, 119.125247801)(30.79646017699115, 119.560359902)(30.872313527180786, 119.09628036100001)(30.94816687737042, 117.418585907)(31.024020227560055, 119.178591423)(31.099873577749683, 117.609734933)(31.17572692793932, 129.55194684100002)(31.251580278128955, 117.41042264000001)(31.327433628318587, 117.265875716)(31.403286978508223, 118.16105389500001)(31.479140328697852, 115.846179001)(31.554993678887488, 116.97160934099999)(31.630847029077117, 118.576332549)(31.706700379266756, 117.968656655)(31.782553729456385, 117.297594364)(31.85840707964602, 115.46523415899999)(31.934260429835657, 117.300742665)(32.010113780025286, 118.234928438)(32.085967130214925, 119.50495388600001)(32.16182048040455, 118.782925613)(32.23767383059419, 119.364237973)(32.31352718078382, 117.63105965)(32.389380530973455, 117.33032343299999)(32.46523388116309, 117.99830141599999)(32.54108723135272, 118.235278905)(32.61694058154236, 119.65967882)(32.692793931731984, 118.452284181)(32.76864728192162, 119.065951177)(32.844500632111256, 118.784966992)(32.92035398230089, 117.498591449)(32.99620733249052, 118.305059033)(33.07206068268015, 116.035546123)(33.147914032869785, 118.73924611)(33.223767383059425, 116.90958051999999)(33.29962073324906, 118.300581377)(33.37547408343869, 116.947832229)(33.45132743362832, 116.138602942)(33.527180783817954, 116.37633851000001)(33.603034134007586, 117.40821093099999)(33.67888748419722, 118.57304123600001)(33.75474083438686, 117.358815888)(33.83059418457648, 118.42085503199999)(33.90644753476612, 118.023743257)(33.98230088495575, 119.18200722)(34.05815423514539, 118.06545292199999)(34.13400758533503, 117.965817983)(34.20986093552465, 117.989290731)(34.28571428571429, 119.41717753900001)(34.36156763590392, 118.56357132699999)(34.43742098609356, 118.650371671)(34.51327433628319, 117.938550799)(34.58912768647282, 120.185857995)(34.664981036662454, 119.035453343)(34.740834386852086, 118.235274576)(34.816687737041725, 116.66268392699999)(34.89254108723136, 118.12421251)(34.96839443742099, 117.96288292)(35.04424778761062, 118.87367191899999)(35.120101137800255, 116.92576403300001)(35.19595448798989, 116.95268544999999)(35.27180783817953, 117.36863696)(35.34766118836915, 117.455735995)(35.42351453855879, 119.339078194)(35.499367888748424, 116.61144886900001)(35.575221238938056, 116.327846351)(35.65107458912769, 117.30480613699999)(35.72692793931732, 118.716365151)(35.80278128950696, 117.806731171)(35.878634639696585, 115.394687214)(35.954487989886225, 118.21688887500001)(36.03034134007585, 116.611685642)(36.10619469026549, 115.142717337)(36.18204804045513, 116.304749818)(36.257901390644754, 115.615798209)(36.333754740834394, 116.17515124900001)(36.40960809102402, 118.24626712400001)(36.48546144121366, 117.27320111099999)(36.56131479140329, 117.515578395)(36.63716814159292, 117.333795463)(36.713021491782555, 117.49726772400001)(36.78887484197219, 117.29285789900001)(36.86472819216183, 116.905214098)(36.94058154235145, 118.09482815499999)(37.01643489254109, 116.141714614)(37.092288242730724, 117.445801639)(37.16814159292036, 117.55426784899998)(37.24399494310999, 119.57076774400001)(37.31984829329962, 119.087389065)(37.395701643489254, 118.735914378)(37.47155499367889, 118.69121713)(37.547408343868526, 115.342569709)(37.62326169405816, 117.43236448399999)(37.69911504424779, 112.340515966)(37.77496839443742, 113.984959418)(37.85082174462706, 110.502215996)(37.92667509481669, 110.820129002)(38.00252844500633, 113.941200665)(38.07838179519595, 117.10721425499999)(38.15423514538559, 117.134507245)(38.23008849557523, 117.53012912599999)(38.305941845764856, 117.971828195)(38.381795195954496, 117.19492797299999)(38.45764854614412, 115.968809294)(38.53350189633376, 117.09674997100001)(38.60935524652339, 118.316544139)(38.685208596713025, 117.00506969899999)(38.76106194690266, 116.44329229)(38.83691529709229, 119.320928533)(38.91276864728193, 117.95879189300001)(38.988621997471554, 122.16467353300001)(39.064475347661194, 115.657065493)(39.140328697850826, 116.083332559)(39.21618204804046, 114.810592153)(39.29203539823009, 116.865268254)(39.36788874841972, 117.38566740099999)(39.443742098609356, 119.10783480199999)(39.519595448798995, 117.76094671199999)(39.59544879898863, 118.490433053)(39.67130214917826, 118.23947649899999)(39.74715549936789, 117.60078931000001)(39.823008849557525, 119.151891476)(39.89886219974716, 116.514672407)(39.97471554993679, 116.172834897)(40.05056890012643, 115.001324199)(40.126422250316054, 116.996624974)(40.20227560050569, 114.480005172)(40.278128950695326, 115.73348981800001)(40.35398230088496, 115.469083864)(40.4298356510746, 116.93651982)(40.50568900126422, 117.38652241)(40.58154235145386, 117.822070514)(40.65739570164349, 119.22398289200001)(40.73324905183313, 116.610164296)(40.80910240202276, 116.927554851)(40.88495575221239, 117.861868002)(40.96080910240203, 117.593682755)(41.036662452591656, 118.102980214)(41.112515802781296, 117.94134618)(41.18836915297093, 117.49855079)(41.26422250316056, 117.716375553)(41.34007585335019, 117.408223714)(41.415929203539825, 116.814612931)(41.49178255372946, 117.120168354)(41.5676359039191, 116.25157297199999)(41.64348925410873, 115.536549653)(41.71934260429836, 117.16747824000001)(41.795195954487994, 117.90922991800001)(41.871049304677626, 116.874405085)(41.94690265486726, 117.046372721)(42.02275600505689, 118.259961921)(42.09860935524653, 118.452733336)(42.174462705436156, 118.88878420399999)(42.250316055625795, 118.460985459)(42.32616940581543, 116.85826543299999)(42.40202275600506, 115.63266648000001)(42.4778761061947, 118.543476773)(42.553729456384325, 118.61352886700001)(42.629582806573964, 117.068357942)(42.70543615676359, 116.772449523)(42.78128950695323, 117.00187143000001)(42.85714285714286, 116.515565546)(42.932996207332494, 116.271555963)(43.00884955752213, 116.255524986)(43.08470290771176, 117.556509929)(43.1605562579014, 117.425497566)(43.23640960809102, 117.11152633799999)(43.31226295828066, 116.977633758)(43.388116308470295, 116.19055108799999)(43.46396965865993, 118.05779859399999)(43.53982300884956, 117.212158125)(43.61567635903919, 116.190499096)(43.69152970922883, 116.766016635)(43.767383059418464, 115.848295428)(43.843236409608096, 117.507773778)(43.91908975979773, 116.473029917)(43.99494310998736, 116.444084697)(44.07079646017699, 115.600072927)(44.14664981036663, 114.094514071)(44.22250316055626, 117.44755488499999)(44.2983565107459, 115.792622072)(44.37420986093552, 114.994194546)(44.45006321112516, 115.79937135899999)(44.5259165613148, 117.48633362199999)(44.60176991150443, 117.40537290500001)(44.677623261694066, 120.764123892)(44.75347661188369, 120.243247807)(44.82932996207333, 119.480016665)(44.90518331226296, 117.352179354)(44.981036662452595, 118.311132479)(45.05689001264223, 101.45818218000001)(45.13274336283186, 103.22630747299999)(45.2085967130215, 99.45856852700001)(45.284450063211125, 99.135340064)(45.360303413400764, 99.454503424)(45.4361567635904, 102.813882298)(45.51201011378003, 118.05086494199999)(45.58786346396966, 118.60162109499998)(45.663716814159294, 116.700994207)(45.739570164348926, 115.39385148599999)(45.815423514538566, 118.42605207)(45.8912768647282, 116.293272669)(45.96713021491783, 116.016948739)(46.04298356510746, 116.343744335)(46.118836915297095, 118.703311786)(46.19469026548673, 119.643728284)(46.27054361567636, 116.91776341500001)(46.346396965866, 115.56438806399998)(46.422250316055624, 116.313624148)(46.498103666245264, 116.090573686)(46.573957016434896, 116.266996294)(46.64981036662453, 116.92645218999999)(46.72566371681417, 115.82496455799999)(46.80151706700379, 115.183527271)(46.87737041719343, 116.06227153900001)(46.95322376738306, 115.449470175)(47.0290771175727, 116.842438572)(47.10493046776233, 118.24978055100001)(47.18078381795196, 116.44220479399999)(47.2566371681416, 116.189991469)(47.33249051833123, 116.780960518)(47.408343868520866, 117.802881637)(47.4841972187105, 118.146698146)(47.56005056890013, 117.485780686)(47.63590391908976, 116.74212031799999)(47.711757269279396, 117.79847523000001)(47.78761061946903, 117.31013449000001)(47.86346396965867, 119.157844942)(47.9393173198483, 116.734808466)(48.01517067003793, 117.02683739899999)(48.091024020227565, 117.723053347)(48.1668773704172, 117.21837104900001)(48.24273072060683, 116.41084835599999)(48.31858407079646, 117.98177124)(48.3944374209861, 117.74436934900001)(48.470290771175726, 117.19141911199999)(48.546144121365366, 116.37598184)(48.621997471555, 115.83429349)(48.69785082174463, 115.78646301699999)(48.77370417193427, 116.01604088900001)(48.849557522123895, 116.67385181)(48.925410872313535, 120.67801946099999)(49.00126422250316, 117.89522902)(49.0771175726928, 117.61443998600001)(49.15297092288243, 116.03781617599999)(49.228824273072064, 117.052075761)(49.3046776232617, 116.830404703)(49.38053097345133, 116.68425443000001)(49.45638432364097, 116.98009898800001)(49.53223767383059, 116.23647160899999)(49.60809102402023, 115.830671313)(49.683944374209865, 116.29366906599999)(49.7597977243995, 116.39508811900001)(49.83565107458913, 117.08221911000001)(49.91150442477876, 115.876953473)(49.9873577749684, 116.78038087499999)(50.063211125158034, 119.991960141)(50.139064475347666, 117.146088326)(50.2149178255373, 118.42228797800001)(50.29077117572693, 118.670072313)(50.36662452591656, 117.970194088)(50.4424778761062, 116.86185789)(50.51833122629583, 117.659058637)(50.59418457648547, 116.34380759300001)(50.6700379266751, 115.58450407000001)(50.74589127686473, 116.696639927)(50.82174462705437, 115.873724553)(50.897597977244, 115.646048945)(50.97345132743364, 115.234047158)(51.04930467762326, 114.944462569)(51.1251580278129, 115.762907249)(51.20101137800253, 115.153141013)(51.276864728192166, 116.172361001)(51.352718078381805, 116.241943971)(51.42857142857143, 116.843890441)(51.50442477876107, 117.028487229)(51.580278128950695, 116.17500622899999)(51.656131479140335, 116.960787923)(51.73198482932997, 115.93340188699999)(51.8078381795196, 116.414536853)(51.88369152970923, 115.816987444)(51.959544879898864, 115.253861579)(52.035398230088504, 119.97756459499999)(52.111251580278136, 118.50052700799999)(52.18710493046777, 118.798534914)(52.2629582806574, 119.13707065700001)(52.33881163084703, 118.681723159)(52.414664981036665, 118.14516407299999)(52.4905183312263, 119.703164485)(52.56637168141593, 61.407006673)(52.64222503160557, 60.372863597000006)(52.7180783817952, 54.915182344)(52.793931731984834, 50.705249027)(52.86978508217447, 59.182684273)(52.9456384323641, 53.806003925)(53.02149178255374, 57.976047534)(53.097345132743364, 56.051000695)(53.173198482933, 56.331695507999996)(53.24905183312263, 57.639688088)(53.32490518331227, 50.567974109999994)(53.4007585335019, 48.364747571)(53.47661188369153, 60.83556798)(53.55246523388117, 60.198222462000004)(53.6283185840708, 60.816652074000004)(53.70417193426044, 53.941681208)(53.78002528445007, 49.132640081)(53.8558786346397, 54.001506749)(53.931731984829334, 53.62035854)(54.007585335018966, 55.641668894)(54.0834386852086, 62.206312610000005)(54.15929203539824, 63.325477135)(54.23514538558787, 36.54809556)(54.3109987357775, 54.316320152)(54.386852085967135, 61.661470704)(54.46270543615677, 59.578831127)(54.5385587863464, 56.949506544)(54.61441213653603, 55.489631113)(54.69026548672567, 57.420429475999995)(54.7661188369153, 57.18555144)(54.841972187104936, 57.035377405999995)(54.91782553729457, 56.685028861999996)(54.9936788874842, 60.656701674000004)(55.06953223767384, 46.333849461)(55.145385587863466, 48.265851591)(55.221238938053105, 55.727940264000004)(55.29709228824273, 59.613653971999994)(55.37294563843237, 56.657207511)(55.448798988622, 55.037852854)(55.524652338811634, 57.508890424)(55.600505689001274, 59.762057833)(55.6763590391909, 60.649796502)(55.75221238938054, 46.082852302)(55.828065739570164, 51.797457115)(55.9039190897598, 58.551343816)(55.979772439949436, 53.920243572000004)(56.05562579013907, 57.916775329)(56.1314791403287, 57.556940951)(56.20733249051833, 51.349796571)(56.28318584070797, 48.162284012)(56.359039190897604, 61.36949072)(56.43489254108724, 60.037157236)(56.51074589127687, 60.481663083)(56.5865992414665, 59.786735834)(56.662452591656134, 55.38663807)(56.73830594184577, 46.821031831999996)(56.8141592920354, 50.074133122)(56.89001264222504, 60.308360272)(56.96586599241467, 59.284612668)(57.0417193426043, 56.228817750999994)(57.11757269279394, 55.929678177)(57.19342604298357, 56.001832747)(57.26927939317321, 55.917592686)(57.34513274336283, 60.207046373)(57.42098609355247, 55.115267677000006)(57.496839443742104, 45.862569784)(57.572692793931736, 49.349789623999996)(57.648546144121376, 61.886972379999996)(57.724399494311, 62.979266667999994)(57.80025284450064, 61.621178012)(57.876106194690266, 42.883099757)(57.951959544879905, 51.306319384)(58.02781289506954, 59.040951645)(58.10366624525917, 58.687175902)(58.1795195954488, 60.227741783)(58.255372945638435, 61.511981992)(58.331226295828074, 45.253847641)(58.407079646017706, 51.968973872999996)(58.48293299620734, 57.938940093)(58.55878634639697, 54.621627129)(58.6346396965866, 58.437110992)(58.710493046776236, 49.732046413)(58.78634639696587, 55.647768883)(58.8621997471555, 61.700035696)(58.93805309734514, 59.017241547)(59.01390644753477, 58.152557126)(59.089759797724405, 60.134374011)(59.16561314791404, 46.892870374)(59.24146649810367, 47.705171667)(59.31731984829331, 52.797324032999995)(59.393173198482934, 57.802108386)(59.46902654867257, 60.271715699)(59.5448798988622, 58.903286078)(59.62073324905184, 54.142437201)(59.69658659924148, 51.33029341300001)(59.7724399494311, 43.95751060400001)(59.84829329962074, 61.167189992999994)(59.92414664981037, 60.329586746)(60.0, 60.734657711000004)
            };
            \addplot[color=blue, mark=none,name path=B] coordinates { %% MIN value
            (0.0, 44.200707942)(0.07585335018963338, 45.91487489400001)(0.15170670037926676, 37.004870037)(0.22756005056890014, 48.688665599)(0.3034134007585335, 45.371447649)(0.3792667509481669, 38.00123746)(0.45512010113780027, 38.257675322)(0.5309734513274336, 45.96097881)(0.606826801517067, 52.882748472)(0.6826801517067005, 48.278497267000006)(0.7585335018963338, 46.490009828999995)(0.8343868520859672, 42.65320486)(0.9102402022756005, 31.217339017)(0.986093552465234, 48.30247563)(1.0619469026548671, 46.337624465999994)(1.1378002528445006, 38.636072942)(1.213653603034134, 50.790408346999996)(1.2895069532237675, 47.458350775)(1.365360303413401, 44.773653008)(1.4412136536030342, 44.685153431)(1.5170670037926677, 43.405809654)(1.5929203539823011, 45.272280020000004)(1.6687737041719344, 29.032137813)(1.7446270543615676, 41.755226495)(1.820480404551201, 12.864625373)(1.8963337547408345, 35.568018402999996)(1.972187104930468, 45.787120683)(2.0480404551201015, 44.24274489)(2.1238938053097343, 37.520633551)(2.199747155499368, 47.568754696)(2.275600505689001, 44.688293607)(2.351453855878635, 32.822095364)(2.427307206068268, 48.394749058)(2.503160556257902, 46.975320749)(2.579013906447535, 44.514114936)(2.6548672566371687, 46.930412745999995)(2.730720606826802, 41.707609593)(2.806573957016435, 46.039135045)(2.8824273072060684, 45.75491737)(2.9582806573957017, 46.011502869000005)(3.0341340075853354, 45.19443576)(3.1099873577749686, 56.055578168000004)(3.1858407079646023, 54.604470387)(3.2616940581542355, 46.681807727000006)(3.3375474083438688, 49.583215187)(3.413400758533502, 40.614568507)(3.4892541087231352, 45.025645677)(3.565107458912769, 50.69348148499999)(3.640960809102402, 38.437912921)(3.716814159292036, 47.491334239)(3.792667509481669, 43.362837728)(3.8685208596713023, 43.691647874)(3.944374209860936, 51.157412789)(4.020227560050569, 40.526996212)(4.096080910240203, 51.375001293000004)(4.171934260429836, 49.187863041)(4.2477876106194685, 38.580926672000004)(4.323640960809103, 45.329144324)(4.399494310998736, 36.507068646)(4.47534766118837, 42.341006817)(4.551201011378002, 53.572448756)(4.6270543615676365, 41.138630374)(4.70290771175727, 16.366055462)(4.778761061946904, 40.731772950999996)(4.854614412136536, 19.627017795)(4.9304677623261695, 54.389854651)(5.006321112515804, 37.815592197)(5.082174462705436, 51.18241957000001)(5.15802781289507, 52.349193484)(5.233881163084703, 44.069735974000004)(5.309734513274337, 38.454426069)(5.38558786346397, 52.105704742)(5.461441213653604, 44.424427957999995)(5.537294563843237, 36.148319049)(5.61314791403287, 46.147582436)(5.689001264222504, 44.268086406)(5.764854614412137, 39.203904005)(5.840707964601771, 45.605108725)(5.916561314791403, 52.724904021)(5.9924146649810375, 46.923060824000004)(6.068268015170671, 38.01242376)(6.144121365360304, 41.592722995)(6.219974715549937, 48.247602492999995)(6.29582806573957, 47.977282622999994)(6.371681415929205, 37.170060297999996)(6.447534766118837, 52.408325913)(6.523388116308471, 37.042021276)(6.599241466498104, 39.219073833)(6.6750948166877375, 37.703777981)(6.750948166877371, 16.651866216)(6.826801517067004, 13.087712414999999)(6.902654867256638, 29.435705732000002)(6.9785082174462705, 12.712734646)(7.054361567635905, 45.310640965)(7.130214917825538, 17.369523603999998)(7.206068268015172, 60.986815340999996)(7.281921618204804, 63.971223174)(7.357774968394438, 96.249232781)(7.433628318584072, 95.95571360699999)(7.509481668773706, 44.191265901)(7.585335018963338, 44.314209259)(7.661188369152971, 51.566979453)(7.737041719342605, 52.348149189)(7.812895069532239, 49.553282022)(7.888748419721872, 46.63689251)(7.964601769911505, 24.267099706000003)(8.040455120101138, 46.583315615)(8.116308470290772, 43.14822454)(8.192161820480406, 48.330037413999996)(8.268015170670038, 43.837203664)(8.343868520859672, 44.146558449000004)(8.419721871049305, 38.665793705)(8.495575221238937, 29.175101646999998)(8.571428571428573, 36.786815686)(8.647281921618205, 43.858163558)(8.72313527180784, 44.976573663)(8.798988621997472, 52.905179555000004)(8.874841972187106, 51.418991275)(8.95069532237674, 35.903368866)(9.026548672566372, 44.746405351)(9.102402022756005, 47.127105158000006)(9.178255372945639, 46.360283063)(9.254108723135273, 37.87288875)(9.329962073324905, 51.982143582000006)(9.40581542351454, 53.733056923)(9.481668773704172, 47.439860962)(9.557522123893808, 45.52693925)(9.63337547408344, 43.731203478999994)(9.709228824273072, 41.624446068000005)(9.785082174462707, 44.42214065)(9.860935524652339, 43.119729591)(9.936788874841973, 45.598075647)(10.012642225031607, 45.773582632)(10.08849557522124, 39.958037731)(10.164348925410872, 47.838540961999996)(10.240202275600508, 51.738302286999996)(10.31605562579014, 49.557843716)(10.391908975979774, 44.13665198300001)(10.467762326169407, 41.9208012)(10.543615676359039, 46.698637384)(10.619469026548675, 40.862349089)(10.695322376738307, 53.069767337)(10.77117572692794, 54.219216846)(10.847029077117574, 42.362648429)(10.922882427307208, 41.071992044)(10.99873577749684, 46.56150571799999)(11.074589127686474, 47.81970025299999)(11.150442477876107, 48.066771269)(11.22629582806574, 17.369132532000002)(11.302149178255375, 51.35992585300001)(11.378002528445007, 51.926964027000004)(11.453855878634641, 46.512714503000005)(11.529709228824274, 47.603638880999995)(11.605562579013906, 53.63800912400001)(11.681415929203542, 50.822564666)(11.757269279393174, 46.217810674999996)(11.833122629582807, 51.480558191)(11.90897597977244, 52.478991894)(11.984829329962075, 46.471099486)(12.060682680151707, 51.014479154)(12.136536030341341, 50.216301731)(12.212389380530974, 45.827127844)(12.288242730720608, 50.348270848)(12.364096080910242, 53.619048746000004)(12.439949431099874, 52.526994362)(12.515802781289509, 39.107448471)(12.59165613147914, 51.0587434)(12.667509481668775, 36.842006258)(12.74336283185841, 44.348377456)(12.819216182048041, 35.268393228)(12.895069532237674, 49.563263526)(12.970922882427308, 38.232153631)(13.046776232616942, 26.841996731000002)(13.122629582806574, 14.045590252)(13.198482932996209, 48.680990645)(13.274336283185841, 55.157487976)(13.350189633375475, 36.576151447)(13.42604298356511, 55.110399445)(13.501896333754742, 52.811761548)(13.577749683944376, 52.617517981999995)(13.653603034134008, 43.972868086)(13.729456384323642, 52.561530899)(13.805309734513276, 46.097072978)(13.881163084702909, 47.152990457)(13.957016434892541, 50.20595263)(14.032869785082175, 49.036058877)(14.10872313527181, 49.118726749)(14.184576485461443, 39.569948276)(14.260429835651076, 25.2139283)(14.336283185840708, 14.053904448)(14.412136536030344, 14.631023053)(14.487989886219976, 30.343637094)(14.563843236409609, 14.842420595)(14.639696586599243, 53.255331706)(14.715549936788875, 58.414493992000004)(14.79140328697851, 58.03074461599999)(14.867256637168143, 55.526068823)(14.943109987357776, 13.632346016000001)(15.018963337547412, 29.714264808)(15.094816687737044, 45.139397832)(15.170670037926676, 48.680692939)(15.24652338811631, 17.962709458)(15.322376738305943, 39.773036246000004)(15.398230088495575, 47.557062242)(15.47408343868521, 44.587429177000004)(15.549936788874842, 45.213350424999994)(15.625790139064478, 44.142110878)(15.701643489254112, 36.119338214)(15.777496839443744, 44.446677133)(15.853350189633378, 13.809023916000001)(15.92920353982301, 45.366997701)(16.005056890012643, 42.955645612)(16.080910240202275, 51.076706469)(16.15676359039191, 52.575171929999996)(16.232616940581543, 44.306121865)(16.30847029077118, 47.389813493000005)(16.38432364096081, 45.004997505000006)(16.460176991150444, 47.698823697)(16.536030341340076, 31.130280179)(16.611883691529712, 47.556368742000004)(16.687737041719345, 45.972397007000005)(16.763590391908977, 22.215363943)(16.83944374209861, 41.447108645)(16.91529709228824, 42.678802805000004)(16.991150442477874, 46.21542104299999)(17.067003792667514, 52.251545642)(17.142857142857146, 26.960996568)(17.21871049304678, 35.384340091)(17.29456384323641, 53.813431031)(17.370417193426043, 24.451501848)(17.44627054361568, 41.848084109999995)(17.52212389380531, 14.187720547)(17.597977243994944, 52.74510703)(17.673830594184576, 44.703972752)(17.749683944374212, 42.343926207)(17.825537294563844, 51.80962241099999)(17.90139064475348, 48.664270103999996)(17.977243994943112, 54.114009902999996)(18.053097345132745, 44.375529291999996)(18.128950695322377, 45.504765558)(18.20480404551201, 50.649784458999996)(18.280657395701645, 52.350646899)(18.356510745891278, 52.191201579)(18.432364096080914, 49.372249583)(18.508217446270546, 43.302023648)(18.58407079646018, 31.911848909999996)(18.65992414664981, 52.400833700999996)(18.735777496839447, 51.524535028)(18.81163084702908, 37.871392948)(18.88748419721871, 46.198252296999996)(18.963337547408344, 53.648522037)(19.039190897597976, 51.471309420000004)(19.115044247787615, 46.215666025999994)(19.190897597977248, 52.214272171999994)(19.26675094816688, 51.332341965)(19.342604298356513, 43.769464037999995)(19.418457648546145, 44.467501627)(19.494310998735777, 41.548904011000005)(19.570164348925413, 45.03272875100001)(19.646017699115045, 41.941852908)(19.721871049304678, 36.900441556)(19.797724399494314, 47.250427716000004)(19.873577749683946, 47.806805608999994)(19.94943109987358, 52.742575868)(20.025284450063214, 37.406678272)(20.101137800252847, 38.785540696)(20.17699115044248, 48.908106616)(20.25284450063211, 48.626308782)(20.328697850821744, 48.106082621)(20.40455120101138, 42.708393572999995)(20.480404551201016, 47.915969653000005)(20.556257901390648, 45.147259043)(20.63211125158028, 48.699170864)(20.707964601769913, 38.490934027)(20.78381795195955, 41.345458809)(20.85967130214918, 52.927560093)(20.935524652338813, 55.716789624)(21.011378002528446, 40.966748837000004)(21.087231352718078, 37.961069197)(21.163084702907714, 46.521223672)(21.23893805309735, 46.906927421000006)(21.314791403286982, 49.665293094000006)(21.390644753476614, 42.154497639)(21.466498103666247, 49.194603888)(21.54235145385588, 43.063420663)(21.61820480404551, 49.533184197000004)(21.694058154235147, 24.832143406)(21.76991150442478, 34.257420166)(21.845764854614416, 12.706278145)(21.921618204804048, 12.326729587)(21.99747155499368, 13.351813443000001)(22.073324905183316, 11.988614371999999)(22.14917825537295, 38.500136411999996)(22.22503160556258, 14.132496660000001)(22.300884955752213, 53.293918319)(22.376738305941846, 13.584057792)(22.45259165613148, 46.775740696)(22.528445006321114, 37.810023012)(22.60429835651075, 51.612878514)(22.680151706700382, 35.501539277999996)(22.756005056890015, 48.281791532)(22.831858407079647, 37.414819124)(22.907711757269283, 35.19166972)(22.983565107458915, 50.978101132999996)(23.059418457648547, 45.153432328)(23.13527180783818, 42.53787659300001)(23.211125158027812, 29.286289783)(23.286978508217448, 46.586134877000006)(23.362831858407084, 38.469918173)(23.438685208596716, 16.788202553)(23.51453855878635, 36.757033827)(23.59039190897598, 44.902317144)(23.666245259165613, 44.470074412)(23.74209860935525, 43.120350017)(23.81795195954488, 38.597889472000006)(23.893805309734514, 46.910167728000005)(23.96965865992415, 51.783506338)(24.045512010113782, 43.670220924)(24.121365360303415, 36.542291158000005)(24.19721871049305, 46.985688147999994)(24.273072060682683, 52.761761414999995)(24.348925410872315, 41.439278214)(24.424778761061948, 46.092068303999994)(24.50063211125158, 46.516787485)(24.576485461441216, 46.202085196)(24.65233881163085, 46.964608928000004)(24.728192161820484, 42.006783938999995)(24.804045512010116, 51.589902601000006)(24.87989886219975, 43.364228197)(24.95575221238938, 13.154825819)(25.031605562579017, 45.889318792999994)(25.10745891276865, 12.899197922)(25.18331226295828, 50.905813971)(25.259165613147914, 50.208702087999995)(25.33501896333755, 39.404988939)(25.410872313527186, 50.560814832)(25.48672566371682, 37.08996725)(25.56257901390645, 43.244591799)(25.638432364096083, 51.727641018)(25.714285714285715, 49.778469718)(25.790139064475348, 49.561996054000005)(25.865992414664984, 31.274724028)(25.941845764854616, 46.402519299999994)(26.017699115044252, 37.544564639)(26.093552465233884, 44.958624216000004)(26.169405815423517, 46.525410856)(26.24525916561315, 14.699325218999999)(26.321112515802785, 35.852483334)(26.396965865992417, 43.118585181)(26.47281921618205, 41.980915782)(26.548672566371682, 50.485643052)(26.624525916561314, 45.08763054)(26.70037926675095, 45.497832571)(26.776232616940586, 36.165296357)(26.85208596713022, 40.59612069)(26.92793931731985, 31.529212696000002)(27.003792667509483, 46.869143053)(27.07964601769912, 46.196820067)(27.15549936788875, 31.234385637000003)(27.231352718078384, 40.101020405999996)(27.307206068268016, 51.762530584000004)(27.38305941845765, 47.990331745)(27.458912768647284, 42.178970758999995)(27.53476611883692, 38.857332937)(27.610619469026553, 39.434931972)(27.686472819216185, 49.042212813)(27.762326169405817, 52.082097097)(27.83817951959545, 36.405573842)(27.914032869785082, 54.950193825)(27.989886219974718, 43.973824971999996)(28.06573957016435, 50.321880186)(28.141592920353986, 45.308179445)(28.21744627054362, 38.285638976)(28.29329962073325, 47.126700408)(28.369152970922887, 48.87053436000001)(28.44500632111252, 42.199600474)(28.52085967130215, 45.386254864)(28.596713021491784, 50.32091107)(28.672566371681416, 43.455269226000006)(28.748419721871052, 36.742297457)(28.824273072060688, 37.074780988)(28.90012642225032, 38.449854465)(28.975979772439953, 39.931796202)(29.051833122629585, 41.790161328)(29.127686472819217, 16.603709002)(29.203539823008853, 16.171359000000002)(29.279393173198486, 31.974667852)(29.355246523388118, 12.476479722)(29.43109987357775, 14.103431016000002)(29.506953223767386, 12.158542123)(29.58280657395702, 10.659851936999999)(29.658659924146654, 53.563779341)(29.734513274336287, 32.414613164)(29.81036662452592, 50.186006195)(29.88621997471555, 46.709342943)(29.962073324905184, 56.410853882)(30.037926675094823, 37.210927033000004)(30.113780025284452, 43.756009935)(30.189633375474088, 47.338076403)(30.265486725663717, 52.51891620000001)(30.341340075853353, 37.314833674)(30.417193426042985, 46.555293457999994)(30.49304677623262, 44.06121478)(30.568900126422257, 44.509191227)(30.644753476611886, 37.437919313)(30.72060682680152, 42.155331770000004)(30.79646017699115, 44.680680264)(30.872313527180786, 49.39397606)(30.94816687737042, 35.425217134)(31.024020227560055, 37.288663924)(31.099873577749683, 41.857346731)(31.17572692793932, 50.671384036999996)(31.251580278128955, 42.938996531)(31.327433628318587, 38.473646394)(31.403286978508223, 53.373043362)(31.479140328697852, 46.182635441)(31.554993678887488, 41.808887724)(31.630847029077117, 39.367266327)(31.706700379266756, 46.73140975)(31.782553729456385, 42.76542488)(31.85840707964602, 45.511097584)(31.934260429835657, 44.380881236)(32.010113780025286, 43.133285999)(32.085967130214925, 44.45884718000001)(32.16182048040455, 40.665414188)(32.23767383059419, 45.968252387)(32.31352718078382, 40.791108875999996)(32.389380530973455, 44.305145674)(32.46523388116309, 41.230989234999996)(32.54108723135272, 30.736677919)(32.61694058154236, 42.253305753)(32.692793931731984, 45.458920015000004)(32.76864728192162, 46.451713129999995)(32.844500632111256, 49.517216985000005)(32.92035398230089, 45.341855089000006)(32.99620733249052, 45.370329208)(33.07206068268015, 47.754314478)(33.147914032869785, 43.52081131600001)(33.223767383059425, 42.658063375)(33.29962073324906, 44.032604518999996)(33.37547408343869, 57.225445764999996)(33.45132743362832, 45.399979515)(33.527180783817954, 39.956482534)(33.603034134007586, 42.08492397)(33.67888748419722, 50.336937724)(33.75474083438686, 52.522933782)(33.83059418457648, 45.400554940999996)(33.90644753476612, 44.912616355)(33.98230088495575, 38.510736281)(34.05815423514539, 44.772259446)(34.13400758533503, 51.531436948)(34.20986093552465, 54.445027384)(34.28571428571429, 38.818127333)(34.36156763590392, 42.349976467000005)(34.43742098609356, 48.417324815)(34.51327433628319, 26.358102621)(34.58912768647282, 44.0804227)(34.664981036662454, 44.586690919)(34.740834386852086, 23.208922206000004)(34.816687737041725, 45.175102619)(34.89254108723136, 14.764668131)(34.96839443742099, 48.971607766)(35.04424778761062, 45.812979015)(35.120101137800255, 26.415779296)(35.19595448798989, 43.697676258)(35.27180783817953, 52.525468414)(35.34766118836915, 44.754294892999994)(35.42351453855879, 44.316554661999994)(35.499367888748424, 36.439651341)(35.575221238938056, 47.479019065)(35.65107458912769, 52.32801344800001)(35.72692793931732, 42.998881599)(35.80278128950696, 45.120187157000004)(35.878634639696585, 44.387194713999996)(35.954487989886225, 47.710041665)(36.03034134007585, 39.587840377)(36.10619469026549, 46.817067484999995)(36.18204804045513, 51.004935008)(36.257901390644754, 49.896639153)(36.333754740834394, 45.64822603)(36.40960809102402, 44.572315156)(36.48546144121366, 44.766196351)(36.56131479140329, 36.252441553)(36.63716814159292, 25.545693728)(36.713021491782555, 31.638530003)(36.78887484197219, 24.265451091000003)(36.86472819216183, 12.328350988)(36.94058154235145, 23.91306834)(37.01643489254109, 14.499638794)(37.092288242730724, 33.830842804999996)(37.16814159292036, 23.137096258)(37.24399494310999, 55.322319496999995)(37.31984829329962, 58.31744775)(37.395701643489254, 57.981604705)(37.47155499367889, 55.467016333)(37.547408343868526, 39.150830885)(37.62326169405816, 46.53741080099999)(37.69911504424779, 44.456568817999994)(37.77496839443742, 47.607233602)(37.85082174462706, 44.484878502)(37.92667509481669, 32.158311861)(38.00252844500633, 35.677660869)(38.07838179519595, 46.521513633)(38.15423514538559, 46.420446423)(38.23008849557523, 40.816530387)(38.305941845764856, 37.346297466)(38.381795195954496, 43.330906121000005)(38.45764854614412, 45.366180517)(38.53350189633376, 42.109306798999995)(38.60935524652339, 37.028781614)(38.685208596713025, 34.870194491999996)(38.76106194690266, 38.134912044000004)(38.83691529709229, 37.79576351)(38.91276864728193, 44.262906603000005)(38.988621997471554, 45.675908921)(39.064475347661194, 43.361401852)(39.140328697850826, 38.746295475000004)(39.21618204804046, 40.576690584000005)(39.29203539823009, 44.358018197999996)(39.36788874841972, 46.956728681)(39.443742098609356, 49.675541511)(39.519595448798995, 50.015292140999996)(39.59544879898863, 44.3752272)(39.67130214917826, 48.320708321000005)(39.74715549936789, 47.478242140000006)(39.823008849557525, 52.798320468)(39.89886219974716, 47.732963337)(39.97471554993679, 46.776370891)(40.05056890012643, 47.581361881)(40.126422250316054, 30.126151569999998)(40.20227560050569, 44.456164228999995)(40.278128950695326, 47.137626122)(40.35398230088496, 51.149710462)(40.4298356510746, 46.155145950999994)(40.50568900126422, 15.532556953)(40.58154235145386, 42.865683592)(40.65739570164349, 51.47741735)(40.73324905183313, 52.662287628)(40.80910240202276, 45.928434777)(40.88495575221239, 35.525176953)(40.96080910240203, 51.585700074)(41.036662452591656, 52.060903286)(41.112515802781296, 52.576583182)(41.18836915297093, 48.712659152)(41.26422250316056, 42.989296831)(41.34007585335019, 45.310400783000006)(41.415929203539825, 48.786317694000005)(41.49178255372946, 54.702259600999994)(41.5676359039191, 49.026452058)(41.64348925410873, 41.475650723)(41.71934260429836, 45.084572151)(41.795195954487994, 40.849079661000005)(41.871049304677626, 47.082920642000005)(41.94690265486726, 52.738546709)(42.02275600505689, 44.869498517000004)(42.09860935524653, 44.438447906)(42.174462705436156, 47.483474586999996)(42.250316055625795, 25.411114469)(42.32616940581543, 43.972725597)(42.40202275600506, 19.789824932)(42.4778761061947, 52.947838644)(42.553729456384325, 47.469490256)(42.629582806573964, 47.657351186)(42.70543615676359, 51.049390402)(42.78128950695323, 43.746702942000006)(42.85714285714286, 41.232750454)(42.932996207332494, 43.044651957999996)(43.00884955752213, 45.006127534)(43.08470290771176, 40.759488858)(43.1605562579014, 56.466239035)(43.23640960809102, 39.724344059)(43.31226295828066, 52.595132798)(43.388116308470295, 46.520827746)(43.46396965865993, 45.6724087)(43.53982300884956, 46.846497291000006)(43.61567635903919, 46.931100822999994)(43.69152970922883, 38.104369412)(43.767383059418464, 43.151983105999996)(43.843236409608096, 51.974215806000004)(43.91908975979773, 36.741228985)(43.99494310998736, 50.042939586)(44.07079646017699, 43.000782051)(44.14664981036663, 45.401468023)(44.22250316055626, 36.232058596)(44.2983565107459, 46.778474284)(44.37420986093552, 47.966278960000004)(44.45006321112516, 39.43154142)(44.5259165613148, 52.213988060999995)(44.60176991150443, 26.059595342)(44.677623261694066, 26.258536855000003)(44.75347661188369, 26.215093523)(44.82932996207333, 13.956776199)(44.90518331226296, 12.711653379000001)(44.981036662452595, 13.525850766000001)(45.05689001264223, 41.211360159)(45.13274336283186, 37.326872048)(45.2085967130215, 42.200458704)(45.284450063211125, 46.402423021)(45.360303413400764, 16.451367129)(45.4361567635904, 40.402309463)(45.51201011378003, 11.593940417999999)(45.58786346396966, 47.390680996)(45.663716814159294, 37.683997045)(45.739570164348926, 53.055887211000005)(45.815423514538566, 42.400022241)(45.8912768647282, 45.960201155)(45.96713021491783, 43.314552697)(46.04298356510746, 47.312474943)(46.118836915297095, 44.698369207999995)(46.19469026548673, 47.101986063)(46.27054361567636, 43.629538108999995)(46.346396965866, 46.452426321000004)(46.422250316055624, 46.529395515000004)(46.498103666245264, 48.672583610000004)(46.573957016434896, 35.358509435)(46.64981036662453, 40.546196314)(46.72566371681417, 45.041396226)(46.80151706700379, 47.524972881)(46.87737041719343, 45.143015941)(46.95322376738306, 36.099642904)(47.0290771175727, 41.612678773)(47.10493046776233, 50.039052659000006)(47.18078381795196, 43.908396747999994)(47.2566371681416, 27.789878133000002)(47.33249051833123, 36.250208409)(47.408343868520866, 44.454520992)(47.4841972187105, 46.272887402)(47.56005056890013, 37.893085755)(47.63590391908976, 38.317013)(47.711757269279396, 40.191467333000006)(47.78761061946903, 36.284377432)(47.86346396965867, 47.928756319)(47.9393173198483, 42.06378229800001)(48.01517067003793, 34.664425283)(48.091024020227565, 43.834039442000005)(48.1668773704172, 37.520679573)(48.24273072060683, 39.254401963999996)(48.31858407079646, 37.683751574)(48.3944374209861, 50.937282114)(48.470290771175726, 44.376164011)(48.546144121365366, 52.576766508999995)(48.621997471555, 40.536066051999995)(48.69785082174463, 39.793868272)(48.77370417193427, 36.45615778)(48.849557522123895, 47.099216674)(48.925410872313535, 41.720009516000005)(49.00126422250316, 46.663612975999996)(49.0771175726928, 44.232019296000004)(49.15297092288243, 51.219270046999995)(49.228824273072064, 36.829147077)(49.3046776232617, 43.051271912000004)(49.38053097345133, 41.866056215)(49.45638432364097, 50.937112165)(49.53223767383059, 38.833500046)(49.60809102402023, 46.548219316)(49.683944374209865, 37.797942277000004)(49.7597977243995, 53.661806816)(49.83565107458913, 47.160925207000005)(49.91150442477876, 46.91955796800001)(49.9873577749684, 34.846366728999996)(50.063211125158034, 43.3722908)(50.139064475347666, 49.798934274000004)(50.2149178255373, 43.37926362)(50.29077117572693, 48.112504431)(50.36662452591656, 43.384413326)(50.4424778761062, 50.669199854)(50.51833122629583, 36.58282483)(50.59418457648547, 52.772962427)(50.6700379266751, 50.766431611)(50.74589127686473, 48.979560750999994)(50.82174462705437, 39.151672863)(50.897597977244, 53.06707282799999)(50.97345132743364, 35.555812118)(51.04930467762326, 51.858749897)(51.1251580278129, 36.547346157999996)(51.20101137800253, 50.198810682)(51.276864728192166, 37.754925561)(51.352718078381805, 48.353827401000004)(51.42857142857143, 39.90823577)(51.50442477876107, 45.284258242)(51.580278128950695, 43.729371028)(51.656131479140335, 49.66771234)(51.73198482932997, 32.609218388)(51.8078381795196, 25.892369521)(51.88369152970923, 16.093757275)(51.959544879898864, 21.650588209000002)(52.035398230088504, 25.424844160000003)(52.111251580278136, 12.988660663000001)(52.18710493046777, 42.418347591)(52.2629582806574, 25.662706304)(52.33881163084703, 52.736165789)(52.414664981036665, 50.012774959)(52.4905183312263, 49.26762916999999)(52.56637168141593, 61.407006673)(52.64222503160557, 60.372863597000006)(52.7180783817952, 54.915182344)(52.793931731984834, 50.705249027)(52.86978508217447, 59.182684273)(52.9456384323641, 53.806003925)(53.02149178255374, 57.976047534)(53.097345132743364, 56.051000695)(53.173198482933, 56.331695507999996)(53.24905183312263, 57.639688088)(53.32490518331227, 50.567974109999994)(53.4007585335019, 48.364747571)(53.47661188369153, 60.83556798)(53.55246523388117, 60.198222462000004)(53.6283185840708, 60.816652074000004)(53.70417193426044, 53.941681208)(53.78002528445007, 49.132640081)(53.8558786346397, 54.001506749)(53.931731984829334, 53.62035854)(54.007585335018966, 55.641668894)(54.0834386852086, 62.206312610000005)(54.15929203539824, 63.325477135)(54.23514538558787, 36.54809556)(54.3109987357775, 54.316320152)(54.386852085967135, 61.661470704)(54.46270543615677, 59.578831127)(54.5385587863464, 56.949506544)(54.61441213653603, 55.489631113)(54.69026548672567, 57.420429475999995)(54.7661188369153, 57.18555144)(54.841972187104936, 57.035377405999995)(54.91782553729457, 56.685028861999996)(54.9936788874842, 60.656701674000004)(55.06953223767384, 46.333849461)(55.145385587863466, 48.265851591)(55.221238938053105, 55.727940264000004)(55.29709228824273, 59.613653971999994)(55.37294563843237, 56.657207511)(55.448798988622, 55.037852854)(55.524652338811634, 57.508890424)(55.600505689001274, 59.762057833)(55.6763590391909, 60.649796502)(55.75221238938054, 46.082852302)(55.828065739570164, 51.797457115)(55.9039190897598, 58.551343816)(55.979772439949436, 53.920243572000004)(56.05562579013907, 57.916775329)(56.1314791403287, 57.556940951)(56.20733249051833, 51.349796571)(56.28318584070797, 48.162284012)(56.359039190897604, 61.36949072)(56.43489254108724, 60.037157236)(56.51074589127687, 60.481663083)(56.5865992414665, 59.786735834)(56.662452591656134, 55.38663807)(56.73830594184577, 46.821031831999996)(56.8141592920354, 50.074133122)(56.89001264222504, 60.308360272)(56.96586599241467, 59.284612668)(57.0417193426043, 56.228817750999994)(57.11757269279394, 55.929678177)(57.19342604298357, 56.001832747)(57.26927939317321, 55.917592686)(57.34513274336283, 60.207046373)(57.42098609355247, 55.115267677000006)(57.496839443742104, 45.862569784)(57.572692793931736, 49.349789623999996)(57.648546144121376, 61.886972379999996)(57.724399494311, 62.979266667999994)(57.80025284450064, 61.621178012)(57.876106194690266, 42.883099757)(57.951959544879905, 51.306319384)(58.02781289506954, 59.040951645)(58.10366624525917, 58.687175902)(58.1795195954488, 60.227741783)(58.255372945638435, 61.511981992)(58.331226295828074, 45.253847641)(58.407079646017706, 51.968973872999996)(58.48293299620734, 57.938940093)(58.55878634639697, 54.621627129)(58.6346396965866, 58.437110992)(58.710493046776236, 49.732046413)(58.78634639696587, 55.647768883)(58.8621997471555, 61.700035696)(58.93805309734514, 59.017241547)(59.01390644753477, 58.152557126)(59.089759797724405, 60.134374011)(59.16561314791404, 46.892870374)(59.24146649810367, 47.705171667)(59.31731984829331, 52.797324032999995)(59.393173198482934, 57.802108386)(59.46902654867257, 60.271715699)(59.5448798988622, 58.903286078)(59.62073324905184, 54.142437201)(59.69658659924148, 51.33029341300001)(59.7724399494311, 43.95751060400001)(59.84829329962074, 61.167189992999994)(59.92414664981037, 60.329586746)(60.0, 60.734657711000004)
            };
            \addplot [pattern=north east lines,pattern color=red] 
            fill between [
                of=A and B,soft clip={domain=0:800},
            ];
            \end{axis}
\end{tikzpicture}
\caption{Measuring instrument: Clamp (Lin)}
\end{subfigure}
\caption{FannkuchRedux, on the workstation measured by the different measuring instrumentst, with the lines representing the minimum, maximum and average energy consumption}\label{fig:time_series_Fankuch_Workstation}
\end{figure}

To gain more insight into the differences between the different measuring instruments a look at a different configuration is useful. Therefore in \cref{fig:time_series_BinaryTrees_Workstation}. This figure shows the test case BinaryTrees on the workstation. Previously we saw that the measuring instruments Intel Power Gadget and LHM showed a consistent average, but with a slight upwards trend in energy consumption. This is not as obvious in \cref{fig:time_series_BinaryTrees_Workstation_IntelPowerGadget} and in \cref{fig:time_series_BinaryTrees_Workstation_LHM} there is a trend downwards. They both also have a peak at around the same time step as in \cref{fig:time_series_Fankuch_Workstation}
However, another thing which can be observed is that Intel Power Gadget is noisier in this configuration as well.  Regarding the clamp measurements in \cref{fig:time_series_BinaryTrees_Workstation_ClampW} there is also a drop off in energy consumption towards the end of the measurement. When looking at \cref{fig:time_series_BinaryTrees_Workstation_ClampL} a very similar pattern occurs although with a bit more variance. 



\begin{figure}[H]  
    \centering 
    \begin{subfigure}[b]{0.49\linewidth}
        \begin{tikzpicture}
            \pgfplotsset{%
            width=1\linewidth,
            % height=1\textheight
            }
            \begin{axis}[ymax=120,
            xlabel={Time (Seconds)},
            ylabel={Energy Consumption (Joules)},
            ]
            \addplot[color=blue, mark=none,] coordinates { %% AVG value
            (0.0, 48.83388333333332)(0.09966777408637874, 65.80850000000001)(0.19933554817275748, 69.20528333333336)(0.2990033222591362, 68.90605833333333)(0.39867109634551495, 69.27005000000001)(0.49833887043189373, 68.95458333333333)(0.5980066445182723, 69.14080833333335)(0.6976744186046512, 69.11959166666666)(0.7973421926910299, 69.3744)(0.8970099667774087, 69.46856666666667)(0.9966777408637875, 69.27143333333332)(1.0963455149501662, 69.138075)(1.1960132890365447, 69.19058333333332)(1.2956810631229236, 68.64076666666665)(1.3953488372093024, 69.59749999999998)(1.495016611295681, 69.32738333333336)(1.5946843853820598, 69.28523333333335)(1.6943521594684385, 69.20231666666665)(1.7940199335548175, 69.46559166666665)(1.893687707641196, 69.32213333333333)(1.993355481727575, 70.19979166666668)(2.0930232558139537, 69.28746666666666)(2.1926910299003324, 69.27536666666668)(2.292358803986711, 69.63652499999996)(2.3920265780730894, 70.5874166666667)(2.4916943521594686, 70.19893333333331)(2.5913621262458473, 70.18903333333333)(2.691029900332226, 70.33845833333332)(2.7906976744186047, 69.27014166666669)(2.8903654485049834, 69.3234)(2.990033222591362, 70.06786666666666)(3.089700996677741, 69.26222500000002)(3.1893687707641196, 69.32241666666663)(3.2890365448504983, 69.32637500000001)(3.388704318936877, 69.13367500000001)(3.488372093023256, 69.20514999999997)(3.588039867109635, 69.30604999999998)(3.6877076411960132, 69.34238333333334)(3.787375415282392, 69.38337500000002)(3.887043189368771, 69.35555833333336)(3.98671096345515, 70.22016666666666)(4.086378737541528, 69.30687499999996)(4.186046511627907, 69.43546666666666)(4.285714285714286, 69.42750000000001)(4.385382059800665, 69.4332916666667)(4.485049833887043, 69.22036666666664)(4.584717607973422, 69.30730833333334)(4.6843853820598005, 69.30380000000001)(4.784053156146179, 69.15316666666668)(4.883720930232559, 69.05794166666666)(4.983388704318937, 69.92380000000001)(5.083056478405315, 69.347875)(5.1827242524916945, 69.34518333333332)(5.282392026578074, 69.12541666666665)(5.382059800664452, 69.20972499999998)(5.48172757475083, 69.15454166666667)(5.5813953488372094, 69.24185000000001)(5.681063122923588, 69.2599416666667)(5.780730897009967, 69.24400833333333)(5.880398671096346, 69.26685833333335)(5.980066445182724, 69.97730000000001)(6.079734219269103, 69.49140000000001)(6.179401993355482, 69.39581666666668)(6.279069767441861, 69.20598333333336)(6.378737541528239, 69.40826666666666)(6.4784053156146175, 69.42967500000002)(6.578073089700997, 69.38406666666666)(6.677740863787376, 69.397025)(6.777408637873754, 69.14420833333331)(6.877076411960133, 69.36653333333334)(6.976744186046512, 69.78277499999996)(7.076411960132891, 69.73548333333333)(7.17607973421927, 69.25557499999998)(7.275747508305648, 69.16725000000004)(7.3754152823920265, 69.01752500000002)(7.475083056478405, 69.25360833333335)(7.574750830564784, 69.26427499999998)(7.674418604651163, 69.22497500000001)(7.774086378737542, 69.32081666666667)(7.8737541528239205, 69.23462500000001)(7.9734219269103, 69.8216333333333)(8.073089700996677, 69.65360000000001)(8.172757475083056, 69.45979166666669)(8.272425249169435, 69.32120833333335)(8.372093023255815, 69.52649166666664)(8.471760797342194, 69.37032500000001)(8.571428571428571, 69.35494999999999)(8.67109634551495, 69.2352)(8.77076411960133, 69.25220833333333)(8.870431893687707, 69.24862500000002)(8.970099667774086, 69.4762083333333)(9.069767441860465, 69.84425)(9.169435215946844, 69.33095833333333)(9.269102990033224, 69.53983333333333)(9.368770764119601, 70.60698333333335)(9.46843853820598, 69.883225)(9.568106312292358, 69.29425833333333)(9.667774086378738, 69.27144166666668)(9.767441860465118, 69.3319833333334)(9.867109634551495, 69.34056666666665)(9.966777408637874, 69.82952500000002)(10.066445182724253, 70.57627499999998)(10.16611295681063, 70.54690000000002)(10.26578073089701, 70.4874916666667)(10.365448504983389, 70.5045583333333)(10.465116279069768, 70.50228333333331)(10.564784053156147, 70.35136666666664)(10.664451827242525, 70.43234166666669)(10.764119601328904, 70.43755833333334)(10.863787375415281, 70.63862500000005)(10.96345514950166, 70.81185)(11.063122923588042, 71.10285833333333)(11.162790697674419, 70.719225)(11.262458471760798, 70.39659166666665)(11.362126245847175, 70.46896666666665)(11.461794019933555, 70.63575000000002)(11.561461794019934, 70.67024999999998)(11.661129568106311, 70.63463333333333)(11.760797342192692, 70.69406666666664)(11.860465116279071, 70.95349999999995)(11.960132890365449, 71.03982500000002)(12.059800664451828, 71.36084166666663)(12.159468438538205, 71.02864166666667)(12.259136212624584, 70.75105)(12.358803986710964, 70.88895833333335)(12.458471760797343, 70.97014999999995)(12.558139534883722, 70.82032499999997)(12.6578073089701, 70.73313333333333)(12.757475083056478, 70.67142500000001)(12.857142857142858, 70.62112500000002)(12.956810631229235, 70.55175833333334)(13.056478405315616, 71.0040583333333)(13.156146179401993, 70.92851666666667)(13.255813953488373, 70.50335833333335)(13.355481727574752, 70.51028333333336)(13.455149501661129, 70.45464166666667)(13.554817275747508, 70.56792500000002)(13.654485049833887, 70.67790000000001)(13.754152823920267, 70.5107166666667)(13.853820598006646, 70.61965833333332)(13.953488372093023, 70.6557666666667)(14.053156146179402, 70.96052499999999)(14.152823920265782, 70.98770833333333)(14.252491694352159, 70.69052500000002)(14.35215946843854, 70.75171666666665)(14.451827242524917, 70.71428333333331)(14.551495016611296, 70.59594999999999)(14.651162790697676, 70.57334166666666)(14.750830564784053, 70.59715833333333)(14.850498338870432, 70.48992499999999)(14.95016611295681, 70.15080833333332)(15.049833887043192, 70.779425)(15.149501661129568, 70.86590000000001)(15.249169435215947, 70.62035833333333)(15.348837209302326, 70.23741666666668)(15.448504983388705, 70.36190833333333)(15.548172757475085, 70.33667500000003)(15.64784053156146, 70.30043333333333)(15.747508305647841, 70.5352416666667)(15.84717607973422, 70.4749666666667)(15.9468438538206, 70.43465833333333)(16.04651162790698, 70.67705833333338)(16.146179401993354, 70.85561666666666)(16.245847176079735, 70.76939999999999)(16.345514950166113, 70.49980000000002)(16.445182724252494, 70.41853333333333)(16.54485049833887, 70.63872499999998)(16.64451827242525, 70.55135833333334)(16.74418604651163, 70.47779166666665)(16.843853820598007, 70.42297500000001)(16.943521594684388, 70.30345833333332)(17.043189368770765, 70.58747500000001)(17.142857142857142, 70.90114166666667)(17.24252491694352, 70.64210000000001)(17.3421926910299, 70.44163333333334)(17.44186046511628, 70.45752499999999)(17.54152823920266, 70.36937499999998)(17.641196013289036, 70.38170833333334)(17.740863787375414, 70.567575)(17.840531561461795, 70.46674166666666)(17.940199335548172, 70.60566666666665)(18.039867109634553, 70.46388333333336)(18.13953488372093, 71.02014166666667)(18.239202657807308, 70.77622500000003)(18.33887043189369, 70.51083333333335)(18.438538205980066, 70.63163333333334)(18.538205980066447, 70.53367500000004)(18.63787375415282, 70.44555)(18.737541528239202, 70.46138333333334)(18.837209302325583, 70.48278333333333)(18.93687707641196, 70.51235833333332)(19.03654485049834, 70.56896666666667)(19.136212624584715, 70.85009999999998)(19.235880398671096, 70.71487500000002)(19.335548172757477, 70.46382500000003)(19.435215946843854, 70.54043333333337)(19.534883720930235, 70.63079166666667)(19.634551495016613, 70.59141666666667)(19.73421926910299, 70.67454166666668)(19.833887043189367, 70.66057500000002)(19.93355481727575, 70.81849999999999)(20.033222591362126, 70.71059166666663)(20.132890365448507, 70.90677499999998)(20.232558139534884, 71.04249999999999)(20.33222591362126, 70.81358333333334)(20.431893687707642, 70.90433333333334)(20.53156146179402, 70.70747499999997)(20.6312292358804, 70.51858333333334)(20.730897009966778, 70.64016666666667)(20.830564784053156, 70.768025)(20.930232558139537, 70.79894166666669)(21.029900332225914, 70.79949166666665)(21.129568106312295, 71.04614999999998)(21.22923588039867, 70.95814166666665)(21.32890365448505, 70.89037499999998)(21.42857142857143, 70.64998333333332)(21.528239202657808, 70.76700833333334)(21.62790697674419, 70.71144999999997)(21.727574750830563, 70.72705833333333)(21.827242524916944, 70.82049999999998)(21.92691029900332, 70.61922499999999)(22.026578073089702, 70.8486583333333)(22.126245847176083, 71.01005000000002)(22.225913621262457, 71.32305000000001)(22.325581395348838, 71.04699999999998)(22.425249169435215, 70.955275)(22.524916943521596, 70.80180833333334)(22.624584717607974, 70.77920833333333)(22.72425249169435, 70.73123333333334)(22.823920265780732, 70.858675)(22.92358803986711, 70.68809166666665)(23.02325581395349, 70.74421666666666)(23.122923588039868, 70.87600833333327)(23.222591362126245, 71.13243333333335)(23.322259136212622, 70.94624166666662)(23.421926910299003, 70.71780833333335)(23.521594684385384, 70.66114999999996)(23.62126245847176, 70.64210833333335)(23.720930232558143, 70.75)(23.820598006644516, 70.65105833333332)(23.920265780730897, 70.78328333333336)(24.01993355481728, 70.73898333333332)(24.119601328903656, 70.83880833333336)(24.219269102990037, 71.21052499999999)(24.31893687707641, 71.18616666666667)(24.41860465116279, 70.7949333333333)(24.51827242524917, 70.971525)(24.61794019933555, 70.855225)(24.717607973421927, 70.9917)(24.817275747508305, 70.84128333333335)(24.916943521594686, 70.82468333333335)(25.016611295681063, 70.79790000000001)(25.116279069767444, 70.78645833333333)(25.21594684385382, 71.06553333333333)(25.3156146179402, 70.98138333333331)(25.41528239202658, 70.70095000000002)(25.514950166112957, 70.86224166666666)(25.614617940199338, 70.89908333333337)(25.714285714285715, 70.78825000000002)(25.813953488372093, 70.84391666666667)(25.91362126245847, 70.94628333333335)(26.01328903654485, 70.87230833333335)(26.112956810631232, 71.00798333333334)(26.21262458471761, 71.1969583333333)(26.312292358803987, 71.23578333333334)(26.411960132890364, 70.93889166666668)(26.511627906976745, 70.88541666666666)(26.611295681063122, 70.73380833333337)(26.710963455149503, 70.85813333333336)(26.81063122923588, 70.88796666666666)(26.910299003322258, 70.81993333333334)(27.00996677740864, 70.87041666666666)(27.109634551495017, 70.93899166666668)(27.209302325581397, 71.0865333333333)(27.308970099667775, 71.09875000000002)(27.408637873754152, 70.88334166666662)(27.508305647840533, 70.70740000000002)(27.60797342192691, 70.79884999999999)(27.70764119601329, 70.79835833333335)(27.80730897009967, 70.84606666666666)(27.906976744186046, 70.77954166666669)(28.006644518272424, 70.89212500000004)(28.106312292358805, 70.80165000000004)(28.205980066445186, 71.09770833333333)(28.305647840531563, 71.29045833333336)(28.40531561461794, 71.17264166666669)(28.504983388704318, 70.9521)(28.6046511627907, 70.94464999999998)(28.70431893687708, 70.84436666666664)(28.803986710963457, 70.7191416666667)(28.903654485049834, 70.8763583333333)(29.003322259136212, 70.81455833333335)(29.102990033222593, 70.95679999999997)(29.20265780730897, 70.8849166666667)(29.30232558139535, 71.1879416666667)(29.40199335548173, 70.88270833333333)(29.501661129568106, 70.83030000000002)(29.601328903654487, 70.79004166666668)(29.700996677740864, 70.74440833333337)(29.800664451827245, 70.89326666666668)(29.90033222591362, 70.90674166666666)(30.0, 70.77172500000002)(30.099667774086384, 71.08649166666669)(30.19933554817276, 71.18899166666668)(30.299003322259136, 71.48609166666668)(30.398671096345517, 71.45039166666668)(30.498338870431894, 71.01985)(30.598006644518275, 71.031475)(30.697674418604652, 70.9927416666667)(30.79734219269103, 71.05656666666664)(30.89700996677741, 70.89274166666665)(30.996677740863788, 71.03013333333335)(31.09634551495017, 71.05097499999998)(31.196013289036546, 70.92367499999999)(31.29568106312292, 71.26499166666666)(31.395348837209305, 71.13604166666666)(31.495016611295682, 70.99915833333333)(31.594684385382063, 70.98874166666671)(31.69435215946844, 71.00941666666667)(31.794019933554814, 71.04865000000001)(31.8936877076412, 71.07087499999999)(31.993355481727576, 71.21381666666669)(32.09302325581396, 71.18503333333334)(32.19269102990033, 71.3258)(32.29235880398671, 71.61909166666665)(32.39202657807309, 71.35365833333331)(32.49169435215947, 71.29865833333335)(32.59136212624585, 71.08742500000002)(32.691029900332225, 71.1585333333333)(32.7906976744186, 71.11009166666668)(32.89036544850499, 71.01214999999999)(32.990033222591364, 71.25681666666667)(33.08970099667774, 71.11715000000002)(33.18936877076412, 71.006275)(33.2890365448505, 71.25249999999997)(33.38870431893688, 71.30369999999996)(33.48837209302326, 71.22094999999999)(33.588039867109636, 71.12214999999999)(33.68770764119601, 71.11868333333335)(33.78737541528239, 71.26234999999997)(33.887043189368775, 71.18199166666668)(33.98671096345515, 71.04325)(34.08637873754153, 71.08444166666663)(34.18604651162791, 71.18739166666667)(34.285714285714285, 71.2854666666667)(34.38538205980067, 71.4834166666667)(34.48504983388704, 71.44269166666665)(34.584717607973424, 71.20037500000002)(34.6843853820598, 71.16696666666667)(34.78405315614618, 71.06864166666666)(34.88372093023256, 71.10324999999997)(34.98338870431893, 71.09559166666668)(35.08305647840532, 71.2213166666667)(35.182724252491695, 71.14665833333332)(35.28239202657807, 71.22984166666664)(35.38205980066445, 71.32149166666667)(35.48172757475083, 71.4391083333333)(35.58139534883721, 71.242025)(35.68106312292359, 71.23162500000002)(35.78073089700997, 71.17466666666664)(35.880398671096344, 71.14929166666666)(35.98006644518272, 71.14752499999999)(36.079734219269106, 71.19005833333334)(36.179401993355484, 71.272125)(36.27906976744186, 71.31125833333333)(36.37873754152824, 71.59743333333336)(36.478405315614616, 71.49257499999996)(36.578073089701, 71.33309166666668)(36.67774086378738, 71.24505833333335)(36.777408637873755, 71.17130000000002)(36.87707641196013, 71.22769166666669)(36.97674418604651, 71.21719166666666)(37.076411960132894, 71.16871666666667)(37.17607973421927, 71.28714166666667)(37.27574750830564, 71.16195000000002)(37.37541528239203, 70.86832499999997)(37.475083056478404, 71.43161666666668)(37.57475083056479, 71.31443333333331)(37.674418604651166, 71.16990833333331)(37.774086378737536, 71.26311666666668)(37.87375415282392, 71.20087500000001)(37.9734219269103, 71.17936666666667)(38.07308970099668, 71.15170833333332)(38.17275747508306, 71.23608333333331)(38.27242524916943, 71.32864166666668)(38.372093023255815, 71.5307916666667)(38.47176079734219, 71.48827499999997)(38.57142857142858, 71.31973333333332)(38.671096345514954, 71.241875)(38.77076411960133, 71.07524166666663)(38.87043189368771, 71.01781666666663)(38.970099667774086, 71.13766666666669)(39.06976744186047, 71.15893333333334)(39.16943521594684, 70.88333333333337)(39.269102990033225, 71.08762499999999)(39.3687707641196, 71.09421666666665)(39.46843853820598, 71.41224999999999)(39.568106312292365, 71.2839)(39.667774086378735, 71.02209166666668)(39.76744186046512, 71.05943333333336)(39.8671096345515, 70.97373333333333)(39.966777408637874, 70.94522500000001)(40.06644518272425, 70.77079166666664)(40.16611295681063, 70.92673333333333)(40.26578073089701, 71.000775)(40.36544850498339, 71.16580833333332)(40.46511627906977, 71.22786666666669)(40.564784053156146, 71.06093333333328)(40.66445182724252, 71.00526666666664)(40.76411960132891, 70.94761666666665)(40.863787375415285, 70.92225833333333)(40.96345514950166, 70.91480000000003)(41.06312292358804, 70.982625)(41.16279069767442, 70.86500000000001)(41.2624584717608, 70.76123333333334)(41.36212624584718, 70.89189166666665)(41.461794019933556, 71.04253333333335)(41.561461794019934, 71.11808333333336)(41.66112956810631, 70.99819166666666)(41.760797342192696, 70.8400583333333)(41.86046511627907, 70.86815833333334)(41.96013289036544, 71.03775000000002)(42.05980066445183, 70.81539999999998)(42.159468438538205, 70.98122500000001)(42.25913621262459, 70.90680833333336)(42.35880398671097, 71.11687500000002)(42.45847176079734, 71.17973333333335)(42.55813953488372, 71.277625)(42.6578073089701, 70.67765)(42.757475083056484, 70.76487500000002)(42.85714285714286, 70.88594166666667)(42.95681063122923, 70.83828333333331)(43.056478405315616, 70.9008333333333)(43.15614617940199, 70.87683333333334)(43.25581395348838, 70.89392500000001)(43.355481727574755, 70.79918333333339)(43.455149501661126, 71.0653333333333)(43.55481727574751, 71.16170000000002)(43.65448504983389, 70.976925)(43.75415282392027, 70.85347500000002)(43.85382059800664, 71.06841666666665)(43.95348837209302, 70.93554166666668)(44.053156146179404, 70.86184999999999)(44.15282392026578, 70.91823333333329)(44.252491694352166, 70.95309166666667)(44.352159468438536, 71.15540833333334)(44.451827242524914, 71.32034166666669)(44.5514950166113, 71.36898333333333)(44.651162790697676, 71.22739166666669)(44.75083056478405, 71.08661666666671)(44.85049833887043, 71.0328833333333)(44.95016611295681, 71.05579166666669)(45.04983388704319, 71.12759999999999)(45.14950166112957, 70.90343333333334)(45.24916943521595, 70.98896666666668)(45.348837209302324, 70.91628333333334)(45.4485049833887, 71.05771666666665)(45.548172757475086, 71.2131083333333)(45.647840531561464, 70.54957499999996)(45.74750830564784, 71.06444166666667)(45.84717607973422, 70.99648333333334)(45.946843853820596, 71.0780583333333)(46.04651162790698, 71.04568333333332)(46.14617940199336, 71.11800833333332)(46.245847176079735, 71.00869166666665)(46.34551495016611, 71.148525)(46.44518272425249, 71.15069999999997)(46.544850498338874, 71.293725)(46.644518272425245, 71.34434999999999)(46.74418604651163, 71.13157499999998)(46.84385382059801, 70.9183)(46.94352159468439, 71.03152500000002)(47.04318936877077, 71.06973333333336)(47.14285714285714, 71.19111666666669)(47.24252491694352, 70.88370833333332)(47.3421926910299, 70.92853333333332)(47.441860465116285, 70.9478)(47.54152823920266, 71.191)(47.64119601328903, 71.2054)(47.74086378737542, 71.15059166666666)(47.840531561461795, 70.98060000000001)(47.94019933554818, 71.02599999999997)(48.03986710963456, 70.96541666666668)(48.13953488372093, 71.06929999999998)(48.23920265780731, 71.05311666666665)(48.33887043189369, 71.21565000000002)(48.43853820598007, 71.20955833333332)(48.538205980066444, 71.17254166666666)(48.63787375415282, 71.29924999999999)(48.737541528239205, 71.08194166666665)(48.83720930232558, 70.95488333333334)(48.93687707641197, 70.92626666666666)(49.03654485049834, 70.96435833333332)(49.136212624584715, 70.93294166666668)(49.2358803986711, 70.92504166666669)(49.33554817275748, 70.94111666666666)(49.435215946843854, 70.93284999999997)(49.53488372093023, 71.14863333333332)(49.63455149501661, 71.24700833333331)(49.734219269102994, 71.1834916666667)(49.83388704318937, 71.05464166666664)(49.93355481727575, 71.03635000000001)(50.033222591362126, 71.02356666666664)(50.1328903654485, 71.23813333333332)(50.23255813953489, 71.33784166666666)(50.332225913621265, 71.38722500000002)(50.43189368770764, 71.41787500000005)(50.53156146179402, 71.44509166666663)(50.6312292358804, 71.56021666666666)(50.73089700996678, 71.50881666666669)(50.83056478405316, 71.30311666666668)(50.93023255813954, 71.09853333333332)(51.029900332225914, 71.29528333333334)(51.12956810631229, 71.27075)(51.229235880398676, 71.19586666666665)(51.328903654485046, 71.13615833333327)(51.42857142857143, 71.38960833333334)(51.52823920265781, 71.46583333333332)(51.627906976744185, 71.538325)(51.72757475083057, 71.44303333333332)(51.82724252491694, 71.4379)(51.926910299003325, 71.36171666666668)(52.0265780730897, 71.33866666666667)(52.12624584717608, 71.41488333333332)(52.225913621262464, 71.41599166666666)(52.325581395348834, 71.51753333333336)(52.42524916943522, 71.47153333333333)(52.524916943521596, 71.48098333333333)(52.62458471760797, 71.52136666666664)(52.72425249169436, 71.56274999999998)(52.82392026578073, 71.45914166666667)(52.92358803986711, 71.37365000000001)(53.02325581395349, 71.46664166666669)(53.12292358803987, 71.34154166666666)(53.222591362126245, 71.48154166666667)(53.32225913621262, 71.41921666666666)(53.42192691029901, 71.56521666666666)(53.521594684385384, 71.64200833333332)(53.62126245847176, 71.72903333333333)(53.72093023255814, 71.76034166666668)(53.820598006644516, 71.37498333333335)(53.9202657807309, 71.49679166666668)(54.01993355481728, 71.24248333333334)(54.119601328903656, 71.34075000000001)(54.21926910299003, 71.3455583333333)(54.31893687707641, 71.48855)(54.418604651162795, 71.36815000000004)(54.51827242524917, 71.45264166666668)(54.61794019933555, 71.57208333333334)(54.71760797342193, 71.60999999999999)(54.817275747508305, 71.38908333333333)(54.91694352159469, 71.3398)(55.01661129568107, 71.34966666666666)(55.11627906976744, 71.34387500000003)(55.21594684385382, 71.30996666666668)(55.3156146179402, 71.22099166666665)(55.41528239202658, 71.33980833333331)(55.51495016611296, 71.33603333333333)(55.61461794019934, 71.45523333333337)(55.714285714285715, 71.54692500000002)(55.81395348837209, 71.51916666666668)(55.91362126245848, 71.38687499999993)(56.01328903654485, 71.28586666666664)(56.11295681063123, 71.26358333333334)(56.21262458471761, 71.36607500000001)(56.31229235880399, 71.481175)(56.41196013289037, 71.55303333333333)(56.51162790697674, 71.62246666666667)(56.611295681063126, 71.65964166666666)(56.7109634551495, 71.5353)(56.81063122923588, 71.63225000000001)(56.910299003322265, 71.43269166666666)(57.009966777408636, 71.36516666666665)(57.10963455149502, 71.34739166666665)(57.2093023255814, 71.26209166666669)(57.308970099667775, 71.17487499999997)(57.40863787375416, 71.13903333333334)(57.50830564784053, 71.20654166666665)(57.607973421926914, 71.30260833333335)(57.70764119601329, 71.44765833333334)(57.80730897009967, 71.45332499999999)(57.906976744186046, 71.48926666666668)(58.006644518272424, 71.2024083333333)(58.10631229235881, 71.27718333333334)(58.205980066445186, 71.54981666666667)(58.30564784053156, 71.32219166666665)(58.40531561461794, 71.4438916666667)(58.50498338870432, 71.19281666666667)(58.6046511627907, 71.54586666666665)(58.70431893687708, 71.49299166666667)(58.80398671096346, 71.4293166666667)(58.903654485049834, 71.32321666666665)(59.00332225913621, 71.38985833333336)(59.102990033222596, 71.41488333333334)(59.202657807308974, 71.3358416666667)(59.30232558139535, 71.26215000000002)(59.40199335548173, 70.93869166666666)(59.501661129568106, 71.23543333333335)(59.60132890365449, 71.31389999999996)(59.70099667774087, 70.73786666666668)(59.80066445182724, 42.140599999999985)(59.90033222591362, 19.990982456140358)(60.0, 6.631)
            };
            \addplot[color=blue, mark=none,name path=A] coordinates { %% MAX value
            (0.0, 60.848)(0.09966777408637874, 77.368)(0.19933554817275748, 77.785)(0.2990033222591362, 78.477)(0.39867109634551495, 78.454)(0.49833887043189373, 72.107)(0.5980066445182723, 71.736)(0.6976744186046512, 72.235)(0.7973421926910299, 72.471)(0.8970099667774087, 72.51)(0.9966777408637875, 72.667)(1.0963455149501662, 80.417)(1.1960132890365447, 78.801)(1.2956810631229236, 72.295)(1.3953488372093024, 80.455)(1.495016611295681, 80.346)(1.5946843853820598, 83.929)(1.6943521594684385, 80.218)(1.7940199335548175, 83.757)(1.893687707641196, 76.045)(1.993355481727575, 72.983)(2.0930232558139537, 78.878)(2.1926910299003324, 77.419)(2.292358803986711, 82.83)(2.3920265780730894, 77.179)(2.4916943521594686, 75.88)(2.5913621262458473, 83.049)(2.691029900332226, 79.011)(2.7906976744186047, 78.336)(2.8903654485049834, 72.203)(2.990033222591362, 74.103)(3.089700996677741, 72.058)(3.1893687707641196, 80.165)(3.2890365448504983, 72.399)(3.388704318936877, 72.253)(3.488372093023256, 72.428)(3.588039867109635, 71.675)(3.6877076411960132, 72.636)(3.787375415282392, 72.118)(3.887043189368771, 72.073)(3.98671096345515, 72.578)(4.086378737541528, 72.194)(4.186046511627907, 71.954)(4.285714285714286, 71.748)(4.385382059800665, 71.675)(4.485049833887043, 72.232)(4.584717607973422, 72.175)(4.6843853820598005, 71.497)(4.784053156146179, 71.681)(4.883720930232559, 71.607)(4.983388704318937, 73.118)(5.083056478405315, 72.437)(5.1827242524916945, 72.805)(5.282392026578074, 72.799)(5.382059800664452, 72.436)(5.48172757475083, 72.912)(5.5813953488372094, 72.744)(5.681063122923588, 72.729)(5.780730897009967, 72.072)(5.880398671096346, 72.301)(5.980066445182724, 74.386)(6.079734219269103, 72.107)(6.179401993355482, 71.888)(6.279069767441861, 71.75)(6.378737541528239, 71.889)(6.4784053156146175, 71.833)(6.578073089700997, 72.431)(6.677740863787376, 72.283)(6.777408637873754, 71.483)(6.877076411960133, 71.945)(6.976744186046512, 72.869)(7.076411960132891, 72.753)(7.17607973421927, 72.154)(7.275747508305648, 72.126)(7.3754152823920265, 71.879)(7.475083056478405, 72.437)(7.574750830564784, 72.068)(7.674418604651163, 72.22)(7.774086378737542, 71.798)(7.8737541528239205, 71.991)(7.9734219269103, 72.754)(8.073089700996677, 72.516)(8.172757475083056, 71.747)(8.272425249169435, 71.863)(8.372093023255815, 72.063)(8.471760797342194, 72.049)(8.571428571428571, 72.253)(8.67109634551495, 71.726)(8.77076411960133, 72.114)(8.870431893687707, 71.718)(8.970099667774086, 73.594)(9.069767441860465, 72.845)(9.169435215946844, 72.759)(9.269102990033224, 73.779)(9.368770764119601, 73.662)(9.46843853820598, 80.68)(9.568106312292358, 81.998)(9.667774086378738, 73.767)(9.767441860465118, 71.982)(9.867109634551495, 71.953)(9.966777408637874, 72.923)(10.066445182724253, 73.172)(10.16611295681063, 73.198)(10.26578073089701, 73.236)(10.365448504983389, 72.541)(10.465116279069768, 72.912)(10.564784053156147, 73.146)(10.664451827242525, 72.962)(10.764119601328904, 77.65)(10.863787375415281, 103.807)(10.96345514950166, 110.488)(11.063122923588042, 104.018)(11.162790697674419, 85.583)(11.262458471760798, 85.822)(11.362126245847175, 86.326)(11.461794019933555, 104.724)(11.561461794019934, 101.791)(11.661129568106311, 102.718)(11.760797342192692, 97.988)(11.860465116279071, 109.774)(11.960132890365449, 109.51)(12.059800664451828, 105.139)(12.159468438538205, 89.675)(12.259136212624584, 102.995)(12.358803986710964, 104.71)(12.458471760797343, 102.652)(12.558139534883722, 99.946)(12.6578073089701, 93.251)(12.757475083056478, 89.007)(12.857142857142858, 89.543)(12.956810631229235, 86.449)(13.056478405315616, 86.215)(13.156146179401993, 87.165)(13.255813953488373, 86.882)(13.355481727574752, 83.944)(13.455149501661129, 81.242)(13.554817275747508, 83.677)(13.654485049833887, 86.248)(13.754152823920267, 85.799)(13.853820598006646, 85.936)(13.953488372093023, 86.166)(14.053156146179402, 86.392)(14.152823920265782, 86.166)(14.252491694352159, 82.42)(14.35215946843854, 81.055)(14.451827242524917, 77.518)(14.551495016611296, 78.323)(14.651162790697676, 77.811)(14.750830564784053, 80.678)(14.850498338870432, 78.555)(14.95016611295681, 80.38)(15.049833887043192, 78.664)(15.149501661129568, 76.611)(15.249169435215947, 80.918)(15.348837209302326, 73.513)(15.448504983388705, 74.671)(15.548172757475085, 73.291)(15.64784053156146, 73.649)(15.747508305647841, 73.088)(15.84717607973422, 73.007)(15.9468438538206, 73.502)(16.04651162790698, 73.066)(16.146179401993354, 73.156)(16.245847176079735, 74.926)(16.345514950166113, 73.299)(16.445182724252494, 79.439)(16.54485049833887, 79.317)(16.64451827242525, 77.442)(16.74418604651163, 72.908)(16.843853820598007, 73.663)(16.943521594684388, 72.624)(17.043189368770765, 73.384)(17.142857142857142, 74.39)(17.24252491694352, 73.688)(17.3421926910299, 73.92)(17.44186046511628, 76.677)(17.54152823920266, 73.408)(17.641196013289036, 72.917)(17.740863787375414, 73.42)(17.840531561461795, 73.698)(17.940199335548172, 76.261)(18.039867109634553, 73.52)(18.13953488372093, 73.975)(18.239202657807308, 73.516)(18.33887043189369, 73.97)(18.438538205980066, 73.255)(18.538205980066447, 73.164)(18.63787375415282, 73.042)(18.737541528239202, 73.586)(18.837209302325583, 74.627)(18.93687707641196, 73.123)(19.03654485049834, 73.387)(19.136212624584715, 74.038)(19.235880398671096, 74.135)(19.335548172757477, 74.442)(19.435215946843854, 81.408)(19.534883720930235, 89.561)(19.634551495016613, 94.124)(19.73421926910299, 93.458)(19.833887043189367, 95.435)(19.93355481727575, 93.423)(20.033222591362126, 90.289)(20.132890365448507, 88.248)(20.232558139534884, 84.206)(20.33222591362126, 89.917)(20.431893687707642, 108.384)(20.53156146179402, 98.81)(20.6312292358804, 99.217)(20.730897009966778, 100.272)(20.830564784053156, 97.611)(20.930232558139537, 95.822)(21.029900332225914, 98.626)(21.129568106312295, 95.934)(21.22923588039867, 95.806)(21.32890365448505, 93.151)(21.42857142857143, 96.166)(21.528239202657808, 95.318)(21.62790697674419, 94.855)(21.727574750830563, 96.796)(21.827242524916944, 98.118)(21.92691029900332, 100.157)(22.026578073089702, 97.458)(22.126245847176083, 98.934)(22.225913621262457, 98.999)(22.325581395348838, 91.944)(22.425249169435215, 96.823)(22.524916943521596, 91.671)(22.624584717607974, 92.457)(22.72425249169435, 89.897)(22.823920265780732, 93.428)(22.92358803986711, 91.482)(23.02325581395349, 89.428)(23.122923588039868, 92.391)(23.222591362126245, 89.344)(23.322259136212622, 90.615)(23.421926910299003, 93.999)(23.521594684385384, 91.013)(23.62126245847176, 91.708)(23.720930232558143, 91.895)(23.820598006644516, 90.004)(23.920265780730897, 91.85)(24.01993355481728, 93.933)(24.119601328903656, 91.805)(24.219269102990037, 93.361)(24.31893687707641, 90.641)(24.41860465116279, 92.239)(24.51827242524917, 93.292)(24.61794019933555, 93.191)(24.717607973421927, 92.489)(24.817275747508305, 91.892)(24.916943521594686, 92.018)(25.016611295681063, 90.056)(25.116279069767444, 90.685)(25.21594684385382, 89.123)(25.3156146179402, 91.308)(25.41528239202658, 89.491)(25.514950166112957, 89.999)(25.614617940199338, 90.053)(25.714285714285715, 89.982)(25.813953488372093, 88.685)(25.91362126245847, 90.761)(26.01328903654485, 90.72)(26.112956810631232, 91.522)(26.21262458471761, 90.809)(26.312292358803987, 92.8)(26.411960132890364, 92.823)(26.511627906976745, 91.878)(26.611295681063122, 92.122)(26.710963455149503, 89.022)(26.81063122923588, 91.642)(26.910299003322258, 89.301)(27.00996677740864, 91.24)(27.109634551495017, 90.068)(27.209302325581397, 91.641)(27.308970099667775, 91.113)(27.408637873754152, 91.519)(27.508305647840533, 92.841)(27.60797342192691, 91.175)(27.70764119601329, 92.932)(27.80730897009967, 92.094)(27.906976744186046, 91.207)(28.006644518272424, 92.024)(28.106312292358805, 90.522)(28.205980066445186, 88.65)(28.305647840531563, 90.649)(28.40531561461794, 90.808)(28.504983388704318, 92.2)(28.6046511627907, 92.106)(28.70431893687708, 91.931)(28.803986710963457, 91.452)(28.903654485049834, 90.005)(29.003322259136212, 90.031)(29.102990033222593, 90.197)(29.20265780730897, 91.167)(29.30232558139535, 90.604)(29.40199335548173, 88.311)(29.501661129568106, 93.519)(29.601328903654487, 90.122)(29.700996677740864, 90.131)(29.800664451827245, 90.159)(29.90033222591362, 91.857)(30.0, 91.268)(30.099667774086384, 91.327)(30.19933554817276, 91.954)(30.299003322259136, 91.04)(30.398671096345517, 92.684)(30.498338870431894, 90.232)(30.598006644518275, 92.637)(30.697674418604652, 91.276)(30.79734219269103, 92.111)(30.89700996677741, 90.588)(30.996677740863788, 91.163)(31.09634551495017, 90.847)(31.196013289036546, 90.575)(31.29568106312292, 92.696)(31.395348837209305, 89.299)(31.495016611295682, 89.408)(31.594684385382063, 90.368)(31.69435215946844, 91.716)(31.794019933554814, 91.172)(31.8936877076412, 90.24)(31.993355481727576, 90.978)(32.09302325581396, 90.783)(32.19269102990033, 91.594)(32.29235880398671, 92.459)(32.39202657807309, 91.148)(32.49169435215947, 90.301)(32.59136212624585, 91.312)(32.691029900332225, 91.514)(32.7906976744186, 90.471)(32.89036544850499, 90.699)(32.990033222591364, 92.078)(33.08970099667774, 92.466)(33.18936877076412, 81.508)(33.2890365448505, 90.718)(33.38870431893688, 92.133)(33.48837209302326, 92.419)(33.588039867109636, 91.263)(33.68770764119601, 91.939)(33.78737541528239, 91.182)(33.887043189368775, 91.627)(33.98671096345515, 92.049)(34.08637873754153, 89.517)(34.18604651162791, 92.154)(34.285714285714285, 90.452)(34.38538205980067, 92.349)(34.48504983388704, 92.129)(34.584717607973424, 92.127)(34.6843853820598, 91.347)(34.78405315614618, 91.341)(34.88372093023256, 92.419)(34.98338870431893, 90.289)(35.08305647840532, 92.687)(35.182724252491695, 90.403)(35.28239202657807, 91.196)(35.38205980066445, 91.006)(35.48172757475083, 92.629)(35.58139534883721, 91.862)(35.68106312292359, 90.252)(35.78073089700997, 91.394)(35.880398671096344, 91.662)(35.98006644518272, 93.432)(36.079734219269106, 91.411)(36.179401993355484, 93.627)(36.27906976744186, 91.834)(36.37873754152824, 93.085)(36.478405315614616, 94.114)(36.578073089701, 94.24)(36.67774086378738, 95.078)(36.777408637873755, 94.551)(36.87707641196013, 92.774)(36.97674418604651, 93.193)(37.076411960132894, 90.912)(37.17607973421927, 92.967)(37.27574750830564, 91.969)(37.37541528239203, 93.025)(37.475083056478404, 91.518)(37.57475083056479, 93.394)(37.674418604651166, 93.121)(37.774086378737536, 93.505)(37.87375415282392, 90.9)(37.9734219269103, 82.049)(38.07308970099668, 83.313)(38.17275747508306, 83.628)(38.27242524916943, 81.895)(38.372093023255815, 82.878)(38.47176079734219, 82.544)(38.57142857142858, 82.063)(38.671096345514954, 88.165)(38.77076411960133, 81.061)(38.87043189368771, 80.862)(38.970099667774086, 80.836)(39.06976744186047, 82.598)(39.16943521594684, 79.801)(39.269102990033225, 81.755)(39.3687707641196, 82.377)(39.46843853820598, 82.062)(39.568106312292365, 76.963)(39.667774086378735, 73.806)(39.76744186046512, 73.685)(39.8671096345515, 73.655)(39.966777408637874, 73.457)(40.06644518272425, 73.941)(40.16611295681063, 73.332)(40.26578073089701, 73.962)(40.36544850498339, 74.029)(40.46511627906977, 74.936)(40.564784053156146, 74.241)(40.66445182724252, 73.71)(40.76411960132891, 73.612)(40.863787375415285, 73.722)(40.96345514950166, 73.617)(41.06312292358804, 73.71)(41.16279069767442, 73.458)(41.2624584717608, 73.807)(41.36212624584718, 73.802)(41.461794019933556, 73.948)(41.561461794019934, 74.529)(41.66112956810631, 73.691)(41.760797342192696, 74.947)(41.86046511627907, 73.598)(41.96013289036544, 73.737)(42.05980066445183, 73.561)(42.159468438538205, 73.03)(42.25913621262459, 73.409)(42.35880398671097, 74.614)(42.45847176079734, 74.164)(42.55813953488372, 74.125)(42.6578073089701, 74.12)(42.757475083056484, 73.344)(42.85714285714286, 73.967)(42.95681063122923, 73.367)(43.056478405315616, 74.122)(43.15614617940199, 73.568)(43.25581395348838, 74.063)(43.355481727574755, 74.226)(43.455149501661126, 74.177)(43.55481727574751, 74.766)(43.65448504983389, 73.973)(43.75415282392027, 73.356)(43.85382059800664, 73.655)(43.95348837209302, 74.92)(44.053156146179404, 73.857)(44.15282392026578, 73.278)(44.252491694352166, 76.08)(44.352159468438536, 83.306)(44.451827242524914, 93.347)(44.5514950166113, 94.436)(44.651162790697676, 95.841)(44.75083056478405, 94.472)(44.85049833887043, 91.865)(44.95016611295681, 90.837)(45.04983388704319, 86.318)(45.14950166112957, 79.88)(45.24916943521595, 81.122)(45.348837209302324, 80.895)(45.4485049833887, 80.505)(45.548172757475086, 81.451)(45.647840531561464, 74.04)(45.74750830564784, 81.718)(45.84717607973422, 78.058)(45.946843853820596, 83.903)(46.04651162790698, 82.23)(46.14617940199336, 82.846)(46.245847176079735, 83.314)(46.34551495016611, 81.067)(46.44518272425249, 81.689)(46.544850498338874, 81.185)(46.644518272425245, 83.819)(46.74418604651163, 82.264)(46.84385382059801, 81.875)(46.94352159468439, 81.983)(47.04318936877077, 80.22)(47.14285714285714, 82.846)(47.24252491694352, 74.323)(47.3421926910299, 74.012)(47.441860465116285, 74.084)(47.54152823920266, 74.24)(47.64119601328903, 74.583)(47.74086378737542, 74.013)(47.840531561461795, 74.664)(47.94019933554818, 73.975)(48.03986710963456, 73.83)(48.13953488372093, 76.736)(48.23920265780731, 79.582)(48.33887043189369, 79.806)(48.43853820598007, 74.21)(48.538205980066444, 74.263)(48.63787375415282, 74.413)(48.737541528239205, 74.246)(48.83720930232558, 73.856)(48.93687707641197, 73.936)(49.03654485049834, 74.462)(49.136212624584715, 73.785)(49.2358803986711, 73.371)(49.33554817275748, 73.974)(49.435215946843854, 73.973)(49.53488372093023, 74.383)(49.63455149501661, 74.062)(49.734219269102994, 74.003)(49.83388704318937, 73.954)(49.93355481727575, 73.933)(50.033222591362126, 73.811)(50.1328903654485, 73.523)(50.23255813953489, 73.697)(50.332225913621265, 73.594)(50.43189368770764, 74.228)(50.53156146179402, 74.354)(50.6312292358804, 74.2)(50.73089700996678, 74.47)(50.83056478405316, 73.641)(50.93023255813954, 73.734)(51.029900332225914, 74.005)(51.12956810631229, 73.802)(51.229235880398676, 74.227)(51.328903654485046, 73.9)(51.42857142857143, 75.349)(51.52823920265781, 74.714)(51.627906976744185, 74.462)(51.72757475083057, 73.763)(51.82724252491694, 74.981)(51.926910299003325, 73.767)(52.0265780730897, 74.201)(52.12624584717608, 73.879)(52.225913621262464, 73.887)(52.325581395348834, 74.015)(52.42524916943522, 73.661)(52.524916943521596, 74.491)(52.62458471760797, 74.371)(52.72425249169436, 75.02)(52.82392026578073, 74.749)(52.92358803986711, 73.555)(53.02325581395349, 77.762)(53.12292358803987, 79.041)(53.222591362126245, 84.37)(53.32225913621262, 84.069)(53.42192691029901, 88.655)(53.521594684385384, 87.129)(53.62126245847176, 92.306)(53.72093023255814, 85.901)(53.820598006644516, 82.684)(53.9202657807309, 78.926)(54.01993355481728, 73.979)(54.119601328903656, 74.194)(54.21926910299003, 73.397)(54.31893687707641, 73.966)(54.418604651162795, 74.023)(54.51827242524917, 74.607)(54.61794019933555, 79.763)(54.71760797342193, 74.332)(54.817275747508305, 74.203)(54.91694352159469, 74.292)(55.01661129568107, 73.985)(55.11627906976744, 74.07)(55.21594684385382, 74.504)(55.3156146179402, 75.381)(55.41528239202658, 73.845)(55.51495016611296, 74.011)(55.61461794019934, 74.659)(55.714285714285715, 74.709)(55.81395348837209, 74.808)(55.91362126245848, 73.938)(56.01328903654485, 74.283)(56.11295681063123, 74.32)(56.21262458471761, 73.812)(56.31229235880399, 74.108)(56.41196013289037, 89.076)(56.51162790697674, 81.907)(56.611295681063126, 81.558)(56.7109634551495, 82.595)(56.81063122923588, 81.241)(56.910299003322265, 80.895)(57.009966777408636, 74.089)(57.10963455149502, 74.181)(57.2093023255814, 74.125)(57.308970099667775, 74.054)(57.40863787375416, 74.301)(57.50830564784053, 74.119)(57.607973421926914, 74.318)(57.70764119601329, 75.038)(57.80730897009967, 74.767)(57.906976744186046, 74.799)(58.006644518272424, 74.256)(58.10631229235881, 74.657)(58.205980066445186, 74.53)(58.30564784053156, 74.161)(58.40531561461794, 74.08)(58.50498338870432, 73.793)(58.6046511627907, 74.373)(58.70431893687708, 74.343)(58.80398671096346, 74.258)(58.903654485049834, 73.837)(59.00332225913621, 74.47)(59.102990033222596, 74.277)(59.202657807308974, 74.703)(59.30232558139535, 74.098)(59.40199335548173, 73.932)(59.501661129568106, 74.428)(59.60132890365449, 74.573)(59.70099667774087, 74.989)(59.80066445182724, 66.382)(59.90033222591362, 52.692)(60.0, 6.631)
            };
            \addplot[color=blue, mark=none,name path=B] coordinates { %% MIN value
            (0.0, 26.363)(0.09966777408637874, 61.173)(0.19933554817275748, 65.024)(0.2990033222591362, 65.707)(0.39867109634551495, 65.218)(0.49833887043189373, 65.392)(0.5980066445182723, 64.763)(0.6976744186046512, 66.173)(0.7973421926910299, 65.958)(0.8970099667774087, 64.154)(0.9966777408637875, 65.938)(1.0963455149501662, 65.339)(1.1960132890365447, 66.15)(1.2956810631229236, 0.0)(1.3953488372093024, 66.178)(1.495016611295681, 66.216)(1.5946843853820598, 65.754)(1.6943521594684385, 66.436)(1.7940199335548175, 65.863)(1.893687707641196, 65.83)(1.993355481727575, 65.331)(2.0930232558139537, 65.974)(2.1926910299003324, 65.454)(2.292358803986711, 65.737)(2.3920265780730894, 66.631)(2.4916943521594686, 66.066)(2.5913621262458473, 66.631)(2.691029900332226, 66.583)(2.7906976744186047, 56.959)(2.8903654485049834, 66.359)(2.990033222591362, 66.513)(3.089700996677741, 66.069)(3.1893687707641196, 66.166)(3.2890365448504983, 66.272)(3.388704318936877, 65.569)(3.488372093023256, 66.203)(3.588039867109635, 65.788)(3.6877076411960132, 65.589)(3.787375415282392, 66.282)(3.887043189368771, 65.729)(3.98671096345515, 67.497)(4.086378737541528, 65.8)(4.186046511627907, 66.294)(4.285714285714286, 66.21)(4.385382059800665, 66.111)(4.485049833887043, 48.494)(4.584717607973422, 66.472)(4.6843853820598005, 66.044)(4.784053156146179, 66.184)(4.883720930232559, 42.209)(4.983388704318937, 62.412)(5.083056478405315, 66.46)(5.1827242524916945, 65.599)(5.282392026578074, 65.429)(5.382059800664452, 66.288)(5.48172757475083, 65.604)(5.5813953488372094, 65.585)(5.681063122923588, 65.85)(5.780730897009967, 64.816)(5.880398671096346, 66.384)(5.980066445182724, 66.608)(6.079734219269103, 66.428)(6.179401993355482, 66.474)(6.279069767441861, 65.919)(6.378737541528239, 66.369)(6.4784053156146175, 66.207)(6.578073089700997, 66.294)(6.677740863787376, 66.264)(6.777408637873754, 66.112)(6.877076411960133, 66.164)(6.976744186046512, 66.141)(7.076411960132891, 66.782)(7.17607973421927, 65.29)(7.275747508305648, 66.215)(7.3754152823920265, 65.595)(7.475083056478405, 66.194)(7.574750830564784, 65.51)(7.674418604651163, 65.817)(7.774086378737542, 65.404)(7.8737541528239205, 66.192)(7.9734219269103, 66.378)(8.073089700996677, 67.14)(8.172757475083056, 66.096)(8.272425249169435, 65.991)(8.372093023255815, 66.329)(8.471760797342194, 66.292)(8.571428571428571, 66.449)(8.67109634551495, 66.585)(8.77076411960133, 66.055)(8.870431893687707, 66.103)(8.970099667774086, 66.64)(9.069767441860465, 65.813)(9.169435215946844, 66.307)(9.269102990033224, 66.374)(9.368770764119601, 66.664)(9.46843853820598, 66.43)(9.568106312292358, 65.675)(9.667774086378738, 66.679)(9.767441860465118, 66.05)(9.867109634551495, 65.671)(9.966777408637874, 67.095)(10.066445182724253, 67.001)(10.16611295681063, 67.671)(10.26578073089701, 66.927)(10.365448504983389, 66.948)(10.465116279069768, 67.043)(10.564784053156147, 67.687)(10.664451827242525, 67.611)(10.764119601328904, 67.127)(10.863787375415281, 66.875)(10.96345514950166, 66.951)(11.063122923588042, 67.638)(11.162790697674419, 65.564)(11.262458471760798, 66.344)(11.362126245847175, 66.428)(11.461794019933555, 66.477)(11.561461794019934, 67.099)(11.661129568106311, 67.563)(11.760797342192692, 66.482)(11.860465116279071, 67.48)(11.960132890365449, 66.226)(12.059800664451828, 67.793)(12.159468438538205, 67.332)(12.259136212624584, 66.637)(12.358803986710964, 67.078)(12.458471760797343, 67.396)(12.558139534883722, 66.686)(12.6578073089701, 67.848)(12.757475083056478, 66.899)(12.857142857142858, 67.336)(12.956810631229235, 67.099)(13.056478405315616, 66.579)(13.156146179401993, 68.137)(13.255813953488373, 66.042)(13.355481727574752, 67.381)(13.455149501661129, 65.217)(13.554817275747508, 67.354)(13.654485049833887, 67.02)(13.754152823920267, 66.576)(13.853820598006646, 66.677)(13.953488372093023, 67.224)(14.053156146179402, 67.744)(14.152823920265782, 68.033)(14.252491694352159, 66.554)(14.35215946843854, 66.923)(14.451827242524917, 67.224)(14.551495016611296, 66.771)(14.651162790697676, 66.414)(14.750830564784053, 66.772)(14.850498338870432, 67.35)(14.95016611295681, 40.532)(15.049833887043192, 66.555)(15.149501661129568, 67.484)(15.249169435215947, 67.005)(15.348837209302326, 67.217)(15.448504983388705, 68.131)(15.548172757475085, 67.604)(15.64784053156146, 68.212)(15.747508305647841, 67.877)(15.84717607973422, 67.939)(15.9468438538206, 67.199)(16.04651162790698, 67.78)(16.146179401993354, 67.885)(16.245847176079735, 67.411)(16.345514950166113, 59.044)(16.445182724252494, 52.427)(16.54485049833887, 67.076)(16.64451827242525, 67.235)(16.74418604651163, 66.891)(16.843853820598007, 67.416)(16.943521594684388, 67.621)(17.043189368770765, 67.784)(17.142857142857142, 68.247)(17.24252491694352, 67.471)(17.3421926910299, 67.765)(17.44186046511628, 68.151)(17.54152823920266, 67.583)(17.641196013289036, 67.772)(17.740863787375414, 68.325)(17.840531561461795, 67.965)(17.940199335548172, 67.937)(18.039867109634553, 67.426)(18.13953488372093, 66.117)(18.239202657807308, 67.92)(18.33887043189369, 67.356)(18.438538205980066, 67.226)(18.538205980066447, 67.35)(18.63787375415282, 66.915)(18.737541528239202, 67.589)(18.837209302325583, 67.302)(18.93687707641196, 67.202)(19.03654485049834, 67.596)(19.136212624584715, 68.114)(19.235880398671096, 67.781)(19.335548172757477, 68.019)(19.435215946843854, 68.298)(19.534883720930235, 67.624)(19.634551495016613, 67.448)(19.73421926910299, 68.008)(19.833887043189367, 68.153)(19.93355481727575, 68.065)(20.033222591362126, 67.479)(20.132890365448507, 65.909)(20.232558139534884, 66.995)(20.33222591362126, 66.552)(20.431893687707642, 66.949)(20.53156146179402, 65.941)(20.6312292358804, 37.718)(20.730897009966778, 63.622)(20.830564784053156, 65.639)(20.930232558139537, 66.074)(21.029900332225914, 66.109)(21.129568106312295, 66.404)(21.22923588039867, 66.046)(21.32890365448505, 67.063)(21.42857142857143, 67.61)(21.528239202657808, 66.921)(21.62790697674419, 67.535)(21.727574750830563, 66.312)(21.827242524916944, 66.147)(21.92691029900332, 55.79)(22.026578073089702, 66.32)(22.126245847176083, 66.538)(22.225913621262457, 66.39)(22.325581395348838, 66.436)(22.425249169435215, 66.296)(22.524916943521596, 66.064)(22.624584717607974, 65.551)(22.72425249169435, 65.723)(22.823920265780732, 66.6)(22.92358803986711, 65.425)(23.02325581395349, 57.176)(23.122923588039868, 65.879)(23.222591362126245, 66.257)(23.322259136212622, 67.248)(23.421926910299003, 67.133)(23.521594684385384, 67.142)(23.62126245847176, 66.356)(23.720930232558143, 66.347)(23.820598006644516, 66.587)(23.920265780730897, 66.817)(24.01993355481728, 66.379)(24.119601328903656, 66.72)(24.219269102990037, 67.422)(24.31893687707641, 67.425)(24.41860465116279, 44.047)(24.51827242524917, 62.975)(24.61794019933555, 66.879)(24.717607973421927, 66.386)(24.817275747508305, 66.36)(24.916943521594686, 67.408)(25.016611295681063, 66.475)(25.116279069767444, 66.217)(25.21594684385382, 66.944)(25.3156146179402, 67.12)(25.41528239202658, 66.496)(25.514950166112957, 66.711)(25.614617940199338, 66.89)(25.714285714285715, 67.214)(25.813953488372093, 66.49)(25.91362126245847, 65.461)(26.01328903654485, 66.538)(26.112956810631232, 66.839)(26.21262458471761, 66.783)(26.312292358803987, 66.825)(26.411960132890364, 66.453)(26.511627906976745, 66.015)(26.611295681063122, 66.293)(26.710963455149503, 66.302)(26.81063122923588, 67.149)(26.910299003322258, 66.982)(27.00996677740864, 66.097)(27.109634551495017, 66.776)(27.209302325581397, 66.971)(27.308970099667775, 67.265)(27.408637873754152, 67.168)(27.508305647840533, 67.117)(27.60797342192691, 66.453)(27.70764119601329, 66.977)(27.80730897009967, 66.368)(27.906976744186046, 66.243)(28.006644518272424, 67.246)(28.106312292358805, 66.564)(28.205980066445186, 65.641)(28.305647840531563, 67.327)(28.40531561461794, 66.939)(28.504983388704318, 66.756)(28.6046511627907, 67.098)(28.70431893687708, 67.084)(28.803986710963457, 66.944)(28.903654485049834, 66.856)(29.003322259136212, 66.836)(29.102990033222593, 66.506)(29.20265780730897, 66.547)(29.30232558139535, 66.985)(29.40199335548173, 67.23)(29.501661129568106, 66.964)(29.601328903654487, 65.896)(29.700996677740864, 66.682)(29.800664451827245, 66.633)(29.90033222591362, 66.43)(30.0, 67.504)(30.099667774086384, 67.044)(30.19933554817276, 67.236)(30.299003322259136, 67.761)(30.398671096345517, 66.833)(30.498338870431894, 66.719)(30.598006644518275, 66.571)(30.697674418604652, 65.771)(30.79734219269103, 66.975)(30.89700996677741, 65.88)(30.996677740863788, 66.313)(31.09634551495017, 66.986)(31.196013289036546, 65.74)(31.29568106312292, 66.224)(31.395348837209305, 66.407)(31.495016611295682, 65.279)(31.594684385382063, 66.94)(31.69435215946844, 66.586)(31.794019933554814, 67.168)(31.8936877076412, 66.899)(31.993355481727576, 66.651)(32.09302325581396, 66.865)(32.19269102990033, 67.528)(32.29235880398671, 67.609)(32.39202657807309, 67.644)(32.49169435215947, 67.102)(32.59136212624585, 66.764)(32.691029900332225, 67.363)(32.7906976744186, 66.557)(32.89036544850499, 67.649)(32.990033222591364, 66.384)(33.08970099667774, 67.234)(33.18936877076412, 67.181)(33.2890365448505, 66.455)(33.38870431893688, 67.614)(33.48837209302326, 67.066)(33.588039867109636, 66.801)(33.68770764119601, 67.074)(33.78737541528239, 66.778)(33.887043189368775, 66.313)(33.98671096345515, 67.223)(34.08637873754153, 66.43)(34.18604651162791, 67.812)(34.285714285714285, 67.508)(34.38538205980067, 68.201)(34.48504983388704, 67.338)(34.584717607973424, 67.745)(34.6843853820598, 67.801)(34.78405315614618, 67.472)(34.88372093023256, 67.175)(34.98338870431893, 66.226)(35.08305647840532, 67.071)(35.182724252491695, 67.129)(35.28239202657807, 67.091)(35.38205980066445, 67.371)(35.48172757475083, 66.964)(35.58139534883721, 67.067)(35.68106312292359, 66.252)(35.78073089700997, 66.546)(35.880398671096344, 66.695)(35.98006644518272, 66.035)(36.079734219269106, 67.794)(36.179401993355484, 67.019)(36.27906976744186, 67.473)(36.37873754152824, 67.521)(36.478405315614616, 68.068)(36.578073089701, 67.64)(36.67774086378738, 67.339)(36.777408637873755, 67.686)(36.87707641196013, 67.372)(36.97674418604651, 67.059)(37.076411960132894, 67.534)(37.17607973421927, 67.443)(37.27574750830564, 66.796)(37.37541528239203, 46.604)(37.475083056478404, 66.27)(37.57475083056479, 67.731)(37.674418604651166, 66.555)(37.774086378737536, 66.979)(37.87375415282392, 66.824)(37.9734219269103, 67.301)(38.07308970099668, 66.656)(38.17275747508306, 67.113)(38.27242524916943, 67.781)(38.372093023255815, 66.803)(38.47176079734219, 68.245)(38.57142857142858, 67.576)(38.671096345514954, 67.036)(38.77076411960133, 67.548)(38.87043189368771, 68.0)(38.970099667774086, 66.791)(39.06976744186047, 67.019)(39.16943521594684, 51.563)(39.269102990033225, 65.312)(39.3687707641196, 65.816)(39.46843853820598, 67.16)(39.568106312292365, 66.796)(39.667774086378735, 66.072)(39.76744186046512, 65.582)(39.8671096345515, 62.064)(39.966777408637874, 65.209)(40.06644518272425, 62.686)(40.16611295681063, 64.611)(40.26578073089701, 65.873)(40.36544850498339, 64.826)(40.46511627906977, 65.115)(40.564784053156146, 63.902)(40.66445182724252, 64.803)(40.76411960132891, 64.671)(40.863787375415285, 65.125)(40.96345514950166, 63.808)(41.06312292358804, 64.921)(41.16279069767442, 65.155)(41.2624584717608, 65.467)(41.36212624584718, 64.336)(41.461794019933556, 65.013)(41.561461794019934, 64.945)(41.66112956810631, 65.629)(41.760797342192696, 64.285)(41.86046511627907, 64.41)(41.96013289036544, 64.712)(42.05980066445183, 66.671)(42.159468438538205, 65.515)(42.25913621262459, 63.366)(42.35880398671097, 67.63)(42.45847176079734, 62.769)(42.55813953488372, 63.428)(42.6578073089701, 27.373)(42.757475083056484, 64.856)(42.85714285714286, 64.892)(42.95681063122923, 64.237)(43.056478405315616, 64.679)(43.15614617940199, 64.875)(43.25581395348838, 65.587)(43.355481727574755, 64.157)(43.455149501661126, 65.285)(43.55481727574751, 64.401)(43.65448504983389, 63.906)(43.75415282392027, 65.754)(43.85382059800664, 64.263)(43.95348837209302, 64.697)(44.053156146179404, 65.71)(44.15282392026578, 64.646)(44.252491694352166, 63.213)(44.352159468438536, 66.419)(44.451827242524914, 64.985)(44.5514950166113, 64.979)(44.651162790697676, 64.585)(44.75083056478405, 64.321)(44.85049833887043, 64.969)(44.95016611295681, 64.925)(45.04983388704319, 65.263)(45.14950166112957, 64.366)(45.24916943521595, 64.528)(45.348837209302324, 64.774)(45.4485049833887, 65.526)(45.548172757475086, 64.86)(45.647840531561464, 0.0)(45.74750830564784, 64.009)(45.84717607973422, 64.826)(45.946843853820596, 65.501)(46.04651162790698, 63.829)(46.14617940199336, 65.907)(46.245847176079735, 51.985)(46.34551495016611, 61.581)(46.44518272425249, 64.418)(46.544850498338874, 64.936)(46.644518272425245, 64.396)(46.74418604651163, 65.22)(46.84385382059801, 65.173)(46.94352159468439, 64.644)(47.04318936877077, 65.109)(47.14285714285714, 64.087)(47.24252491694352, 65.318)(47.3421926910299, 64.952)(47.441860465116285, 65.744)(47.54152823920266, 63.908)(47.64119601328903, 65.248)(47.74086378737542, 65.013)(47.840531561461795, 64.394)(47.94019933554818, 65.23)(48.03986710963456, 64.641)(48.13953488372093, 65.42)(48.23920265780731, 61.308)(48.33887043189369, 62.102)(48.43853820598007, 64.739)(48.538205980066444, 64.514)(48.63787375415282, 64.437)(48.737541528239205, 64.876)(48.83720930232558, 64.785)(48.93687707641197, 65.46)(49.03654485049834, 64.91)(49.136212624584715, 65.065)(49.2358803986711, 64.833)(49.33554817275748, 65.194)(49.435215946843854, 64.728)(49.53488372093023, 64.747)(49.63455149501661, 64.662)(49.734219269102994, 64.663)(49.83388704318937, 65.726)(49.93355481727575, 65.125)(50.033222591362126, 64.317)(50.1328903654485, 64.037)(50.23255813953489, 65.773)(50.332225913621265, 63.92)(50.43189368770764, 65.77)(50.53156146179402, 63.918)(50.6312292358804, 65.084)(50.73089700996678, 66.131)(50.83056478405316, 64.079)(50.93023255813954, 48.34)(51.029900332225914, 64.507)(51.12956810631229, 65.216)(51.229235880398676, 64.952)(51.328903654485046, 64.377)(51.42857142857143, 65.334)(51.52823920265781, 65.522)(51.627906976744185, 64.964)(51.72757475083057, 64.836)(51.82724252491694, 64.838)(51.926910299003325, 65.46)(52.0265780730897, 63.853)(52.12624584717608, 64.954)(52.225913621262464, 65.488)(52.325581395348834, 64.843)(52.42524916943522, 65.019)(52.524916943521596, 64.873)(52.62458471760797, 65.276)(52.72425249169436, 63.971)(52.82392026578073, 65.14)(52.92358803986711, 64.971)(53.02325581395349, 64.448)(53.12292358803987, 65.45)(53.222591362126245, 64.418)(53.32225913621262, 65.064)(53.42192691029901, 64.069)(53.521594684385384, 66.301)(53.62126245847176, 63.767)(53.72093023255814, 61.889)(53.820598006644516, 64.939)(53.9202657807309, 63.703)(54.01993355481728, 65.487)(54.119601328903656, 63.103)(54.21926910299003, 63.857)(54.31893687707641, 65.137)(54.418604651162795, 66.29)(54.51827242524917, 63.078)(54.61794019933555, 63.768)(54.71760797342193, 64.499)(54.817275747508305, 64.672)(54.91694352159469, 64.741)(55.01661129568107, 64.846)(55.11627906976744, 63.897)(55.21594684385382, 64.568)(55.3156146179402, 64.775)(55.41528239202658, 65.083)(55.51495016611296, 63.907)(55.61461794019934, 61.741)(55.714285714285715, 64.291)(55.81395348837209, 65.339)(55.91362126245848, 64.328)(56.01328903654485, 64.442)(56.11295681063123, 55.15)(56.21262458471761, 63.51)(56.31229235880399, 64.369)(56.41196013289037, 64.703)(56.51162790697674, 64.753)(56.611295681063126, 65.567)(56.7109634551495, 54.602)(56.81063122923588, 63.382)(56.910299003322265, 64.889)(57.009966777408636, 65.012)(57.10963455149502, 64.073)(57.2093023255814, 64.752)(57.308970099667775, 63.465)(57.40863787375416, 64.266)(57.50830564784053, 63.091)(57.607973421926914, 63.331)(57.70764119601329, 65.015)(57.80730897009967, 63.812)(57.906976744186046, 64.99)(58.006644518272424, 63.621)(58.10631229235881, 64.155)(58.205980066445186, 64.629)(58.30564784053156, 64.989)(58.40531561461794, 63.737)(58.50498338870432, 63.843)(58.6046511627907, 63.916)(58.70431893687708, 64.713)(58.80398671096346, 64.291)(58.903654485049834, 64.454)(59.00332225913621, 64.772)(59.102990033222596, 63.993)(59.202657807308974, 64.863)(59.30232558139535, 64.385)(59.40199335548173, 33.259)(59.501661129568106, 64.302)(59.60132890365449, 62.004)(59.70099667774087, 42.771)(59.80066445182724, 6.99)(59.90033222591362, 4.366)(60.0, 6.631)
            };
            \addplot [pattern=north east lines,pattern color=red] 
            fill between [
                of=A and B,soft clip={domain=0:800},
            ];
            \end{axis}
    \end{tikzpicture}
    \caption{Measuring instrument: IntelPowerGadget} \label{fig:time_series_BinaryTrees_Workstation_IntelPowerGadget}
    \end{subfigure}
    \begin{subfigure}[b]{0.49\linewidth}
        \begin{tikzpicture}
            \pgfplotsset{%
        width=1\linewidth,
        % height=1\textheight
        }
            \begin{axis}[ymax=120,
            xlabel={Time (Seconds)},
            ylabel={Energy Consumption (Joules)},
            ]
            \addplot[color=blue, mark=none,] coordinates { %% AVG value
            (0.0, 23.416895866666664)(0.5042016806722689, 70.43832874999998)(1.0084033613445378, 70.27720612499998)(1.5126050420168067, 69.85395337500002)(2.0168067226890756, 70.21574831666663)(2.5210084033613445, 70.08271050833332)(3.0252100840336134, 69.824715775)(3.5294117647058822, 69.74630001666668)(4.033613445378151, 69.87819336666668)(4.53781512605042, 69.90863196666669)(5.042016806722689, 69.66062571666666)(5.546218487394958, 69.72514245000002)(6.050420168067227, 69.89253426666669)(6.554621848739496, 69.87968071666666)(7.0588235294117645, 69.78443164166667)(7.563025210084033, 69.72227704166667)(8.067226890756302, 69.90545618333334)(8.571428571428571, 69.82972974166663)(9.07563025210084, 69.73229585833332)(9.579831932773109, 69.71789203333333)(10.084033613445378, 70.99870813333331)(10.588235294117647, 71.12630793333331)(11.092436974789916, 71.21068106666667)(11.596638655462183, 70.92620206666669)(12.100840336134453, 71.15646347499998)(12.605042016806722, 71.03932365833332)(13.109243697478991, 71.2106782916667)(13.61344537815126, 71.16464235)(14.117647058823529, 71.35119578333334)(14.621848739495798, 71.1313096666667)(15.126050420168067, 71.01446871666666)(15.630252100840334, 70.87618886666668)(16.134453781512605, 71.11003458333327)(16.638655462184875, 70.94767568333336)(17.142857142857142, 70.96109860833333)(17.647058823529413, 70.81874900833334)(18.15126050420168, 71.15326863333331)(18.65546218487395, 71.1037189416667)(19.159663865546218, 71.03208677499998)(19.66386554621849, 70.80316070833334)(20.168067226890756, 71.35580353333334)(20.672268907563026, 71.25077093333334)(21.176470588235293, 71.30420534166669)(21.680672268907564, 71.48944710833335)(22.18487394957983, 71.48346858333336)(22.6890756302521, 71.3523941)(23.193277310924366, 71.32974611666668)(23.697478991596636, 71.11283331666668)(24.201680672268907, 71.45101959999997)(24.705882352941174, 71.53615614999998)(25.210084033613445, 71.2761910833333)(25.71428571428571, 71.21235852499996)(26.218487394957982, 71.50729149999998)(26.72268907563025, 71.44895375833333)(27.22689075630252, 71.37398051666669)(27.731092436974787, 71.3031527666667)(28.235294117647058, 71.42737105833332)(28.739495798319325, 71.43909808333336)(29.243697478991596, 71.35627554166666)(29.747899159663863, 71.20882117499998)(30.252100840336134, 71.461599025)(30.7563025210084, 71.43061898333332)(31.260504201680668, 71.27226176666665)(31.764705882352942, 71.18654032499997)(32.26890756302521, 71.39260682499999)(32.773109243697476, 71.47434678333336)(33.27731092436975, 71.12725934999999)(33.78151260504202, 70.7025737916667)(34.285714285714285, 70.50878081666666)(34.78991596638655, 70.04048762500001)(35.294117647058826, 69.38029229166665)(35.79831932773109, 68.8597276333333)(36.30252100840336, 68.23411577500002)(36.80672268907563, 67.99371766666668)(37.3109243697479, 67.59132504999998)(37.81512605042017, 67.24314874166667)(38.319327731092436, 67.22006698333333)(38.8235294117647, 66.97897898333332)(39.32773109243698, 66.65912832500003)(39.831932773109244, 66.43761511666666)(40.33613445378151, 66.34046674999999)(40.840336134453786, 66.25137243333334)(41.34453781512605, 66.06322604166668)(41.84873949579832, 65.79897645833333)(42.35294117647059, 65.70897540833333)(42.85714285714286, 65.60212908333332)(43.36134453781513, 65.37750570833333)(43.865546218487395, 65.276101125)(44.36974789915966, 65.1807441)(44.87394957983193, 65.15235120833333)(45.3781512605042, 64.98593985833332)(45.882352941176464, 64.92952544166667)(46.38655462184873, 64.93676580833332)(46.890756302521005, 64.95085582499998)(47.39495798319327, 64.93694430833334)(47.89915966386554, 64.93274870000002)(48.403361344537814, 64.94459794166669)(48.90756302521008, 64.963498825)(49.41176470588235, 64.92195080833336)(49.915966386554615, 64.97305836666666)(50.42016806722689, 64.86660004166666)(50.924369747899156, 64.98105498333331)(51.42857142857142, 64.89631540833334)(51.93277310924369, 64.93707439166667)(52.436974789915965, 65.03422160000004)(52.94117647058823, 64.94498948333336)(53.4453781512605, 64.882828175)(53.949579831932766, 64.93446131666664)(54.45378151260504, 64.90047826666664)(54.95798319327731, 64.89443483333335)(55.462184873949575, 64.87628775833335)(55.96638655462184, 64.929239125)(56.470588235294116, 64.92449320000003)(56.97478991596638, 64.90633440000002)(57.47899159663865, 64.83700954999998)(57.983193277310924, 64.9943992)(58.48739495798319, 64.903208925)(58.99159663865546, 64.88976876666665)(59.495798319327726, 64.8927814)(60.0, 59.24813614166664)
            };
            \addplot[color=blue, mark=none,name path=A] coordinates { %% MAX value
            (0.0, 61.271233)(0.5042016806722689, 81.64015)(1.0084033613445378, 79.81014)(1.5126050420168067, 78.17006)(2.0168067226890756, 72.32718)(2.5210084033613445, 71.93325)(3.0252100840336134, 71.853546)(3.5294117647058822, 71.93668)(4.033613445378151, 72.04971)(4.53781512605042, 72.122604)(5.042016806722689, 71.50974)(5.546218487394958, 71.90124)(6.050420168067227, 71.973625)(6.554621848739496, 71.89263)(7.0588235294117645, 72.00995)(7.563025210084033, 71.78397)(8.067226890756302, 72.37895)(8.571428571428571, 71.91038)(9.07563025210084, 71.82325)(9.579831932773109, 72.55537)(10.084033613445378, 76.701836)(10.588235294117647, 95.99717)(11.092436974789916, 97.1732)(11.596638655462183, 92.06893)(12.100840336134453, 82.32669)(12.605042016806722, 87.12804)(13.109243697478991, 86.022484)(13.61344537815126, 102.589195)(14.117647058823529, 94.11612)(14.621848739495798, 92.51221)(15.126050420168067, 91.71041)(15.630252100840334, 82.30687)(16.134453781512605, 83.04216)(16.638655462184875, 84.47312)(17.142857142857142, 83.94907)(17.647058823529413, 83.62598)(18.15126050420168, 85.05263)(18.65546218487395, 85.64606)(19.159663865546218, 84.23027)(19.66386554621849, 73.12096)(20.168067226890756, 73.85229)(20.672268907563026, 74.85238)(21.176470588235293, 73.52595)(21.680672268907564, 74.0644)(22.18487394957983, 73.84261)(22.6890756302521, 73.30397)(23.193277310924366, 73.443245)(23.697478991596636, 73.33567)(24.201680672268907, 73.38718)(24.705882352941174, 84.41633)(25.210084033613445, 73.19115)(25.71428571428571, 73.380424)(26.218487394957982, 73.831116)(26.72268907563025, 73.71988)(27.22689075630252, 73.60463)(27.731092436974787, 73.4046)(28.235294117647058, 74.664536)(28.739495798319325, 73.44525)(29.243697478991596, 74.7752)(29.747899159663863, 73.61175)(30.252100840336134, 73.97805)(30.7563025210084, 73.317505)(31.260504201680668, 73.40418)(31.764705882352942, 75.45147)(32.26890756302521, 74.27496)(32.773109243697476, 74.021255)(33.27731092436975, 75.36065)(33.78151260504202, 73.354935)(34.285714285714285, 73.47484)(34.78991596638655, 73.58304)(35.294117647058826, 73.59847)(35.79831932773109, 76.190834)(36.30252100840336, 73.30951)(36.80672268907563, 73.98367)(37.3109243697479, 74.040344)(37.81512605042017, 73.25062)(38.319327731092436, 72.94955)(38.8235294117647, 73.26242)(39.32773109243698, 73.46287)(39.831932773109244, 73.36294)(40.33613445378151, 73.386856)(40.840336134453786, 73.6881)(41.34453781512605, 72.73827)(41.84873949579832, 73.17738)(42.35294117647059, 72.638115)(42.85714285714286, 72.92528)(43.36134453781513, 72.14933)(43.865546218487395, 72.52189)(44.36974789915966, 71.90965)(44.87394957983193, 72.37901)(45.3781512605042, 71.80592)(45.882352941176464, 70.4129)(46.38655462184873, 71.35316)(46.890756302521005, 71.985825)(47.39495798319327, 72.384766)(47.89915966386554, 70.65356)(48.403361344537814, 70.96515)(48.90756302521008, 72.31185)(49.41176470588235, 71.96436)(49.915966386554615, 70.91257)(50.42016806722689, 71.65452)(50.924369747899156, 73.09978)(51.42857142857142, 72.53971)(51.93277310924369, 71.75415)(52.436974789915965, 71.01151)(52.94117647058823, 66.34218)(53.4453781512605, 66.91687)(53.949579831932766, 66.21502)(54.45378151260504, 65.80798)(54.95798319327731, 66.90805)(55.462184873949575, 66.62919)(55.96638655462184, 66.80185)(56.470588235294116, 66.89875)(56.97478991596638, 66.07221)(57.47899159663865, 66.22711)(57.983193277310924, 66.43668)(58.48739495798319, 66.85684)(58.99159663865546, 67.65092)(59.495798319327726, 68.18349)(60.0, 64.08069)
            };
            \addplot[color=blue, mark=none,name path=B] coordinates { %% MIN value
            (0.0, 22.038027)(0.5042016806722689, 66.967316)(1.0084033613445378, 67.320274)(1.5126050420168067, 65.88031)(2.0168067226890756, 65.00479)(2.5210084033613445, 66.83649)(3.0252100840336134, 66.94874)(3.5294117647058822, 64.1441)(4.033613445378151, 64.67671)(4.53781512605042, 66.07049)(5.042016806722689, 64.0964)(5.546218487394958, 66.8465)(6.050420168067227, 66.461105)(6.554621848739496, 66.6577)(7.0588235294117645, 66.51022)(7.563025210084033, 66.61196)(8.067226890756302, 66.73348)(8.571428571428571, 66.40518)(9.07563025210084, 64.48417)(9.579831932773109, 66.39599)(10.084033613445378, 67.95544)(10.588235294117647, 67.23137)(11.092436974789916, 67.76965)(11.596638655462183, 68.267075)(12.100840336134453, 67.852325)(12.605042016806722, 67.474785)(13.109243697478991, 67.34062)(13.61344537815126, 67.90885)(14.117647058823529, 67.77252)(14.621848739495798, 67.127914)(15.126050420168067, 63.674793)(15.630252100840334, 68.271935)(16.134453781512605, 67.51884)(16.638655462184875, 62.305)(17.142857142857142, 67.307076)(17.647058823529413, 68.094536)(18.15126050420168, 67.69046)(18.65546218487395, 67.09185)(19.159663865546218, 67.28248)(19.66386554621849, 68.17464)(20.168067226890756, 67.42943)(20.672268907563026, 64.02499)(21.176470588235293, 68.3652)(21.680672268907564, 67.075096)(22.18487394957983, 64.68437)(22.6890756302521, 65.115776)(23.193277310924366, 64.57315)(23.697478991596636, 64.888115)(24.201680672268907, 64.69574)(24.705882352941174, 64.92887)(25.210084033613445, 64.5568)(25.71428571428571, 64.94167)(26.218487394957982, 64.70517)(26.72268907563025, 64.92303)(27.22689075630252, 64.85413)(27.731092436974787, 64.9086)(28.235294117647058, 64.946594)(28.739495798319325, 64.46408)(29.243697478991596, 64.73588)(29.747899159663863, 64.8247)(30.252100840336134, 64.75919)(30.7563025210084, 64.15462)(31.260504201680668, 64.63937)(31.764705882352942, 64.539116)(32.26890756302521, 64.30251)(32.773109243697476, 64.33435)(33.27731092436975, 64.475945)(33.78151260504202, 64.71981)(34.285714285714285, 64.630196)(34.78991596638655, 64.26852)(35.294117647058826, 64.25337)(35.79831932773109, 64.1658)(36.30252100840336, 64.34611)(36.80672268907563, 63.599346)(37.3109243697479, 63.810883)(37.81512605042017, 62.6874)(38.319327731092436, 64.18695)(38.8235294117647, 63.468353)(39.32773109243698, 63.76725)(39.831932773109244, 63.217182)(40.33613445378151, 63.608425)(40.840336134453786, 64.19289)(41.34453781512605, 64.06326)(41.84873949579832, 61.59704)(42.35294117647059, 64.0435)(42.85714285714286, 64.17912)(43.36134453781513, 63.88923)(43.865546218487395, 62.72429)(44.36974789915966, 63.749065)(44.87394957983193, 64.422005)(45.3781512605042, 63.89843)(45.882352941176464, 63.80422)(46.38655462184873, 63.829735)(46.890756302521005, 63.934307)(47.39495798319327, 63.87919)(47.89915966386554, 63.763943)(48.403361344537814, 63.911606)(48.90756302521008, 64.1659)(49.41176470588235, 63.884922)(49.915966386554615, 63.75834)(50.42016806722689, 60.02938)(50.924369747899156, 62.18238)(51.42857142857142, 63.856277)(51.93277310924369, 59.939682)(52.436974789915965, 63.86818)(52.94117647058823, 63.91892)(53.4453781512605, 63.84714)(53.949579831932766, 63.932144)(54.45378151260504, 63.906)(54.95798319327731, 63.886932)(55.462184873949575, 63.793884)(55.96638655462184, 63.940945)(56.470588235294116, 64.0917)(56.97478991596638, 63.902042)(57.47899159663865, 63.849182)(57.983193277310924, 64.35588)(58.48739495798319, 61.00222)(58.99159663865546, 60.841984)(59.495798319327726, 63.935726)(60.0, 56.104656)
            };
            \addplot [pattern=north east lines,pattern color=red] 
            fill between [
                of=A and B,soft clip={domain=0:800},
            ];
            \end{axis}
    \end{tikzpicture}
    \caption{Measuring instrument: LHM} \label{fig:time_series_BinaryTrees_Workstation_LHM}
\end{subfigure}
\begin{subfigure}[b]{0.49\linewidth}
    \begin{tikzpicture}
        \pgfplotsset{%
        width=1\linewidth,
        % height=1\textheight
        }
        \begin{axis}[ymax=120,
        xlabel={Time (Seconds)},
        ylabel={Energy Consumption (Joules)},
        ]
        \addplot[color=blue, mark=none,] coordinates { %% AVG value
        (0.0, 0.0)(0.1001669449081803, 57.7940715)(0.2003338898163606, 63.80245441666663)(0.3005008347245409, 59.70699591666668)(0.4006677796327212, 59.425786)(0.5008347245409015, 59.939936333333335)(0.6010016694490818, 60.0352364166667)(0.7011686143572621, 60.04882149999998)(0.8013355592654424, 60.205361500000016)(0.9015025041736228, 60.99890783333334)(1.001669449081803, 60.196038666666674)(1.1018363939899833, 59.58499166666666)(1.2020033388981637, 60.108549583333335)(1.3021702838063438, 60.137952999999996)(1.4023372287145242, 60.19749333333335)(1.5025041736227045, 60.137505583333336)(1.6026711185308848, 60.11290833333333)(1.7028380634390652, 59.96604850000001)(1.8030050083472455, 60.178674250000014)(1.9031719532554257, 60.059162083333334)(2.003338898163606, 60.21851441666669)(2.1035058430717863, 59.967731583333325)(2.2036727879799667, 60.44271633333332)(2.303839732888147, 60.356377250000016)(2.4040066777963274, 60.16260675000001)(2.5041736227045073, 60.182259166666704)(2.6043405676126876, 60.18638516666666)(2.704507512520868, 59.915150166666656)(2.8046744574290483, 60.25216524999998)(2.9048414023372287, 60.28135)(3.005008347245409, 60.37591325)(3.1051752921535893, 60.00882358333335)(3.2053422370617697, 60.118803583333374)(3.30550918196995, 60.27501241666671)(3.4056761268781304, 60.079827333333334)(3.5058430717863107, 60.18649691666666)(3.606010016694491, 60.402595999999996)(3.7061769616026714, 60.418673750000025)(3.8063439065108513, 60.19580424999995)(3.906510851419032, 60.338544999999996)(4.006677796327212, 60.292071833333345)(4.106844741235393, 60.250333999999995)(4.207011686143573, 60.26594908333335)(4.3071786310517535, 60.40605925)(4.407345575959933, 60.18090724999999)(4.507512520868114, 60.35219133333332)(4.607679465776294, 60.266396166666674)(4.707846410684475, 60.24959600000001)(4.808013355592655, 60.25648875)(4.908180300500835, 60.23536033333334)(5.0083472454090145, 60.51462591666663)(5.108514190317195, 60.49057766666667)(5.208681135225375, 60.24679933333335)(5.308848080133556, 60.27384224999999)(5.409015025041736, 60.40763708333332)(5.509181969949917, 60.169212833333326)(5.609348914858097, 60.32672949999999)(5.709515859766277, 60.57405424999999)(5.809682804674457, 60.15213391666669)(5.909849749582638, 60.368400500000014)(6.010016694490818, 60.28043975)(6.110183639398999, 60.165140166666646)(6.210350584307179, 60.371716333333325)(6.3105175292153595, 60.29423841666666)(6.410684474123539, 60.330803916666696)(6.510851419031719, 60.33832608333334)(6.6110183639399, 60.38646791666665)(6.71118530884808, 60.32951624999999)(6.811352253756261, 60.17364933333333)(6.911519198664441, 60.38885241666666)(7.011686143572621, 60.29364383333334)(7.111853088480801, 60.17931433333334)(7.212020033388982, 60.44979691666668)(7.312186978297162, 60.38820700000005)(7.412353923205343, 60.52332258333332)(7.512520868113523, 60.32151666666667)(7.612687813021703, 60.417839333333355)(7.712854757929884, 60.561485083333345)(7.813021702838064, 60.31200524999999)(7.913188647746244, 60.49232249999997)(8.013355592654424, 60.356611249999986)(8.113522537562606, 60.489397916666704)(8.213689482470786, 60.328565249999976)(8.313856427378965, 59.986001833333354)(8.414023372287145, 60.52388816666666)(8.514190317195327, 60.40567291666665)(8.614357262103507, 60.29866374999998)(8.714524207011687, 60.36799891666669)(8.814691151919867, 60.18856691666665)(8.914858096828048, 60.416669833333344)(9.015025041736228, 60.300845166666676)(9.115191986644408, 60.51730633333334)(9.215358931552588, 60.41613066666667)(9.31552587646077, 60.33716666666668)(9.41569282136895, 60.4153875)(9.51585976627713, 60.37372216666667)(9.61602671118531, 60.24925016666664)(9.71619365609349, 60.44772683333333)(9.81636060100167, 60.41916683333332)(9.916527545909851, 60.28459416666667)(10.016694490818029, 60.64863883333333)(10.11686143572621, 60.76521508333333)(10.21702838063439, 61.0895905)(10.31719532554257, 60.964347166666656)(10.41736227045075, 61.195130250000005)(10.51752921535893, 61.02308841666666)(10.617696160267112, 61.21909108333336)(10.717863105175292, 60.84979491666668)(10.818030050083472, 60.995164333333335)(10.918196994991652, 60.78358141666665)(11.018363939899833, 60.77994983333331)(11.118530884808013, 60.75591250000002)(11.218697829716193, 61.18048166666666)(11.318864774624373, 60.86851741666669)(11.419031719532555, 60.91355608333338)(11.519198664440735, 61.05471925)(11.619365609348915, 60.827139916666646)(11.719532554257095, 60.86003758333334)(11.819699499165276, 60.99945725000001)(11.919866444073456, 60.89096791666665)(12.020033388981636, 60.94379316666666)(12.120200333889816, 60.93660208333334)(12.220367278797998, 60.841594999999984)(12.320534223706177, 60.78538208333332)(12.420701168614357, 61.02982208333332)(12.520868113522537, 61.04433866666667)(12.621035058430719, 60.94567516666668)(12.721202003338899, 61.15881391666669)(12.821368948247079, 60.937197)(12.921535893155259, 60.691007333333346)(13.021702838063439, 61.02273641666669)(13.12186978297162, 60.95710424999999)(13.2220367278798, 60.89361250000002)(13.32220367278798, 60.95298916666664)(13.42237061769616, 61.08974341666667)(13.522537562604342, 60.99898916666666)(13.622704507512521, 60.8261738333333)(13.722871452420701, 60.98121341666667)(13.823038397328881, 60.77967008333336)(13.923205342237063, 60.8235338333333)(14.023372287145243, 61.11858283333332)(14.123539232053423, 61.200104416666626)(14.223706176961603, 60.852565833333365)(14.323873121869784, 61.04004074999998)(14.424040066777964, 60.95868633333331)(14.524207011686144, 60.92924091666665)(14.624373956594324, 61.22128899999999)(14.724540901502506, 60.73802933333331)(14.824707846410686, 60.96690066666667)(14.924874791318866, 61.07944841666666)(15.025041736227045, 60.90268625)(15.125208681135225, 61.042527666666665)(15.225375626043405, 60.883032916666664)(15.325542570951589, 61.00814458333333)(15.425709515859769, 61.19163624999999)(15.525876460767948, 60.90828125000001)(15.626043405676128, 61.21082641666667)(15.726210350584308, 60.78140991666667)(15.826377295492488, 61.08484508333334)(15.926544240400668, 60.84901666666667)(16.026711185308848, 60.82262849999998)(16.126878130217026, 61.05716600000002)(16.22704507512521, 60.689658999999956)(16.32721202003339, 60.99616083333333)(16.42737896494157, 60.9036225)(16.52754590984975, 61.096762250000005)(16.62771285475793, 61.067989666666655)(16.72787979966611, 60.88105416666665)(16.82804674457429, 60.93068108333334)(16.92821368948247, 60.8957945833333)(17.028380634390654, 61.05015191666668)(17.128547579298832, 61.16173875)(17.228714524207014, 60.94413966666669)(17.328881469115192, 60.97779424999997)(17.429048414023374, 60.88866383333332)(17.529215358931552, 60.996436416666675)(17.629382303839733, 60.98030183333332)(17.72954924874791, 60.94239483333333)(17.829716193656097, 60.89054066666667)(17.929883138564275, 60.9689400833333)(18.030050083472457, 61.17646850000001)(18.130217028380635, 61.01235625000003)(18.230383973288816, 61.02599708333332)(18.330550918196995, 60.882275416666694)(18.430717863105176, 60.84084241666669)(18.530884808013354, 61.030748250000016)(18.63105175292154, 61.22533208333338)(18.731218697829718, 61.155554333333335)(18.8313856427379, 61.0709295)(18.931552587646078, 60.96715916666669)(19.03171953255426, 61.22600908333337)(19.131886477462437, 60.63148791666668)(19.23205342237062, 61.043559583333305)(19.332220367278797, 60.888094250000044)(19.43238731218698, 61.342957083333346)(19.53255425709516, 60.83488641666667)(19.63272120200334, 60.93961191666665)(19.73288814691152, 61.133206083333334)(19.833055091819702, 61.04247158333333)(19.93322203672788, 61.01959908333333)(20.033388981636058, 61.30606141666668)(20.13355592654424, 61.11330791666668)(20.23372287145242, 61.35175158333332)(20.333889816360603, 61.21818616666668)(20.43405676126878, 61.29849783333335)(20.534223706176963, 61.48611991666669)(20.63439065108514, 61.3598745)(20.734557595993323, 61.324728000000015)(20.8347245409015, 61.52751641666668)(20.934891485809683, 61.19001375000002)(21.03505843071786, 61.49009224999999)(21.135225375626046, 61.28558966666666)(21.235392320534224, 61.3931483333333)(21.335559265442406, 61.25419191666667)(21.435726210350584, 61.25824025000004)(21.535893155258766, 61.4735619166667)(21.636060100166944, 61.43869558333335)(21.736227045075125, 61.25114066666669)(21.836393989983303, 61.34022575000004)(21.93656093489149, 61.29973974999999)(22.036727879799667, 61.369954249999985)(22.13689482470785, 61.384537749999986)(22.237061769616027, 61.58045424999998)(22.33722871452421, 61.36048491666666)(22.437395659432386, 61.366196249999994)(22.537562604340568, 61.37972516666667)(22.637729549248746, 61.149506833333334)(22.737896494156928, 61.31521675000001)(22.83806343906511, 61.35717333333334)(22.93823038397329, 61.38604775000002)(23.03839732888147, 61.47958908333332)(23.13856427378965, 61.08311074999998)(23.23873121869783, 61.2924913333333)(23.33889816360601, 61.23666975000002)(23.43906510851419, 61.370382666666664)(23.53923205342237, 61.143997999999996)(23.639398998330552, 61.39009208333336)(23.739565943238734, 61.40644291666667)(23.839732888146912, 61.27509641666669)(23.939899833055094, 61.56962124999999)(24.040066777963272, 61.47033208333334)(24.140233722871454, 61.308146916666686)(24.24040066777963, 61.3004465)(24.340567612687813, 61.58761566666669)(24.440734557595995, 61.282227666666685)(24.540901502504177, 61.362681083333335)(24.641068447412355, 61.512863833333356)(24.741235392320537, 61.19368083333333)(24.841402337228715, 61.14600766666664)(24.941569282136896, 61.61427741666664)(25.041736227045075, 61.52165291666665)(25.141903171953256, 61.21052583333333)(25.242070116861438, 61.35980866666664)(25.34223706176962, 61.530613833333334)(25.442404006677798, 61.3649198333333)(25.54257095158598, 61.60506699999998)(25.642737896494157, 61.53290249999999)(25.74290484140234, 61.50602816666663)(25.843071786310517, 61.15455224999998)(25.9432387312187, 61.62108324999999)(26.043405676126877, 61.253302166666664)(26.143572621035062, 61.70309908333329)(26.24373956594324, 61.693084)(26.34390651085142, 61.51294058333333)(26.4440734557596, 61.34507266666669)(26.544240400667782, 61.25377983333335)(26.64440734557596, 61.57712841666669)(26.744574290484138, 61.299001833333335)(26.84474123539232, 61.42871649999995)(26.9449081803005, 61.62739533333333)(27.045075125208683, 61.52557349999999)(27.14524207011686, 61.380849999999995)(27.245409015025043, 61.43880233333333)(27.34557595993322, 61.85000066666668)(27.445742904841403, 61.81530141666667)(27.54590984974958, 61.51875316666669)(27.646076794657763, 61.42199791666666)(27.746243739565944, 61.44813600000001)(27.846410684474126, 61.42576674999997)(27.946577629382304, 61.65102583333333)(28.046744574290486, 61.38394708333335)(28.146911519198664, 61.35779433333333)(28.247078464106846, 61.741047)(28.347245409015024, 61.490535333333355)(28.447412353923205, 61.82743774999997)(28.547579298831387, 61.84448716666666)(28.64774624373957, 61.56499191666667)(28.747913188647747, 62.03253533333334)(28.84808013355593, 61.55611683333336)(28.948247078464107, 61.9182935)(29.04841402337229, 61.47733616666668)(29.148580968280466, 61.557540666666675)(29.248747913188648, 61.74994916666664)(29.348914858096826, 61.89189066666669)(29.44908180300501, 61.58875016666668)(29.54924874791319, 61.54537400000002)(29.64941569282137, 61.897542333333334)(29.74958263772955, 61.81869416666665)(29.84974958263773, 61.61930808333332)(29.94991652754591, 61.50146533333336)(30.05008347245409, 61.67316674999997)(30.15025041736227, 61.84909525)(30.25041736227045, 61.30754666666669)(30.35058430717863, 61.58731025000001)(30.45075125208681, 61.30751699999999)(30.55091819699499, 61.41162150000001)(30.651085141903177, 61.77350783333332)(30.751252086811355, 61.62727308333334)(30.851419031719537, 61.23680758333334)(30.951585976627715, 61.683231999999975)(31.051752921535897, 61.25283925)(31.151919866444075, 61.69802341666669)(31.252086811352257, 61.78307549999998)(31.352253756260435, 61.77882333333333)(31.452420701168617, 61.67873608333336)(31.552587646076795, 61.423492749999994)(31.652754590984976, 61.35286016666666)(31.752921535893154, 61.39640333333334)(31.853088480801336, 61.61474099999999)(31.953255425709514, 61.71657308333333)(32.053422370617696, 61.39479583333333)(32.15358931552588, 61.57190958333331)(32.25375626043405, 61.5165405)(32.35392320534224, 61.76733366666667)(32.45409015025042, 61.66829458333332)(32.554257095158604, 61.50992341666667)(32.65442404006678, 61.480962000000005)(32.75459098497496, 61.48546408333333)(32.85475792988314, 61.59574925000001)(32.95492487479132, 61.607284166666666)(33.0550918196995, 61.393620916666684)(33.15525876460768, 61.606887833333325)(33.25542570951586, 61.466594166666674)(33.35559265442404, 61.59081041666668)(33.45575959933222, 61.620574249999976)(33.5559265442404, 61.58214841666667)(33.65609348914858, 61.62198375000003)(33.756260434056756, 61.58765683333331)(33.85642737896494, 61.77643741666668)(33.95659432387313, 61.450419499999974)(34.05676126878131, 61.203029499999985)(34.15692821368948, 61.76669291666667)(34.257095158597664, 61.54540991666669)(34.357262103505846, 61.50400366666664)(34.45742904841403, 61.57567808333337)(34.5575959933222, 61.53216058333333)(34.657762938230384, 61.64867174999998)(34.757929883138566, 61.58487975)(34.85809682804675, 61.67661516666668)(34.95826377295492, 61.69606466666667)(35.058430717863104, 61.702585750000026)(35.158597662771285, 61.785048833333335)(35.25876460767947, 61.46685791666668)(35.35893155258764, 61.91199183333335)(35.45909849749582, 61.43084291666667)(35.559265442404005, 61.699487666666705)(35.659432387312194, 61.69932524999998)(35.75959933222037, 61.897216000000036)(35.85976627712855, 61.659449166666676)(35.95993322203673, 61.67231725)(36.06010016694491, 61.65982508333333)(36.16026711185309, 61.640859000000006)(36.26043405676127, 61.47971641666667)(36.36060100166945, 61.65524225000001)(36.46076794657763, 61.696680916666665)(36.56093489148581, 61.57928983333333)(36.66110183639399, 61.87003525000001)(36.76126878130217, 61.75618925000002)(36.86143572621035, 61.67555208333335)(36.96160267111853, 61.654785333333315)(37.06176961602671, 61.556310166666634)(37.16193656093489, 61.577183583333344)(37.26210350584308, 61.60321066666667)(37.362270450751254, 61.81627866666669)(37.462437395659435, 61.35552016666665)(37.56260434056762, 61.802906833333324)(37.6627712854758, 61.480051916666675)(37.76293823038397, 61.67654416666666)(37.863105175292155, 61.62284799999996)(37.96327212020034, 61.69437191666668)(38.06343906510852, 61.614939083333326)(38.16360601001669, 61.657831249999994)(38.263772954924875, 61.49304250000001)(38.363939899833056, 61.775303750000006)(38.46410684474124, 61.49519899999999)(38.56427378964941, 61.79237800000001)(38.664440734557594, 61.602411416666676)(38.764607679465776, 61.565806416666675)(38.86477462437396, 61.73516833333336)(38.96494156928214, 61.74035083333333)(39.06510851419032, 61.560714083333345)(39.1652754590985, 61.88214575000003)(39.26544240400668, 61.63548333333333)(39.36560934891486, 61.771738000000006)(39.46577629382304, 61.54447375000003)(39.56594323873122, 61.77936808333336)(39.666110183639404, 61.70485933333332)(39.76627712854758, 61.405787583333314)(39.86644407345576, 61.74842791666664)(39.96661101836394, 61.532704250000016)(40.066777963272116, 61.89468891666663)(40.1669449081803, 61.72465441666666)(40.26711185308848, 61.70606991666666)(40.36727879799666, 61.695347583333316)(40.46744574290484, 61.513911333333354)(40.567612687813025, 61.93411141666665)(40.667779632721206, 61.49984325000005)(40.76794657762939, 61.60420133333335)(40.86811352253756, 62.00969941666665)(40.968280467445744, 61.633045916666696)(41.068447412353926, 61.770807250000004)(41.16861435726211, 61.61435966666667)(41.26878130217028, 61.55798341666668)(41.368948247078464, 61.80717900000002)(41.469115191986646, 61.718596666666684)(41.56928213689483, 61.716319250000005)(41.669449081803, 61.981154083333344)(41.769616026711184, 61.54389974999997)(41.869782971619365, 61.866642583333324)(41.96994991652755, 61.430949500000004)(42.07011686143572, 61.828866833333336)(42.1702838063439, 62.18060625000001)(42.27045075125209, 61.43257691666667)(42.370617696160274, 61.51089516666666)(42.47078464106845, 61.70638958333332)(42.57095158597663, 62.05361358333333)(42.67111853088481, 61.68316083333331)(42.77128547579299, 61.55980016666669)(42.87145242070117, 61.904824499999954)(42.97161936560935, 61.76059466666667)(43.07178631051753, 61.78499266666668)(43.17195325542571, 61.69599400000002)(43.27212020033389, 61.97305691666666)(43.37228714524207, 61.59453375000001)(43.47245409015025, 61.5904685)(43.57262103505843, 61.87355008333334)(43.67278797996661, 61.66939774999999)(43.77295492487479, 61.866932916666656)(43.87312186978298, 61.782073333333344)(43.97328881469116, 61.540669666666666)(44.073455759599334, 61.706110166666654)(44.173622704507515, 61.528905499999986)(44.2737896494157, 61.72950758333334)(44.37395659432388, 61.61952116666664)(44.47412353923205, 61.917963166666645)(44.574290484140235, 61.877440666666686)(44.67445742904842, 61.509481083333334)(44.7746243739566, 61.92065883333335)(44.87479131886477, 61.44632016666664)(44.974958263772955, 61.64944408333331)(45.075125208681136, 61.86713574999999)(45.17529215358932, 61.89352791666665)(45.27545909849749, 62.07966516666667)(45.375626043405674, 61.673111333333324)(45.475792988313856, 61.85286908333335)(45.57595993322204, 62.06362241666666)(45.67612687813022, 61.572703000000025)(45.7762938230384, 61.78447441666665)(45.87646076794658, 61.68374091666665)(45.976627712854764, 61.730402333333345)(46.07679465776294, 61.62381983333333)(46.17696160267112, 61.887257166666686)(46.2771285475793, 61.9343101666667)(46.37729549248748, 61.53286766666665)(46.47746243739566, 61.74630150000003)(46.57762938230384, 62.02570516666667)(46.67779632721202, 61.7687425)(46.7779632721202, 61.46372008333333)(46.87813021702838, 62.03506333333332)(46.97829716193656, 61.92095366666669)(47.07846410684474, 61.73186183333333)(47.17863105175292, 61.87447033333335)(47.278797996661105, 61.938317916666634)(47.378964941569286, 61.91874091666664)(47.47913188647747, 61.89146833333333)(47.57929883138564, 61.964883916666665)(47.679465776293824, 61.71888199999999)(47.779632721202006, 61.959166333333336)(47.87979966611019, 61.72896316666664)(47.97996661101836, 62.12710391666668)(48.080133555926544, 61.984089416666684)(48.180300500834726, 61.617548416666665)(48.28046744574291, 62.067050750000035)(48.38063439065108, 61.74981708333333)(48.48080133555926, 61.76605658333332)(48.580968280467445, 61.82574383333331)(48.68113522537563, 61.764368500000025)(48.7813021702838, 61.90880741666665)(48.88146911519199, 61.657481000000026)(48.98163606010017, 62.11769441666667)(49.081803005008354, 61.60123191666664)(49.18196994991653, 61.64368700000001)(49.28213689482471, 62.28719441666669)(49.38230383973289, 61.64799983333333)(49.48247078464107, 62.13684933333333)(49.58263772954925, 61.81796700000003)(49.68280467445743, 61.92013458333335)(49.78297161936561, 61.842386333333344)(49.88313856427379, 61.53159158333334)(49.98330550918197, 62.22936291666667)(50.08347245409015, 61.70774325000004)(50.18363939899833, 61.78212916666669)(50.28380634390651, 62.23482074999997)(50.38397328881469, 61.784296750000024)(50.484140233722876, 61.837142000000014)(50.58430717863106, 61.739175833333334)(50.68447412353924, 61.87357058333332)(50.784641068447414, 61.87806575)(50.884808013355595, 61.73561575)(50.98497495826378, 62.21994908333332)(51.08514190317196, 61.60091650000004)(51.18530884808013, 61.54906658333332)(51.285475792988315, 62.30517974999999)(51.3856427378965, 61.751860583333375)(51.48580968280468, 61.95909033333333)(51.58597662771285, 61.93246883333333)(51.686143572621035, 61.988992916666675)(51.786310517529216, 61.752140749999995)(51.8864774624374, 61.69964541666666)(51.98664440734557, 62.14552183333337)(52.086811352253754, 61.92377108333337)(52.18697829716194, 62.12459691666666)(52.287145242070125, 61.79213399999999)(52.3873121869783, 61.791015250000015)(52.48747913188648, 61.93137525000001)(52.58764607679466, 61.82186791666668)(52.68781302170284, 62.21235041666668)(52.78797996661102, 61.850167916666656)(52.8881469115192, 61.61434933333332)(52.98831385642738, 62.10032000000002)(53.088480801335564, 61.98020375000002)(53.18864774624374, 61.82971608333334)(53.28881469115192, 61.96628766666668)(53.3889816360601, 61.82462991666667)(53.489148580968276, 62.08432874999996)(53.58931552587646, 61.86589000000001)(53.68948247078464, 61.99963233333335)(53.78964941569283, 61.55037483333332)(53.889816360601, 61.616205999999984)(53.989983305509185, 62.14438716666667)(54.090150250417366, 62.09283349999998)(54.19031719532555, 61.84919608333333)(54.29048414023372, 61.47191349999999)(54.390651085141904, 61.868662916666686)(54.490818030050086, 61.821425583333344)(54.59098497495827, 61.924076333333325)(54.69115191986644, 62.30334274999998)(54.791318864774624, 61.70664466666667)(54.891485809682806, 61.61206608333335)(54.99165275459099, 62.14132016666666)(55.09181969949916, 62.07598750000001)(55.19198664440734, 61.61523433333332)(55.292153589315525, 62.215441916666656)(55.39232053422371, 62.08371366666668)(55.49248747913189, 61.807957833333326)(55.59265442404007, 61.70070341666668)(55.69282136894825, 62.15571441666666)(55.79298831385643, 61.827401916666645)(55.89315525876461, 62.19985299999998)(55.99332220367279, 62.0472558333333)(56.09348914858097, 61.686761916666676)(56.19365609348915, 61.787877)(56.29382303839733, 62.206062916666674)(56.39398998330551, 62.06733116666667)(56.49415692821369, 61.99412874999998)(56.59432387312187, 61.800379916666664)(56.69449081803005, 62.066236500000024)(56.79465776293823, 61.83323133333332)(56.89482470784641, 61.97709491666668)(56.99499165275459, 62.089385166666645)(57.095158597662774, 62.073520416666675)(57.195325542570956, 62.11131158333332)(57.29549248747914, 61.91701708333333)(57.39565943238732, 61.96328191666667)(57.495826377295494, 61.76019724999999)(57.595993322203675, 62.08104816666666)(57.69616026711186, 62.205412)(57.79632721202004, 61.88396608333333)(57.89649415692821, 61.770284083333316)(57.996661101836395, 61.94094216666665)(58.09682804674458, 62.109236666666646)(58.19699499165276, 61.806924333333335)(58.29716193656093, 61.93247950000002)(58.397328881469114, 62.15529725000002)(58.497495826377296, 61.75043758333332)(58.59766277128548, 61.80088666666665)(58.69782971619365, 62.16149749999999)(58.79799666110184, 62.018106083333315)(58.89816360601002, 61.68694583333335)(58.9983305509182, 62.101744000000025)(59.09849749582638, 62.25498316666666)(59.19866444073456, 62.075087333333315)(59.29883138564274, 61.99443425000001)(59.398998330550924, 62.09195866666664)(59.4991652754591, 61.80208275000003)(59.59933222036728, 61.92220458333332)(59.69949916527546, 62.184217666666655)(59.79966611018364, 61.86635275000001)(59.89983305509182, 62.20994424999999)(60.0, 61.96823016666667)
        };
        \addplot[color=blue, mark=none,name path=A] coordinates { %% MAX value
        (0.0, 0.0)(0.1001669449081803, 62.37411)(0.2003338898163606, 67.08173000000001)(0.3005008347245409, 68.20661)(0.4006677796327212, 73.45439999999999)(0.5008347245409015, 64.15877)(0.6010016694490818, 67.33076)(0.7011686143572621, 78.27251000000001)(0.8013355592654424, 68.01802)(0.9015025041736228, 66.44331)(1.001669449081803, 63.792559999999995)(1.1018363939899833, 62.93502)(1.2020033388981637, 63.276199999999996)(1.3021702838063438, 63.72238)(1.4023372287145242, 64.84725)(1.5025041736227045, 63.49105)(1.6026711185308848, 64.1191)(1.7028380634390652, 62.8221)(1.8030050083472455, 63.73519)(1.9031719532554257, 63.927440000000004)(2.003338898163606, 63.92501)(2.1035058430717863, 64.29671)(2.2036727879799667, 64.32356999999999)(2.303839732888147, 69.19599)(2.4040066777963274, 64.18563)(2.5041736227045073, 63.46542)(2.6043405676126876, 63.23287)(2.704507512520868, 63.20968)(2.8046744574290483, 63.49227)(2.9048414023372287, 65.84394)(3.005008347245409, 64.81307)(3.1051752921535893, 63.18832)(3.2053422370617697, 63.475789999999996)(3.30550918196995, 63.60091)(3.4056761268781304, 63.256069999999994)(3.5058430717863107, 63.94942)(3.606010016694491, 63.31526)(3.7061769616026714, 63.18344)(3.8063439065108513, 62.87948)(3.906510851419032, 63.908519999999996)(4.006677796327212, 63.025349999999996)(4.106844741235393, 63.71566)(4.207011686143573, 63.962849999999996)(4.3071786310517535, 64.43892)(4.407345575959933, 66.8437)(4.507512520868114, 63.83529)(4.607679465776294, 63.29573)(4.707846410684475, 65.02119)(4.808013355592655, 64.17830000000001)(4.908180300500835, 67.62311)(5.0083472454090145, 63.33907)(5.108514190317195, 64.99434)(5.208681135225375, 62.92586)(5.308848080133556, 64.29732)(5.409015025041736, 63.04977)(5.509181969949917, 63.666219999999996)(5.609348914858097, 63.49105)(5.709515859766277, 64.02205000000001)(5.809682804674457, 63.8243)(5.909849749582638, 63.183429999999994)(6.010016694490818, 63.66194)(6.110183639398999, 63.40194)(6.210350584307179, 64.17769)(6.3105175292153595, 63.83406)(6.410684474123539, 63.52218)(6.510851419031719, 64.0428)(6.6110183639399, 62.98018)(6.71118530884808, 63.550250000000005)(6.811352253756261, 63.025349999999996)(6.911519198664441, 65.48201)(7.011686143572621, 63.6119)(7.111853088480801, 63.487989999999996)(7.212020033388982, 64.13374)(7.312186978297162, 64.35285999999999)(7.412353923205343, 65.86165)(7.512520868113523, 65.68709)(7.612687813021703, 64.85029)(7.712854757929884, 63.48921)(7.813021702838064, 68.69245000000001)(7.913188647746244, 63.93477)(8.013355592654424, 63.48922)(8.113522537562606, 64.1661)(8.213689482470786, 62.468709999999994)(8.313856427378965, 63.38912)(8.414023372287145, 63.80599)(8.514190317195327, 63.71199)(8.614357262103507, 62.99056)(8.714524207011687, 64.39376)(8.814691151919867, 63.67782)(8.914858096828048, 64.88081)(9.015025041736228, 63.38728)(9.115191986644408, 72.18731)(9.215358931552588, 64.53597)(9.31552587646077, 63.754720000000006)(9.41569282136895, 63.45077)(9.51585976627713, 64.22652000000001)(9.61602671118531, 64.29732)(9.71619365609349, 63.990919999999996)(9.81636060100167, 63.478229999999996)(9.916527545909851, 63.4709)(10.016694490818029, 63.06563)(10.11686143572621, 64.43404)(10.21702838063439, 65.20125)(10.31719532554257, 63.9714)(10.41736227045075, 65.12007)(10.51752921535893, 63.974450000000004)(10.617696160267112, 64.1014)(10.717863105175292, 64.46639)(10.818030050083472, 64.27291)(10.918196994991652, 64.29854)(11.018363939899833, 63.205400000000004)(11.118530884808013, 65.04073)(11.218697829716193, 65.44416)(11.318864774624373, 63.68697)(11.419031719532555, 64.9095)(11.519198664440735, 64.43647999999999)(11.619365609348915, 63.59969)(11.719532554257095, 64.0544)(11.819699499165276, 64.05378999999999)(11.919866444073456, 63.56979)(12.020033388981636, 65.2983)(12.120200333889816, 64.81307)(12.220367278797998, 64.77338999999999)(12.320534223706177, 63.74374)(12.420701168614357, 65.72432)(12.520868113522537, 63.99337)(12.621035058430719, 64.13863)(12.721202003338899, 63.98299)(12.821368948247079, 64.14901)(12.921535893155259, 64.31014)(13.021702838063439, 63.45443)(13.12186978297162, 64.33699)(13.2220367278798, 63.67476)(13.32220367278798, 64.19112)(13.42237061769616, 64.05135)(13.522537562604342, 64.83565)(13.622704507512521, 63.84566)(13.722871452420701, 64.69405)(13.823038397328881, 65.21223)(13.923205342237063, 64.07699)(14.023372287145243, 64.05196000000001)(14.123539232053423, 64.41146)(14.223706176961603, 64.05989)(14.323873121869784, 66.87604999999999)(14.424040066777964, 63.941489999999995)(14.524207011686144, 63.8896)(14.624373956594324, 64.84419)(14.724540901502506, 64.40108000000001)(14.824707846410686, 64.71908)(14.924874791318866, 64.06539000000001)(15.025041736227045, 64.04035999999999)(15.125208681135225, 64.36934)(15.225375626043405, 64.08553)(15.325542570951589, 63.890820000000005)(15.425709515859769, 65.82869)(15.525876460767948, 64.02144)(15.626043405676128, 65.33614)(15.726210350584308, 64.05624)(15.826377295492488, 64.467)(15.926544240400668, 64.26558)(16.026711185308848, 64.60371)(16.126878130217026, 64.57075999999999)(16.22704507512521, 63.77181)(16.32721202003339, 64.00008)(16.42737896494157, 64.33760000000001)(16.52754590984975, 64.71114)(16.62771285475793, 64.56953)(16.72787979966611, 64.56099)(16.82804674457429, 64.75202999999999)(16.92821368948247, 66.29683)(17.028380634390654, 65.37092)(17.128547579298832, 65.28425)(17.228714524207014, 64.11788)(17.328881469115192, 66.70332)(17.429048414023374, 65.13656)(17.529215358931552, 63.995810000000006)(17.629382303839733, 65.81953)(17.72954924874791, 63.85482)(17.829716193656097, 63.4703)(17.929883138564275, 65.5052)(18.030050083472457, 63.82552)(18.130217028380635, 64.69771)(18.230383973288816, 64.44686)(18.330550918196995, 65.25801)(18.430717863105176, 63.50143)(18.530884808013354, 65.22323)(18.63105175292154, 64.71236)(18.731218697829718, 65.55404)(18.8313856427379, 64.24239)(18.931552587646078, 64.80452)(19.03171953255426, 63.78646)(19.131886477462437, 64.8973)(19.23205342237062, 65.38985)(19.332220367278797, 64.07149000000001)(19.43238731218698, 64.29427)(19.53255425709516, 64.23141)(19.63272120200334, 63.50753)(19.73288814691152, 64.3254)(19.833055091819702, 63.86398)(19.93322203672788, 63.41781)(20.033388981636058, 65.86653)(20.13355592654424, 63.95736)(20.23372287145242, 64.85884)(20.333889816360603, 64.96443000000001)(20.43405676126878, 64.8088)(20.534223706176963, 64.59456)(20.63439065108514, 64.21431)(20.734557595993323, 64.56527)(20.8347245409015, 64.07088)(20.934891485809683, 64.96748)(21.03505843071786, 64.15389)(21.135225375626046, 64.63485)(21.235392320534224, 67.40339)(21.335559265442406, 65.33736)(21.435726210350584, 64.89363)(21.535893155258766, 64.49202)(21.636060100166944, 64.85762)(21.736227045075125, 65.30989)(21.836393989983303, 65.38253)(21.93656093489149, 64.80025)(22.036727879799667, 65.30989)(22.13689482470785, 64.80452)(22.237061769616027, 64.61043000000001)(22.33722871452421, 65.5937)(22.437395659432386, 64.71786)(22.537562604340568, 66.13936)(22.637729549248746, 64.69588)(22.737896494156928, 65.17072999999999)(22.83806343906511, 63.76693)(22.93823038397329, 65.26534)(23.03839732888147, 65.17806)(23.13856427378965, 63.80904)(23.23873121869783, 64.09407)(23.33889816360601, 66.00935)(23.43906510851419, 66.05391)(23.53923205342237, 65.16279)(23.639398998330552, 63.98055)(23.739565943238734, 64.44319)(23.839732888146912, 64.83931)(23.939899833055094, 66.55074)(24.040066777963272, 65.72187)(24.140233722871454, 64.32356999999999)(24.24040066777963, 64.89364)(24.340567612687813, 64.86068)(24.440734557595995, 64.91744)(24.540901502504177, 65.16341)(24.641068447412355, 65.5107)(24.741235392320537, 64.0721)(24.841402337228715, 64.69649000000001)(24.941569282136896, 65.9062)(25.041736227045075, 65.82869)(25.141903171953256, 63.98666)(25.242070116861438, 65.25252)(25.34223706176962, 64.72883999999999)(25.442404006677798, 64.49508)(25.54257095158598, 65.22567)(25.642737896494157, 64.86556)(25.74290484140234, 65.63094000000001)(25.843071786310517, 65.03402)(25.9432387312187, 64.7563)(26.043405676126877, 64.55305)(26.143572621035062, 65.28975)(26.24373956594324, 65.79756)(26.34390651085142, 65.33675)(26.4440734557596, 64.73006)(26.544240400667782, 64.64095)(26.64440734557596, 65.95259)(26.744574290484138, 64.07943)(26.84474123539232, 65.47469000000001)(26.9449081803005, 72.68109)(27.045075125208683, 65.12129)(27.14524207011686, 64.77523)(27.245409015025043, 70.36664)(27.34557595993322, 85.32632)(27.445742904841403, 79.98026)(27.54590984974958, 69.00984)(27.646076794657763, 65.14571000000001)(27.746243739565944, 65.49116)(27.846410684474126, 65.42586)(27.946577629382304, 65.31905)(28.046744574290486, 63.86885)(28.146911519198664, 64.71297)(28.247078464106846, 66.96821)(28.347245409015024, 70.79267)(28.447412353923205, 73.93109)(28.547579298831387, 75.62359)(28.64774624373957, 73.33417)(28.747913188647747, 77.98686000000001)(28.84808013355593, 74.96074)(28.948247078464107, 71.74419999999999)(29.04841402337229, 72.25140999999999)(29.148580968280466, 72.76226)(29.248747913188648, 74.26678)(29.348914858096826, 74.55914)(29.44908180300501, 71.91021)(29.54924874791319, 74.88934)(29.64941569282137, 73.0473)(29.74958263772955, 78.72783)(29.84974958263773, 74.7636)(29.94991652754591, 65.20185)(30.05008347245409, 65.19819)(30.15025041736227, 64.81978)(30.25041736227045, 64.94246000000001)(30.35058430717863, 64.74043)(30.45075125208681, 64.51888)(30.55091819699499, 64.88569)(30.651085141903177, 64.98945)(30.751252086811355, 64.09224)(30.851419031719537, 64.66414)(30.951585976627715, 66.43721000000001)(31.051752921535897, 64.7325)(31.151919866444075, 64.99679)(31.252086811352257, 66.28462)(31.352253756260435, 73.96404000000001)(31.452420701168617, 69.76666)(31.552587646076795, 65.90682)(31.652754590984976, 64.22407)(31.752921535893154, 65.2635)(31.853088480801336, 65.32087)(31.953255425709514, 67.73909)(32.053422370617696, 64.52132)(32.15358931552588, 64.67818)(32.25375626043405, 65.07369)(32.35392320534224, 64.88508999999999)(32.45409015025042, 65.19391999999999)(32.554257095158604, 64.58052)(32.65442404006678, 65.11153)(32.75459098497496, 65.48812)(32.85475792988314, 64.38765000000001)(32.95492487479132, 65.77253999999999)(33.0550918196995, 65.32149000000001)(33.15525876460768, 65.82808)(33.25542570951586, 65.25312)(33.35559265442404, 65.222)(33.45575959933222, 64.42183)(33.5559265442404, 65.14509000000001)(33.65609348914858, 65.35749)(33.756260434056756, 66.72713)(33.85642737896494, 65.45332)(33.95659432387313, 64.99494999999999)(34.05676126878131, 64.62324)(34.15692821368948, 64.94491)(34.257095158597664, 64.73188999999999)(34.357262103505846, 65.59797999999999)(34.45742904841403, 64.90644999999999)(34.5575959933222, 64.01229)(34.657762938230384, 65.43929)(34.757929883138566, 64.92659)(34.85809682804675, 66.85957)(34.95826377295492, 65.17744)(35.058430717863104, 66.77351)(35.158597662771285, 64.58541)(35.25876460767947, 64.25641999999999)(35.35893155258764, 66.77595)(35.45909849749582, 64.40108000000001)(35.559265442404005, 65.70235)(35.659432387312194, 66.05696)(35.75959933222037, 66.55378999999999)(35.85976627712855, 64.30587)(35.95993322203673, 64.52009000000001)(36.06010016694491, 66.05024)(36.16026711185309, 65.50093000000001)(36.26043405676127, 65.82381)(36.36060100166945, 65.8055)(36.46076794657763, 64.96748)(36.56093489148581, 65.77986)(36.66110183639399, 65.08588999999999)(36.76126878130217, 65.70600999999999)(36.86143572621035, 65.13228)(36.96160267111853, 65.77131)(37.06176961602671, 65.02852)(37.16193656093489, 65.1103)(37.26210350584308, 65.63033)(37.362270450751254, 65.90376)(37.462437395659435, 65.15914000000001)(37.56260434056762, 64.95161999999999)(37.6627712854758, 64.8973)(37.76293823038397, 66.19307)(37.863105175292155, 65.23909)(37.96327212020034, 65.39046)(38.06343906510852, 64.71419)(38.16360601001669, 65.3166)(38.263772954924875, 65.51558)(38.363939899833056, 64.86922)(38.46410684474124, 65.41243)(38.56427378964941, 65.37214)(38.664440734557594, 64.47249000000001)(38.764607679465776, 65.94038)(38.86477462437396, 64.84908)(38.96494156928214, 65.86409)(39.06510851419032, 65.15364)(39.1652754590985, 66.69660999999999)(39.26544240400668, 67.05794)(39.36560934891486, 66.67157999999999)(39.46577629382304, 65.68464)(39.56594323873122, 65.83052)(39.666110183639404, 64.64034)(39.76627712854758, 65.15486)(39.86644407345576, 65.46674999999999)(39.96661101836394, 64.50422999999999)(40.066777963272116, 66.95296)(40.1669449081803, 65.24091999999999)(40.26711185308848, 66.23945)(40.36727879799666, 65.3929)(40.46744574290484, 66.25288)(40.567612687813025, 65.11702)(40.667779632721206, 65.56136000000001)(40.76794657762939, 64.95161999999999)(40.86811352253756, 65.20186)(40.968280467445744, 65.29097)(41.068447412353926, 65.3575)(41.16861435726211, 64.91255)(41.26878130217028, 65.10909)(41.368948247078464, 66.01118)(41.469115191986646, 65.68098)(41.56928213689483, 64.59517)(41.669449081803, 65.07307)(41.769616026711184, 65.73957)(41.869782971619365, 64.80330000000001)(41.96994991652755, 65.04256)(42.07011686143572, 65.72919999999999)(42.1702838063439, 65.12862)(42.27045075125209, 64.60615)(42.370617696160274, 65.55098)(42.47078464106845, 66.34566000000001)(42.57095158597663, 65.21101)(42.67111853088481, 65.18233000000001)(42.77128547579299, 64.65132)(42.87145242070117, 65.48506)(42.97161936560935, 64.97847)(43.07178631051753, 65.96785)(43.17195325542571, 66.30476)(43.27212020033389, 65.34956)(43.37228714524207, 66.9731)(43.47245409015025, 65.05476)(43.57262103505843, 66.61726)(43.67278797996661, 65.24032)(43.77295492487479, 65.60408)(43.87312186978298, 65.1097)(43.97328881469116, 65.21529)(44.073455759599334, 66.12837)(44.173622704507515, 65.03768)(44.2737896494157, 67.25691)(44.37395659432388, 65.29707)(44.47412353923205, 65.24886000000001)(44.574290484140235, 65.67122)(44.67445742904842, 64.70931)(44.7746243739566, 66.74971000000001)(44.87479131886477, 64.32661999999999)(44.974958263772955, 64.67757)(45.075125208681136, 65.21894999999999)(45.17529215358932, 72.5694)(45.27545909849749, 72.44305)(45.375626043405674, 65.62056)(45.475792988313856, 65.70478)(45.57595993322204, 66.11739)(45.67612687813022, 64.83626)(45.7762938230384, 65.60408)(45.87646076794658, 64.66719)(45.976627712854764, 65.6761)(46.07679465776294, 64.7441)(46.17696160267112, 65.43502000000001)(46.2771285475793, 66.90169)(46.37729549248748, 66.08503)(46.47746243739566, 65.25069)(46.57762938230384, 65.55037)(46.67779632721202, 65.26106999999999)(46.7779632721202, 64.29488)(46.87813021702838, 64.91438)(46.97829716193656, 65.6755)(47.07846410684474, 65.74568000000001)(47.17863105175292, 66.201)(47.278797996661105, 68.89997)(47.378964941569286, 65.87386000000001)(47.47913188647747, 64.85822999999999)(47.57929883138564, 66.08443)(47.679465776293824, 65.22444)(47.779632721202006, 65.45699)(47.87979966611019, 67.24653)(47.97996661101836, 65.31966)(48.080133555926544, 65.1927)(48.180300500834726, 65.2043)(48.28046744574291, 65.11764000000001)(48.38063439065108, 66.25167)(48.48080133555926, 64.45296)(48.580968280467445, 65.86165)(48.68113522537563, 65.41365)(48.7813021702838, 64.97359)(48.88146911519199, 65.49605)(48.98163606010017, 65.51558)(49.081803005008354, 65.29342)(49.18196994991653, 64.6971)(49.28213689482471, 65.34834000000001)(49.38230383973289, 65.1451)(49.48247078464107, 65.30074)(49.58263772954925, 66.5843)(49.68280467445743, 65.95442)(49.78297161936561, 65.22017)(49.88313856427379, 64.87716)(49.98330550918197, 65.32576)(50.08347245409015, 65.80183)(50.18363939899833, 64.92049)(50.28380634390651, 65.8116)(50.38397328881469, 65.61079000000001)(50.484140233722876, 65.50093000000001)(50.58430717863106, 65.78535)(50.68447412353924, 65.15059)(50.784641068447414, 65.00715)(50.884808013355595, 65.39595)(50.98497495826378, 65.41304)(51.08514190317196, 65.26044999999999)(51.18530884808013, 64.76668000000001)(51.285475792988315, 65.30806)(51.3856427378965, 64.91988)(51.48580968280468, 65.67488)(51.58597662771285, 65.43074)(51.686143572621035, 65.65413000000001)(51.786310517529216, 66.09846)(51.8864774624374, 65.16951)(51.98664440734557, 67.06648)(52.086811352253754, 64.84969000000001)(52.18697829716194, 65.24091999999999)(52.287145242070125, 65.86104)(52.3873121869783, 65.32271)(52.48747913188648, 66.25227)(52.58764607679466, 65.54366)(52.68781302170284, 66.40364)(52.78797996661102, 65.44478)(52.8881469115192, 64.86434)(52.98831385642738, 65.30074)(53.088480801335564, 65.96174)(53.18864774624374, 64.84908)(53.28881469115192, 66.30964)(53.3889816360601, 68.18831)(53.489148580968276, 64.98641)(53.58931552587646, 64.75447)(53.68948247078464, 66.61543)(53.78964941569283, 65.05171)(53.889816360601, 65.91719)(53.989983305509185, 66.1656)(54.090150250417366, 66.11617)(54.19031719532555, 65.92085)(54.29048414023372, 64.75264)(54.390651085141904, 65.58455000000001)(54.490818030050086, 64.82711)(54.59098497495827, 66.94868)(54.69115191986644, 64.79781)(54.791318864774624, 64.94246000000001)(54.891485809682806, 65.53573)(54.99165275459099, 66.12105)(55.09181969949916, 68.15778)(55.19198664440734, 65.94588)(55.292153589315525, 65.68890999999999)(55.39232053422371, 65.46003)(55.49248747913189, 65.02425)(55.59265442404007, 66.12654)(55.69282136894825, 65.82258999999999)(55.79298831385643, 64.79293)(55.89315525876461, 66.89619)(55.99332220367279, 65.88728)(56.09348914858097, 65.34651)(56.19365609348915, 64.95101)(56.29382303839733, 66.04903)(56.39398998330551, 64.87105)(56.49415692821369, 64.93514)(56.59432387312187, 65.28609)(56.69449081803005, 66.72834)(56.79465776293823, 65.22689)(56.89482470784641, 65.48994)(56.99499165275459, 65.5638)(57.095158597662774, 66.68867)(57.195325542570956, 65.46369999999999)(57.29549248747914, 66.58369)(57.39565943238732, 65.20735)(57.495826377295494, 66.00019999999999)(57.595993322203675, 65.62299999999999)(57.69616026711186, 66.05879)(57.79632721202004, 66.32735)(57.89649415692821, 65.37215)(57.996661101836395, 65.99348)(58.09682804674458, 68.37812)(58.19699499165276, 66.40303)(58.29716193656093, 68.07234)(58.397328881469114, 65.65779)(58.497495826377296, 65.70235)(58.59766277128548, 65.18233000000001)(58.69782971619365, 66.3304)(58.79799666110184, 65.41487000000001)(58.89816360601002, 64.68794)(58.9983305509182, 64.82466)(59.09849749582638, 65.69929)(59.19866444073456, 65.64192)(59.29883138564274, 65.15303)(59.398998330550924, 66.23091)(59.4991652754591, 65.06452999999999)(59.59933222036728, 65.68769999999999)(59.69949916527546, 65.80671)(59.79966611018364, 64.3138)(59.89983305509182, 65.98188999999999)(60.0, 66.53059)
        };
        \addplot[color=blue, mark=none,name path=B] coordinates { %% MIN value
        (0.0, 0.0)(0.1001669449081803, 54.10142)(0.2003338898163606, 59.94491)(0.3005008347245409, 55.776830000000004)(0.4006677796327212, 55.622420000000005)(0.5008347245409015, 56.45737)(0.6010016694490818, 56.42381)(0.7011686143572621, 55.197010000000006)(0.8013355592654424, 57.03538)(0.9015025041736228, 57.2905)(1.001669449081803, 57.02134)(1.1018363939899833, 56.03868)(1.2020033388981637, 56.13571999999999)(1.3021702838063438, 56.39512)(1.4023372287145242, 56.62705)(1.5025041736227045, 56.64597)(1.6026711185308848, 56.96885)(1.7028380634390652, 57.189189999999996)(1.8030050083472455, 56.89073)(1.9031719532554257, 56.14487)(2.003338898163606, 57.076269999999994)(2.1035058430717863, 56.14488)(2.2036727879799667, 57.30637)(2.303839732888147, 56.754619999999996)(2.4040066777963274, 55.47288)(2.5041736227045073, 56.259620000000005)(2.6043405676126876, 57.06102)(2.704507512520868, 55.73106)(2.8046744574290483, 56.69663)(2.9048414023372287, 56.15647)(3.005008347245409, 56.505590000000005)(3.1051752921535893, 55.70359)(3.2053422370617697, 45.392340000000004)(3.30550918196995, 56.75279)(3.4056761268781304, 56.65757)(3.5058430717863107, 56.80466)(3.606010016694491, 56.75339)(3.7061769616026714, 55.32274)(3.8063439065108513, 55.488749999999996)(3.906510851419032, 57.67869)(4.006677796327212, 57.173930000000006)(4.106844741235393, 55.25742)(4.207011686143573, 57.32651)(4.3071786310517535, 57.769009999999994)(4.407345575959933, 56.89622)(4.507512520868114, 56.274879999999996)(4.607679465776294, 56.603849999999994)(4.707846410684475, 57.011570000000006)(4.808013355592655, 56.56358)(4.908180300500835, 56.35727)(5.0083472454090145, 57.13121)(5.108514190317195, 57.266690000000004)(5.208681135225375, 55.7512)(5.308848080133556, 56.18272)(5.409015025041736, 57.26914)(5.509181969949917, 56.22117)(5.609348914858097, 56.431740000000005)(5.709515859766277, 56.53733)(5.809682804674457, 56.5709)(5.909849749582638, 57.02439)(6.010016694490818, 56.75584)(6.110183639398999, 56.875460000000004)(6.210350584307179, 56.61485)(6.3105175292153595, 57.24412)(6.410684474123539, 57.45224999999999)(6.510851419031719, 57.543800000000005)(6.6110183639399, 56.45188)(6.71118530884808, 57.04759)(6.811352253756261, 56.559309999999996)(6.911519198664441, 56.40244)(7.011686143572621, 56.97068)(7.111853088480801, 57.30332)(7.212020033388982, 57.028059999999996)(7.312186978297162, 57.0842)(7.412353923205343, 57.562110000000004)(7.512520868113523, 57.360699999999994)(7.612687813021703, 57.17026)(7.712854757929884, 56.01609)(7.813021702838064, 56.46226)(7.913188647746244, 56.78269)(8.013355592654424, 57.03843)(8.113522537562606, 57.130590000000005)(8.213689482470786, 56.1345)(8.313856427378965, 56.65818)(8.414023372287145, 55.82566)(8.514190317195327, 57.20078)(8.614357262103507, 57.092749999999995)(8.714524207011687, 56.79307)(8.814691151919867, 56.57761)(8.914858096828048, 57.12876)(9.015025041736228, 57.81785)(9.115191986644408, 57.20689)(9.215358931552588, 56.333479999999994)(9.31552587646077, 56.55259)(9.41569282136895, 57.1135)(9.51585976627713, 56.515969999999996)(9.61602671118531, 56.49583)(9.71619365609349, 57.3082)(9.81636060100167, 57.63474000000001)(9.916527545909851, 57.03049)(10.016694490818029, 55.695660000000004)(10.11686143572621, 56.96946)(10.21702838063439, 58.26096)(10.31719532554257, 56.887679999999996)(10.41736227045075, 56.99509)(10.51752921535893, 57.20444)(10.617696160267112, 58.16269)(10.717863105175292, 56.715559999999996)(10.818030050083472, 57.28928)(10.918196994991652, 57.31004)(11.018363939899833, 57.713469999999994)(11.118530884808013, 57.40098)(11.218697829716193, 58.001560000000005)(11.318864774624373, 58.18894)(11.419031719532555, 57.5261)(11.519198664440735, 57.2551)(11.619365609348915, 57.399750000000004)(11.719532554257095, 57.51694)(11.819699499165276, 56.38901)(11.919866444073456, 57.91245)(12.020033388981636, 57.24534)(12.120200333889816, 57.19529)(12.220367278797998, 57.78672)(12.320534223706177, 57.131809999999994)(12.420701168614357, 57.369240000000005)(12.520868113522537, 57.57675)(12.621035058430719, 58.36045)(12.721202003338899, 57.87339)(12.821368948247079, 57.16905)(12.921535893155259, 57.78306)(13.021702838063439, 57.026219999999995)(13.12186978297162, 57.387550000000005)(13.2220367278798, 57.97288)(13.32220367278798, 56.64597)(13.42237061769616, 57.93809)(13.522537562604342, 57.83248999999999)(13.622704507512521, 57.49008)(13.722871452420701, 57.51634)(13.823038397328881, 57.14095999999999)(13.923205342237063, 57.392430000000004)(14.023372287145243, 57.46262)(14.123539232053423, 57.08054)(14.223706176961603, 56.98228)(14.323873121869784, 57.640840000000004)(14.424040066777964, 58.31895)(14.524207011686144, 57.41867)(14.624373956594324, 58.182219999999994)(14.724540901502506, 57.01523)(14.824707846410686, 57.628029999999995)(14.924874791318866, 58.0864)(15.025041736227045, 57.071389999999994)(15.125208681135225, 55.793929999999996)(15.225375626043405, 57.54929)(15.325542570951589, 55.63157)(15.425709515859769, 58.17429)(15.525876460767948, 57.014630000000004)(15.626043405676128, 57.12632)(15.726210350584308, 56.41648)(15.826377295492488, 56.450050000000005)(15.926544240400668, 57.081160000000004)(16.026711185308848, 57.47971)(16.126878130217026, 56.97923)(16.22704507512521, 56.438449999999996)(16.32721202003339, 57.20078)(16.42737896494157, 57.543800000000005)(16.52754590984975, 57.47605)(16.62771285475793, 57.6677)(16.72787979966611, 57.454679999999996)(16.82804674457429, 57.131809999999994)(16.92821368948247, 56.4763)(17.028380634390654, 56.571509999999996)(17.128547579298832, 58.00705)(17.228714524207014, 56.526959999999995)(17.328881469115192, 56.67344)(17.429048414023374, 57.14219)(17.529215358931552, 58.02537)(17.629382303839733, 57.86302)(17.72954924874791, 57.51023)(17.829716193656097, 57.61582)(17.929883138564275, 57.075050000000005)(18.030050083472457, 58.13645)(18.130217028380635, 58.111430000000006)(18.230383973288816, 56.84616)(18.330550918196995, 56.68748)(18.430717863105176, 56.004490000000004)(18.530884808013354, 57.667080000000006)(18.63105175292154, 57.83615)(18.731218697829718, 58.16941)(18.8313856427379, 56.79429)(18.931552587646078, 57.343599999999995)(19.03171953255426, 58.27256)(19.131886477462437, 56.59348)(19.23205342237062, 57.98386)(19.332220367278797, 56.327369999999995)(19.43238731218698, 57.97288)(19.53255425709516, 57.69883)(19.63272120200334, 58.38058)(19.73288814691152, 58.047940000000004)(19.833055091819702, 56.807100000000005)(19.93322203672788, 56.25352)(20.033388981636058, 56.61851)(20.13355592654424, 56.86814)(20.23372287145242, 55.37461999999999)(20.333889816360603, 57.173930000000006)(20.43405676126878, 56.364599999999996)(20.534223706176963, 57.272800000000004)(20.63439065108514, 58.18101)(20.734557595993323, 57.532199999999996)(20.8347245409015, 58.45017)(20.934891485809683, 57.700050000000005)(21.03505843071786, 58.571020000000004)(21.135225375626046, 58.26462)(21.235392320534224, 57.494969999999995)(21.335559265442406, 56.36399)(21.435726210350584, 57.555389999999996)(21.535893155258766, 58.03207999999999)(21.636060100166944, 57.90208)(21.736227045075125, 56.54405)(21.836393989983303, 57.65304999999999)(21.93656093489149, 58.033910000000006)(22.036727879799667, 58.708349999999996)(22.13689482470785, 57.76353)(22.237061769616027, 57.72507)(22.33722871452421, 57.77879)(22.437395659432386, 57.72018)(22.537562604340568, 57.928309999999996)(22.637729549248746, 56.472030000000004)(22.737896494156928, 56.84251)(22.83806343906511, 58.381809999999994)(22.93823038397329, 57.69394)(23.03839732888147, 57.84226)(23.13856427378965, 58.30979)(23.23873121869783, 58.83896)(23.33889816360601, 57.08786)(23.43906510851419, 57.387550000000005)(23.53923205342237, 57.562110000000004)(23.639398998330552, 45.786629999999995)(23.739565943238734, 57.14951)(23.839732888146912, 57.37351)(23.939899833055094, 58.024139999999996)(24.040066777963272, 58.3464)(24.140233722871454, 57.865460000000006)(24.24040066777963, 58.10655)(24.340567612687813, 56.47813)(24.440734557595995, 57.886810000000004)(24.540901502504177, 58.22678)(24.641068447412355, 58.24082)(24.741235392320537, 57.945409999999995)(24.841402337228715, 57.77024)(24.941569282136896, 58.14498999999999)(25.041736227045075, 58.08518)(25.141903171953256, 58.51974)(25.242070116861438, 57.92587)(25.34223706176962, 58.257909999999995)(25.442404006677798, 58.016819999999996)(25.54257095158598, 56.534890000000004)(25.642737896494157, 57.82089)(25.74290484140234, 58.00584)(25.843071786310517, 56.87852)(25.9432387312187, 56.99326)(26.043405676126877, 57.227639999999994)(26.143572621035062, 58.26828)(26.24373956594324, 58.422090000000004)(26.34390651085142, 58.067479999999996)(26.4440734557596, 58.40622)(26.544240400667782, 58.120580000000004)(26.64440734557596, 58.09372)(26.744574290484138, 56.89805)(26.84474123539232, 58.29086)(26.9449081803005, 58.27377)(27.045075125208683, 57.84714)(27.14524207011686, 58.654019999999996)(27.245409015025043, 57.72568)(27.34557595993322, 58.15659)(27.445742904841403, 56.623999999999995)(27.54590984974958, 57.900850000000005)(27.646076794657763, 57.97898)(27.746243739565944, 57.37473)(27.846410684474126, 58.31589)(27.946577629382304, 58.18588)(28.046744574290486, 57.47421)(28.146911519198664, 58.04184)(28.247078464106846, 58.63266)(28.347245409015024, 58.228609999999996)(28.447412353923205, 57.94663)(28.547579298831387, 58.210300000000004)(28.64774624373957, 57.9216)(28.747913188647747, 58.26523)(28.84808013355593, 57.98142)(28.948247078464107, 57.59262)(29.04841402337229, 58.11142)(29.148580968280466, 57.213589999999996)(29.248747913188648, 57.701260000000005)(29.348914858096826, 57.61216)(29.44908180300501, 58.44956)(29.54924874791319, 57.61888)(29.64941569282137, 58.253029999999995)(29.74958263772955, 58.34335)(29.84974958263773, 58.374480000000005)(29.94991652754591, 57.889860000000006)(30.05008347245409, 57.812960000000004)(30.15025041736227, 58.76022)(30.25041736227045, 57.945409999999995)(30.35058430717863, 57.308820000000004)(30.45075125208681, 57.1782)(30.55091819699499, 57.08909)(30.651085141903177, 57.90024)(30.751252086811355, 58.86947)(30.851419031719537, 57.61338000000001)(30.951585976627715, 58.44894)(31.051752921535897, 57.66831)(31.151919866444075, 57.84226)(31.252086811352257, 58.48984)(31.352253756260435, 58.50815)(31.452420701168617, 58.18772)(31.552587646076795, 57.68784)(31.652754590984976, 57.37046)(31.752921535893154, 57.10801)(31.853088480801336, 58.25302)(31.953255425709514, 57.64207)(32.053422370617696, 58.35678)(32.15358931552588, 58.162079999999996)(32.25375626043405, 58.3458)(32.35392320534224, 58.74619)(32.45409015025042, 58.046119999999995)(32.554257095158604, 57.63168)(32.65442404006678, 57.638400000000004)(32.75459098497496, 58.38547)(32.85475792988314, 58.25058)(32.95492487479132, 57.85019)(33.0550918196995, 57.603)(33.15525876460768, 58.574070000000006)(33.25542570951586, 57.096410000000006)(33.35559265442404, 58.73276)(33.45575959933222, 58.16391)(33.5559265442404, 58.60214)(33.65609348914858, 57.82517)(33.756260434056756, 57.9094)(33.85642737896494, 57.716519999999996)(33.95659432387313, 57.60911)(34.05676126878131, 57.34299)(34.15692821368948, 58.60825)(34.257095158597664, 57.49375)(34.357262103505846, 57.73484)(34.45742904841403, 57.23313)(34.5575959933222, 57.70677)(34.657762938230384, 57.9503)(34.757929883138566, 57.78184)(34.85809682804675, 57.97165)(34.95826377295492, 56.86142)(35.058430717863104, 58.113859999999995)(35.158597662771285, 58.14744)(35.25876460767947, 57.1544)(35.35893155258764, 58.40318)(35.45909849749582, 58.03696000000001)(35.559265442404005, 57.759859999999996)(35.659432387312194, 58.63571)(35.75959933222037, 57.62802)(35.85976627712855, 59.440160000000006)(35.95993322203673, 57.07078)(36.06010016694491, 57.80259)(36.16026711185309, 56.5355)(36.26043405676127, 57.74521)(36.36060100166945, 58.45078)(36.46076794657763, 56.82176)(36.56093489148581, 58.28293000000001)(36.66110183639399, 57.916109999999996)(36.76126878130217, 58.53195)(36.86143572621035, 57.9802)(36.96160267111853, 58.347629999999995)(37.06176961602671, 57.74888)(37.16193656093489, 57.428439999999995)(37.26210350584308, 57.44798)(37.362270450751254, 57.58164)(37.462437395659435, 57.98873999999999)(37.56260434056762, 58.48679)(37.6627712854758, 57.69455)(37.76293823038397, 58.09311)(37.863105175292155, 58.080290000000005)(37.96327212020034, 57.74276999999999)(38.06343906510852, 57.97898)(38.16360601001669, 57.92466)(38.263772954924875, 57.524879999999996)(38.363939899833056, 57.51817)(38.46410684474124, 57.32895)(38.56427378964941, 57.548069999999996)(38.664440734557594, 58.37021)(38.764607679465776, 58.58994)(38.86477462437396, 58.231049999999996)(38.96494156928214, 57.42356)(39.06510851419032, 56.855320000000006)(39.1652754590985, 58.660740000000004)(39.26544240400668, 58.515480000000004)(39.36560934891486, 58.33848)(39.46577629382304, 57.368629999999996)(39.56594323873122, 57.58775)(39.666110183639404, 58.06564)(39.76627712854758, 57.92465)(39.86644407345576, 58.23227)(39.96661101836394, 58.62228)(40.066777963272116, 57.89109)(40.1669449081803, 58.11936)(40.26711185308848, 57.587140000000005)(40.36727879799666, 57.69089)(40.46744574290484, 57.94785)(40.567612687813025, 58.67538)(40.667779632721206, 58.66868)(40.76794657762939, 57.93076)(40.86811352253756, 57.74705)(40.968280467445744, 57.93748)(41.068447412353926, 57.9802)(41.16861435726211, 57.55845)(41.26878130217028, 58.261570000000006)(41.368948247078464, 57.86362)(41.469115191986646, 57.607879999999994)(41.56928213689483, 57.711639999999996)(41.669449081803, 57.6854)(41.769616026711184, 58.40622)(41.869782971619365, 58.14194)(41.96994991652755, 57.52854)(42.07011686143572, 57.985079999999996)(42.1702838063439, 58.784029999999994)(42.27045075125209, 57.47483)(42.370617696160274, 57.94968)(42.47078464106845, 58.699799999999996)(42.57095158597663, 58.722390000000004)(42.67111853088481, 58.202369999999995)(42.77128547579299, 58.64426)(42.87145242070117, 58.41659)(42.97161936560935, 58.22983)(43.07178631051753, 57.5145)(43.17195325542571, 58.627770000000005)(43.27212020033389, 57.65793000000001)(43.37228714524207, 58.257909999999995)(43.47245409015025, 57.186130000000006)(43.57262103505843, 58.462990000000005)(43.67278797996661, 57.89597)(43.77295492487479, 58.55576)(43.87312186978298, 58.415369999999996)(43.97328881469116, 57.52548)(44.073455759599334, 57.8624)(44.173622704507515, 57.116550000000004)(44.2737896494157, 57.47483)(44.37395659432388, 57.62985)(44.47412353923205, 56.72349)(44.574290484140235, 57.63962)(44.67445742904842, 57.936249999999994)(44.7746243739566, 58.80417)(44.87479131886477, 58.23532)(44.974958263772955, 57.62802)(45.075125208681136, 58.281099999999995)(45.17529215358932, 58.43369)(45.27545909849749, 58.297579999999996)(45.375626043405674, 57.651219999999995)(45.475792988313856, 58.08945)(45.57595993322204, 57.60483)(45.67612687813022, 57.76353)(45.7762938230384, 58.11203)(45.87646076794658, 58.08274)(45.976627712854764, 58.07602)(46.07679465776294, 58.43674)(46.17696160267112, 58.19443)(46.2771285475793, 57.78366)(46.37729549248748, 58.55454)(46.47746243739566, 58.24998)(46.57762938230384, 58.173069999999996)(46.67779632721202, 59.00924)(46.7779632721202, 57.196509999999996)(46.87813021702838, 57.95151)(46.97829716193656, 58.75717)(47.07846410684474, 57.86606)(47.17863105175292, 58.0272)(47.278797996661105, 58.4752)(47.378964941569286, 58.39829)(47.47913188647747, 58.441010000000006)(47.57929883138564, 57.900850000000005)(47.679465776293824, 58.113859999999995)(47.779632721202006, 59.20517)(47.87979966611019, 57.90025)(47.97996661101836, 58.56492)(48.080133555926544, 58.23532)(48.180300500834726, 56.73447)(48.28046744574291, 58.20297)(48.38063439065108, 57.93565)(48.48080133555926, 58.43369)(48.580968280467445, 58.5405)(48.68113522537563, 58.492290000000004)(48.7813021702838, 58.5173)(48.88146911519199, 57.442479999999996)(48.98163606010017, 58.7114)(49.081803005008354, 58.063810000000004)(49.18196994991653, 58.015600000000006)(49.28213689482471, 58.26707)(49.38230383973289, 58.87314)(49.48247078464107, 58.709559999999996)(49.58263772954925, 58.16696999999999)(49.68280467445743, 59.44321000000001)(49.78297161936561, 56.68016)(49.88313856427379, 58.11875)(49.98330550918197, 59.05869)(50.08347245409015, 58.60703)(50.18363939899833, 57.50718)(50.28380634390651, 57.86363)(50.38397328881469, 58.46665)(50.484140233722876, 57.97287)(50.58430717863106, 57.60544)(50.68447412353924, 58.098)(50.784641068447414, 58.04307)(50.884808013355595, 58.00767)(50.98497495826378, 57.71226)(51.08514190317196, 57.77756)(51.18530884808013, 58.24204)(51.285475792988315, 57.681740000000005)(51.3856427378965, 58.343360000000004)(51.48580968280468, 58.384859999999996)(51.58597662771285, 57.9216)(51.686143572621035, 58.938449999999996)(51.786310517529216, 58.22495)(51.8864774624374, 58.00828)(51.98664440734557, 58.57895)(52.086811352253754, 57.93564)(52.18697829716194, 59.130700000000004)(52.287145242070125, 58.87069)(52.3873121869783, 56.90659)(52.48747913188648, 57.67563)(52.58764607679466, 57.922219999999996)(52.68781302170284, 59.52255)(52.78797996661102, 58.08823)(52.8881469115192, 58.73642)(52.98831385642738, 58.61252)(53.088480801335564, 58.4929)(53.18864774624374, 58.23532)(53.28881469115192, 57.21848)(53.3889816360601, 58.01865)(53.489148580968276, 58.9537)(53.58931552587646, 56.3054)(53.68948247078464, 58.68088)(53.78964941569283, 57.7855)(53.889816360601, 58.493500000000004)(53.989983305509185, 57.922830000000005)(54.090150250417366, 58.77792)(54.19031719532555, 58.19626)(54.29048414023372, 58.11813)(54.390651085141904, 58.248749999999994)(54.490818030050086, 57.71165)(54.59098497495827, 59.04038)(54.69115191986644, 57.01218)(54.791318864774624, 58.67783)(54.891485809682806, 58.57285)(54.99165275459099, 58.01804)(55.09181969949916, 59.18869)(55.19198664440734, 57.31675)(55.292153589315525, 58.805389999999996)(55.39232053422371, 58.49655)(55.49248747913189, 57.34971)(55.59265442404007, 58.58323)(55.69282136894825, 59.33823)(55.79298831385643, 58.05588)(55.89315525876461, 58.994600000000005)(55.99332220367279, 58.07297)(56.09348914858097, 57.480320000000006)(56.19365609348915, 56.963359999999994)(56.29382303839733, 58.20725)(56.39398998330551, 58.80051)(56.49415692821369, 58.2512)(56.59432387312187, 58.72483)(56.69449081803005, 57.32468)(56.79465776293823, 59.384)(56.89482470784641, 58.768159999999995)(56.99499165275459, 58.18345)(57.095158597662774, 58.397059999999996)(57.195325542570956, 59.1655)(57.29549248747914, 57.62742)(57.39565943238732, 58.35373)(57.495826377295494, 58.36106)(57.595993322203675, 58.17429)(57.69616026711186, 59.034890000000004)(57.79632721202004, 57.941739999999996)(57.89649415692821, 58.275)(57.996661101836395, 57.227019999999996)(58.09682804674458, 58.76755)(58.19699499165276, 56.53184)(58.29716193656093, 57.978370000000005)(58.397328881469114, 58.08518)(58.497495826377296, 57.91245)(58.59766277128548, 58.69187)(58.69782971619365, 57.745819999999995)(58.79799666110184, 58.24387)(58.89816360601002, 58.024139999999996)(58.9983305509182, 58.21396)(59.09849749582638, 59.01962)(59.19866444073456, 58.34153)(59.29883138564274, 58.536229999999996)(59.398998330550924, 58.37693)(59.4991652754591, 57.90329)(59.59933222036728, 58.00217)(59.69949916527546, 58.85178)(59.79966611018364, 58.32199)(59.89983305509182, 59.00436)(60.0, 57.72873)
        };
        \addplot [pattern=north east lines,pattern color=red] 
        fill between [
            of=A and B,soft clip={domain=0:800},
        ];
        \end{axis}
\end{tikzpicture}
\caption{Measuring instrument: RAPL}\label{fig:time_series_BinaryTrees_Workstation_RAPL}
\end{subfigure}
\begin{subfigure}[b]{0.49\linewidth}
    \begin{tikzpicture}
        \pgfplotsset{%
        width=1\linewidth,
        % height=1\textheight
        }
        \begin{axis}[ymax=120,
        xlabel={Time (Seconds)},
        ylabel={Energy Consumption (Joules)},
        ]
        \addplot[color=blue, mark=none,] coordinates { %% AVG value
        (0.0, 105.90970157122501)(0.07585335018963338, 106.13518260043335)(0.15170670037926676, 105.42104600893335)(0.22756005056890014, 104.99850978976656)(0.3034134007585335, 105.01533609652498)(0.3792667509481669, 104.90442054541664)(0.45512010113780027, 105.42552585189165)(0.5309734513274336, 105.80192312599999)(0.606826801517067, 106.12116505491672)(0.6826801517067005, 105.97326023024165)(0.7585335018963338, 105.22996055239996)(0.8343868520859672, 105.87733483760832)(0.9102402022756005, 106.44192809479166)(0.986093552465234, 105.6819366578167)(1.0619469026548671, 106.60087718019169)(1.1378002528445006, 106.6681430741167)(1.213653603034134, 106.80788514994168)(1.2895069532237675, 107.64868045830002)(1.365360303413401, 107.27915775404999)(1.4412136536030342, 107.29767519581671)(1.5170670037926677, 107.51444463219167)(1.5929203539823011, 107.77385983749164)(1.6687737041719344, 107.61589943202497)(1.7446270543615676, 107.44011603124163)(1.820480404551201, 108.11543984068327)(1.8963337547408345, 108.5262022284167)(1.972187104930468, 108.06857168524999)(2.0480404551201015, 107.8258774767333)(2.1238938053097343, 108.12237790655828)(2.199747155499368, 108.76120541940831)(2.275600505689001, 109.5069558094917)(2.351453855878635, 108.72394447894995)(2.427307206068268, 108.03771647015832)(2.503160556257902, 107.59908417589999)(2.579013906447535, 106.74109854129999)(2.6548672566371687, 107.45415855640003)(2.730720606826802, 107.10668010631669)(2.806573957016435, 107.18152682793335)(2.8824273072060684, 107.73020993129163)(2.9582806573957017, 107.43255117720834)(3.0341340075853354, 107.24315850070829)(3.1099873577749686, 109.37798840020832)(3.1858407079646023, 108.70372670412503)(3.2616940581542355, 108.23044351752492)(3.3375474083438688, 108.3238918323667)(3.413400758533502, 108.68680939130834)(3.4892541087231352, 107.66846525293332)(3.565107458912769, 108.65897885254167)(3.640960809102402, 108.46696125545)(3.716814159292036, 108.29646774097496)(3.792667509481669, 107.57662736730835)(3.8685208596713023, 108.41091098153336)(3.944374209860936, 107.79271274535832)(4.020227560050569, 108.66548501284166)(4.096080910240203, 108.94682532143335)(4.171934260429836, 108.9355819269833)(4.2477876106194685, 108.34028392392499)(4.323640960809103, 108.49137839110833)(4.399494310998736, 109.06146652830007)(4.47534766118837, 108.95626436374167)(4.551201011378002, 108.81343180927502)(4.6270543615676365, 108.90753349334165)(4.70290771175727, 108.97293380871666)(4.778761061946904, 109.05775619025836)(4.854614412136536, 109.30477573806664)(4.9304677623261695, 109.18442362573334)(5.006321112515804, 109.54569335783327)(5.082174462705436, 109.52918811613334)(5.15802781289507, 109.50728742411668)(5.233881163084703, 109.57195999220833)(5.309734513274337, 109.55034482368335)(5.38558786346397, 109.48054490740006)(5.461441213653604, 109.13499694399167)(5.537294563843237, 108.77146555984999)(5.61314791403287, 109.37385300078336)(5.689001264222504, 109.11497226649163)(5.764854614412137, 109.247869497175)(5.840707964601771, 109.5516030438)(5.916561314791403, 108.95779626350837)(5.9924146649810375, 108.92360045872505)(6.068268015170671, 109.64710437745)(6.144121365360304, 108.40417614559166)(6.219974715549937, 105.37162959475835)(6.29582806573957, 105.02466040384998)(6.371681415929205, 104.42461810582505)(6.447534766118837, 104.73436226495832)(6.523388116308471, 105.62928438114173)(6.599241466498104, 107.58405006039999)(6.6750948166877375, 107.48090279148332)(6.750948166877371, 109.62592956736668)(6.826801517067004, 109.01533799939999)(6.902654867256638, 108.19895585695829)(6.9785082174462705, 107.35952209793336)(7.054361567635905, 105.65206003699998)(7.130214917825538, 106.68908678567503)(7.206068268015172, 106.55703700469167)(7.281921618204804, 106.7143756004)(7.357774968394438, 107.48341234571663)(7.433628318584072, 107.7732190245917)(7.509481668773706, 106.02193179970827)(7.585335018963338, 104.08191982259171)(7.661188369152971, 105.11517239387506)(7.737041719342605, 103.95522009075832)(7.812895069532239, 104.95923665651667)(7.888748419721872, 104.3678481177083)(7.964601769911505, 105.00929246863339)(8.040455120101138, 104.42107520783333)(8.116308470290772, 104.10315681775)(8.192161820480406, 104.13251213636669)(8.268015170670038, 105.37888431101668)(8.343868520859672, 105.37826881189167)(8.419721871049305, 105.98603673764168)(8.495575221238937, 105.83863261529169)(8.571428571428573, 106.42743445225831)(8.647281921618205, 106.43599072875834)(8.72313527180784, 106.22269689433331)(8.798988621997472, 106.584643971675)(8.874841972187106, 106.10500038098333)(8.95069532237674, 106.67691577412504)(9.026548672566372, 107.00484878107497)(9.102402022756005, 107.59459156364164)(9.178255372945639, 107.16285712700834)(9.254108723135273, 106.58040411476667)(9.329962073324905, 106.9659894440334)(9.40581542351454, 106.52437775938331)(9.481668773704172, 106.44630994537503)(9.557522123893808, 107.12541439264162)(9.63337547408344, 106.85177145186667)(9.709228824273072, 107.07763062299999)(9.785082174462707, 106.94323934128334)(9.860935524652339, 107.61031375179999)(9.936788874841973, 106.78679814458327)(10.012642225031607, 106.19473072685838)(10.08849557522124, 105.73961768865)(10.164348925410872, 105.8799027809417)(10.240202275600508, 104.98935104769173)(10.31605562579014, 106.67132235982498)(10.391908975979774, 106.73977567054165)(10.467762326169407, 105.36271575460827)(10.543615676359039, 107.72133847180835)(10.619469026548675, 107.55569052672504)(10.695322376738307, 107.69489090065004)(10.77117572692794, 108.92169689030835)(10.847029077117574, 106.75200621224992)(10.922882427307208, 108.47703942524164)(10.99873577749684, 107.17081151635837)(11.074589127686474, 107.87561031905834)(11.150442477876107, 106.91056284174165)(11.22629582806574, 106.61503087706663)(11.302149178255375, 108.11325032094997)(11.378002528445007, 107.42830577645834)(11.453855878634641, 107.12927876618336)(11.529709228824274, 107.11353058876664)(11.605562579013906, 107.37830912924169)(11.681415929203542, 107.69462785856666)(11.757269279393174, 107.38827520104999)(11.833122629582807, 106.73157174718328)(11.90897597977244, 106.70695004024999)(11.984829329962075, 106.26980878635828)(12.060682680151707, 106.64221225242503)(12.136536030341341, 106.54158394640834)(12.212389380530974, 106.447359792125)(12.288242730720608, 106.12682781199996)(12.364096080910242, 106.33083846460836)(12.439949431099874, 105.66498458391668)(12.515802781289509, 106.258497460575)(12.59165613147914, 106.42552876969998)(12.667509481668775, 106.3850439551)(12.74336283185841, 105.69608726618334)(12.819216182048041, 106.15430887723333)(12.895069532237674, 107.3751821510917)(12.970922882427308, 107.25456930605833)(13.046776232616942, 106.65061818761669)(13.122629582806574, 107.33173191411662)(13.198482932996209, 106.43130689359165)(13.274336283185841, 108.57674766215838)(13.350189633375475, 107.88847253827498)(13.42604298356511, 107.34459904460832)(13.501896333754742, 108.89711448377503)(13.577749683944376, 106.39837976360835)(13.653603034134008, 107.44642607214998)(13.729456384323642, 105.10921659041658)(13.805309734513276, 104.93358877501666)(13.881163084702909, 103.06688288983325)(13.957016434892541, 102.87297316543335)(14.032869785082175, 104.32988746900834)(14.10872313527181, 103.54648952158335)(14.184576485461443, 107.35066736697499)(14.260429835651076, 109.12666788219165)(14.336283185840708, 109.39289775445839)(14.412136536030344, 108.95109414805002)(14.487989886219976, 108.31105542239999)(14.563843236409609, 107.86206926304168)(14.639696586599243, 107.54621100916671)(14.715549936788875, 107.46929131683333)(14.79140328697851, 107.85300505954166)(14.867256637168143, 107.739202357275)(14.943109987357776, 108.44726139959998)(15.018963337547412, 107.27475186321674)(15.094816687737044, 107.31974336565831)(15.170670037926676, 106.77141135073329)(15.24652338811631, 107.01664751631667)(15.322376738305943, 106.37993714214998)(15.398230088495575, 106.38900133483337)(15.47408343868521, 107.52598910910004)(15.549936788874842, 107.40163584859168)(15.625790139064478, 106.70079179774167)(15.701643489254112, 107.15442431084163)(15.777496839443744, 107.22804918918341)(15.853350189633378, 107.17787331176665)(15.92920353982301, 106.79234904689999)(16.005056890012643, 107.74034315612502)(16.080910240202275, 107.60095581570832)(16.15676359039191, 108.4708449728417)(16.232616940581543, 108.06241134807506)(16.30847029077118, 107.97290128646665)(16.38432364096081, 108.73539831852507)(16.460176991150444, 108.65713394919172)(16.536030341340076, 109.48938387429162)(16.611883691529712, 109.21754717975001)(16.687737041719345, 109.29518171724996)(16.763590391908977, 110.00834076477503)(16.83944374209861, 110.7968316056667)(16.91529709228824, 110.17447749583333)(16.991150442477874, 109.84639381376665)(17.067003792667514, 110.32360130339164)(17.142857142857146, 110.93809468600831)(17.21871049304678, 110.58140089016669)(17.29456384323641, 110.21957390544999)(17.370417193426043, 110.2990437017417)(17.44627054361568, 110.54622985908337)(17.52212389380531, 109.9865138581167)(17.597977243994944, 109.94595477034994)(17.673830594184576, 107.87936005105833)(17.749683944374212, 109.38626906459997)(17.825537294563844, 108.07396980659996)(17.90139064475348, 109.72866104788332)(17.977243994943112, 108.36116055838335)(18.053097345132745, 108.90500913684164)(18.128950695322377, 108.76232648212502)(18.20480404551201, 108.97989993108334)(18.280657395701645, 108.71393328599999)(18.356510745891278, 110.47041908064169)(18.432364096080914, 106.86206042373334)(18.508217446270546, 108.69240349342493)(18.58407079646018, 108.40186460188329)(18.65992414664981, 108.05544756295004)(18.735777496839447, 107.6232616843917)(18.81163084702908, 107.9714161547917)(18.88748419721871, 107.67689582380837)(18.963337547408344, 109.10912911047507)(19.039190897597976, 108.03808130325838)(19.115044247787615, 108.08659096362494)(19.190897597977248, 108.18350531991668)(19.26675094816688, 107.56319142358339)(19.342604298356513, 108.51327283503333)(19.418457648546145, 107.40626743171661)(19.494310998735777, 107.16991473722494)(19.570164348925413, 106.9563713682)(19.646017699115045, 107.03196426024166)(19.721871049304678, 106.75312330284164)(19.797724399494314, 106.95572953002504)(19.873577749683946, 106.988335061975)(19.94943109987358, 107.50108964391667)(20.025284450063214, 107.53116070089996)(20.101137800252847, 106.42057327479998)(20.17699115044248, 106.70005638846662)(20.25284450063211, 107.05628778311666)(20.328697850821744, 107.02408235236672)(20.40455120101138, 106.64326777804996)(20.480404551201016, 105.89136121435004)(20.556257901390648, 106.68160569976665)(20.63211125158028, 105.74580983932496)(20.707964601769913, 106.84997441951674)(20.78381795195955, 107.6506410142917)(20.85967130214918, 107.40306867615833)(20.935524652338813, 107.71255188836669)(21.011378002528446, 106.83249506873334)(21.087231352718078, 107.04791556097499)(21.163084702907714, 105.37830542431666)(21.23893805309735, 104.58202839370828)(21.314791403286982, 104.59934025419169)(21.390644753476614, 103.77710992450835)(21.466498103666247, 102.89221581414168)(21.54235145385588, 103.13488675486671)(21.61820480404551, 104.83303969226664)(21.694058154235147, 103.51025506193336)(21.76991150442478, 106.20069823003337)(21.845764854614416, 106.48832063440001)(21.921618204804048, 104.73113464165831)(21.99747155499368, 105.03798285722502)(22.073324905183316, 104.88716509117499)(22.14917825537295, 103.53702688467499)(22.22503160556258, 103.42320325520004)(22.300884955752213, 103.41846615032499)(22.376738305941846, 103.84551100523332)(22.45259165613148, 104.77420190673334)(22.528445006321114, 108.65621065565003)(22.60429835651075, 108.08081668131668)(22.680151706700382, 107.59130560772495)(22.756005056890015, 107.90103462268334)(22.831858407079647, 108.59460283225006)(22.907711757269283, 108.7016650129333)(22.983565107458915, 108.20553656668334)(23.059418457648547, 109.30265152868338)(23.13527180783818, 109.00343003625001)(23.211125158027812, 108.09272939864164)(23.286978508217448, 108.36764232045006)(23.362831858407084, 108.88949100389165)(23.438685208596716, 108.20706174263334)(23.51453855878635, 109.3532549598)(23.59039190897598, 109.15978303704998)(23.666245259165613, 108.76777989983333)(23.74209860935525, 109.13573172615004)(23.81795195954488, 109.39174583976673)(23.893805309734514, 110.33114593535834)(23.96965865992415, 110.69414156480838)(24.045512010113782, 110.25380496281667)(24.121365360303415, 109.42518972370002)(24.19721871049305, 109.52762254710832)(24.273072060682683, 109.64552810655833)(24.348925410872315, 110.18758238459996)(24.424778761061948, 111.00702890582502)(24.50063211125158, 110.9520615587)(24.576485461441216, 110.6548442710833)(24.65233881163085, 110.22051831234172)(24.728192161820484, 109.97834295124167)(24.804045512010116, 110.28449663184168)(24.87989886219975, 110.61003255989165)(24.95575221238938, 110.36448302560832)(25.031605562579017, 109.79665728536663)(25.10745891276865, 110.08247865285836)(25.18331226295828, 111.26848019362504)(25.259165613147914, 108.14579223054166)(25.33501896333755, 109.95186281810001)(25.410872313527186, 110.28969102228331)(25.48672566371682, 109.69873424609999)(25.56257901390645, 108.06899173272501)(25.638432364096083, 109.97859493724158)(25.714285714285715, 107.54962658063333)(25.790139064475348, 109.33565255381669)(25.865992414664984, 109.21243079440832)(25.941845764854616, 108.44605272513334)(26.017699115044252, 109.34498520720832)(26.093552465233884, 108.72385916874994)(26.169405815423517, 109.16799497565835)(26.24525916561315, 109.53768181739997)(26.321112515802785, 109.04183727122496)(26.396965865992417, 108.87291240117499)(26.47281921618205, 108.40354138394167)(26.548672566371682, 108.87685140914164)(26.624525916561314, 108.93742150035828)(26.70037926675095, 109.24324761170834)(26.776232616940586, 108.79854369375832)(26.85208596713022, 109.17732532854998)(26.92793931731985, 109.59276486057504)(27.003792667509483, 109.37022244581664)(27.07964601769912, 108.87821146717496)(27.15549936788875, 108.39687310777505)(27.231352718078384, 108.36269259174169)(27.307206068268016, 108.22896515074167)(27.38305941845765, 108.43481151933332)(27.458912768647284, 108.47289733001666)(27.53476611883692, 109.34342698400836)(27.610619469026553, 108.64142267815002)(27.686472819216185, 108.43235392310001)(27.762326169405817, 108.27146536167503)(27.83817951959545, 108.01261698812499)(27.914032869785082, 108.71428156543328)(27.989886219974718, 108.9714390933834)(28.06573957016435, 108.95497378440832)(28.141592920353986, 108.65372655769998)(28.21744627054362, 109.39322983086667)(28.29329962073325, 108.86815837516669)(28.369152970922887, 109.68974918951663)(28.44500632111252, 110.14977278143336)(28.52085967130215, 109.7178731645333)(28.596713021491784, 111.07410208718338)(28.672566371681416, 110.59799269535004)(28.748419721871052, 109.81146304015836)(28.824273072060688, 109.8906977403583)(28.90012642225032, 109.61210437085002)(28.975979772439953, 110.0789889236499)(29.051833122629585, 110.6338895976333)(29.127686472819217, 111.0807505232083)(29.203539823008853, 113.45488502893338)(29.279393173198486, 114.3731799339333)(29.355246523388118, 113.76355167039996)(29.43109987357775, 114.02422591137497)(29.506953223767386, 113.47694765002504)(29.58280657395702, 112.1923921920334)(29.658659924146654, 112.57710995609166)(29.734513274336287, 112.48450162160832)(29.81036662452592, 112.05088785369995)(29.88621997471555, 112.41542627837497)(29.962073324905184, 111.97231609617499)(30.037926675094823, 105.04908751857498)(30.113780025284452, 105.29455610798333)(30.189633375474088, 105.25939408064166)(30.265486725663717, 105.50061578405841)(30.341340075853353, 105.551107209725)(30.417193426042985, 105.47879205454996)(30.49304677623262, 105.40902780283336)(30.568900126422257, 105.62231081393337)(30.644753476611886, 106.016763368375)(30.72060682680152, 106.19317603964996)(30.79646017699115, 105.72377930644166)(30.872313527180786, 106.28485910069165)(30.94816687737042, 107.0432476886)(31.024020227560055, 107.0254712761417)(31.099873577749683, 107.15036453870829)(31.17572692793932, 106.99643532331667)(31.251580278128955, 107.68564056383327)(31.327433628318587, 107.5588225879584)(31.403286978508223, 107.84637202220004)(31.479140328697852, 107.12153431983333)(31.554993678887488, 107.89878378916664)(31.630847029077117, 107.83188989628334)(31.706700379266756, 108.05359758235832)(31.782553729456385, 108.89011971000004)(31.85840707964602, 109.54235667906671)(31.934260429835657, 109.08480737424166)(32.010113780025286, 108.77163937285)(32.085967130214925, 108.52677504340826)(32.16182048040455, 108.81441275544168)(32.23767383059419, 108.83646136942501)(32.31352718078382, 108.50413380663339)(32.389380530973455, 108.41767541701665)(32.46523388116309, 108.53454071349168)(32.54108723135272, 107.41865144775002)(32.61694058154236, 107.50357601575001)(32.692793931731984, 106.54276544855828)(32.76864728192162, 107.40511241628333)(32.844500632111256, 107.31852026819163)(32.92035398230089, 106.08578536312504)(32.99620733249052, 107.07569153200835)(33.07206068268015, 105.67263523143326)(33.147914032869785, 107.64209991337496)(33.223767383059425, 106.58497906996668)(33.29962073324906, 105.208346807125)(33.37547408343869, 106.52177067003339)(33.45132743362832, 105.05766113529168)(33.527180783817954, 105.64043386150836)(33.603034134007586, 106.07893851375836)(33.67888748419722, 105.35497919644166)(33.75474083438686, 105.39264203004998)(33.83059418457648, 105.99705017598336)(33.90644753476612, 105.81834406484168)(33.98230088495575, 106.34819664140838)(34.05815423514539, 105.57305808964162)(34.13400758533503, 105.31124477514165)(34.20986093552465, 105.96334925044167)(34.28571428571429, 105.9865669409167)(34.36156763590392, 105.1071753471083)(34.43742098609356, 106.02206771560824)(34.51327433628319, 105.77026309702502)(34.58912768647282, 105.63944414507499)(34.664981036662454, 106.32152053209997)(34.740834386852086, 105.38192830667496)(34.816687737041725, 104.96679311026666)(34.89254108723136, 104.812663058825)(34.96839443742099, 104.97112677624996)(35.04424778761062, 106.4265160166584)(35.120101137800255, 105.40150052401671)(35.19595448798989, 104.7673391375)(35.27180783817953, 105.53493862314167)(35.34766118836915, 106.1045579050833)(35.42351453855879, 104.79387588532505)(35.499367888748424, 106.00400881518333)(35.575221238938056, 105.88542777903336)(35.65107458912769, 105.46468795727499)(35.72692793931732, 105.3507458509167)(35.80278128950696, 106.12291393335836)(35.878634639696585, 106.01407986456664)(35.954487989886225, 105.0889655305)(36.03034134007585, 105.96907523302502)(36.10619469026549, 104.98822620493337)(36.18204804045513, 104.48405736209172)(36.257901390644754, 104.17993737431667)(36.333754740834394, 103.39856000070833)(36.40960809102402, 104.58839486788331)(36.48546144121366, 104.1173697152333)(36.56131479140329, 105.56941947950001)(36.63716814159292, 106.12687743390825)(36.713021491782555, 108.21622804892495)(36.78887484197219, 109.42303462372497)(36.86472819216183, 107.76298500264997)(36.94058154235145, 109.14809180003336)(37.01643489254109, 108.85211421846667)(37.092288242730724, 109.18366600418335)(37.16814159292036, 109.55743596959164)(37.24399494310999, 110.51499174290835)(37.31984829329962, 109.58470670585002)(37.395701643489254, 111.03764499706666)(37.47155499367889, 110.49806850667501)(37.547408343868526, 108.294990236925)(37.62326169405816, 107.92984416524997)(37.69911504424779, 107.40541920314996)(37.77496839443742, 107.55859830572503)(37.85082174462706, 107.81121615105002)(37.92667509481669, 107.12814086163333)(38.00252844500633, 107.75475327552498)(38.07838179519595, 107.84257721263332)(38.15423514538559, 108.1123140758833)(38.23008849557523, 108.32290599391666)(38.305941845764856, 106.79360777090837)(38.381795195954496, 107.83890516891667)(38.45764854614412, 109.69176615397492)(38.53350189633376, 108.83065031115001)(38.60935524652339, 108.54568877073332)(38.685208596713025, 108.60561629435003)(38.76106194690266, 107.03813172203331)(38.83691529709229, 108.48195539347503)(38.91276864728193, 109.19878685688332)(38.988621997471554, 108.032068720525)(39.064475347661194, 108.92274892872499)(39.140328697850826, 108.98529157695002)(39.21618204804046, 108.87914027986666)(39.29203539823009, 109.31052077650001)(39.36788874841972, 109.27300865178334)(39.443742098609356, 110.3547138260083)(39.519595448798995, 110.77196709745832)(39.59544879898863, 110.63563438287498)(39.67130214917826, 110.5147447707834)(39.74715549936789, 110.327468851725)(39.823008849557525, 110.99960256609171)(39.89886219974716, 110.46858168545005)(39.97471554993679, 109.95924670853337)(40.05056890012643, 108.99846644485838)(40.126422250316054, 108.79937693372509)(40.20227560050569, 108.92078507760833)(40.278128950695326, 108.50674291824996)(40.35398230088496, 107.2837380760917)(40.4298356510746, 108.01718143025826)(40.50568900126422, 105.89266627540833)(40.58154235145386, 107.62823387761668)(40.65739570164349, 106.32614289286673)(40.73324905183313, 107.77555619419168)(40.80910240202276, 108.50832112878334)(40.88495575221239, 106.37107382176669)(40.96080910240203, 106.22621390599166)(41.036662452591656, 106.92247688448334)(41.112515802781296, 106.91238271076665)(41.18836915297093, 106.57006683023333)(41.26422250316056, 106.1458553183583)(41.34007585335019, 105.97414901006664)(41.415929203539825, 106.6270202859667)(41.49178255372946, 106.66274921155001)(41.5676359039191, 107.18294272036663)(41.64348925410873, 106.23335831731667)(41.71934260429836, 106.70550872269999)(41.795195954487994, 107.02975461097503)(41.871049304677626, 106.39342562070831)(41.94690265486726, 106.18098645582498)(42.02275600505689, 106.52401976185001)(42.09860935524653, 107.0709020925583)(42.174462705436156, 107.27750628389997)(42.250316055625795, 106.83738918783327)(42.32616940581543, 105.7026622085917)(42.40202275600506, 105.72442681673331)(42.4778761061947, 106.25062664622497)(42.553729456384325, 106.99343376380831)(42.629582806573964, 106.32118522405835)(42.70543615676359, 105.61246772031667)(42.78128950695323, 105.7393386940583)(42.85714285714286, 106.12869061600829)(42.932996207332494, 106.40038911233327)(43.00884955752213, 106.58778742808337)(43.08470290771176, 106.17996547499169)(43.1605562579014, 106.49705683835835)(43.23640960809102, 107.11640480289165)(43.31226295828066, 105.99350404398328)(43.388116308470295, 106.887823554725)(43.46396965865993, 106.65714637025835)(43.53982300884956, 105.88252939976668)(43.61567635903919, 106.346835463425)(43.69152970922883, 105.01027388039167)(43.767383059418464, 104.34485258128338)(43.843236409608096, 103.69542345605005)(43.91908975979773, 103.6054256339917)(43.99494310998736, 102.9017514325083)(44.07079646017699, 104.14472148189999)(44.14664981036663, 105.33163542097503)(44.22250316055626, 106.63249541929163)(44.2983565107459, 109.12583664610828)(44.37420986093552, 109.75688463341666)(44.45006321112516, 107.51597195421662)(44.5259165613148, 108.13191905265002)(44.60176991150443, 107.68418498529171)(44.677623261694066, 107.42988704320835)(44.75347661188369, 107.50257073555831)(44.82932996207333, 107.85867906564167)(44.90518331226296, 107.39063492932502)(44.981036662452595, 108.49803788438335)(45.05689001264223, 107.74555724504995)(45.13274336283186, 108.21148739269171)(45.2085967130215, 108.66213990274997)(45.284450063211125, 107.90791066250839)(45.360303413400764, 108.46196488694999)(45.4361567635904, 108.732534794225)(45.51201011378003, 108.0383082737)(45.58786346396966, 108.72196643649168)(45.663716814159294, 109.06112982695002)(45.739570164348926, 108.24458153641662)(45.815423514538566, 108.68826753063334)(45.8912768647282, 108.74794857480829)(45.96713021491783, 109.49055789525832)(46.04298356510746, 109.58814534199168)(46.118836915297095, 109.6139654278656)(46.19469026548673, 109.49728053876473)(46.27054361567636, 109.90136238196635)(46.346396965866, 110.84931049666388)(46.422250316055624, 111.02266706405047)(46.498103666245264, 111.23794561479833)(46.573957016434896, 111.08861391921852)(46.64981036662453, 111.38501597081515)(46.72566371681417, 110.89431817505884)(46.80151706700379, 111.03457443198323)(46.87737041719343, 111.48396732605048)(46.95322376738306, 111.78995392425213)(47.0290771175727, 111.74541480080673)(47.10493046776233, 111.25942370742011)(47.18078381795196, 110.8231989634118)(47.2566371681416, 110.90780552325208)(47.33249051833123, 111.63799382099153)(47.408343868520866, 111.34438553180505)(47.4841972187105, 110.94019199827963)(47.56005056890013, 109.58775819137286)(47.63590391908976, 109.33585275848306)(47.711757269279396, 110.08557884485595)(47.78761061946903, 109.15712249288137)(47.86346396965867, 110.51694754899998)(47.9393173198483, 109.1345452602712)(48.01517067003793, 109.52614248233895)(48.091024020227565, 108.98752377744829)(48.1668773704172, 110.68505731212072)(48.24273072060683, 109.79653475021544)(48.31858407079646, 109.63521314002585)(48.3944374209861, 109.88922510317393)(48.470290771175726, 110.36192327523473)(48.546144121365366, 109.00790063532173)(48.621997471555, 109.16777767751303)(48.69785082174463, 108.85817655595646)(48.77370417193427, 109.88777558494778)(48.849557522123895, 109.44366228566665)(48.925410872313535, 110.28530455588603)(49.00126422250316, 110.32307510956136)(49.0771175726928, 110.31674356814159)(49.15297092288243, 110.59997477053982)(49.228824273072064, 109.33079274820717)(49.3046776232617, 109.14994970265768)(49.38053097345133, 109.46404810630625)(49.45638432364097, 109.32028086625228)(49.53223767383059, 109.24394975703606)(49.60809102402023, 108.89462635904505)(49.683944374209865, 109.19570169254956)(49.7597977243995, 109.8998513563513)(49.83565107458913, 109.80994049563058)(49.91150442477876, 109.93479311467567)(49.9873577749684, 110.0939961334234)(50.063211125158034, 110.21183767756362)(50.139064475347666, 109.91057262933634)(50.2149178255373, 110.00172988352729)(50.29077117572693, 109.62522522398184)(50.36662452591656, 109.02890220588077)(50.4424778761062, 109.88894946279818)(50.51833122629583, 109.5919877114496)(50.59418457648547, 109.60966805068803)(50.6700379266751, 109.78913824309176)(50.74589127686473, 109.84823314292592)(50.82174462705437, 109.88186947710282)(50.897597977244, 109.27915739756072)(50.97345132743364, 109.95245875273831)(51.04930467762326, 109.53830246924528)(51.1251580278129, 109.75622690465099)(51.20101137800253, 110.45694521732074)(51.276864728192166, 109.34871042178298)(51.352718078381805, 106.66518849720757)(51.42857142857143, 107.17610085020954)(51.50442477876107, 108.34073861554808)(51.580278128950695, 107.84077676191343)(51.656131479140335, 109.90748783883494)(51.73198482932997, 110.39663882591265)(51.8078381795196, 110.49439509549515)(51.88369152970923, 110.35404951026466)(51.959544879898864, 108.92250558168317)(52.035398230088504, 108.62444956984156)(52.111251580278136, 108.12907791122768)(52.18710493046777, 108.20653181050498)(52.2629582806574, 107.97944494180996)(52.33881163084703, 108.56644191814004)(52.414664981036665, 108.52304968809275)(52.4905183312263, 108.04411611988664)(52.56637168141593, 102.1802356814872)(52.64222503160557, 103.93523277810257)(52.7180783817952, 102.48682652728206)(52.793931731984834, 104.12967621566669)(52.86978508217447, 102.99496935687179)(52.9456384323641, 106.2320508485641)(53.02149178255374, 104.19727673410259)(53.097345132743364, 103.03398033417947)(53.173198482933, 104.50479740556408)(53.24905183312263, 105.29927997776925)(53.32490518331227, 105.21039017089748)(53.4007585335019, 105.06995265661538)(53.47661188369153, 104.44979089771795)(53.55246523388117, 105.30903345635899)(53.6283185840708, 105.85779514517948)(53.70417193426044, 106.10677611687181)(53.78002528445007, 105.56169662802564)(53.8558786346397, 105.38617678379487)(53.931731984829334, 105.98290715215386)(54.007585335018966, 104.19948870315385)(54.0834386852086, 104.69236053376923)(54.15929203539824, 105.94585248187181)(54.23514538558787, 104.63641767492307)(54.3109987357775, 107.32547645225641)(54.386852085967135, 106.1918744634359)(54.46270543615677, 106.13892991164099)(54.5385587863464, 104.94738369921053)(54.61441213653603, 104.15009460450001)(54.69026548672567, 103.51999729789475)(54.7661188369153, 105.0724020404737)(54.841972187104936, 105.95736340294738)(54.91782553729457, 104.80087510442105)(54.9936788874842, 102.65402785584212)(55.06953223767384, 106.05342837705264)(55.145385587863466, 103.4492488109737)(55.221238938053105, 103.6672207228158)(55.29709228824273, 102.60045769086113)(55.37294563843237, 103.5502334113611)(55.448798988622, 104.78529254924999)(55.524652338811634, 103.24203490944444)(55.600505689001274, 101.66341461550002)(55.6763590391909, 101.8620759787222)(55.75221238938054, 102.56591814433332)(55.828065739570164, 101.20023406274998)(55.9039190897598, 103.1749651927222)(55.979772439949436, 102.18317669658335)(56.05562579013907, 102.55139391188888)(56.1314791403287, 101.27872911747224)(56.20733249051833, 103.49623825639999)(56.28318584070797, 101.98360477342858)(56.359039190897604, 103.1009665024)(56.43489254108724, 104.07885219482857)(56.51074589127687, 103.9920642340857)(56.5865992414665, 104.29187641594287)(56.662452591656134, 102.31299014548568)(56.73830594184577, 103.50262950331431)(56.8141592920354, 103.7720729490571)(56.89001264222504, 102.8633719457143)(56.96586599241467, 102.76788568475757)(57.0417193426043, 101.88432946121213)(57.11757269279394, 102.85552751118182)(57.19342604298357, 101.67355320721212)(57.26927939317321, 101.50704254975003)(57.34513274336283, 102.7707325409375)(57.42098609355247, 102.28009977196876)(57.496839443742104, 100.46229958348387)(57.572692793931736, 100.62302877025807)(57.648546144121376, 102.59919224283871)(57.724399494311, 101.74214585283872)(57.80025284450064, 101.77402899819354)(57.876106194690266, 101.62991146419355)(57.951959544879905, 100.45382825796776)(58.02781289506954, 102.9047683455)(58.10366624525917, 103.16559513556666)(58.1795195954488, 101.83803994053336)(58.255372945638435, 104.1263100034333)(58.331226295828074, 103.35951743006667)(58.407079646017706, 102.35138982603334)(58.48293299620734, 104.25472273763334)(58.55878634639697, 105.8550753543)(58.6346396965866, 103.4310772229)(58.710493046776236, 102.55779626393331)(58.78634639696587, 99.6489411749)(58.8621997471555, 92.32307580240003)(58.93805309734514, 93.08821291733332)(59.01390644753477, 90.69681401526665)(59.089759797724405, 97.02490745616666)(59.16561314791404, 91.13404032706664)(59.24146649810367, 101.99462448665517)(59.31731984829331, 95.14345359010345)(59.393173198482934, 96.67892507637929)(59.46902654867257, 95.52201792848277)(59.5448798988622, 93.71521731762067)(59.62073324905184, 91.85250807279307)(59.69658659924148, 93.10230003524137)(59.7724399494311, 92.84233020803448)(59.84829329962074, 93.82362876384613)(59.92414664981037, 92.87716586256003)(60.0, 95.26981413668)
        };
        \addplot[color=blue, mark=none,name path=A] coordinates { %% MAX value
        (0.0, 131.02766915)(0.07585335018963338, 131.736773089)(0.15170670037926676, 130.115155163)(0.22756005056890014, 132.916698562)(0.3034134007585335, 131.586454532)(0.3792667509481669, 130.8918981)(0.45512010113780027, 131.699372722)(0.5309734513274336, 134.94317728800002)(0.606826801517067, 135.201293329)(0.6826801517067005, 134.492288429)(0.7585335018963338, 134.997824335)(0.8343868520859672, 132.844707588)(0.9102402022756005, 136.765360319)(0.986093552465234, 131.354496125)(1.0619469026548671, 132.24557149100002)(1.1378002528445006, 131.956630869)(1.213653603034134, 131.39147212)(1.2895069532237675, 144.28930361000002)(1.365360303413401, 143.956328399)(1.4412136536030342, 142.20252676799998)(1.5170670037926677, 142.630014627)(1.5929203539823011, 143.721909897)(1.6687737041719344, 143.887209734)(1.7446270543615676, 147.235722974)(1.820480404551201, 144.812992786)(1.8963337547408345, 147.108051883)(1.972187104930468, 147.904141329)(2.0480404551201015, 146.765698338)(2.1238938053097343, 147.188379733)(2.199747155499368, 145.96012523800002)(2.275600505689001, 147.632723214)(2.351453855878635, 145.71159056300002)(2.427307206068268, 152.94394198999998)(2.503160556257902, 152.77527904299998)(2.579013906447535, 154.529990413)(2.6548672566371687, 154.471537404)(2.730720606826802, 151.528964783)(2.806573957016435, 155.307713083)(2.8824273072060684, 154.532566001)(2.9582806573957017, 151.82833944)(3.0341340075853354, 156.10480886)(3.1099873577749686, 152.184508975)(3.1858407079646023, 164.470034462)(3.2616940581542355, 152.623885546)(3.3375474083438688, 154.866310435)(3.413400758533502, 151.194079213)(3.4892541087231352, 155.022429631)(3.565107458912769, 153.900909749)(3.640960809102402, 151.180729079)(3.716814159292036, 155.958093782)(3.792667509481669, 152.786913331)(3.8685208596713023, 153.75887309499998)(3.944374209860936, 152.000806)(4.020227560050569, 155.933121114)(4.096080910240203, 157.333734179)(4.171934260429836, 157.073971546)(4.2477876106194685, 157.86810791800002)(4.323640960809103, 167.955577715)(4.399494310998736, 165.249581522)(4.47534766118837, 168.855593027)(4.551201011378002, 166.8135891)(4.6270543615676365, 168.766917508)(4.70290771175727, 167.938872064)(4.778761061946904, 170.65024948299998)(4.854614412136536, 171.85468913600002)(4.9304677623261695, 169.41937481699998)(5.006321112515804, 168.295980199)(5.082174462705436, 159.88868857)(5.15802781289507, 159.984518173)(5.233881163084703, 160.04106444200002)(5.309734513274337, 160.945674198)(5.38558786346397, 161.60898538600003)(5.461441213653604, 159.524003348)(5.537294563843237, 155.526356415)(5.61314791403287, 153.248350896)(5.689001264222504, 149.112906362)(5.764854614412137, 150.148014596)(5.840707964601771, 147.854239258)(5.916561314791403, 142.352819636)(5.9924146649810375, 136.088838284)(6.068268015170671, 134.556585843)(6.144121365360304, 132.698518719)(6.219974715549937, 132.355555732)(6.29582806573957, 133.490600885)(6.371681415929205, 132.348896252)(6.447534766118837, 137.621680271)(6.523388116308471, 146.45583482499998)(6.599241466498104, 133.375736734)(6.6750948166877375, 144.12719524599999)(6.750948166877371, 138.193644979)(6.826801517067004, 135.154209781)(6.902654867256638, 136.717059549)(6.9785082174462705, 133.558369942)(7.054361567635905, 136.099949777)(7.130214917825538, 135.162015634)(7.206068268015172, 134.254673064)(7.281921618204804, 133.213970128)(7.357774968394438, 134.807976376)(7.433628318584072, 134.163152725)(7.509481668773706, 128.377292542)(7.585335018963338, 128.802172004)(7.661188369152971, 127.883554535)(7.737041719342605, 130.048198663)(7.812895069532239, 129.496312321)(7.888748419721872, 130.030838677)(7.964601769911505, 143.22698583)(8.040455120101138, 136.473600174)(8.116308470290772, 136.621163014)(8.192161820480406, 135.121163126)(8.268015170670038, 168.40024643700002)(8.343868520859672, 181.37944373599998)(8.419721871049305, 182.651407209)(8.495575221238937, 182.775900148)(8.571428571428573, 182.287608764)(8.647281921618205, 183.37196570100002)(8.72313527180784, 183.964388178)(8.798988621997472, 183.114899073)(8.874841972187106, 184.761610308)(8.95069532237674, 183.504313571)(9.026548672566372, 184.550299512)(9.102402022756005, 183.690185782)(9.178255372945639, 182.99632656400001)(9.254108723135273, 184.014012508)(9.329962073324905, 183.268814039)(9.40581542351454, 185.04275913)(9.481668773704172, 182.862779264)(9.557522123893808, 183.379441947)(9.63337547408344, 184.270763399)(9.709228824273072, 181.535154921)(9.785082174462707, 183.543843525)(9.860935524652339, 184.38914760900002)(9.936788874841973, 183.73518614900001)(10.012642225031607, 183.454521392)(10.08849557522124, 182.851528948)(10.164348925410872, 181.473572231)(10.240202275600508, 181.727949312)(10.31605562579014, 182.12714343300001)(10.391908975979774, 182.008814804)(10.467762326169407, 181.99713998)(10.543615676359039, 183.927647686)(10.619469026548675, 184.072792671)(10.695322376738307, 182.60751381)(10.77117572692794, 183.834009037)(10.847029077117574, 181.356480836)(10.922882427307208, 183.51159945)(10.99873577749684, 182.287115185)(11.074589127686474, 182.80880575)(11.150442477876107, 180.82103864099997)(11.22629582806574, 182.51153319899998)(11.302149178255375, 183.67154396)(11.378002528445007, 182.72751054600002)(11.453855878634641, 182.23008327)(11.529709228824274, 182.889784383)(11.605562579013906, 180.912863123)(11.681415929203542, 182.90942481599998)(11.757269279393174, 182.982465347)(11.833122629582807, 182.857596818)(11.90897597977244, 182.75581677000002)(11.984829329962075, 182.313555608)(12.060682680151707, 181.96850237700002)(12.136536030341341, 182.605601021)(12.212389380530974, 180.358790817)(12.288242730720608, 181.72705473899998)(12.364096080910242, 181.80439657800002)(12.439949431099874, 181.231267699)(12.515802781289509, 182.57248519400002)(12.59165613147914, 181.940843484)(12.667509481668775, 181.902127546)(12.74336283185841, 182.435825299)(12.819216182048041, 183.154233888)(12.895069532237674, 182.424149233)(12.970922882427308, 182.39742588399997)(13.046776232616942, 183.162850664)(13.122629582806574, 180.053706371)(13.198482932996209, 181.63543373)(13.274336283185841, 180.56601595499998)(13.350189633375475, 181.04729152099998)(13.42604298356511, 179.68209258200002)(13.501896333754742, 179.62715275)(13.577749683944376, 181.878336682)(13.653603034134008, 182.409924546)(13.729456384323642, 182.057826024)(13.805309734513276, 181.452274384)(13.881163084702909, 179.65916624)(13.957016434892541, 183.417787317)(14.032869785082175, 181.999076114)(14.10872313527181, 180.16716966)(14.184576485461443, 181.196086291)(14.260429835651076, 180.79070736900002)(14.336283185840708, 180.808737423)(14.412136536030344, 181.574941341)(14.487989886219976, 181.050530258)(14.563843236409609, 180.486595382)(14.639696586599243, 180.52468148)(14.715549936788875, 179.393591761)(14.79140328697851, 179.67583657)(14.867256637168143, 179.52709049999999)(14.943109987357776, 180.099207164)(15.018963337547412, 123.274707454)(15.094816687737044, 121.726948756)(15.170670037926676, 123.68069781999999)(15.24652338811631, 121.927849476)(15.322376738305943, 121.153015195)(15.398230088495575, 120.879125079)(15.47408343868521, 122.855407953)(15.549936788874842, 121.541583354)(15.625790139064478, 121.041883949)(15.701643489254112, 121.273356866)(15.777496839443744, 125.39283429899999)(15.853350189633378, 121.37462622599999)(15.92920353982301, 121.762049612)(16.005056890012643, 126.743193249)(16.080910240202275, 122.244549084)(16.15676359039191, 121.330206538)(16.232616940581543, 121.70766231)(16.30847029077118, 121.123898695)(16.38432364096081, 123.158096833)(16.460176991150444, 122.711240568)(16.536030341340076, 122.288326654)(16.611883691529712, 122.39786628399999)(16.687737041719345, 124.122257079)(16.763590391908977, 125.842175496)(16.83944374209861, 128.606936788)(16.91529709228824, 129.822618884)(16.991150442477874, 122.295761976)(17.067003792667514, 121.218208807)(17.142857142857146, 121.8937882)(17.21871049304678, 122.673348627)(17.29456384323641, 121.778161274)(17.370417193426043, 123.46867376200001)(17.44627054361568, 125.583363215)(17.52212389380531, 122.288355371)(17.597977243994944, 125.859834042)(17.673830594184576, 130.95180996599998)(17.749683944374212, 129.48855377499999)(17.825537294563844, 121.54716852199999)(17.90139064475348, 121.349055833)(17.977243994943112, 121.3781669)(18.053097345132745, 121.028011159)(18.128950695322377, 122.05681134000001)(18.20480404551201, 129.03011805900002)(18.280657395701645, 126.397681374)(18.356510745891278, 158.495587842)(18.432364096080914, 120.98543187300001)(18.508217446270546, 121.04523816300001)(18.58407079646018, 132.456017019)(18.65992414664981, 122.327397042)(18.735777496839447, 120.966210315)(18.81163084702908, 135.7947655)(18.88748419721871, 131.96160512900002)(18.963337547408344, 122.650869943)(19.039190897597976, 122.13222507799999)(19.115044247787615, 124.61545856500001)(19.190897597977248, 121.788991943)(19.26675094816688, 123.25414193)(19.342604298356513, 129.905268546)(19.418457648546145, 122.14618533199999)(19.494310998735777, 121.53882533299999)(19.570164348925413, 122.71586531)(19.646017699115045, 126.006024873)(19.721871049304678, 124.159748089)(19.797724399494314, 131.074179535)(19.873577749683946, 129.450254932)(19.94943109987358, 121.05316361999999)(20.025284450063214, 121.473027712)(20.101137800252847, 121.899454962)(20.17699115044248, 125.896822872)(20.25284450063211, 121.37531955200001)(20.328697850821744, 123.11879511000001)(20.40455120101138, 120.989622019)(20.480404551201016, 121.27478634799999)(20.556257901390648, 126.953593326)(20.63211125158028, 136.524512045)(20.707964601769913, 121.408693616)(20.78381795195955, 126.063113682)(20.85967130214918, 121.819540778)(20.935524652338813, 127.804888337)(21.011378002528446, 127.796226708)(21.087231352718078, 123.57768953600001)(21.163084702907714, 122.10359074899999)(21.23893805309735, 124.78697738099999)(21.314791403286982, 140.176881705)(21.390644753476614, 140.41567252)(21.466498103666247, 132.96433888299998)(21.54235145385588, 135.302273681)(21.61820480404551, 129.892333046)(21.694058154235147, 126.95562833)(21.76991150442478, 131.300329644)(21.845764854614416, 129.396034617)(21.921618204804048, 125.04061996000001)(21.99747155499368, 128.091280261)(22.073324905183316, 126.36567658000001)(22.14917825537295, 122.227210451)(22.22503160556258, 120.951572631)(22.300884955752213, 121.030882398)(22.376738305941846, 123.68735905)(22.45259165613148, 121.524035861)(22.528445006321114, 123.237785242)(22.60429835651075, 122.890962308)(22.680151706700382, 122.223445868)(22.756005056890015, 122.860141492)(22.831858407079647, 121.84482777599999)(22.907711757269283, 141.42472422999998)(22.983565107458915, 121.497571127)(23.059418457648547, 122.407578363)(23.13527180783818, 128.908267469)(23.211125158027812, 128.573362995)(23.286978508217448, 121.068515378)(23.362831858407084, 122.111997991)(23.438685208596716, 122.010787435)(23.51453855878635, 121.30304498299999)(23.59039190897598, 123.408903426)(23.666245259165613, 121.81188585)(23.74209860935525, 122.030905935)(23.81795195954488, 122.28956476100001)(23.893805309734514, 124.063301049)(23.96965865992415, 122.678433591)(24.045512010113782, 121.84738919799999)(24.121365360303415, 122.19478028099999)(24.19721871049305, 122.45390478399999)(24.273072060682683, 121.40309298)(24.348925410872315, 121.862366094)(24.424778761061948, 124.444386411)(24.50063211125158, 124.973555443)(24.576485461441216, 127.59996147800001)(24.65233881163085, 123.05586915999999)(24.728192161820484, 121.410248686)(24.804045512010116, 122.42532497699999)(24.87989886219975, 123.121386435)(24.95575221238938, 122.802934519)(25.031605562579017, 123.471811785)(25.10745891276865, 121.65077671600001)(25.18331226295828, 127.846999247)(25.259165613147914, 137.366754673)(25.33501896333755, 140.456343178)(25.410872313527186, 135.334142941)(25.48672566371682, 121.88266745)(25.56257901390645, 121.985929054)(25.638432364096083, 123.29060787)(25.714285714285715, 121.567253658)(25.790139064475348, 124.426773625)(25.865992414664984, 121.491097364)(25.941845764854616, 120.72137271800001)(26.017699115044252, 122.393976009)(26.093552465233884, 122.61228211900001)(26.169405815423517, 123.590143252)(26.24525916561315, 123.444329165)(26.321112515802785, 122.788673908)(26.396965865992417, 124.424454612)(26.47281921618205, 122.083496182)(26.548672566371682, 121.188325539)(26.624525916561314, 122.363861293)(26.70037926675095, 121.661949827)(26.776232616940586, 121.989063153)(26.85208596713022, 123.56858561099999)(26.92793931731985, 122.011859001)(27.003792667509483, 121.604071014)(27.07964601769912, 133.377414882)(27.15549936788875, 130.998268355)(27.231352718078384, 135.190201435)(27.307206068268016, 140.090443683)(27.38305941845765, 137.670504693)(27.458912768647284, 133.667848227)(27.53476611883692, 135.57159838500002)(27.610619469026553, 129.46666316099999)(27.686472819216185, 129.207838536)(27.762326169405817, 129.870685409)(27.83817951959545, 130.611452164)(27.914032869785082, 136.790432776)(27.989886219974718, 135.551747783)(28.06573957016435, 135.347220735)(28.141592920353986, 136.142265724)(28.21744627054362, 135.599110453)(28.29329962073325, 136.090023256)(28.369152970922887, 135.35761194999998)(28.44500632111252, 133.002349783)(28.52085967130215, 134.19533426799998)(28.596713021491784, 134.036264411)(28.672566371681416, 132.339727856)(28.748419721871052, 131.736948652)(28.824273072060688, 139.192286363)(28.90012642225032, 140.294617541)(28.975979772439953, 143.009118729)(29.051833122629585, 148.070196203)(29.127686472819217, 149.56371353)(29.203539823008853, 153.412482775)(29.279393173198486, 153.269756727)(29.355246523388118, 133.897766967)(29.43109987357775, 133.79442566)(29.506953223767386, 140.110564228)(29.58280657395702, 131.08987405800002)(29.658659924146654, 143.379202367)(29.734513274336287, 152.40065025)(29.81036662452592, 146.623987027)(29.88621997471555, 133.023120019)(29.962073324905184, 132.762500388)(30.037926675094823, 130.367539019)(30.113780025284452, 130.654307164)(30.189633375474088, 131.719237842)(30.265486725663717, 143.596448385)(30.341340075853353, 145.971915772)(30.417193426042985, 147.503117723)(30.49304677623262, 145.881058489)(30.568900126422257, 147.30851315200002)(30.644753476611886, 148.17128084)(30.72060682680152, 146.08742044)(30.79646017699115, 145.394380362)(30.872313527180786, 146.227292629)(30.94816687737042, 146.913617536)(31.024020227560055, 146.495495354)(31.099873577749683, 146.91602174)(31.17572692793932, 145.392778133)(31.251580278128955, 146.202057321)(31.327433628318587, 146.540302298)(31.403286978508223, 151.889908912)(31.479140328697852, 153.46484297)(31.554993678887488, 151.567927688)(31.630847029077117, 153.775721193)(31.706700379266756, 154.48406717)(31.782553729456385, 153.620238237)(31.85840707964602, 153.671222157)(31.934260429835657, 155.17509136799998)(32.010113780025286, 154.866532417)(32.085967130214925, 149.334823325)(32.16182048040455, 152.99577792)(32.23767383059419, 151.578045742)(32.31352718078382, 154.484082752)(32.389380530973455, 154.411945227)(32.46523388116309, 153.98198825)(32.54108723135272, 156.122332113)(32.61694058154236, 151.869012658)(32.692793931731984, 153.885164325)(32.76864728192162, 156.423223643)(32.844500632111256, 157.738381481)(32.92035398230089, 153.83614166499999)(32.99620733249052, 159.241455987)(33.07206068268015, 158.429085618)(33.147914032869785, 157.24189663)(33.223767383059425, 161.06258519600001)(33.29962073324906, 172.47628615099998)(33.37547408343869, 172.360584804)(33.45132743362832, 171.547782278)(33.527180783817954, 173.818281871)(33.603034134007586, 173.47681558600001)(33.67888748419722, 173.012326316)(33.75474083438686, 171.73245071899998)(33.83059418457648, 172.582632805)(33.90644753476612, 171.66539101200001)(33.98230088495575, 160.34658871000002)(34.05815423514539, 158.527537396)(34.13400758533503, 161.19910440400002)(34.20986093552465, 160.769963219)(34.28571428571429, 161.994456614)(34.36156763590392, 158.441011755)(34.43742098609356, 156.40094003)(34.51327433628319, 154.880446631)(34.58912768647282, 151.565441142)(34.664981036662454, 148.663566708)(34.740834386852086, 149.435900263)(34.816687737041725, 153.002221296)(34.89254108723136, 152.71317630000001)(34.96839443742099, 150.098940556)(35.04424778761062, 152.03979397199998)(35.120101137800255, 151.143964026)(35.19595448798989, 152.28103517399998)(35.27180783817953, 151.439425734)(35.34766118836915, 152.90319745099998)(35.42351453855879, 153.310363835)(35.499367888748424, 160.82551983)(35.575221238938056, 165.175008149)(35.65107458912769, 153.71953915)(35.72692793931732, 154.855136194)(35.80278128950696, 152.13573322)(35.878634639696585, 151.08818810100001)(35.954487989886225, 130.908810622)(36.03034134007585, 128.96257265)(36.10619469026549, 134.848055668)(36.18204804045513, 137.548494143)(36.257901390644754, 136.866125493)(36.333754740834394, 152.404872589)(36.40960809102402, 153.51258658)(36.48546144121366, 142.085713671)(36.56131479140329, 137.140613991)(36.63716814159292, 132.005855812)(36.713021491782555, 135.276475474)(36.78887484197219, 129.43173276599998)(36.86472819216183, 122.51668030399999)(36.94058154235145, 127.743474929)(37.01643489254109, 141.727399284)(37.092288242730724, 122.26354788)(37.16814159292036, 122.535055233)(37.24399494310999, 150.535836082)(37.31984829329962, 131.464806178)(37.395701643489254, 141.752461391)(37.47155499367889, 124.75209613000001)(37.547408343868526, 133.62054197100002)(37.62326169405816, 135.529388282)(37.69911504424779, 139.333407514)(37.77496839443742, 144.747242368)(37.85082174462706, 143.77665947600002)(37.92667509481669, 146.525966861)(38.00252844500633, 147.624778614)(38.07838179519595, 144.841015264)(38.15423514538559, 147.72487761300002)(38.23008849557523, 147.49119064)(38.305941845764856, 147.785655244)(38.381795195954496, 144.809369232)(38.45764854614412, 144.860317813)(38.53350189633376, 145.442845403)(38.60935524652339, 147.512656318)(38.685208596713025, 146.170510131)(38.76106194690266, 146.013644944)(38.83691529709229, 150.066939656)(38.91276864728193, 154.828515638)(38.988621997471554, 153.373021488)(39.064475347661194, 154.5776353)(39.140328697850826, 151.680657993)(39.21618204804046, 154.062670094)(39.29203539823009, 154.585475855)(39.36788874841972, 153.201413726)(39.443742098609356, 150.522935468)(39.519595448798995, 150.893408003)(39.59544879898863, 152.87027778200002)(39.67130214917826, 154.263583206)(39.74715549936789, 151.206493414)(39.823008849557525, 151.345554685)(39.89886219974716, 153.214035308)(39.97471554993679, 155.80503867599998)(40.05056890012643, 154.21463885)(40.126422250316054, 152.091598657)(40.20227560050569, 155.951189821)(40.278128950695326, 153.736967832)(40.35398230088496, 167.680778697)(40.4298356510746, 150.003260759)(40.50568900126422, 154.04058447100002)(40.58154235145386, 161.984713427)(40.65739570164349, 153.733519877)(40.73324905183313, 154.86465364900002)(40.80910240202276, 156.68580964400002)(40.88495575221239, 163.842570421)(40.96080910240203, 172.084313324)(41.036662452591656, 172.639750325)(41.112515802781296, 171.330957415)(41.18836915297093, 171.317977456)(41.26422250316056, 170.74829866599998)(41.34007585335019, 171.54167410300002)(41.415929203539825, 163.82043823)(41.49178255372946, 158.515736894)(41.5676359039191, 160.929140928)(41.64348925410873, 160.405491983)(41.71934260429836, 158.00088295)(41.795195954487994, 155.87869641199998)(41.871049304677626, 157.616423511)(41.94690265486726, 156.617682617)(42.02275600505689, 149.499980234)(42.09860935524653, 149.483119252)(42.174462705436156, 148.34886484700002)(42.250316055625795, 147.895356704)(42.32616940581543, 145.02142315700002)(42.40202275600506, 140.144323116)(42.4778761061947, 140.665916045)(42.553729456384325, 131.0836098)(42.629582806573964, 130.44358946399998)(42.70543615676359, 130.55540987700002)(42.78128950695323, 130.350970836)(42.85714285714286, 131.00382384899999)(42.932996207332494, 129.319273067)(43.00884955752213, 128.759259072)(43.08470290771176, 137.89557285)(43.1605562579014, 137.081252387)(43.23640960809102, 130.294497135)(43.31226295828066, 133.25931842199998)(43.388116308470295, 144.37254298)(43.46396965865993, 127.085859056)(43.53982300884956, 132.009272974)(43.61567635903919, 129.634303791)(43.69152970922883, 129.001670217)(43.767383059418464, 121.633101877)(43.843236409608096, 121.013722943)(43.91908975979773, 123.316069248)(43.99494310998736, 120.919089302)(44.07079646017699, 125.329607157)(44.14664981036663, 133.732479776)(44.22250316055626, 134.980599026)(44.2983565107459, 140.885597204)(44.37420986093552, 140.73427416799998)(44.45006321112516, 136.729009578)(44.5259165613148, 137.18002734)(44.60176991150443, 134.963578985)(44.677623261694066, 133.808293252)(44.75347661188369, 129.971129472)(44.82932996207333, 129.274796578)(44.90518331226296, 133.058450443)(44.981036662452595, 132.171880475)(45.05689001264223, 181.55414772)(45.13274336283186, 182.37078048200001)(45.2085967130215, 179.38205231400002)(45.284450063211125, 181.04110702300002)(45.360303413400764, 181.507322037)(45.4361567635904, 179.549558099)(45.51201011378003, 181.164218616)(45.58786346396966, 179.69489405000002)(45.663716814159294, 178.75773860700002)(45.739570164348926, 180.361813258)(45.815423514538566, 180.710007588)(45.8912768647282, 180.06624086)(45.96713021491783, 178.74682595)(46.04298356510746, 180.540629765)(46.118836915297095, 179.489138285)(46.19469026548673, 178.82739512)(46.27054361567636, 180.141140315)(46.346396965866, 178.019909108)(46.422250316055624, 178.281738732)(46.498103666245264, 178.155740974)(46.573957016434896, 177.76831594200002)(46.64981036662453, 177.867903451)(46.72566371681417, 177.53403752)(46.80151706700379, 179.994001554)(46.87737041719343, 179.545257354)(46.95322376738306, 178.761810417)(47.0290771175727, 180.041552549)(47.10493046776233, 178.847890374)(47.18078381795196, 180.231536893)(47.2566371681416, 179.56184789999998)(47.33249051833123, 179.181905252)(47.408343868520866, 179.61988003199997)(47.4841972187105, 176.77065947399998)(47.56005056890013, 176.780641803)(47.63590391908976, 177.39620197)(47.711757269279396, 179.338628717)(47.78761061946903, 179.27634168400002)(47.86346396965867, 177.69036702699998)(47.9393173198483, 177.086129978)(48.01517067003793, 177.91580063)(48.091024020227565, 178.600340103)(48.1668773704172, 179.393664102)(48.24273072060683, 177.605520812)(48.31858407079646, 178.77875906100002)(48.3944374209861, 178.478985573)(48.470290771175726, 178.20375145500003)(48.546144121365366, 178.378017656)(48.621997471555, 178.488713575)(48.69785082174463, 177.74183756)(48.77370417193427, 178.625655717)(48.849557522123895, 178.523602869)(48.925410872313535, 178.058015242)(49.00126422250316, 175.38803468400002)(49.0771175726928, 177.081151548)(49.15297092288243, 176.78932177299998)(49.228824273072064, 177.33493331199998)(49.3046776232617, 177.28563938000002)(49.38053097345133, 177.144808923)(49.45638432364097, 175.55515828)(49.53223767383059, 176.509988696)(49.60809102402023, 177.091551394)(49.683944374209865, 175.787250369)(49.7597977243995, 176.133885325)(49.83565107458913, 175.10419036999997)(49.91150442477876, 177.580481369)(49.9873577749684, 175.95436587199998)(50.063211125158034, 175.45598131100002)(50.139064475347666, 177.41105975300002)(50.2149178255373, 176.553174053)(50.29077117572693, 175.00063193900002)(50.36662452591656, 174.945519967)(50.4424778761062, 174.08390305)(50.51833122629583, 173.965889391)(50.59418457648547, 174.9604922)(50.6700379266751, 176.109519148)(50.74589127686473, 175.213053217)(50.82174462705437, 175.294136687)(50.897597977244, 177.31558751)(50.97345132743364, 177.358974708)(51.04930467762326, 176.863143299)(51.1251580278129, 176.66746110399998)(51.20101137800253, 175.21976238899998)(51.276864728192166, 173.787188609)(51.352718078381805, 177.399468613)(51.42857142857143, 174.07130441700002)(51.50442477876107, 172.581524471)(51.580278128950695, 176.135692571)(51.656131479140335, 146.515362241)(51.73198482932997, 123.92201848299999)(51.8078381795196, 122.66368499800001)(51.88369152970923, 122.111190487)(51.959544879898864, 121.602190184)(52.035398230088504, 124.074441569)(52.111251580278136, 121.77315571099999)(52.18710493046777, 121.46763768299999)(52.2629582806574, 122.94994791600001)(52.33881163084703, 121.057990883)(52.414664981036665, 121.58209561)(52.4905183312263, 120.8428557)(52.56637168141593, 120.222651211)(52.64222503160557, 120.479248227)(52.7180783817952, 120.240560812)(52.793931731984834, 122.12973399699999)(52.86978508217447, 121.387668621)(52.9456384323641, 131.196079994)(53.02149178255374, 122.015032988)(53.097345132743364, 121.64659913599999)(53.173198482933, 120.772790776)(53.24905183312263, 123.079549174)(53.32490518331227, 121.35671242500001)(53.4007585335019, 121.720927783)(53.47661188369153, 121.733925255)(53.55246523388117, 122.04380020100001)(53.6283185840708, 121.812228533)(53.70417193426044, 120.70405730499999)(53.78002528445007, 122.285846822)(53.8558786346397, 120.58524367700001)(53.931731984829334, 122.69427416799999)(54.007585335018966, 121.68754725699999)(54.0834386852086, 125.798754431)(54.15929203539824, 121.563981009)(54.23514538558787, 119.569452852)(54.3109987357775, 120.90450862499999)(54.386852085967135, 120.90813374)(54.46270543615677, 121.819021118)(54.5385587863464, 121.492530375)(54.61441213653603, 121.013869374)(54.69026548672567, 120.38685540899999)(54.7661188369153, 121.41688247399999)(54.841972187104936, 120.81034875)(54.91782553729457, 122.694392567)(54.9936788874842, 122.52712128600001)(55.06953223767384, 123.19004803300001)(55.145385587863466, 123.301644216)(55.221238938053105, 121.786902271)(55.29709228824273, 122.07911737399999)(55.37294563843237, 121.769115717)(55.448798988622, 122.07663029900002)(55.524652338811634, 121.346698084)(55.600505689001274, 121.662734502)(55.6763590391909, 121.10091800299999)(55.75221238938054, 122.066208279)(55.828065739570164, 120.617519188)(55.9039190897598, 120.187842578)(55.979772439949436, 119.96106865)(56.05562579013907, 119.749305931)(56.1314791403287, 120.50236507899999)(56.20733249051833, 119.912839286)(56.28318584070797, 119.511830099)(56.359039190897604, 122.669496212)(56.43489254108724, 121.220264802)(56.51074589127687, 127.225299301)(56.5865992414665, 120.39434096800001)(56.662452591656134, 121.421326178)(56.73830594184577, 121.723571994)(56.8141592920354, 122.23209411399999)(56.89001264222504, 122.26430517400001)(56.96586599241467, 120.280979913)(57.0417193426043, 120.62135360699999)(57.11757269279394, 120.588553817)(57.19342604298357, 120.45724881699999)(57.26927939317321, 120.545801085)(57.34513274336283, 119.33437832799999)(57.42098609355247, 120.30817324799999)(57.496839443742104, 120.322032519)(57.572692793931736, 120.20204792)(57.648546144121376, 135.338313466)(57.724399494311, 122.364244864)(57.80025284450064, 119.79137763400001)(57.876106194690266, 120.670391973)(57.951959544879905, 122.222600155)(58.02781289506954, 120.76868750499999)(58.10366624525917, 121.14461308599999)(58.1795195954488, 122.357569924)(58.255372945638435, 123.015955116)(58.331226295828074, 121.703616617)(58.407079646017706, 121.895662516)(58.48293299620734, 126.472399247)(58.55878634639697, 121.66654767200001)(58.6346396965866, 122.805035891)(58.710493046776236, 125.692992803)(58.78634639696587, 121.958267395)(58.8621997471555, 125.537733666)(58.93805309734514, 142.918204184)(59.01390644753477, 119.786955286)(59.089759797724405, 119.710484504)(59.16561314791404, 122.03739792600001)(59.24146649810367, 119.31841982099999)(59.31731984829331, 119.50242424300001)(59.393173198482934, 120.813931118)(59.46902654867257, 121.58007321299999)(59.5448798988622, 123.062756139)(59.62073324905184, 121.97550694)(59.69658659924148, 122.479481088)(59.7724399494311, 122.258599237)(59.84829329962074, 120.148084982)(59.92414664981037, 121.199022646)(60.0, 119.94247434100001)
        };
        \addplot[color=blue, mark=none,name path=B] coordinates { %% MIN value
        (0.0, 14.043494637)(0.07585335018963338, 43.13862234)(0.15170670037926676, 38.322631268)(0.22756005056890014, 35.293649397)(0.3034134007585335, 37.897336888)(0.3792667509481669, 36.57902844)(0.45512010113780027, 40.422352010000004)(0.5309734513274336, 43.086640697)(0.606826801517067, 37.476316509)(0.6826801517067005, 53.537893809)(0.7585335018963338, 36.331815014)(0.8343868520859672, 43.039574800000004)(0.9102402022756005, 43.463991733)(0.986093552465234, 10.696389585999999)(1.0619469026548671, 35.195372674)(1.1378002528445006, 51.300427451)(1.213653603034134, 33.420343327)(1.2895069532237675, 44.194734557000004)(1.365360303413401, 44.013582889)(1.4412136536030342, 40.910392066)(1.5170670037926677, 44.686789169)(1.5929203539823011, 47.645492829)(1.6687737041719344, 46.070193284)(1.7446270543615676, 37.741037881000004)(1.820480404551201, 48.312324336)(1.8963337547408345, 54.655474975000004)(1.972187104930468, 44.094788315)(2.0480404551201015, 38.259187968)(2.1238938053097343, 47.634730497999996)(2.199747155499368, 29.369455504999998)(2.275600505689001, 56.235815838)(2.351453855878635, 44.90632057)(2.427307206068268, 21.853730305)(2.503160556257902, 33.879466971)(2.579013906447535, 23.053847001999998)(2.6548672566371687, 14.127956209999999)(2.730720606826802, 12.216771481999999)(2.806573957016435, 35.548458427999996)(2.8824273072060684, 12.993635346000001)(2.9582806573957017, 22.284627537000002)(3.0341340075853354, 12.712998712000001)(3.1099873577749686, 51.758871155)(3.1858407079646023, 32.452759669)(3.2616940581542355, 27.042020806)(3.3375474083438688, 13.856347061000001)(3.413400758533502, 11.617359921999999)(3.4892541087231352, 11.807270822)(3.565107458912769, 54.154711142)(3.640960809102402, 51.407847127000004)(3.716814159292036, 26.439005774)(3.792667509481669, 12.776828773)(3.8685208596713023, 38.620413069)(3.944374209860936, 11.168355940000001)(4.020227560050569, 40.372041163)(4.096080910240203, 43.5174867)(4.171934260429836, 46.488701942999995)(4.2477876106194685, 39.378450762)(4.323640960809103, 36.934973883)(4.399494310998736, 43.666880569)(4.47534766118837, 42.205045357)(4.551201011378002, 48.92477911900001)(4.6270543615676365, 37.004613825999996)(4.70290771175727, 41.613760326999994)(4.778761061946904, 43.992534626)(4.854614412136536, 47.184612235)(4.9304677623261695, 38.83605017)(5.006321112515804, 50.756368617999996)(5.082174462705436, 46.098285622999995)(5.15802781289507, 42.714615595)(5.233881163084703, 45.363557608)(5.309734513274337, 34.743143148)(5.38558786346397, 48.384000854)(5.461441213653604, 44.733888017999995)(5.537294563843237, 24.110866086999998)(5.61314791403287, 34.004351139)(5.689001264222504, 31.937267542)(5.764854614412137, 18.982429743)(5.840707964601771, 28.904919710999998)(5.916561314791403, 12.688823458)(5.9924146649810375, 10.901643893)(6.068268015170671, 44.390958387)(6.144121365360304, 14.484308229)(6.219974715549937, 28.210558329)(6.29582806573957, 47.17321572)(6.371681415929205, 34.072013423)(6.447534766118837, 14.149270122999999)(6.523388116308471, 29.378925658)(6.599241466498104, 34.149110317)(6.6750948166877375, 39.810696871000005)(6.750948166877371, 39.637101451)(6.826801517067004, 37.575956481)(6.902654867256638, 35.526556403)(6.9785082174462705, 48.407614527999996)(7.054361567635905, 11.411881765)(7.130214917825538, 47.580878497)(7.206068268015172, 36.259890794)(7.281921618204804, 41.664673099)(7.357774968394438, 42.98699017)(7.433628318584072, 42.275606191)(7.509481668773706, 50.50505211)(7.585335018963338, 20.556827151)(7.661188369152971, 36.778477841)(7.737041719342605, 13.919323934000001)(7.812895069532239, 13.017441663)(7.888748419721872, 12.137304083)(7.964601769911505, 25.685082541)(8.040455120101138, 46.929887448)(8.116308470290772, 27.706772348)(8.192161820480406, 24.98557429)(8.268015170670038, 36.951596632000005)(8.343868520859672, 14.7943055)(8.419721871049305, 44.645875577)(8.495575221238937, 51.243710676000006)(8.571428571428573, 40.056337695)(8.647281921618205, 45.182794297)(8.72313527180784, 50.206894618)(8.798988621997472, 48.378930714)(8.874841972187106, 12.966221568)(8.95069532237674, 48.631591914000005)(9.026548672566372, 43.027208501000004)(9.102402022756005, 48.502181236000006)(9.178255372945639, 45.549711536000004)(9.254108723135273, 35.722252463)(9.329962073324905, 50.052833279)(9.40581542351454, 48.033791924000006)(9.481668773704172, 42.144060882)(9.557522123893808, 33.820483278000005)(9.63337547408344, 12.560851976)(9.709228824273072, 19.762351792)(9.785082174462707, 28.672460354000002)(9.860935524652339, 41.732342878)(9.936788874841973, 36.949461555999996)(10.012642225031607, 12.657213137000001)(10.08849557522124, 11.999691228)(10.164348925410872, 20.72431456)(10.240202275600508, 14.015016)(10.31605562579014, 28.617569924)(10.391908975979774, 28.833629740000003)(10.467762326169407, 21.71476826)(10.543615676359039, 43.846852995)(10.619469026548675, 31.152266416)(10.695322376738307, 11.498623814)(10.77117572692794, 46.778524211)(10.847029077117574, 12.906790046000001)(10.922882427307208, 46.420867936)(10.99873577749684, 44.413300725999996)(11.074589127686474, 48.653554711)(11.150442477876107, 38.789893467)(11.22629582806574, 12.300066999999999)(11.302149178255375, 44.72885015)(11.378002528445007, 39.640236186)(11.453855878634641, 30.811960579)(11.529709228824274, 43.756042050000005)(11.605562579013906, 35.000042139)(11.681415929203542, 48.207282066000005)(11.757269279393174, 44.234643938999994)(11.833122629582807, 11.462336977)(11.90897597977244, 37.448211109999995)(11.984829329962075, 30.669437166999998)(12.060682680151707, 48.092040557000004)(12.136536030341341, 10.809914924)(12.212389380530974, 36.196991905)(12.288242730720608, 41.220048902)(12.364096080910242, 45.03596061099999)(12.439949431099874, 27.900798877)(12.515802781289509, 26.35774407)(12.59165613147914, 37.50273221)(12.667509481668775, 37.033067831)(12.74336283185841, 35.202263171)(12.819216182048041, 26.271471931999997)(12.895069532237674, 49.849690418)(12.970922882427308, 29.085601616)(13.046776232616942, 22.581246202)(13.122629582806574, 36.328510673)(13.198482932996209, 13.461587779999999)(13.274336283185841, 12.314783484)(13.350189633375475, 13.309386972999999)(13.42604298356511, 30.337734382)(13.501896333754742, 13.017496838)(13.577749683944376, 12.196679635)(13.653603034134008, 22.513923171000002)(13.729456384323642, 13.552536785000001)(13.805309734513276, 39.16829536)(13.881163084702909, 30.798819448)(13.957016434892541, 32.715564081)(14.032869785082175, 32.978274156)(14.10872313527181, 16.076479114)(14.184576485461443, 47.110690672000004)(14.260429835651076, 26.167083496)(14.336283185840708, 33.638636756000004)(14.412136536030344, 43.405044732)(14.487989886219976, 20.672350104000003)(14.563843236409609, 23.44874096)(14.639696586599243, 47.011302136)(14.715549936788875, 37.192963147)(14.79140328697851, 45.72643917)(14.867256637168143, 12.10257222)(14.943109987357776, 47.530371902)(15.018963337547412, 48.921745294000004)(15.094816687737044, 35.467196634000004)(15.170670037926676, 38.811056295)(15.24652338811631, 45.44539507499999)(15.322376738305943, 41.800302075)(15.398230088495575, 36.540345480999996)(15.47408343868521, 43.170387549999994)(15.549936788874842, 43.826552548)(15.625790139064478, 40.357342030999995)(15.701643489254112, 34.689030866)(15.777496839443744, 35.165169474)(15.853350189633378, 48.260683144999994)(15.92920353982301, 28.883386246999997)(16.005056890012643, 45.176587946)(16.080910240202275, 22.687645136999997)(16.15676359039191, 49.101087212)(16.232616940581543, 35.512649337)(16.30847029077118, 30.839028825)(16.38432364096081, 48.537286742)(16.460176991150444, 44.394979357000004)(16.536030341340076, 54.901221456)(16.611883691529712, 51.062010292000004)(16.687737041719345, 48.562524497000005)(16.763590391908977, 51.023357047999994)(16.83944374209861, 48.534480531)(16.91529709228824, 42.708482270000005)(16.991150442477874, 50.341191668)(17.067003792667514, 49.935706558)(17.142857142857146, 51.743340221)(17.21871049304678, 49.569599728)(17.29456384323641, 47.799474019)(17.370417193426043, 47.520685441)(17.44627054361568, 50.895435055)(17.52212389380531, 50.309710325)(17.597977243994944, 36.897285775)(17.673830594184576, 12.376661294)(17.749683944374212, 41.004217495999995)(17.825537294563844, 11.584443221)(17.90139064475348, 48.639973438000006)(17.977243994943112, 36.268732379)(18.053097345132745, 33.411561314)(18.128950695322377, 18.180228488)(18.20480404551201, 14.85445322)(18.280657395701645, 12.824653158999999)(18.356510745891278, 52.19608256)(18.432364096080914, 11.747246012999998)(18.508217446270546, 26.208688468000002)(18.58407079646018, 19.489437347)(18.65992414664981, 37.363958725)(18.735777496839447, 11.502089735)(18.81163084702908, 12.366936008)(18.88748419721871, 10.405671403000001)(18.963337547408344, 40.756340003)(19.039190897597976, 37.204643032999996)(19.115044247787615, 22.162835984)(19.190897597977248, 49.114404345000004)(19.26675094816688, 10.892108890000001)(19.342604298356513, 47.558046637000004)(19.418457648546145, 41.171816152999995)(19.494310998735777, 45.148701392)(19.570164348925413, 39.540345873999996)(19.646017699115045, 36.683018319)(19.721871049304678, 38.924016776)(19.797724399494314, 39.678564695)(19.873577749683946, 42.776875957)(19.94943109987358, 39.202853569)(20.025284450063214, 42.683862785)(20.101137800252847, 37.027541948999996)(20.17699115044248, 35.615722708)(20.25284450063211, 34.483943834)(20.328697850821744, 40.425401855000004)(20.40455120101138, 32.870126037)(20.480404551201016, 25.121553398)(20.556257901390648, 11.70952792)(20.63211125158028, 9.430218883)(20.707964601769913, 18.575334461)(20.78381795195955, 10.541486977000002)(20.85967130214918, 11.81034953)(20.935524652338813, 24.721275782)(21.011378002528446, 11.829572320999999)(21.087231352718078, 26.174094323000002)(21.163084702907714, 34.473008053)(21.23893805309735, 13.681768808000001)(21.314791403286982, 13.040607545)(21.390644753476614, 22.002843198)(21.466498103666247, 25.666823409)(21.54235145385588, 25.051522518)(21.61820480404551, 26.840002106)(21.694058154235147, 25.693771241999997)(21.76991150442478, 22.594117022)(21.845764854614416, 42.685753248999994)(21.921618204804048, 12.513250128)(21.99747155499368, 49.666530259000005)(22.073324905183316, 50.748134613)(22.14917825537295, 31.285894238999997)(22.22503160556258, 38.641111417)(22.300884955752213, 32.676956408)(22.376738305941846, 32.233862902)(22.45259165613148, 34.232824747)(22.528445006321114, 12.720843618)(22.60429835651075, 9.910331074)(22.680151706700382, 20.846098075)(22.756005056890015, 23.952316985)(22.831858407079647, 14.585121773000001)(22.907711757269283, 30.114787610999997)(22.983565107458915, 36.358520079)(23.059418457648547, 51.715378442)(23.13527180783818, 30.328920842)(23.211125158027812, 35.974608364999995)(23.286978508217448, 13.291221541)(23.362831858407084, 36.771413378000005)(23.438685208596716, 10.653879542)(23.51453855878635, 39.499687064)(23.59039190897598, 50.926378908)(23.666245259165613, 41.013883251)(23.74209860935525, 37.239158825000004)(23.81795195954488, 34.742824635)(23.893805309734514, 43.415042615)(23.96965865992415, 56.795898772)(24.045512010113782, 49.165016413000004)(24.121365360303415, 41.483600366999994)(24.19721871049305, 43.46393508)(24.273072060682683, 48.99448519)(24.348925410872315, 50.256566111999994)(24.424778761061948, 48.975563353)(24.50063211125158, 49.620990195000005)(24.576485461441216, 44.2462359)(24.65233881163085, 46.033650126000005)(24.728192161820484, 36.690129373000005)(24.804045512010116, 43.938339102)(24.87989886219975, 57.685839361999996)(24.95575221238938, 48.291859210999995)(25.031605562579017, 33.37036952)(25.10745891276865, 12.914340664)(25.18331226295828, 56.045808956)(25.259165613147914, 13.916187496000001)(25.33501896333755, 24.353114048000002)(25.410872313527186, 13.569776771999999)(25.48672566371682, 14.776782391000001)(25.56257901390645, 12.6280528)(25.638432364096083, 44.484278456000006)(25.714285714285715, 12.751851245000001)(25.790139064475348, 47.155482854999995)(25.865992414664984, 48.870232294999994)(25.941845764854616, 19.108120145)(26.017699115044252, 11.347858974000001)(26.093552465233884, 10.512956534)(26.169405815423517, 12.525450391)(26.24525916561315, 27.215557422)(26.321112515802785, 35.177867662)(26.396965865992417, 47.598706803)(26.47281921618205, 12.549295118)(26.548672566371682, 40.121605818)(26.624525916561314, 37.040733477)(26.70037926675095, 45.490167245)(26.776232616940586, 24.144948108999998)(26.85208596713022, 35.063342242000004)(26.92793931731985, 39.550242618)(27.003792667509483, 36.603371623)(27.07964601769912, 45.010951495)(27.15549936788875, 37.409069781)(27.231352718078384, 39.710774197)(27.307206068268016, 40.898064233999996)(27.38305941845765, 44.97710944799999)(27.458912768647284, 16.661974926)(27.53476611883692, 33.498161175)(27.610619469026553, 42.956100273000004)(27.686472819216185, 39.284750237)(27.762326169405817, 48.551601961)(27.83817951959545, 15.901502465)(27.914032869785082, 45.100842438)(27.989886219974718, 40.53705531600001)(28.06573957016435, 45.734380968)(28.141592920353986, 28.807765055)(28.21744627054362, 46.190100192)(28.29329962073325, 28.19098185)(28.369152970922887, 39.721855874)(28.44500632111252, 46.742034206999996)(28.52085967130215, 24.698677077000003)(28.596713021491784, 54.335543605)(28.672566371681416, 25.090317252)(28.748419721871052, 34.189682323)(28.824273072060688, 47.939223682)(28.90012642225032, 17.065031304)(28.975979772439953, 43.667610625)(29.051833122629585, 48.281201191)(29.127686472819217, 31.858476805)(29.203539823008853, 52.912286636000005)(29.279393173198486, 54.692909637999996)(29.355246523388118, 51.356131039)(29.43109987357775, 45.330121381)(29.506953223767386, 41.309261125)(29.58280657395702, 32.190571574)(29.658659924146654, 44.580803511999996)(29.734513274336287, 46.336984926999996)(29.81036662452592, 13.129870825000001)(29.88621997471555, 35.906716216)(29.962073324905184, 41.337777323)(30.037926675094823, 26.056841667)(30.113780025284452, 43.008142038)(30.189633375474088, 24.368400545)(30.265486725663717, 13.58342959)(30.341340075853353, 34.283529187)(30.417193426042985, 50.436511197)(30.49304677623262, 30.178961394999998)(30.568900126422257, 38.530280917)(30.644753476611886, 43.22959847)(30.72060682680152, 44.720918860000005)(30.79646017699115, 35.463418973)(30.872313527180786, 45.814945009)(30.94816687737042, 27.933344783)(31.024020227560055, 36.842622849)(31.099873577749683, 39.391564532)(31.17572692793932, 23.946143208000002)(31.251580278128955, 36.359312378)(31.327433628318587, 36.580021507)(31.403286978508223, 42.589070346)(31.479140328697852, 27.312633667)(31.554993678887488, 48.291280239)(31.630847029077117, 45.234976112)(31.706700379266756, 41.120562496000005)(31.782553729456385, 43.780362620000005)(31.85840707964602, 45.563237259)(31.934260429835657, 39.17081518)(32.010113780025286, 34.998438837)(32.085967130214925, 53.561048191999994)(32.16182048040455, 57.044847769)(32.23767383059419, 48.355038139)(32.31352718078382, 35.661089816)(32.389380530973455, 47.860812556)(32.46523388116309, 53.937522575)(32.54108723135272, 14.906538862000001)(32.61694058154236, 12.252341396999999)(32.692793931731984, 12.046383139000001)(32.76864728192162, 24.460731711)(32.844500632111256, 28.052083886000002)(32.92035398230089, 24.968415588)(32.99620733249052, 11.514784585)(33.07206068268015, 10.077817066999998)(33.147914032869785, 9.176912895000001)(33.223767383059425, 12.910346287000001)(33.29962073324906, 25.475640447)(33.37547408343869, 13.992042454)(33.45132743362832, 12.077268949)(33.527180783817954, 26.52228317)(33.603034134007586, 29.949519567)(33.67888748419722, 38.211683329)(33.75474083438686, 35.645278797)(33.83059418457648, 26.227860271)(33.90644753476612, 36.184120752)(33.98230088495575, 44.878646102)(34.05815423514539, 43.277391685)(34.13400758533503, 42.242710892000005)(34.20986093552465, 51.314069001)(34.28571428571429, 50.602121276999995)(34.36156763590392, 12.963456857)(34.43742098609356, 36.084472221)(34.51327433628319, 43.954428553)(34.58912768647282, 16.640995786999998)(34.664981036662454, 39.853062142)(34.740834386852086, 42.192492647)(34.816687737041725, 39.430411326)(34.89254108723136, 34.935221677)(34.96839443742099, 13.332522285)(35.04424778761062, 50.055144184)(35.120101137800255, 33.001342948)(35.19595448798989, 36.577556692)(35.27180783817953, 33.094523353)(35.34766118836915, 43.305438581000004)(35.42351453855879, 25.819512696000004)(35.499367888748424, 31.642795376000002)(35.575221238938056, 43.72192603800001)(35.65107458912769, 27.393396777)(35.72692793931732, 28.785435215)(35.80278128950696, 37.559121351)(35.878634639696585, 40.603318076)(35.954487989886225, 13.292647238999999)(36.03034134007585, 11.036698282)(36.10619469026549, 10.207536447)(36.18204804045513, 20.439612770000004)(36.257901390644754, 24.544419762)(36.333754740834394, 37.349552808)(36.40960809102402, 36.310419443)(36.48546144121366, 13.523057421)(36.56131479140329, 13.425712984)(36.63716814159292, 31.439240472)(36.713021491782555, 35.376840486)(36.78887484197219, 26.221231248)(36.86472819216183, 36.718152493)(36.94058154235145, 45.879037773)(37.01643489254109, 24.565530154)(37.092288242730724, 47.847609674)(37.16814159292036, 28.970801123999998)(37.24399494310999, 47.994375051000006)(37.31984829329962, 32.595111456)(37.395701643489254, 48.275602649999996)(37.47155499367889, 24.651475900999998)(37.547408343868526, 42.800229984)(37.62326169405816, 37.417295894)(37.69911504424779, 35.886255974)(37.77496839443742, 45.100026407)(37.85082174462706, 34.546813036)(37.92667509481669, 10.379352695)(38.00252844500633, 40.107780833999996)(38.07838179519595, 46.045554471)(38.15423514538559, 10.335056)(38.23008849557523, 13.338455352)(38.305941845764856, 10.915015465)(38.381795195954496, 13.738800175999998)(38.45764854614412, 44.315136685)(38.53350189633376, 39.571959678)(38.60935524652339, 17.306713364)(38.685208596713025, 48.572102262)(38.76106194690266, 27.596359207)(38.83691529709229, 46.31166161)(38.91276864728193, 43.07616475)(38.988621997471554, 12.275290765)(39.064475347661194, 37.031520117)(39.140328697850826, 45.604332537000005)(39.21618204804046, 47.423165292)(39.29203539823009, 42.869069063)(39.36788874841972, 42.376291994999995)(39.443742098609356, 35.901691751)(39.519595448798995, 48.466661164)(39.59544879898863, 55.345591821999996)(39.67130214917826, 49.418396699)(39.74715549936789, 37.643355768)(39.823008849557525, 48.797425946999994)(39.89886219974716, 26.517701072)(39.97471554993679, 46.740806796)(40.05056890012643, 17.013213479)(40.126422250316054, 13.051710028)(40.20227560050569, 14.446074713999998)(40.278128950695326, 35.314979664999996)(40.35398230088496, 11.912586516000001)(40.4298356510746, 27.49468325)(40.50568900126422, 13.381026151)(40.58154235145386, 14.52736375)(40.65739570164349, 11.804645829)(40.73324905183313, 14.683737018)(40.80910240202276, 44.217729002999995)(40.88495575221239, 28.101180176999996)(40.96080910240203, 38.487082414999996)(41.036662452591656, 24.468342697)(41.112515802781296, 44.913235291)(41.18836915297093, 51.364764021999996)(41.26422250316056, 36.418798596)(41.34007585335019, 37.36714012)(41.415929203539825, 37.037105636)(41.49178255372946, 11.415349281000001)(41.5676359039191, 35.22666987)(41.64348925410873, 34.210375422)(41.71934260429836, 36.603196481000005)(41.795195954487994, 51.647156725)(41.871049304677626, 47.052593459)(41.94690265486726, 24.410294358)(42.02275600505689, 37.002161091)(42.09860935524653, 46.961218932)(42.174462705436156, 36.877562477)(42.250316055625795, 42.567493801)(42.32616940581543, 42.958065461)(42.40202275600506, 32.141594517)(42.4778761061947, 36.030552915)(42.553729456384325, 54.223001846)(42.629582806573964, 46.329142321)(42.70543615676359, 35.203635921)(42.78128950695323, 35.248361091)(42.85714285714286, 35.057769584999996)(42.932996207332494, 36.061995231)(43.00884955752213, 38.475426354)(43.08470290771176, 37.266592247)(43.1605562579014, 37.726658613)(43.23640960809102, 45.834465199)(43.31226295828066, 31.462300931999998)(43.388116308470295, 11.659066736000002)(43.46396965865993, 19.378544206999997)(43.53982300884956, 11.853761063999999)(43.61567635903919, 12.255462099)(43.69152970922883, 41.570619387)(43.767383059418464, 10.644857111999999)(43.843236409608096, 33.42600069)(43.91908975979773, 47.096223391)(43.99494310998736, 12.359837091000001)(44.07079646017699, 9.217870645)(44.14664981036663, 11.453741841)(44.22250316055626, 38.057212883)(44.2983565107459, 41.921212823)(44.37420986093552, 48.508511043)(44.45006321112516, 41.556356414999996)(44.5259165613148, 47.442099275000004)(44.60176991150443, 36.909227645)(44.677623261694066, 42.006602621000006)(44.75347661188369, 25.130625747000003)(44.82932996207333, 43.423909185999996)(44.90518331226296, 21.572507300999998)(44.981036662452595, 26.826889591)(45.05689001264223, 36.481006352)(45.13274336283186, 15.676959691)(45.2085967130215, 32.551410428)(45.284450063211125, 34.475189895999996)(45.360303413400764, 33.568934746000004)(45.4361567635904, 12.412703371)(45.51201011378003, 12.52027053)(45.58786346396966, 21.808582458)(45.663716814159294, 44.239636008999994)(45.739570164348926, 26.808399292999997)(45.815423514538566, 47.429472671)(45.8912768647282, 48.602625581)(45.96713021491783, 46.358940701)(46.04298356510746, 50.209923614000004)(46.118836915297095, 49.355537082)(46.19469026548673, 43.325253802)(46.27054361567636, 45.634963259)(46.346396965866, 48.528861094)(46.422250316055624, 31.750020604999996)(46.498103666245264, 54.600763517999994)(46.573957016434896, 27.407536631)(46.64981036662453, 50.36074184900001)(46.72566371681417, 48.21430916)(46.80151706700379, 47.988336005)(46.87737041719343, 47.732785027000006)(46.95322376738306, 53.984703994)(47.0290771175727, 40.10502453)(47.10493046776233, 48.410571522000005)(47.18078381795196, 42.389543133)(47.2566371681416, 42.598583739000006)(47.33249051833123, 50.489029474000006)(47.408343868520866, 54.903394453000004)(47.4841972187105, 11.207132256000001)(47.56005056890013, 29.34023494)(47.63590391908976, 12.23222155)(47.711757269279396, 10.317495298999999)(47.78761061946903, 10.697335836999999)(47.86346396965867, 29.772607783999998)(47.9393173198483, 31.279288487)(48.01517067003793, 27.066672350999998)(48.091024020227565, 17.049427677)(48.1668773704172, 52.223492568000005)(48.24273072060683, 22.221105502)(48.31858407079646, 14.312984647)(48.3944374209861, 41.22271851)(48.470290771175726, 49.869765959)(48.546144121365366, 33.886628164)(48.621997471555, 36.435101514)(48.69785082174463, 15.810660075000001)(48.77370417193427, 47.806312563000006)(48.849557522123895, 41.920444599999996)(48.925410872313535, 39.876000241999996)(49.00126422250316, 40.748234528999994)(49.0771175726928, 44.467726555)(49.15297092288243, 49.138641119999996)(49.228824273072064, 43.468316724)(49.3046776232617, 34.868080692)(49.38053097345133, 44.126059166999994)(49.45638432364097, 44.226760307)(49.53223767383059, 44.675567722)(49.60809102402023, 36.778335219)(49.683944374209865, 48.902625586)(49.7597977243995, 51.710494865)(49.83565107458913, 50.165194878)(49.91150442477876, 42.706328328)(49.9873577749684, 39.991831876)(50.063211125158034, 46.331324613999996)(50.139064475347666, 41.628562453)(50.2149178255373, 46.199695954)(50.29077117572693, 38.948399789)(50.36662452591656, 40.30135278)(50.4424778761062, 46.728624294)(50.51833122629583, 35.680496923999996)(50.59418457648547, 43.744394451000005)(50.6700379266751, 29.337742184)(50.74589127686473, 15.850716991)(50.82174462705437, 14.072526296000001)(50.897597977244, 23.024447715999997)(50.97345132743364, 14.803415087000001)(51.04930467762326, 11.267276586)(51.1251580278129, 12.530593887000002)(51.20101137800253, 42.965654563)(51.276864728192166, 47.661511270999995)(51.352718078381805, 34.841098298)(51.42857142857143, 40.953464621)(51.50442477876107, 46.4206253)(51.580278128950695, 34.895177338)(51.656131479140335, 26.028812529)(51.73198482932997, 52.599351944999995)(51.8078381795196, 34.412143149)(51.88369152970923, 51.87232607)(51.959544879898864, 50.062393995)(52.035398230088504, 51.500293823999996)(52.111251580278136, 37.377965687999996)(52.18710493046777, 35.219916981)(52.2629582806574, 12.894293997999998)(52.33881163084703, 35.822459842)(52.414664981036665, 48.196821947000004)(52.4905183312263, 40.273543094)(52.56637168141593, 38.313428799)(52.64222503160557, 54.999413295000004)(52.7180783817952, 9.684398263)(52.793931731984834, 44.467346877000004)(52.86978508217447, 12.37072535)(52.9456384323641, 50.702047056)(53.02149178255374, 49.264062589000005)(53.097345132743364, 35.930149130000004)(53.173198482933, 45.574736029)(53.24905183312263, 42.130383352)(53.32490518331227, 47.061877236)(53.4007585335019, 41.506603075)(53.47661188369153, 48.035950656)(53.55246523388117, 52.71567735399999)(53.6283185840708, 47.884715022)(53.70417193426044, 52.943597319000006)(53.78002528445007, 46.845078)(53.8558786346397, 45.53013456000001)(53.931731984829334, 45.676274186)(54.007585335018966, 36.888956492)(54.0834386852086, 48.715466293)(54.15929203539824, 52.723298701000004)(54.23514538558787, 13.658228921)(54.3109987357775, 46.549746498)(54.386852085967135, 40.681126234000004)(54.46270543615677, 45.34447594699999)(54.5385587863464, 46.710164646)(54.61441213653603, 40.173566893)(54.69026548672567, 36.749622527)(54.7661188369153, 55.570440597)(54.841972187104936, 55.79610481899999)(54.91782553729457, 55.334056268999994)(54.9936788874842, 32.018275674)(55.06953223767384, 49.954146048)(55.145385587863466, 22.568024742000002)(55.221238938053105, 44.78999145)(55.29709228824273, 33.86367407)(55.37294563843237, 40.681006618)(55.448798988622, 45.321284213000006)(55.524652338811634, 44.928454634000005)(55.600505689001274, 46.364595037)(55.6763590391909, 35.267445049)(55.75221238938054, 49.311784052)(55.828065739570164, 35.765359648)(55.9039190897598, 57.24401403)(55.979772439949436, 37.011430597)(56.05562579013907, 44.969772918000004)(56.1314791403287, 45.087955027999996)(56.20733249051833, 46.320742861)(56.28318584070797, 28.587776552999998)(56.359039190897604, 41.810979380999996)(56.43489254108724, 47.396074741999996)(56.51074589127687, 48.615311414000004)(56.5865992414665, 48.643590261)(56.662452591656134, 35.255095392)(56.73830594184577, 52.221913494)(56.8141592920354, 45.407682333)(56.89001264222504, 28.904846442999997)(56.96586599241467, 39.339720841)(57.0417193426043, 48.390116196)(57.11757269279394, 49.267661792000006)(57.19342604298357, 49.463816174)(57.26927939317321, 42.636619599)(57.34513274336283, 47.753295767)(57.42098609355247, 41.769022762999995)(57.496839443742104, 39.623549753)(57.572692793931736, 43.361527931)(57.648546144121376, 50.293952442)(57.724399494311, 34.935647263999996)(57.80025284450064, 45.270803465)(57.876106194690266, 42.802928953)(57.951959544879905, 32.528004366)(58.02781289506954, 40.251413082)(58.10366624525917, 39.643905434)(58.1795195954488, 13.685064193)(58.255372945638435, 55.593484555)(58.331226295828074, 37.144820267)(58.407079646017706, 37.019186325999996)(58.48293299620734, 42.378798785)(58.55878634639697, 49.465116204)(58.6346396965866, 15.800314491)(58.710493046776236, 59.601582838)(58.78634639696587, 45.905500344)(58.8621997471555, 12.173553479999999)(58.93805309734514, 34.74986245)(59.01390644753477, 10.651460275)(59.089759797724405, 47.305382562)(59.16561314791404, 37.351077569000005)(59.24146649810367, 56.090255969)(59.31731984829331, 44.025522411000004)(59.393173198482934, 37.840270405)(59.46902654867257, 41.229299415)(59.5448798988622, 49.757049474)(59.62073324905184, 46.040302462999996)(59.69658659924148, 47.177436724)(59.7724399494311, 50.792262189999995)(59.84829329962074, 47.756561689)(59.92414664981037, 12.017192816)(60.0, 30.331720699)
        };
        \addplot [pattern=north east lines,pattern color=red] 
        fill between [
            of=A and B,soft clip={domain=0:800},
        ];
        \end{axis}
\end{tikzpicture}
\caption{Measuring instrument: Clamp(Win)} \label{fig:time_series_BinaryTrees_Workstation_ClampW}
\end{subfigure}
\begin{subfigure}[b]{0.49\linewidth}
    \begin{tikzpicture}[]
        \begin{axis}[ymax=120,
        xlabel={Time (Seconds)},
        ylabel={Energy Consumption (Joules)},
        ]
        \addplot[color=blue, mark=none,] coordinates { %% AVG value
        (0.0, 99.44048762010834)(0.07585335018963338, 99.51782094550832)(0.15170670037926676, 100.09295460200832)(0.22756005056890014, 99.73194373522502)(0.3034134007585335, 99.825320785125)(0.3792667509481669, 100.19179167825828)(0.45512010113780027, 100.52634431509166)(0.5309734513274336, 100.72192464209162)(0.606826801517067, 100.67573449831667)(0.6826801517067005, 100.25259136425832)(0.7585335018963338, 99.95173657338333)(0.8343868520859672, 100.30784432919164)(0.9102402022756005, 99.5536006581167)(0.986093552465234, 100.22054782428333)(1.0619469026548671, 100.6311540331)(1.1378002528445006, 100.91940604495002)(1.213653603034134, 100.82131262323333)(1.2895069532237675, 100.47326279840836)(1.365360303413401, 101.15359842076668)(1.4412136536030342, 101.37928553655003)(1.5170670037926677, 101.36879667845831)(1.5929203539823011, 101.00855208676667)(1.6687737041719344, 100.89430341447499)(1.7446270543615676, 101.13618495366671)(1.820480404551201, 100.88212061540831)(1.8963337547408345, 101.4984280066167)(1.972187104930468, 101.30042267539999)(2.0480404551201015, 101.26231632408333)(2.1238938053097343, 101.40234606495831)(2.199747155499368, 101.09563021703329)(2.275600505689001, 101.1286900436667)(2.351453855878635, 101.5013304378083)(2.427307206068268, 101.46597016320831)(2.503160556257902, 101.52184835492498)(2.579013906447535, 101.11381587715832)(2.6548672566371687, 101.13453141371669)(2.730720606826802, 101.13896071249997)(2.806573957016435, 101.07863197680003)(2.8824273072060684, 101.22595891200831)(2.9582806573957017, 101.26650151886663)(3.0341340075853354, 101.3310865006)(3.1099873577749686, 101.42140816751662)(3.1858407079646023, 101.33576907174167)(3.2616940581542355, 101.40146287316666)(3.3375474083438688, 101.52233214786663)(3.413400758533502, 101.45643982714165)(3.4892541087231352, 101.24111792657501)(3.565107458912769, 101.43661583300835)(3.640960809102402, 101.29380738399168)(3.716814159292036, 101.16939702756665)(3.792667509481669, 101.33427547728336)(3.8685208596713023, 101.05412735890835)(3.944374209860936, 101.00466586711666)(4.020227560050569, 101.31606810410831)(4.096080910240203, 101.19866953025831)(4.171934260429836, 101.13305191830831)(4.2477876106194685, 101.41128031300829)(4.323640960809103, 101.50922007913336)(4.399494310998736, 101.10583507903334)(4.47534766118837, 101.1252837053417)(4.551201011378002, 101.24039956431669)(4.6270543615676365, 101.20934993065006)(4.70290771175727, 100.91158159410001)(4.778761061946904, 100.8971938985833)(4.854614412136536, 100.4112848901)(4.9304677623261695, 100.8105090657584)(5.006321112515804, 101.05882090093324)(5.082174462705436, 100.56408275700826)(5.15802781289507, 101.06480156905835)(5.233881163084703, 101.0503599403)(5.309734513274337, 101.0031005890583)(5.38558786346397, 100.964938093175)(5.461441213653604, 100.63951564147497)(5.537294563843237, 100.46044465735835)(5.61314791403287, 100.76369354497498)(5.689001264222504, 101.08957563234164)(5.764854614412137, 101.17585150830003)(5.840707964601771, 100.9719171573583)(5.916561314791403, 100.80483671265834)(5.9924146649810375, 101.49137409526669)(6.068268015170671, 101.9215281777917)(6.144121365360304, 101.69339526031665)(6.219974715549937, 101.58312973627498)(6.29582806573957, 101.81436702484997)(6.371681415929205, 101.6778355744333)(6.447534766118837, 101.373976347225)(6.523388116308471, 100.38241180989165)(6.599241466498104, 97.68655872085833)(6.6750948166877375, 95.23875853662503)(6.750948166877371, 93.52666474410002)(6.826801517067004, 95.05049756185835)(6.902654867256638, 95.71077471373336)(6.9785082174462705, 95.83437002323335)(7.054361567635905, 101.17712229548336)(7.130214917825538, 104.13990726345001)(7.206068268015172, 105.79739146079996)(7.281921618204804, 106.45343087545834)(7.357774968394438, 106.45193928829167)(7.433628318584072, 106.97450561898337)(7.509481668773706, 96.90094748398336)(7.585335018963338, 97.05644695706668)(7.661188369152971, 96.8148519901667)(7.737041719342605, 97.03886117791659)(7.812895069532239, 97.33997484089998)(7.888748419721872, 97.12356497119995)(7.964601769911505, 97.02062071714163)(8.040455120101138, 97.12497464472496)(8.116308470290772, 97.23913196789165)(8.192161820480406, 97.59410360736665)(8.268015170670038, 97.5278661329917)(8.343868520859672, 97.28909834839168)(8.419721871049305, 97.88754873738336)(8.495575221238937, 98.00163018260834)(8.571428571428573, 97.48902213276666)(8.647281921618205, 97.3397360554)(8.72313527180784, 97.38026963668337)(8.798988621997472, 97.30818076879999)(8.874841972187106, 97.26479228819998)(8.95069532237674, 98.12993007161666)(9.026548672566372, 97.81259509170839)(9.102402022756005, 97.83657663850836)(9.178255372945639, 97.97459994927503)(9.254108723135273, 97.90441881264165)(9.329962073324905, 97.91304700040001)(9.40581542351454, 98.03777231819168)(9.481668773704172, 97.68289840796673)(9.557522123893808, 97.94868666948334)(9.63337547408344, 98.07981345803334)(9.709228824273072, 97.93571333363333)(9.785082174462707, 97.54677720226665)(9.860935524652339, 97.94548635283336)(9.936788874841973, 97.97231736261666)(10.012642225031607, 98.44137922454998)(10.08849557522124, 98.46922086086668)(10.164348925410872, 98.70919676293332)(10.240202275600508, 97.651654145475)(10.31605562579014, 97.81177286909997)(10.391908975979774, 97.9469498843167)(10.467762326169407, 98.2705889976333)(10.543615676359039, 98.10840506756666)(10.619469026548675, 97.96683106315002)(10.695322376738307, 97.71530950798336)(10.77117572692794, 97.77055298166668)(10.847029077117574, 97.80464848540001)(10.922882427307208, 97.79285923081662)(10.99873577749684, 97.58822185437499)(11.074589127686474, 97.943340935975)(11.150442477876107, 97.36215480603337)(11.22629582806574, 97.61685571839162)(11.302149178255375, 98.02265416374172)(11.378002528445007, 97.41925847420836)(11.453855878634641, 97.84143638410833)(11.529709228824274, 98.0800484545917)(11.605562579013906, 98.20017788692503)(11.681415929203542, 98.394727597225)(11.757269279393174, 98.52163974175829)(11.833122629582807, 98.36178067840834)(11.90897597977244, 99.03792675422504)(11.984829329962075, 98.78678856570004)(12.060682680151707, 98.79671714953336)(12.136536030341341, 98.85317242035832)(12.212389380530974, 98.97179089130005)(12.288242730720608, 98.82577560739998)(12.364096080910242, 97.99135037170836)(12.439949431099874, 98.81303515684164)(12.515802781289509, 98.78247304707504)(12.59165613147914, 98.80399950956665)(12.667509481668775, 98.54667660659167)(12.74336283185841, 98.43450053651667)(12.819216182048041, 98.16157061005002)(12.895069532237674, 99.70443352871668)(12.970922882427308, 100.02622558302497)(13.046776232616942, 100.2467368145833)(13.122629582806574, 100.3249909382833)(13.198482932996209, 100.146860017225)(13.274336283185841, 100.25434254339999)(13.350189633375475, 100.58141901694162)(13.42604298356511, 100.5345242995)(13.501896333754742, 100.69758032276664)(13.577749683944376, 100.25586460106666)(13.653603034134008, 100.08682285489999)(13.729456384323642, 100.03497697276663)(13.805309734513276, 100.13818898878334)(13.881163084702909, 99.73491311246669)(13.957016434892541, 100.20642627182502)(14.032869785082175, 98.41490493809992)(14.10872313527181, 97.8153372781584)(14.184576485461443, 95.65929043133332)(14.260429835651076, 93.59595293726669)(14.336283185840708, 93.05149536357496)(14.412136536030344, 95.20787791551669)(14.487989886219976, 95.44207059545005)(14.563843236409609, 98.4653851280834)(14.639696586599243, 101.92159472026665)(14.715549936788875, 104.59832158768337)(14.79140328697851, 105.7754865372)(14.867256637168143, 105.32088722467498)(14.943109987357776, 106.29452132330827)(15.018963337547412, 96.73902677802504)(15.094816687737044, 96.7621996523833)(15.170670037926676, 97.03250796681665)(15.24652338811631, 96.96346919496669)(15.322376738305943, 96.84605632722499)(15.398230088495575, 96.63193846493338)(15.47408343868521, 97.030608573925)(15.549936788874842, 96.829785903875)(15.625790139064478, 96.41164859845831)(15.701643489254112, 96.96837081254166)(15.777496839443744, 96.97540064460831)(15.853350189633378, 97.2127292261083)(15.92920353982301, 97.01018753605831)(16.005056890012643, 96.86866376970836)(16.080910240202275, 96.89235375464169)(16.15676359039191, 96.64181669484998)(16.232616940581543, 96.92053483361664)(16.30847029077118, 96.89200977119161)(16.38432364096081, 96.75105651851665)(16.460176991150444, 96.71211579484164)(16.536030341340076, 96.89332339843332)(16.611883691529712, 97.0108404010417)(16.687737041719345, 96.57907733667498)(16.763590391908977, 96.49816610629999)(16.83944374209861, 96.66233963524996)(16.91529709228824, 96.8850442407583)(16.991150442477874, 96.62290989627496)(17.067003792667514, 96.45242172338332)(17.142857142857146, 96.92152309010004)(17.21871049304678, 96.80629254512502)(17.29456384323641, 97.15752533359166)(17.370417193426043, 96.56567666780832)(17.44627054361568, 96.61112614084163)(17.52212389380531, 96.90054171739162)(17.597977243994944, 96.82557483548332)(17.673830594184576, 96.92427860125837)(17.749683944374212, 96.92671238963331)(17.825537294563844, 97.31311028690834)(17.90139064475348, 96.82149702273331)(17.977243994943112, 96.67815911969166)(18.053097345132745, 96.90150477296667)(18.128950695322377, 97.23764744663337)(18.20480404551201, 96.85339107273326)(18.280657395701645, 97.06708685945)(18.356510745891278, 97.26332078415)(18.432364096080914, 97.04705074595832)(18.508217446270546, 97.27199814929166)(18.58407079646018, 96.72747204128332)(18.65992414664981, 97.0736341304333)(18.735777496839447, 97.11469085380001)(18.81163084702908, 97.26008671591669)(18.88748419721871, 97.33228181248336)(18.963337547408344, 97.28515815648332)(19.039190897597976, 96.83599716709166)(19.115044247787615, 97.07242449985)(19.190897597977248, 97.42221957896665)(19.26675094816688, 97.47236728678331)(19.342604298356513, 97.08933741015002)(19.418457648546145, 97.06824768652501)(19.494310998735777, 97.19282954149168)(19.570164348925413, 96.83018795085835)(19.646017699115045, 96.4160859512167)(19.721871049304678, 97.04069210084167)(19.797724399494314, 97.52629426977502)(19.873577749683946, 97.79678726419165)(19.94943109987358, 97.80147122884999)(20.025284450063214, 97.64731776420003)(20.101137800252847, 98.07586935107499)(20.17699115044248, 97.83824541019997)(20.25284450063211, 98.23100592410833)(20.328697850821744, 98.47759434148333)(20.40455120101138, 98.40217840729997)(20.480404551201016, 98.22624583782503)(20.556257901390648, 98.23433465508334)(20.63211125158028, 98.71195136574163)(20.707964601769913, 98.98722450036666)(20.78381795195955, 98.83574751108335)(20.85967130214918, 99.55200460359994)(20.935524652338813, 99.48999821915835)(21.011378002528446, 99.5374723810584)(21.087231352718078, 98.93788369761673)(21.163084702907714, 98.65881752865833)(21.23893805309735, 98.61342849505)(21.314791403286982, 98.81018515884999)(21.390644753476614, 98.76260030153331)(21.466498103666247, 98.80895673745835)(21.54235145385588, 98.34556029365834)(21.61820480404551, 97.91094403818336)(21.694058154235147, 97.59865667718336)(21.76991150442478, 96.83192113711668)(21.845764854614416, 96.48476302624164)(21.921618204804048, 95.03181178093335)(21.99747155499368, 95.43382996212497)(22.073324905183316, 97.73825510534165)(22.14917825537295, 99.46146653390832)(22.22503160556258, 101.30128625980834)(22.300884955752213, 102.97522142275834)(22.376738305941846, 104.20233105759172)(22.45259165613148, 104.48527369744167)(22.528445006321114, 97.32379246830835)(22.60429835651075, 97.93492283739997)(22.680151706700382, 97.94819427570833)(22.756005056890015, 97.83605897999168)(22.831858407079647, 98.09140836537499)(22.907711757269283, 97.62350563965836)(22.983565107458915, 97.68495192635001)(23.059418457648547, 97.692989776)(23.13527180783818, 97.79621476666668)(23.211125158027812, 97.8761913005)(23.286978508217448, 97.80903564199171)(23.362831858407084, 98.19078171876666)(23.438685208596716, 98.76603417875835)(23.51453855878635, 98.87921030295004)(23.59039190897598, 98.87572618914996)(23.666245259165613, 99.11692672239165)(23.74209860935525, 98.74098748380003)(23.81795195954488, 98.27328782385831)(23.893805309734514, 99.0820036221)(23.96965865992415, 99.60480177942496)(24.045512010113782, 100.01305630974164)(24.121365360303415, 99.74990169089165)(24.19721871049305, 99.65002576204998)(24.273072060682683, 99.95151550843326)(24.348925410872315, 99.69047141138331)(24.424778761061948, 99.72587778729995)(24.50063211125158, 99.90663265514159)(24.576485461441216, 99.84257588776669)(24.65233881163085, 99.83798032353329)(24.728192161820484, 99.23431087321661)(24.804045512010116, 99.09981951775835)(24.87989886219975, 99.4214776754)(24.95575221238938, 99.72449262295831)(25.031605562579017, 99.21940642017503)(25.10745891276865, 99.60008086442504)(25.18331226295828, 100.290208926925)(25.259165613147914, 100.46610790455837)(25.33501896333755, 100.39081161669169)(25.410872313527186, 100.49473225835)(25.48672566371682, 100.26087088453333)(25.56257901390645, 99.9636144364334)(25.638432364096083, 100.58584483682505)(25.714285714285715, 100.7576143201)(25.790139064475348, 100.61465465198337)(25.865992414664984, 100.62910790938339)(25.941845764854616, 100.97366137270836)(26.017699115044252, 100.78840710488332)(26.093552465233884, 100.59245180143336)(26.169405815423517, 100.57094875820836)(26.24525916561315, 100.78837711577495)(26.321112515802785, 100.54276928435836)(26.396965865992417, 100.43973045095002)(26.47281921618205, 100.7658034470333)(26.548672566371682, 100.65774115376668)(26.624525916561314, 100.60124278681674)(26.70037926675095, 100.58136062933337)(26.776232616940586, 100.36465840290838)(26.85208596713022, 100.43606860917498)(26.92793931731985, 100.89360853110828)(27.003792667509483, 100.86811817459997)(27.07964601769912, 100.49607774339174)(27.15549936788875, 100.29919885371666)(27.231352718078384, 100.15163529179162)(27.307206068268016, 100.0149043791667)(27.38305941845765, 100.12918942243336)(27.458912768647284, 100.15075490760836)(27.53476611883692, 100.31178943159999)(27.610619469026553, 101.14229879590832)(27.686472819216185, 100.9869494133417)(27.762326169405817, 100.69718637256668)(27.83817951959545, 100.94918263584168)(27.914032869785082, 101.0257350270667)(27.989886219974718, 101.07868054748333)(28.06573957016435, 100.6627399031333)(28.141592920353986, 100.45917359941672)(28.21744627054362, 100.9853064044334)(28.29329962073325, 100.47047167550002)(28.369152970922887, 100.47728477411663)(28.44500632111252, 100.36098513984167)(28.52085967130215, 100.2820955774583)(28.596713021491784, 100.881689834575)(28.672566371681416, 100.9897507553667)(28.748419721871052, 100.83128517390001)(28.824273072060688, 100.90708169883335)(28.90012642225032, 100.79909488331667)(28.975979772439953, 101.28075305189166)(29.051833122629585, 100.95013752832503)(29.127686472819217, 98.63946031683334)(29.203539823008853, 98.19702972350002)(29.279393173198486, 97.10072687104997)(29.355246523388118, 96.59222779485007)(29.43109987357775, 96.7541004837083)(29.506953223767386, 97.63708980646668)(29.58280657395702, 101.10556990654999)(29.658659924146654, 102.20803813401666)(29.734513274336287, 104.74843915585001)(29.81036662452592, 105.4150393142667)(29.88621997471555, 106.02686783446666)(29.962073324905184, 105.39016877685835)(30.037926675094823, 95.90763650216667)(30.113780025284452, 96.86307994410002)(30.189633375474088, 97.6072787494667)(30.265486725663717, 97.3766869685667)(30.341340075853353, 96.90101243037499)(30.417193426042985, 97.01569101233336)(30.49304677623262, 97.27590834874998)(30.568900126422257, 97.58666540367501)(30.644753476611886, 97.40918288601671)(30.72060682680152, 97.57928862391664)(30.79646017699115, 97.56377214492495)(30.872313527180786, 97.68715605981666)(30.94816687737042, 97.54050098126665)(31.024020227560055, 97.41776015274169)(31.099873577749683, 97.40643028988332)(31.17572692793932, 97.35136117683335)(31.251580278128955, 97.28032686454165)(31.327433628318587, 97.39896767121664)(31.403286978508223, 98.07442149058329)(31.479140328697852, 97.84780627244162)(31.554993678887488, 97.94688442949995)(31.630847029077117, 97.563894736125)(31.706700379266756, 97.31416215505001)(31.782553729456385, 97.1968610752167)(31.85840707964602, 97.59830162526666)(31.934260429835657, 97.55591584727502)(32.010113780025286, 97.71786031705832)(32.085967130214925, 97.3718275544833)(32.16182048040455, 97.80764650429167)(32.23767383059419, 97.96065193200836)(32.31352718078382, 97.42369971273332)(32.389380530973455, 97.62611617370835)(32.46523388116309, 97.76469159338329)(32.54108723135272, 97.79117537559162)(32.61694058154236, 97.5249107763)(32.692793931731984, 97.49979721023335)(32.76864728192162, 97.64026879745832)(32.844500632111256, 97.85703817350002)(32.92035398230089, 97.89312187479999)(32.99620733249052, 97.66072868109168)(33.07206068268015, 97.50289311924172)(33.147914032869785, 97.43987216544997)(33.223767383059425, 97.95234931903336)(33.29962073324906, 97.81012148329172)(33.37547408343869, 97.5393239225501)(33.45132743362832, 97.20446393513332)(33.527180783817954, 97.50833499037502)(33.603034134007586, 97.31093490612503)(33.67888748419722, 97.39743525585835)(33.75474083438686, 97.20699136153333)(33.83059418457648, 97.7704320691167)(33.90644753476612, 97.46410962169172)(33.98230088495575, 97.46101896155002)(34.05815423514539, 97.97304460323332)(34.13400758533503, 97.91948509222506)(34.20986093552465, 97.72421300075)(34.28571428571429, 97.1099347243583)(34.36156763590392, 97.06822608513329)(34.43742098609356, 97.09082768471664)(34.51327433628319, 97.42072504151669)(34.58912768647282, 96.91112089696664)(34.664981036662454, 96.55362690957496)(34.740834386852086, 97.03549998208328)(34.816687737041725, 96.63446118745837)(34.89254108723136, 97.54272283483331)(34.96839443742099, 97.50895053076671)(35.04424778761062, 97.53708324816667)(35.120101137800255, 97.76077937534164)(35.19595448798989, 97.81198797334169)(35.27180783817953, 97.92994449950001)(35.34766118836915, 97.45124099934168)(35.42351453855879, 97.11876282524162)(35.499367888748424, 97.52738297096667)(35.575221238938056, 97.29107689766664)(35.65107458912769, 97.49976483857498)(35.72692793931732, 96.83922744132498)(35.80278128950696, 97.94518940926663)(35.878634639696585, 97.9002792527)(35.954487989886225, 97.65941349551667)(36.03034134007585, 97.811012977075)(36.10619469026549, 98.265475890175)(36.18204804045513, 97.83549254387505)(36.257901390644754, 97.80861231974995)(36.333754740834394, 97.98753036882499)(36.40960809102402, 98.4562679338084)(36.48546144121366, 98.46842732979168)(36.56131479140329, 97.44257457951667)(36.63716814159292, 96.30940565556664)(36.713021491782555, 95.59584742008336)(36.78887484197219, 95.67674423709997)(36.86472819216183, 96.11086573824997)(36.94058154235145, 96.86610711640832)(37.01643489254109, 96.35318657011666)(37.092288242730724, 100.89659483539167)(37.16814159292036, 102.593460600775)(37.24399494310999, 103.76293734186665)(37.31984829329962, 103.204222545025)(37.395701643489254, 104.14977707240833)(37.47155499367889, 104.23217324965836)(37.547408343868526, 99.0409585268667)(37.62326169405816, 98.75834523650833)(37.69911504424779, 98.8467898443)(37.77496839443742, 98.76638099014167)(37.85082174462706, 98.69736290572502)(37.92667509481669, 97.91217254706665)(38.00252844500633, 98.39090808602504)(38.07838179519595, 98.18655809951666)(38.15423514538559, 98.80040909565)(38.23008849557523, 99.00332217747494)(38.305941845764856, 98.90285178158338)(38.381795195954496, 98.87638388803333)(38.45764854614412, 99.14780282335002)(38.53350189633376, 98.78475656627504)(38.60935524652339, 98.50033331885)(38.685208596713025, 98.99340745363332)(38.76106194690266, 98.8945532839)(38.83691529709229, 98.51661959672505)(38.91276864728193, 98.91748827984163)(38.988621997471554, 99.30506463045002)(39.064475347661194, 99.41957772642499)(39.140328697850826, 99.2800107638916)(39.21618204804046, 99.32329488644999)(39.29203539823009, 99.20272928821667)(39.36788874841972, 99.46273159765839)(39.443742098609356, 99.41204000288332)(39.519595448798995, 99.61366672389168)(39.59544879898863, 99.68504204259166)(39.67130214917826, 99.16862024831667)(39.74715549936789, 99.54131325879997)(39.823008849557525, 99.027123300975)(39.89886219974716, 98.91157498152498)(39.97471554993679, 99.07910205198333)(40.05056890012643, 99.51492784274998)(40.126422250316054, 99.03970969517498)(40.20227560050569, 98.53563918135)(40.278128950695326, 99.00429704012505)(40.35398230088496, 99.71551384770831)(40.4298356510746, 99.68980026897499)(40.50568900126422, 99.55592479601661)(40.58154235145386, 99.92152121381672)(40.65739570164349, 99.83538293872499)(40.73324905183313, 99.87556860809998)(40.80910240202276, 99.61208170839171)(40.88495575221239, 99.23411053485002)(40.96080910240203, 99.72265186154169)(41.036662452591656, 100.02399636736666)(41.112515802781296, 99.84763856851669)(41.18836915297093, 100.08006316664996)(41.26422250316056, 99.97228801602496)(41.34007585335019, 100.27069975208335)(41.415929203539825, 99.81755900016668)(41.49178255372946, 100.1370992798)(41.5676359039191, 100.13514895910832)(41.64348925410873, 99.82817251870834)(41.71934260429836, 101.03974448657497)(41.795195954487994, 100.06678893066665)(41.871049304677626, 100.84876170181673)(41.94690265486726, 100.99978411991665)(42.02275600505689, 100.881510805175)(42.09860935524653, 100.67952122803334)(42.174462705436156, 100.68598248598333)(42.250316055625795, 100.89761096764171)(42.32616940581543, 101.0457516849667)(42.40202275600506, 100.59993344284166)(42.4778761061947, 100.72047396428334)(42.553729456384325, 100.97207459784165)(42.629582806573964, 101.14286580297501)(42.70543615676359, 101.06688603644164)(42.78128950695323, 100.80342243255001)(42.85714285714286, 101.21816554414168)(42.932996207332494, 101.0999046401833)(43.00884955752213, 101.09649632557502)(43.08470290771176, 101.1972632880583)(43.1605562579014, 101.56052982904173)(43.23640960809102, 101.28304401396663)(43.31226295828066, 101.31027086138337)(43.388116308470295, 101.35772400555003)(43.46396965865993, 101.11887778302498)(43.53982300884956, 101.600422136)(43.61567635903919, 101.2052413562083)(43.69152970922883, 101.03774403655831)(43.767383059418464, 101.01506030483331)(43.843236409608096, 101.26110250330834)(43.91908975979773, 101.492322561)(43.99494310998736, 101.40007145340834)(44.07079646017699, 100.62481892221012)(44.14664981036663, 99.95967201421008)(44.22250316055626, 98.55072561497477)(44.2983565107459, 97.51759790076474)(44.37420986093552, 97.03044956436979)(44.45006321112516, 97.0354556999916)(44.5259165613148, 98.6304299704538)(44.60176991150443, 98.37703116774793)(44.677623261694066, 101.90550653068067)(44.75347661188369, 104.43586715889076)(44.82932996207333, 103.9796631946218)(44.90518331226296, 105.76795766199162)(44.981036662452595, 106.14547087887395)(45.05689001264223, 97.24902172291179)(45.13274336283186, 97.679671113549)(45.2085967130215, 97.90511156735296)(45.284450063211125, 98.07118730551963)(45.360303413400764, 97.49193923385293)(45.4361567635904, 97.59075854400979)(45.51201011378003, 97.91818428269609)(45.58786346396966, 97.80263367040192)(45.663716814159294, 97.8671586352255)(45.739570164348926, 98.16273999749014)(45.815423514538566, 97.84255635578432)(45.8912768647282, 97.20787558474512)(45.96713021491783, 97.77374618180392)(46.04298356510746, 97.88377619477455)(46.118836915297095, 97.6908997811765)(46.19469026548673, 98.17881348991178)(46.27054361567636, 98.37167779510891)(46.346396965866, 98.29178043079209)(46.422250316055624, 97.98970799068312)(46.498103666245264, 98.07058993558415)(46.573957016434896, 98.05371358370294)(46.64981036662453, 98.43054021038614)(46.72566371681417, 98.33998473817822)(46.80151706700379, 98.23293900211878)(46.87737041719343, 98.72077841822777)(46.95322376738306, 99.12472298513862)(47.0290771175727, 99.15450824857427)(47.10493046776233, 98.82294231487126)(47.18078381795196, 98.63737090436632)(47.2566371681416, 98.28673873747523)(47.33249051833123, 98.38537725267325)(47.408343868520866, 98.7481940810891)(47.4841972187105, 99.05794033763365)(47.56005056890013, 98.71844877160395)(47.63590391908976, 98.65071291560398)(47.711757269279396, 99.38095737375248)(47.78761061946903, 99.40209868142003)(47.86346396965867, 99.12417335523007)(47.9393173198483, 99.12368129813999)(48.01517067003793, 99.12897193122998)(48.091024020227565, 99.14712253817999)(48.1668773704172, 99.36820609363998)(48.24273072060683, 99.23892630434)(48.31858407079646, 99.18363684614002)(48.3944374209861, 98.99354114514996)(48.470290771175726, 99.36226037956997)(48.546144121365366, 99.22081303409)(48.621997471555, 99.39589579369002)(48.69785082174463, 99.00980785936004)(48.77370417193427, 98.77054593832004)(48.849557522123895, 99.25696196710004)(48.925410872313535, 99.40737305422003)(49.00126422250316, 99.41589435141003)(49.0771175726928, 99.39607408499002)(49.15297092288243, 99.56922594765001)(49.228824273072064, 98.81527619501996)(49.3046776232617, 99.22828802763)(49.38053097345133, 98.93162341138998)(49.45638432364097, 99.42980508248)(49.53223767383059, 99.18865585411999)(49.60809102402023, 98.99601703878001)(49.683944374209865, 98.88204569820996)(49.7597977243995, 98.41344429026263)(49.83565107458913, 99.12128829649491)(49.91150442477876, 99.5200883719899)(49.9873577749684, 99.06314054094946)(50.063211125158034, 98.6049137771818)(50.139064475347666, 99.01570164060607)(50.2149178255373, 98.81401092113127)(50.29077117572693, 98.00997226013264)(50.36662452591656, 98.58091118654079)(50.4424778761062, 98.75843955913273)(50.51833122629583, 100.088035920102)(50.59418457648547, 99.69230452145361)(50.6700379266751, 99.21056590143299)(50.74589127686473, 99.71290244783332)(50.82174462705437, 99.91426344506249)(50.897597977244, 99.7492951179167)(50.97345132743364, 99.56190828634732)(51.04930467762326, 99.5136311440105)(51.1251580278129, 100.10954911250526)(51.20101137800253, 99.86436026601052)(51.276864728192166, 99.42137226398944)(51.352718078381805, 99.08835882482103)(51.42857142857143, 99.59539812757893)(51.50442477876107, 99.84654265605263)(51.580278128950695, 98.86128793690527)(51.656131479140335, 98.14347721597893)(51.73198482932997, 98.29089752029473)(51.8078381795196, 94.96678296257899)(51.88369152970923, 96.02097808322101)(51.959544879898864, 96.86648683329473)(52.035398230088504, 97.18658697081916)(52.111251580278136, 100.47954052368821)(52.18710493046777, 102.00892146111828)(52.2629582806574, 103.13471506178496)(52.33881163084703, 104.02147805023658)(52.414664981036665, 104.29601572806456)(52.4905183312263, 104.51796733203227)(52.56637168141593, 106.22622827200001)(52.64222503160557, 107.118803528)(52.7180783817952, 100.17219821399999)(52.793931731984834, 100.46277358699999)(52.86978508217447, 103.173278149)(52.9456384323641, 106.22820269900001)(53.02149178255374, 105.39075886699999)(53.097345132743364, 106.762209981)(53.173198482933, 106.43230039800001)(53.24905183312263, 104.740957451)(53.32490518331227, 104.68585039800001)(53.4007585335019, 105.888446284)(53.47661188369153, 107.838189566)(53.55246523388117, 104.809019128)(53.6283185840708, 106.080161163)(53.70417193426044, 105.82346770499998)(53.78002528445007, 105.44551113)(53.8558786346397, 106.481421832)(53.931731984829334, 105.96592037100001)(54.007585335018966, 105.76660781499999)(54.0834386852086, 104.346775744)(54.15929203539824, 107.02436949000001)(54.23514538558787, 107.89337250499999)(54.3109987357775, 106.37539989700001)(54.386852085967135, 105.390465885)(54.46270543615677, 106.017264244)(54.5385587863464, 104.123852913)(54.61441213653603, 104.39643271899999)(54.69026548672567, 106.608766454)(54.7661188369153, 105.313244118)(54.841972187104936, 105.16393120000001)(54.91782553729457, 105.828586981)(54.9936788874842, 104.512047729)(55.06953223767384, 104.56099431300001)(55.145385587863466, 106.092773319)(55.221238938053105, 105.878032949)(55.29709228824273, 105.222117459)(55.37294563843237, 106.195434534)(55.448798988622, 104.888747599)(55.524652338811634, 106.310285841)(55.600505689001274, 106.41952545000001)(55.6763590391909, 106.43827463099998)(55.75221238938054, 105.52375376)(55.828065739570164, 105.176745868)(55.9039190897598, 106.201585809)(55.979772439949436, 101.473788235)(56.05562579013907, 105.28739829099999)(56.1314791403287, 101.692329336)(56.20733249051833, 108.128393582)(56.28318584070797, 105.861125455)(56.359039190897604, 104.68534827699999)(56.43489254108724, 108.749914802)(56.51074589127687, 105.708160139)(56.5865992414665, 105.38128665100001)(56.662452591656134, 105.54211256900001)(56.73830594184577, 105.312356191)(56.8141592920354, 105.308716076)(56.89001264222504, 105.548327565)(56.96586599241467, 105.941258558)(57.0417193426043, 106.037746311)(57.11757269279394, 105.548458513)(57.19342604298357, 105.437408625)(57.26927939317321, 105.794552722)(57.34513274336283, 110.468799903)(57.42098609355247, 105.92084124799999)(57.496839443742104, 103.048106608)(57.572692793931736, 105.69496253300001)(57.648546144121376, 105.423128311)(57.724399494311, 105.827116946)(57.80025284450064, 104.342480341)(57.876106194690266, 105.442508226)(57.951959544879905, 107.419906338)(58.02781289506954, 106.712287829)(58.10366624525917, 106.108780208)(58.1795195954488, 104.593476701)(58.255372945638435, 107.16351327699999)(58.331226295828074, 104.54839941499999)(58.407079646017706, 105.06910306)(58.48293299620734, 103.70350178700001)(58.55878634639697, 103.22558217099999)(58.6346396965866, 103.31456894200001)(58.710493046776236, 104.751774386)(58.78634639696587, 105.825144828)(58.8621997471555, 106.46723717500001)(58.93805309734514, 107.69228284)(59.01390644753477, 104.41576853000001)(59.089759797724405, 104.538384394)(59.16561314791404, 104.82516561700001)(59.24146649810367, 103.590742078)(59.31731984829331, 104.08612032900001)(59.393173198482934, 104.136943647)(59.46902654867257, 104.571130311)(59.5448798988622, 102.750323283)(59.62073324905184, 103.620999086)(59.69658659924148, 104.797008661)(59.7724399494311, 105.107100493)(59.84829329962074, 102.649525665)(59.92414664981037, 103.68518040699999)(60.0, 103.207670183)
        };
        \addplot[color=blue, mark=none,name path=A] coordinates { %% MAX value
        (0.0, 118.541488585)(0.07585335018963338, 118.18682194300001)(0.15170670037926676, 118.855736051)(0.22756005056890014, 118.224936417)(0.3034134007585335, 117.95843210999999)(0.3792667509481669, 118.820326209)(0.45512010113780027, 119.931117298)(0.5309734513274336, 119.992467826)(0.606826801517067, 118.56247263)(0.6826801517067005, 118.89125723299999)(0.7585335018963338, 118.485046655)(0.8343868520859672, 119.057426929)(0.9102402022756005, 118.387266096)(0.986093552465234, 118.015934254)(1.0619469026548671, 118.22559945)(1.1378002528445006, 120.39920002000001)(1.213653603034134, 119.537966665)(1.2895069532237675, 119.360440031)(1.365360303413401, 119.282590061)(1.4412136536030342, 119.521778948)(1.5170670037926677, 119.225692864)(1.5929203539823011, 118.267474045)(1.6687737041719344, 119.46114053100001)(1.7446270543615676, 119.22545381799999)(1.820480404551201, 118.40631146999999)(1.8963337547408345, 120.168106987)(1.972187104930468, 119.880098473)(2.0480404551201015, 120.09587937299999)(2.1238938053097343, 120.85551579400001)(2.199747155499368, 120.47151916800001)(2.275600505689001, 119.148461183)(2.351453855878635, 120.78854573599999)(2.427307206068268, 118.69812183600001)(2.503160556257902, 119.79442950299999)(2.579013906447535, 120.72586820699999)(2.6548672566371687, 120.73643454)(2.730720606826802, 119.164676715)(2.806573957016435, 119.13930783699999)(2.8824273072060684, 118.66564200399999)(2.9582806573957017, 119.588254678)(3.0341340075853354, 118.320765507)(3.1099873577749686, 120.086092455)(3.1858407079646023, 119.124623241)(3.2616940581542355, 119.599349632)(3.3375474083438688, 120.076446447)(3.413400758533502, 119.395875995)(3.4892541087231352, 118.54336306)(3.565107458912769, 120.603612241)(3.640960809102402, 118.121508141)(3.716814159292036, 119.18983555)(3.792667509481669, 118.92206531299999)(3.8685208596713023, 118.01389802599999)(3.944374209860936, 118.977306452)(4.020227560050569, 119.138633739)(4.096080910240203, 119.691409846)(4.171934260429836, 120.34178662800001)(4.2477876106194685, 121.13471319300001)(4.323640960809103, 121.833844047)(4.399494310998736, 118.756644105)(4.47534766118837, 119.832262254)(4.551201011378002, 118.813645173)(4.6270543615676365, 120.123597912)(4.70290771175727, 117.97624933099999)(4.778761061946904, 119.28066876899999)(4.854614412136536, 118.463709917)(4.9304677623261695, 120.009674273)(5.006321112515804, 119.42568461600001)(5.082174462705436, 118.72839854700001)(5.15802781289507, 117.738224729)(5.233881163084703, 119.78028426899999)(5.309734513274337, 119.652789261)(5.38558786346397, 118.607162331)(5.461441213653604, 119.772323231)(5.537294563843237, 119.383066885)(5.61314791403287, 118.558757832)(5.689001264222504, 119.638780994)(5.764854614412137, 118.822117464)(5.840707964601771, 118.986656834)(5.916561314791403, 119.609830649)(5.9924146649810375, 120.013662882)(6.068268015170671, 119.165720579)(6.144121365360304, 118.105377102)(6.219974715549937, 119.35967531099999)(6.29582806573957, 119.26960295800001)(6.371681415929205, 118.57233288399999)(6.447534766118837, 121.75667454699999)(6.523388116308471, 120.218632355)(6.599241466498104, 119.44033730199999)(6.6750948166877375, 119.92193227499999)(6.750948166877371, 120.29803683300001)(6.826801517067004, 118.911563155)(6.902654867256638, 118.60078471199999)(6.9785082174462705, 120.486552175)(7.054361567635905, 119.697389226)(7.130214917825538, 119.91387185900001)(7.206068268015172, 120.466861213)(7.281921618204804, 120.454699963)(7.357774968394438, 120.38757496400001)(7.433628318584072, 119.675420874)(7.509481668773706, 118.58838074799999)(7.585335018963338, 117.847332514)(7.661188369152971, 119.364233579)(7.737041719342605, 121.71455141999999)(7.812895069532239, 118.943007164)(7.888748419721872, 119.42433441700001)(7.964601769911505, 119.382123246)(8.040455120101138, 119.29077428)(8.116308470290772, 122.922717776)(8.192161820480406, 120.192098896)(8.268015170670038, 119.926727893)(8.343868520859672, 119.724129454)(8.419721871049305, 118.098258507)(8.495575221238937, 119.43343371099999)(8.571428571428573, 120.36659474700001)(8.647281921618205, 119.721875942)(8.72313527180784, 119.41173616900001)(8.798988621997472, 118.693427552)(8.874841972187106, 118.33555925600001)(8.95069532237674, 119.708437471)(9.026548672566372, 118.842961422)(9.102402022756005, 119.111102017)(9.178255372945639, 118.818971121)(9.254108723135273, 119.35577988200001)(9.329962073324905, 120.76331549000001)(9.40581542351454, 119.19264719699999)(9.481668773704172, 120.44163784700001)(9.557522123893808, 119.58621763)(9.63337547408344, 120.118285574)(9.709228824273072, 119.206435828)(9.785082174462707, 117.78691881)(9.860935524652339, 118.96346977900001)(9.936788874841973, 118.349950427)(10.012642225031607, 119.889548452)(10.08849557522124, 120.376934041)(10.164348925410872, 119.190660716)(10.240202275600508, 120.530993962)(10.31605562579014, 120.290912663)(10.391908975979774, 119.99222871100001)(10.467762326169407, 117.677982291)(10.543615676359039, 118.33671406100001)(10.619469026548675, 120.071377719)(10.695322376738307, 121.279331189)(10.77117572692794, 119.32630059899999)(10.847029077117574, 119.172944507)(10.922882427307208, 119.173502482)(10.99873577749684, 118.735514666)(11.074589127686474, 120.712341839)(11.150442477876107, 119.40340689099999)(11.22629582806574, 118.23395057600001)(11.302149178255375, 119.555290064)(11.378002528445007, 118.28194799100001)(11.453855878634641, 119.103020382)(11.529709228824274, 121.43783707600001)(11.605562579013906, 119.096376956)(11.681415929203542, 118.874360028)(11.757269279393174, 118.93537353599999)(11.833122629582807, 119.22978694899999)(11.90897597977244, 119.816594257)(11.984829329962075, 118.275836328)(12.060682680151707, 121.67442864099999)(12.136536030341341, 116.91922782700001)(12.212389380530974, 118.116223428)(12.288242730720608, 115.024977969)(12.364096080910242, 118.670976674)(12.439949431099874, 117.964564615)(12.515802781289509, 117.988543318)(12.59165613147914, 120.428759876)(12.667509481668775, 116.39230373800001)(12.74336283185841, 118.74194302999999)(12.819216182048041, 118.92558453299999)(12.895069532237674, 118.536911479)(12.970922882427308, 120.00099052899999)(13.046776232616942, 118.672666521)(13.122629582806574, 117.489677671)(13.198482932996209, 119.151770457)(13.274336283185841, 119.220985889)(13.350189633375475, 118.35376796700001)(13.42604298356511, 118.042557116)(13.501896333754742, 120.087386923)(13.577749683944376, 117.85691694100001)(13.653603034134008, 120.259403655)(13.729456384323642, 119.611615496)(13.805309734513276, 118.757297577)(13.881163084702909, 119.687838969)(13.957016434892541, 117.14102243299999)(14.032869785082175, 117.641093039)(14.10872313527181, 118.730884549)(14.184576485461443, 119.75051409800001)(14.260429835651076, 120.907309014)(14.336283185840708, 118.351069964)(14.412136536030344, 120.43724904599999)(14.487989886219976, 120.17631336099998)(14.563843236409609, 119.61713874399999)(14.639696586599243, 120.09681261400002)(14.715549936788875, 120.021070911)(14.79140328697851, 122.08362220699999)(14.867256637168143, 121.51381948199999)(14.943109987357776, 120.20012183600001)(15.018963337547412, 117.861007465)(15.094816687737044, 117.785320211)(15.170670037926676, 119.48411114500001)(15.24652338811631, 118.51717452100002)(15.322376738305943, 117.35345498999999)(15.398230088495575, 117.597531552)(15.47408343868521, 117.82069917999999)(15.549936788874842, 117.831920246)(15.625790139064478, 117.757994776)(15.701643489254112, 117.641294616)(15.777496839443744, 116.49793659400001)(15.853350189633378, 117.41104850299999)(15.92920353982301, 117.937636755)(16.005056890012643, 117.730550772)(16.080910240202275, 118.04259884700001)(16.15676359039191, 117.18765126999999)(16.232616940581543, 112.512694655)(16.30847029077118, 109.17525483700001)(16.38432364096081, 109.71095376199999)(16.460176991150444, 112.178282549)(16.536030341340076, 110.851868911)(16.611883691529712, 110.428739014)(16.687737041719345, 110.379120556)(16.763590391908977, 108.535614747)(16.83944374209861, 109.97834368699999)(16.91529709228824, 112.477645811)(16.991150442477874, 111.192995559)(17.067003792667514, 110.227310084)(17.142857142857146, 109.58486411799998)(17.21871049304678, 111.395897959)(17.29456384323641, 110.449359273)(17.370417193426043, 109.946357872)(17.44627054361568, 109.776683488)(17.52212389380531, 109.656716391)(17.597977243994944, 110.06254818400001)(17.673830594184576, 110.537227894)(17.749683944374212, 110.281331142)(17.825537294563844, 111.17559433400001)(17.90139064475348, 110.995767625)(17.977243994943112, 110.03483307799999)(18.053097345132745, 110.87214335600001)(18.128950695322377, 108.45720062)(18.20480404551201, 113.386956258)(18.280657395701645, 111.435647283)(18.356510745891278, 110.09184658400001)(18.432364096080914, 110.760092445)(18.508217446270546, 109.06727222900001)(18.58407079646018, 110.113239477)(18.65992414664981, 116.921401055)(18.735777496839447, 119.390634392)(18.81163084702908, 110.56807907299999)(18.88748419721871, 110.377163303)(18.963337547408344, 110.618487786)(19.039190897597976, 110.77821199300001)(19.115044247787615, 110.37182806700001)(19.190897597977248, 116.298375071)(19.26675094816688, 124.83370640700001)(19.342604298356513, 109.861320862)(19.418457648546145, 110.159080288)(19.494310998735777, 110.223659145)(19.570164348925413, 110.416999696)(19.646017699115045, 110.35246141900001)(19.721871049304678, 116.64997376100001)(19.797724399494314, 109.444038831)(19.873577749683946, 116.11450053600001)(19.94943109987358, 116.669180109)(20.025284450063214, 117.694102675)(20.101137800252847, 120.153032467)(20.17699115044248, 119.913681754)(20.25284450063211, 119.798044232)(20.328697850821744, 123.47914876600001)(20.40455120101138, 116.90195899700001)(20.480404551201016, 117.27828180899999)(20.556257901390648, 118.45151858999999)(20.63211125158028, 118.66660676800001)(20.707964601769913, 116.46078240599999)(20.78381795195955, 118.844248236)(20.85967130214918, 121.528264325)(20.935524652338813, 116.949655366)(21.011378002528446, 122.269850325)(21.087231352718078, 122.962026948)(21.163084702907714, 118.744975707)(21.23893805309735, 117.223851064)(21.314791403286982, 115.278340486)(21.390644753476614, 119.327882293)(21.466498103666247, 115.981738672)(21.54235145385588, 118.07491112999999)(21.61820480404551, 117.458506051)(21.694058154235147, 125.70238593699999)(21.76991150442478, 133.24382996100002)(21.845764854614416, 118.242080329)(21.921618204804048, 117.207372864)(21.99747155499368, 118.00954641000001)(22.073324905183316, 116.592863929)(22.14917825537295, 118.15002764200001)(22.22503160556258, 120.536525436)(22.300884955752213, 121.540394568)(22.376738305941846, 121.414247449)(22.45259165613148, 120.85159667299999)(22.528445006321114, 110.28653801000002)(22.60429835651075, 112.010015882)(22.680151706700382, 111.25237675599999)(22.756005056890015, 109.945037914)(22.831858407079647, 113.811676706)(22.907711757269283, 110.32317802899999)(22.983565107458915, 110.464349848)(23.059418457648547, 109.019278003)(23.13527180783818, 109.827716722)(23.211125158027812, 109.15903189299999)(23.286978508217448, 111.85472771500001)(23.362831858407084, 111.34556572999999)(23.438685208596716, 119.681018043)(23.51453855878635, 120.84041122900001)(23.59039190897598, 116.11658876)(23.666245259165613, 120.05827264300001)(23.74209860935525, 118.116651988)(23.81795195954488, 121.55843063200001)(23.893805309734514, 119.264578371)(23.96965865992415, 118.273293899)(24.045512010113782, 120.39684940900001)(24.121365360303415, 119.61666392)(24.19721871049305, 116.841146721)(24.273072060682683, 120.218077131)(24.348925410872315, 118.46230939)(24.424778761061948, 117.202133992)(24.50063211125158, 117.99760807)(24.576485461441216, 118.53071382499999)(24.65233881163085, 120.447768358)(24.728192161820484, 119.394404493)(24.804045512010116, 117.64020474)(24.87989886219975, 117.358657714)(24.95575221238938, 119.936643881)(25.031605562579017, 117.643900603)(25.10745891276865, 118.63153252800001)(25.18331226295828, 117.23238391500001)(25.259165613147914, 121.06220054399999)(25.33501896333755, 119.85264349299999)(25.410872313527186, 119.423611722)(25.48672566371682, 116.937461238)(25.56257901390645, 118.831800504)(25.638432364096083, 118.036824051)(25.714285714285715, 121.352927243)(25.790139064475348, 118.019429797)(25.865992414664984, 118.088321767)(25.941845764854616, 119.442460191)(26.017699115044252, 118.490590387)(26.093552465233884, 119.65138375000001)(26.169405815423517, 118.56760918699999)(26.24525916561315, 120.695344578)(26.321112515802785, 118.049557741)(26.396965865992417, 117.276348591)(26.47281921618205, 120.431567269)(26.548672566371682, 119.085632721)(26.624525916561314, 118.920303764)(26.70037926675095, 119.36215845999999)(26.776232616940586, 115.528767817)(26.85208596713022, 118.895958463)(26.92793931731985, 118.965956052)(27.003792667509483, 118.366220389)(27.07964601769912, 119.695061232)(27.15549936788875, 118.653121436)(27.231352718078384, 120.442398656)(27.307206068268016, 117.105056535)(27.38305941845765, 119.39695777200001)(27.458912768647284, 120.55215327399999)(27.53476611883692, 117.518983894)(27.610619469026553, 118.239076805)(27.686472819216185, 119.77083796000001)(27.762326169405817, 118.597235656)(27.83817951959545, 118.463800664)(27.914032869785082, 121.697480833)(27.989886219974718, 118.25980235899999)(28.06573957016435, 118.864348864)(28.141592920353986, 118.36553108800001)(28.21744627054362, 119.420190278)(28.29329962073325, 118.83582340000001)(28.369152970922887, 118.74529863999999)(28.44500632111252, 118.66094473700001)(28.52085967130215, 118.453183558)(28.596713021491784, 117.53112250499998)(28.672566371681416, 119.726396972)(28.748419721871052, 118.730192616)(28.824273072060688, 119.32154628699999)(28.90012642225032, 119.396878259)(28.975979772439953, 118.285411969)(29.051833122629585, 120.69910323)(29.127686472819217, 118.629743216)(29.203539823008853, 119.144414967)(29.279393173198486, 120.188256076)(29.355246523388118, 119.595516472)(29.43109987357775, 119.28333838900001)(29.506953223767386, 118.08539441600001)(29.58280657395702, 119.174819777)(29.658659924146654, 119.828370549)(29.734513274336287, 119.65014309700001)(29.81036662452592, 121.085766103)(29.88621997471555, 121.146992722)(29.962073324905184, 120.086292663)(30.037926675094823, 108.893613845)(30.113780025284452, 110.207847108)(30.189633375474088, 109.712800003)(30.265486725663717, 109.608988541)(30.341340075853353, 109.57592031200001)(30.417193426042985, 110.26056240999999)(30.49304677623262, 110.85180870800001)(30.568900126422257, 110.13553140300002)(30.644753476611886, 111.352066185)(30.72060682680152, 109.74858374)(30.79646017699115, 109.530019787)(30.872313527180786, 110.047859221)(30.94816687737042, 112.29973984300001)(31.024020227560055, 108.676213214)(31.099873577749683, 110.199841593)(31.17572692793932, 114.76671451499999)(31.251580278128955, 109.94147038999999)(31.327433628318587, 110.030196836)(31.403286978508223, 113.05591615799999)(31.479140328697852, 109.74924965)(31.554993678887488, 110.30413705699999)(31.630847029077117, 112.622219654)(31.706700379266756, 109.85945194199999)(31.782553729456385, 109.246993842)(31.85840707964602, 110.277888477)(31.934260429835657, 110.45669168)(32.010113780025286, 110.92630116800001)(32.085967130214925, 108.43604620600001)(32.16182048040455, 109.243413422)(32.23767383059419, 110.773758995)(32.31352718078382, 111.27502930200001)(32.389380530973455, 109.692247388)(32.46523388116309, 109.76294239799999)(32.54108723135272, 111.735954033)(32.61694058154236, 109.82363834399999)(32.692793931731984, 110.54914304100001)(32.76864728192162, 110.070155444)(32.844500632111256, 113.753342586)(32.92035398230089, 110.74500849)(32.99620733249052, 109.98094437)(33.07206068268015, 109.72539397700001)(33.147914032869785, 109.38028686799998)(33.223767383059425, 111.826653081)(33.29962073324906, 110.185852525)(33.37547408343869, 108.350113112)(33.45132743362832, 109.641322856)(33.527180783817954, 110.99406586500001)(33.603034134007586, 107.904791395)(33.67888748419722, 109.494969299)(33.75474083438686, 108.957958345)(33.83059418457648, 108.408175176)(33.90644753476612, 109.19457384799999)(33.98230088495575, 110.052201911)(34.05815423514539, 116.574898406)(34.13400758533503, 111.88386462400001)(34.20986093552465, 111.885057754)(34.28571428571429, 113.05821231799999)(34.36156763590392, 117.427305287)(34.43742098609356, 115.88306787600001)(34.51327433628319, 109.759715029)(34.58912768647282, 109.541727669)(34.664981036662454, 113.822675865)(34.740834386852086, 107.60048356)(34.816687737041725, 109.132498492)(34.89254108723136, 110.598589615)(34.96839443742099, 116.616120372)(35.04424778761062, 115.13710093200001)(35.120101137800255, 116.58986840600001)(35.19595448798989, 113.645241646)(35.27180783817953, 115.861265376)(35.34766118836915, 115.796320345)(35.42351453855879, 117.644507683)(35.499367888748424, 117.41847306700001)(35.575221238938056, 115.192474978)(35.65107458912769, 117.30421606)(35.72692793931732, 115.816239861)(35.80278128950696, 118.51261217499999)(35.878634639696585, 119.07381235900002)(35.954487989886225, 117.011695386)(36.03034134007585, 117.582398921)(36.10619469026549, 116.732265832)(36.18204804045513, 116.785153287)(36.257901390644754, 118.503678535)(36.333754740834394, 117.45529101800001)(36.40960809102402, 116.84856568500001)(36.48546144121366, 117.74448057500001)(36.56131479140329, 116.640371048)(36.63716814159292, 123.248115112)(36.713021491782555, 115.869313077)(36.78887484197219, 116.69429746200001)(36.86472819216183, 116.785590147)(36.94058154235145, 117.99157744)(37.01643489254109, 117.736396764)(37.092288242730724, 118.84914803199999)(37.16814159292036, 120.161898476)(37.24399494310999, 119.20859174)(37.31984829329962, 119.826665337)(37.395701643489254, 121.74674919099999)(37.47155499367889, 120.71654736699999)(37.547408343868526, 108.837233936)(37.62326169405816, 110.549307401)(37.69911504424779, 112.688297727)(37.77496839443742, 109.533552189)(37.85082174462706, 110.674145045)(37.92667509481669, 110.827904284)(38.00252844500633, 111.55414797)(38.07838179519595, 110.28652308299999)(38.15423514538559, 119.170760874)(38.23008849557523, 118.4398184)(38.305941845764856, 116.933236771)(38.381795195954496, 118.92587711899999)(38.45764854614412, 118.014060068)(38.53350189633376, 116.78576590700001)(38.60935524652339, 119.500769332)(38.685208596713025, 117.965068447)(38.76106194690266, 118.589943324)(38.83691529709229, 118.15813702)(38.91276864728193, 118.64320633)(38.988621997471554, 117.710752384)(39.064475347661194, 119.942710853)(39.140328697850826, 116.971047761)(39.21618204804046, 119.33265175000001)(39.29203539823009, 117.44863926500001)(39.36788874841972, 119.935959809)(39.443742098609356, 119.20299702999999)(39.519595448798995, 118.87033479899999)(39.59544879898863, 117.03933343499999)(39.67130214917826, 118.880823706)(39.74715549936789, 118.148693846)(39.823008849557525, 118.141474619)(39.89886219974716, 119.578109385)(39.97471554993679, 118.632566642)(40.05056890012643, 118.892790707)(40.126422250316054, 119.771787556)(40.20227560050569, 119.35993570400001)(40.278128950695326, 118.191539116)(40.35398230088496, 118.013026892)(40.4298356510746, 119.245952567)(40.50568900126422, 118.26343408599999)(40.58154235145386, 122.79393638)(40.65739570164349, 117.985568672)(40.73324905183313, 120.68521626500001)(40.80910240202276, 119.267107171)(40.88495575221239, 119.04414705)(40.96080910240203, 118.568486999)(41.036662452591656, 118.610731891)(41.112515802781296, 117.894495182)(41.18836915297093, 120.881559589)(41.26422250316056, 136.164884425)(41.34007585335019, 118.894129206)(41.415929203539825, 118.686993921)(41.49178255372946, 120.055695018)(41.5676359039191, 119.638470271)(41.64348925410873, 119.458483771)(41.71934260429836, 119.745450366)(41.795195954487994, 119.84834896500001)(41.871049304677626, 118.33021638299999)(41.94690265486726, 118.91363699499999)(42.02275600505689, 118.755722494)(42.09860935524653, 120.24303931700001)(42.174462705436156, 118.293523441)(42.250316055625795, 118.884528324)(42.32616940581543, 120.543408968)(42.40202275600506, 118.86587202499999)(42.4778761061947, 118.729782011)(42.553729456384325, 118.98480812000001)(42.629582806573964, 119.470464044)(42.70543615676359, 120.29898234299999)(42.78128950695323, 121.075038412)(42.85714285714286, 118.38581272300002)(42.932996207332494, 119.368552284)(43.00884955752213, 119.355343186)(43.08470290771176, 118.215793098)(43.1605562579014, 119.15927934899999)(43.23640960809102, 120.300830229)(43.31226295828066, 119.49054678799999)(43.388116308470295, 119.00366171499999)(43.46396965865993, 118.546564815)(43.53982300884956, 118.152463862)(43.61567635903919, 119.66010979)(43.69152970922883, 120.541901896)(43.767383059418464, 119.93999511300001)(43.843236409608096, 119.31951658999999)(43.91908975979773, 120.663068645)(43.99494310998736, 118.14494697100001)(44.07079646017699, 118.435904475)(44.14664981036663, 121.582932058)(44.22250316055626, 118.705461198)(44.2983565107459, 122.006394254)(44.37420986093552, 121.021511366)(44.45006321112516, 118.712452631)(44.5259165613148, 119.575338441)(44.60176991150443, 118.44461946199999)(44.677623261694066, 119.305131153)(44.75347661188369, 120.9463392)(44.82932996207333, 121.438036244)(44.90518331226296, 120.606420265)(44.981036662452595, 119.722326649)(45.05689001264223, 109.90504168499999)(45.13274336283186, 109.767502105)(45.2085967130215, 108.483249434)(45.284450063211125, 109.16761485)(45.360303413400764, 109.30135270899999)(45.4361567635904, 108.86211775)(45.51201011378003, 108.69800584000001)(45.58786346396966, 109.511728878)(45.663716814159294, 111.42717689899999)(45.739570164348926, 109.522659551)(45.815423514538566, 110.157414738)(45.8912768647282, 111.90111192399999)(45.96713021491783, 109.03730565000001)(46.04298356510746, 110.056992265)(46.118836915297095, 110.26518927699999)(46.19469026548673, 110.283340912)(46.27054361567636, 108.294836873)(46.346396965866, 109.87399761900001)(46.422250316055624, 108.231669961)(46.498103666245264, 110.499309947)(46.573957016434896, 110.59206440700001)(46.64981036662453, 109.05083531100001)(46.72566371681417, 109.667434145)(46.80151706700379, 110.06539154400001)(46.87737041719343, 110.54293162900001)(46.95322376738306, 111.929075689)(47.0290771175727, 111.091019748)(47.10493046776233, 110.773792246)(47.18078381795196, 108.971753875)(47.2566371681416, 109.134763948)(47.33249051833123, 108.80541403000001)(47.408343868520866, 109.697622694)(47.4841972187105, 107.74353031800001)(47.56005056890013, 111.64237324300001)(47.63590391908976, 111.70379040200001)(47.711757269279396, 111.217185171)(47.78761061946903, 110.34164681)(47.86346396965867, 108.297151457)(47.9393173198483, 109.752423001)(48.01517067003793, 108.26867272)(48.091024020227565, 112.27096959400001)(48.1668773704172, 110.777091214)(48.24273072060683, 111.26573487900001)(48.31858407079646, 111.66587473199999)(48.3944374209861, 112.721494687)(48.470290771175726, 108.822856774)(48.546144121365366, 110.847553963)(48.621997471555, 110.16555746099999)(48.69785082174463, 109.355302424)(48.77370417193427, 109.43750702999999)(48.849557522123895, 111.073937645)(48.925410872313535, 110.521832005)(49.00126422250316, 110.73453906699999)(49.0771175726928, 110.98402167)(49.15297092288243, 110.257101309)(49.228824273072064, 108.81960921499999)(49.3046776232617, 112.41435012)(49.38053097345133, 109.784264918)(49.45638432364097, 110.553227001)(49.53223767383059, 110.74346876199999)(49.60809102402023, 107.49765480800001)(49.683944374209865, 112.066276461)(49.7597977243995, 107.96743783400001)(49.83565107458913, 108.746598098)(49.91150442477876, 108.34401832699999)(49.9873577749684, 109.323026212)(50.063211125158034, 109.989337455)(50.139064475347666, 110.588932559)(50.2149178255373, 110.291408223)(50.29077117572693, 108.855701532)(50.36662452591656, 108.829024199)(50.4424778761062, 110.518208727)(50.51833122629583, 110.499361153)(50.59418457648547, 119.427030859)(50.6700379266751, 119.606357384)(50.74589127686473, 117.860917212)(50.82174462705437, 119.355132894)(50.897597977244, 118.975455036)(50.97345132743364, 117.531653659)(51.04930467762326, 119.12463839699998)(51.1251580278129, 117.35794487199999)(51.20101137800253, 117.75653270699999)(51.276864728192166, 117.84584949299999)(51.352718078381805, 118.539027799)(51.42857142857143, 118.4870927)(51.50442477876107, 119.061166853)(51.580278128950695, 118.263417534)(51.656131479140335, 120.196523353)(51.73198482932997, 118.11239160299999)(51.8078381795196, 118.425761948)(51.88369152970923, 117.49010198100001)(51.959544879898864, 117.32625198199999)(52.035398230088504, 118.531802789)(52.111251580278136, 119.10942693799998)(52.18710493046777, 117.52473356899999)(52.2629582806574, 119.248433107)(52.33881163084703, 120.08248127)(52.414664981036665, 119.291132885)(52.4905183312263, 119.001389513)(52.56637168141593, 106.22622827200001)(52.64222503160557, 107.118803528)(52.7180783817952, 100.17219821399999)(52.793931731984834, 100.46277358699999)(52.86978508217447, 103.173278149)(52.9456384323641, 106.22820269900001)(53.02149178255374, 105.39075886699999)(53.097345132743364, 106.762209981)(53.173198482933, 106.43230039800001)(53.24905183312263, 104.740957451)(53.32490518331227, 104.68585039800001)(53.4007585335019, 105.888446284)(53.47661188369153, 107.838189566)(53.55246523388117, 104.809019128)(53.6283185840708, 106.080161163)(53.70417193426044, 105.82346770499998)(53.78002528445007, 105.44551113)(53.8558786346397, 106.481421832)(53.931731984829334, 105.96592037100001)(54.007585335018966, 105.76660781499999)(54.0834386852086, 104.346775744)(54.15929203539824, 107.02436949000001)(54.23514538558787, 107.89337250499999)(54.3109987357775, 106.37539989700001)(54.386852085967135, 105.390465885)(54.46270543615677, 106.017264244)(54.5385587863464, 104.123852913)(54.61441213653603, 104.39643271899999)(54.69026548672567, 106.608766454)(54.7661188369153, 105.313244118)(54.841972187104936, 105.16393120000001)(54.91782553729457, 105.828586981)(54.9936788874842, 104.512047729)(55.06953223767384, 104.56099431300001)(55.145385587863466, 106.092773319)(55.221238938053105, 105.878032949)(55.29709228824273, 105.222117459)(55.37294563843237, 106.195434534)(55.448798988622, 104.888747599)(55.524652338811634, 106.310285841)(55.600505689001274, 106.41952545000001)(55.6763590391909, 106.43827463099998)(55.75221238938054, 105.52375376)(55.828065739570164, 105.176745868)(55.9039190897598, 106.201585809)(55.979772439949436, 101.473788235)(56.05562579013907, 105.28739829099999)(56.1314791403287, 101.692329336)(56.20733249051833, 108.128393582)(56.28318584070797, 105.861125455)(56.359039190897604, 104.68534827699999)(56.43489254108724, 108.749914802)(56.51074589127687, 105.708160139)(56.5865992414665, 105.38128665100001)(56.662452591656134, 105.54211256900001)(56.73830594184577, 105.312356191)(56.8141592920354, 105.308716076)(56.89001264222504, 105.548327565)(56.96586599241467, 105.941258558)(57.0417193426043, 106.037746311)(57.11757269279394, 105.548458513)(57.19342604298357, 105.437408625)(57.26927939317321, 105.794552722)(57.34513274336283, 110.468799903)(57.42098609355247, 105.92084124799999)(57.496839443742104, 103.048106608)(57.572692793931736, 105.69496253300001)(57.648546144121376, 105.423128311)(57.724399494311, 105.827116946)(57.80025284450064, 104.342480341)(57.876106194690266, 105.442508226)(57.951959544879905, 107.419906338)(58.02781289506954, 106.712287829)(58.10366624525917, 106.108780208)(58.1795195954488, 104.593476701)(58.255372945638435, 107.16351327699999)(58.331226295828074, 104.54839941499999)(58.407079646017706, 105.06910306)(58.48293299620734, 103.70350178700001)(58.55878634639697, 103.22558217099999)(58.6346396965866, 103.31456894200001)(58.710493046776236, 104.751774386)(58.78634639696587, 105.825144828)(58.8621997471555, 106.46723717500001)(58.93805309734514, 107.69228284)(59.01390644753477, 104.41576853000001)(59.089759797724405, 104.538384394)(59.16561314791404, 104.82516561700001)(59.24146649810367, 103.590742078)(59.31731984829331, 104.08612032900001)(59.393173198482934, 104.136943647)(59.46902654867257, 104.571130311)(59.5448798988622, 102.750323283)(59.62073324905184, 103.620999086)(59.69658659924148, 104.797008661)(59.7724399494311, 105.107100493)(59.84829329962074, 102.649525665)(59.92414664981037, 103.68518040699999)(60.0, 103.207670183)
        };
        \addplot[color=blue, mark=none,name path=B] coordinates { %% MIN value
        (0.0, 45.228535001)(0.07585335018963338, 48.036800737)(0.15170670037926676, 52.334553771)(0.22756005056890014, 33.308686858)(0.3034134007585335, 46.958409931)(0.3792667509481669, 43.798106585)(0.45512010113780027, 46.946185252)(0.5309734513274336, 54.103319361000004)(0.606826801517067, 55.337850555)(0.6826801517067005, 43.413252316)(0.7585335018963338, 30.297202668)(0.8343868520859672, 44.556282798)(0.9102402022756005, 37.083095833)(0.986093552465234, 50.08795713799999)(1.0619469026548671, 50.537318645)(1.1378002528445006, 49.200208188000005)(1.213653603034134, 46.185488174)(1.2895069532237675, 49.744675493)(1.365360303413401, 42.77623452)(1.4412136536030342, 50.150043362999995)(1.5170670037926677, 43.136999069)(1.5929203539823011, 52.053414792000005)(1.6687737041719344, 46.14316096899999)(1.7446270543615676, 44.466766969)(1.820480404551201, 48.39757066)(1.8963337547408345, 54.178818322)(1.972187104930468, 54.554573991)(2.0480404551201015, 40.508702232999994)(2.1238938053097343, 52.76416047399999)(2.199747155499368, 42.523299892)(2.275600505689001, 41.755800212)(2.351453855878635, 42.747482571999996)(2.427307206068268, 49.255535398)(2.503160556257902, 54.511489351)(2.579013906447535, 47.627992465000005)(2.6548672566371687, 45.95146381)(2.730720606826802, 51.152584331)(2.806573957016435, 45.075135379)(2.8824273072060684, 49.641852833)(2.9582806573957017, 52.502679229)(3.0341340075853354, 51.011021507)(3.1099873577749686, 48.330974581)(3.1858407079646023, 52.39708301)(3.2616940581542355, 51.347950018999995)(3.3375474083438688, 52.654997748)(3.413400758533502, 53.17812644)(3.4892541087231352, 44.388336859)(3.565107458912769, 46.930589780999995)(3.640960809102402, 46.285841893)(3.716814159292036, 44.642787235)(3.792667509481669, 52.330252187)(3.8685208596713023, 37.948537306)(3.944374209860936, 43.295480221)(4.020227560050569, 49.377620045)(4.096080910240203, 37.948611639999996)(4.171934260429836, 38.34301778)(4.2477876106194685, 38.47374797)(4.323640960809103, 54.554080393)(4.399494310998736, 45.322986805)(4.47534766118837, 47.358152342000004)(4.551201011378002, 38.983559942)(4.6270543615676365, 49.017625155)(4.70290771175727, 45.258703127000004)(4.778761061946904, 47.553060754)(4.854614412136536, 37.176775166999995)(4.9304677623261695, 48.732853928)(5.006321112515804, 49.160177098000005)(5.082174462705436, 37.101556356)(5.15802781289507, 45.517816993000004)(5.233881163084703, 45.74754299)(5.309734513274337, 47.047935953999996)(5.38558786346397, 47.228400006)(5.461441213653604, 44.55164026600001)(5.537294563843237, 41.891443308999996)(5.61314791403287, 47.476422067)(5.689001264222504, 45.76903500900001)(5.764854614412137, 50.918817183)(5.840707964601771, 42.496558586)(5.916561314791403, 46.013489693)(5.9924146649810375, 44.388163186)(6.068268015170671, 49.025619731)(6.144121365360304, 35.627676918)(6.219974715549937, 36.97435924)(6.29582806573957, 51.699450281)(6.371681415929205, 39.590034814999996)(6.447534766118837, 44.482311261999996)(6.523388116308471, 42.103718462)(6.599241466498104, 37.253435522000004)(6.6750948166877375, 33.089038081)(6.750948166877371, 25.185139001)(6.826801517067004, 13.380246862)(6.902654867256638, 37.740972385)(6.9785082174462705, 14.225221401999999)(7.054361567635905, 18.107828216999998)(7.130214917825538, 35.452568979999995)(7.206068268015172, 27.218374933)(7.281921618204804, 49.18488855299999)(7.357774968394438, 43.737744974)(7.433628318584072, 62.095350485000004)(7.509481668773706, 36.575274425)(7.585335018963338, 52.403158033)(7.661188369152971, 37.574536304)(7.737041719342605, 44.478751458000005)(7.812895069532239, 48.543069239000005)(7.888748419721872, 46.982440435)(7.964601769911505, 42.628692585)(8.040455120101138, 39.536967341)(8.116308470290772, 41.908588049)(8.192161820480406, 49.141540749)(8.268015170670038, 37.375507936)(8.343868520859672, 38.708813849)(8.419721871049305, 45.728072751999996)(8.495575221238937, 53.806760657000005)(8.571428571428573, 36.411331242)(8.647281921618205, 39.173369291)(8.72313527180784, 49.08607993300001)(8.798988621997472, 46.592322862)(8.874841972187106, 45.207047742)(8.95069532237674, 44.502238984)(9.026548672566372, 37.254460496)(9.102402022756005, 45.493785231)(9.178255372945639, 48.754620771)(9.254108723135273, 47.405660938)(9.329962073324905, 47.902451228000004)(9.40581542351454, 49.785387255)(9.481668773704172, 40.466719224)(9.557522123893808, 44.232729152000005)(9.63337547408344, 37.927320933000004)(9.709228824273072, 45.265933027)(9.785082174462707, 39.453123477999995)(9.860935524652339, 44.364139027)(9.936788874841973, 43.924622462)(10.012642225031607, 50.210314421)(10.08849557522124, 37.002015842999995)(10.164348925410872, 46.728118067000004)(10.240202275600508, 43.557126917)(10.31605562579014, 43.453304686)(10.391908975979774, 50.451691021)(10.467762326169407, 48.252963143)(10.543615676359039, 44.606637187000004)(10.619469026548675, 49.950100207)(10.695322376738307, 37.449119251)(10.77117572692794, 42.290130294)(10.847029077117574, 43.738095776)(10.922882427307208, 42.976033484)(10.99873577749684, 44.874108369)(11.074589127686474, 46.108913142999995)(11.150442477876107, 39.720480229)(11.22629582806574, 44.466693211999996)(11.302149178255375, 49.565766158)(11.378002528445007, 38.932169679)(11.453855878634641, 44.653314348)(11.529709228824274, 40.64231307)(11.605562579013906, 53.210974714)(11.681415929203542, 37.850093438)(11.757269279393174, 46.031967931000004)(11.833122629582807, 37.641359157)(11.90897597977244, 48.700205021)(11.984829329962075, 37.673763905)(12.060682680151707, 44.417965624000004)(12.136536030341341, 39.621095388)(12.212389380530974, 43.614088935)(12.288242730720608, 47.910351006)(12.364096080910242, 38.184868066)(12.439949431099874, 48.813703612)(12.515802781289509, 36.559549054)(12.59165613147914, 37.611344103)(12.667509481668775, 43.376300697000005)(12.74336283185841, 44.233266598)(12.819216182048041, 13.853450013)(12.895069532237674, 47.288057752)(12.970922882427308, 47.83943575800001)(13.046776232616942, 53.332228289)(13.122629582806574, 45.736977994)(13.198482932996209, 44.982747566)(13.274336283185841, 46.294224849)(13.350189633375475, 44.41883254300001)(13.42604298356511, 53.902529965)(13.501896333754742, 46.904715133)(13.577749683944376, 51.018502657999996)(13.653603034134008, 45.822278929)(13.729456384323642, 45.359620565)(13.805309734513276, 47.87245914)(13.881163084702909, 44.207259878)(13.957016434892541, 43.766233595)(14.032869785082175, 37.748906863)(14.10872313527181, 41.41375841200001)(14.184576485461443, 41.681930482)(14.260429835651076, 25.597594973)(14.336283185840708, 15.053400357)(14.412136536030344, 37.500344496000004)(14.487989886219976, 14.664081554)(14.563843236409609, 13.328843774000001)(14.639696586599243, 32.459368649)(14.715549936788875, 13.385647463000002)(14.79140328697851, 48.473896739000004)(14.867256637168143, 19.695978638)(14.943109987357776, 47.831557817)(15.018963337547412, 47.810092205)(15.094816687737044, 52.2318442)(15.170670037926676, 52.583766901000004)(15.24652338811631, 51.64927358)(15.322376738305943, 45.117188259)(15.398230088495575, 41.661392068)(15.47408343868521, 49.519908242)(15.549936788874842, 43.457360741)(15.625790139064478, 46.408936548)(15.701643489254112, 47.622788561)(15.777496839443744, 46.286821198999995)(15.853350189633378, 42.926284501)(15.92920353982301, 50.26162935000001)(16.005056890012643, 46.376978117)(16.080910240202275, 42.096796141)(16.15676359039191, 44.375410333)(16.232616940581543, 38.993038067)(16.30847029077118, 45.698833278)(16.38432364096081, 43.420354074)(16.460176991150444, 43.48865021)(16.536030341340076, 50.241446259)(16.611883691529712, 46.003779264)(16.687737041719345, 46.362745593)(16.763590391908977, 42.217863205)(16.83944374209861, 48.146032812)(16.91529709228824, 43.894377780999996)(16.991150442477874, 43.25039366200001)(17.067003792667514, 37.899191667)(17.142857142857146, 42.129896523)(17.21871049304678, 45.635068098)(17.29456384323641, 50.295203395)(17.370417193426043, 39.459276647)(17.44627054361568, 38.611458869)(17.52212389380531, 44.206390816)(17.597977243994944, 44.742768670000004)(17.673830594184576, 45.752743188)(17.749683944374212, 44.441993321)(17.825537294563844, 44.063606799999995)(17.90139064475348, 37.813538285)(17.977243994943112, 45.667017822)(18.053097345132745, 37.377926362)(18.128950695322377, 37.733160108)(18.20480404551201, 43.39608273099999)(18.280657395701645, 40.303305826000006)(18.356510745891278, 46.28314699)(18.432364096080914, 35.359543945)(18.508217446270546, 45.266082781)(18.58407079646018, 42.039917270000004)(18.65992414664981, 42.655969666000004)(18.735777496839447, 36.774316061)(18.81163084702908, 42.089905965)(18.88748419721871, 53.730652096)(18.963337547408344, 43.291760961)(19.039190897597976, 38.698786704)(19.115044247787615, 44.924774794)(19.190897597977248, 45.112513959999994)(19.26675094816688, 41.119712562000004)(19.342604298356513, 38.237488792)(19.418457648546145, 29.120138629)(19.494310998735777, 37.009166373)(19.570164348925413, 38.151863865)(19.646017699115045, 41.576855376)(19.721871049304678, 37.693645634999996)(19.797724399494314, 44.349546751000005)(19.873577749683946, 42.735901084999995)(19.94943109987358, 43.375234732)(20.025284450063214, 41.641572892)(20.101137800252847, 43.839064820999994)(20.17699115044248, 33.270008218)(20.25284450063211, 43.235771777)(20.328697850821744, 41.392129667)(20.40455120101138, 50.252506678)(20.480404551201016, 36.284795961)(20.556257901390648, 36.013242322)(20.63211125158028, 51.606584091)(20.707964601769913, 38.830919151)(20.78381795195955, 37.999849976)(20.85967130214918, 50.906454885)(20.935524652338813, 45.501634976)(21.011378002528446, 37.961662358)(21.087231352718078, 45.648198626)(21.163084702907714, 47.023271533999996)(21.23893805309735, 38.080949387000004)(21.314791403286982, 27.451321096)(21.390644753476614, 29.676199548)(21.466498103666247, 18.902202155)(21.54235145385588, 31.74452478)(21.61820480404551, 46.365314161)(21.694058154235147, 36.737955661)(21.76991150442478, 26.995595697)(21.845764854614416, 30.583668487)(21.921618204804048, 25.516842801)(21.99747155499368, 12.392487681999999)(22.073324905183316, 27.322317120999998)(22.14917825537295, 28.006558963)(22.22503160556258, 40.44724508)(22.300884955752213, 52.773539424000006)(22.376738305941846, 52.24436822999999)(22.45259165613148, 56.572590510999994)(22.528445006321114, 43.385931816)(22.60429835651075, 45.318867788)(22.680151706700382, 37.427174205)(22.756005056890015, 36.903285357)(22.831858407079647, 42.41257844)(22.907711757269283, 37.541002199000005)(22.983565107458915, 37.665185818000005)(23.059418457648547, 32.662129103)(23.13527180783818, 41.839469253)(23.211125158027812, 41.253415303)(23.286978508217448, 44.000285127999994)(23.362831858407084, 40.15892848)(23.438685208596716, 45.145925636)(23.51453855878635, 43.113914126)(23.59039190897598, 44.288337654)(23.666245259165613, 50.889575334)(23.74209860935525, 48.602576474)(23.81795195954488, 38.394965279000004)(23.893805309734514, 45.768188985)(23.96965865992415, 46.95810752)(24.045512010113782, 47.507461586000005)(24.121365360303415, 36.841715904)(24.19721871049305, 49.780310805)(24.273072060682683, 45.342547904)(24.348925410872315, 45.327253878)(24.424778761061948, 43.121326808999996)(24.50063211125158, 46.288815881)(24.576485461441216, 50.876230627999995)(24.65233881163085, 53.531046917)(24.728192161820484, 44.639998739999996)(24.804045512010116, 39.431317816)(24.87989886219975, 42.952974409)(24.95575221238938, 46.534586579)(25.031605562579017, 37.458759218000004)(25.10745891276865, 43.388625727000004)(25.18331226295828, 51.524841712000004)(25.259165613147914, 46.919506441)(25.33501896333755, 48.558665433)(25.410872313527186, 46.83153926)(25.48672566371682, 40.380961151)(25.56257901390645, 43.350952871)(25.638432364096083, 46.861436741000006)(25.714285714285715, 53.433649771000006)(25.790139064475348, 45.583314123)(25.865992414664984, 44.987483012)(25.941845764854616, 45.147956528)(26.017699115044252, 45.524993889)(26.093552465233884, 38.423646497)(26.169405815423517, 38.768842316000004)(26.24525916561315, 52.722556812)(26.321112515802785, 39.210142174)(26.396965865992417, 38.153939777)(26.47281921618205, 38.879741258)(26.548672566371682, 47.195127045999996)(26.624525916561314, 43.053230374)(26.70037926675095, 47.066742296)(26.776232616940586, 43.620830039999994)(26.85208596713022, 45.240480749)(26.92793931731985, 45.697208083999996)(27.003792667509483, 45.903080663000004)(27.07964601769912, 36.905687023)(27.15549936788875, 56.089941935)(27.231352718078384, 54.787806433)(27.307206068268016, 40.464725238)(27.38305941845765, 42.971768858)(27.458912768647284, 36.813461033)(27.53476611883692, 44.463273096)(27.610619469026553, 51.735069323)(27.686472819216185, 51.899402787)(27.762326169405817, 38.058382989)(27.83817951959545, 52.277312342)(27.914032869785082, 46.177779367)(27.989886219974718, 53.349955899)(28.06573957016435, 37.320160096)(28.141592920353986, 46.204061878999994)(28.21744627054362, 46.64152524400001)(28.29329962073325, 23.18199963)(28.369152970922887, 46.21260590600001)(28.44500632111252, 40.78917525599999)(28.52085967130215, 16.770679201)(28.596713021491784, 45.115506062)(28.672566371681416, 50.235554452)(28.748419721871052, 44.879783370000006)(28.824273072060688, 37.800421968)(28.90012642225032, 37.859155761)(28.975979772439953, 55.301217642)(29.051833122629585, 45.796654239999995)(29.127686472819217, 25.874436088000003)(29.203539823008853, 45.500901238000004)(29.279393173198486, 25.210116967)(29.355246523388118, 14.714164284999999)(29.43109987357775, 28.408265792999998)(29.506953223767386, 14.542059207)(29.58280657395702, 29.899295713999997)(29.658659924146654, 15.885765455000001)(29.734513274336287, 52.65909412)(29.81036662452592, 40.226767345)(29.88621997471555, 68.298639874)(29.962073324905184, 40.344271491)(30.037926675094823, 37.691030033000004)(30.113780025284452, 41.330139427)(30.189633375474088, 47.968308760999996)(30.265486725663717, 47.88748321)(30.341340075853353, 39.434245608)(30.417193426042985, 45.454662684)(30.49304677623262, 43.20986525400001)(30.568900126422257, 49.398596883)(30.644753476611886, 37.314910446)(30.72060682680152, 46.784350394)(30.79646017699115, 46.928481612999995)(30.872313527180786, 53.694697392)(30.94816687737042, 42.412600753999996)(31.024020227560055, 41.012986472)(31.099873577749683, 40.963689284000004)(31.17572692793932, 45.544247377)(31.251580278128955, 45.076461337)(31.327433628318587, 39.524802863000005)(31.403286978508223, 47.848070201999995)(31.479140328697852, 46.358686135999996)(31.554993678887488, 51.750611336999995)(31.630847029077117, 40.332048117)(31.706700379266756, 42.487497866000005)(31.782553729456385, 42.767441907999995)(31.85840707964602, 45.371622002)(31.934260429835657, 40.027621059)(32.010113780025286, 36.380835831999995)(32.085967130214925, 38.051715294)(32.16182048040455, 47.446773904)(32.23767383059419, 51.773518978000006)(32.31352718078382, 38.933328298)(32.389380530973455, 38.107166537)(32.46523388116309, 41.454293598999996)(32.54108723135272, 45.024783195)(32.61694058154236, 43.083606882)(32.692793931731984, 42.124732807)(32.76864728192162, 40.043156208)(32.844500632111256, 53.06060978399999)(32.92035398230089, 50.235838992)(32.99620733249052, 43.978848385999996)(33.07206068268015, 36.960857825)(33.147914032869785, 45.402216067000005)(33.223767383059425, 53.785139009999995)(33.29962073324906, 49.212791567)(33.37547408343869, 45.700837328)(33.45132743362832, 45.308188959000006)(33.527180783817954, 44.491121281000005)(33.603034134007586, 38.862948707)(33.67888748419722, 49.378325206)(33.75474083438686, 42.144393867)(33.83059418457648, 52.83094615)(33.90644753476612, 46.621601633000004)(33.98230088495575, 39.627783043)(34.05815423514539, 41.281100617)(34.13400758533503, 44.83958818799999)(34.20986093552465, 45.965634444)(34.28571428571429, 44.551761182999996)(34.36156763590392, 40.254652296)(34.43742098609356, 44.325303393000006)(34.51327433628319, 51.562083186)(34.58912768647282, 43.784599191)(34.664981036662454, 48.690615089999994)(34.740834386852086, 43.054233616)(34.816687737041725, 23.067405869)(34.89254108723136, 47.466667694)(34.96839443742099, 45.540599405)(35.04424778761062, 34.441307042)(35.120101137800255, 46.232340287999996)(35.19595448798989, 41.038795517000004)(35.27180783817953, 51.489393182)(35.34766118836915, 44.428596764000005)(35.42351453855879, 43.727710912999996)(35.499367888748424, 42.821899332)(35.575221238938056, 24.733072979)(35.65107458912769, 45.433274327)(35.72692793931732, 18.760292974)(35.80278128950696, 41.356998882)(35.878634639696585, 38.932477076)(35.954487989886225, 43.19430388)(36.03034134007585, 39.850472927)(36.10619469026549, 56.206628697)(36.18204804045513, 40.927676742)(36.257901390644754, 44.685776062)(36.333754740834394, 38.389643842)(36.40960809102402, 52.482502176)(36.48546144121366, 51.323075799)(36.56131479140329, 45.252323475000004)(36.63716814159292, 36.588132279)(36.713021491782555, 36.456323198999996)(36.78887484197219, 15.820195542)(36.86472819216183, 13.131567505)(36.94058154235145, 27.746597039999997)(37.01643489254109, 25.085097946)(37.092288242730724, 48.810408944)(37.16814159292036, 46.006999862)(37.24399494310999, 56.53827596)(37.31984829329962, 40.115220566)(37.395701643489254, 56.103399158)(37.47155499367889, 57.006560491)(37.547408343868526, 51.821837524)(37.62326169405816, 47.302484668)(37.69911504424779, 53.759861622)(37.77496839443742, 51.788431382000006)(37.85082174462706, 46.493686791)(37.92667509481669, 35.521252663)(38.00252844500633, 44.876001969)(38.07838179519595, 14.353625516)(38.15423514538559, 48.925267653000006)(38.23008849557523, 42.868505076999995)(38.305941845764856, 49.46985728599999)(38.381795195954496, 44.072815717999994)(38.45764854614412, 51.034167386)(38.53350189633376, 45.260977533)(38.60935524652339, 39.756738794)(38.685208596713025, 47.281157179)(38.76106194690266, 45.271773728999996)(38.83691529709229, 44.164183011000006)(38.91276864728193, 42.815060196000005)(38.988621997471554, 44.551752259)(39.064475347661194, 51.737327705)(39.140328697850826, 36.085313598)(39.21618204804046, 43.232400497)(39.29203539823009, 40.880926031)(39.36788874841972, 43.044405481999995)(39.443742098609356, 44.736896626000004)(39.519595448798995, 49.952240233999994)(39.59544879898863, 41.430759076)(39.67130214917826, 39.847204041)(39.74715549936789, 44.064938239)(39.823008849557525, 43.279026366000004)(39.89886219974716, 44.318764437)(39.97471554993679, 49.810294478)(40.05056890012643, 51.780882598)(40.126422250316054, 39.930951707000006)(40.20227560050569, 36.722434191)(40.278128950695326, 45.449331208)(40.35398230088496, 51.288453118)(40.4298356510746, 46.393717551)(40.50568900126422, 45.567647218)(40.58154235145386, 43.167053652)(40.65739570164349, 46.394341256000004)(40.73324905183313, 44.33599401400001)(40.80910240202276, 40.562238156)(40.88495575221239, 41.208437686)(40.96080910240203, 43.860591568000004)(41.036662452591656, 51.456366061)(41.112515802781296, 43.979576871)(41.18836915297093, 45.721114621)(41.26422250316056, 44.318579548)(41.34007585335019, 45.793239202)(41.415929203539825, 45.04932779)(41.49178255372946, 43.927953742)(41.5676359039191, 39.480802987)(41.64348925410873, 18.090382382)(41.71934260429836, 56.107845577999996)(41.795195954487994, 17.261874735)(41.871049304677626, 54.305270527999994)(41.94690265486726, 39.404150598)(42.02275600505689, 51.423609976)(42.09860935524653, 40.050566003)(42.174462705436156, 46.358341067)(42.250316055625795, 43.873935848)(42.32616940581543, 49.131288696000006)(42.40202275600506, 46.026752974000004)(42.4778761061947, 39.745305125)(42.553729456384325, 45.224059672)(42.629582806573964, 47.269434041)(42.70543615676359, 53.657957929000005)(42.78128950695323, 39.970336543)(42.85714285714286, 46.699319402)(42.932996207332494, 49.10534137)(43.00884955752213, 49.063729584)(43.08470290771176, 46.705948181)(43.1605562579014, 50.374674277000004)(43.23640960809102, 44.930278363999996)(43.31226295828066, 50.362324959000006)(43.388116308470295, 43.117692472)(43.46396965865993, 37.217457784)(43.53982300884956, 49.170151632)(43.61567635903919, 54.157894899)(43.69152970922883, 46.042120713)(43.767383059418464, 40.492187465)(43.843236409608096, 44.830078068)(43.91908975979773, 45.966846691)(43.99494310998736, 36.971915388999996)(44.07079646017699, 39.608711621)(44.14664981036663, 34.945086178000004)(44.22250316055626, 45.529813874)(44.2983565107459, 36.7018301)(44.37420986093552, 15.310289566)(44.45006321112516, 13.040337119)(44.5259165613148, 29.013299861)(44.60176991150443, 13.190474656)(44.677623261694066, 13.691158901)(44.75347661188369, 47.050245716)(44.82932996207333, 14.945697166)(44.90518331226296, 58.67770783)(44.981036662452595, 59.11220968400001)(45.05689001264223, 40.973846708)(45.13274336283186, 48.911954226)(45.2085967130215, 39.657216571)(45.284450063211125, 52.572381439999994)(45.360303413400764, 36.550707012000004)(45.4361567635904, 41.669184408999996)(45.51201011378003, 45.969393013)(45.58786346396966, 45.298574077)(45.663716814159294, 41.385416305)(45.739570164348926, 48.855054171)(45.815423514538566, 51.546315469)(45.8912768647282, 37.12191377)(45.96713021491783, 45.084659083999995)(46.04298356510746, 46.971199952999996)(46.118836915297095, 35.023423027999996)(46.19469026548673, 34.204661603000005)(46.27054361567636, 47.847962337)(46.346396965866, 42.568425936)(46.422250316055624, 35.061842972)(46.498103666245264, 41.896226111)(46.573957016434896, 41.289008708)(46.64981036662453, 52.807738498)(46.72566371681417, 35.592796043999996)(46.80151706700379, 47.331427564)(46.87737041719343, 45.758215221)(46.95322376738306, 52.898032051)(47.0290771175727, 43.918173548999995)(47.10493046776233, 47.720121523)(47.18078381795196, 49.114440114000004)(47.2566371681416, 36.164966242)(47.33249051833123, 45.352966529999996)(47.408343868520866, 52.994765679000004)(47.4841972187105, 56.438348125999994)(47.56005056890013, 42.974852974)(47.63590391908976, 43.972114104999996)(47.711757269279396, 43.544779673)(47.78761061946903, 50.545109941)(47.86346396965867, 44.607846642)(47.9393173198483, 43.719306453)(48.01517067003793, 49.766360164)(48.091024020227565, 38.426260266)(48.1668773704172, 44.789561119999995)(48.24273072060683, 38.663443877999995)(48.31858407079646, 43.873017783)(48.3944374209861, 37.59301162)(48.470290771175726, 48.863272935)(48.546144121365366, 45.371438092)(48.621997471555, 46.830119302)(48.69785082174463, 48.09756523)(48.77370417193427, 43.865736485)(48.849557522123895, 49.486600005)(48.925410872313535, 38.495201819)(49.00126422250316, 37.992632383)(49.0771175726928, 39.468765659999995)(49.15297092288243, 51.30147406)(49.228824273072064, 45.217240136)(49.3046776232617, 43.631405347000005)(49.38053097345133, 37.413833944000004)(49.45638432364097, 51.361988214)(49.53223767383059, 40.562370484)(49.60809102402023, 48.612151516)(49.683944374209865, 42.108128168)(49.7597977243995, 42.012683640999995)(49.83565107458913, 47.092620916)(49.91150442477876, 52.388323969000005)(49.9873577749684, 37.312003597)(50.063211125158034, 38.687265814999996)(50.139064475347666, 46.13479714500001)(50.2149178255373, 44.9595006)(50.29077117572693, 16.534064912999998)(50.36662452591656, 37.86738861)(50.4424778761062, 24.705761011999996)(50.51833122629583, 55.684768319)(50.59418457648547, 39.91497516)(50.6700379266751, 38.049096763)(50.74589127686473, 48.577425100999996)(50.82174462705437, 43.999865858)(50.897597977244, 48.46465594599999)(50.97345132743364, 43.187578016)(51.04930467762326, 42.98370459)(51.1251580278129, 57.513677187)(51.20101137800253, 52.589418425)(51.276864728192166, 44.806944318999996)(51.352718078381805, 42.232246984)(51.42857142857143, 48.864255718)(51.50442477876107, 45.88761714100001)(51.580278128950695, 45.655464015999996)(51.656131479140335, 37.333984846)(51.73198482932997, 52.587347139)(51.8078381795196, 13.71439185)(51.88369152970923, 19.118541519)(51.959544879898864, 17.50818094)(52.035398230088504, 24.911613622)(52.111251580278136, 15.035636982)(52.18710493046777, 27.171130066)(52.2629582806574, 14.657140680000001)(52.33881163084703, 40.352671720000004)(52.414664981036665, 47.750127398000004)(52.4905183312263, 54.386902164000006)(52.56637168141593, 106.22622827200001)(52.64222503160557, 107.118803528)(52.7180783817952, 100.17219821399999)(52.793931731984834, 100.46277358699999)(52.86978508217447, 103.173278149)(52.9456384323641, 106.22820269900001)(53.02149178255374, 105.39075886699999)(53.097345132743364, 106.762209981)(53.173198482933, 106.43230039800001)(53.24905183312263, 104.740957451)(53.32490518331227, 104.68585039800001)(53.4007585335019, 105.888446284)(53.47661188369153, 107.838189566)(53.55246523388117, 104.809019128)(53.6283185840708, 106.080161163)(53.70417193426044, 105.82346770499998)(53.78002528445007, 105.44551113)(53.8558786346397, 106.481421832)(53.931731984829334, 105.96592037100001)(54.007585335018966, 105.76660781499999)(54.0834386852086, 104.346775744)(54.15929203539824, 107.02436949000001)(54.23514538558787, 107.89337250499999)(54.3109987357775, 106.37539989700001)(54.386852085967135, 105.390465885)(54.46270543615677, 106.017264244)(54.5385587863464, 104.123852913)(54.61441213653603, 104.39643271899999)(54.69026548672567, 106.608766454)(54.7661188369153, 105.313244118)(54.841972187104936, 105.16393120000001)(54.91782553729457, 105.828586981)(54.9936788874842, 104.512047729)(55.06953223767384, 104.56099431300001)(55.145385587863466, 106.092773319)(55.221238938053105, 105.878032949)(55.29709228824273, 105.222117459)(55.37294563843237, 106.195434534)(55.448798988622, 104.888747599)(55.524652338811634, 106.310285841)(55.600505689001274, 106.41952545000001)(55.6763590391909, 106.43827463099998)(55.75221238938054, 105.52375376)(55.828065739570164, 105.176745868)(55.9039190897598, 106.201585809)(55.979772439949436, 101.473788235)(56.05562579013907, 105.28739829099999)(56.1314791403287, 101.692329336)(56.20733249051833, 108.128393582)(56.28318584070797, 105.861125455)(56.359039190897604, 104.68534827699999)(56.43489254108724, 108.749914802)(56.51074589127687, 105.708160139)(56.5865992414665, 105.38128665100001)(56.662452591656134, 105.54211256900001)(56.73830594184577, 105.312356191)(56.8141592920354, 105.308716076)(56.89001264222504, 105.548327565)(56.96586599241467, 105.941258558)(57.0417193426043, 106.037746311)(57.11757269279394, 105.548458513)(57.19342604298357, 105.437408625)(57.26927939317321, 105.794552722)(57.34513274336283, 110.468799903)(57.42098609355247, 105.92084124799999)(57.496839443742104, 103.048106608)(57.572692793931736, 105.69496253300001)(57.648546144121376, 105.423128311)(57.724399494311, 105.827116946)(57.80025284450064, 104.342480341)(57.876106194690266, 105.442508226)(57.951959544879905, 107.419906338)(58.02781289506954, 106.712287829)(58.10366624525917, 106.108780208)(58.1795195954488, 104.593476701)(58.255372945638435, 107.16351327699999)(58.331226295828074, 104.54839941499999)(58.407079646017706, 105.06910306)(58.48293299620734, 103.70350178700001)(58.55878634639697, 103.22558217099999)(58.6346396965866, 103.31456894200001)(58.710493046776236, 104.751774386)(58.78634639696587, 105.825144828)(58.8621997471555, 106.46723717500001)(58.93805309734514, 107.69228284)(59.01390644753477, 104.41576853000001)(59.089759797724405, 104.538384394)(59.16561314791404, 104.82516561700001)(59.24146649810367, 103.590742078)(59.31731984829331, 104.08612032900001)(59.393173198482934, 104.136943647)(59.46902654867257, 104.571130311)(59.5448798988622, 102.750323283)(59.62073324905184, 103.620999086)(59.69658659924148, 104.797008661)(59.7724399494311, 105.107100493)(59.84829329962074, 102.649525665)(59.92414664981037, 103.68518040699999)(60.0, 103.207670183)
        };
        \addplot [pattern=north east lines,pattern color=red] 
        fill between [
            of=A and B,soft clip={domain=0:800},
        ];
        \end{axis}
\end{tikzpicture}
\caption{Measuring instrument: Clamp(Lin)}\label{fig:time_series_BinaryTrees_Workstation_ClampL}
\end{subfigure}
\caption{BinaryTress on the workstation measured by the different measuring instrumentst, with the lines representing the minimum, maximum and average energy consumption}\label{fig:time_series_BinaryTrees_Workstation}
\end{figure}








%\subsection{Binarytress}
%
                            \begin{figure}
                                \centering
                                \begin{tikzpicture}[]
                                    \pgfplotsset{%
                                        width=.85\textwidth,
                                        height=.15\textheight
                                    }
                                    \begin{axis}[xlabel={Average energy consumption (Watts)}, title={Cores - BinaryTrees - Energy - without outliers}, ytick={},
                                    yticklabels={
                                        
                                        },
                                        xmin=0,xmax=20,
                                        ]
                                    
                                    \end{axis}
                                \end{tikzpicture}
                            \caption{A comparison of of Cores energy consumption for test case BinaryTrees for the Surface4Pro,  experiment \#2 (without outliers)} \label{fig:BinaryTrees_Cores_comparison_energy_without_outliers_Surface4Pro_avg_watts_exp2}
                            \end{figure}
                            
%
                            \begin{figure}
                                \centering
                                \begin{tikzpicture}[]
                                    \pgfplotsset{%
                                        width=.85\textwidth,
                                        height=.15\textheight
                                    }
                                    \begin{axis}[xlabel={Average energy consumption (Watts)}, title={Cores - BinaryTrees - Energy - without outliers}, ytick={},
                                    yticklabels={
                                        
                                        },
                                        xmin=0,xmax=20,
                                        ]
                                    
                                    \end{axis}
                                \end{tikzpicture}
                            \caption{A comparison of of Cores energy consumption for test case BinaryTrees for the Surface4Pro,  experiment \#2 (without outliers)} \label{fig:BinaryTrees_Cores_comparison_energy_without_outliers_Surface4Pro_avg_watts_exp2}
                            \end{figure}
                            
%
                            \begin{figure}
                                \centering
                                \begin{tikzpicture}[]
                                    \pgfplotsset{%
                                        width=.85\textwidth,
                                        height=.15\textheight
                                    }
                                    \begin{axis}[xlabel={Average energy consumption (Watts)}, title={Cores - BinaryTrees - Energy - without outliers}, ytick={},
                                    yticklabels={
                                        
                                        },
                                        xmin=0,xmax=20,
                                        ]
                                    
                                    \end{axis}
                                \end{tikzpicture}
                            \caption{A comparison of of Cores energy consumption for test case BinaryTrees for the Surface4Pro,  experiment \#2 (without outliers)} \label{fig:BinaryTrees_Cores_comparison_energy_without_outliers_Surface4Pro_avg_watts_exp2}
                            \end{figure}
                            
%
                            \begin{figure}
                                \centering
                                \begin{tikzpicture}[]
                                    \pgfplotsset{%
                                        width=.85\textwidth,
                                        height=.15\textheight
                                    }
                                    \begin{axis}[xlabel={Average energy consumption (Watts)}, title={Cores - BinaryTrees - Energy - without outliers}, ytick={},
                                    yticklabels={
                                        
                                        },
                                        xmin=0,xmax=20,
                                        ]
                                    
                                    \end{axis}
                                \end{tikzpicture}
                            \caption{A comparison of of Cores energy consumption for test case BinaryTrees for the Surface4Pro,  experiment \#2 (without outliers)} \label{fig:BinaryTrees_Cores_comparison_energy_without_outliers_Surface4Pro_avg_watts_exp2}
                            \end{figure}
                            
%
                            \begin{figure}
                                \centering
                                \begin{tikzpicture}[]
                                    \pgfplotsset{%
                                        width=.85\textwidth,
                                        height=.15\textheight
                                    }
                                    \begin{axis}[xlabel={Average energy consumption (Watts)}, title={Cores - BinaryTrees - Energy - without outliers}, ytick={},
                                    yticklabels={
                                        
                                        },
                                        xmin=0,xmax=20,
                                        ]
                                    
                                    \end{axis}
                                \end{tikzpicture}
                            \caption{A comparison of of Cores energy consumption for test case BinaryTrees for the Surface4Pro,  experiment \#2 (without outliers)} \label{fig:BinaryTrees_Cores_comparison_energy_without_outliers_Surface4Pro_avg_watts_exp2}
                            \end{figure}
                            
%
                            \begin{figure}
                                \centering
                                \begin{tikzpicture}[]
                                    \pgfplotsset{%
                                        width=.85\textwidth,
                                        height=.15\textheight
                                    }
                                    \begin{axis}[xlabel={Average energy consumption (Watts)}, title={Cores - BinaryTrees - Energy - without outliers}, ytick={},
                                    yticklabels={
                                        
                                        },
                                        xmin=0,xmax=20,
                                        ]
                                    
                                    \end{axis}
                                \end{tikzpicture}
                            \caption{A comparison of of Cores energy consumption for test case BinaryTrees for the Surface4Pro,  experiment \#2 (without outliers)} \label{fig:BinaryTrees_Cores_comparison_energy_without_outliers_Surface4Pro_avg_watts_exp2}
                            \end{figure}
                            
%
                            \begin{figure}
                                \centering
                                \begin{tikzpicture}[]
                                    \pgfplotsset{%
                                        width=.85\textwidth,
                                        height=.15\textheight
                                    }
                                    \begin{axis}[xlabel={Average energy consumption (Watts)}, title={Cores - BinaryTrees - Energy - without outliers}, ytick={},
                                    yticklabels={
                                        
                                        },
                                        xmin=0,xmax=20,
                                        ]
                                    
                                    \end{axis}
                                \end{tikzpicture}
                            \caption{A comparison of of Cores energy consumption for test case BinaryTrees for the Surface4Pro,  experiment \#2 (without outliers)} \label{fig:BinaryTrees_Cores_comparison_energy_without_outliers_Surface4Pro_avg_watts_exp2}
                            \end{figure}
                            
%
                            \begin{figure}
                                \centering
                                \begin{tikzpicture}[]
                                    \pgfplotsset{%
                                        width=.85\textwidth,
                                        height=.15\textheight
                                    }
                                    \begin{axis}[xlabel={Average energy consumption (Watts)}, title={Cores - BinaryTrees - Energy - without outliers}, ytick={},
                                    yticklabels={
                                        
                                        },
                                        xmin=0,xmax=20,
                                        ]
                                    
                                    \end{axis}
                                \end{tikzpicture}
                            \caption{A comparison of of Cores energy consumption for test case BinaryTrees for the Surface4Pro,  experiment \#2 (without outliers)} \label{fig:BinaryTrees_Cores_comparison_energy_without_outliers_Surface4Pro_avg_watts_exp2}
                            \end{figure}
                            
%
                            \begin{figure}
                                \centering
                                \begin{tikzpicture}[]
                                    \pgfplotsset{%
                                        width=.85\textwidth,
                                        height=.15\textheight
                                    }
                                    \begin{axis}[xlabel={Average energy consumption (Watts)}, title={Cores - BinaryTrees - Energy - without outliers}, ytick={},
                                    yticklabels={
                                        
                                        },
                                        xmin=0,xmax=20,
                                        ]
                                    
                                    \end{axis}
                                \end{tikzpicture}
                            \caption{A comparison of of Cores energy consumption for test case BinaryTrees for the Surface4Pro,  experiment \#2 (without outliers)} \label{fig:BinaryTrees_Cores_comparison_energy_without_outliers_Surface4Pro_avg_watts_exp2}
                            \end{figure}
                            
%
                            \begin{figure}
                                \centering
                                \begin{tikzpicture}[]
                                    \pgfplotsset{%
                                        width=.85\textwidth,
                                        height=.15\textheight
                                    }
                                    \begin{axis}[xlabel={Average energy consumption (Watts)}, title={Cores - BinaryTrees - Energy - without outliers}, ytick={},
                                    yticklabels={
                                        
                                        },
                                        xmin=0,xmax=20,
                                        ]
                                    
                                    \end{axis}
                                \end{tikzpicture}
                            \caption{A comparison of of Cores energy consumption for test case BinaryTrees for the Surface4Pro,  experiment \#2 (without outliers)} \label{fig:BinaryTrees_Cores_comparison_energy_without_outliers_Surface4Pro_avg_watts_exp2}
                            \end{figure}
                            
%
                            \begin{figure}
                                \centering
                                \begin{tikzpicture}[]
                                    \pgfplotsset{%
                                        width=.85\textwidth,
                                        height=.15\textheight
                                    }
                                    \begin{axis}[xlabel={Average energy consumption (Watts)}, title={Cores - BinaryTrees - Energy - without outliers}, ytick={},
                                    yticklabels={
                                        
                                        },
                                        xmin=0,xmax=20,
                                        ]
                                    
                                    \end{axis}
                                \end{tikzpicture}
                            \caption{A comparison of of Cores energy consumption for test case BinaryTrees for the Surface4Pro,  experiment \#2 (without outliers)} \label{fig:BinaryTrees_Cores_comparison_energy_without_outliers_Surface4Pro_avg_watts_exp2}
                            \end{figure}
                            
%\subsection{FannkuchRedux}
%
                            \begin{figure}
                                \centering
                                \begin{tikzpicture}[]
                                    \pgfplotsset{%
                                        width=.85\textwidth,
                                        height=.15\textheight
                                    }
                                    \begin{axis}[xlabel={Average energy consumption (Watts)}, title={Cores - FannkuchRedux - Energy - without outliers}, ytick={},
                                    yticklabels={
                                        
                                        },
                                        xmin=0,xmax=20,
                                        ]
                                    
                                    \end{axis}
                                \end{tikzpicture}
                            \caption{A comparison of of Cores energy consumption for test case FannkuchRedux for the Surface4Pro,  experiment \#2 (without outliers)} \label{fig:FannkuchRedux_Cores_comparison_energy_without_outliers_Surface4Pro_avg_watts_exp2}
                            \end{figure}
                            
%
                            \begin{figure}
                                \centering
                                \begin{tikzpicture}[]
                                    \pgfplotsset{%
                                        width=.85\textwidth,
                                        height=.15\textheight
                                    }
                                    \begin{axis}[xlabel={Average energy consumption (Watts)}, title={Cores - FannkuchRedux - Energy - without outliers}, ytick={},
                                    yticklabels={
                                        
                                        },
                                        xmin=0,xmax=20,
                                        ]
                                    
                                    \end{axis}
                                \end{tikzpicture}
                            \caption{A comparison of of Cores energy consumption for test case FannkuchRedux for the Surface4Pro,  experiment \#2 (without outliers)} \label{fig:FannkuchRedux_Cores_comparison_energy_without_outliers_Surface4Pro_avg_watts_exp2}
                            \end{figure}
                            
%
                            \begin{figure}
                                \centering
                                \begin{tikzpicture}[]
                                    \pgfplotsset{%
                                        width=.85\textwidth,
                                        height=.15\textheight
                                    }
                                    \begin{axis}[xlabel={Average energy consumption (Watts)}, title={Cores - FannkuchRedux - Energy - without outliers}, ytick={},
                                    yticklabels={
                                        
                                        },
                                        xmin=0,xmax=20,
                                        ]
                                    
                                    \end{axis}
                                \end{tikzpicture}
                            \caption{A comparison of of Cores energy consumption for test case FannkuchRedux for the Surface4Pro,  experiment \#2 (without outliers)} \label{fig:FannkuchRedux_Cores_comparison_energy_without_outliers_Surface4Pro_avg_watts_exp2}
                            \end{figure}
                            
%
                            \begin{figure}
                                \centering
                                \begin{tikzpicture}[]
                                    \pgfplotsset{%
                                        width=.85\textwidth,
                                        height=.15\textheight
                                    }
                                    \begin{axis}[xlabel={Average energy consumption (Watts)}, title={Cores - FannkuchRedux - Energy - without outliers}, ytick={},
                                    yticklabels={
                                        
                                        },
                                        xmin=0,xmax=20,
                                        ]
                                    
                                    \end{axis}
                                \end{tikzpicture}
                            \caption{A comparison of of Cores energy consumption for test case FannkuchRedux for the Surface4Pro,  experiment \#2 (without outliers)} \label{fig:FannkuchRedux_Cores_comparison_energy_without_outliers_Surface4Pro_avg_watts_exp2}
                            \end{figure}
                            
%
                            \begin{figure}
                                \centering
                                \begin{tikzpicture}[]
                                    \pgfplotsset{%
                                        width=.85\textwidth,
                                        height=.15\textheight
                                    }
                                    \begin{axis}[xlabel={Average energy consumption (Watts)}, title={Cores - FannkuchRedux - Energy - without outliers}, ytick={},
                                    yticklabels={
                                        
                                        },
                                        xmin=0,xmax=20,
                                        ]
                                    
                                    \end{axis}
                                \end{tikzpicture}
                            \caption{A comparison of of Cores energy consumption for test case FannkuchRedux for the Surface4Pro,  experiment \#2 (without outliers)} \label{fig:FannkuchRedux_Cores_comparison_energy_without_outliers_Surface4Pro_avg_watts_exp2}
                            \end{figure}
                            
%
                            \begin{figure}
                                \centering
                                \begin{tikzpicture}[]
                                    \pgfplotsset{%
                                        width=.85\textwidth,
                                        height=.15\textheight
                                    }
                                    \begin{axis}[xlabel={Average energy consumption (Watts)}, title={Cores - FannkuchRedux - Energy - without outliers}, ytick={},
                                    yticklabels={
                                        
                                        },
                                        xmin=0,xmax=20,
                                        ]
                                    
                                    \end{axis}
                                \end{tikzpicture}
                            \caption{A comparison of of Cores energy consumption for test case FannkuchRedux for the Surface4Pro,  experiment \#2 (without outliers)} \label{fig:FannkuchRedux_Cores_comparison_energy_without_outliers_Surface4Pro_avg_watts_exp2}
                            \end{figure}
                            
%
                            \begin{figure}
                                \centering
                                \begin{tikzpicture}[]
                                    \pgfplotsset{%
                                        width=.85\textwidth,
                                        height=.15\textheight
                                    }
                                    \begin{axis}[xlabel={Average energy consumption (Watts)}, title={Cores - FannkuchRedux - Energy - without outliers}, ytick={},
                                    yticklabels={
                                        
                                        },
                                        xmin=0,xmax=20,
                                        ]
                                    
                                    \end{axis}
                                \end{tikzpicture}
                            \caption{A comparison of of Cores energy consumption for test case FannkuchRedux for the Surface4Pro,  experiment \#2 (without outliers)} \label{fig:FannkuchRedux_Cores_comparison_energy_without_outliers_Surface4Pro_avg_watts_exp2}
                            \end{figure}
                            
%
                            \begin{figure}
                                \centering
                                \begin{tikzpicture}[]
                                    \pgfplotsset{%
                                        width=.85\textwidth,
                                        height=.15\textheight
                                    }
                                    \begin{axis}[xlabel={Average energy consumption (Watts)}, title={Cores - FannkuchRedux - Energy - without outliers}, ytick={},
                                    yticklabels={
                                        
                                        },
                                        xmin=0,xmax=20,
                                        ]
                                    
                                    \end{axis}
                                \end{tikzpicture}
                            \caption{A comparison of of Cores energy consumption for test case FannkuchRedux for the Surface4Pro,  experiment \#2 (without outliers)} \label{fig:FannkuchRedux_Cores_comparison_energy_without_outliers_Surface4Pro_avg_watts_exp2}
                            \end{figure}
                            
%
                            \begin{figure}
                                \centering
                                \begin{tikzpicture}[]
                                    \pgfplotsset{%
                                        width=.85\textwidth,
                                        height=.15\textheight
                                    }
                                    \begin{axis}[xlabel={Average energy consumption (Watts)}, title={Cores - FannkuchRedux - Energy - without outliers}, ytick={},
                                    yticklabels={
                                        
                                        },
                                        xmin=0,xmax=20,
                                        ]
                                    
                                    \end{axis}
                                \end{tikzpicture}
                            \caption{A comparison of of Cores energy consumption for test case FannkuchRedux for the Surface4Pro,  experiment \#2 (without outliers)} \label{fig:FannkuchRedux_Cores_comparison_energy_without_outliers_Surface4Pro_avg_watts_exp2}
                            \end{figure}
                            
%
                            \begin{figure}
                                \centering
                                \begin{tikzpicture}[]
                                    \pgfplotsset{%
                                        width=.85\textwidth,
                                        height=.15\textheight
                                    }
                                    \begin{axis}[xlabel={Average energy consumption (Watts)}, title={Cores - FannkuchRedux - Energy - without outliers}, ytick={},
                                    yticklabels={
                                        
                                        },
                                        xmin=0,xmax=20,
                                        ]
                                    
                                    \end{axis}
                                \end{tikzpicture}
                            \caption{A comparison of of Cores energy consumption for test case FannkuchRedux for the Surface4Pro,  experiment \#2 (without outliers)} \label{fig:FannkuchRedux_Cores_comparison_energy_without_outliers_Surface4Pro_avg_watts_exp2}
                            \end{figure}
                            
%
                            \begin{figure}
                                \centering
                                \begin{tikzpicture}[]
                                    \pgfplotsset{%
                                        width=.85\textwidth,
                                        height=.15\textheight
                                    }
                                    \begin{axis}[xlabel={Average energy consumption (Watts)}, title={Cores - FannkuchRedux - Energy - without outliers}, ytick={},
                                    yticklabels={
                                        
                                        },
                                        xmin=0,xmax=20,
                                        ]
                                    
                                    \end{axis}
                                \end{tikzpicture}
                            \caption{A comparison of of Cores energy consumption for test case FannkuchRedux for the Surface4Pro,  experiment \#2 (without outliers)} \label{fig:FannkuchRedux_Cores_comparison_energy_without_outliers_Surface4Pro_avg_watts_exp2}
                            \end{figure}
                            
%\subsection{Fasta}
%
                            \begin{figure}
                                \centering
                                \begin{tikzpicture}[]
                                    \pgfplotsset{%
                                        width=.7\textwidth,
                                        height=.2\textheight
                                    }
                                    \begin{axis}[xlabel={Average energy consumption (Watts)}, title={Cores - Fasta - Energy - without outliers}, ytick={1, 2},
                                    yticklabels={
                                        IntelPowerGadget , HardwareMonitor 
                                        },
                                        xmin=0,xmax=80,
                                        ]
                                    
                                    \addplot+ [boxplot prepared={
                                    lower whisker=54.22445449466433,
                                    lower quartile=54.45751347572185,
                                    median=54.54624547297937,
                                    upper quartile=54.72020125409266,
                                    upper whisker=55.103791721157386},
                                    ] table[row sep=\\,y index=0] {\\};
                                    
                                    \addplot+ [boxplot prepared={
                                    lower whisker=51.45082608392138,
                                    lower quartile=51.933860038023106,
                                    median=52.121433545941585,
                                    upper quartile=52.479201309170854,
                                    upper whisker=54.95103920614709},
                                    ] table[row sep=\\,y index=0] {\\};
                                    
                                    \end{axis}
                                \end{tikzpicture}
                            \caption{A comparison of of Cores energy consumption for test case Fasta for the workstation (without outliers)} \label{fig:Fasta_Cores_comparison_energy_without_outliers_PowerKomplett_avg_watts_exp2}
                            \end{figure}
                            
%
                            \begin{figure}
                                \centering
                                \begin{tikzpicture}[]
                                    \pgfplotsset{%
                                        width=.7\textwidth,
                                        height=.2\textheight
                                    }
                                    \begin{axis}[xlabel={Average energy consumption (Watts)}, title={Cores - Fasta - Energy - without outliers}, ytick={1, 2},
                                    yticklabels={
                                        IntelPowerGadget , HardwareMonitor 
                                        },
                                        xmin=0,xmax=80,
                                        ]
                                    
                                    \addplot+ [boxplot prepared={
                                    lower whisker=54.22445449466433,
                                    lower quartile=54.45751347572185,
                                    median=54.54624547297937,
                                    upper quartile=54.72020125409266,
                                    upper whisker=55.103791721157386},
                                    ] table[row sep=\\,y index=0] {\\};
                                    
                                    \addplot+ [boxplot prepared={
                                    lower whisker=51.45082608392138,
                                    lower quartile=51.933860038023106,
                                    median=52.121433545941585,
                                    upper quartile=52.479201309170854,
                                    upper whisker=54.95103920614709},
                                    ] table[row sep=\\,y index=0] {\\};
                                    
                                    \end{axis}
                                \end{tikzpicture}
                            \caption{A comparison of of Cores energy consumption for test case Fasta for the workstation (without outliers)} \label{fig:Fasta_Cores_comparison_energy_without_outliers_PowerKomplett_avg_watts_exp2}
                            \end{figure}
                            
%
                            \begin{figure}
                                \centering
                                \begin{tikzpicture}[]
                                    \pgfplotsset{%
                                        width=.7\textwidth,
                                        height=.2\textheight
                                    }
                                    \begin{axis}[xlabel={Average energy consumption (Watts)}, title={Cores - Fasta - Energy - without outliers}, ytick={1, 2},
                                    yticklabels={
                                        IntelPowerGadget , HardwareMonitor 
                                        },
                                        xmin=0,xmax=80,
                                        ]
                                    
                                    \addplot+ [boxplot prepared={
                                    lower whisker=54.22445449466433,
                                    lower quartile=54.45751347572185,
                                    median=54.54624547297937,
                                    upper quartile=54.72020125409266,
                                    upper whisker=55.103791721157386},
                                    ] table[row sep=\\,y index=0] {\\};
                                    
                                    \addplot+ [boxplot prepared={
                                    lower whisker=51.45082608392138,
                                    lower quartile=51.933860038023106,
                                    median=52.121433545941585,
                                    upper quartile=52.479201309170854,
                                    upper whisker=54.95103920614709},
                                    ] table[row sep=\\,y index=0] {\\};
                                    
                                    \end{axis}
                                \end{tikzpicture}
                            \caption{A comparison of of Cores energy consumption for test case Fasta for the workstation (without outliers)} \label{fig:Fasta_Cores_comparison_energy_without_outliers_PowerKomplett_avg_watts_exp2}
                            \end{figure}
                            
%
                            \begin{figure}
                                \centering
                                \begin{tikzpicture}[]
                                    \pgfplotsset{%
                                        width=.7\textwidth,
                                        height=.2\textheight
                                    }
                                    \begin{axis}[xlabel={Average energy consumption (Watts)}, title={Cores - Fasta - Energy - without outliers}, ytick={1, 2},
                                    yticklabels={
                                        IntelPowerGadget , HardwareMonitor 
                                        },
                                        xmin=0,xmax=80,
                                        ]
                                    
                                    \addplot+ [boxplot prepared={
                                    lower whisker=54.22445449466433,
                                    lower quartile=54.45751347572185,
                                    median=54.54624547297937,
                                    upper quartile=54.72020125409266,
                                    upper whisker=55.103791721157386},
                                    ] table[row sep=\\,y index=0] {\\};
                                    
                                    \addplot+ [boxplot prepared={
                                    lower whisker=51.45082608392138,
                                    lower quartile=51.933860038023106,
                                    median=52.121433545941585,
                                    upper quartile=52.479201309170854,
                                    upper whisker=54.95103920614709},
                                    ] table[row sep=\\,y index=0] {\\};
                                    
                                    \end{axis}
                                \end{tikzpicture}
                            \caption{A comparison of of Cores energy consumption for test case Fasta for the workstation (without outliers)} \label{fig:Fasta_Cores_comparison_energy_without_outliers_PowerKomplett_avg_watts_exp2}
                            \end{figure}
                            
%
                            \begin{figure}
                                \centering
                                \begin{tikzpicture}[]
                                    \pgfplotsset{%
                                        width=.7\textwidth,
                                        height=.2\textheight
                                    }
                                    \begin{axis}[xlabel={Average energy consumption (Watts)}, title={Cores - Fasta - Energy - without outliers}, ytick={1, 2},
                                    yticklabels={
                                        IntelPowerGadget , HardwareMonitor 
                                        },
                                        xmin=0,xmax=80,
                                        ]
                                    
                                    \addplot+ [boxplot prepared={
                                    lower whisker=54.22445449466433,
                                    lower quartile=54.45751347572185,
                                    median=54.54624547297937,
                                    upper quartile=54.72020125409266,
                                    upper whisker=55.103791721157386},
                                    ] table[row sep=\\,y index=0] {\\};
                                    
                                    \addplot+ [boxplot prepared={
                                    lower whisker=51.45082608392138,
                                    lower quartile=51.933860038023106,
                                    median=52.121433545941585,
                                    upper quartile=52.479201309170854,
                                    upper whisker=54.95103920614709},
                                    ] table[row sep=\\,y index=0] {\\};
                                    
                                    \end{axis}
                                \end{tikzpicture}
                            \caption{A comparison of of Cores energy consumption for test case Fasta for the workstation (without outliers)} \label{fig:Fasta_Cores_comparison_energy_without_outliers_PowerKomplett_avg_watts_exp2}
                            \end{figure}
                            
%
                            \begin{figure}
                                \centering
                                \begin{tikzpicture}[]
                                    \pgfplotsset{%
                                        width=.7\textwidth,
                                        height=.2\textheight
                                    }
                                    \begin{axis}[xlabel={Average energy consumption (Watts)}, title={Cores - Fasta - Energy - without outliers}, ytick={1, 2},
                                    yticklabels={
                                        IntelPowerGadget , HardwareMonitor 
                                        },
                                        xmin=0,xmax=80,
                                        ]
                                    
                                    \addplot+ [boxplot prepared={
                                    lower whisker=54.22445449466433,
                                    lower quartile=54.45751347572185,
                                    median=54.54624547297937,
                                    upper quartile=54.72020125409266,
                                    upper whisker=55.103791721157386},
                                    ] table[row sep=\\,y index=0] {\\};
                                    
                                    \addplot+ [boxplot prepared={
                                    lower whisker=51.45082608392138,
                                    lower quartile=51.933860038023106,
                                    median=52.121433545941585,
                                    upper quartile=52.479201309170854,
                                    upper whisker=54.95103920614709},
                                    ] table[row sep=\\,y index=0] {\\};
                                    
                                    \end{axis}
                                \end{tikzpicture}
                            \caption{A comparison of of Cores energy consumption for test case Fasta for the workstation (without outliers)} \label{fig:Fasta_Cores_comparison_energy_without_outliers_PowerKomplett_avg_watts_exp2}
                            \end{figure}
                            
%
                            \begin{figure}
                                \centering
                                \begin{tikzpicture}[]
                                    \pgfplotsset{%
                                        width=.7\textwidth,
                                        height=.2\textheight
                                    }
                                    \begin{axis}[xlabel={Average energy consumption (Watts)}, title={Cores - Fasta - Energy - without outliers}, ytick={1, 2},
                                    yticklabels={
                                        IntelPowerGadget , HardwareMonitor 
                                        },
                                        xmin=0,xmax=80,
                                        ]
                                    
                                    \addplot+ [boxplot prepared={
                                    lower whisker=54.22445449466433,
                                    lower quartile=54.45751347572185,
                                    median=54.54624547297937,
                                    upper quartile=54.72020125409266,
                                    upper whisker=55.103791721157386},
                                    ] table[row sep=\\,y index=0] {\\};
                                    
                                    \addplot+ [boxplot prepared={
                                    lower whisker=51.45082608392138,
                                    lower quartile=51.933860038023106,
                                    median=52.121433545941585,
                                    upper quartile=52.479201309170854,
                                    upper whisker=54.95103920614709},
                                    ] table[row sep=\\,y index=0] {\\};
                                    
                                    \end{axis}
                                \end{tikzpicture}
                            \caption{A comparison of of Cores energy consumption for test case Fasta for the workstation (without outliers)} \label{fig:Fasta_Cores_comparison_energy_without_outliers_PowerKomplett_avg_watts_exp2}
                            \end{figure}
                            
%
                            \begin{figure}
                                \centering
                                \begin{tikzpicture}[]
                                    \pgfplotsset{%
                                        width=.7\textwidth,
                                        height=.2\textheight
                                    }
                                    \begin{axis}[xlabel={Average energy consumption (Watts)}, title={Cores - Fasta - Energy - without outliers}, ytick={1, 2},
                                    yticklabels={
                                        IntelPowerGadget , HardwareMonitor 
                                        },
                                        xmin=0,xmax=80,
                                        ]
                                    
                                    \addplot+ [boxplot prepared={
                                    lower whisker=54.22445449466433,
                                    lower quartile=54.45751347572185,
                                    median=54.54624547297937,
                                    upper quartile=54.72020125409266,
                                    upper whisker=55.103791721157386},
                                    ] table[row sep=\\,y index=0] {\\};
                                    
                                    \addplot+ [boxplot prepared={
                                    lower whisker=51.45082608392138,
                                    lower quartile=51.933860038023106,
                                    median=52.121433545941585,
                                    upper quartile=52.479201309170854,
                                    upper whisker=54.95103920614709},
                                    ] table[row sep=\\,y index=0] {\\};
                                    
                                    \end{axis}
                                \end{tikzpicture}
                            \caption{A comparison of of Cores energy consumption for test case Fasta for the workstation (without outliers)} \label{fig:Fasta_Cores_comparison_energy_without_outliers_PowerKomplett_avg_watts_exp2}
                            \end{figure}
                            
%
                            \begin{figure}
                                \centering
                                \begin{tikzpicture}[]
                                    \pgfplotsset{%
                                        width=.7\textwidth,
                                        height=.2\textheight
                                    }
                                    \begin{axis}[xlabel={Average energy consumption (Watts)}, title={Cores - Fasta - Energy - without outliers}, ytick={1, 2},
                                    yticklabels={
                                        IntelPowerGadget , HardwareMonitor 
                                        },
                                        xmin=0,xmax=80,
                                        ]
                                    
                                    \addplot+ [boxplot prepared={
                                    lower whisker=54.22445449466433,
                                    lower quartile=54.45751347572185,
                                    median=54.54624547297937,
                                    upper quartile=54.72020125409266,
                                    upper whisker=55.103791721157386},
                                    ] table[row sep=\\,y index=0] {\\};
                                    
                                    \addplot+ [boxplot prepared={
                                    lower whisker=51.45082608392138,
                                    lower quartile=51.933860038023106,
                                    median=52.121433545941585,
                                    upper quartile=52.479201309170854,
                                    upper whisker=54.95103920614709},
                                    ] table[row sep=\\,y index=0] {\\};
                                    
                                    \end{axis}
                                \end{tikzpicture}
                            \caption{A comparison of of Cores energy consumption for test case Fasta for the workstation (without outliers)} \label{fig:Fasta_Cores_comparison_energy_without_outliers_PowerKomplett_avg_watts_exp2}
                            \end{figure}
                            
%
                            \begin{figure}
                                \centering
                                \begin{tikzpicture}[]
                                    \pgfplotsset{%
                                        width=.7\textwidth,
                                        height=.2\textheight
                                    }
                                    \begin{axis}[xlabel={Average energy consumption (Watts)}, title={Cores - Fasta - Energy - without outliers}, ytick={1, 2},
                                    yticklabels={
                                        IntelPowerGadget , HardwareMonitor 
                                        },
                                        xmin=0,xmax=80,
                                        ]
                                    
                                    \addplot+ [boxplot prepared={
                                    lower whisker=54.22445449466433,
                                    lower quartile=54.45751347572185,
                                    median=54.54624547297937,
                                    upper quartile=54.72020125409266,
                                    upper whisker=55.103791721157386},
                                    ] table[row sep=\\,y index=0] {\\};
                                    
                                    \addplot+ [boxplot prepared={
                                    lower whisker=51.45082608392138,
                                    lower quartile=51.933860038023106,
                                    median=52.121433545941585,
                                    upper quartile=52.479201309170854,
                                    upper whisker=54.95103920614709},
                                    ] table[row sep=\\,y index=0] {\\};
                                    
                                    \end{axis}
                                \end{tikzpicture}
                            \caption{A comparison of of Cores energy consumption for test case Fasta for the workstation (without outliers)} \label{fig:Fasta_Cores_comparison_energy_without_outliers_PowerKomplett_avg_watts_exp2}
                            \end{figure}
                            
%
                            \begin{figure}
                                \centering
                                \begin{tikzpicture}[]
                                    \pgfplotsset{%
                                        width=.7\textwidth,
                                        height=.2\textheight
                                    }
                                    \begin{axis}[xlabel={Average energy consumption (Watts)}, title={Cores - Fasta - Energy - without outliers}, ytick={1, 2},
                                    yticklabels={
                                        IntelPowerGadget , HardwareMonitor 
                                        },
                                        xmin=0,xmax=80,
                                        ]
                                    
                                    \addplot+ [boxplot prepared={
                                    lower whisker=54.22445449466433,
                                    lower quartile=54.45751347572185,
                                    median=54.54624547297937,
                                    upper quartile=54.72020125409266,
                                    upper whisker=55.103791721157386},
                                    ] table[row sep=\\,y index=0] {\\};
                                    
                                    \addplot+ [boxplot prepared={
                                    lower whisker=51.45082608392138,
                                    lower quartile=51.933860038023106,
                                    median=52.121433545941585,
                                    upper quartile=52.479201309170854,
                                    upper whisker=54.95103920614709},
                                    ] table[row sep=\\,y index=0] {\\};
                                    
                                    \end{axis}
                                \end{tikzpicture}
                            \caption{A comparison of of Cores energy consumption for test case Fasta for the workstation (without outliers)} \label{fig:Fasta_Cores_comparison_energy_without_outliers_PowerKomplett_avg_watts_exp2}
                            \end{figure}
                            
%\subsection{NBody}
%
                \begin{figure}[H]
                    \centering
                    \begin{tikzpicture}
                        \pgfplotsset{%
                            width=1\textwidth,
                            height=0.4\textheight
                        }
                        \begin{axis}[
                            xlabel={Start battery level},
                            ylabel={Average dynamic energy (watt)},
                            ymin=0,ymax=20,
                        ]
                        
                            \addplot [mark=none, ultra thick, red]  coordinates {
                            (40, 0.006595684402444291)(45, 0.007295220674193181)(50, 0.007999659608910881)(55, 0.007202467971243639)(60, 0.007280865079495958)(65, 0.006858423509955077)(70, 0.008369444141141925)(75, 0.007144940647010992)(80, 0.004753236834232667)
                            };
                            \addlegendentry{Surface4Pro - IntelPowerGadget}
                            
                            \addplot [mark=none, ultra thick, blue]  coordinates {
                            (40, 0.002385328427460912)(45, 0.0017856511573015649)(50, 0.0025901992189954056)(55, 0.00210998144366709)(60, 0.00286452646862575)(65, 0.0020038487280194628)(70, 0.002908483770774353)(75, 0.0005098111150931342)(80, -0.005995916826693459)
                            };
                            \addlegendentry{Surface4Pro - HardwareMonitor}
                            
                            \addplot [mark=none, ultra thick, orange]  coordinates {
                            (50, 251.8661014299577)(55, 214.7597259251014)(60, 162.23880995564176)(65, 107.43421593849766)(70, 53.9187387386668)(75, 0.7554852536149699)(80, -44.463085003568885)
                            };
                            \addlegendentry{Surface4Pro - RAPL}
                            
                            \addplot [mark=none, dashdotted, red]  coordinates {
                            (40, -0.004038062354887025)(45, -0.004312668500277529)(50, -0.003808663911021498)(55, -0.0037057407527755254)(60, -0.004478257932982471)(65, -0.0026308734995501644)(70, -0.0034090674446925874)(75, -0.003041497079697436)(80, -0.0020334307266356875)
                            };
                            \addlegendentry{SurfaceBook - IntelPowerGadget}
                            
                            \addplot [mark=none, dashdotted, blue]  coordinates {
                            (40, -0.002443518930616523)(45, -0.0029256880137447055)(50, -0.002551777773312929)(55, -0.0027567211782433486)(60, -0.002206859154402231)(65, -0.0026388300848279207)(70, -0.0024597736945479324)(75, -0.002445350799195827)(80, -0.000541271265011134)
                            };
                            \addlegendentry{SurfaceBook - HardwareMonitor}
                            
                            \addplot [mark=none, dashdotted, orange]  coordinates {
                            (40, 101.92702155010147)(45, 86.7212700795567)(50, 69.40754598562454)(55, 51.343785407669614)(60, 32.443112755251185)(65, 14.52577919077786)(70, -4.520517378551423)(75, -23.811706977384954)(80, -35.700372165212755)
                            };
                            \addlegendentry{SurfaceBook - RAPL}
                            
                        \end{axis}
                    \end{tikzpicture} 
                \caption{A graph illustrating the energy consumption of Dram for test case Nbody with regards to the battey level of the DUT (with outliers)} \label{fig:Nbody_Dram_charge}
                \end{figure}
                
%
                \begin{figure}[H]
                    \centering
                    \begin{tikzpicture}
                        \pgfplotsset{%
                            width=1\textwidth,
                            height=0.4\textheight
                        }
                        \begin{axis}[
                            xlabel={Start battery level},
                            ylabel={Average dynamic energy (watt)},
                            ymin=0,ymax=20,
                        ]
                        
                            \addplot [mark=none, ultra thick, red]  coordinates {
                            (40, 0.006595684402444291)(45, 0.007295220674193181)(50, 0.007999659608910881)(55, 0.007202467971243639)(60, 0.007280865079495958)(65, 0.006858423509955077)(70, 0.008369444141141925)(75, 0.007144940647010992)(80, 0.004753236834232667)
                            };
                            \addlegendentry{Surface4Pro - IntelPowerGadget}
                            
                            \addplot [mark=none, ultra thick, blue]  coordinates {
                            (40, 0.002385328427460912)(45, 0.0017856511573015649)(50, 0.0025901992189954056)(55, 0.00210998144366709)(60, 0.00286452646862575)(65, 0.0020038487280194628)(70, 0.002908483770774353)(75, 0.0005098111150931342)(80, -0.005995916826693459)
                            };
                            \addlegendentry{Surface4Pro - HardwareMonitor}
                            
                            \addplot [mark=none, ultra thick, orange]  coordinates {
                            (50, 251.8661014299577)(55, 214.7597259251014)(60, 162.23880995564176)(65, 107.43421593849766)(70, 53.9187387386668)(75, 0.7554852536149699)(80, -44.463085003568885)
                            };
                            \addlegendentry{Surface4Pro - RAPL}
                            
                            \addplot [mark=none, dashdotted, red]  coordinates {
                            (40, -0.004038062354887025)(45, -0.004312668500277529)(50, -0.003808663911021498)(55, -0.0037057407527755254)(60, -0.004478257932982471)(65, -0.0026308734995501644)(70, -0.0034090674446925874)(75, -0.003041497079697436)(80, -0.0020334307266356875)
                            };
                            \addlegendentry{SurfaceBook - IntelPowerGadget}
                            
                            \addplot [mark=none, dashdotted, blue]  coordinates {
                            (40, -0.002443518930616523)(45, -0.0029256880137447055)(50, -0.002551777773312929)(55, -0.0027567211782433486)(60, -0.002206859154402231)(65, -0.0026388300848279207)(70, -0.0024597736945479324)(75, -0.002445350799195827)(80, -0.000541271265011134)
                            };
                            \addlegendentry{SurfaceBook - HardwareMonitor}
                            
                            \addplot [mark=none, dashdotted, orange]  coordinates {
                            (40, 101.92702155010147)(45, 86.7212700795567)(50, 69.40754598562454)(55, 51.343785407669614)(60, 32.443112755251185)(65, 14.52577919077786)(70, -4.520517378551423)(75, -23.811706977384954)(80, -35.700372165212755)
                            };
                            \addlegendentry{SurfaceBook - RAPL}
                            
                        \end{axis}
                    \end{tikzpicture} 
                \caption{A graph illustrating the energy consumption of Dram for test case Nbody with regards to the battey level of the DUT (with outliers)} \label{fig:Nbody_Dram_charge}
                \end{figure}
                
%
                \begin{figure}[H]
                    \centering
                    \begin{tikzpicture}
                        \pgfplotsset{%
                            width=1\textwidth,
                            height=0.4\textheight
                        }
                        \begin{axis}[
                            xlabel={Start battery level},
                            ylabel={Average dynamic energy (watt)},
                            ymin=0,ymax=20,
                        ]
                        
                            \addplot [mark=none, ultra thick, red]  coordinates {
                            (40, 0.006595684402444291)(45, 0.007295220674193181)(50, 0.007999659608910881)(55, 0.007202467971243639)(60, 0.007280865079495958)(65, 0.006858423509955077)(70, 0.008369444141141925)(75, 0.007144940647010992)(80, 0.004753236834232667)
                            };
                            \addlegendentry{Surface4Pro - IntelPowerGadget}
                            
                            \addplot [mark=none, ultra thick, blue]  coordinates {
                            (40, 0.002385328427460912)(45, 0.0017856511573015649)(50, 0.0025901992189954056)(55, 0.00210998144366709)(60, 0.00286452646862575)(65, 0.0020038487280194628)(70, 0.002908483770774353)(75, 0.0005098111150931342)(80, -0.005995916826693459)
                            };
                            \addlegendentry{Surface4Pro - HardwareMonitor}
                            
                            \addplot [mark=none, ultra thick, orange]  coordinates {
                            (50, 251.8661014299577)(55, 214.7597259251014)(60, 162.23880995564176)(65, 107.43421593849766)(70, 53.9187387386668)(75, 0.7554852536149699)(80, -44.463085003568885)
                            };
                            \addlegendentry{Surface4Pro - RAPL}
                            
                            \addplot [mark=none, dashdotted, red]  coordinates {
                            (40, -0.004038062354887025)(45, -0.004312668500277529)(50, -0.003808663911021498)(55, -0.0037057407527755254)(60, -0.004478257932982471)(65, -0.0026308734995501644)(70, -0.0034090674446925874)(75, -0.003041497079697436)(80, -0.0020334307266356875)
                            };
                            \addlegendentry{SurfaceBook - IntelPowerGadget}
                            
                            \addplot [mark=none, dashdotted, blue]  coordinates {
                            (40, -0.002443518930616523)(45, -0.0029256880137447055)(50, -0.002551777773312929)(55, -0.0027567211782433486)(60, -0.002206859154402231)(65, -0.0026388300848279207)(70, -0.0024597736945479324)(75, -0.002445350799195827)(80, -0.000541271265011134)
                            };
                            \addlegendentry{SurfaceBook - HardwareMonitor}
                            
                            \addplot [mark=none, dashdotted, orange]  coordinates {
                            (40, 101.92702155010147)(45, 86.7212700795567)(50, 69.40754598562454)(55, 51.343785407669614)(60, 32.443112755251185)(65, 14.52577919077786)(70, -4.520517378551423)(75, -23.811706977384954)(80, -35.700372165212755)
                            };
                            \addlegendentry{SurfaceBook - RAPL}
                            
                        \end{axis}
                    \end{tikzpicture} 
                \caption{A graph illustrating the energy consumption of Dram for test case Nbody with regards to the battey level of the DUT (with outliers)} \label{fig:Nbody_Dram_charge}
                \end{figure}
                
%
                \begin{figure}[H]
                    \centering
                    \begin{tikzpicture}
                        \pgfplotsset{%
                            width=1\textwidth,
                            height=0.4\textheight
                        }
                        \begin{axis}[
                            xlabel={Start battery level},
                            ylabel={Average dynamic energy (watt)},
                            ymin=0,ymax=20,
                        ]
                        
                            \addplot [mark=none, ultra thick, red]  coordinates {
                            (40, 0.006595684402444291)(45, 0.007295220674193181)(50, 0.007999659608910881)(55, 0.007202467971243639)(60, 0.007280865079495958)(65, 0.006858423509955077)(70, 0.008369444141141925)(75, 0.007144940647010992)(80, 0.004753236834232667)
                            };
                            \addlegendentry{Surface4Pro - IntelPowerGadget}
                            
                            \addplot [mark=none, ultra thick, blue]  coordinates {
                            (40, 0.002385328427460912)(45, 0.0017856511573015649)(50, 0.0025901992189954056)(55, 0.00210998144366709)(60, 0.00286452646862575)(65, 0.0020038487280194628)(70, 0.002908483770774353)(75, 0.0005098111150931342)(80, -0.005995916826693459)
                            };
                            \addlegendentry{Surface4Pro - HardwareMonitor}
                            
                            \addplot [mark=none, ultra thick, orange]  coordinates {
                            (50, 251.8661014299577)(55, 214.7597259251014)(60, 162.23880995564176)(65, 107.43421593849766)(70, 53.9187387386668)(75, 0.7554852536149699)(80, -44.463085003568885)
                            };
                            \addlegendentry{Surface4Pro - RAPL}
                            
                            \addplot [mark=none, dashdotted, red]  coordinates {
                            (40, -0.004038062354887025)(45, -0.004312668500277529)(50, -0.003808663911021498)(55, -0.0037057407527755254)(60, -0.004478257932982471)(65, -0.0026308734995501644)(70, -0.0034090674446925874)(75, -0.003041497079697436)(80, -0.0020334307266356875)
                            };
                            \addlegendentry{SurfaceBook - IntelPowerGadget}
                            
                            \addplot [mark=none, dashdotted, blue]  coordinates {
                            (40, -0.002443518930616523)(45, -0.0029256880137447055)(50, -0.002551777773312929)(55, -0.0027567211782433486)(60, -0.002206859154402231)(65, -0.0026388300848279207)(70, -0.0024597736945479324)(75, -0.002445350799195827)(80, -0.000541271265011134)
                            };
                            \addlegendentry{SurfaceBook - HardwareMonitor}
                            
                            \addplot [mark=none, dashdotted, orange]  coordinates {
                            (40, 101.92702155010147)(45, 86.7212700795567)(50, 69.40754598562454)(55, 51.343785407669614)(60, 32.443112755251185)(65, 14.52577919077786)(70, -4.520517378551423)(75, -23.811706977384954)(80, -35.700372165212755)
                            };
                            \addlegendentry{SurfaceBook - RAPL}
                            
                        \end{axis}
                    \end{tikzpicture} 
                \caption{A graph illustrating the energy consumption of Dram for test case Nbody with regards to the battey level of the DUT (with outliers)} \label{fig:Nbody_Dram_charge}
                \end{figure}
                
%
                \begin{figure}[H]
                    \centering
                    \begin{tikzpicture}
                        \pgfplotsset{%
                            width=1\textwidth,
                            height=0.4\textheight
                        }
                        \begin{axis}[
                            xlabel={Start battery level},
                            ylabel={Average dynamic energy (watt)},
                            ymin=0,ymax=20,
                        ]
                        
                            \addplot [mark=none, ultra thick, red]  coordinates {
                            (40, 0.006595684402444291)(45, 0.007295220674193181)(50, 0.007999659608910881)(55, 0.007202467971243639)(60, 0.007280865079495958)(65, 0.006858423509955077)(70, 0.008369444141141925)(75, 0.007144940647010992)(80, 0.004753236834232667)
                            };
                            \addlegendentry{Surface4Pro - IntelPowerGadget}
                            
                            \addplot [mark=none, ultra thick, blue]  coordinates {
                            (40, 0.002385328427460912)(45, 0.0017856511573015649)(50, 0.0025901992189954056)(55, 0.00210998144366709)(60, 0.00286452646862575)(65, 0.0020038487280194628)(70, 0.002908483770774353)(75, 0.0005098111150931342)(80, -0.005995916826693459)
                            };
                            \addlegendentry{Surface4Pro - HardwareMonitor}
                            
                            \addplot [mark=none, ultra thick, orange]  coordinates {
                            (50, 251.8661014299577)(55, 214.7597259251014)(60, 162.23880995564176)(65, 107.43421593849766)(70, 53.9187387386668)(75, 0.7554852536149699)(80, -44.463085003568885)
                            };
                            \addlegendentry{Surface4Pro - RAPL}
                            
                            \addplot [mark=none, dashdotted, red]  coordinates {
                            (40, -0.004038062354887025)(45, -0.004312668500277529)(50, -0.003808663911021498)(55, -0.0037057407527755254)(60, -0.004478257932982471)(65, -0.0026308734995501644)(70, -0.0034090674446925874)(75, -0.003041497079697436)(80, -0.0020334307266356875)
                            };
                            \addlegendentry{SurfaceBook - IntelPowerGadget}
                            
                            \addplot [mark=none, dashdotted, blue]  coordinates {
                            (40, -0.002443518930616523)(45, -0.0029256880137447055)(50, -0.002551777773312929)(55, -0.0027567211782433486)(60, -0.002206859154402231)(65, -0.0026388300848279207)(70, -0.0024597736945479324)(75, -0.002445350799195827)(80, -0.000541271265011134)
                            };
                            \addlegendentry{SurfaceBook - HardwareMonitor}
                            
                            \addplot [mark=none, dashdotted, orange]  coordinates {
                            (40, 101.92702155010147)(45, 86.7212700795567)(50, 69.40754598562454)(55, 51.343785407669614)(60, 32.443112755251185)(65, 14.52577919077786)(70, -4.520517378551423)(75, -23.811706977384954)(80, -35.700372165212755)
                            };
                            \addlegendentry{SurfaceBook - RAPL}
                            
                        \end{axis}
                    \end{tikzpicture} 
                \caption{A graph illustrating the energy consumption of Dram for test case Nbody with regards to the battey level of the DUT (with outliers)} \label{fig:Nbody_Dram_charge}
                \end{figure}
                
%
                \begin{figure}[H]
                    \centering
                    \begin{tikzpicture}
                        \pgfplotsset{%
                            width=1\textwidth,
                            height=0.4\textheight
                        }
                        \begin{axis}[
                            xlabel={Start battery level},
                            ylabel={Average dynamic energy (watt)},
                            ymin=0,ymax=20,
                        ]
                        
                            \addplot [mark=none, ultra thick, red]  coordinates {
                            (40, 0.006595684402444291)(45, 0.007295220674193181)(50, 0.007999659608910881)(55, 0.007202467971243639)(60, 0.007280865079495958)(65, 0.006858423509955077)(70, 0.008369444141141925)(75, 0.007144940647010992)(80, 0.004753236834232667)
                            };
                            \addlegendentry{Surface4Pro - IntelPowerGadget}
                            
                            \addplot [mark=none, ultra thick, blue]  coordinates {
                            (40, 0.002385328427460912)(45, 0.0017856511573015649)(50, 0.0025901992189954056)(55, 0.00210998144366709)(60, 0.00286452646862575)(65, 0.0020038487280194628)(70, 0.002908483770774353)(75, 0.0005098111150931342)(80, -0.005995916826693459)
                            };
                            \addlegendentry{Surface4Pro - HardwareMonitor}
                            
                            \addplot [mark=none, ultra thick, orange]  coordinates {
                            (50, 251.8661014299577)(55, 214.7597259251014)(60, 162.23880995564176)(65, 107.43421593849766)(70, 53.9187387386668)(75, 0.7554852536149699)(80, -44.463085003568885)
                            };
                            \addlegendentry{Surface4Pro - RAPL}
                            
                            \addplot [mark=none, dashdotted, red]  coordinates {
                            (40, -0.004038062354887025)(45, -0.004312668500277529)(50, -0.003808663911021498)(55, -0.0037057407527755254)(60, -0.004478257932982471)(65, -0.0026308734995501644)(70, -0.0034090674446925874)(75, -0.003041497079697436)(80, -0.0020334307266356875)
                            };
                            \addlegendentry{SurfaceBook - IntelPowerGadget}
                            
                            \addplot [mark=none, dashdotted, blue]  coordinates {
                            (40, -0.002443518930616523)(45, -0.0029256880137447055)(50, -0.002551777773312929)(55, -0.0027567211782433486)(60, -0.002206859154402231)(65, -0.0026388300848279207)(70, -0.0024597736945479324)(75, -0.002445350799195827)(80, -0.000541271265011134)
                            };
                            \addlegendentry{SurfaceBook - HardwareMonitor}
                            
                            \addplot [mark=none, dashdotted, orange]  coordinates {
                            (40, 101.92702155010147)(45, 86.7212700795567)(50, 69.40754598562454)(55, 51.343785407669614)(60, 32.443112755251185)(65, 14.52577919077786)(70, -4.520517378551423)(75, -23.811706977384954)(80, -35.700372165212755)
                            };
                            \addlegendentry{SurfaceBook - RAPL}
                            
                        \end{axis}
                    \end{tikzpicture} 
                \caption{A graph illustrating the energy consumption of Dram for test case Nbody with regards to the battey level of the DUT (with outliers)} \label{fig:Nbody_Dram_charge}
                \end{figure}
                
%
                \begin{figure}[H]
                    \centering
                    \begin{tikzpicture}
                        \pgfplotsset{%
                            width=1\textwidth,
                            height=0.4\textheight
                        }
                        \begin{axis}[
                            xlabel={Start battery level},
                            ylabel={Average dynamic energy (watt)},
                            ymin=0,ymax=20,
                        ]
                        
                            \addplot [mark=none, ultra thick, red]  coordinates {
                            (40, 0.006595684402444291)(45, 0.007295220674193181)(50, 0.007999659608910881)(55, 0.007202467971243639)(60, 0.007280865079495958)(65, 0.006858423509955077)(70, 0.008369444141141925)(75, 0.007144940647010992)(80, 0.004753236834232667)
                            };
                            \addlegendentry{Surface4Pro - IntelPowerGadget}
                            
                            \addplot [mark=none, ultra thick, blue]  coordinates {
                            (40, 0.002385328427460912)(45, 0.0017856511573015649)(50, 0.0025901992189954056)(55, 0.00210998144366709)(60, 0.00286452646862575)(65, 0.0020038487280194628)(70, 0.002908483770774353)(75, 0.0005098111150931342)(80, -0.005995916826693459)
                            };
                            \addlegendentry{Surface4Pro - HardwareMonitor}
                            
                            \addplot [mark=none, ultra thick, orange]  coordinates {
                            (50, 251.8661014299577)(55, 214.7597259251014)(60, 162.23880995564176)(65, 107.43421593849766)(70, 53.9187387386668)(75, 0.7554852536149699)(80, -44.463085003568885)
                            };
                            \addlegendentry{Surface4Pro - RAPL}
                            
                            \addplot [mark=none, dashdotted, red]  coordinates {
                            (40, -0.004038062354887025)(45, -0.004312668500277529)(50, -0.003808663911021498)(55, -0.0037057407527755254)(60, -0.004478257932982471)(65, -0.0026308734995501644)(70, -0.0034090674446925874)(75, -0.003041497079697436)(80, -0.0020334307266356875)
                            };
                            \addlegendentry{SurfaceBook - IntelPowerGadget}
                            
                            \addplot [mark=none, dashdotted, blue]  coordinates {
                            (40, -0.002443518930616523)(45, -0.0029256880137447055)(50, -0.002551777773312929)(55, -0.0027567211782433486)(60, -0.002206859154402231)(65, -0.0026388300848279207)(70, -0.0024597736945479324)(75, -0.002445350799195827)(80, -0.000541271265011134)
                            };
                            \addlegendentry{SurfaceBook - HardwareMonitor}
                            
                            \addplot [mark=none, dashdotted, orange]  coordinates {
                            (40, 101.92702155010147)(45, 86.7212700795567)(50, 69.40754598562454)(55, 51.343785407669614)(60, 32.443112755251185)(65, 14.52577919077786)(70, -4.520517378551423)(75, -23.811706977384954)(80, -35.700372165212755)
                            };
                            \addlegendentry{SurfaceBook - RAPL}
                            
                        \end{axis}
                    \end{tikzpicture} 
                \caption{A graph illustrating the energy consumption of Dram for test case Nbody with regards to the battey level of the DUT (with outliers)} \label{fig:Nbody_Dram_charge}
                \end{figure}
                
%
                \begin{figure}[H]
                    \centering
                    \begin{tikzpicture}
                        \pgfplotsset{%
                            width=1\textwidth,
                            height=0.4\textheight
                        }
                        \begin{axis}[
                            xlabel={Start battery level},
                            ylabel={Average dynamic energy (watt)},
                            ymin=0,ymax=20,
                        ]
                        
                            \addplot [mark=none, ultra thick, red]  coordinates {
                            (40, 0.006595684402444291)(45, 0.007295220674193181)(50, 0.007999659608910881)(55, 0.007202467971243639)(60, 0.007280865079495958)(65, 0.006858423509955077)(70, 0.008369444141141925)(75, 0.007144940647010992)(80, 0.004753236834232667)
                            };
                            \addlegendentry{Surface4Pro - IntelPowerGadget}
                            
                            \addplot [mark=none, ultra thick, blue]  coordinates {
                            (40, 0.002385328427460912)(45, 0.0017856511573015649)(50, 0.0025901992189954056)(55, 0.00210998144366709)(60, 0.00286452646862575)(65, 0.0020038487280194628)(70, 0.002908483770774353)(75, 0.0005098111150931342)(80, -0.005995916826693459)
                            };
                            \addlegendentry{Surface4Pro - HardwareMonitor}
                            
                            \addplot [mark=none, ultra thick, orange]  coordinates {
                            (50, 251.8661014299577)(55, 214.7597259251014)(60, 162.23880995564176)(65, 107.43421593849766)(70, 53.9187387386668)(75, 0.7554852536149699)(80, -44.463085003568885)
                            };
                            \addlegendentry{Surface4Pro - RAPL}
                            
                            \addplot [mark=none, dashdotted, red]  coordinates {
                            (40, -0.004038062354887025)(45, -0.004312668500277529)(50, -0.003808663911021498)(55, -0.0037057407527755254)(60, -0.004478257932982471)(65, -0.0026308734995501644)(70, -0.0034090674446925874)(75, -0.003041497079697436)(80, -0.0020334307266356875)
                            };
                            \addlegendentry{SurfaceBook - IntelPowerGadget}
                            
                            \addplot [mark=none, dashdotted, blue]  coordinates {
                            (40, -0.002443518930616523)(45, -0.0029256880137447055)(50, -0.002551777773312929)(55, -0.0027567211782433486)(60, -0.002206859154402231)(65, -0.0026388300848279207)(70, -0.0024597736945479324)(75, -0.002445350799195827)(80, -0.000541271265011134)
                            };
                            \addlegendentry{SurfaceBook - HardwareMonitor}
                            
                            \addplot [mark=none, dashdotted, orange]  coordinates {
                            (40, 101.92702155010147)(45, 86.7212700795567)(50, 69.40754598562454)(55, 51.343785407669614)(60, 32.443112755251185)(65, 14.52577919077786)(70, -4.520517378551423)(75, -23.811706977384954)(80, -35.700372165212755)
                            };
                            \addlegendentry{SurfaceBook - RAPL}
                            
                        \end{axis}
                    \end{tikzpicture} 
                \caption{A graph illustrating the energy consumption of Dram for test case Nbody with regards to the battey level of the DUT (with outliers)} \label{fig:Nbody_Dram_charge}
                \end{figure}
                
%
                \begin{figure}[H]
                    \centering
                    \begin{tikzpicture}
                        \pgfplotsset{%
                            width=1\textwidth,
                            height=0.4\textheight
                        }
                        \begin{axis}[
                            xlabel={Start battery level},
                            ylabel={Average dynamic energy (watt)},
                            ymin=0,ymax=20,
                        ]
                        
                            \addplot [mark=none, ultra thick, red]  coordinates {
                            (40, 0.006595684402444291)(45, 0.007295220674193181)(50, 0.007999659608910881)(55, 0.007202467971243639)(60, 0.007280865079495958)(65, 0.006858423509955077)(70, 0.008369444141141925)(75, 0.007144940647010992)(80, 0.004753236834232667)
                            };
                            \addlegendentry{Surface4Pro - IntelPowerGadget}
                            
                            \addplot [mark=none, ultra thick, blue]  coordinates {
                            (40, 0.002385328427460912)(45, 0.0017856511573015649)(50, 0.0025901992189954056)(55, 0.00210998144366709)(60, 0.00286452646862575)(65, 0.0020038487280194628)(70, 0.002908483770774353)(75, 0.0005098111150931342)(80, -0.005995916826693459)
                            };
                            \addlegendentry{Surface4Pro - HardwareMonitor}
                            
                            \addplot [mark=none, ultra thick, orange]  coordinates {
                            (50, 251.8661014299577)(55, 214.7597259251014)(60, 162.23880995564176)(65, 107.43421593849766)(70, 53.9187387386668)(75, 0.7554852536149699)(80, -44.463085003568885)
                            };
                            \addlegendentry{Surface4Pro - RAPL}
                            
                            \addplot [mark=none, dashdotted, red]  coordinates {
                            (40, -0.004038062354887025)(45, -0.004312668500277529)(50, -0.003808663911021498)(55, -0.0037057407527755254)(60, -0.004478257932982471)(65, -0.0026308734995501644)(70, -0.0034090674446925874)(75, -0.003041497079697436)(80, -0.0020334307266356875)
                            };
                            \addlegendentry{SurfaceBook - IntelPowerGadget}
                            
                            \addplot [mark=none, dashdotted, blue]  coordinates {
                            (40, -0.002443518930616523)(45, -0.0029256880137447055)(50, -0.002551777773312929)(55, -0.0027567211782433486)(60, -0.002206859154402231)(65, -0.0026388300848279207)(70, -0.0024597736945479324)(75, -0.002445350799195827)(80, -0.000541271265011134)
                            };
                            \addlegendentry{SurfaceBook - HardwareMonitor}
                            
                            \addplot [mark=none, dashdotted, orange]  coordinates {
                            (40, 101.92702155010147)(45, 86.7212700795567)(50, 69.40754598562454)(55, 51.343785407669614)(60, 32.443112755251185)(65, 14.52577919077786)(70, -4.520517378551423)(75, -23.811706977384954)(80, -35.700372165212755)
                            };
                            \addlegendentry{SurfaceBook - RAPL}
                            
                        \end{axis}
                    \end{tikzpicture} 
                \caption{A graph illustrating the energy consumption of Dram for test case Nbody with regards to the battey level of the DUT (with outliers)} \label{fig:Nbody_Dram_charge}
                \end{figure}
                
%
                \begin{figure}[H]
                    \centering
                    \begin{tikzpicture}
                        \pgfplotsset{%
                            width=1\textwidth,
                            height=0.4\textheight
                        }
                        \begin{axis}[
                            xlabel={Start battery level},
                            ylabel={Average dynamic energy (watt)},
                            ymin=0,ymax=20,
                        ]
                        
                            \addplot [mark=none, ultra thick, red]  coordinates {
                            (40, 0.006595684402444291)(45, 0.007295220674193181)(50, 0.007999659608910881)(55, 0.007202467971243639)(60, 0.007280865079495958)(65, 0.006858423509955077)(70, 0.008369444141141925)(75, 0.007144940647010992)(80, 0.004753236834232667)
                            };
                            \addlegendentry{Surface4Pro - IntelPowerGadget}
                            
                            \addplot [mark=none, ultra thick, blue]  coordinates {
                            (40, 0.002385328427460912)(45, 0.0017856511573015649)(50, 0.0025901992189954056)(55, 0.00210998144366709)(60, 0.00286452646862575)(65, 0.0020038487280194628)(70, 0.002908483770774353)(75, 0.0005098111150931342)(80, -0.005995916826693459)
                            };
                            \addlegendentry{Surface4Pro - HardwareMonitor}
                            
                            \addplot [mark=none, ultra thick, orange]  coordinates {
                            (50, 251.8661014299577)(55, 214.7597259251014)(60, 162.23880995564176)(65, 107.43421593849766)(70, 53.9187387386668)(75, 0.7554852536149699)(80, -44.463085003568885)
                            };
                            \addlegendentry{Surface4Pro - RAPL}
                            
                            \addplot [mark=none, dashdotted, red]  coordinates {
                            (40, -0.004038062354887025)(45, -0.004312668500277529)(50, -0.003808663911021498)(55, -0.0037057407527755254)(60, -0.004478257932982471)(65, -0.0026308734995501644)(70, -0.0034090674446925874)(75, -0.003041497079697436)(80, -0.0020334307266356875)
                            };
                            \addlegendentry{SurfaceBook - IntelPowerGadget}
                            
                            \addplot [mark=none, dashdotted, blue]  coordinates {
                            (40, -0.002443518930616523)(45, -0.0029256880137447055)(50, -0.002551777773312929)(55, -0.0027567211782433486)(60, -0.002206859154402231)(65, -0.0026388300848279207)(70, -0.0024597736945479324)(75, -0.002445350799195827)(80, -0.000541271265011134)
                            };
                            \addlegendentry{SurfaceBook - HardwareMonitor}
                            
                            \addplot [mark=none, dashdotted, orange]  coordinates {
                            (40, 101.92702155010147)(45, 86.7212700795567)(50, 69.40754598562454)(55, 51.343785407669614)(60, 32.443112755251185)(65, 14.52577919077786)(70, -4.520517378551423)(75, -23.811706977384954)(80, -35.700372165212755)
                            };
                            \addlegendentry{SurfaceBook - RAPL}
                            
                        \end{axis}
                    \end{tikzpicture} 
                \caption{A graph illustrating the energy consumption of Dram for test case Nbody with regards to the battey level of the DUT (with outliers)} \label{fig:Nbody_Dram_charge}
                \end{figure}
                
%
                \begin{figure}[H]
                    \centering
                    \begin{tikzpicture}
                        \pgfplotsset{%
                            width=1\textwidth,
                            height=0.4\textheight
                        }
                        \begin{axis}[
                            xlabel={Start battery level},
                            ylabel={Average dynamic energy (watt)},
                            ymin=0,ymax=20,
                        ]
                        
                            \addplot [mark=none, ultra thick, red]  coordinates {
                            (40, 0.006595684402444291)(45, 0.007295220674193181)(50, 0.007999659608910881)(55, 0.007202467971243639)(60, 0.007280865079495958)(65, 0.006858423509955077)(70, 0.008369444141141925)(75, 0.007144940647010992)(80, 0.004753236834232667)
                            };
                            \addlegendentry{Surface4Pro - IntelPowerGadget}
                            
                            \addplot [mark=none, ultra thick, blue]  coordinates {
                            (40, 0.002385328427460912)(45, 0.0017856511573015649)(50, 0.0025901992189954056)(55, 0.00210998144366709)(60, 0.00286452646862575)(65, 0.0020038487280194628)(70, 0.002908483770774353)(75, 0.0005098111150931342)(80, -0.005995916826693459)
                            };
                            \addlegendentry{Surface4Pro - HardwareMonitor}
                            
                            \addplot [mark=none, ultra thick, orange]  coordinates {
                            (50, 251.8661014299577)(55, 214.7597259251014)(60, 162.23880995564176)(65, 107.43421593849766)(70, 53.9187387386668)(75, 0.7554852536149699)(80, -44.463085003568885)
                            };
                            \addlegendentry{Surface4Pro - RAPL}
                            
                            \addplot [mark=none, dashdotted, red]  coordinates {
                            (40, -0.004038062354887025)(45, -0.004312668500277529)(50, -0.003808663911021498)(55, -0.0037057407527755254)(60, -0.004478257932982471)(65, -0.0026308734995501644)(70, -0.0034090674446925874)(75, -0.003041497079697436)(80, -0.0020334307266356875)
                            };
                            \addlegendentry{SurfaceBook - IntelPowerGadget}
                            
                            \addplot [mark=none, dashdotted, blue]  coordinates {
                            (40, -0.002443518930616523)(45, -0.0029256880137447055)(50, -0.002551777773312929)(55, -0.0027567211782433486)(60, -0.002206859154402231)(65, -0.0026388300848279207)(70, -0.0024597736945479324)(75, -0.002445350799195827)(80, -0.000541271265011134)
                            };
                            \addlegendentry{SurfaceBook - HardwareMonitor}
                            
                            \addplot [mark=none, dashdotted, orange]  coordinates {
                            (40, 101.92702155010147)(45, 86.7212700795567)(50, 69.40754598562454)(55, 51.343785407669614)(60, 32.443112755251185)(65, 14.52577919077786)(70, -4.520517378551423)(75, -23.811706977384954)(80, -35.700372165212755)
                            };
                            \addlegendentry{SurfaceBook - RAPL}
                            
                        \end{axis}
                    \end{tikzpicture} 
                \caption{A graph illustrating the energy consumption of Dram for test case Nbody with regards to the battey level of the DUT (with outliers)} \label{fig:Nbody_Dram_charge}
                \end{figure}
                
%\subsection{TestCaseIdle}
%
            \begin{figure}
                \centering
                \begin{tikzpicture}
                    \pgfplotsset{%
                        width=1\textwidth,
                        height=0.5\textheight
                    }
                    \begin{axis}[
                        xlabel={Start temperature},
                        ylabel={Average dynamic energy (watt)},
                    ]
                    
                    \end{axis}
                \end{tikzpicture} 
            \caption{A graph illustrating the energy consumption of Cores for test case TestCaseIdle with regards to the temperature of the DUT} \label{fig:TestCaseIdle_Cores}
            \end{figure}
            
%
            \begin{figure}
                \centering
                \begin{tikzpicture}
                    \pgfplotsset{%
                        width=1\textwidth,
                        height=0.5\textheight
                    }
                    \begin{axis}[
                        xlabel={Start temperature},
                        ylabel={Average dynamic energy (watt)},
                    ]
                    
                    \end{axis}
                \end{tikzpicture} 
            \caption{A graph illustrating the energy consumption of Cores for test case TestCaseIdle with regards to the temperature of the DUT} \label{fig:TestCaseIdle_Cores}
            \end{figure}
            
%
            \begin{figure}
                \centering
                \begin{tikzpicture}
                    \pgfplotsset{%
                        width=1\textwidth,
                        height=0.5\textheight
                    }
                    \begin{axis}[
                        xlabel={Start temperature},
                        ylabel={Average dynamic energy (watt)},
                    ]
                    
                    \end{axis}
                \end{tikzpicture} 
            \caption{A graph illustrating the energy consumption of Cores for test case TestCaseIdle with regards to the temperature of the DUT} \label{fig:TestCaseIdle_Cores}
            \end{figure}
            
%
            \begin{figure}
                \centering
                \begin{tikzpicture}
                    \pgfplotsset{%
                        width=1\textwidth,
                        height=0.5\textheight
                    }
                    \begin{axis}[
                        xlabel={Start temperature},
                        ylabel={Average dynamic energy (watt)},
                    ]
                    
                    \end{axis}
                \end{tikzpicture} 
            \caption{A graph illustrating the energy consumption of Cores for test case TestCaseIdle with regards to the temperature of the DUT} \label{fig:TestCaseIdle_Cores}
            \end{figure}
            
%
            \begin{figure}
                \centering
                \begin{tikzpicture}
                    \pgfplotsset{%
                        width=1\textwidth,
                        height=0.5\textheight
                    }
                    \begin{axis}[
                        xlabel={Start temperature},
                        ylabel={Average dynamic energy (watt)},
                    ]
                    
                    \end{axis}
                \end{tikzpicture} 
            \caption{A graph illustrating the energy consumption of Cores for test case TestCaseIdle with regards to the temperature of the DUT} \label{fig:TestCaseIdle_Cores}
            \end{figure}
            
%
            \begin{figure}
                \centering
                \begin{tikzpicture}
                    \pgfplotsset{%
                        width=1\textwidth,
                        height=0.5\textheight
                    }
                    \begin{axis}[
                        xlabel={Start temperature},
                        ylabel={Average dynamic energy (watt)},
                    ]
                    
                    \end{axis}
                \end{tikzpicture} 
            \caption{A graph illustrating the energy consumption of Cores for test case TestCaseIdle with regards to the temperature of the DUT} \label{fig:TestCaseIdle_Cores}
            \end{figure}
            
%
            \begin{figure}
                \centering
                \begin{tikzpicture}
                    \pgfplotsset{%
                        width=1\textwidth,
                        height=0.5\textheight
                    }
                    \begin{axis}[
                        xlabel={Start temperature},
                        ylabel={Average dynamic energy (watt)},
                    ]
                    
                    \end{axis}
                \end{tikzpicture} 
            \caption{A graph illustrating the energy consumption of Cores for test case TestCaseIdle with regards to the temperature of the DUT} \label{fig:TestCaseIdle_Cores}
            \end{figure}
            
%
            \begin{figure}
                \centering
                \begin{tikzpicture}
                    \pgfplotsset{%
                        width=1\textwidth,
                        height=0.5\textheight
                    }
                    \begin{axis}[
                        xlabel={Start temperature},
                        ylabel={Average dynamic energy (watt)},
                    ]
                    
                    \end{axis}
                \end{tikzpicture} 
            \caption{A graph illustrating the energy consumption of Cores for test case TestCaseIdle with regards to the temperature of the DUT} \label{fig:TestCaseIdle_Cores}
            \end{figure}
            
%
            \begin{figure}
                \centering
                \begin{tikzpicture}
                    \pgfplotsset{%
                        width=1\textwidth,
                        height=0.5\textheight
                    }
                    \begin{axis}[
                        xlabel={Start temperature},
                        ylabel={Average dynamic energy (watt)},
                    ]
                    
                    \end{axis}
                \end{tikzpicture} 
            \caption{A graph illustrating the energy consumption of Cores for test case TestCaseIdle with regards to the temperature of the DUT} \label{fig:TestCaseIdle_Cores}
            \end{figure}
            
%
            \begin{figure}
                \centering
                \begin{tikzpicture}
                    \pgfplotsset{%
                        width=1\textwidth,
                        height=0.5\textheight
                    }
                    \begin{axis}[
                        xlabel={Start temperature},
                        ylabel={Average dynamic energy (watt)},
                    ]
                    
                    \end{axis}
                \end{tikzpicture} 
            \caption{A graph illustrating the energy consumption of Cores for test case TestCaseIdle with regards to the temperature of the DUT} \label{fig:TestCaseIdle_Cores}
            \end{figure}
            
%
            \begin{figure}
                \centering
                \begin{tikzpicture}
                    \pgfplotsset{%
                        width=1\textwidth,
                        height=0.5\textheight
                    }
                    \begin{axis}[
                        xlabel={Start temperature},
                        ylabel={Average dynamic energy (watt)},
                    ]
                    
                    \end{axis}
                \end{tikzpicture} 
            \caption{A graph illustrating the energy consumption of Cores for test case TestCaseIdle with regards to the temperature of the DUT} \label{fig:TestCaseIdle_Cores}
            \end{figure}
            