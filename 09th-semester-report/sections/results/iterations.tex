\section{Iterations}\label{sec:iterations}

It is now time to take a look at how the performance of the different DUTs changes over time. This will be done by plotting the average dynamic energy consumption for each time a test case is executed after the DUT is restarted.

\subsection{Expectations}

When considering what is expected from this experiment, it would be a slightly increased performance when the DUT has been running for a longer time. The expected reason for this increased energy consumption would be the background processes. 

\subsection{Results} 

When considering the results, a graph exists for each of the test cases, where two of them can be seen in \cref{fig:FannkuchRedux_Cores_iteration} and \cref{fig:Nbody_Cores_iteration}, and the others can be found in \cref{app:iterations}. It can first of all be noted some of the DUTs has no values on some of the higher x-values. This is the case for both DUTs with a battery, meaning they reach the lower limit of 40\% before executing the test cases 30 times. It can also be seen how energy consumption does not increase over time, which was expected. This indicates the impact of background processes is smaller than expected, which also was found to be the case when considering the impact of R3 validation in \cref{subsec:exp_r3}. In the following, the different DUTs will be covered one at a time.

\paragraph{Workstation:} The workstation is for all test cases the DUT with the highest energy consumption, which is according to our expectation. This DUT is the only one with Clamp measurements, where this measurement type seems to be the one which deviates the most, which is most likely due to the impact of the energy consumption of the different components within the DUT. This is opposed to the software measuring instruments where only the CPU and RAM are measured. When comparing the Clamp measurements across OSes, Windows seem to have the higher energy consumption overall, as can be observed in \cref{fig:BinaryTrees_Cores_iteration}. When looking at the different software-based measuring instruments, Intel Power Gadget and LHM are observed to be very close when measuring the energy consumption, with RAPL measuring lower energy consumption, in all cases except NBody where RAPL records a higher energy consumption in \cref{fig:Nbody_Cores_iteration}. When comparing hardware and software measuring instruments, they are in some cases very similar, as in \cref{fig:Nbody_Cores_iteration}, but vary more in other cases like in \cref{fig:BinaryTrees_Cores_iteration}.

\paragraph{Surface Book:} When considering the Surface Book, it has the lowest energy consumption in all cases, except when comparing the RAPL measurement for the Surface Pro 4 with the Intel Power Gadget measurements in \cref{fig:FannkuchRedux_Cores_iteration}. Intel Power Gadget and LHM are in all cases except for Nbody in \cref{fig:Nbody_Cores_iteration} close to the same measurement. When considering RAPL, it measures a higher energy consumption compared to Intel Power Gadget and LHM for Fasta, BinaryTrees and NBody, and lower for FannkuchRedux.

\paragraph{Surface Pro 4:} For the Surface Pro 4, the energy consumption is lower compared to the workstation in all cases, and higher compared to the Surface Book in all cases except one. When comparing the software measuring instruments, RAPL will in all cases give the lowest measurement, where Intel Power Gadget and LHM are very close to each other.


                            \begin{figure}
                                \centering
                                \begin{tikzpicture}[]
                                    \pgfplotsset{%
                                        width=.85\textwidth,
                                        height=.15\textheight
                                    }
                                    \begin{axis}[xlabel={Average energy consumption (Watts)}, title={Cores - FannkuchRedux - Energy - without outliers}, ytick={},
                                    yticklabels={
                                        
                                        },
                                        xmin=0,xmax=20,
                                        ]
                                    
                                    \end{axis}
                                \end{tikzpicture}
                            \caption{A comparison of of Cores energy consumption for test case FannkuchRedux for the Surface4Pro,  experiment \#2 (without outliers)} \label{fig:FannkuchRedux_Cores_comparison_energy_without_outliers_Surface4Pro_avg_watts_exp2}
                            \end{figure}
                            

                            \begin{figure}
                                \centering
                                \begin{tikzpicture}[]
                                    \pgfplotsset{%
                                        width=.85\textwidth,
                                        height=.15\textheight
                                    }
                                    \begin{axis}[xlabel={Average energy consumption (Watts)}, title={Cores - BinaryTrees - Energy - without outliers}, ytick={},
                                    yticklabels={
                                        
                                        },
                                        xmin=0,xmax=20,
                                        ]
                                    
                                    \end{axis}
                                \end{tikzpicture}
                            \caption{A comparison of of Cores energy consumption for test case BinaryTrees for the Surface4Pro,  experiment \#2 (without outliers)} \label{fig:BinaryTrees_Cores_comparison_energy_without_outliers_Surface4Pro_avg_watts_exp2}
                            \end{figure}
                            
