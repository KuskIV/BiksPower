\todo{Fix comments from Lone from 27/10}
In the study by Hackenberg et al.\cite{hackenberg2013} the accuracies of several different measuring instruments are tested. Here they compare several different hardware measurement approaches to RAPL for intel and APM for AMD CPUs. There is a focus on the granularity of the measurements concerning the impact of the granularity on the results. For AC measurements it is found that sampling down to 50ms yields more accurate information. This is contrary to common beliefs that large capacitors in PSUs level out the consumption, at least from PSUs in 2013. They also show that a granularity of $1-20$ samples per second is necessary to analyze the consumption of the different phases of an application. This is also close to the granularity provided by RAPL \todo{What is sampling rate of RAPL. In Khan we say 1000 hz, here we say close to 20 hz} and APM though with some worse accuracies. They conduct AC measurement with a ZES ZIMMER LMG450, IPMI and PDU from the outlet leading into the power supply, and DC measurement using a custom setup using a Hall sensor where they only measure the CPU, RAM and the motherboard excluding the fans and disk from the measurements. For high granularity of $100+$ samples per second, the AC techniques cannot sample fast enough. They achieve good results with their DC measurement, but they note that some filtering is needed, as a lot of noise is present in the measurements. They also remark that the success of their measurements is highly dependent on the type of motherboard as different manufacturers use different kinds and sizes of capacitors that could smooth out the measurements. In conclusion, they determine that RAPL and APM could need some improvements, but that RAPL is decently accurate and good enough to be used, while APM is distinctly lacking in certain scenarios. For the hardware measurements, the AC measurements were accurate enough and could provide them with good granularity and their DC setup was surprisingly accurate with a high sampling rate.\cite*{hackenberg2013}\newline
