In this study the accuracies of several different energy consumption measurements is tested "Power Measurement Techniques on Standard
Compute Nodes: A Quantitative Comparison" \cite{hackenberg2013}. Here they compare several different hardware measurement approaches to RAPL for intel and APM for AMD cpu's. There is especially a focus on the resolution of the measurements and when it matters. They find the for AC energy measurements sample down to 50ms yields more accurate information and granular information about the system. This is contrary to common beliefs that large capacitors in modern PSUs level out the consumption. They show that a resolution of $1-20$ samples per second is necessary to analyses the consumption of the different phases of an application.This is also close to the resolution provided by RAPL and APM though with some worse accuracies. For high resolution of $100+$ samples per second the AC technics cannot provide a granular enough view. They achieve good results with their DC measurement, but they note that some filtering is needed, as a lot of noise is present in the measurements. They also remark that the success of their measurements are highly dependent on the type of motherboard as different manufactures use different kinds and sizes of capacitors that could smooth out the measurements. As a conclusion they determine that RAPL and APM could need some improvements, but that RAPL is decently accurate and good enough to be used. For the hardware measurements the AC measurements were accurate enough and could provide them with a good granularity and that their DC setup was surprisingly accurate with a high sampling rate.