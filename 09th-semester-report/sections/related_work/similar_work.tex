% Has anyone else tried to compare energy measurement tools?

\section{Energy consumption of different languages and constructs}\label[section]{sec:rw_energy_consumption_diff_lang}

% One is the work by Hackenberg et al.\cite[]{hackenberg2013} which aims to find which instrument measures the energy consumption of a system best. This is done by measuring using RAPL, and AMD's APM and comparing them against both AC and DC measuring while running a sequence of synthetic code designed to provide a controllable workload. Hackenberg et al. found AC to be the most accurate one, as $95\%$ and $96\%$ of samples vary with less than 2W in the two tests conducted.

Multiple works focus on the differences in energy consumption of different languages and various language constructs within the same language. One such work is by Lindholt et al.\cite[]{Lindholt2022}, where a comparison was made of four groups of language constructs within C\#, including concurrency, casting parameter mechanisms and lambda expressions. This is achieved using 150 micro benchmarks, where further analysis is conducted on each of the constructs to understand the results. Based on the findings, 68 macro benchmarks are created to test the observations on a larger scale and to see how the constructs work in combination with each other.\newline

Another such work is the work by Rasmussen et al.\cite[]{Rasmussen2021}, in which another work by Kumar et al.\cite[]{Kumar2017} made in Java is analyzed and reproduced in C\# to see how the results compare in a new language. The study includes different primitive types like selections, collections, objects inheritance and exceptions, and when comparing the results it is concluded there can be drastic differences between Java and C\# in some situations. Another study by \cite[]{Theilmann2022} looks into two different implementations of a microservice architecture, one with a shared database connection and one with a database connection per service. The energy consumption of these implementations are analyzed and compared to a monolithic architecture. This study was done on two similar implementations on C\# and Java, where the energy consumption is compared, and a significant difference can be found, both when comparing languages, but also between the type of database connection.\newline

Other studies also exist, where more than two programming languages are compared. An example of this is the work by Pereira et al.\cite[]{Pereira2017}, where 27 of the most popular languages are compared. The languages are compared with each other, but also in groups considering compiled, interpreted and virtual machine languages or object-oriented, functional and imperative languages. The languages are tested across ten test cases with different loads, and the results show how the language can have a huge impact on the energy consumption of the program, and that a faster program is not always greener.\newline

In the work by Najmuddin et al.\cite[]{Najmuddin2021} the energy consumption of different OSs are compared on different aspects, including color schemes, resolutions, brightness, memory management, process management, scheduler tasks and other modules. The study is based on observations where Linux uses more energy than Windows and aims to find out why. Based on an analysis of 15 existing works, the reason is found to be drivers. Linux is found to lack optimized drivers, as a result of hardware manufactures unwillingness to give details about the hardware. This makes it difficult to make optimized drivers for open source projects, like Linux. They also find the kernel in Linux to be poorly optimized in some versions.