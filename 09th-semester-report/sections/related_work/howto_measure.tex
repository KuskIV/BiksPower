% How to measure the energy consumption of code

\section{Different Measuring Instruments}\label[section]{sec:rw_diff_measuring_instruments}
Energy measurements are needed to be able to compare the energy consumption of e.g. different language constructs. However, there are several methods to get these measurements. In this section some work focused on different measuring instruments is presented.

%\subsection*{Software based measuring instruments}

One such work is the work by Jagroep et al.\cite[]{Jagroep2015} which compares different software-based measurement instruments. This work evaluates different measurement instruments through a series of experiments on both idle and full load scenarios where the estimated energy consumption is compared to the actual consumption as reported by hardware measurements. Here it is noted that despite having access to roughly the same data the results can differ significantly between the different measurement instruments. The study includes 14 different measurement instruments, picked based on four criteria. These include: at least a beta version has to be available, it has to be able to sample at least once per second, it has to run on Linux or Windows and it has to be compatible with their hardware. They do however have some issues when trying to download and set up most of the measuring instruments, and only successfully use two measuring methods one on Windows 7 and one on Ubuntu 12.04. Both were found to be inaccurate.\cite*{Jagroep2015} \newline

Another work which aims to analyze Intel's RAPL compared with a hardware-based measuring instrument is the work by Khan et al.\cite*[]{RAPL_in_action}. In this work, they argue the advantages and disadvantages are yet to be fully uncovered, which is what they aim to do. This is achieved on an Intel Skylake and an Intel Haswell CPU, which also uncovers the differences, where the newer versions are deemed more optimized. The key findings of the work are first of all in regards to the accuracy of RAPL, where they find the correlation between the wall power and the package power from RAPL to be $0.99$ for the Haswell machine. In addition to this, the overhead of RAPL is deemed low and negligible through experiments. This was done by pinning the benchmark and RAPL on the same core, and analyzing if the benchmark would slow down depending on the sampling rate of the measurement program, where with the highest sampling rate, the performance overhead is still less than $2\%$. The work also analyzes what impact the temperature has on the power consumption measured by RAPL, where the correlation coefficient between this was found to be $0.34$ and $0.93$ for Skylake and Haswell respectively, meaning the temperature does measurably impact the power consumption, especially for the older Haswell processor. One big advantage of RAPL, as argued by Khan et al.\cite*[]{RAPL_in_action}, is the sampling rate of $1.000$Hz. This sampling rate is argued to be much higher than most external power meters and enables RAPL to distinguish between different phases within the benchmark. This sampling rate is however also one of the roots of errors in RAPL, as the measurements are inconsistent and non-atomic. According to the documentation, RAPL updates every $0.976$ms, but this value deviates and is overall found to be closer to $1$ms. Because of this, cases can occur where the energy is measured at a consistent interval and will contain a different number of updates, resulting in spikes in energy consumption. And since RAPL does not give any timestamps on updates, it is difficult to say when updates are made, unless a check is made more often than every $1$ms, resulting in some sampling being made without it being an update, but these values are removed. The work\cite*[]{RAPL_in_action} also notes things like poor driver support, the inability to measure the energy consumption from a single core and the static sampling rate of $1.000$ as disadvantages with RAPL.\cite*{RAPL_in_action}\nytafsnit

% In the work by Hackenberg et al.\cite*[]{hackenberg2013} Intelligent Platform Management Interface (IMPI) based platforms like power supplies with measurement features, model-based interfaces like RAPL or APM and physical measurements including AC and DC are compared, with an aim of uncovering aspects of accuracy, temporal resolution and measurement overhead.

% Here it is noted that Intel's RAPL has previously been shown to be reasonably accurate, and is deemed a viable alternative to physical measurements\cite*[]{Dongarra2012,hackenberg2013}.

%\subsection*{Hardware based measuring instruments}

In this study the accuracies of several different energy consumption measurements is tested "Power Measurement Techniques on Standard
Compute Nodes: A Quantitative Comparison" \cite{hackenberg2013}. Here they compare several different hardware measurement approaches to RAPL for intel and APM for AMD cpu's. There is especially a focus on the resolution of the measurements and when it matters. They find the for AC energy measurements sample down to 50ms yields more accurate information and granular information about the system. This is contrary to common beliefs that large capacitors in modern PSUs level out the consumption. They show that a resolution of $1-20$ samples per second is necessary to analyses the consumption of the different phases of an application. This is also close to the resolution provided by RAPL and APM though with some worse accuracies. For high resolution of $100+$ samples per second the AC technics cannot provide a granular enough view. They achieve good results with their DC measurement, but they note that some filtering is needed, as a lot of noise is present in the measurements. They also remark that the success of their measurements are highly dependent on the type of motherboard as different manufactures use different kinds and sizes of capacitors that could smooth out the measurements. As a conclusion they determine that RAPL and APM could need some improvements, but that RAPL is decently accurate and good enough to be used. For the hardware measurements the AC measurements were accurate enough and could provide them with a good granularity and that their DC setup was surprisingly accurate with a high sampling rate.

Another similar work that also aims to compare different measurement method in "A Comparative Study of Methods for Measurement of Energy of Computing"\cite{fahad2019comparative}. To do this a hardware measurements from a HCLWattsUp was used and then compared with RAPLand on-chip-sensors. To get granular measurements from the HCLWattUp they introduced the concept of Dynamic Energy consumption, which would make the results more comparable to RAPL. Dynamic energy consumption is an estimation of the applications power consumption, ans is calculated using the following formula:

\begin{equation}
    E_D = E_T -(P_S * T_E)
\end{equation}

Where $E_D$ is the dynamic energy consumption, $T_E$ is the duration of the program execution and $P_S$ is the energy consumption when the system is idle. The dynamic energy consumption will then represent the energy consumption of the test case. The hardware measurements are the ground truth in their experiments. It is found that RAPL generally under reports the energy consumption of the system. The accuracy of RAPL is also tested for different use cases to see if possible calibrations are feasible. They found that calibration for the RAPL and on-chip-sensors were generally not possible with their technics. They provide some recommendation for future works and setup which will be covered in more depth later.



% When measuring the energy consumption of a process, different energy profilers exist that have been used in existing work. One popular approach is model-based power consumption estimators provided by x86 microprocessors, including Intel's RAPL for intel chips, and Application Power Management (APM) for AMD.

% \todo{I think this sentence is weird.}Work exists for RAPL, here it is deemed a viable alternative to physical measurements, where it consistently under-estimates\cite[]{Dongarra2012, Hackenberg2013}, but it does have some limitations, as RAPL values are not physical measurements, it is rather based on modelling approach\cite[]{Hackenberg2013}. Another limitation is how RAPL returns energy data and not power data. Because of this, the average update interval is used as timestamps for each update, and this value is assumed to be accurate enough.\cite[]{Hackenberg2013}

% When considering APM, it is reasonably accurate but suffers from for example systematic inaccuracies most likely caused by novelty of the interface.\cite[]{Hackenberg2013}


% When measuring the energy consumption using hardware, it can be done on either AC or DC. When measuring AC, the measuring device is located between the power supply and the electrical outlet, and for DC it is from the power supply to the system.\cite[]{Hackenberg2013} Another hardware approach could also be using the Watts Up Pro power consumption meter, also used in multiple works as it has a very high accuracy of $1.5\%$.\cite[]{Jagroep2015}\nytafsnit

%% Intel Energy Checker SDK
%% Application Power Management for AMD


%% FEETINGS: Framework for Energy Efficiency Testing to Improve Environmental Goal of the Software:


%%In this study the accuracies of several different energy consumption measurements is tested "Power Measurement Techniques on Standard
Compute Nodes: A Quantitative Comparison" \cite{hackenberg2013}. Here they compare several different hardware measurement approaches to RAPL for intel and APM for AMD cpu's. There is especially a focus on the resolution of the measurements and when it matters. They find the for AC energy measurements sample down to 50ms yields more accurate information and granular information about the system. This is contrary to common beliefs that large capacitors in modern PSUs level out the consumption. They show that a resolution of $1-20$ samples per second is necessary to analyses the consumption of the different phases of an application. This is also close to the resolution provided by RAPL and APM though with some worse accuracies. For high resolution of $100+$ samples per second the AC technics cannot provide a granular enough view. They achieve good results with their DC measurement, but they note that some filtering is needed, as a lot of noise is present in the measurements. They also remark that the success of their measurements are highly dependent on the type of motherboard as different manufactures use different kinds and sizes of capacitors that could smooth out the measurements. As a conclusion they determine that RAPL and APM could need some improvements, but that RAPL is decently accurate and good enough to be used. For the hardware measurements the AC measurements were accurate enough and could provide them with a good granularity and that their DC setup was surprisingly accurate with a high sampling rate.