% How to measure the energy consumption of code

\section{Measure Energy Consumption of Code}

When measuring the energy consumption of a process, different energy profilers exists which has been used in existing work. One popular approach are model-based power consumption estimators provided by x86 microprocessors, including Intel's RAPL for intel chips, and Application Power Management (APM) for AMD.

Work exists for RAPL, here it is deemed a viable alternative to physical measurements, where it consistently under-estimates\cite[]{Dongarra2012, Hackenberg2013}, but it does have some limitations, as RAPL values are not physical measurements, it is rather based on modeling approach\cite[]{Hackenberg2013}. Another limitation is how RAPL returns energy data and not power data. Because of this, the average update interval is used as timestamps for each update, and this value is assumed to be accurate enough.\cite[]{Hackenberg2013}

When considering APM, it is reasonably accurate, but suffers from for example systematic inaccuracies most likely caused by novelty of the interface.\cite[]{Hackenberg2013}


When measuring the energy consumption using hardware, it can be done on either AC or DC. When measuring AC, the measuring device is located between the power supply and the electrical outlet, and for DC it is from the power supply to the system.\cite[]{Hackenberg2013} Another hardware approach could also be using the Watts Up Pro power consumption meter, also used in multiple works as it has a very high accuracy of $1.5\%$.\cite[]{Jagroep2015}

%% Intel Energy Checker SDK
%% Application Power Management for AMD