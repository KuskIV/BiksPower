% How to measure the energy consumption of code

\section{Measure Energy Consumption of Code}

When measuring the energy consumption of a process, different energy profilers exist that have been used in existing work. One popular approach is model-based power consumption estimators provided by x86 microprocessors, including Intel's RAPL for intel chips, and Application Power Management (APM) for AMD.

\todo{I think this sentence is weird.}Work exists for RAPL, here it is deemed a viable alternative to physical measurements, where it consistently under-estimates\cite[]{Dongarra2012, Hackenberg2013}, but it does have some limitations, as RAPL values are not physical measurements, it is rather based on modelling approach\cite[]{Hackenberg2013}. Another limitation is how RAPL returns energy data and not power data. Because of this, the average update interval is used as timestamps for each update, and this value is assumed to be accurate enough.\cite[]{Hackenberg2013}

When considering APM, it is reasonably accurate but suffers from for example systematic inaccuracies most likely caused by novelty of the interface.\cite[]{Hackenberg2013}


When measuring the energy consumption using hardware, it can be done on either AC or DC. When measuring AC, the measuring device is located between the power supply and the electrical outlet, and for DC it is from the power supply to the system.\cite[]{Hackenberg2013} Another hardware approach could also be using the Watts Up Pro power consumption meter, also used in multiple works as it has a very high accuracy of $1.5\%$.\cite[]{Jagroep2015}\nytafsnit

%% Intel Energy Checker SDK
%% Application Power Management for AMD


%% FEETINGS: Framework for Energy Efficiency Testing to Improve Environmental Goal of the Software:


Mancebo et al. found that there are three types of problems that occur in the domain of evaluating software's energy consumption.\cite{MANCEBO2021100558} 
\begin{enumerate}
    \item There are inconsistencies in the terminology, with different terms being used for the same concept or even the same term being used for different concepts. This lack of consistency hurts the understanding of the subject.
    \item There is no agreed-upon methodology, which makes it difficult to compare and replicate results.
    \item Choosing the correct measuring instruments that are fitting for the particular experiment evaluation.
\end{enumerate}

To solve these problems they created \textit{Framework for Energy Efficiency Testing to Improve Environmental Goals of the Software} (FEETINGS). FEETINGS consist of three main components which aim to solve these issues these are the Conceptual component, Methodological component and Technological component.\nytafsnit

%\begin{tabular}{}

\begin{table}[ht]
    \begin{tabular}{|| p{0.29\linewidth} | p{0.5\linewidth}| p{0.15\linewidth}||}
        \hline
        \textbf{Term}& \textbf{Definition} & \textbf{Source} \\ [0.5ex]\hline\hline
        Software entity & Software that is to be characterized by measuring its attributes.         & FEETINGS\cite{GarciaFEETINGS}        \\ \hline
        Software entity class & The collection of all the entities that satisfy the determined objective. & FEETINGS\cite{GarciaFEETINGS}        \\ \hline
        Test Case & A representation of functionality of the software entity to be measured.  & FEETINGS\cite{GarciaFEETINGS}         \\ \hline
        Test Case Measurement & A set of energy consumption measurements of all the runs in a test case.  & FEETINGS\cite{GarciaFEETINGS}         \\ \hline
        Measurement & A set of energy consumption samples from a single test case run.       & FEETINGS\cite{GarciaFEETINGS}         \\ \hline
        Samples & Each energy consumption record taken by a measuring instrument.           & FEETINGS\cite{GarciaFEETINGS}         \\ \hline
        Device Under Test (DUT) & A device where the software entity to be measured is run.                 & FEETINGS\cite{GarciaFEETINGS}         \\ \hline
        Measuring Instrument & A method used to make energy consumption measurements.                    & FEETINGS\cite{GarciaFEETINGS}         \\ \hline
        Setup & A defined step of procedures executes at DUT startup.                     & R3\cite{Bokhari2020r3}              \\ \hline
        Batch\ & A set of test cases executed in sequence with a cool of periods.            & NEW             \\ \hline
        Configuration\ & A combinations of a DUT, measuring instrument and test case, temperature and battery.            & NEW             \\ \hline
    
    \end{tabular}
    \caption{Terminology used throughout our work.}
    \label{tab:TerminologyAlert}
    \end{table}
\todo{Check terminology is consistent}

The Conceptual component provides an extension of the work by Garcia et al.\cite{GARCIA2006631} which defines Software Measurement Ontology. This extension is called Green Software Measurement Ontology (GSMO) and provides terminology and the corresponding definitions. The extended terminology from Mancebo et al. can be seen in \cref{tab:feetTable} along with extensions from other work and new extensions useful for this work. The Methodological component is a process called Green Software Measurement Process (GSMP) which are guidelines to assist with the study design, analysis and presenting the results presented in phases. The Technological component consists of two parts. The first part is the Energy Efficiency Tester (EET) which is a hardware-based approach to measuring energy consumption. The second part is called ELLIOT, which is a software-based tool which handles the data provided by EET with the main goal of providing a visual representation that can be used to process, analyze and construct graphs and tables of the data collected by the EET.

In this work, the Conceptual components providing the terminology will be followed with some additional terminology from other work and our own as shown in \cref{tab:feetTable}\nytafsnit

