% How to measure the energy consumption of code

\section{Measure Energy Consumption of Code}\label[section]{sec:measure_energy}

When measuring the energy consumption of a test case, there exists various different ways of proceeding. 

% In the work by Hackenberg et al.\cite*[]{hackenberg2013} Intelligent Platform Management Interface (IMPI) based platforms like power supplies with measurement features, model-based interfaces like RAPL or APM and physical measurements including AC and DC are compared, with an aim of uncovering aspects of accuracy, temporal resolution and measurement overhead.

% Here it is noted that Intel's RAPL has previously been shown to be reasonably accurate, and is deemed a viable alternative to physical measurements\cite*[]{Dongarra2012,hackenberg2013}.

In this study the accuracies of several different energy consumption measurements is tested "Power Measurement Techniques on Standard
Compute Nodes: A Quantitative Comparison" \cite{hackenberg2013}. Here they compare several different hardware measurement approaches to RAPL for intel and APM for AMD cpu's. There is especially a focus on the resolution of the measurements and when it matters. They find the for AC energy measurements sample down to 50ms yields more accurate information and granular information about the system. This is contrary to common beliefs that large capacitors in modern PSUs level out the consumption. They show that a resolution of $1-20$ samples per second is necessary to analyses the consumption of the different phases of an application. This is also close to the resolution provided by RAPL and APM though with some worse accuracies. For high resolution of $100+$ samples per second the AC technics cannot provide a granular enough view. They achieve good results with their DC measurement, but they note that some filtering is needed, as a lot of noise is present in the measurements. They also remark that the success of their measurements are highly dependent on the type of motherboard as different manufactures use different kinds and sizes of capacitors that could smooth out the measurements. As a conclusion they determine that RAPL and APM could need some improvements, but that RAPL is decently accurate and good enough to be used. For the hardware measurements the AC measurements were accurate enough and could provide them with a good granularity and that their DC setup was surprisingly accurate with a high sampling rate.



% When measuring the energy consumption of a process, different energy profilers exist that have been used in existing work. One popular approach is model-based power consumption estimators provided by x86 microprocessors, including Intel's RAPL for intel chips, and Application Power Management (APM) for AMD.

% \todo{I think this sentence is weird.}Work exists for RAPL, here it is deemed a viable alternative to physical measurements, where it consistently under-estimates\cite[]{Dongarra2012, Hackenberg2013}, but it does have some limitations, as RAPL values are not physical measurements, it is rather based on modelling approach\cite[]{Hackenberg2013}. Another limitation is how RAPL returns energy data and not power data. Because of this, the average update interval is used as timestamps for each update, and this value is assumed to be accurate enough.\cite[]{Hackenberg2013}

% When considering APM, it is reasonably accurate but suffers from for example systematic inaccuracies most likely caused by novelty of the interface.\cite[]{Hackenberg2013}


% When measuring the energy consumption using hardware, it can be done on either AC or DC. When measuring AC, the measuring device is located between the power supply and the electrical outlet, and for DC it is from the power supply to the system.\cite[]{Hackenberg2013} Another hardware approach could also be using the Watts Up Pro power consumption meter, also used in multiple works as it has a very high accuracy of $1.5\%$.\cite[]{Jagroep2015}\nytafsnit

%% Intel Energy Checker SDK
%% Application Power Management for AMD


%% FEETINGS: Framework for Energy Efficiency Testing to Improve Environmental Goal of the Software:


Mancebo et al.\cite{GarciaFEETINGS}  found that there are three types of problems that occur in the domain of evaluating software's energy consumption.\cite{GarciaFEETINGS} 
\begin{enumerate}
    \item There are inconsistencies in the terminology, with different terms being used for the same concept or even the same term being used for different concepts. This lack of consistency hurts the understanding of the subject.
    \item There is no agreed-upon methodology, which makes it difficult to compare and replicate results.
    \item Choosing the correct measuring instruments that are fitting for the particular experiment evaluation.
\end{enumerate}

To solve these problems they created Framework for Energy Efficiency Testing to Improve Environmental Goals of the Software (FEETINGS). FEETINGS consist of three main components which aim to solve these issues. These are the conceptual component, methodological component and technological component.\cite{GarciaFEETINGS}\nytafsnit

%%\begin{tabular}{}

\begin{table}[ht]
    \begin{tabular}{|| p{0.29\linewidth} | p{0.5\linewidth}| p{0.15\linewidth}||}
        \hline
        \textbf{Term}& \textbf{Definition} & \textbf{Source} \\ [0.5ex]\hline\hline
        Software entity & Software that is to be characterized by measuring its attributes.         & FEETINGS\cite{GarciaFEETINGS}        \\ \hline
        Software entity class & The collection of all the entities that satisfy the determined objective. & FEETINGS\cite{GarciaFEETINGS}        \\ \hline
        Test Case & A representation of functionality of the software entity to be measured.  & FEETINGS\cite{GarciaFEETINGS}         \\ \hline
        Test Case Measurement & A set of energy consumption measurements of all the runs in a test case.  & FEETINGS\cite{GarciaFEETINGS}         \\ \hline
        Measurement & A set of energy consumption samples from a single test case run.       & FEETINGS\cite{GarciaFEETINGS}         \\ \hline
        Samples & Each energy consumption record taken by a measuring instrument.           & FEETINGS\cite{GarciaFEETINGS}         \\ \hline
        Device Under Test (DUT) & A device where the software entity to be measured is run.                 & FEETINGS\cite{GarciaFEETINGS}         \\ \hline
        Measuring Instrument & A method used to make energy consumption measurements.                    & FEETINGS\cite{GarciaFEETINGS}         \\ \hline
        Setup & A defined step of procedures executes at DUT startup.                     & R3\cite{Bokhari2020r3}              \\ \hline
        Batch\ & A set of test cases executed in sequence with a cool of periods.            & NEW             \\ \hline
        Configuration\ & A combinations of a DUT, measuring instrument and test case, temperature and battery.            & NEW             \\ \hline
    
    \end{tabular}
    \caption{Terminology used throughout our work.}
    \label{tab:TerminologyAlert}
    \end{table}
\todo{Check terminology is consistent}

The conceptual component provides an extension of the work by García et al.\cite{GarciaSMO} which defines Software Measurement Ontology. This extension is called Green Software Measurement Ontology (GSMO) and provides terminology and the corresponding definitions to help authors of papers about energy measurements describe their process with a specified terminology. This should help readers of such papers understand and compare results from different papers.\todo{Mention table with definitions} The Methodological component is a process called Green Software Measurement Process (GSMP) which are guidelines to assist with the study design, analysis and presenting the results. It is presented in seven phases, described as follows by Mancebo et al\cite*{GarciaFEETINGS}
\begin{enumerate}
    \item \textit{"Scope Definition"}: A requirements specification for evaluation of the results is made and the test cases are chosen.
    \item \textit{"Measurement Environment Setting"}: The DUT is chosen as well as the measuring instruments. The baseline measurements should also be acquired in this phase.
    \item \textit{"Measurement Environment Preparation"}: Preparation of environment
    \item \textit{"Perform the measurements"}: Experiment is conducted and data is recorded. 
    \item \textit{"Test Case Data analysis"}: The data is processed and analyzed. Here outliers can be found and a sanity check of the results is done.
    \item \textit{"Software Entity Data analysis"}: Conclusions about the experiments are started from the analysis in the previous step.
    \item \textit{"Reporting the results"}: Here the process of the study is documented, as well as the results. 
\end{enumerate} 

Mancebo et al.\cite*{GarciaFEETINGS} claim that the methodological component aid in giving more reliable and consistent measurements which in turn should make them more comparable and replicable. 

The technological component consists of two parts. The first part is the Energy Efficiency Tester (EET) which is a hardware-based approach to measuring energy consumption. It is presented in more detail by Mancebo et al. in \cite*{GarciaEET}. EET is a portable measuring instrument, which has three main components: A system microcontroller which is a Mega Arduino Development Board that is responsible for getting the data. Then there are a set of 9 sensors for measuring the energy consumption of different components of the DUT and the temperature. Lastly, a power supply, which should be used instead of the DUT's original power supply because the sensors are connected to the power supply.\cite*{GarciaEET} The second part is called ELLIOT, which is a software-based tool which handles the data provided by EET with the main goal of providing a visual representation that can be used to process, analyze and construct graphs and tables of the data collected by the EET. It can identify potential outliers and calculate different statistical variables.\cite{GarciaFEETINGS} However it should be noted that to the extent of our knowledge the ELLIOT tool is not available to the public and can therefore not be used.

In this study the accuracies of several different energy consumption measurements is tested "Power Measurement Techniques on Standard
Compute Nodes: A Quantitative Comparison" \cite{hackenberg2013}. Here they compare several different hardware measurement approaches to RAPL for intel and APM for AMD cpu's. There is especially a focus on the resolution of the measurements and when it matters. They find the for AC energy measurements sample down to 50ms yields more accurate information and granular information about the system. This is contrary to common beliefs that large capacitors in modern PSUs level out the consumption. They show that a resolution of $1-20$ samples per second is necessary to analyses the consumption of the different phases of an application. This is also close to the resolution provided by RAPL and APM though with some worse accuracies. For high resolution of $100+$ samples per second the AC technics cannot provide a granular enough view. They achieve good results with their DC measurement, but they note that some filtering is needed, as a lot of noise is present in the measurements. They also remark that the success of their measurements are highly dependent on the type of motherboard as different manufactures use different kinds and sizes of capacitors that could smooth out the measurements. As a conclusion they determine that RAPL and APM could need some improvements, but that RAPL is decently accurate and good enough to be used. For the hardware measurements the AC measurements were accurate enough and could provide them with a good granularity and that their DC setup was surprisingly accurate with a high sampling rate.