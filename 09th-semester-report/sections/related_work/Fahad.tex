Another work which compares different measuring instruments is the work by Fahad et al.\cite{fahad2019comparative}. To do this hardware measurements from an HCLWattsUp were used and then compared with energy predictive models, represented by RAPL and on-chip-sensors for Intel CPUs (MPSS). For the experiments conducted a high granularity is required to get the energy consumption of the different application phases. To achieve this and make the measurements comparable, the HCLWattsUp Meter is used and the concept of Dynamic Energy consumption is introduced. The results are comparable as the HCLWattsUp Meter measurements are comparable to RAPL's and on-chip-sensors measurements. Dynamic energy consumption is an estimation of the application's energy consumption and is calculated using the following formula:\cite{fahad2019comparative}

\begin{equation}\label{eq:dynamicEnergy}
    E_D = E_T - (P_S * T_E)
\end{equation}

Where $E_D$ is the dynamic energy consumption, $T_E$ is the duration of the program execution and $P_S$ is the energy consumption when the system is idle. $E_T$ is the total energy consumption of the system. The dynamic energy consumption will then represent the energy consumption of the test case. The HCLWattsUp meter measurements represent the ground truth of the experiments. The study shows that RAPL generally under reports the energy consumption of the system. The accuracy of RAPL is also tested in different scenarios to see if calibrations are feasible. Calibrations for RAPL and on-chip-sensors were generally not possible with their techniques in most scenarios. They provide some recommendations for future works and setup which will be covered in more depth later in \cref{sec:Measurement_Methodology}.