Another work comparing different measuring instruments is the work by Fahad et al.\cite{fahad2019comparative}. The hardware measurements were made using an HCLWattsUp and then compared with energy predictive models, represented by RAPL and on-chip-sensors for Intel CPUs Manycore Platform Software Stack (MPSS). For the experiments, a high granularity was required to get the energy consumption of the different phases of the test cases. To make the measurements made by the software and hardware-based measuring instruments comparable, the notion of dynamic energy consumption is introduced. This is required as the HCLWattsUp measures the energy consumption of the entire DUT, whereas the software-based measuring instruments only measure the energy consumption of the CPU and RAM. Dynamic energy consumption is an estimation of the test cases energy consumption and is calculated using the following formula:\cite{fahad2019comparative}

\begin{equation}\label{eq:dynamicEnergy}
    E_D = E_T - (P_S * T_E)
\end{equation}

Where $E_D$ is the dynamic energy consumption, $T_E$ is the duration of the program execution and $P_S$ is the energy consumption when the system is idle. $E_T$ is the total energy consumption of the system. The dynamic energy consumption will then represent the energy consumption of the test case. The HCLWattsUp meter measurements represent the ground truth of the experiments. The study shows that RAPL generally underreports the energy consumption of the system. The accuracy of RAPL is also tested in different scenarios to see if calibrations are feasible. Calibrations for RAPL and on-chip sensors were generally not possible with their techniques in most scenarios. They provide some recommendations for future works and setup which will be covered in more depth later in \cref{sec:Measurement_Methodology}.