\section{Key Takeaways}\label[section]{sec:rw_key_takeaways}

In this section, the major aspects of the related works covered in \cref{sec:rw_energy_consumption_diff_lang,sec:rw_diff_measuring_instruments,sec:rw_measureing_methodology_setup} are presented. 

In \cref{sec:rw_diff_measuring_instruments} the idea of comparing different measuring instruments as Japgroep et al.\cite{Jagroep2015} did was presented. However, only two measuring instruments could be set up to conduct measurements. Those were Joulemeter and Energy Consumption Tools (EC Tools). Joulemeter was last updated in 2011\cite{Joulemeter} and EC Tools was last updated in 2013\cite{ECTools}. Since the last updates are many years ago we decided against using them as we assumed they are discontinued. The idea of comparing different measuring instruments as Jagroep et al. did, is however the essence of our work as well. A measuring instrument used in multiple related works is RAPL\cite{RAPL_in_action,hackenberg2013,fahad2019comparative} and is as such relevant to include. However, as shown by Khan et al.\cite{RAPL_in_action} RAPL performs differently on different Intel CPU architectures, but still deems it to be a viable alternative to hardware-based measurements in many cases. A similar conclusion was made in the work by Fahad et al.\cite{fahad2019comparative} and Hackenberg et al.\cite{hackenberg2013} where RAPL was shown be be inaccurate compared to the hardware-based measuring instruments. A hardware-based measuring instrument will therefore be used as a ground truth in our work. In \cref{sec:rw_energy_consumption_diff_lang,sec:rw_diff_measuring_instruments,sec:rw_measureing_methodology_setup} different hardware measurement approaches are used, as can be seen in \cref{tab:Hardware_based_Measuring_instruments}. Another hardware based measuring instrument seen in the literature is Kill A Watt\cite{weaver2012measuring}, which is also shown in \cref{tab:Hardware_based_Measuring_instruments}. These will be considered when choosing the hardware-based approach to utilize in our work in \cref{subsec:HardwareMeasurementsIntro} and \cref{subsec:Equipment}.
\begin{table}[ht]
    \centering
    \begin{tabular}{| l | c |}
    \hline
    \textbf{Hardware-based Measuring Instrument}  & \textbf{Occurrences}                    \\ \hline
    Watts Up Pro        & 2      \\ \hline
    Plugwise            & 1      \\ \hline
    ZES ZIMMER LMG450   & 1      \\ \hline
    Custom setup        & 1      \\ \hline
    \end{tabular}
    \caption{The different hardware-based measuring instruments and how often they are used.}
    \label{tab:Hardware_based_Measuring_instruments}
    \end{table}

%  Watts  Up? PRO Jagroep
%  Plugwise Khan
% ZES ZIMMER LMG450 Hackenberg
% Watt up Pro Fahad
% Custom setup Mancebo

Furthermore, Hackenberg et al.\cite{hackenberg2013} provide insight to be considered regarding optimal sample rate depending on the type of measurements.

Another aspect presented by Fahad et al.\cite{fahad2019comparative} is the concept of dynamic energy consumption. Dynamic energy enables a comparison between the hardware-based measurements and software-based measurements used in our work, which will be chosen in \cref{ch:method}.\newline

% Fahad
% Hackenberg
% Khan

% Add terminology alert
In \cref{sec:rw_measureing_methodology_setup} different frameworks and aspects of methodology were presented. In this work, some of the contributions from Mancebo et al. \cite{GarciaFEETINGS} are incorporated to some extent. The incorporated contributions include some of the terminology defined in the GSMO, where some aspects in the GSMO are deemed irrelevant to our use case and will not be used. Some additional terminology is however also needed, which is included in the terminology used in this work in \cref{tab:TerminologyAlert}. The two other components are the methodological component and the technological components, both of which are not used. The methodological component which are guidelines to assist with the study design are not explicitly followed, since we were not aware of it in the initial phase of our work, however a similar processes is used. The technological component consisting of a custom hardware-based measuring instrument where the software tool is not available to the public and can therefore not be used.\newline

Sestoft\cite{sestoft2013microbenchmarks} presents several things to consider when running benchmarks. For example how the JIT compilation can affect the execution time of benchmarks and how garbage collection can also be an uncontrollable factor. Furthermore, a list of potential pitfalls is presented, which are considered when implementing the framework used in our work, as presented in \cref{sec:rw_measureing_methodology_setup}.\newline

Bokhari et al.\cite{Bokhari2020r3} came up with \textit{R3-VALIDATION} to avoid system software states changing from restarting the DUT and to generally improve the fairness of running several different test cases. The ideas of \textit{R3-VALIDATION} will be adapted in \cref{subsec:exp_rep} to better fit our use case although the core idea will remain the same.\newline

Finally, it was noted by Dongarra et al.\cite{Dongarra2012} that when comparing results gathered from different DUTs or measuring instruments it is useful to have the same sampling rate across the setups. This will also be implemented if possible within the limits of the different measuring instruments used in our work.

% Mancebo
% Sestoft
% Bokhari