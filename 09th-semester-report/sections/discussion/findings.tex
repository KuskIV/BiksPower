\section{Findings}

This part of the discussion will aim to discuss different results covered in \cref{ch:results}

\paragraph*{Energy consumption of Windows:} When comparing measurements made on Windows on Linux on the clamp, different conclusions can be made in experiment \#1 and \#2. In \#1, the energy consumption was in most cases higher on Windows, as can be seen in \cref*{sec:iterations}, where the energy consumption over time was analyzed. This conclusion is the opposite to what was found in Najmuddin et al. in \cite*[]{Najmuddin2021}. Because of this, the second experiment was conducted, where the C-states were disabled, as was covered in \cref*{sec:disable_c_states_exp}. In this experiment, the dynamic energy consumption of Windows decreased, while it remained the same for Linux, which shows two things. This first of all mean we have the same conclusion as Najmuddin et al.\cite*[]{Najmuddin2021}, but it also shows how the conclusion made by Najmuddin et al.\cite*[]{Najmuddin2021} could be correct. The conclusion stated how the energy consumption on Linux is higher compared to Windows because of a lack of well written drivers for Linux and based on the results in this work, it seems like Linux is unable to utilize the different C-states within a CPU, because the measurements remained the same when they were disabled or enabled.

\paragraph*{Experiment \#1}
%% can these values be used for anything?

\paragraph*{Experiment \#2}
%% are these more realistic with C-states disabled