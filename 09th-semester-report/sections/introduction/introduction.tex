In recent years there has been an increased focus on our emissions, and power consumption. These emissions are partly because of the amount of energy we as a species produce and subsequently use. 
The computer has after the IBM personal computer become a mainstay in pretty much every household in the developed world, and even in the developing world an increasing amount of people have access to their own computational devices such as computers and smartphones\cite{DevelopedWorldPC}. 

Because of this development, the energy consumed from these devices has become more and more relevant as they become increasingly common, and thus they also take up a larger percentage of the energy consumption. 
While hardware has become much more efficient over the years as transistor sizes have become smaller and thus uses less power, this will not be enough to handle the growth in the market for computation devices energy consumption. Because of this, a more in depth understanding of the energy consumption of hardware is interesting as software could help reduce the energy usage\cite{somavat2011energy}. 
In recent year studies of how and when software uses energy has been conducted, but it is still a very new field of study. It has been found that some software, depending on programming languages used, can have a large effect on the power consumption, even when they are functionally identical, where some languages use over 90x more power than others.
A majority of these studies have been conducted on Linux devices as they utilize the tool Intel RAPL, which is exclusive on linux. While the studies for linux are important, the literature does not indicate if the finding for linux energy consumption translates directly to Windows\cite{Pereira2017}. 

Windows has been the most commonly used operating system for personal computers for many years. This is still true at the time of writing, and it is therefore hugely relevant how energy is consumed on these devices as they make up a large percent of computers world-wide\cite{OSShare}. To monitor energy consumption on windows other programs with similar functionality to RAPL exist, but the actual reliability of these solutions are, to our knowledge, yet to be verified in the literature.

The aim of this work is therefore to compare different methods of measuring power consumption of software to see how the solutions for windows perform. 
In order to verify these results, hardware measurements will be utilized and the results will also be compared to Intel RAPL.

In order to conduct this research, four different research questions are formulated. The overall idea is to compare different energy profilers on the same hardware, but as some profilers either requires specific hardware or claims to perform better on certain hardware, this will also be explored. This means several energy profilers will measure the energy consumption on several systems in order to see how well they perform compared to RAPL and a hardware measuring.


\begin{itemize}
    \item \textbf{RQ1:} How do we systematically measure the power consumption of software on different hardware and OS and how do we make the results comparable.
    \item \textbf{RQ2:} How do existing solutions to power measurements compare to each other?
    \item \textbf{RQ3:} How does the measured power consumption compare between windows and linux measurements?
    \item \textbf{RQ4:} How does the measured power consumption compare between different hardware?
\end{itemize}

Since multiple hardware setups are tested, \textbf{RQ1} addresses how a way to compare the results is required. This is required before a comparison can be made in between the different energy profilers in \textbf{RQ2}. Based on \textbf{RQ2}, an analysis will be made comparing the profilers with each other based on performance on different operating systems systems (\textbf{RQ3}) and different hardware (\textbf{RQ4}).

This work is motivated by how most research today mainly uses Intel RAPL on Linux\cite*[]{Rasmussen2021,Pereira2017,Theilmann2022,Lindholt2022}, and those who do not, has the disclaimer of not knowing how accurate their results actually are\cite[]{Bruce2015ReducingEC, Ozturk2019, Unlu2021}. This is despite claims from Microsoft how their tool can perform with 98\% accuracy in certain cases\cite[]{E3WinHec}. The main contribution of this work is to gather an independent source of how accurate different energy profilers are.



%% Last question should not be a yes/no question. 

% Break down research question to a plan of attack, how do we actually find out.

