In recent years there has been an increased focus on our emissions, and power consumption.

These emissions are partly because of the amount of energy we as a species produce and subsequently use. 
The computer has after the IBM personal computer become a mainstay in pretty much every household in the developed world, and even in the developing world an increasing amount of people have access to their own computational devices such as computers and smartphones. 

Because of this development, the energy consumed from these devices has become more and more relevant as they become increasingly common, and thus they also take up a larger percentage of the energy consumption. 
While hardware has become much more efficient over the years as transistor sizes have become smaller and thus uses less power, this will not be enough to handle the growth in the market for computation devicess energy consumption. Because of this, a more in depth understanding of the energy consumption of hardware is interesting as software could help reduce the energy usage. 
In recent year studies of how and when software uses energy has been conducted, but it is still a very new field of study. It has been found that some software, depending on programming languages used, can have a large effect on the power consumption, even when they are functionally identical, where some languages use over 90x more power.
A majority of these studies have been conducted on Linux devices as they utilize the tool Intel RAPL, which is exclusive on linux. While the studies for linux are important, the literature does not indicate if the finding for linux energy consumption translates directly to Windows. 

Windows has been the most commonly used operating system for personal computers for many years. This is still true at the time of writing, and it is therefore hugely relevant how energy is consumed on these devices as they make up a large percent of computers world-wide. To monitor energy consumption on windows other programs with similar funcationality to RAPL exist, but the actual reliability of these solutions are, to our knowledge, yet to be verified in the literature.

The aim of this work is therefore to compare the different methods of measuring power consumption of programs to see if the solutions for windows are reliable. 
In order to verify these results, hardware measurements will be utilized and the results will also be compared to RAPL.

To do this we have formulated 4 different research questions that we would like to answer during, to gain insight into the solution on windows.

\begin{itemize}
    \item How do we systematically measure the power consumption of software on different hardware and OS and how do we make the results comparable.
    \item How do existing solutions to power measurements compare to each other?
    \item How does the measured power consumption compare between windows and linux measurements?
    \item Do some problems favour some OS or hardware, or does the same trend continue on both machines.
\end{itemize}

%% Last question should not be a yes/no question. 

% Break down research question to a plan of attack, how do we actually find out.

