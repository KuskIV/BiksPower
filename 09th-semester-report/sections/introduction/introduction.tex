
In recent years there has been an increased focus on emissions and energy consumption. One factor impacting the increased energy consumption is the rise of the computer after the IBM personal computer became a mainstay in most households. The computer can be found in most households in the developed world, and an increasing amount of people have access to a computational device in the developing world, like computers and smartphones.\cite{DevelopedWorldPC}. 

\paragraph*{}
Because an increasing number of computers exist, they are responsible for a growing percentage of global electricity consumption, where the percentage of energy consumed by computers on a global scale is estimated to be $\approx 6\%$\cite{somavat2011energy}. While hardware has become much more efficient over the years as transistor sizes have become smaller and thus uses less energy, this will not be enough to handle the growth in the market for computational devices\cite{procaccianti2011profiling}. Because of this, a more in-depth understanding of the energy consumption of hardware\todo{software?} is interesting as software could help reduce the energy usage\cite{somavat2011energy}. In recent years several studies of software's energy consumption have been conducted, but it is still a newer field of study. Pereira et al.\cite{Pereira2017} found that for some test cases the programming languages used, can have a large effect on the energy consumption, even when they are functionally identical, where some languages use over 90x more energy than others. A majority of these studies have been conducted on Linux devices, where energy measurements are performed using the measuring instrument Intel's Running Average Power Limit (RAPL) on Linux. While the studies for Linux are important, the literature does not indicate if the findings for Linux energy consumption translate directly to Windows\cite{Pereira2017}. 

\paragraph*{}
Considering the energy consumption of Windows is relevant since Windows has been the most commonly used operating system for personal computers for multiple years. This is still true at the time of writing, and it is, therefore, relevant to know how energy is consumed on these devices as they make up a large percentage of computers worldwide\cite{OSShare}. To monitor energy consumption on Windows, other measuring instruments with similar functionality to RAPL exist, but the actual reliability of these solutions is, to our knowledge, yet to be verified in the literature.

\paragraph*{}
The aim of this work is therefore to compare different measuring instruments and to explore how the solutions for Windows perform. To verify these results, a hardware-based measuring instrument will be utilized as a ground truth and the results will also be compared to RAPL. To conduct this study, four different research questions are formulated. The overall idea is to compare different measuring instruments on the same Device Under Test (DUT), but as one measuring instruments require specific hardware or claim to perform better on certain hardware, this will also be explored. This means several measuring instruments will measure the energy consumption on several different DUTs to see how well they perform compared to RAPL and a hardware-based measuring instrument. The main contributions of this work are formulated as research questions, defined as the following:


\begin{itemize}
    \item \textbf{RQ1:} How do we systematically measure the energy consumption of a test case on different DUTs and OSs and how do we make the test case measurements comparable?\label[research question]{RQ1}
    \item \textbf{RQ2:} How do existing energy measuring instruments compare to each other?\label[research question]{RQ2}
    \item \textbf{RQ3:} How does the test case measurement compare between Windows and Linux?\label[research question]{RQ3}
    \item \textbf{RQ4:} How does the test case measurement compare between different DUTs?\label[research question]{RQ4}
\end{itemize}

The overall goal of this work is to compare existing measuring instruments (\textbf{RQ2}). This will be done based on operating systems (OS) (\textbf{RQ3}) and DUTs (\textbf{RQ4}). When comparing measuring instruments, it will be done with different test cases. To achieve this, a systematic way of measuring energy consumption and making them comparable across different DUTs and OSs, as stated in \textbf{RQ1} is required.

\paragraph*{}
This work is motivated by how most research today mainly uses RAPL on Linux\cite[]{Rasmussen2021,Pereira2017,Theilmann2022,Lindholt2022}, and those who do not, have the disclaimer of not knowing how accurate their results are\cite[]{Bruce2015ReducingEC, Ozturk2019, Unlu2021}. This is despite claims from Microsoft that their tool, Energy Estimation Engine, (E3) can measure the energy consumption of software with 98\% accuracy in certain cases\cite[]{E3WinHec}. The main contribution of this work is to create an independent source of how accurate different measuring instruments are.

\paragraph*{}
In \cref{ch:related_work} a look into existing work will be made, for inspiration. Following this, \cref{ch:method} will aim to answer \textbf{RQ1}, by introducing the framework created, and how to make measurements comparable. In \cref{ch:experiments}, the different experiments performed in order to answer \textbf{RQ2-4} will be explored, and analyzed in \cref{ch:results}. Lastly, the results will be discussed and concluded in \cref{ch:discussion} and \cref{ch:conclusion} respectively.






%% Last question should not be a yes/no question. 

% Break down research question to a plan of attack, how do we actually find out.

