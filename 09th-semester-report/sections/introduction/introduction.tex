In recent years there has been an increased focus on our emissions, and energy consumption. 
The computer has after the IBM personal computer become a mainstay in most households in the developed world, and even in the developing world an increasing amount of people have access to their own computational devices such as computers and smartphones\cite{DevelopedWorldPC}. 

\paragraph*{}
Because an increasing number of computers exists, they consume and increasing percentage of the global electrical consumption, where the percentage of energy consumed by computers on a global scale is estimated to be 6\%\cite{somavat2011energy}. While hardware has become much more efficient over the years as transistor sizes have become smaller and thus uses less energy, this will not be enough to handle the growth in the market for computational devices\cite{procaccianti2011profiling}. Because of this, a more in depth understanding of the energy consumption of hardware is interesting as software could help reduce the energy usage\cite{somavat2011energy}. In recent years several studies of software's energy consumption has been conducted, but it is still a very new field of study. Pereira et al.\cite{Pereira2017} found that for some test cases the programming languages used, can have a large effect on the energy consumption, even when they are functionally identical, where some languages use over 90x more energy than others. A majority of these studies have been conducted on Linux devices, where energy measurements are performed using the Linux exclusive measuring instrument Intel's Running Average Power Limit (RAPL). While the studies for Linux are important, the literature does not indicate if the findings for Linux energy consumption translates directly to Windows\cite{Pereira2017}. 

\paragraph*{}
When considering the energy consumption on Windows, this is an important question to answer as Windows has been the most commonly used operating system for personal computers for many years. This is still true at the time of writing, and it is therefore hugely relevant how energy is consumed on these devices as they make up a large percent of computers world-wide\cite{OSShare}. To monitor energy consumption on Windows other measuring instruments with similar functionality to RAPL exist, but the actual reliability of these solutions are, to our knowledge, yet to be verified in the literature.

\paragraph*{}
The aim of this work is therefore to compare different measuring instruments and to explore how the solutions for Windows perform. In order to verify these results, a hardware-based measuring instrument will be utilized and the results will also be compared to RAPL. In order to conduct this study, four different research questions are formulated. The overall idea is to compare different measuring instruments on the same Device Under Test (DUT), but as some measuring instruments require specific hardware or claims to perform better on certain hardware, this will also be explored. This means several measuring instruments will measure the energy consumption on several different DUTs in order to see how well they perform compared to RAPL and a hardware-based measuring instrument.


\begin{itemize}
    \item \textbf{RQ1:} How do we systematically measure the energy consumption of a test case on different DUTs and OSs and how do we make the test case measurements comparable?\label[research question]{RQ1}
    \item \textbf{RQ2:} How do existing energy measurements instruments compare to each other?\label[research question]{RQ2}
    \item \textbf{RQ3:} How does the test case measurement compare between Windows and Linux?\label[research question]{RQ3}
    \item \textbf{RQ4:} How does the test case measurement compare between different DUTs?\label[research question]{RQ4}
\end{itemize}

The overall goal of this work is to compare existing measurement instruments (\textbf{RQ2}). This will be done based on both operating systems (\textbf{RQ3}) and DUTs (\textbf{RQ4}). When comparing measurement instruments, it will be done with different test cases. In order to achieve this, a systematic way of measuring energy consumption and make the comparisons comparable across different DUTs and OSs, as stated in \textbf{RQ1}.

\paragraph*{}
This work is motivated by how most research today mainly uses RAPL on Linux\cite[]{Rasmussen2021,Pereira2017,Theilmann2022,Lindholt2022}, and those who do not, have the disclaimer of not knowing how accurate their results actually are\cite[]{Bruce2015ReducingEC, Ozturk2019, Unlu2021}. This is despite claims from Microsoft how their tool (Enegy Esimation Engine (E3)) can perform with 98\% accuracy in certain cases\cite[]{E3WinHec}. The main contribution of this work is to create an independent source of how accurate different measuring instruments are.

\paragraph*{}
In \cref{ch:related_work} a look into existing work will be made, for inspiration. Following this, \cref{ch:method} will aim to answer \textbf{RQ1}, by introducing the framework created, and how to make measurements comparable. In \cref{ch:experiments}, the different experiments performed in order to answer \textbf{RQ2-4} will be explained, and analyzed in \cref{ch:results}. Lastly, the results will be discussed and concluded upon in \cref{ch:discussion} and \cref{ch:conclusion} respectively.






%% Last question should not be a yes/no question. 

% Break down research question to a plan of attack, how do we actually find out.

