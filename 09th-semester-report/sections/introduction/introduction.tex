\todo{Beskriv vores arbejdsmetode}In recent years there has been an increased focus on our emissions, and energy consumption. These emissions are partly because of the amount of energy we as a species produce and subsequently use. 
The computer has after the IBM personal computer become a mainstay in pretty much every household in the developed world, and even in the developing world an increasing amount of people have access to their own computational devices such as computers and smartphones\cite{DevelopedWorldPC}. 

Because of this development, the energy consumed from these devices has become more and more relevant as they become increasingly common, and thus they also take up a larger percentage of the energy consumption.\todo{How much?} 
While hardware has become much more efficient over the years as transistor sizes have become smaller and thus uses less energy, this will not be enough to handle the growth in the market for computational devices. Because of this, a more in depth understanding of the energy consumption of hardware is interesting as software could help reduce the energy usage\cite{somavat2011energy}. 
In recent years several studies of software's energy consumption has been conducted, but it is still a very new field of study. Pereira et al.\cite*{Pereira2017} found that for some test cases the programming languages used, can have a large effect on the energy consumption, even when they are functionally identical, where some languages use over 90x more energy than others. Where a test case is defined as \textit{"A representation of the functionality of the software entity to be measured"} by Garcia et al.\cite*{GarciaFEETINGS}
A majority of these studies have been conducted on Linux devices as they utilize a measuring instrument, which is \textit{"a method used to make energy consumption measurements"} by Garcia et al.\cite*{GarciaFEETINGS} The measuring instrument is Intel Running Average Power Limit (RAPL), which is exclusive to Linux. While the studies for Linux are important, the literature does not indicate if the findings for Linux energy consumption translates directly to Windows\cite{Pereira2017}. 

Windows has been the most commonly used operating system for personal computers for many years. This is still true at the time of writing, and it is therefore hugely relevant how energy is consumed on these devices as they make up a large percent of computers world-wide\cite{OSShare}. To monitor energy consumption on windows other programs with similar functionality to Intel RAPL exist, but the actual reliability of these solutions are, to our knowledge, yet to be verified in the literature.

The aim of this work is therefore to compare different measuring instruments and to explore how the solutions for windows perform. 
In order to verify these results, a hardware based measuring instrument will be utilized and the results will also be compared to Intel RAPL.

In order to conduct this study, four different research questions are formulated. The overall idea is to compare different measuring instruments on the same Device Under Test(DUT), but as some measuring instruments require specific hardware or claims to perform better on certain hardware, this will also be explored. This means several measuring instruments will measure the energy consumption on several different DUTs in order to see how well they perform compared to Intel RAPL and a hardware based measuring instrument.


\begin{itemize}
    \item \textbf{RQ1:} How do we systematically measure the energy consumption of a test case on different DUTs and OSes and how do we make the test case measurements comparable?
    \item \textbf{RQ2:} How do existing solutions to energy measurements compare to each other?
    \item \textbf{RQ3:} How does the test case measurement compare between windows and Linux?
    \item \textbf{RQ4:} How does the test case measurement compare between different DUTs?
\end{itemize}

\textbf{RQ1} signifies that several DUTs have to be tested and a method to compare the results is required, so that the different measuring instruments can be compared in \textbf{RQ2}. Based on \textbf{RQ2}, an analysis will be made comparing the measuring instruments with each other based on performance on Linux and Windows (\textbf{RQ3}) and different DUTs with different hardware components (\textbf{RQ4}).

This work is motivated by how most research today mainly uses Intel RAPL on Linux\cite[]{Rasmussen2021,Pereira2017,Theilmann2022,Lindholt2022}, and those who do not, have the disclaimer of not knowing how accurate their results actually are\cite[]{Bruce2015ReducingEC, Ozturk2019, Unlu2021}. This is despite claims from Microsoft how their tool (Enegy Esimation Engine (E3)) can perform with 98\% accuracy in certain cases\cite[]{E3WinHec}. The main contribution of this work is to create an independent source of how accurate different measuring instruments are.



%% Last question should not be a yes/no question. 

% Break down research question to a plan of attack, how do we actually find out.

