\subsection{Sample size formulas}
It is important to get enough measurements for each test case such that the test case measurements reflect a representative mean and standard deviation. Mean while gathering too many measurement, would be a waste of resources therefore an approach to determine a sufficient amount is needed. There are several approaches used in the literature. For example, some papers use what seems like an arbitrary number of samples\cite{Pereira2017,Koedijk2022diff,Georgiou2020}.\todo{probably central limit theorem, but don't remember.} Another method is to base the number of samples on how many times the experiment can be run within a time-frame\cite{sestoft2013microbenchmarks}. In this report, a formula will be used to calculate the sample size. One formula or family of formulas are Cochran's formula.\cite{Cochran, kotrlik2001organizational,israel1992determining}, which gives the minimum number of required observations for performance metrics within a specified standard deviation. However, there are several versions of the formula.
The first one in \cref{cochransEQ1}\footnote{Note the terminology used by Cochran is different to the one utilized in this report} is for categorical data when the population is large\cite{israel1992determining}:

\begin{equation}
    n_0 = \frac{Z^2*p*q}{e^2}
    \label[equation]{cochransEQ1}
\end{equation}

\begin{itemize}
    \item $n_0$ is the number of samples
    \item $Z$ is the abscissa of the normal curve which removes an area $\alpha$ at the tails of the distribution. (Z-score) Where $1 - \alpha$ is the desired confidence level.%is the z-value, which is found using the z-table and represents the confidence level
    \item $pq$ is an estimate of variance, where
    \begin{itemize}
        \item $p$ is the estimated proportion of an attribute in the data
        \item $q$ is $1 - p$
    \end{itemize}
    \item $e$ is desired level of precision (acceptable margin of error)
\end{itemize}


Then there is a correction step for when the sample size is larger than 5\% of the population. As well as for smaller populations a smaller sample size is necessary.\cite{israel1992determining,kotrlik2001organizational} The equation is shown in \cref{cochransCorrection}.

\begin{equation}
    n = \frac{n_0}{1+\frac{(n_0-1)}{N}}
    \label[equation]{cochransCorrection}
\end{equation}

Where $n$ is the new sample size and $N$ population size.



Furthermore, there is a simplified formula called Yamane's formula, shown in \cref{yamane}\cite{israel1992determining}. 

\begin{equation}
    n = \frac{N}{1 - N(e)^2}
    \label[equation]{yamane}
\end{equation}


Lastly, there is another version of Cochran's formula shown in \cref{cochransEQ2}, Which is for continuous data. 

\begin{equation}
    n_0 = \frac{Z^2*\sigma^2}{e^2}
    \label[equation]{cochransEQ2}
\end{equation}

Where instead of $pq$ we have $\sigma$ which is the standard deviation of the data. The correction formula can also be used in combination with this formula.\nytafsnit



It should still be considered if categorical variables will play an important role in the data analysis, then \cref{cochransEQ1} should be used\cite{kotrlik2001organizational}. For our data \cref{cochransEQ2} seem to most accurately fit the description and is therefore chosen. The correction formula \cref{cochransCorrection} is not needed since we, in theory, have an infinite population, while in practice it is of course limited by time. With Cochran's formula given a desired margin of error, desired confidence level and an estimate of the standard deviation a sample size can be calculated. So these variables are determined as follows. 

The confidence level is how little the results must deviate. A confidence level of 95\% would represent 95\% of the data points would match 95\% of the times. If a confidence level of 95\% is desired and confidence level $= 1 - \alpha$ then $\alpha = 0.05$. 
Kotrlik et al found that a margin of error of 5\% is commonly chosen and is found acceptable for categorical data, but for continuous data, a 3\% margin of error is found acceptable\cite{kotrlik2001organizational}. Since the variables measured are continuous data a margin of error of 3\% is chosen.

The Z-score reflects how many standard deviations a sample is from the mean. An approximate score can be found in a Z-table when there is a desired confidence level or $\alpha$. The most commonly used $\alpha$ is $0.05$ or $0.01$.\cite{kotrlik2001organizational} In this report $0.05$ is chosen which gives a confidence level of 95\%. \todo{I have not found a reason as to why yet.}From the Z-table the estimated Z-value is then $1.96$. 

Lastly, an estimate of the standard deviation is also needed, which is not available. Cochran listed four methods for estimating the standard deviation in case it is not initially available:
\begin{enumerate}
    \item The sample is taken in two steps. The first step is used to determine how many further samples are required in step two, based on the standard deviation in the first set of samples.
    \item Utilizing results of a pilot study
    \item Utilizing results from a similar study
    \item Come up with a guess assisted with logical-mathematical results.
\end{enumerate}


Method one and two are both based on the same principle of getting an initial smaller sample. Method three is to the extent of our knowledge not feasible since no results from a study similar enough are available. Method four requires more knowledge than we possess to give a qualified estimate. Therefore method one is chosen. Now for step one when making the small sample it is required to know how many samples to acquire. The Central Limit Theorem says that the mean of a sample will become close to the mean of the overall population in correlation with an increase in the sample size. Ross found that at least 30 samples is enough for the central limit theorem to hold\cite{Ross}. Which means the distributions of the sample means are close to normally distributed.

% Influence of pragmming paradigms picks 100 based on 9781118805350.

From the initial sample, an estimated standard deviation of each parameter measured can be acquired. Then the minimum number of samples required can be calculated for each parameter. Whereafter the largest of the values are chosen and all of the experiments are run that number of times to get the minimum required samples.


% Focus on report influence of programming paradigms.

% Probabply not normal distributed - Jeppe alcholt