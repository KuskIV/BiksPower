\subsection{Correlation}
Correlation is a statistical measurement that essentially informs us of the relationship between two samples. The correlation can be either positive or negative and a measure of how correlated they are. A strong correlation does however not mean that changes i one variable effects the other just that it statically seems to be the case.
\paragraph{Pearson correlation coefficient} The Pearson correlation coefficient is one of if not the most used method for finding the correlations in linear relationships. The strength of the correlation is spans between $-1 \dotsc 1$, were the directions is given by the polarity sign. Depending on the context different strength are considered as strong or weak, but commonly 0 means that there is no correlation between the two samples. Some Assumptions have to be true about the data for us to utilize Pearson correlation coefficient:
\begin{itemize}
    \item The data must be quantitative
    \item The two samples must be normally distributed
    \item There must be no outliers in the data
    \item The relationship between the two must also be linear
\end{itemize}
If all of these assumptions are true for the data then the Pearson correlation coefficient can be calculated with the formula:
$$r=\frac{n\sum{xy- (\sum{x})(\sum{y})}}{\sqrt{[n\sum{x^2}-(\sum{x})^2][n\sum{y^2}-(\sum{y}^2)]}}$$
If however the data does not follow the conditions then another method of calculating the correlation will have to be used. 