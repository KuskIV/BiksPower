\subsection{Correlation}\todo{sources}
Correlation is a statistical measurement informing about the relationship between two samples. The correlation can be positive or negative and is a measure of how correlated they are. A strong correlation does not mean that changes in one variable affect the other just that it statically seems to be the case. Commonly $0$ means there is no correlation between the two samples. A positive value means that the two samples are positively correlated, meaning that they seem to grow together. A negative value inversely means as one shrinks the other grows.

\paragraph{Pearson correlation coefficient} The Pearson correlation coefficient is a frequently used method for finding correlations in linear relationships\cite{armstrong2019should}. The strength of the correlation spans between $-1 \dotsc 1$, where the directions are given by the polarity sign. Depending on the context, different strengths are considered strong or weak. Some assumptions have to be true about the data for us to utilize the Pearson correlation coefficient:

\begin{itemize}
    \item The data must be quantitative
    \item The two samples must be normally distributed
    \item There must be no outliers in the data
    \item The relationship between the two must be linear
\end{itemize}

If all of these assumptions are true for the data then the Pearson correlation coefficient can be calculated with the formula:

\begin{equation}
    r=\frac{\sum{(x_i-\bar{x})(y_i-\bar{y})}}{\sqrt{\sum{(x_i-\bar{x})^2}\sum{(y_i-\bar{y})^2}}}
\end{equation}

If however, the data does not follow the conditions, another method of calculating the correlation will have to be used. 

\paragraph{Kendall Tau correlation}
There are two common choices when doing non-parametric rank correlation, which can be used on non-normally distributed data. These are Spearman's rank correlation coefficient and Kendall Tau's correlation coefficient[han1987non]. While both can be used in the same scenarios there are some differences between them. Both are ranked with the same procedure that Mann-Whitney U tests are. The Spearman rank seems to be the most popular way of measuring correlations, but some argue that Kendall Tau is better in most cases \cite{gilpin1993table}. Some of the big differences between the two are the effect of outliers, Spearman's rank is largely affected by a few outliers in the data while Kendall Tau in comparison seems more robust. Because of this the choice to utilize Kendall Tau seems the most natural and is what will be continued with. The results from Kendall Tau can be interpreted the same way as from Pearson, the range is from $-1$ to $1$. The formula for Kendall Tau is:

\begin{equation}
    \frac{C-D}{C+D}
\end{equation}

$C$ is Concordant pairs and $D$ is Discordant. A Concordant pair is ordered, while Discordant is disordered pairs\cite{kendall1938new}. To evaluate the correlations the scale presented by Guildford in \cite[219]{guilford1950fundamental} can be used.
\begin{itemize}
    \item $<.20$: Slight; almost negligible relationship
    \item $.20-.40$: Low correlation; definite but small relationship
    \item $.40-.70$: Moderate correlation; substantial relationship
    \item $.70-.90$: High Correlation; marked relationship
    \item $.90-1$: Very high correlation; very dependable relationship
\end{itemize}

This is the scale that will be worked with to evaluate the correlations of the experiments.

