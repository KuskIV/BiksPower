\subsection{Distribution}
Here our method for statistically comparing the distribution of the different measurement will be elaborated and explained.
\todo{Some inconsistency about p}
\paragraph{Shapiro-Wilk}
To answer the first question it is important to find out which statistical tests that could be used on the data. One of the most common preconditions for statistical test is that the data should be a normal distribution for the results to work. To test if a distribution is a normal Distribution a normalcy test have to be used. While there are many different normalcy tests it seems that the Shapiro-Wilk test is generally more powerful than the others that are commonly used \cite{razali2011power}. Because of this the Shapiro-Wilk test will be used to check for normal distributions. The formula for the Shapiro-Wilk test:
$$W=\frac{( \sum{a_i x_i} )^2}{\sum{(x_i - \bar{x})^2}}$$
The null hypnosis's $H_0$ for the test is:
\textit{The sample comes from a normally-distributes population}

To find out whether $H_0$ can be rejected, the $p_{value}$ is used. The $p_{Value}$ or probability value is essentially the probability of the achieving the same results under $H_0$ of a statistical test. We also have a value alpha which is essentials a value we set to determine the statistical significance, a common threshold is $0.05$. If the $p_{Value}$ is less than alpha the we can reject $H_0$. Using the $p_{value}$ is commonplace in statistical testing, and it is also used for the rest of the tests in this section, unless specified otherwise all test are conducted with an alpha of $0.05$.
\paragraph{T-Test}
If the $H_0$ for the Shapiro-Wilk test could not be rejected we assume that the distribution are normal or close enough not to be important.

In the case of a normal distribution we a can use a students t-test to check if there is a statistically significant difference between the two samples. There are a couple of conditions when using t-test these are:
\begin{itemize}
    \item Both samples should follow a normal Distribution.
    \item The samples are independent of each other.
    \item Both the samples should also have approximately the same variance.
\end{itemize}
The null hypothesis $H_0$ for the t-test is as follows: \textit{There is no statistically significant difference between the samples}

The t-test is essentially performed using two values the $t_{value}$ and the $p_{value}$:
The formula for the calculation of the t value can be seens:
$$t_{Value} = \frac{|\bar{x_1}- \bar{x_2}|}{\sqrt{\frac{S_1^2}{n_1} + \frac{S_2^2}{n_2}}}$$ 

\paragraph{Mann–Whitney U test}
If however it was determined the the distribution are not normally distributed then another test will have to be made instead of the student t-test. There are some important differences between the two methods. The t-test calculates the difference in the mean between two samples, while the Mann–Whitney U test check is there is a difference in the rank sum of the samples\cite{mann1947test}. 

The idea behind the rank sum is essentially that each datapoint is given a rank, this rank is based one its position in a sorted order between all of the other datapoints. 
The basic requirements for using the Mann–Whitney U test is to have ordinal variables, this essentially just means that there is a way to order and sort them.

The null hypothesis $H_0$ for the Mann–Whitney U test is defined as \textit{In the population, the sum of the rankings in the two group's does nor differ.}
Actually doing the calculations require a series of calculations where the initial is finding $U_1$ and $U_2$.

The actual calculations for each group consist of the number of cases in the group $n_1$ and rank sum $T_1$. These are then used in the formula:
$$U_1 = n_1*n_2+\frac{n_1*(n_1+1)}{2}-T_1$$ 
$$U_2 = n_1*n_2+\frac{n_2*(n_2+1)}{2}-T_2$$

The smallest of these two will then be used in the test as $U=min(U_1,U_2)$
The next calculations cover the calculations of $\varphi U$ and $\sigma U$.
$\varphi U$ is the expected value of $U$, while $\sigma U$ is the standard error of $U$.
$$\varphi U = \frac{n_1*n_2}{2}$$
$$\sigma U = \sqrt{\frac{n_1*n_2*(n_1+n_2+1)}{12}}$$
Then to calculated the value $z$.
$$z = \frac{U-\varphi U}{\sigma U}$$

using z we can now find the p-value to check if we can reject $H_0$, again the alpha is set to $0.05$. 