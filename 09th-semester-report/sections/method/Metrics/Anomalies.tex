\subsection{Anomalies}
Anomaly detection is important when using the data as i will influence the results from the experiments. When an anomaly is detected it will be removed from teh data as to not corrupt the rest of teh results.
\paragraph{DBScan}
To find and remove these anomalies the DBScan algorithm will be used. The DBScan algorithm is essentials a clustering algorithm that is used to cluster data into groups, but is can also be used for density based abnormality detection. The psudo code for the DBScan algorithm\cite{ester1996density}.
\begin{lstlisting}
DBSCAN(D(*@\feetnote{D is the dataset}@*), eps(*@\feetnote{esp is a radius around a point}@*), MinPts(*@\feetnote{MinPts is the minimum amount of points found in the radius of a point to be a core point}@*))
{
    C = 0;
    foreach unvisited point p in dataset D
    {
        mark P as visited;
        NeighborPts = regionQuery(P, eps);
        if(sizeof(neighborPts) < MinPts)
        {
            mark P as NOISE;
        }else
        {
            C = next cluster;
            expandCluster(P, NeighborPts, C, eps, MinPts);
        }
    }
}
expandCluster(P, NeighborPts, C, eps, MinPts){
    add P to cluster C;
    foreach( point P' in NeighborPts)
    {
        if(P' is not visited)
        {
            mark P' as visited;
            NeighborPts' = regionQuery(P', eps);
            if(sizeof(NeighborPts')>= MinPts)
            {
                NeighborPts = NeighborPts joined with NeighborPts';
            }
        }
        if(P' is not yet member of any cluster)
        {
            add P' to cluster C
        }
    }

}
regionQuery(P, esp)
{
    return all points within P's eps-neighborhood (including P)
}
\end{lstlisting}

Essentially the algorithm goes trough each point and count how many other point are in the radius around if this i equal to or larger than the a set threshold then the point will be labeled a core point. The points that does not fit this criteria, but are still inside the radius of a core point, are instead labeled border points. The points that have not received a label yet are then outlier or anomalies. The variable assignment of esp and MinPts will be done via the following procedures. According to Ester et al a goog MinPts value for 2 dimensional data is 4\cite{ester1996density}. For the esp it has to be calculated, from each point the distance to the nearest $MinPts$ neighbors are found. These distances are then sorted and plotted on a graph, where the line breaks would be a good value of esp.


