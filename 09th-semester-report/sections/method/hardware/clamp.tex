\paragraph*{AC Current clamp MN60}

This section will cover the electrical probe found that meet the criteria from \ref{sec:clampIntro}. For the measurements, an AC clamp was used specifically the MN60 created by Chauvin Arnoux. The MN60 comes from reputable manufacturer\todo[]{Add Source}, and is available from multiple vendors, at a relatively low price of around 400 euros. The MN60 is an analog sensor meaning that the output is an analog signal\cite{agarwal2005foundations}. The analog sensor's measurements are continuos meaning that it does not really have a sampling rate as the sensor responds immediately to the environment. The MN600 sampling rate is controlled entirely by the device used the read the signal from it. To read the signal the current clamp has to be connected to an oscilloscope, and the clamp itself is connected to the electrical phase of the wire leading to the power supply for the computer. An illustration of the proposed usage of the current clamp together with an oscilloscope can be seen in \cref{fig:clampSetup}
\begin{figure}[h!]
    \centering
    \includegraphics*[scale=0.4]{figures/CLAMP.png}
    \caption{Here the MN60 current clamp can be seen connected onto the phase of the wire into the computer}
    \label{fig:clampSetup}
\end{figure}
The output from the current is in $mv$ that has a specific conversion rate to a current root mean square (RMS) value, which together with the wall socket specification, this being 230 volts in the EU, can be used to calculate the exact amount of joules flowing through the wire at any point in time. The sampling frequency of the current clamps is depended on the oscilloscope used, the accuracy of the clamp is claimed to be $<1\%$\cite{ClampDoc}. As can be seen in \ref{fig:clampSetup} there is no exposed wiring, that could be accidentally touched or shorted out, making it really safe and easy to use probe, especially when measuring mains voltage.


% The current clamp used and some of the specifications like the accuracy and the potentially hight sample rate was briefly covered in \cref{sec:clampIntro}.
% The current clamp can measure the current passing trough the phase of a wire, which can be used to derive the joule passing trough it. The benefits of using a current clamp as opposed to other methods, is that there is not exposed wiring that could accidentally be interacted with. This was a priority as mains voltage can be lethal and direct measurements could come with some risk. Though this added safety does come with certain drawbacks, current clamps are less accurate than more direct ways of measuring the power when looking at a price to accuracy ration\todo[]{source on this, and also, isn't this irrelevant?}. Another limitation with the current clamp compared to for example the WattsUpMeter is that an oscilloscope is required to read the output from the clamp.
