\subsection{Setup}\label[subsec]{subsec:setup}

As was covered in \cref{sec:measuring_instruments}, four different measuring instruments were chosen, based on different criteria. Because these measuring instruments require different operating systems, the setup will be for both Windows and Linux. Additionally, E3 has some specific demands, regarding under which circumstances the measuring instruments perform best\cite{E3Doc,E3Video}. E3 first of all requires the DUT to have a battery and second of all claims to measure most accurate when the DUT has a special MAXIM chip. In order to obtain a ground truth using a hardware measuring instrument, a DUT without a battery is required, as this makes it possible to measure the energy consumption from the power outlet. Because of this, three different DUTs were chosen, to test all these cases:

\begin{itemize}
    \item A DUT with a battery and a MAXIM chip
    \item A DUT with a battery and no MAXIM chip
    \item A DUT with no battery and no maxim chip
\end{itemize}

Furthermore, E3 also claimed to work best when the charger is disabled\cite{E3Doc}, which is an additional requirement when answering \textbf{RQ1}. When considering how to disable/enable the charger of the DUT, smart plugs are a good solution for this. In order to compare the performance of the different software-based measuring instruments, a hardware solution is also required as a ground truth. This hardware solution should be able to log to a computer so that the measurements can be analyzed.