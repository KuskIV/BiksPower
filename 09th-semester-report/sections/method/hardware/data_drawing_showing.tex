\subsection{Setup}\label[subsec]{subsec:setup}

As was covered in \cref{sec:measuring_instruments}, four different measuring instruments were chosen, based on different criteria. Because these measuring instruments require different operating systems, the setup will be for Windows and Linux. Additionally, E3 has some specific demands, regarding under which circumstances the measuring instruments perform best\cite{E3Doc,E3Video}. The first demand was regarding a DUT with a battery and a DUT with the special MAXIM chip. Because of this, three different DUTs were chosen, to test all these cases:

\begin{itemize}
    \item A DUT with a battery and a MAXIM chip
    \item A DUT with a battery and no MAXIM chip
    \item A DUT with no battery and no maxim chip
\end{itemize}

Furthermore, E3 also claimed to work best when the charger is disabled\cite{E3Doc}, therefore to automate the process, smart plugs are a good solution. The computer should be able to turn these smart plugs on and off through the framework used for running the test cases. To compare the performance of the different software measuring instruments, a hardware solution is also required as a ground truth. This hardware solution should be able to log to a computer so that the measurements can be analyzed.