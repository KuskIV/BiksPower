\chapter{Method}\label{ch:method}

Following the introduction to this work, it is now time to take a further look into how the research questions can be answered. This will be achieved by looking at the different components required in the experiments, in order to find the best fit.

\section{Measuring Software}\label{sec:measuring_software}

When considering research question one, different ways of measuring the energy consumption is needed. In this section different existing approached will be presented, and in the end compared.


                        \begin{figure}
                            \centering
                            \begin{tikzpicture}[]
                                \pgfplotsset{%
                                    width=.6\textwidth,
                                    height=0.4\textheight
                                }
                                \begin{axis}[xlabel={Average dynamic energy (Watts)}, title={SurfaceBook - RAPL}, ytick={1, 2, 3, 4},
                                yticklabels={
                                    BinaryTrees - RAPL, FannkuchRedux - RAPL, Nbody - RAPL, Fasta - RAPL
                                    },
                                    xmin=0,xmax=80,
                                    ]
                                
                                    \addplot+ [boxplot prepared={
                                    lower whisker=4.828474728535248,
                                    lower quartile=4.965529391276659,
                                    median=5.123137462250352,
                                    upper quartile=5.29291940764101,
                                    upper whisker=5.6252351289872085},
                                    ] table[row sep=\\,y index=0] {\\};
                                    
                                    \addplot+ [boxplot prepared={
                                    lower whisker=5.062875916575935,
                                    lower quartile=5.15876774202521,
                                    median=5.1819262057960325,
                                    upper quartile=5.198190098979957,
                                    upper whisker=5.255077811846938},
                                    ] table[row sep=\\,y index=0] {\\};
                                    
                                    \addplot+ [boxplot prepared={
                                    lower whisker=4.780004016964808,
                                    lower quartile=4.8070152384092,
                                    median=4.82418743669118,
                                    upper quartile=4.838536851432698,
                                    upper whisker=4.886202997659311},
                                    ] table[row sep=\\,y index=0] {\\};
                                    
                                    \addplot+ [boxplot prepared={
                                    lower whisker=5.036917950104407,
                                    lower quartile=5.263664537834856,
                                    median=5.398020641615526,
                                    upper quartile=5.549948759622877,
                                    upper whisker=5.807261082297773},
                                    ] table[row sep=\\,y index=0] {\\};
                                    
                                \end{axis}
                            \end{tikzpicture}
                        \caption{R3 validation for dynamic energy measurements by RAPL for the Dram for all DUT's on Unix and test cases where the impact of the first profiler can be seen (with outliers)} \label{fig:SurfaceBook_RAPL_Dram_R3_dynamic_energy_with_outliers_Unix_avg_watts}
                        \end{figure}
                        
\subsection*{Microsoft energy estimation engine}
Microsoft estimation energy estimation engine(E3), is a tool created to monitor battery usage on windows devices, because of this it is only available on windows devices with battery's. E3 monitors the computer at all times to created battery energy usage reports, that can be accessed in the power & battery 
\section{Intel PowerGadget}

The Intel Power Gadget\cite[]{powergadget} is a software tool for measuring power consumption of Intel Core processors from the 2nd to 10th generation, for both Windows and macOS.

What this software tool offers, is real-time estimations of the energy consumption in watts using the energy counters in the processors.

The tool also contains a command line version called Powerlog, in addition to this, the newest versions also includes estimations of energy consumption on multi socket systems and externally callable APIs. These APIs can b used to extract information within sections of code. This is achieved by evaluating the energy MSR on a per-socket basis.

When measuring energy consumption the average power is measured in watts, cumulative energy in joules, cumulative energy in miliWatt-hours, temperature in Celsius and frequency in GHz.

When considering the API, the sampling frequency can range from 1 to 1000 milliseconds. Here it is noted a high frequency will bring a greater accuracy, but poorer performance of the system, and a frequency of 100 milliseconds, which is the default value, is recommended.\cite*[]{powergadget_api}.
\subsection{Open Hardware Monitor / Libre Hardware Monitor}\label[subsec]{subsec:HardwareMonitor}
Open Hardware Monitor is a free open-source piece of software able to monitor metrics like temperature, fan speeds, voltages, loads and clock speeds on a computer for both CPU and GPU. According to the documentation, the software supports most hardware monitoring chips on motherboards and works for both Intel and AMD chips, in addition to both 32bit and 64bit windows, and any x86-based Linux. When running, the software will publish all data to the Windows Management Instrumentation (WMI), which allows other applications to use the data.\cite[]{open_hardware_monitor}

This tool is not mentioned in any articles, and accuracy is not mentioned anywhere in the documentation. Generally, the information in the literature and its documentation pertaining to accuracy are either very sparse on non-existent. A fork of Open Hardware Monitor called Libre Hardware Monitor (LHM) is available on Github\cite{libre_hardware_monitor}. From here a version without a GUI is used to minimize the energy consumption from the measuring instrument itself.
\subsection{Comparison}\label{subsec:software_comparison}

After the introduction of different existing energy profilers, a comparison can be made. Firstly, when considering the four tools it can be seen that most of them provide the measurements of the same components which is ideal for comparing their results later. The most commonly used software-based measuring instrument is Intel's RAPL. However, it is limited to Linux. When considering the Windows tools, none of the tools has any credible sources documenting their accuracies. One interesting tool is E3, as it estimates the power of all processes running on the computer individually, and how a Maxim chip can, according to Microsoft, bring the accuracy up to 98\%.\cite[]{E3WinHec} E3 does however only work on devices with a battery, as opposed to Open Hardware Monitor and Intel Power Gadget. When comparing the latter two, they seem very similar, where one big advantage of Open Hardware Monitor is that it is open source, so any uncertainty could be verified by looking at the actual implementation, and the implementation can be altered to fit the purpose. The advantage of Intel's Power Gadget and Open Hardware Monitor is that they have an implemented interface, which makes it easier to access the relevant data. Opposed to this we have E3, where a log needs to be reset, and a log file needs to be parsed, which is not ideal. We have chosen to use Intel's RAPL, Microsoft's E3, Intel's Power Power Gadget and Open Hardware Monitor. 

% WHHY?

%% AMD version  AMP. Source siger bad. Hold os inden for en producent af CPU.

\section{Hardware}\label{sec:hardware}

When measuring the energy consumption of hardware, different software measuring tools were introduced in \cref{sec:measuring_software}. In this section the hardware these tools will measure the energy consumption on will be introduced. This will include an analysis of the different requirements the tools had, in order to find which combination of hardware will give the best result.

\section{Metrics}\label{sec:metrics}

As this work will aim to compare different ways to measure the energy consumption of software, not only across different measuring approaches, but also different hardware and operating systems, a way to represent the data is required. In this section different ways to make the data presentable and comparable will be presented. To make this easier to read the following sub-questions have been formulated which will help in clarifying the goals of each of the metrics, and the relevance.
\begin{itemize}
    \item How would different implementations of a functionally identical programs from the same languages be compared to see differences in energy consumption?
    \item How are the data from the different systems comparable, can the correlated is the same experiment across the different machines and operating systems?
    \item How are outliers detected and how often do they occurs in the data?.
\end{itemize}
To answer
\subsection{Distribution}
To answer the first question it is important to find out which statistical tests that could be used on the data. One of the most common preconditions for statistical test is that the data should be a normal distribution for the results to work. To test if a distribution is a normal Distribution a normalcy test have to be used. While there are many different normalcy tests it seems that the Shapiro-Wilk test is generally more powerful than the others that are commonly used \cite{razali2011power}.

sub question a standard 
t-test could be used to differentiate the two sample means to see if the difference between them is real or occurrence because of a random outside variables. Using a t-test does however assumes something thing about the data used.
\begin{itemize}
    \item Both samples should follow a normal Distribution.
    \item The samples are independent of each other.
    \item Both the samples should also have approximately the same variance.
\end{itemize}
\subsection{Correlation}
Correlation is a statistical measurement that essentially informs us of the relationship between two samples. The correlation can be either positive or negative and a measure of how correlated they are. A strong correlation does however not mean that changes i one variable effects the other just that it statically seems to be the case.
\paragraph{Pearson correlation coefficient} The Pearson correlation coefficient is one of if not the most used method for finding the correlations in linear relationships. The strength of the correlation is spans between $-1 \dotsc 1$, were the directions is given by the polarity sign. Depending on the context different strength are considered as strong or weak, but commonly 0 means that there is no correlation between the two samples. Some Assumptions have to be true about the data for us to utilize Pearson correlation coefficient:
\begin{itemize}
    \item The data must be quantitative
    \item The two samples must be normally distributed
    \item There must be no outliers in the data
    \item The relationship between the two must also be linear
\end{itemize}
If all of these assumptions are true for the data then the Pearson correlation coefficient can be calculated with the formula:
% $$r=\frac{n\sum{xy- (\sum{x})(\sum{y})}}{\sqrt{[n\sum{x^2}-(\sum{x})^2][n\sum{y^2}-(\sum{y}^2)]}}$$
$$r=\frac{\sum{(x_i-\bar{x})(y_i-\bar{y})}}{\sqrt{\sum{(x_i-\bar{x})^2}\sum{(y_i-\bar{y})^2}}}$$
If however the data does not follow the conditions then another method of calculating the correlation will have to be used. 
\paragraph{Kendall tau correlation}
There are two common choices when doing non-parametric rank correlation, which we can use on non-normally distributed data. These are the Spearman's rank correlation coefficient and Kendall tau's correlation coefficient. While both can be used in the same scenarios there are some differences betweens them.Both are ranked with the same procedure that Mann–Whitney U tests are. The Spearman rank seems to be the most popular way measuring correlations, but some argue that Kendall tau is better in most cases \cite{gilpin1993table}. Some of the big differences between the two the effect of outliers Spearman's rank is largely effected by a few outliers in the data while Kendall tau in comparison seems more robust. Because of this the choice to utilize Kendall tau seems the most natural, and is what will be continued with. The results from Kendall tau can be interpreted the same way as from Pearson, the range is from $-1$ to $1$. The formula for kendall tau is also pretty straight forward:
$$\frac{C-D}{C+D}$$
C is Concordant pairs and D is Discordant. Essentially a Concordant pairs are every time the rankings are in order while Discordant is disordered pairs\cite{kendall1938new}.


\subsection{Anomalies}
Anomaly detection is important when using the data as i will influence the results from the experiments. When an anomaly is detected it will be removed from teh data as to not corrupt the rest of teh results.
\paragraph{DBScan}
To find and remove these anomalies the DBScan algorithm will be used. The DBScan algorithm is essentials a clustering algorithm that is used to cluster data into groups, but is can also be used for density based abnormality detection.
\begin{lstlisting}
    
\end{lstlisting}


\section{Experimental/Test Setup}\label[subsection]{ExperimentalSetup}
After the different components required to execute the experiment have been introduced, it is now time to take a look at the experiments. This will be regarding the languages used, how the data is saved, how many times the experiments are run etc.
\todo{Hvad vi vælger fra sestoft, Feetings osv og hvorfor vi vælger det}

%% save in DB in SQL
%% Script written in C#
%% Cochrans 

