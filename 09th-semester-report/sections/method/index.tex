\chapter{Method}\label{ch:method}

Following the introduction to this work, it is now time to take a further look into how the research questions can be answered. This will be achieved by looking at the different components required in the experiments, in order to find the best fit.

\section{Measuring Software}\label{sec:measuring_software}

When considering research question one, different ways of measuring the energy consumption is needed. In this section different existing approached will be presented, and in the end compared.


                            \begin{figure}
                                \centering
                                \begin{tikzpicture}[]
                                    \pgfplotsset{%
                                        width=.85\textwidth,
                                        height=0.15\textheight
                                    }
                                    \begin{axis}[xlabel={Average energy (Watts)}, title={workstation - RAPL}, ytick={1, 2, 3, 4, 5, 6},
                                    yticklabels={
                                        Fasta - IPG, Fasta - LHM, Fasta - CLAMP, Fasta - IPG, Fasta - LHM, Fasta - CLAMP
                                        },
                                        xmin=0,xmax=50,
                                        ]
                                    
                                        \addplot+ [boxplot prepared={
                                        lower whisker=40.30526797215659,
                                        lower quartile=40.97327163025656,
                                        median=41.72166726874962,
                                        upper quartile=42.639938738927185,
                                        upper whisker=44.425335327916834},
                                        ] table[row sep=\\,y index=0] {\\};
                                        
                                        \addplot+ [boxplot prepared={
                                        lower whisker=40.48550261459181,
                                        lower quartile=40.831039235552794,
                                        median=41.5846158762684,
                                        upper quartile=42.207863208562216,
                                        upper whisker=43.46991921513228},
                                        ] table[row sep=\\,y index=0] {\\};
                                        
                                        \addplot+ [boxplot prepared={
                                        lower whisker=40.2697084750618,
                                        lower quartile=41.035098465232245,
                                        median=41.38869099577169,
                                        upper quartile=42.44694669562162,
                                        upper whisker=43.91629239371539},
                                        ] table[row sep=\\,y index=0] {\\};
                                        
                                        \addplot+ [boxplot prepared={
                                        lower whisker=40.758440129667676,
                                        lower quartile=41.39236498066475,
                                        median=41.773795862544134,
                                        upper quartile=42.42892778500295,
                                        upper whisker=44.19480610081178},
                                        ] table[row sep=\\,y index=0] {\\};
                                        
                                        \addplot+ [boxplot prepared={
                                        lower whisker=40.81938120830589,
                                        lower quartile=41.27530056939251,
                                        median=41.61423462219319,
                                        upper quartile=42.13919903223857,
                                        upper whisker=43.40666243037317},
                                        ] table[row sep=\\,y index=0] {\\};
                                        
                                        \addplot+ [boxplot prepared={
                                        lower whisker=40.864166136835486,
                                        lower quartile=41.51662118233571,
                                        median=41.88433916576963,
                                        upper quartile=42.86202552311207,
                                        upper whisker=45.290678405053065},
                                        ] table[row sep=\\,y index=0] {\\};
                                        
                                    \end{axis}
                                \end{tikzpicture}
                            \caption{R3 validation for energy measurements by RAPL for the Cores for DUT PowerKomplett, OS Win32NT and test case Fasta, where the impact of the first profiler can be seen (with outliers)} \label{fig:PowerKomplett_RAPL_Cores_R3_energy_with_outliers_Win32NT_avg_watts}
                            \end{figure}
                            
\subsection{Microsoft's Energy Estimation Engine (E3)}\label[subsec]{subsec:e3}
Microsoft's Energy Estimation Engine (E3) is a tool created to monitor the energy consumption of batteries on Windows devices. E3 monitors the DUT at all times to create battery energy usage reports, accessible in the power battery settings under Battery usage.\cite[p.43]{E3WinHec} The reported accuracy of E3 varies if the DUT has a special energy measurement chip or not. The reported accuracies for different compoents of the DUT is presented in \cite{E3WinHec} can be seen in \cref{tab:E3_acc_Table}.

\begin{table}[ht]
    \centering
    \begin{tabular}{|c|c|c|c|c|}
    \hline
    \textit{}    & CPU  & Storage         & Display         & Network         \\ \hline
    Without chip & 89\% & \textless{}70\% & \textless{}70\% & \textless{}70\% \\ \hline
    With chip    & 98\% & 98\%            & 98\%            & 98\%            \\ \hline
    \end{tabular}
    \caption{E3 accuracies. For devices with and without Maxim chips}
    \label{tab:E3_acc_Table}
\end{table}

In \cref{tab:E3_acc_Table} it can be observed that there is a large difference in the accuracy between devices with and without the chip on the motherboard. The chips referred to here are specifically the MAXIM MAX34407 or the MAXIM MAX34417, contained for example in different Microsoft Surface devices.\newline

% , finding the models of computers that actually contain these chips have proven difficult, with very few computer known to actually contain the chip. 

When comparing E3 to other measuring instruments like RAPL, the primary difference is how E3 is able to estimate the energy consumption for each proces running on the DUT, opposed to RAPL where only the energy consumption of the CPU and RAM are reported. This makes it easier to draw certain conclusions about energy usage from E3 if the estimations are accurate.\newline

The reports created by E3 will by default contain logs of power usage for the previous $7$ days, where each process and their power usage is logged. Each log entry accounts for $1-5$ minutes depending on the actual length of the process execution\cite[]{E3Video}. Each entry in the report contains many attributes, where some of the relevant attributes to this study can be seen in \cref{tab:E3_attr_Table}

\begin{table}[ht]
    \centering
    \begin{tabular}{||c|c||}
    \hline
    \textbf{Attribute Name}  & \textbf{Description}                            \\ [0.5ex] \hline\hline
    AppId                    & Unique id for each process and subprocess       \\ \hline
    TimeStamp                & Start time of Measurement                       \\ \hline
    TimeInMs                 & Duration of the Measurement in Ms               \\ \hline
    CPUEnergyConsumption     & Cpu energy consumption in millijoules           \\ \hline
    NetworkEnergyConsumption & Network energy consumption in millijoules       \\ \hline
    SocEnergyConsumption     & System-On-Chip energy consumption in milijoules \\ \hline
    DiskEnergyConsumption    & Disk energy consumption in millijoules          \\ \hline
    TotalEnergyConsumption   & Total energy consumption in millijoules         \\ \hline
    \end{tabular}
    \caption{The attributes in the E3 report used in this report}
    \label{tab:E3_attr_Table}
    \end{table}

As of the time of writing no study or external entity apart from Microsoft themselves has neither used nor verified the accuracy and reliability of E3 as reported by Microsoft. Because of this, the information about E3 mostly comes from blog posts or presentations by Microsoft and is very sparse in the information provided. An additional limitation is how this software only works on Windows devices with a battery, meaning that E3 is not accessible on Windows desktops.\newline

In order to generate the raw energy report from E3, navigate to the desired destination of the report in an elevated command shell and type \texttt{powercfg.exe \\srumutil}. Upon success, the output should be \texttt{Completed with status 0 (0x00000000)}.

% start normal listing
% \begin{lstlisting}[language=Python, caption=Python example]

% From file
% \lstinputlisting[language=Octave]{BitXorMatrix.m}
\section{Intel PowerGadget}

The Intel Power Gadget\cite[]{powergadget} is a software tool for measuring power consumption of Intel Core processors from the 2nd to 10th generation, for both Windows and macOS. What this software tool offers, is real-time estimations of the energy consumption in watts using the energy counters in the processors. 
The tool also contains a command line version called Powerlog, in addition to this, the newest versions also includes estimations of energy consumption on multi socket systems and externally callable APIs. These APIs can b used to extract information within sections of code. This is achieved by evaluating the energy Model Specific Registers (MSR) on a per-socket basis.

When measuring energy consumption the average power is measured in watts, cumulative energy in joules, cumulative energy in miliWatt-hours, temperature in Celsius and frequency in GHz. When considering the API, the sampling frequency can range from 1 to 1000 milliseconds. Here it is noted a high frequency will bring a greater accuracy, but poorer performance of the system, and a frequency of 100 milliseconds, which is the default value, is recommended.\cite*[]{powergadget_api}.

When using the Power Gadget API a reference to \texttt{EnergyLib64.dll} will make functions available. When using the library, it must first be initialized, by calling 

\texttt{IntelEnergyLibInitialize()}, where the drivers are loaded. In addition to this, the library also contains contains other method. Similar for all these methods is how they return a boolean, representing if the call was a success. In this study, this tool will only be used for power measuring, despite also having additional features like temperatures or frequencies of the processor. All functions can be found in the official documentation\cite*[]{powergadget_api}, where the relevant functions can be found in \cref{tab:intel_power_gadget_functions}

\begin{table}
    \centering
    \begin{tabular}{ | c | c |}
        \hline
        \thead{Function} & Description \\
        \hline
        \texttt{bool IntelEnergyLibInitialize();} &  \makecell{The library is initialized and a connection is\\made to the driver}  \\
        \hline
        \texttt{bool ReadSample();} &  \makecell{Sample data is read from all supported MSRs\\through the driver}  \\
        \hline
        \makecell{\texttt{bool GetPowerData(}\\ \texttt{int iNode, int iMSR,}\\ \texttt{double *pResult, int *nResult);}} &  \makecell{The data collected by the most recent\\\texttt{ReadSample()} is returned. The data\\returned is only for the package iNode\\specified, from the iMSR MSR specified.\\pResult represents the data returned,\\and nResults represents the number of\\double results in pResult.}  \\
        \hline
        \texttt{bool StartLog(wchart szFileName);} & \makecell{Data is collected and written to the\\file specified by calling \texttt{ReadSample()}\\until \texttt{StopLog()} }  \\
        \hline
        \texttt{bool StopLog();} &  \makecell{All the saved data is written to\\ the file specified in the \texttt{StartLog()}\\ call, and stops saving data.
        }  \\
        \hline
    \end{tabular}
    \caption{Function prototypes of relevant functions for Intel Power Gadget.}
    \label{tab:intel_power_gadget_functions}
\end{table}

% Returns the data collected by the most recent call to ReadSample(). The returned data is for the data on the package specified by iNode, from the MSR specified by iMSR. The data is returned in pResult, and the number of double results returned in pResult is returned in nResult. Refer Table 1: MSR Functions.

% The data collected by the most recent ReadSample() is returned. The data returned is only for the package iNode specified, from the iMSR MSR specified. pResult represents the data returned, and nResults represents the number of double results in pResult.

% Starts saving the data collected by ReadSample() until StopLog() is called. Data will be written to the file specified by szFileName.

% Data is collected and written to the file specified by calling \texttt{ReadSample()} until \texttt{StopLog()} 

% Stops saving data and writes all saved data to the file specified by the call to StartLog().

% All the saved data is written to the file specified in the \texttt{StartLog()} call, and stops saving data.
\section{Hardware Monitor}

\section{Hardware}\label{sec:hardware}

When measuring the energy consumption of hardware, different software measuring tools were introduced in \cref{sec:measuring_software}. In this section the hardware these tools will measure the energy consumption on will be introduced. This will include an analysis of the different requirements the tools had, in order to find which combination of hardware will give the best result.

\section{Metrics}\label{sec:metrics}

As this work will aim to compare different ways to measure the energy consumption of software, not only across different measuring approaches, but also different hardware and operating systems, a way to represent the data is required. In this section different ways to make the data presentable and comparable will be presented.

\section{Experimental Approach}\label{sec:experimental_approach}

After the different components required in order to execute the experiment has been introduced, it is not time to take a deeper look into the experiment. This will be in regards to the languages used, how the data is saved, how many times the experiments are run ect.

%% save in DB in SQL
%% Script written in C#
%% Cochrans 

