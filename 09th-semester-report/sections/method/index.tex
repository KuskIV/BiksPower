\chapter{Method}\label{ch:method}

Following the introduction to this work, it is now time to take a further look into how the research questions can be answered. This will be achieved by looking at the different components required in the experiments, in order to find the best fit.

\section{Measuring Software}\label{sec:measuring_software}

When considering research question one, different ways of measuring the energy consumption is needed. In this section different existing approached will be presented, and in the end compared.


                        \begin{figure}
                            \centering
                            \begin{tikzpicture}[]
                                \pgfplotsset{%
                                    width=.6\textwidth,
                                    height=0.4\textheight
                                }
                                \begin{axis}[xlabel={Average dynamic energy (Watts)}, title={SurfaceBook - RAPL}, ytick={1, 2, 3, 4},
                                yticklabels={
                                    BinaryTrees - RAPL, FannkuchRedux - RAPL, Nbody - RAPL, Fasta - RAPL
                                    },
                                    xmin=0,xmax=80,
                                    ]
                                
                                    \addplot+ [boxplot prepared={
                                    lower whisker=4.828474728535248,
                                    lower quartile=4.965529391276659,
                                    median=5.123137462250352,
                                    upper quartile=5.29291940764101,
                                    upper whisker=5.6252351289872085},
                                    ] table[row sep=\\,y index=0] {\\};
                                    
                                    \addplot+ [boxplot prepared={
                                    lower whisker=5.062875916575935,
                                    lower quartile=5.15876774202521,
                                    median=5.1819262057960325,
                                    upper quartile=5.198190098979957,
                                    upper whisker=5.255077811846938},
                                    ] table[row sep=\\,y index=0] {\\};
                                    
                                    \addplot+ [boxplot prepared={
                                    lower whisker=4.780004016964808,
                                    lower quartile=4.8070152384092,
                                    median=4.82418743669118,
                                    upper quartile=4.838536851432698,
                                    upper whisker=4.886202997659311},
                                    ] table[row sep=\\,y index=0] {\\};
                                    
                                    \addplot+ [boxplot prepared={
                                    lower whisker=5.036917950104407,
                                    lower quartile=5.263664537834856,
                                    median=5.398020641615526,
                                    upper quartile=5.549948759622877,
                                    upper whisker=5.807261082297773},
                                    ] table[row sep=\\,y index=0] {\\};
                                    
                                \end{axis}
                            \end{tikzpicture}
                        \caption{R3 validation for dynamic energy measurements by RAPL for the Dram for all DUT's on Unix and test cases where the impact of the first profiler can be seen (with outliers)} \label{fig:SurfaceBook_RAPL_Dram_R3_dynamic_energy_with_outliers_Unix_avg_watts}
                        \end{figure}
                        
\subsection*{Microsoft energy estimation engine}
Microsoft estimation energy estimation engine(E3), is a tool created to monitor battery usage on windows devices, because of this it is only available on windows devices with battery's. E3 monitors the computer at all times to created battery energy usage reports, that can be accessed in the power & battery 
\section{Intel PowerGadget}

The Intel Power Gadget\cite[]{powergadget} is a software tool for measuring power consumption of Intel Core processors from the 2nd to 10th generation, for both Windows and macOS.

What this software tool offers, is real-time estimations of the energy consumption in watts using the energy counters in the processors.

The tool also contains a command line version called Powerlog, in addition to this, the newest versions also includes estimations of energy consumption on multi socket systems and externally callable APIs. These APIs can b used to extract information within sections of code. This is achieved by evaluating the energy MSR on a per-socket basis.

When measuring energy consumption the average power is measured in watts, cumulative energy in joules, cumulative energy in miliWatt-hours, temperature in Celsius and frequency in GHz.

When considering the API, the sampling frequency can range from 1 to 1000 milliseconds. Here it is noted a high frequency will bring a greater accuracy, but poorer performance of the system, and a frequency of 100 milliseconds, which is the default value, is recommended.\cite*[]{powergadget_api}.
\subsection{Open Hardware Monitor / Libre Hardware Monitor}\label[subsec]{subsec:HardwareMonitor}
Open Hardware Monitor is a free open-source piece of software able to monitor metrics like temperature, fan speeds, voltages, loads and clock speeds on a computer for both CPU and GPU. According to the documentation, the software supports most hardware monitoring chips on motherboards and works for both Intel and AMD chips, in addition to both 32bit and 64bit windows, and any x86-based Linux. When running, the software will publish all data to the Windows Management Instrumentation (WMI), which allows other applications to use the data.\cite[]{open_hardware_monitor}

This tool is not mentioned in any articles, and accuracy is not mentioned anywhere in the documentation. Generally, the information in the literature and its documentation pertaining to accuracy are either very sparse on non-existent. A fork of Open Hardware Monitor called Libre Hardware Monitor (LHM) is available on Github\cite{libre_hardware_monitor}. From here a version without a GUI is used to minimize the energy consumption from the measuring instrument itself.

\section{Hardware}\label{sec:hardware}

When measuring the energy consumption of hardware, different software measuring tools were introduced in \cref{sec:measuring_software}. In this section the hardware these tools will measure the energy consumption on will be introduced. This will include an analysis of the different requirements the tools had, in order to find which combination of hardware will give the best result.

\section{Metrics}\label{sec:metrics}

As this work will aim to compare different ways to measure the energy consumption of software, not only across different measuring approaches, but also different hardware and operating systems, a way to represent the data is required. In this section different ways to make the data presentable and comparable will be presented.

\section{Experimental Approach}\label{sec:experimental_approach}

After the different components required in order to execute the experiment has been introduced, it is not time to take a deeper look into the experiment. This will be in regards to the languages used, how the data is saved, how many times the experiments are run ect.

%% save in DB in SQL
%% Script written in C#
%% Cochrans 

