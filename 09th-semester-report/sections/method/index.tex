\chapter{Method}\label{ch:method}

Following the introduction to this work, it is now time to take a further look into how the research questions can be answered. This will be achieved by looking at the different components required in the experiments, in order to find the best fit. This will first of all be energy profilers in \cref*{sec:energy_profilers}, hardware in \cref*{sec:hardware}, how the data is saved in \cref*{sec:save_data}  and statistical methods in \cref*{sec:stat}

\section{Energy Profilers}\label{sec:energy_profilers}

When considering \textbf{RQ2}, different energy profilers are needed. In this section different existing approached will be presented, and in the end compared.

When selecting the profilers, promising ones were chosen in different categories, including software manufactures, hardware manufacture and at last open source. When considering hardware manufactures, Intel's RAPL and Power Gadget were chosen as their solutions with potential for hardware support could be promising. For software manufactures, Microsoft's E3 was chosen as their domain knowledge could prove valuable for a good energy estimation. Lastly, an open source project is included, this being Open Hardware Monitor, to see how this compared to three, potentially more sophisticated solutions. All the different profilers will be introduced now.


                            \begin{figure}
                                \centering
                                \begin{tikzpicture}[]
                                    \pgfplotsset{%
                                        width=.85\textwidth,
                                        height=0.15\textheight
                                    }
                                    \begin{axis}[xlabel={Average energy (Watts)}, title={workstation - RAPL}, ytick={1, 2, 3, 4, 5, 6},
                                    yticklabels={
                                        Fasta - IPG, Fasta - LHM, Fasta - CLAMP, Fasta - IPG, Fasta - LHM, Fasta - CLAMP
                                        },
                                        xmin=0,xmax=50,
                                        ]
                                    
                                        \addplot+ [boxplot prepared={
                                        lower whisker=40.30526797215659,
                                        lower quartile=40.97327163025656,
                                        median=41.72166726874962,
                                        upper quartile=42.639938738927185,
                                        upper whisker=44.425335327916834},
                                        ] table[row sep=\\,y index=0] {\\};
                                        
                                        \addplot+ [boxplot prepared={
                                        lower whisker=40.48550261459181,
                                        lower quartile=40.831039235552794,
                                        median=41.5846158762684,
                                        upper quartile=42.207863208562216,
                                        upper whisker=43.46991921513228},
                                        ] table[row sep=\\,y index=0] {\\};
                                        
                                        \addplot+ [boxplot prepared={
                                        lower whisker=40.2697084750618,
                                        lower quartile=41.035098465232245,
                                        median=41.38869099577169,
                                        upper quartile=42.44694669562162,
                                        upper whisker=43.91629239371539},
                                        ] table[row sep=\\,y index=0] {\\};
                                        
                                        \addplot+ [boxplot prepared={
                                        lower whisker=40.758440129667676,
                                        lower quartile=41.39236498066475,
                                        median=41.773795862544134,
                                        upper quartile=42.42892778500295,
                                        upper whisker=44.19480610081178},
                                        ] table[row sep=\\,y index=0] {\\};
                                        
                                        \addplot+ [boxplot prepared={
                                        lower whisker=40.81938120830589,
                                        lower quartile=41.27530056939251,
                                        median=41.61423462219319,
                                        upper quartile=42.13919903223857,
                                        upper whisker=43.40666243037317},
                                        ] table[row sep=\\,y index=0] {\\};
                                        
                                        \addplot+ [boxplot prepared={
                                        lower whisker=40.864166136835486,
                                        lower quartile=41.51662118233571,
                                        median=41.88433916576963,
                                        upper quartile=42.86202552311207,
                                        upper whisker=45.290678405053065},
                                        ] table[row sep=\\,y index=0] {\\};
                                        
                                    \end{axis}
                                \end{tikzpicture}
                            \caption{R3 validation for energy measurements by RAPL for the Cores for DUT PowerKomplett, OS Win32NT and test case Fasta, where the impact of the first profiler can be seen (with outliers)} \label{fig:PowerKomplett_RAPL_Cores_R3_energy_with_outliers_Win32NT_avg_watts}
                            \end{figure}
                            
\subsection{Microsoft's Energy Estimation Engine (E3)}\label[subsec]{subsec:e3}
Microsoft's Energy Estimation Engine (E3) is a tool created to monitor the energy consumption of batteries on Windows devices. E3 monitors the DUT at all times to create battery energy usage reports, accessible in the power battery settings under Battery usage.\cite[p.43]{E3WinHec} The reported accuracy of E3 varies if the DUT has a special energy measurement chip or not. The reported accuracies for different compoents of the DUT is presented in \cite{E3WinHec} can be seen in \cref{tab:E3_acc_Table}.

\begin{table}[ht]
    \centering
    \begin{tabular}{|c|c|c|c|c|}
    \hline
    \textit{}    & CPU  & Storage         & Display         & Network         \\ \hline
    Without chip & 89\% & \textless{}70\% & \textless{}70\% & \textless{}70\% \\ \hline
    With chip    & 98\% & 98\%            & 98\%            & 98\%            \\ \hline
    \end{tabular}
    \caption{E3 accuracies. For devices with and without Maxim chips}
    \label{tab:E3_acc_Table}
\end{table}

In \cref{tab:E3_acc_Table} it can be observed that there is a large difference in the accuracy between devices with and without the chip on the motherboard. The chips referred to here are specifically the MAXIM MAX34407 or the MAXIM MAX34417, contained for example in different Microsoft Surface devices.\newline

% , finding the models of computers that actually contain these chips have proven difficult, with very few computer known to actually contain the chip. 

When comparing E3 to other measuring instruments like RAPL, the primary difference is how E3 is able to estimate the energy consumption for each proces running on the DUT, opposed to RAPL where only the energy consumption of the CPU and RAM are reported. This makes it easier to draw certain conclusions about energy usage from E3 if the estimations are accurate.\newline

The reports created by E3 will by default contain logs of power usage for the previous $7$ days, where each process and their power usage is logged. Each log entry accounts for $1-5$ minutes depending on the actual length of the process execution\cite[]{E3Video}. Each entry in the report contains many attributes, where some of the relevant attributes to this study can be seen in \cref{tab:E3_attr_Table}

\begin{table}[ht]
    \centering
    \begin{tabular}{||c|c||}
    \hline
    \textbf{Attribute Name}  & \textbf{Description}                            \\ [0.5ex] \hline\hline
    AppId                    & Unique id for each process and subprocess       \\ \hline
    TimeStamp                & Start time of Measurement                       \\ \hline
    TimeInMs                 & Duration of the Measurement in Ms               \\ \hline
    CPUEnergyConsumption     & Cpu energy consumption in millijoules           \\ \hline
    NetworkEnergyConsumption & Network energy consumption in millijoules       \\ \hline
    SocEnergyConsumption     & System-On-Chip energy consumption in milijoules \\ \hline
    DiskEnergyConsumption    & Disk energy consumption in millijoules          \\ \hline
    TotalEnergyConsumption   & Total energy consumption in millijoules         \\ \hline
    \end{tabular}
    \caption{The attributes in the E3 report used in this report}
    \label{tab:E3_attr_Table}
    \end{table}

As of the time of writing no study or external entity apart from Microsoft themselves has neither used nor verified the accuracy and reliability of E3 as reported by Microsoft. Because of this, the information about E3 mostly comes from blog posts or presentations by Microsoft and is very sparse in the information provided. An additional limitation is how this software only works on Windows devices with a battery, meaning that E3 is not accessible on Windows desktops.\newline

In order to generate the raw energy report from E3, navigate to the desired destination of the report in an elevated command shell and type \texttt{powercfg.exe \\srumutil}. Upon success, the output should be \texttt{Completed with status 0 (0x00000000)}.

% start normal listing
% \begin{lstlisting}[language=Python, caption=Python example]

% From file
% \lstinputlisting[language=Octave]{BitXorMatrix.m}
\section{Intel PowerGadget}

The Intel Power Gadget\cite[]{powergadget} is a software tool for measuring power consumption of Intel Core processors from the 2nd to 10th generation, for both Windows and macOS. What this software tool offers, is real-time estimations of the energy consumption in watts using the energy counters in the processors. 
The tool also contains a command line version called Powerlog, in addition to this, the newest versions also includes estimations of energy consumption on multi socket systems and externally callable APIs. These APIs can b used to extract information within sections of code. This is achieved by evaluating the energy Model Specific Registers (MSR) on a per-socket basis.

When measuring energy consumption the average power is measured in watts, cumulative energy in joules, cumulative energy in miliWatt-hours, temperature in Celsius and frequency in GHz. When considering the API, the sampling frequency can range from 1 to 1000 milliseconds. Here it is noted a high frequency will bring a greater accuracy, but poorer performance of the system, and a frequency of 100 milliseconds, which is the default value, is recommended.\cite*[]{powergadget_api}.

When using the Power Gadget API a reference to \texttt{EnergyLib64.dll} will make functions available. When using the library, it must first be initialized, by calling 

\texttt{IntelEnergyLibInitialize()}, where the drivers are loaded. In addition to this, the library also contains contains other method. Similar for all these methods is how they return a boolean, representing if the call was a success. In this study, this tool will only be used for power measuring, despite also having additional features like temperatures or frequencies of the processor. All functions can be found in the official documentation\cite*[]{powergadget_api}, where the relevant functions can be found in \cref{tab:intel_power_gadget_functions}

\begin{table}
    \centering
    \begin{tabular}{ | c | c |}
        \hline
        \thead{Function} & Description \\
        \hline
        \texttt{bool IntelEnergyLibInitialize();} &  \makecell{The library is initialized and a connection is\\made to the driver}  \\
        \hline
        \texttt{bool ReadSample();} &  \makecell{Sample data is read from all supported MSRs\\through the driver}  \\
        \hline
        \makecell{\texttt{bool GetPowerData(}\\ \texttt{int iNode, int iMSR,}\\ \texttt{double *pResult, int *nResult);}} &  \makecell{The data collected by the most recent\\\texttt{ReadSample()} is returned. The data\\returned is only for the package iNode\\specified, from the iMSR MSR specified.\\pResult represents the data returned,\\and nResults represents the number of\\double results in pResult.}  \\
        \hline
        \texttt{bool StartLog(wchart szFileName);} & \makecell{Data is collected and written to the\\file specified by calling \texttt{ReadSample()}\\until \texttt{StopLog()} }  \\
        \hline
        \texttt{bool StopLog();} &  \makecell{All the saved data is written to\\ the file specified in the \texttt{StartLog()}\\ call, and stops saving data.
        }  \\
        \hline
    \end{tabular}
    \caption{Function prototypes of relevant functions for Intel Power Gadget.}
    \label{tab:intel_power_gadget_functions}
\end{table}

% Returns the data collected by the most recent call to ReadSample(). The returned data is for the data on the package specified by iNode, from the MSR specified by iMSR. The data is returned in pResult, and the number of double results returned in pResult is returned in nResult. Refer Table 1: MSR Functions.

% The data collected by the most recent ReadSample() is returned. The data returned is only for the package iNode specified, from the iMSR MSR specified. pResult represents the data returned, and nResults represents the number of double results in pResult.

% Starts saving the data collected by ReadSample() until StopLog() is called. Data will be written to the file specified by szFileName.

% Data is collected and written to the file specified by calling \texttt{ReadSample()} until \texttt{StopLog()} 

% Stops saving data and writes all saved data to the file specified by the call to StartLog().

% All the saved data is written to the file specified in the \texttt{StartLog()} call, and stops saving data.
\section{Hardware Monitor}
\section{Comparison}\label{sec:comparison}

In this section of the results, a comparison between the different DUTs and measuring instruments will be performed. This will be done in a way where the different measuring instruments will be compared for each DUT, before comparing all DUTs and all measuring instruments. Before this, the expectations will be presented.


\subsection{Expectations:} Based on what was seen in \cref{sec:iterations}, similar observations are expected. This will include a clamp with a high standard deviation compared to the different software measuring instruments, and cases where Intel Power Gadget and LHM measurements a close to each other. For RAPL, a low standard deviation is expected, in addition to lower measured energy consumption in most cases compared to the other measuring instruments.


\subsection{Results}
\paragraph{Workstation}


                            \begin{figure}
                                \centering
                                \begin{tikzpicture}[]
                                    \pgfplotsset{%
                                        width=.7\textwidth,
                                        height=.2\textheight
                                    }
                                    \begin{axis}[xlabel={Average energy consumption (Watts)}, title={Cores - Fasta - Energy - without outliers}, ytick={1, 2},
                                    yticklabels={
                                        IntelPowerGadget , HardwareMonitor 
                                        },
                                        xmin=0,xmax=80,
                                        ]
                                    
                                    \addplot+ [boxplot prepared={
                                    lower whisker=54.22445449466433,
                                    lower quartile=54.45751347572185,
                                    median=54.54624547297937,
                                    upper quartile=54.72020125409266,
                                    upper whisker=55.103791721157386},
                                    ] table[row sep=\\,y index=0] {\\};
                                    
                                    \addplot+ [boxplot prepared={
                                    lower whisker=51.45082608392138,
                                    lower quartile=51.933860038023106,
                                    median=52.121433545941585,
                                    upper quartile=52.479201309170854,
                                    upper whisker=54.95103920614709},
                                    ] table[row sep=\\,y index=0] {\\};
                                    
                                    \end{axis}
                                \end{tikzpicture}
                            \caption{A comparison of of Cores energy consumption for test case Fasta for the workstation (without outliers)} \label{fig:Fasta_Cores_comparison_energy_without_outliers_PowerKomplett_avg_watts_exp2}
                            \end{figure}
                            

                \begin{figure}[H]
                    \centering
                    \begin{tikzpicture}
                        \pgfplotsset{%
                            width=1\textwidth,
                            height=0.4\textheight
                        }
                        \begin{axis}[
                            xlabel={Start battery level},
                            ylabel={Average dynamic energy (watt)},
                            ymin=0,ymax=20,
                        ]
                        
                            \addplot [mark=none, ultra thick, red]  coordinates {
                            (40, 0.006595684402444291)(45, 0.007295220674193181)(50, 0.007999659608910881)(55, 0.007202467971243639)(60, 0.007280865079495958)(65, 0.006858423509955077)(70, 0.008369444141141925)(75, 0.007144940647010992)(80, 0.004753236834232667)
                            };
                            \addlegendentry{Surface4Pro - IntelPowerGadget}
                            
                            \addplot [mark=none, ultra thick, blue]  coordinates {
                            (40, 0.002385328427460912)(45, 0.0017856511573015649)(50, 0.0025901992189954056)(55, 0.00210998144366709)(60, 0.00286452646862575)(65, 0.0020038487280194628)(70, 0.002908483770774353)(75, 0.0005098111150931342)(80, -0.005995916826693459)
                            };
                            \addlegendentry{Surface4Pro - HardwareMonitor}
                            
                            \addplot [mark=none, ultra thick, orange]  coordinates {
                            (50, 251.8661014299577)(55, 214.7597259251014)(60, 162.23880995564176)(65, 107.43421593849766)(70, 53.9187387386668)(75, 0.7554852536149699)(80, -44.463085003568885)
                            };
                            \addlegendentry{Surface4Pro - RAPL}
                            
                            \addplot [mark=none, dashdotted, red]  coordinates {
                            (40, -0.004038062354887025)(45, -0.004312668500277529)(50, -0.003808663911021498)(55, -0.0037057407527755254)(60, -0.004478257932982471)(65, -0.0026308734995501644)(70, -0.0034090674446925874)(75, -0.003041497079697436)(80, -0.0020334307266356875)
                            };
                            \addlegendentry{SurfaceBook - IntelPowerGadget}
                            
                            \addplot [mark=none, dashdotted, blue]  coordinates {
                            (40, -0.002443518930616523)(45, -0.0029256880137447055)(50, -0.002551777773312929)(55, -0.0027567211782433486)(60, -0.002206859154402231)(65, -0.0026388300848279207)(70, -0.0024597736945479324)(75, -0.002445350799195827)(80, -0.000541271265011134)
                            };
                            \addlegendentry{SurfaceBook - HardwareMonitor}
                            
                            \addplot [mark=none, dashdotted, orange]  coordinates {
                            (40, 101.92702155010147)(45, 86.7212700795567)(50, 69.40754598562454)(55, 51.343785407669614)(60, 32.443112755251185)(65, 14.52577919077786)(70, -4.520517378551423)(75, -23.811706977384954)(80, -35.700372165212755)
                            };
                            \addlegendentry{SurfaceBook - RAPL}
                            
                        \end{axis}
                    \end{tikzpicture} 
                \caption{A graph illustrating the energy consumption of Dram for test case Nbody with regards to the battey level of the DUT (with outliers)} \label{fig:Nbody_Dram_charge}
                \end{figure}
                

The first DUT to consider is the workstation. For this DUT, Fasta and NBody measurements can be seen in \cref{fig:Fasta_Cores_comparison_dynamic_energy_without_outliers_PowerKomplett_avg_watts} and \cref{fig:Nbody_Cores_comparison_dynamic_energy_without_outliers_PowerKomplett_avg_watts} respectively, and the rest can be found in \cref{app:comparison_workstation}. When comparing the software measuring instruments, Intel Power Gadget and LHM measure close to each other in most cases. This includes BinaryTrees, Fasta and NBody where for FannkuchRedux a bigger difference can be observed. When considering RAPL compare to this, RAPL will in all test cases except NBody measure an energy consumption lower than its windows equivalents. When comparing the software measuring instruments against the hardware measurements, RAPL will against the clamp on Linux in all cases except NBody get a measurement higher than the clamp. For Windows, both Intel Power Gadget and LHM will get a measurement higher than the clamp.

\paragraph{Surface Pro 4}


                            \begin{figure}
                                \centering
                                \begin{tikzpicture}[]
                                    \pgfplotsset{%
                                        width=.7\textwidth,
                                        height=.2\textheight
                                    }
                                    \begin{axis}[xlabel={Average energy consumption (Watts)}, title={Cores - Fasta - Energy - without outliers}, ytick={1, 2},
                                    yticklabels={
                                        IntelPowerGadget , HardwareMonitor 
                                        },
                                        xmin=0,xmax=80,
                                        ]
                                    
                                    \addplot+ [boxplot prepared={
                                    lower whisker=54.22445449466433,
                                    lower quartile=54.45751347572185,
                                    median=54.54624547297937,
                                    upper quartile=54.72020125409266,
                                    upper whisker=55.103791721157386},
                                    ] table[row sep=\\,y index=0] {\\};
                                    
                                    \addplot+ [boxplot prepared={
                                    lower whisker=51.45082608392138,
                                    lower quartile=51.933860038023106,
                                    median=52.121433545941585,
                                    upper quartile=52.479201309170854,
                                    upper whisker=54.95103920614709},
                                    ] table[row sep=\\,y index=0] {\\};
                                    
                                    \end{axis}
                                \end{tikzpicture}
                            \caption{A comparison of of Cores energy consumption for test case Fasta for the workstation (without outliers)} \label{fig:Fasta_Cores_comparison_energy_without_outliers_PowerKomplett_avg_watts_exp2}
                            \end{figure}
                            

                            \begin{figure}
                                \centering
                                \begin{tikzpicture}[]
                                    \pgfplotsset{%
                                        width=.85\textwidth,
                                        height=.15\textheight
                                    }
                                    \begin{axis}[xlabel={Average energy consumption (Watts)}, title={Cores - FannkuchRedux - Energy - without outliers}, ytick={},
                                    yticklabels={
                                        
                                        },
                                        xmin=0,xmax=20,
                                        ]
                                    
                                    \end{axis}
                                \end{tikzpicture}
                            \caption{A comparison of of Cores energy consumption for test case FannkuchRedux for the Surface4Pro,  experiment \#2 (without outliers)} \label{fig:FannkuchRedux_Cores_comparison_energy_without_outliers_Surface4Pro_avg_watts_exp2}
                            \end{figure}
                            

The next DUT is the Surface Pro 4, where FannkuchRedux and Fasta are illustrated in \cref{fig:FannkuchRedux_Cores_comparison_dynamic_energy_without_outliers_Surface4Pro_avg_watts} and \cref{fig:Fasta_Cores_comparison_dynamic_energy_without_outliers_Surface4Pro_avg_watts} respectively, and the rest can be found in \cref{app:comparison_surfacepro4}. On the Surface Pro 4, a few things can be observed. When comparing the standard deviation, Intel Power Gadget deviates the most, which for example can be observed in the FannkuchRedux in \cref{fig:FannkuchRedux_Cores_comparison_dynamic_energy_without_outliers_Surface4Pro_avg_watts}. Intel Power Gadgets measurements are close the those made by LHM. When comparing RAPL against the measuring instruments on Windows, the difference changes from test case to test case,  where the measurements differ by $~6$W in FannkuchRedux in \cref{fig:FannkuchRedux_Cores_comparison_dynamic_energy_without_outliers_Surface4Pro_avg_watts} and $~1$W in Fasta in \cref{fig:Fasta_Cores_comparison_dynamic_energy_without_outliers_Surface4Pro_avg_watts}.

\paragraph{Surface Book}


                            \begin{figure}
                                \centering
                                \begin{tikzpicture}[]
                                    \pgfplotsset{%
                                        width=.85\textwidth,
                                        height=.15\textheight
                                    }
                                    \begin{axis}[xlabel={Average energy consumption (Watts)}, title={Cores - BinaryTrees - Energy - without outliers}, ytick={},
                                    yticklabels={
                                        
                                        },
                                        xmin=0,xmax=20,
                                        ]
                                    
                                    \end{axis}
                                \end{tikzpicture}
                            \caption{A comparison of of Cores energy consumption for test case BinaryTrees for the Surface4Pro,  experiment \#2 (without outliers)} \label{fig:BinaryTrees_Cores_comparison_energy_without_outliers_Surface4Pro_avg_watts_exp2}
                            \end{figure}
                            

                \begin{figure}[H]
                    \centering
                    \begin{tikzpicture}
                        \pgfplotsset{%
                            width=1\textwidth,
                            height=0.4\textheight
                        }
                        \begin{axis}[
                            xlabel={Start battery level},
                            ylabel={Average dynamic energy (watt)},
                            ymin=0,ymax=20,
                        ]
                        
                            \addplot [mark=none, ultra thick, red]  coordinates {
                            (40, 0.006595684402444291)(45, 0.007295220674193181)(50, 0.007999659608910881)(55, 0.007202467971243639)(60, 0.007280865079495958)(65, 0.006858423509955077)(70, 0.008369444141141925)(75, 0.007144940647010992)(80, 0.004753236834232667)
                            };
                            \addlegendentry{Surface4Pro - IntelPowerGadget}
                            
                            \addplot [mark=none, ultra thick, blue]  coordinates {
                            (40, 0.002385328427460912)(45, 0.0017856511573015649)(50, 0.0025901992189954056)(55, 0.00210998144366709)(60, 0.00286452646862575)(65, 0.0020038487280194628)(70, 0.002908483770774353)(75, 0.0005098111150931342)(80, -0.005995916826693459)
                            };
                            \addlegendentry{Surface4Pro - HardwareMonitor}
                            
                            \addplot [mark=none, ultra thick, orange]  coordinates {
                            (50, 251.8661014299577)(55, 214.7597259251014)(60, 162.23880995564176)(65, 107.43421593849766)(70, 53.9187387386668)(75, 0.7554852536149699)(80, -44.463085003568885)
                            };
                            \addlegendentry{Surface4Pro - RAPL}
                            
                            \addplot [mark=none, dashdotted, red]  coordinates {
                            (40, -0.004038062354887025)(45, -0.004312668500277529)(50, -0.003808663911021498)(55, -0.0037057407527755254)(60, -0.004478257932982471)(65, -0.0026308734995501644)(70, -0.0034090674446925874)(75, -0.003041497079697436)(80, -0.0020334307266356875)
                            };
                            \addlegendentry{SurfaceBook - IntelPowerGadget}
                            
                            \addplot [mark=none, dashdotted, blue]  coordinates {
                            (40, -0.002443518930616523)(45, -0.0029256880137447055)(50, -0.002551777773312929)(55, -0.0027567211782433486)(60, -0.002206859154402231)(65, -0.0026388300848279207)(70, -0.0024597736945479324)(75, -0.002445350799195827)(80, -0.000541271265011134)
                            };
                            \addlegendentry{SurfaceBook - HardwareMonitor}
                            
                            \addplot [mark=none, dashdotted, orange]  coordinates {
                            (40, 101.92702155010147)(45, 86.7212700795567)(50, 69.40754598562454)(55, 51.343785407669614)(60, 32.443112755251185)(65, 14.52577919077786)(70, -4.520517378551423)(75, -23.811706977384954)(80, -35.700372165212755)
                            };
                            \addlegendentry{SurfaceBook - RAPL}
                            
                        \end{axis}
                    \end{tikzpicture} 
                \caption{A graph illustrating the energy consumption of Dram for test case Nbody with regards to the battey level of the DUT (with outliers)} \label{fig:Nbody_Dram_charge}
                \end{figure}
                

For the Surface Book, test case BinaryTrees and Nbody can be seen in \cref{fig:BinaryTrees_Cores_comparison_dynamic_energy_without_outliers_PowerKomplett_avg_watts} and \cref{fig:Nbody_Cores_comparison_dynamic_energy_without_outliers_PowerKomplett_avg_watts} respectively, where the other test cases can be found in \cref{app:comparison_surfacebook}. On the Surface Book, the patterns are not as clear as they were on the Surface Pro 4. This is first of all because of the increased uncertainty in some of the results on the software measuring instruments on Windows, like for BinaryTrees in \cref{fig:BinaryTrees_Cores_comparison_dynamic_energy_without_outliers_PowerKomplett_avg_watts}. In one case, the difference between the median value for Intel Power Gadget and LHM stands out, with a difference of $~2$W for Nbody in \cref{fig:Nbody_Cores_comparison_dynamic_energy_without_outliers_PowerKomplett_avg_watts}. When comparing RAPL against Windows measurements, RAPL measures the highest in all cases except for FannkuchRedux.

\subsection{DUT and Measuring Instrument}


                            \begin{figure}
                                \centering
                                \begin{tikzpicture}[]
                                    \pgfplotsset{%
                                        width=.7\textwidth,
                                        height=.2\textheight
                                    }
                                    \begin{axis}[xlabel={Average energy consumption (Watts)}, title={Cores - Fasta - Energy - without outliers}, ytick={1, 2},
                                    yticklabels={
                                        IntelPowerGadget , HardwareMonitor 
                                        },
                                        xmin=0,xmax=80,
                                        ]
                                    
                                    \addplot+ [boxplot prepared={
                                    lower whisker=54.22445449466433,
                                    lower quartile=54.45751347572185,
                                    median=54.54624547297937,
                                    upper quartile=54.72020125409266,
                                    upper whisker=55.103791721157386},
                                    ] table[row sep=\\,y index=0] {\\};
                                    
                                    \addplot+ [boxplot prepared={
                                    lower whisker=51.45082608392138,
                                    lower quartile=51.933860038023106,
                                    median=52.121433545941585,
                                    upper quartile=52.479201309170854,
                                    upper whisker=54.95103920614709},
                                    ] table[row sep=\\,y index=0] {\\};
                                    
                                    \end{axis}
                                \end{tikzpicture}
                            \caption{A comparison of of Cores energy consumption for test case Fasta for the workstation (without outliers)} \label{fig:Fasta_Cores_comparison_energy_without_outliers_PowerKomplett_avg_watts_exp2}
                            \end{figure}
                            

                \begin{figure}[H]
                    \centering
                    \begin{tikzpicture}
                        \pgfplotsset{%
                            width=1\textwidth,
                            height=0.4\textheight
                        }
                        \begin{axis}[
                            xlabel={Start battery level},
                            ylabel={Average dynamic energy (watt)},
                            ymin=0,ymax=20,
                        ]
                        
                            \addplot [mark=none, ultra thick, red]  coordinates {
                            (40, 0.006595684402444291)(45, 0.007295220674193181)(50, 0.007999659608910881)(55, 0.007202467971243639)(60, 0.007280865079495958)(65, 0.006858423509955077)(70, 0.008369444141141925)(75, 0.007144940647010992)(80, 0.004753236834232667)
                            };
                            \addlegendentry{Surface4Pro - IntelPowerGadget}
                            
                            \addplot [mark=none, ultra thick, blue]  coordinates {
                            (40, 0.002385328427460912)(45, 0.0017856511573015649)(50, 0.0025901992189954056)(55, 0.00210998144366709)(60, 0.00286452646862575)(65, 0.0020038487280194628)(70, 0.002908483770774353)(75, 0.0005098111150931342)(80, -0.005995916826693459)
                            };
                            \addlegendentry{Surface4Pro - HardwareMonitor}
                            
                            \addplot [mark=none, ultra thick, orange]  coordinates {
                            (50, 251.8661014299577)(55, 214.7597259251014)(60, 162.23880995564176)(65, 107.43421593849766)(70, 53.9187387386668)(75, 0.7554852536149699)(80, -44.463085003568885)
                            };
                            \addlegendentry{Surface4Pro - RAPL}
                            
                            \addplot [mark=none, dashdotted, red]  coordinates {
                            (40, -0.004038062354887025)(45, -0.004312668500277529)(50, -0.003808663911021498)(55, -0.0037057407527755254)(60, -0.004478257932982471)(65, -0.0026308734995501644)(70, -0.0034090674446925874)(75, -0.003041497079697436)(80, -0.0020334307266356875)
                            };
                            \addlegendentry{SurfaceBook - IntelPowerGadget}
                            
                            \addplot [mark=none, dashdotted, blue]  coordinates {
                            (40, -0.002443518930616523)(45, -0.0029256880137447055)(50, -0.002551777773312929)(55, -0.0027567211782433486)(60, -0.002206859154402231)(65, -0.0026388300848279207)(70, -0.0024597736945479324)(75, -0.002445350799195827)(80, -0.000541271265011134)
                            };
                            \addlegendentry{SurfaceBook - HardwareMonitor}
                            
                            \addplot [mark=none, dashdotted, orange]  coordinates {
                            (40, 101.92702155010147)(45, 86.7212700795567)(50, 69.40754598562454)(55, 51.343785407669614)(60, 32.443112755251185)(65, 14.52577919077786)(70, -4.520517378551423)(75, -23.811706977384954)(80, -35.700372165212755)
                            };
                            \addlegendentry{SurfaceBook - RAPL}
                            
                        \end{axis}
                    \end{tikzpicture} 
                \caption{A graph illustrating the energy consumption of Dram for test case Nbody with regards to the battey level of the DUT (with outliers)} \label{fig:Nbody_Dram_charge}
                \end{figure}
                

The last comparison in this experiment is between all DUTs and measuring instruments. The test cases Fasta and Nbody are used, where the rest can be found in \cref{app:comparison}. When comparing the different DUTs, some of the same tendencies can be observed. This could be the similarity between Intel Power Gadget and LHM in most cases, with a few outliers like Nbody on Surface Book in \cref{fig:Nbody_Cores_comparison_dynamic_energy_without_outliers_avg_watts}. An outlier like this, is however not consistent for the DUT, meaning, the Intel Power Gadget and LHM measurements for the workstation and Surface Pro 4 for Nbody are still similar. Another thing to consider is the correlation between when RAPL measurements are either higher/lower compared to the Windows measurements across the same DUT. Here it is however more difficult to find a pattern. In most cases, RAPL will measure a lower value across all measuring instruments for all DUTS, with some exceptions. Exceptions include Surface Book on Fasta in \cref{fig:Fasta_Cores_comparison_dynamic_energy_without_outliers_avg_watts}, where the RAPL measurement is higher. For the other two DUT's, RAPL will however measure lower energy consumption. Another example is the energy consumption for the test case Nbody in \cref{fig:Nbody_Cores_comparison_dynamic_energy_without_outliers_avg_watts}, where both the Surface Book and the Workstation, RAPL reports a higher energy consumption, this is however not the case for the Surface Pro 4.



\section{Hardware}\label{sec:hardware}

When measuring the energy consumption of hardware, different software measuring tools were introduced in \cref{sec:energy_profilers}. In this section the hardware these tools will measure the energy consumption on will be introduced. This will include an analysis of the different requirements the tools had, in order to find which combination of hardware will give the best result.

\section{How to Save the Data}\label{sec:save_data}

When measuring the energy consumption, the results needs to be saved somewhere, for later analysis. The data should therefore be saved in a structured way, where it can be easily accessed later. Another point is that the data should be collected in one place, despite coming from multiple computers, and lastly during the experiments, the solution should consume no power. For this, and SQL database was chosen. This will be discovered further now.

\subsection{SQL}\label{subsec:sql}

SQL makes a lot of sense to use when working with structured data, where it enables the user to process the data later in an optimized manner. The structure of the data can be seen in \cref{fig:uml_diagram}, and is used for two things during the experiments. The first thing is to capture the results, where a connection to the database will be made after each run of the experiments, to upload to data. The connection is then shut down again, before the next run, to avoid using energy on this, which could affect the results. The relevant components for capturing results from \cref*{fig:uml_diagram} are as following:

\begin{itemize}
    \item Test case: Represents what test program was used in the experiment through a unique ID and a name.
    \item Measuring instrument: Represents what measuring instrument was used in the experiment through a unique ID and name
    \item DUT: Represents which hardware specifications the test case was executed on through a unique ID, name, operating system and version\feetnote{The version is to separate different versions of the framework. Meaning each time a change is made to the framework, the version increments.}. 
    \item RawData: Represents the output from the measuring instrument, in the experiment. This is represented through a unique ID, an experiment ID, a value and the date of execution. The value here is a serialized object representing the data, as the different measuring instruments had different formats in their output.
    \item Configuration: Represents what configuration was used during the experiments. Here min/max temp and battery represent the limits the system had to exceed before the results were no longer useful, and a system restart and cooldown period are required. Between represents the cooldown period between two test case runs in minutes, and duration specifies the minimum duration the test case had to run for denoted in minutes. This is also represented through a unique ID.
    \item Temperature: Represents a temperature measurement made on the system during the experiments. The measurement is represented through a unique ID, an experiment ID, a time, a name and a value. Here the name is the sensor name e.g. 'CORE \#1'.
    \item Experiment: The experiment ties all other tables together, where one experiment is represented with a unique ID, configuration ID, measuring instrument ID, DUT ID, test case ID, the programming language and start/end times. This enables one experiment to have multiple temperature measurements, one program to be used in several experiments etc. In addition to this, runs represent how many times the test case is executed in order to run for at least the configurations duration value, and iteration represents how many times the test case has been measured at the given point in time, by the given measuring instrument since the last restart of the computer. This is relevant for the R3-Validation, where for the first ABC, the iteration will be 1 for all measuring instruments, the second BCA will be 2 etc.
\end{itemize}

The second use case for the data is to save the state of the program upon a restart. This is relevant for R3-Validation, where after each restart, a new measuring instrument needs to start. The relevant tables for this are as follows:

\begin{itemize}
    \item Test case: Same definition as before
    \item DUT: Same definition as before
    \item Run: Represents what measuring instrument was the first measuring instrument last time an experiment was run on for a given test case on a given DUT. This is represented using a unique run ID, a test case ID and a DUT ID. In addition to this is the value, which represents the different measuring instruments used in the experiment, and which one started last time.
\end{itemize}

\begin{figure}[H]
    \centering
    \begin{tikzpicture}
        \begin{object}[text width=4 cm]{TestCase}{0 ,1}
            \attribute{TestCaseId : INT}
            \attribute{Name : VARCHAR}
            \attribute{Language : VARCHAR}
        \end{object}
        \begin{object}[text width=4 cm]{Dut}{10,1}
            \attribute{DutId : INT}
            \attribute{Name : VARCHAR}
            \attribute{OS : VARCHAR}
            \attribute{Version : INT}
        \end{object}
        \begin{object}[text width=4 cm]{Run}{5,5}
            \attribute{RunId : INT}
            \attribute{DutId : INT}
            \attribute{TestCaseId : INT}
            \attribute{Value : VARCHAR}
        \end{object}
        \begin{object}[text width = 4 cm]{MeasuringInstrument}{5,1}
            \attribute{InstrumentId : INT}
            \attribute{Name : VARCHAR}
        \end{object}
        \begin{object}{Measurement}{6.5,-2.5}
            \attribute{MeasurementId : INT}
            \attribute{ConfigId : INT}
            \attribute{InstrumentId : INT}
            \attribute{DutId : INT}
            \attribute{TestCaseId : INT}
            \attribute{Runs : INT}
            \attribute{Iteration : INT}
            \attribute{FirstMeasuring : VARCHAR}
            \attribute{StartTime : DATETIME(6)}
            \attribute{EndTime : DATETIME(6)}
        \end{object}
        \begin{object}[text width=4 cm]{RawData}{0, -10}
            \attribute{RawDataId : INT}
            \attribute{MeasurementId : INT}
            \attribute{Value : VARCHAR}
            \attribute{Time : DATETIME(6)}
        \end{object}
        \begin{object}[text width=4 cm]{TimeSeries}{0, -4.5}
            \attribute{TimeSeriesId : INT}
            \attribute{MeasurementId : INT}
            \attribute{Value : VARCHAR}
            \attribute{Time : DATETIME(6)}
        \end{object}
        \begin{object}[text width = 4 cm]{Configuration}{5,-10}
            \attribute{ConfigurationId : INT}
            \attribute{MinTemp : INT}
            \attribute{MaxTemp : INT}
            \attribute{Between : INT}
            \attribute{Duration : INT}
            \attribute{MinBattery : INT}
            \attribute{MaxBattery : INT}
            \attribute{Version : INT}
        \end{object}
        \begin{object}[text width = 4 cm]{Environment}{10, -10}
            \attribute{EnvironmentId : INT}
            \attribute{MeasurementId : INT}
            \attribute{Time : DATETIME(6)}
            \attribute{Name : VARCHAR}
            \attribute{Value : INT}
            \attribute{Type : VARCHAR}
        \end{object}
        
        \association{Run}{}{0..*}{TestCase}{}{1}
        \association{Run}{}{0..*}{Dut}{}{1}
        \association{Measurement}{}{0..*}{Dut}{}{1}
        \association{Measurement}{}{0..*}{MeasuringInstrument}{}{1}
        \association{Measurement}{}{0..*}{Dut}{}{1}
        \association{Measurement}{}{1}{RawData}{}{1}
        \association{Measurement}{}{1}{TimeSeries}{}{1}
        \association{Measurement}{}{0..*}{Configuration}{}{1}
        \association{Measurement}{}{0..*}{Environment}{}{0..*}
        \association{Measurement}{}{0..*}{TestCase}{}{1}
    \end{tikzpicture}
    \caption{An UML diagram representing the tables in the SQL database} \label{fig:uml_diagram}
\end{figure}





\section{Statistical methods}\label{sec:stat}

As this work will aim to compare different ways to measure the energy consumption of software, not only across different measuring approaches, but also different hardware and operating systems, a way to represent the data is required. In this section different ways to make the data presentable and comparable will be presented. The selection of the varies statical methods are in partly inspired by Koedijk et al\cite{koedijk2022finding} and Fahad et al \cite{fahad2019comparative}.
\subsection{Distribution}
To answer the first question it is important to find out which statistical tests that could be used on the data. One of the most common preconditions for statistical test is that the data should be a normal distribution for the results to work. To test if a distribution is a normal Distribution a normalcy test have to be used. While there are many different normalcy tests it seems that the Shapiro-Wilk test is generally more powerful than the others that are commonly used \cite{razali2011power}. Because of this the Shapiro-Wilk test will be used to check for normal distributions. The formula for the Shapiro-Wilk test:
$$W=\frac{( \sum{a_i x_i} )^2}{\sum{(x_i - \bar{x})^2}}$$
The null hypnosis's $H_0$ for the test is:
\textit{The sample comes from a normally-distributes population}
There is also the concept of a P-value which is essentially a threshold for wether we can reject $H_0$ or not. The P-value is then compared with a alpha value that we chose, essentially if $P \geq \alpha$ then $H_0$ can be rejected.

If the $H_0$ for the Shapiro-Wilk test could not be rejected we assume that the distribution are normal or close enough not to be important.

t-test could be used to differentiate the two sample means to see if the difference between them is real or occurrence because of a random outside variables. Using a t-test does however assumes something thing about the data used.
\begin{itemize}
    \item Both samples should follow a normal Distribution.
    \item The samples are independent of each other.
    \item Both the samples should also have approximately the same variance.
\end{itemize}
\subsection{Correlation}
Correlation is a statistical measurement that essentially informs us of the relationship between two samples. The correlation can be either positive or negative and a measure of how correlated they are. A strong correlation does however not mean that changes i one variable effects the other just that it statically seems to be the case.
\paragraph{Pearson correlation coefficient} The Pearson correlation coefficient is one of if not the most used method for finding the correlations in linear relationships. The strength of the correlation is spans between $-1 \dotsc 1$, were the directions is given by the polarity sign. Depending on the context different strength are considered as strong or weak, but commonly 0 means that there is no correlation between the two samples. Some Assumptions have to be true about the data for us to utilize Pearson correlation coefficient:
\begin{itemize}
    \item The data must be quantitative
    \item The two samples must be normally distributed
    \item There must be no outliers in the data
    \item The relationship between the two must also be linear
\end{itemize}
If all of these assumptions are true for the data then the Pearson correlation coefficient can be calculated with the formula:
% $$r=\frac{n\sum{xy- (\sum{x})(\sum{y})}}{\sqrt{[n\sum{x^2}-(\sum{x})^2][n\sum{y^2}-(\sum{y}^2)]}}$$
$$r=\frac{\sum{(x_i-\bar{x})(y_i-\bar{y})}}{\sqrt{\sum{(x_i-\bar{x})^2}\sum{(y_i-\bar{y})^2}}}$$
If however the data does not follow the conditions then another method of calculating the correlation will have to be used. 
\paragraph{Kendall tau correlation}
There are two common choices when doing non-parametric rank correlation, which we can use on non-normally distributed data. These are the Spearman's rank correlation coefficient and Kendall tau's correlation coefficient. While both can be used in the same scenarios there are some differences betweens them.Both are ranked with the same procedure that Mann–Whitney U tests are. The Spearman rank seems to be the most popular way measuring correlations, but some argue that Kendall tau is better in most cases \cite{gilpin1993table}. Some of the big differences between the two the effect of outliers Spearman's rank is largely effected by a few outliers in the data while Kendall tau in comparison seems more robust. Because of this the choice to utilize Kendall tau seems the most natural, and is what will be continued with. The results from Kendall tau can be interpreted the same way as from Pearson, the range is from $-1$ to $1$. The formula for kendall tau is also pretty straight forward:
$$\frac{C-D}{C+D}$$
C is Concordant pairs and D is Discordant. Essentially a Concordant pairs are every time the rankings are in order while Discordant is disordered pairs\cite{kendall1938new}.
 

\subsection{Anomalies}
Yeet
\subsection{Sample size formulas}\label[subsec]{subsec:cock}
It is important to get enough measurements for each test case such that the test case measurements reflect a representative mean and standard deviation. Meanwhile gathering too many measurements, would be a waste of resources therefore an approach to determine a sufficient amount is needed. In other words, the required sample size needs to be determined. There are several approaches used in the literature. For example, some papers use what seems like an arbitrary number of measurements\cite{Pereira2017,Koedijk2022diff,Georgiou2020}. Another method is to base the number of measurements on how many times the experiment can be run within a time-frame\cite{sestoft2013microbenchmarks}. In this report, a formula will be used to calculate the number of measurements needed i.e. the sample size. One formula or family of formulas are Cochran's formula.\cite{Cochran, kotrlik2001organizational,israel1992determining}, which gives the minimum number of required observations for performance metrics within a specified standard deviation. However, there are several versions of the formula.
The first one in \cref{cochransEQ1}\footnote{The terminology used by Cochran is different to the one utilized in this report} is for categorical data when the population is large\cite{israel1992determining}:

\begin{equation}
    n_0 = \frac{Z^2*p*q}{e^2}
    \label[equation]{cochransEQ1}
\end{equation}

Where

\begin{itemize}
    \item $n_0$ is the number of measurements (sample size)
    \item $Z$ is the abscissa of the normal curve which removes an area $\alpha$ at the tails of the distribution. (Z-score) Where $1 - \alpha$ is the desired confidence level.%is the z-value, which is found using the z-table and represents the confidence level
    \item $pq$ is an estimate of variance, where
    \begin{itemize}
        \item $p$ is the estimated proportion of an attribute in the data
        \item $q$ is $1 - p$
    \end{itemize}
    \item $e$ is desired level of precision (acceptable margin of error)
\end{itemize}


Then there is a correction step for when the sample size is larger than 5\% of the population. As well as for smaller populations a smaller sample size is necessary.\cite{israel1992determining,kotrlik2001organizational} The equation is shown in \cref{cochransCorrection}.

\begin{equation}
    n = \frac{n_0}{1+\frac{(n_0-1)}{N}}
    \label[equation]{cochransCorrection}
\end{equation}

Where $n$ is the new sample size and $N$ population size.



Furthermore, there is a simplified formula called Yamane's formula, shown in \cref{yamane}\cite{israel1992determining}. 

\begin{equation}
    n = \frac{N}{1 - N(e)^2}
    \label[equation]{yamane}
\end{equation}


Lastly, there is another version of Cochran's formula shown in \cref{cochransEQ2}, Which is for continuous data. 

\begin{equation}
    n_0 = \frac{Z^2*\sigma^2}{e^2}
    \label[equation]{cochransEQ2}
\end{equation}

Where instead of $pq$ we have $\sigma$ which is the standard deviation of the data. The correction formula can also be used in combination with this formula.\nytafsnit



It should still be considered if categorical variables will play an important role in the data analysis, then \cref{cochransEQ1} should be used\cite{kotrlik2001organizational}. For our data \cref{cochransEQ2} seem to most accurately fit the description and is therefore chosen. The correction formula \cref{cochransCorrection} is not needed since we, in theory, have an infinite population, while in practice it is of course limited by time. With Cochran's formula given a desired margin of error, desired confidence level and an estimate of the standard deviation a sample size can be calculated. So these variables are determined as follows. 

The confidence level is how little the results must deviate. A confidence level of 95\% would represent 95\% of the data points would match 95\% of the times. If a confidence level of 95\% is desired and confidence level $= 1 - \alpha$ then $\alpha = 0.05$. 
Kotrlik et al. found that a margin of error of 5\% is commonly chosen and is found acceptable for categorical data, but for continuous data, a 3\% margin of error is found acceptable\cite{kotrlik2001organizational}. Since the variables measured are continuous data a margin of error of 3\% is chosen.

The Z-score reflects how many standard deviations a measurement is from the mean. An approximate score can be found in a Z-table when there is a desired confidence level or $\alpha$. The most commonly used $\alpha$ is $0.05$ or $0.01$.\cite{kotrlik2001organizational} In this report $0.05$ is chosen which gives a confidence level of 95\%. \todo{I have not found a reason as to why.}From the Z-table the estimated Z-value is then $1.96$. 

Lastly, an estimate of the standard deviation is also needed, which is not available. Cochran listed four methods for estimating the standard deviation in case it is not initially available:
\begin{enumerate}
    \item The measurements are taken in two steps. The first step is used to determine how many further measurements are required in step two, based on the standard deviation in the first set of measurements.
    \item Utilizing results of a pilot study
    \item Utilizing results from a similar study
    \item Come up with a guess assisted with logical-mathematical results.
\end{enumerate}


Method one and two are both based on the same principle of getting an initial smaller amount of measurements. Method three is to the extent of our knowledge not feasible since no results from a study similar enough are available. Method four requires more knowledge than we possess to give a qualified estimate. Therefore method one is chosen. Now for step one when making a small number of measurements it is required to know how many measurements to acquire. The Central Limit Theorem says that the mean of a small number of samples will become close to the mean of the overall population in correlation with an increase in the sample size. Ross found that at least 30 samples are enough for the central limit theorem to hold\cite{Ross}. Which means the distributions of the sample means are close to normally distributed.

% Influence of pragmming paradigms picks 100 based on 9781118805350.

From the initial sample, an estimated standard deviation of each parameter measured can be acquired. Then the minimum number of measurements required can be calculated for each parameter. Whereafter the largest of the values are chosen and all of the experiments are run that number of times to get the minimum required measurements.\todo{Is this true actually what we do?}


% Focus on report influence of programming paradigms.

% Probabply not normal distributed - Jeppe alcholt


\section{Experimental/Test Setup}\label[subsection]{ExperimentalSetup}
After the different components required to execute the experiment have been introduced, it is now time to take a look at the experiments. This will be regarding the languages used, how the data is saved, how many times the experiments are run etc.
\todo{Hvad vi vælger fra sestoft, Feetings osv og hvorfor vi vælger det}
\subsection{Measurement Methodology} \label[]{Measure_meth}
When conducting the experiments three different kinds of measuring approaches are being utilized from the tools in \cref{sec:hardware}.
The first is hardware-based measurements, which are system-level physical measurements. Another approach is software based measurements: RAPL, LHM, Intel Power Gadget and sometimes E3, where measurements are performed using on-chip sensors. Lastly, there are energy predictive models, which are utilized by E3, when there is no power meter chip in the DUT.\todo{We could remove this from here as it should be described in a previous section.} There is however some uncertainty about how E3 works, this is covered in \cref{sec:E3Experiments}. Of these three approaches, the most accurate is the system-level physical measurements which can provide highly accurate measurements for the DUT's energy consumption, but these measurements are for the entire DUT, so they cannot provide a view of the power consumption of applications on an individual level, which the E3 can do. It is however not necessary to get a per-application energy consumption, as only the test cases' energy consumption is to be calculated, dynamic energy consumption can solve this, which is expanded upon in \cref{subsec:how_to_measure}

%Dynamic energy consumption is an estimation of the application's power consumption. To calculate the dynamic energy consumption $E_D$, as presented by Fahad et al \cite{fahad2019comparative}, the following formula is used:
%$$E_D = E_T -(P_S * T_E)$$
%Where $E_T$ is the total energy consumption by the DUT when running the experiment, $T_E$ is the duration of the program execution and $P_S$ is the energy consumption when the DUT is idle. The dynamic energy consumption will then represent the energy consumption of the test case. To do this a certain amount of control over the DUT is required and some precautions are necessary. The procedures that will be used are different for the DUTs and OSes, but some will be the same for all DUTs as shown here:, and are a combination of procedures presented in \cref{sec:rw_measureing_methodology_setup} by Sestoft\cite*[]{sestoft2013microbenchmarks} and Fahad et al.\cite*[]{fahad2019comparative}:
%
%\begin{itemize}
%    \item The machines are reserved exclusively for the experiments, this is to prevent deviations in the results.
%    \item The networking will be disabled on the machines to ensure that these do not affect the results.
%    \item The processes and the temperature of the machines will be measured before and after each experiment.
%    \item After each batch, the DUT will be restarted and its setup phase is performed before continuing to the next batch.
%    \item The memory of the test cases\todo{Hvordan ved vi det?} will not exceed the DUT's main memory to avoid memory swapping
%    \item Some baseline power usage will be established by measurements before and after the experiments
%\end{itemize}

%Each batch will be tested on every DUT with each type of OS these being Windows 10 and Ubuntu.
%The specifications for the different DUT vary, both in age, hardware specifications and manufacturer, this is done to get a wider scope of tests to see if certain measurements work better on certain DUTs. For the experiment one desktop workstation and two laptops will be used. The specs for the workstation can be seen in \cref{tab:komplett}, the Surface Pro 4 can be seen in \cref{tab:surfacePro} and Surface Book can be seen in \cref{tab:surfaceBook}. Comparing these will help evaluate the different measuring methods for the different DUTs, and provide insight into why they may or may not differ in performance.
%
%
%Each DUT will have a slightly different setup depending on the hardware and the OS of the DUT. For the Windows and Linux machines, the various unnecessary background processes will be disabled to ensure as little noise in the measurements as possible\cite*[]{sestoft2013microbenchmarks}, the specific process disabled can be seen in \cref{tab:disabled_proc}.\todo{How did we decide this} There are no disabled processes on the Linux OS, this was choice was made on the general assumption that Ubuntu has less unnecessary background processes than Windows.\todo{Does a proper source exist for this statement} A difference between the DUTs is if whether they are desktops or laptops. On the desktop it is possible to perform system-level physical measurements because the measurement is system-wide it is important to have as little non test case related power fluctuation as possible, this is why the fans of the DUT will be set to max on the workstation. For the laptops, this is not an issue as the measurements do not include the whole system. For the laptops, some special controls are necessary for the measurements because measurement quality can become worse if the laptops are connected to the charger during measurements\cite{E3Video}. To prevent this the charger is shut off during the experiments, and then turned on to fully recharge the DUT during the setup phase, this cycle will repeat until all of the experiments are done.

%\begin{table}[ht]
    \centering
    \begin{tabular}{| l | l |}
    \hline
    \textbf{Windows}    & \textbf{Linux}    \\ \hline
    AsusUpdateCheck     &   \\ \hline
    AsusDownLoadLicense &   \\ \hline
    msedge              &   \\ \hline
    OneDrive            &   \\ \hline
    GitHubDesktop       &   \\ \hline
    Microsoft.Photos    &   \\ \hline
    SkypeApp            &   \\ \hline
    SkypeBackgroundHost &   \\ \hline
    \end{tabular}
    \caption{The background processes which are disabled when running a test case}
    \label{tab:disabled_proc}
\end{table}

%% save in DB in SQL
%% Script written in C#
%% Cochrans 

