\section{Software-based measuring instruments}\label{sec:measuring_instruments}

When considering \textbf{RQ2}, different measuring instruments are needed. In this section, different existing approaches will be presented, and then compared.

When selecting software-based measuring instruments there are several options which are chosen from different categories, including software manufacturers, hardware manufacture and open source. From the hardware manufacturers, Intel's RAPL and Intel's Power Gadget were chosen as their solutions with potential for hardware support could be promising. For software manufacturers, Microsoft's E3 was chosen as their domain knowledge could prove valuable for energy estimation. Lastly, an open source project is included, this being Open Hardware Monitor, to see how this compared to three, potentially more sophisticated solutions. Open Hardware Monitor was chosen as a C\# implementation was found, which made it easy to use, and based on the assumption that all open source versions will measure in the same way. As the most used approach in existing work, Intels RAPL will also be used. All the different software-based measuring instruments will be introduced now, and compared in the end.


                        \begin{figure}
                            \centering
                            \begin{tikzpicture}[]
                                \pgfplotsset{%
                                    width=.6\textwidth,
                                    height=0.4\textheight
                                }
                                \begin{axis}[xlabel={Average dynamic energy (Watts)}, title={SurfaceBook - RAPL}, ytick={1, 2, 3, 4},
                                yticklabels={
                                    BinaryTrees - RAPL, FannkuchRedux - RAPL, Nbody - RAPL, Fasta - RAPL
                                    },
                                    xmin=0,xmax=80,
                                    ]
                                
                                    \addplot+ [boxplot prepared={
                                    lower whisker=4.828474728535248,
                                    lower quartile=4.965529391276659,
                                    median=5.123137462250352,
                                    upper quartile=5.29291940764101,
                                    upper whisker=5.6252351289872085},
                                    ] table[row sep=\\,y index=0] {\\};
                                    
                                    \addplot+ [boxplot prepared={
                                    lower whisker=5.062875916575935,
                                    lower quartile=5.15876774202521,
                                    median=5.1819262057960325,
                                    upper quartile=5.198190098979957,
                                    upper whisker=5.255077811846938},
                                    ] table[row sep=\\,y index=0] {\\};
                                    
                                    \addplot+ [boxplot prepared={
                                    lower whisker=4.780004016964808,
                                    lower quartile=4.8070152384092,
                                    median=4.82418743669118,
                                    upper quartile=4.838536851432698,
                                    upper whisker=4.886202997659311},
                                    ] table[row sep=\\,y index=0] {\\};
                                    
                                    \addplot+ [boxplot prepared={
                                    lower whisker=5.036917950104407,
                                    lower quartile=5.263664537834856,
                                    median=5.398020641615526,
                                    upper quartile=5.549948759622877,
                                    upper whisker=5.807261082297773},
                                    ] table[row sep=\\,y index=0] {\\};
                                    
                                \end{axis}
                            \end{tikzpicture}
                        \caption{R3 validation for dynamic energy measurements by RAPL for the Dram for all DUT's on Unix and test cases where the impact of the first profiler can be seen (with outliers)} \label{fig:SurfaceBook_RAPL_Dram_R3_dynamic_energy_with_outliers_Unix_avg_watts}
                        \end{figure}
                        
\subsection*{Microsoft energy estimation engine}
Microsoft estimation energy estimation engine(E3), is a tool created to monitor battery usage on windows devices, because of this it is only available on windows devices with battery's. E3 monitors the computer at all times to created battery energy usage reports, that can be accessed in the power & battery 
\section{Intel PowerGadget}

The Intel Power Gadget\cite[]{powergadget} is a software tool for measuring power consumption of Intel Core processors from the 2nd to 10th generation, for both Windows and macOS.

What this software tool offers, is real-time estimations of the energy consumption in watts using the energy counters in the processors.

The tool also contains a command line version called Powerlog, in addition to this, the newest versions also includes estimations of energy consumption on multi socket systems and externally callable APIs. These APIs can b used to extract information within sections of code. This is achieved by evaluating the energy MSR on a per-socket basis.

When measuring energy consumption the average power is measured in watts, cumulative energy in joules, cumulative energy in miliWatt-hours, temperature in Celsius and frequency in GHz.

When considering the API, the sampling frequency can range from 1 to 1000 milliseconds. Here it is noted a high frequency will bring a greater accuracy, but poorer performance of the system, and a frequency of 100 milliseconds, which is the default value, is recommended.\cite*[]{powergadget_api}.
\subsection{Open Hardware Monitor / Libre Hardware Monitor}\label[subsec]{subsec:HardwareMonitor}
Open Hardware Monitor is a free open-source piece of software able to monitor metrics like temperature, fan speeds, voltages, loads and clock speeds on a computer for both CPU and GPU. According to the documentation, the software supports most hardware monitoring chips on motherboards and works for both Intel and AMD chips, in addition to both 32bit and 64bit windows, and any x86-based Linux. When running, the software will publish all data to the Windows Management Instrumentation (WMI), which allows other applications to use the data.\cite[]{open_hardware_monitor}

This tool is not mentioned in any articles, and accuracy is not mentioned anywhere in the documentation. Generally, the information in the literature and its documentation pertaining to accuracy are either very sparse on non-existent. A fork of Open Hardware Monitor called Libre Hardware Monitor (LHM) is available on Github\cite{libre_hardware_monitor}. From here a version without a GUI is used to minimize the energy consumption from the measuring instrument itself.
\subsection{Comparison}\label{subsec:software_comparison}

After the introduction of different existing energy profilers, a comparison can be made. Firstly, when considering the four tools it can be seen that most of them provide the measurements of the same components which is ideal for comparing their results later. The most commonly used software-based measuring instrument is Intel's RAPL. However, it is limited to Linux. When considering the Windows tools, none of the tools has any credible sources documenting their accuracies. One interesting tool is E3, as it estimates the power of all processes running on the computer individually, and how a Maxim chip can, according to Microsoft, bring the accuracy up to 98\%.\cite[]{E3WinHec} E3 does however only work on devices with a battery, as opposed to Open Hardware Monitor and Intel Power Gadget. When comparing the latter two, they seem very similar, where one big advantage of Open Hardware Monitor is that it is open source, so any uncertainty could be verified by looking at the actual implementation, and the implementation can be altered to fit the purpose. The advantage of Intel's Power Gadget and Open Hardware Monitor is that they have an implemented interface, which makes it easier to access the relevant data. Opposed to this we have E3, where a log needs to be reset, and a log file needs to be parsed, which is not ideal. We have chosen to use Intel's RAPL, Microsoft's E3, Intel's Power Power Gadget and Open Hardware Monitor. 

% WHHY?

%% AMD version  AMP. Source siger bad. Hold os inden for en producent af CPU.