\section{Measuring instruments}\label{sec:measuring_instruments}

The aim of this work is to compare different measurement instruments, across different DUT's and OS's, as is covered in \textbf{RQ2-4}. In this section, different measurement instruments used in this work will be introduced.

When considering measuring approahces, three different kinds exists:

\begin{itemize}
    \item Hardware-based measurements: System-level physical measurements
    \item Software based measurements: RAPL, LHM, Intel Power Gadget and E3s.
    \item Energy predictive models: A model estimating the energy consumption based on performance monitoring counters.
\end{itemize}

\paragraph*{}
When selecting measurement instruments, they can be either a energy predictive model, software or hardware based. When considering software based measurement instruments, multiple versions exists, where this work will aim to include a wide variety. This will include a Linux and Windows versions created by Intel, this being RAPL and Intel Power Gadget. RAPL was chosen as \cref*{ch:related_work} found this to be widely used in existing work, and Intel Power Gadget makes sense, as this is the Windows version, created by the same company. Microsoft has also made their own versions, this being E3. This was chosen based on the assumption of how their domain knowledge could prove valuable. Another measurement instrument included is Libre Hardware Monitor (LHM), representing open source. LHM was chosen as it was chosen as their implementation was easy to use. An assumption here is that all open source versions will measure energy in the same way. Energy predictive models are models estimating the energy consumption based on performance monitoring counters, and will be used by E3 in some cases. Lastly a hardware measurement instrument is also included, and will serve as the ground truth.



% When considering the measurement methodology, it is in regards to a few things. This is first of all the different kinds of measurement for both software and hardware approahces.



% A thing to note here is some uncertainty regarding E3, as covered in \cref{sec:E3Experiments}. In this case the assumption is that this measurement instrument will utilize on-chip sensors if a chip like the MAXIM chip is available, otherwise an energy predictive model will be used. Of these three approaches, the most accurate is the system-level physical measurements which can provide highly accurate measurements for the DUT's energy consumption, but these measurements are for the entire DUT, so they cannot provide a view of the power consumption of applications on an individual level, which the E3 can do. It is however not necessary to get a per-application energy consumption, as only the test cases' energy consumption is to be calculated, dynamic energy consumption can solve this, which is expanded upon in \cref{subsec:how_to_measure}



                        \begin{figure}
                            \centering
                            \begin{tikzpicture}[]
                                \pgfplotsset{%
                                    width=.6\textwidth,
                                    height=0.4\textheight
                                }
                                \begin{axis}[xlabel={Average dynamic energy (Watts)}, title={SurfaceBook - RAPL}, ytick={1, 2, 3, 4},
                                yticklabels={
                                    BinaryTrees - RAPL, FannkuchRedux - RAPL, Nbody - RAPL, Fasta - RAPL
                                    },
                                    xmin=0,xmax=80,
                                    ]
                                
                                    \addplot+ [boxplot prepared={
                                    lower whisker=4.828474728535248,
                                    lower quartile=4.965529391276659,
                                    median=5.123137462250352,
                                    upper quartile=5.29291940764101,
                                    upper whisker=5.6252351289872085},
                                    ] table[row sep=\\,y index=0] {\\};
                                    
                                    \addplot+ [boxplot prepared={
                                    lower whisker=5.062875916575935,
                                    lower quartile=5.15876774202521,
                                    median=5.1819262057960325,
                                    upper quartile=5.198190098979957,
                                    upper whisker=5.255077811846938},
                                    ] table[row sep=\\,y index=0] {\\};
                                    
                                    \addplot+ [boxplot prepared={
                                    lower whisker=4.780004016964808,
                                    lower quartile=4.8070152384092,
                                    median=4.82418743669118,
                                    upper quartile=4.838536851432698,
                                    upper whisker=4.886202997659311},
                                    ] table[row sep=\\,y index=0] {\\};
                                    
                                    \addplot+ [boxplot prepared={
                                    lower whisker=5.036917950104407,
                                    lower quartile=5.263664537834856,
                                    median=5.398020641615526,
                                    upper quartile=5.549948759622877,
                                    upper whisker=5.807261082297773},
                                    ] table[row sep=\\,y index=0] {\\};
                                    
                                \end{axis}
                            \end{tikzpicture}
                        \caption{R3 validation for dynamic energy measurements by RAPL for the Dram for all DUT's on Unix and test cases where the impact of the first profiler can be seen (with outliers)} \label{fig:SurfaceBook_RAPL_Dram_R3_dynamic_energy_with_outliers_Unix_avg_watts}
                        \end{figure}
                        
\subsection*{Microsoft energy estimation engine}
Microsoft estimation energy estimation engine(E3), is a tool created to monitor battery usage on windows devices, because of this it is only available on windows devices with battery's. E3 monitors the computer at all times to created battery energy usage reports, that can be accessed in the power & battery 
\section{Intel PowerGadget}

The Intel Power Gadget\cite[]{powergadget} is a software tool for measuring power consumption of Intel Core processors from the 2nd to 10th generation, for both Windows and macOS.

What this software tool offers, is real-time estimations of the energy consumption in watts using the energy counters in the processors.

The tool also contains a command line version called Powerlog, in addition to this, the newest versions also includes estimations of energy consumption on multi socket systems and externally callable APIs. These APIs can b used to extract information within sections of code. This is achieved by evaluating the energy MSR on a per-socket basis.

When measuring energy consumption the average power is measured in watts, cumulative energy in joules, cumulative energy in miliWatt-hours, temperature in Celsius and frequency in GHz.

When considering the API, the sampling frequency can range from 1 to 1000 milliseconds. Here it is noted a high frequency will bring a greater accuracy, but poorer performance of the system, and a frequency of 100 milliseconds, which is the default value, is recommended.\cite*[]{powergadget_api}.
\subsection{Open Hardware Monitor / Libre Hardware Monitor}\label[subsec]{subsec:HardwareMonitor}
Open Hardware Monitor is a free open-source piece of software able to monitor metrics like temperature, fan speeds, voltages, loads and clock speeds on a computer for both CPU and GPU. According to the documentation, the software supports most hardware monitoring chips on motherboards and works for both Intel and AMD chips, in addition to both 32bit and 64bit windows, and any x86-based Linux. When running, the software will publish all data to the Windows Management Instrumentation (WMI), which allows other applications to use the data.\cite[]{open_hardware_monitor}

This tool is not mentioned in any articles, and accuracy is not mentioned anywhere in the documentation. Generally, the information in the literature and its documentation pertaining to accuracy are either very sparse on non-existent. A fork of Open Hardware Monitor called Libre Hardware Monitor (LHM) is available on Github\cite{libre_hardware_monitor}. From here a version without a GUI is used to minimize the energy consumption from the measuring instrument itself.
\subsection{Hardware measurements}\label{sec:clampIntro}
Hardware-based measuring instruments are often from the literature seen as the most accurate way to measure the energy consumption of a system, and it is sometimes used to compare the software based measuring instruments to estimate their accuracy\cite{fahad2019comparative}. The hardware-based measuring instruments are usually using some electrical probe, but the exact make and model are not consistent in the literature, which can be seen in \cref{tab:Hardware_based_Measuring_instruments}.
To determine what to use in our work a MoSCoW analysis has been made below. To make the criteria knowledge gathered from \cref{ch:related_work} is used.

\begin{itemize}
    \item Must have
    \begin{itemize}
        \item Accuracy: The measurement should be as least as accurate as Watts Up Pro ($1.5\%$)\todo{cite pls}
        \item Frequency: The measurements should be as frequent as Watts Up Pro\todo{what is frequency of watts up pro}.
        \item Safety: as the measurements could involve high voltages safety is a concern.\cite{sik}.
    \end{itemize}
    \item Should have
    \begin{itemize}
        \item Ease of use: it should not be difficult to set up or obtain logging from the device, as we have limited experience with power
    \end{itemize}
    \item Could have
    \begin{itemize}
        \item Availability: It should be available to purchase for a private consumer to promote reproducibility
        \item Price: It should be affordable for a private consumer to promote reproducibility 
    \end{itemize}
    \item Wont have
    \begin{itemize}
        \item DC measuring
    \end{itemize}
\end{itemize}

The Watts Up Pro is not in production anymore thus the availability is poor. and the other devices used in \cref{ch:related_work} do not meet the criteria. The Plugwise has a reported accuracy of 5\%\feetnote{https://www.plugwise.com/product/plug/?lang=en-+}\todo{Til Lone og Bent: footnote eller cite?}, which is not accurate enough since a higher degree of accuracy is prioritized in our work. The ZES ZIMMER LMG450 is extremely accurate, but could not be used because of the steep price point of around \EUR{6800}, and this exceeds out budget. None of the devices fits the exact needs of our work, therefore a custom setup has to be made. It also has to measure in AC, because measuring the wire from the wall socket to the DUT provides measurements for the whole system, which is desired for a ground truth measurement. The setup will involve some type of electrical probe able to measure the current of the energy going into the power supply of the DUT. The probe will then be connected to some sort of logging device with the ability to provide the data in a useable format. 


\subsection{Comparison}\label{subsec:software_comparison}

After the introduction of different existing energy profilers, a comparison can be made. Firstly, when considering the four tools it can be seen that most of them provide the measurements of the same components which is ideal for comparing their results later. The most commonly used software-based measuring instrument is Intel's RAPL. However, it is limited to Linux. When considering the Windows tools, none of the tools has any credible sources documenting their accuracies. One interesting tool is E3, as it estimates the power of all processes running on the computer individually, and how a Maxim chip can, according to Microsoft, bring the accuracy up to 98\%.\cite[]{E3WinHec} E3 does however only work on devices with a battery, as opposed to Open Hardware Monitor and Intel Power Gadget. When comparing the latter two, they seem very similar, where one big advantage of Open Hardware Monitor is that it is open source, so any uncertainty could be verified by looking at the actual implementation, and the implementation can be altered to fit the purpose. The advantage of Intel's Power Gadget and Open Hardware Monitor is that they have an implemented interface, which makes it easier to access the relevant data. Opposed to this we have E3, where a log needs to be reset, and a log file needs to be parsed, which is not ideal. We have chosen to use Intel's RAPL, Microsoft's E3, Intel's Power Power Gadget and Open Hardware Monitor. 

% WHHY?

%% AMD version  AMP. Source siger bad. Hold os inden for en producent af CPU.