\section{Measuring instruments}\label{sec:measuring_instruments}

When considering \textbf{RQ2}, different measuring instruments are needed. In this section, different existing approaches will be presented, and then compared.

When selecting software-based measuring instruments there are several options which are chosen from different categories, including software manufacturers, hardware manufacture and open source. From the hardware manufacturers, Intel's RAPL and Intel's Power Gadget were chosen as their solutions with potential for hardware support could be promising. For software manufacturers, Microsoft's E3 was chosen as their domain knowledge could prove valuable for energy estimation. Lastly, an open source project is included, this being Open Hardware Monitor, to see how this compared to three, potentially more sophisticated solutions. Open Hardware Monitor was chosen as a C\# implementation was found, which made it easy to use, and based on the assumption that all open source versions will measure in the same way. As the most used approach in existing work, Intels RAPL will also be used. All the different software-based measuring instruments will be introduced now, and compared in the end.


                            \begin{figure}
                                \centering
                                \begin{tikzpicture}[]
                                    \pgfplotsset{%
                                        width=.85\textwidth,
                                        height=0.15\textheight
                                    }
                                    \begin{axis}[xlabel={Average energy (Watts)}, title={workstation - RAPL}, ytick={1, 2, 3, 4, 5, 6},
                                    yticklabels={
                                        Fasta - IPG, Fasta - LHM, Fasta - CLAMP, Fasta - IPG, Fasta - LHM, Fasta - CLAMP
                                        },
                                        xmin=0,xmax=50,
                                        ]
                                    
                                        \addplot+ [boxplot prepared={
                                        lower whisker=40.30526797215659,
                                        lower quartile=40.97327163025656,
                                        median=41.72166726874962,
                                        upper quartile=42.639938738927185,
                                        upper whisker=44.425335327916834},
                                        ] table[row sep=\\,y index=0] {\\};
                                        
                                        \addplot+ [boxplot prepared={
                                        lower whisker=40.48550261459181,
                                        lower quartile=40.831039235552794,
                                        median=41.5846158762684,
                                        upper quartile=42.207863208562216,
                                        upper whisker=43.46991921513228},
                                        ] table[row sep=\\,y index=0] {\\};
                                        
                                        \addplot+ [boxplot prepared={
                                        lower whisker=40.2697084750618,
                                        lower quartile=41.035098465232245,
                                        median=41.38869099577169,
                                        upper quartile=42.44694669562162,
                                        upper whisker=43.91629239371539},
                                        ] table[row sep=\\,y index=0] {\\};
                                        
                                        \addplot+ [boxplot prepared={
                                        lower whisker=40.758440129667676,
                                        lower quartile=41.39236498066475,
                                        median=41.773795862544134,
                                        upper quartile=42.42892778500295,
                                        upper whisker=44.19480610081178},
                                        ] table[row sep=\\,y index=0] {\\};
                                        
                                        \addplot+ [boxplot prepared={
                                        lower whisker=40.81938120830589,
                                        lower quartile=41.27530056939251,
                                        median=41.61423462219319,
                                        upper quartile=42.13919903223857,
                                        upper whisker=43.40666243037317},
                                        ] table[row sep=\\,y index=0] {\\};
                                        
                                        \addplot+ [boxplot prepared={
                                        lower whisker=40.864166136835486,
                                        lower quartile=41.51662118233571,
                                        median=41.88433916576963,
                                        upper quartile=42.86202552311207,
                                        upper whisker=45.290678405053065},
                                        ] table[row sep=\\,y index=0] {\\};
                                        
                                    \end{axis}
                                \end{tikzpicture}
                            \caption{R3 validation for energy measurements by RAPL for the Cores for DUT PowerKomplett, OS Win32NT and test case Fasta, where the impact of the first profiler can be seen (with outliers)} \label{fig:PowerKomplett_RAPL_Cores_R3_energy_with_outliers_Win32NT_avg_watts}
                            \end{figure}
                            
\subsection{Microsoft's Energy Estimation Engine (E3)}\label[subsec]{subsec:e3}
Microsoft's Energy Estimation Engine (E3) is a tool created to monitor the energy consumption of batteries on Windows devices. E3 monitors the DUT at all times to create battery energy usage reports, accessible in the power battery settings under Battery usage.\cite[p.43]{E3WinHec} The reported accuracy of E3 varies if the DUT has a special energy measurement chip or not. The reported accuracies for different compoents of the DUT is presented in \cite{E3WinHec} can be seen in \cref{tab:E3_acc_Table}.

\begin{table}[ht]
    \centering
    \begin{tabular}{|c|c|c|c|c|}
    \hline
    \textit{}    & CPU  & Storage         & Display         & Network         \\ \hline
    Without chip & 89\% & \textless{}70\% & \textless{}70\% & \textless{}70\% \\ \hline
    With chip    & 98\% & 98\%            & 98\%            & 98\%            \\ \hline
    \end{tabular}
    \caption{E3 accuracies. For devices with and without Maxim chips}
    \label{tab:E3_acc_Table}
\end{table}

In \cref{tab:E3_acc_Table} it can be observed that there is a large difference in the accuracy between devices with and without the chip on the motherboard. The chips referred to here are specifically the MAXIM MAX34407 or the MAXIM MAX34417, contained for example in different Microsoft Surface devices.\newline

% , finding the models of computers that actually contain these chips have proven difficult, with very few computer known to actually contain the chip. 

When comparing E3 to other measuring instruments like RAPL, the primary difference is how E3 is able to estimate the energy consumption for each proces running on the DUT, opposed to RAPL where only the energy consumption of the CPU and RAM are reported. This makes it easier to draw certain conclusions about energy usage from E3 if the estimations are accurate.\newline

The reports created by E3 will by default contain logs of power usage for the previous $7$ days, where each process and their power usage is logged. Each log entry accounts for $1-5$ minutes depending on the actual length of the process execution\cite[]{E3Video}. Each entry in the report contains many attributes, where some of the relevant attributes to this study can be seen in \cref{tab:E3_attr_Table}

\begin{table}[ht]
    \centering
    \begin{tabular}{||c|c||}
    \hline
    \textbf{Attribute Name}  & \textbf{Description}                            \\ [0.5ex] \hline\hline
    AppId                    & Unique id for each process and subprocess       \\ \hline
    TimeStamp                & Start time of Measurement                       \\ \hline
    TimeInMs                 & Duration of the Measurement in Ms               \\ \hline
    CPUEnergyConsumption     & Cpu energy consumption in millijoules           \\ \hline
    NetworkEnergyConsumption & Network energy consumption in millijoules       \\ \hline
    SocEnergyConsumption     & System-On-Chip energy consumption in milijoules \\ \hline
    DiskEnergyConsumption    & Disk energy consumption in millijoules          \\ \hline
    TotalEnergyConsumption   & Total energy consumption in millijoules         \\ \hline
    \end{tabular}
    \caption{The attributes in the E3 report used in this report}
    \label{tab:E3_attr_Table}
    \end{table}

As of the time of writing no study or external entity apart from Microsoft themselves has neither used nor verified the accuracy and reliability of E3 as reported by Microsoft. Because of this, the information about E3 mostly comes from blog posts or presentations by Microsoft and is very sparse in the information provided. An additional limitation is how this software only works on Windows devices with a battery, meaning that E3 is not accessible on Windows desktops.\newline

In order to generate the raw energy report from E3, navigate to the desired destination of the report in an elevated command shell and type \texttt{powercfg.exe \\srumutil}. Upon success, the output should be \texttt{Completed with status 0 (0x00000000)}.

% start normal listing
% \begin{lstlisting}[language=Python, caption=Python example]

% From file
% \lstinputlisting[language=Octave]{BitXorMatrix.m}
\section{Intel PowerGadget}

The Intel Power Gadget\cite[]{powergadget} is a software tool for measuring power consumption of Intel Core processors from the 2nd to 10th generation, for both Windows and macOS. What this software tool offers, is real-time estimations of the energy consumption in watts using the energy counters in the processors. 
The tool also contains a command line version called Powerlog, in addition to this, the newest versions also includes estimations of energy consumption on multi socket systems and externally callable APIs. These APIs can b used to extract information within sections of code. This is achieved by evaluating the energy Model Specific Registers (MSR) on a per-socket basis.

When measuring energy consumption the average power is measured in watts, cumulative energy in joules, cumulative energy in miliWatt-hours, temperature in Celsius and frequency in GHz. When considering the API, the sampling frequency can range from 1 to 1000 milliseconds. Here it is noted a high frequency will bring a greater accuracy, but poorer performance of the system, and a frequency of 100 milliseconds, which is the default value, is recommended.\cite*[]{powergadget_api}.

When using the Power Gadget API a reference to \texttt{EnergyLib64.dll} will make functions available. When using the library, it must first be initialized, by calling 

\texttt{IntelEnergyLibInitialize()}, where the drivers are loaded. In addition to this, the library also contains contains other method. Similar for all these methods is how they return a boolean, representing if the call was a success. In this study, this tool will only be used for power measuring, despite also having additional features like temperatures or frequencies of the processor. All functions can be found in the official documentation\cite*[]{powergadget_api}, where the relevant functions can be found in \cref{tab:intel_power_gadget_functions}

\begin{table}
    \centering
    \begin{tabular}{ | c | c |}
        \hline
        \thead{Function} & Description \\
        \hline
        \texttt{bool IntelEnergyLibInitialize();} &  \makecell{The library is initialized and a connection is\\made to the driver}  \\
        \hline
        \texttt{bool ReadSample();} &  \makecell{Sample data is read from all supported MSRs\\through the driver}  \\
        \hline
        \makecell{\texttt{bool GetPowerData(}\\ \texttt{int iNode, int iMSR,}\\ \texttt{double *pResult, int *nResult);}} &  \makecell{The data collected by the most recent\\\texttt{ReadSample()} is returned. The data\\returned is only for the package iNode\\specified, from the iMSR MSR specified.\\pResult represents the data returned,\\and nResults represents the number of\\double results in pResult.}  \\
        \hline
        \texttt{bool StartLog(wchart szFileName);} & \makecell{Data is collected and written to the\\file specified by calling \texttt{ReadSample()}\\until \texttt{StopLog()} }  \\
        \hline
        \texttt{bool StopLog();} &  \makecell{All the saved data is written to\\ the file specified in the \texttt{StartLog()}\\ call, and stops saving data.
        }  \\
        \hline
    \end{tabular}
    \caption{Function prototypes of relevant functions for Intel Power Gadget.}
    \label{tab:intel_power_gadget_functions}
\end{table}

% Returns the data collected by the most recent call to ReadSample(). The returned data is for the data on the package specified by iNode, from the MSR specified by iMSR. The data is returned in pResult, and the number of double results returned in pResult is returned in nResult. Refer Table 1: MSR Functions.

% The data collected by the most recent ReadSample() is returned. The data returned is only for the package iNode specified, from the iMSR MSR specified. pResult represents the data returned, and nResults represents the number of double results in pResult.

% Starts saving the data collected by ReadSample() until StopLog() is called. Data will be written to the file specified by szFileName.

% Data is collected and written to the file specified by calling \texttt{ReadSample()} until \texttt{StopLog()} 

% Stops saving data and writes all saved data to the file specified by the call to StartLog().

% All the saved data is written to the file specified in the \texttt{StartLog()} call, and stops saving data.
\section{Hardware Monitor}
\paragraph*{AC Current clamp MN60}
The current clamp was briefly covered in the \cref{clampIntro}, with some important specification of the version of the device that is used, such as its high reported accuracy and the potentially high sampling rate. 
The current clamp can measure the current passing trough the phase of a wire, which can then be used to derive the joule passing trough it. The benefits from using a current clamp as opposed to other methods of probing is that there is no exposed wiring that could accidentally be interacted with, this was a priority as mains voltage can be lethal and direct measurements could come with some risk. Though this added safety does come with certain drawbacks, current clamps are less accurate than more direct ways of measuring the power when looking at a price to accuracy ration. Another limitation with the current clamp versus other measurement equipment is that it alone cannot measure anything and needs an oscilloscope to function.
\section{Comparison}\label{sec:comparison}

In this section of the results, a comparison between the different DUTs and measuring instruments will be performed. This will be done in a way where the different measuring instruments will be compared for each DUT, before comparing all DUTs and all measuring instruments. Before this, the expectations will be presented.


\subsection{Expectations:} Based on what was seen in \cref{sec:iterations}, similar observations are expected. This will include a clamp with a high standard deviation compared to the different software measuring instruments, and cases where Intel Power Gadget and LHM measurements a close to each other. For RAPL, a low standard deviation is expected, in addition to lower measured energy consumption in most cases compared to the other measuring instruments.


\subsection{Results}
\paragraph{Workstation}


                            \begin{figure}
                                \centering
                                \begin{tikzpicture}[]
                                    \pgfplotsset{%
                                        width=.7\textwidth,
                                        height=.2\textheight
                                    }
                                    \begin{axis}[xlabel={Average energy consumption (Watts)}, title={Cores - Fasta - Energy - without outliers}, ytick={1, 2},
                                    yticklabels={
                                        IntelPowerGadget , HardwareMonitor 
                                        },
                                        xmin=0,xmax=80,
                                        ]
                                    
                                    \addplot+ [boxplot prepared={
                                    lower whisker=54.22445449466433,
                                    lower quartile=54.45751347572185,
                                    median=54.54624547297937,
                                    upper quartile=54.72020125409266,
                                    upper whisker=55.103791721157386},
                                    ] table[row sep=\\,y index=0] {\\};
                                    
                                    \addplot+ [boxplot prepared={
                                    lower whisker=51.45082608392138,
                                    lower quartile=51.933860038023106,
                                    median=52.121433545941585,
                                    upper quartile=52.479201309170854,
                                    upper whisker=54.95103920614709},
                                    ] table[row sep=\\,y index=0] {\\};
                                    
                                    \end{axis}
                                \end{tikzpicture}
                            \caption{A comparison of of Cores energy consumption for test case Fasta for the workstation (without outliers)} \label{fig:Fasta_Cores_comparison_energy_without_outliers_PowerKomplett_avg_watts_exp2}
                            \end{figure}
                            

                \begin{figure}[H]
                    \centering
                    \begin{tikzpicture}
                        \pgfplotsset{%
                            width=1\textwidth,
                            height=0.4\textheight
                        }
                        \begin{axis}[
                            xlabel={Start battery level},
                            ylabel={Average dynamic energy (watt)},
                            ymin=0,ymax=20,
                        ]
                        
                            \addplot [mark=none, ultra thick, red]  coordinates {
                            (40, 0.006595684402444291)(45, 0.007295220674193181)(50, 0.007999659608910881)(55, 0.007202467971243639)(60, 0.007280865079495958)(65, 0.006858423509955077)(70, 0.008369444141141925)(75, 0.007144940647010992)(80, 0.004753236834232667)
                            };
                            \addlegendentry{Surface4Pro - IntelPowerGadget}
                            
                            \addplot [mark=none, ultra thick, blue]  coordinates {
                            (40, 0.002385328427460912)(45, 0.0017856511573015649)(50, 0.0025901992189954056)(55, 0.00210998144366709)(60, 0.00286452646862575)(65, 0.0020038487280194628)(70, 0.002908483770774353)(75, 0.0005098111150931342)(80, -0.005995916826693459)
                            };
                            \addlegendentry{Surface4Pro - HardwareMonitor}
                            
                            \addplot [mark=none, ultra thick, orange]  coordinates {
                            (50, 251.8661014299577)(55, 214.7597259251014)(60, 162.23880995564176)(65, 107.43421593849766)(70, 53.9187387386668)(75, 0.7554852536149699)(80, -44.463085003568885)
                            };
                            \addlegendentry{Surface4Pro - RAPL}
                            
                            \addplot [mark=none, dashdotted, red]  coordinates {
                            (40, -0.004038062354887025)(45, -0.004312668500277529)(50, -0.003808663911021498)(55, -0.0037057407527755254)(60, -0.004478257932982471)(65, -0.0026308734995501644)(70, -0.0034090674446925874)(75, -0.003041497079697436)(80, -0.0020334307266356875)
                            };
                            \addlegendentry{SurfaceBook - IntelPowerGadget}
                            
                            \addplot [mark=none, dashdotted, blue]  coordinates {
                            (40, -0.002443518930616523)(45, -0.0029256880137447055)(50, -0.002551777773312929)(55, -0.0027567211782433486)(60, -0.002206859154402231)(65, -0.0026388300848279207)(70, -0.0024597736945479324)(75, -0.002445350799195827)(80, -0.000541271265011134)
                            };
                            \addlegendentry{SurfaceBook - HardwareMonitor}
                            
                            \addplot [mark=none, dashdotted, orange]  coordinates {
                            (40, 101.92702155010147)(45, 86.7212700795567)(50, 69.40754598562454)(55, 51.343785407669614)(60, 32.443112755251185)(65, 14.52577919077786)(70, -4.520517378551423)(75, -23.811706977384954)(80, -35.700372165212755)
                            };
                            \addlegendentry{SurfaceBook - RAPL}
                            
                        \end{axis}
                    \end{tikzpicture} 
                \caption{A graph illustrating the energy consumption of Dram for test case Nbody with regards to the battey level of the DUT (with outliers)} \label{fig:Nbody_Dram_charge}
                \end{figure}
                

The first DUT to consider is the workstation. For this DUT, Fasta and NBody measurements can be seen in \cref{fig:Fasta_Cores_comparison_dynamic_energy_without_outliers_PowerKomplett_avg_watts} and \cref{fig:Nbody_Cores_comparison_dynamic_energy_without_outliers_PowerKomplett_avg_watts} respectively, and the rest can be found in \cref{app:comparison_workstation}. When comparing the software measuring instruments, Intel Power Gadget and LHM measure close to each other in most cases. This includes BinaryTrees, Fasta and NBody where for FannkuchRedux a bigger difference can be observed. When considering RAPL compare to this, RAPL will in all test cases except NBody measure an energy consumption lower than its windows equivalents. When comparing the software measuring instruments against the hardware measurements, RAPL will against the clamp on Linux in all cases except NBody get a measurement higher than the clamp. For Windows, both Intel Power Gadget and LHM will get a measurement higher than the clamp.

\paragraph{Surface Pro 4}


                            \begin{figure}
                                \centering
                                \begin{tikzpicture}[]
                                    \pgfplotsset{%
                                        width=.7\textwidth,
                                        height=.2\textheight
                                    }
                                    \begin{axis}[xlabel={Average energy consumption (Watts)}, title={Cores - Fasta - Energy - without outliers}, ytick={1, 2},
                                    yticklabels={
                                        IntelPowerGadget , HardwareMonitor 
                                        },
                                        xmin=0,xmax=80,
                                        ]
                                    
                                    \addplot+ [boxplot prepared={
                                    lower whisker=54.22445449466433,
                                    lower quartile=54.45751347572185,
                                    median=54.54624547297937,
                                    upper quartile=54.72020125409266,
                                    upper whisker=55.103791721157386},
                                    ] table[row sep=\\,y index=0] {\\};
                                    
                                    \addplot+ [boxplot prepared={
                                    lower whisker=51.45082608392138,
                                    lower quartile=51.933860038023106,
                                    median=52.121433545941585,
                                    upper quartile=52.479201309170854,
                                    upper whisker=54.95103920614709},
                                    ] table[row sep=\\,y index=0] {\\};
                                    
                                    \end{axis}
                                \end{tikzpicture}
                            \caption{A comparison of of Cores energy consumption for test case Fasta for the workstation (without outliers)} \label{fig:Fasta_Cores_comparison_energy_without_outliers_PowerKomplett_avg_watts_exp2}
                            \end{figure}
                            

                            \begin{figure}
                                \centering
                                \begin{tikzpicture}[]
                                    \pgfplotsset{%
                                        width=.85\textwidth,
                                        height=.15\textheight
                                    }
                                    \begin{axis}[xlabel={Average energy consumption (Watts)}, title={Cores - FannkuchRedux - Energy - without outliers}, ytick={},
                                    yticklabels={
                                        
                                        },
                                        xmin=0,xmax=20,
                                        ]
                                    
                                    \end{axis}
                                \end{tikzpicture}
                            \caption{A comparison of of Cores energy consumption for test case FannkuchRedux for the Surface4Pro,  experiment \#2 (without outliers)} \label{fig:FannkuchRedux_Cores_comparison_energy_without_outliers_Surface4Pro_avg_watts_exp2}
                            \end{figure}
                            

The next DUT is the Surface Pro 4, where FannkuchRedux and Fasta are illustrated in \cref{fig:FannkuchRedux_Cores_comparison_dynamic_energy_without_outliers_Surface4Pro_avg_watts} and \cref{fig:Fasta_Cores_comparison_dynamic_energy_without_outliers_Surface4Pro_avg_watts} respectively, and the rest can be found in \cref{app:comparison_surfacepro4}. On the Surface Pro 4, a few things can be observed. When comparing the standard deviation, Intel Power Gadget deviates the most, which for example can be observed in the FannkuchRedux in \cref{fig:FannkuchRedux_Cores_comparison_dynamic_energy_without_outliers_Surface4Pro_avg_watts}. Intel Power Gadgets measurements are close the those made by LHM. When comparing RAPL against the measuring instruments on Windows, the difference changes from test case to test case,  where the measurements differ by $~6$W in FannkuchRedux in \cref{fig:FannkuchRedux_Cores_comparison_dynamic_energy_without_outliers_Surface4Pro_avg_watts} and $~1$W in Fasta in \cref{fig:Fasta_Cores_comparison_dynamic_energy_without_outliers_Surface4Pro_avg_watts}.

\paragraph{Surface Book}


                            \begin{figure}
                                \centering
                                \begin{tikzpicture}[]
                                    \pgfplotsset{%
                                        width=.85\textwidth,
                                        height=.15\textheight
                                    }
                                    \begin{axis}[xlabel={Average energy consumption (Watts)}, title={Cores - BinaryTrees - Energy - without outliers}, ytick={},
                                    yticklabels={
                                        
                                        },
                                        xmin=0,xmax=20,
                                        ]
                                    
                                    \end{axis}
                                \end{tikzpicture}
                            \caption{A comparison of of Cores energy consumption for test case BinaryTrees for the Surface4Pro,  experiment \#2 (without outliers)} \label{fig:BinaryTrees_Cores_comparison_energy_without_outliers_Surface4Pro_avg_watts_exp2}
                            \end{figure}
                            

                \begin{figure}[H]
                    \centering
                    \begin{tikzpicture}
                        \pgfplotsset{%
                            width=1\textwidth,
                            height=0.4\textheight
                        }
                        \begin{axis}[
                            xlabel={Start battery level},
                            ylabel={Average dynamic energy (watt)},
                            ymin=0,ymax=20,
                        ]
                        
                            \addplot [mark=none, ultra thick, red]  coordinates {
                            (40, 0.006595684402444291)(45, 0.007295220674193181)(50, 0.007999659608910881)(55, 0.007202467971243639)(60, 0.007280865079495958)(65, 0.006858423509955077)(70, 0.008369444141141925)(75, 0.007144940647010992)(80, 0.004753236834232667)
                            };
                            \addlegendentry{Surface4Pro - IntelPowerGadget}
                            
                            \addplot [mark=none, ultra thick, blue]  coordinates {
                            (40, 0.002385328427460912)(45, 0.0017856511573015649)(50, 0.0025901992189954056)(55, 0.00210998144366709)(60, 0.00286452646862575)(65, 0.0020038487280194628)(70, 0.002908483770774353)(75, 0.0005098111150931342)(80, -0.005995916826693459)
                            };
                            \addlegendentry{Surface4Pro - HardwareMonitor}
                            
                            \addplot [mark=none, ultra thick, orange]  coordinates {
                            (50, 251.8661014299577)(55, 214.7597259251014)(60, 162.23880995564176)(65, 107.43421593849766)(70, 53.9187387386668)(75, 0.7554852536149699)(80, -44.463085003568885)
                            };
                            \addlegendentry{Surface4Pro - RAPL}
                            
                            \addplot [mark=none, dashdotted, red]  coordinates {
                            (40, -0.004038062354887025)(45, -0.004312668500277529)(50, -0.003808663911021498)(55, -0.0037057407527755254)(60, -0.004478257932982471)(65, -0.0026308734995501644)(70, -0.0034090674446925874)(75, -0.003041497079697436)(80, -0.0020334307266356875)
                            };
                            \addlegendentry{SurfaceBook - IntelPowerGadget}
                            
                            \addplot [mark=none, dashdotted, blue]  coordinates {
                            (40, -0.002443518930616523)(45, -0.0029256880137447055)(50, -0.002551777773312929)(55, -0.0027567211782433486)(60, -0.002206859154402231)(65, -0.0026388300848279207)(70, -0.0024597736945479324)(75, -0.002445350799195827)(80, -0.000541271265011134)
                            };
                            \addlegendentry{SurfaceBook - HardwareMonitor}
                            
                            \addplot [mark=none, dashdotted, orange]  coordinates {
                            (40, 101.92702155010147)(45, 86.7212700795567)(50, 69.40754598562454)(55, 51.343785407669614)(60, 32.443112755251185)(65, 14.52577919077786)(70, -4.520517378551423)(75, -23.811706977384954)(80, -35.700372165212755)
                            };
                            \addlegendentry{SurfaceBook - RAPL}
                            
                        \end{axis}
                    \end{tikzpicture} 
                \caption{A graph illustrating the energy consumption of Dram for test case Nbody with regards to the battey level of the DUT (with outliers)} \label{fig:Nbody_Dram_charge}
                \end{figure}
                

For the Surface Book, test case BinaryTrees and Nbody can be seen in \cref{fig:BinaryTrees_Cores_comparison_dynamic_energy_without_outliers_PowerKomplett_avg_watts} and \cref{fig:Nbody_Cores_comparison_dynamic_energy_without_outliers_PowerKomplett_avg_watts} respectively, where the other test cases can be found in \cref{app:comparison_surfacebook}. On the Surface Book, the patterns are not as clear as they were on the Surface Pro 4. This is first of all because of the increased uncertainty in some of the results on the software measuring instruments on Windows, like for BinaryTrees in \cref{fig:BinaryTrees_Cores_comparison_dynamic_energy_without_outliers_PowerKomplett_avg_watts}. In one case, the difference between the median value for Intel Power Gadget and LHM stands out, with a difference of $~2$W for Nbody in \cref{fig:Nbody_Cores_comparison_dynamic_energy_without_outliers_PowerKomplett_avg_watts}. When comparing RAPL against Windows measurements, RAPL measures the highest in all cases except for FannkuchRedux.

\subsection{DUT and Measuring Instrument}


                            \begin{figure}
                                \centering
                                \begin{tikzpicture}[]
                                    \pgfplotsset{%
                                        width=.7\textwidth,
                                        height=.2\textheight
                                    }
                                    \begin{axis}[xlabel={Average energy consumption (Watts)}, title={Cores - Fasta - Energy - without outliers}, ytick={1, 2},
                                    yticklabels={
                                        IntelPowerGadget , HardwareMonitor 
                                        },
                                        xmin=0,xmax=80,
                                        ]
                                    
                                    \addplot+ [boxplot prepared={
                                    lower whisker=54.22445449466433,
                                    lower quartile=54.45751347572185,
                                    median=54.54624547297937,
                                    upper quartile=54.72020125409266,
                                    upper whisker=55.103791721157386},
                                    ] table[row sep=\\,y index=0] {\\};
                                    
                                    \addplot+ [boxplot prepared={
                                    lower whisker=51.45082608392138,
                                    lower quartile=51.933860038023106,
                                    median=52.121433545941585,
                                    upper quartile=52.479201309170854,
                                    upper whisker=54.95103920614709},
                                    ] table[row sep=\\,y index=0] {\\};
                                    
                                    \end{axis}
                                \end{tikzpicture}
                            \caption{A comparison of of Cores energy consumption for test case Fasta for the workstation (without outliers)} \label{fig:Fasta_Cores_comparison_energy_without_outliers_PowerKomplett_avg_watts_exp2}
                            \end{figure}
                            

                \begin{figure}[H]
                    \centering
                    \begin{tikzpicture}
                        \pgfplotsset{%
                            width=1\textwidth,
                            height=0.4\textheight
                        }
                        \begin{axis}[
                            xlabel={Start battery level},
                            ylabel={Average dynamic energy (watt)},
                            ymin=0,ymax=20,
                        ]
                        
                            \addplot [mark=none, ultra thick, red]  coordinates {
                            (40, 0.006595684402444291)(45, 0.007295220674193181)(50, 0.007999659608910881)(55, 0.007202467971243639)(60, 0.007280865079495958)(65, 0.006858423509955077)(70, 0.008369444141141925)(75, 0.007144940647010992)(80, 0.004753236834232667)
                            };
                            \addlegendentry{Surface4Pro - IntelPowerGadget}
                            
                            \addplot [mark=none, ultra thick, blue]  coordinates {
                            (40, 0.002385328427460912)(45, 0.0017856511573015649)(50, 0.0025901992189954056)(55, 0.00210998144366709)(60, 0.00286452646862575)(65, 0.0020038487280194628)(70, 0.002908483770774353)(75, 0.0005098111150931342)(80, -0.005995916826693459)
                            };
                            \addlegendentry{Surface4Pro - HardwareMonitor}
                            
                            \addplot [mark=none, ultra thick, orange]  coordinates {
                            (50, 251.8661014299577)(55, 214.7597259251014)(60, 162.23880995564176)(65, 107.43421593849766)(70, 53.9187387386668)(75, 0.7554852536149699)(80, -44.463085003568885)
                            };
                            \addlegendentry{Surface4Pro - RAPL}
                            
                            \addplot [mark=none, dashdotted, red]  coordinates {
                            (40, -0.004038062354887025)(45, -0.004312668500277529)(50, -0.003808663911021498)(55, -0.0037057407527755254)(60, -0.004478257932982471)(65, -0.0026308734995501644)(70, -0.0034090674446925874)(75, -0.003041497079697436)(80, -0.0020334307266356875)
                            };
                            \addlegendentry{SurfaceBook - IntelPowerGadget}
                            
                            \addplot [mark=none, dashdotted, blue]  coordinates {
                            (40, -0.002443518930616523)(45, -0.0029256880137447055)(50, -0.002551777773312929)(55, -0.0027567211782433486)(60, -0.002206859154402231)(65, -0.0026388300848279207)(70, -0.0024597736945479324)(75, -0.002445350799195827)(80, -0.000541271265011134)
                            };
                            \addlegendentry{SurfaceBook - HardwareMonitor}
                            
                            \addplot [mark=none, dashdotted, orange]  coordinates {
                            (40, 101.92702155010147)(45, 86.7212700795567)(50, 69.40754598562454)(55, 51.343785407669614)(60, 32.443112755251185)(65, 14.52577919077786)(70, -4.520517378551423)(75, -23.811706977384954)(80, -35.700372165212755)
                            };
                            \addlegendentry{SurfaceBook - RAPL}
                            
                        \end{axis}
                    \end{tikzpicture} 
                \caption{A graph illustrating the energy consumption of Dram for test case Nbody with regards to the battey level of the DUT (with outliers)} \label{fig:Nbody_Dram_charge}
                \end{figure}
                

The last comparison in this experiment is between all DUTs and measuring instruments. The test cases Fasta and Nbody are used, where the rest can be found in \cref{app:comparison}. When comparing the different DUTs, some of the same tendencies can be observed. This could be the similarity between Intel Power Gadget and LHM in most cases, with a few outliers like Nbody on Surface Book in \cref{fig:Nbody_Cores_comparison_dynamic_energy_without_outliers_avg_watts}. An outlier like this, is however not consistent for the DUT, meaning, the Intel Power Gadget and LHM measurements for the workstation and Surface Pro 4 for Nbody are still similar. Another thing to consider is the correlation between when RAPL measurements are either higher/lower compared to the Windows measurements across the same DUT. Here it is however more difficult to find a pattern. In most cases, RAPL will measure a lower value across all measuring instruments for all DUTS, with some exceptions. Exceptions include Surface Book on Fasta in \cref{fig:Fasta_Cores_comparison_dynamic_energy_without_outliers_avg_watts}, where the RAPL measurement is higher. For the other two DUT's, RAPL will however measure lower energy consumption. Another example is the energy consumption for the test case Nbody in \cref{fig:Nbody_Cores_comparison_dynamic_energy_without_outliers_avg_watts}, where both the Surface Book and the Workstation, RAPL reports a higher energy consumption, this is however not the case for the Surface Pro 4.

