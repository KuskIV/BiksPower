\subsection{Microsoft's Energy Estimation Engine (E3)}\label[subsec]{subsec:e3}
Microsoft's Energy Estimation Engine (E3) is a tool created to monitor the energy consumption of batteries on Windows devices. E3 monitors the DUT at all times to create battery energy usage reports, accessible in the power battery settings under Battery usage.\cite[p.43]{E3WinHec} The reported accuracy of E3 varies depending on whether the DUT has a special energy measurement chip or not. The reported accuracies are presented in \cite{E3WinHec} can be seen in \cref{tab:E3_acc_Table}.

\begin{table}[ht]
    \centering
    \begin{tabular}{|c|c|c|c|c|}
    \hline
    \textit{}    & CPU  & Storage         & Display         & Network         \\ \hline
    Without chip & 89\% & \textless{}70\% & \textless{}70\% & \textless{}70\% \\ \hline
    With chip    & 98\% & 98\%            & 98\%            & 98\%            \\ \hline
    \end{tabular}
    \caption{E3 accuracies. For devices with and without Maxim chips}
    \label{tab:E3_acc_Table}
\end{table}

In \cref{tab:E3_acc_Table} it can be observed that there is a large difference in the accuracy between devices with and without the chip on the motherboard. The chips referred to here are specifically the MAXIM MAX34407 or the MAXIM MAX34417, present in for example different Microsoft Surface devices.\newline

% , finding the models of computers that actually contain these chips have proven difficult, with very few computer known to actually contain the chip. 

When comparing E3 to other measuring instruments like RAPL, the primary difference is how E3 is able to estimate the energy consumption for each process running on the DUT, as opposed to RAPL where only the energy consumption of the CPU and RAM are reported. This makes it easier to draw certain conclusions about energy usage from E3 given that the estimations are accurate.\newline

The reports created by E3 will by default contain logs of power usage for the previous $7$ days, where each process and their power usage is logged. Each log entry accounts for $1-5$ minutes depending on the actual length of the process execution\cite[]{E3Video}. Each entry in the report contains many attributes, where some of the relevant attributes to this study can be seen in \cref{tab:E3_attr_Table}

\begin{table}[ht]
    \centering
    \begin{tabular}{||c|c||}
    \hline
    \textbf{Attribute Name}  & \textbf{Description}                            \\ [0.5ex] \hline\hline
    AppId                    & Unique id for each process and subprocess       \\ \hline
    TimeStamp                & Start time of Measurement                       \\ \hline
    TimeInMs                 & Duration of the Measurement in Ms               \\ \hline
    CPUEnergyConsumption     & Cpu energy consumption in millijoules           \\ \hline
    NetworkEnergyConsumption & Network energy consumption in millijoules       \\ \hline
    SocEnergyConsumption     & System-On-Chip energy consumption in milijoules \\ \hline
    DiskEnergyConsumption    & Disk energy consumption in millijoules          \\ \hline
    TotalEnergyConsumption   & Total energy consumption in millijoules         \\ \hline
    \end{tabular}
    \caption{The attributes in the E3 report used in this report}
    \label{tab:E3_attr_Table}
    \end{table}

As of the time of writing to the extent of our knowledge, no study or external entity apart from Microsoft themselves has neither used nor verified the accuracy and reliability of E3 as reported by Microsoft. Because of this, the information about E3 mostly comes from blog posts and presentations by Microsoft and the information is therefore very sparse. An additional limitation is how this software only works on Windows devices with a battery, meaning that E3 is not accessible on Windows desktops.\newline

In order to generate the raw energy report from E3, navigate to the desired destination of the report in an elevated command shell and type \texttt{powercfg.exe \\srumutil}. Upon success, the output should be \texttt{Completed with status 0 (0x00000000)}.

% start normal listing
% \begin{lstlisting}[language=Python, caption=Python example]

% From file
% \lstinputlisting[language=Octave]{BitXorMatrix.m}