\subsection{Comparison}\label{subsec:software_comparison}

After the introduction of different existing energy profilers, a comparison can be made. Firstly, when considering the four tools it can be seen that most of them provide the measurements of the same components which is ideal for comparing their results later. The most commonly used software-based measuring instrument is RAPL. However, it is limited to Linux. When considering the Windows tools, none of the tools has any credible sources documenting their accuracies. One interesting tool is E3, as it estimates the power of all processes running on the computer individually, and how a Maxim chip can, according to Microsoft, bring the accuracy up to 98\%.\cite[]{E3WinHec} E3 does however only work on devices with a battery, as opposed to LHM and Intel Power Gadget. When comparing the latter two, they seem very similar, where one big advantage of LHM is that it is open source, so any uncertainty could be verified by looking at the actual implementation, and the implementation can be altered to fit the purpose. The advantage of Intel's Power Gadget and LHM is that they have an implemented interface, which makes it easier to access the relevant data. Opposed to this we have E3, where a log needs to be reset, and a log file needs to be parsed, which is not ideal. We have chosen to use RAPL, Microsoft's E3, Intel's Power Power Gadget and LHM. 

% WHHY?

%% AMD version  AMP. Source siger bad. Hold os inden for en producent af CPU.