\subsection{Comparison}\label{subsec:software_comparison}

After the introduction to different existing energy profilers, a comparison can be made. Firstly, when considering the four tools it can be seen that most of them provide the measurements of the  same components which is ideal for comparing their results later. Here it is noted that the one tool used in existing studies is Intel's RAPL, which gives it some credibility in regards to accuracy and reliability. The downside with RAPL is however that it is limited to Linux. When considering the Windows tools, none of the tools have any credible sources vouching for their accuracies. One interesting tool is E3, as it estimates the power of all processes running on the computer individually, and how a Maxim chip can, according to Microsoft, bring the accuracy up to 98\%.\cite[]{E3WinHec} E3 does however only work on devices with battery, opposed to Hardware Monitor and Intel Power Gadget. When comparing the latter two, they seem very similar, where one big advantage of Hardware Monitor is that it is open source, so any uncertainty could be verified by looking in the actual implementations, and the implementation can be altered to fit the purpose. Advantages of Intel's Power Gadget and Hardware Monitor is how they have an implemented interface, which makes it easier to access the relevant data. Opposed to this we have E3, where a log needs to be reset, and a log file needs to be parsed, which is not ideal.