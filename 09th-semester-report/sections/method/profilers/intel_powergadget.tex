\subsection{Intel PowerGadget}

The Intel Power Gadget\cite[]{powergadget} is a software tool for estimating the power consumption of Intel Core processors from the 2nd to 10th generation, for both Windows and macOS. What this software tool offers, is real-time estimations of the energy consumption in watts using the energy counters in the processors. 
The tool also contains a command line version called Powerlog and estimations of energy consumption on multi-socket systems and externally callable APIs.\todo[]{Check if this sentence means what we think it does} The API can be used to extract information within sections of code. This is achieved by evaluating the energy Model Specific Registers (MSR) on a per-socket basis.

When measuring energy consumption the average power is measured in watts, cumulative energy in joules, cumulative energy in milliwatt-hours, temperature in Celsius and frequency in GHz. When considering the API, the sampling frequency can range from 1 to 1000 milliseconds. Here it is noted a high frequency will bring greater accuracy but worsen the performance of the system, and a frequency of 100 milliseconds, which is the default value, is recommended.\cite[]{powergadget_api}.

When using the Power Gadget API a reference to \texttt{EnergyLib64.dll} will make functions available. When using the library, it must first be initialized, by calling 

\texttt{IntelEnergyLibInitialize()}, where the drivers are loaded. In addition to this, the library also contains other methods. Similar to all these methods is how they return a boolean, representing if the call was a success. In this study, this tool will only be used for power measuring, despite also having additional features like temperatures or frequency measurements of the processor. All functions can be found in the official documentation\cite[]{powergadget_api}, where the relevant functions can be found in \cref{tab:intel_power_gadget_functions}.

Intel Power Gadget and its API have so far been used in studies\cite[]{Bruce2015ReducingEC, Ozturk2019, Unlu2021}, but, as is noted in the studies, no study exists looking at the accuracy of the tool. In addition to this, it is also noted that the tool only measures the energy consumption of the CPU and not the GPU.

\begin{table}
    \centering
    \begin{tabular}{ | c | c |}
        \hline
        \thead{Function} & Description \\
        \hline
        \texttt{bool IntelEnergyLibInitialize();} &  \makecell{The library is initialized and a connection is\\made to the driver}  \\
        \hline
        \texttt{bool ReadSample();} &  \makecell{Sample data is read from all supported MSRs\\through the driver}  \\
        \hline
        \makecell{\texttt{bool GetPowerData(}\\ \texttt{int iNode, int iMSR,}\\ \texttt{double *pResult, int *nResult);}} &  \makecell{The data collected by the most recent\\\texttt{ReadSample()} is returned. The data\\returned is only for the package iNode\\specified, from the iMSR MSR specified.\\pResult represents the data returned,\\and nResults represents the number of\\double results in pResult.}  \\
        \hline
        \texttt{bool StartLog(wchart szFileName);} & \makecell{Data is collected and written to the\\file specified by calling \texttt{ReadSample()}\\until \texttt{StopLog()} }  \\
        \hline
        \texttt{bool StopLog();} &  \makecell{All the saved data is written to\\ the file specified in the \texttt{StartLog()}\\ call, and stops saving data.
        }  \\
        \hline
    \end{tabular}
    \caption{Function prototypes of relevant functions for Intel Power Gadget.}
    \label{tab:intel_power_gadget_functions}
\end{table}

% Returns the data collected by the most recent call to ReadSample(). The returned data is for the data on the package specified by iNode, from the MSR specified by iMSR. The data is returned in pResult, and the number of double results returned in pResult is returned in nResult. Refer Table 1: MSR Functions.

% The data collected by the most recent ReadSample() is returned. The data returned is only for the package iNode specified, from the iMSR MSR specified. pResult represents the data returned, and nResults represents the number of double results in pResult.

% Starts saving the data collected by ReadSample() until StopLog() is called. Data will be written to the file specified by szFileName.

% Data is collected and written to the file specified by calling \texttt{ReadSample()} until \texttt{StopLog()} 

% Stops saving data and writes all saved data to the file specified by the call to StartLog().

% All the saved data is written to the file specified in the \texttt{StartLog()} call, and stops saving data.