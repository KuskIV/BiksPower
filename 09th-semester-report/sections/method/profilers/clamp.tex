\subsection{Hardware current measurements} \label[]{clampIntro}
This section will cover the tool used to conduct hardware measurements and how to use it. For the measurements, an AC clamp was used specifically the MN60 created by Chauvin Arnoux. The current clamp is an analogue device meaning it does not use a digital signal but instead outputs an analogue signal which has a direct relation to the measurements. To read the signal the current clamp has to be connected to an oscilloscope, and the clamp itself is connected to the electrical phase of the wire leading to the power supply for the computer. An illustration of the proposed usage of the current clamp together with an oscilloscope can be seen in \cref{fig:clampSetup}
\begin{figure}[h!]
    \centering
    \includegraphics*[scale=0.4]{figures/CLAMP.png}
    \caption{Here the MN60 current clamp can be seen connected onto the phase of the wire into the computer}
    \label{fig:clampSetup}
\end{figure}
The output from the current is in mv that then has a specific conversion rate to a current root mean square (rms) value which together with the wall socket specification, this being 230 volts in the EU, can be used to calculate the exact amount of joules flowing through the wire at any point in time. The sampling frequency of the current clamps is depended on the oscilloscope used, the accuracy of the clamp is claimed to be $<1\%$ \cite{ClampDoc}, which is within the expected range for hardware measurements used in other studies such as the HCLWattsUp with an accuracy of $<1.5\%$, but our setups potential in terms of measurement frequency is higher.