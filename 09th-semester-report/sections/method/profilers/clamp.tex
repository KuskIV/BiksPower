\subsection{Hardware measurements}\label{sec:clampIntro}
Hardware-based measuring instruments are often from the literature seen as the most accurate way to measure the energy consumption of a system, and it is sometimes used to compare the software based measuring instruments to estimate their accuracy\cite{fahad2019comparative}. The hardware-based measuring instruments are usually using some electrical probe, but the exact make and model are not consistent in the literature, which can be seen in \cref{tab:Hardware_based_Measuring_instruments}.
To determine what to use in our work a MoSCoW analysis has been made below. To make the criteria knowledge gathered from \cref{ch:related_work} is used.

\begin{itemize}
    \item Must have
    \begin{itemize}
        \item Accuracy: The measurement should be as least as accurate as Watts Up Pro ($1.5\%$)\todo{cite pls}
        \item Frequency: The measurements should be as frequent as Watts Up Pro\todo{what is frequency of watts up pro}.
        \item Safety: as the measurements could involve high voltages safety is a concern.\cite{sik}.
    \end{itemize}
    \item Should have
    \begin{itemize}
        \item Ease of use: it should not be difficult to set up or obtain logging from the device, as we have limited experience with power
    \end{itemize}
    \item Could have
    \begin{itemize}
        \item Availability: It should be available to purchase for a private consumer to promote reproducibility
        \item Price: It should be affordable for a private consumer to promote reproducibility 
    \end{itemize}
    \item Wont have
    \begin{itemize}
        \item DC measuring
    \end{itemize}
\end{itemize}

The Watts Up Pro is not in production anymore thus the availability is poor. and the other devices used in \cref{ch:related_work} do not meet the criteria. The Plugwise has a reported accuracy of 5\%\feetnote{https://www.plugwise.com/product/plug/?lang=en-+}\todo{Til Lone og Bent: footnote eller cite?}, which is not accurate enough since a higher degree of accuracy is prioritized in our work. The ZES ZIMMER LMG450 is extremely accurate, but could not be used because of the steep price point of around \EUR{6800}, and this exceeds out budget. None of the devices fits the exact needs of our work, therefore a custom setup has to be made. It also has to measure in AC, because measuring the wire from the wall socket to the DUT provides measurements for the whole system, which is desired for a ground truth measurement. The setup will involve some type of electrical probe able to measure the current of the energy going into the power supply of the DUT. The probe will then be connected to some sort of logging device with the ability to provide the data in a useable format. 

