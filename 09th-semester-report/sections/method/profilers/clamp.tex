\subsection{Hardware measurements} \label{sec:clampIntro}
Hardware measurements are often from the literature seen as the most accurate way to measure the energy consumption of a system, it is sometimes used to compare the software measurements to access their accuracy\cite{fahad2019comparative}. The hardware measurements are usually using some electrical probe, but the exact make and type are not consistent in the literature, which can be seen in \cref{tab:Hardware_based_Measuring_instruments}.
For what would be best for this project certain criteria and priorities for the chosen measurement instrument would have to be made, by looking at the instruments used in similar studies.
\begin{itemize}
    \item Availability, should be easy to purchase and receive to promote reproducibility.
    \item Price, high end power measurement instruments can quickly become very expensive.
    \item Accuracy, the measurements should be accurate.
    \item Frequency, the measurements should be taken several times a second.
    \item Safety, as the measurements could involved high voltages safety should be a concern.
    \item Ease of use, it should not be difficult to setup or obtain logging from the device.
\end{itemize}
The devices used in related works does not meet the criteria. The Watt Up Pro is not in production anymore thus the availability is poor. The plugwise have a reported 5\% accuracy which is not particularly accurate when compared to other devices. The ZES ZIMMER LMG450 was extremely accurate, but could not be used because of the steep price point of around 6800 euro. None of the devices fitted the exact need of the project, because of that a custom setup would have to be made. 

The setup will involve some type of electrical probe that measures the current of the energy going into the power supply of the DUT. The probe will then be connected to some sort of logging device that can provide the data in useable format. 

