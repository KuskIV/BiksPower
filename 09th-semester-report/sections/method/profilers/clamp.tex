\subsection{Hardware measurements} \label{sec:clampIntro}
Hardware measurements are often from the literature seen as the most accurate way to measure the energy consumption of a system, it is sometimes used to compare the software measurements to access their accuracy\cite{fahad2019comparative}. The hardware measurements are usually using some electrical probe, but the exact make and model are not consistent in the literature, which can be seen in \cref{tab:Hardware_based_Measuring_instruments}.
To determine what to use in our work several criteria and aspects to prioritize has to be made. To make the criteria knowledge gathered from related work is used.
\begin{itemize}
    \item Availability: it should be easy to purchase and receive to promote reproducibility.
    \item Price: high-end power measurement instruments can quickly become very expensive, but our budget is limited.
    \item Accuracy: the measurements should be accurate.\todo{too vague}
    \item Frequency: the measurements should be taken several times a second.\todo{too vague}
    \item Safety: as the measurements could involve high voltages safety should be a concern \cite{sik}.
    \item Ease of use: it should not be difficult to set up or obtain logging from the device.
\end{itemize}
The devices used in \cref{ch:related_work} do not meet the criteria. The Watt Up Pro is not in production anymore thus the availability is poor. The Plugwise has a reported accuracy of 5\%, which is not high enough since a higher degree of accuracy is prioritized in our work. The ZES ZIMMER LMG450 is extremely accurate, but could not be used because of the steep price point of around \EUR{6800}. None of the devices fits the exact needs of our work, therefore a custom setup has to be made. It also has to measure in AC, because measuring the wire from the wall socket to the DUT provides measurements for the whole system, which is desired for a ground truth measurement. The setup will involve some type of electrical probe that measures the current of the energy going into the power supply of the DUT. The probe will then be connected to some sort of logging device that can provide the data in a useable format. 

