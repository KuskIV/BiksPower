\subsection{Measurement Methodology}
When conducting the experiments three different kinds of measuring approach are being utilized from the tools in \cref{sec:hardware}.
The hardware measurements are system-level physical measurements, while RAPL, OpenHardwareMonitor and sometimes E3 are measurement using on-chip sensor's. The final approach are Energy predictive models with are E3, when the power meter chip is not available. The most accurate of these is the system-level physical measurement which can provide highly accurate measurements for the systems energy consumption, but these measurements are for the system, so it cannot provide a fine grained view of an applications power consumption. All of the tools used except for E3 does not provide the fine grained measurements for the applications energy consumption. To achieve these anyway dynamic energy consumption is calculated, this is an estimation of the applications power consumption. To Calculate the dynamic energy consumption $E_D$, as presented by Fahad et al \cite{fahad2019comparative}, the following formula can be used:
$$E_D = E_T -(P_S * T_E)$$
$E_T$ is the total amount of energy consumption by the system when running the experiment. $T_E$ is the duration of the program execution. $P_S$ is the amount of energy consumption when the system is idle.