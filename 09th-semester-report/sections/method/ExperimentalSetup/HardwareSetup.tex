\subsection{Measurement Methodology}
When conducting the experiments two different kinds of measuring approach are being utilized from the tools in \cref{sec:hardware}.
The hardware measurements are system-level physical measurements, while RAPL, OpenHardwareMonitor and sometimes E3 are measurement using on-chip sensor's. The final approach are Energy predictive models which are E3, when the power meter chip is not available. The most accurate of these is the system-level physical measurement which can provide highly accurate measurements for the systems energy consumption, but these measurements are for the system, so it cannot provide a fine grained view of an applications power consumption. All of the tools used except for E3 does not provide the fine grained measurements for the applications energy consumption. To achieve these anyway dynamic energy consumption is calculated, this is an estimation of the applications power consumption. To Calculate the dynamic energy consumption $E_D$, as presented by Fahad et al \cite{fahad2019comparative}, the following formula can be used:
$$E_D = E_T -(P_S * T_E)$$
$E_T$ is the total amount of energy consumption by the system when running the experiment. $T_E$ is the duration of the program execution. $P_S$ is the amount of energy consumption when the system is idle. The dynamic energy consumption will then represent the energy consumption of the application. To do this a certain amount of control over the system is required and some precautions are necessary. The procedures that will be used are different for the DUT's and OS, but some will be the same for all DUT.
\begin{itemize}
    \item The machines are reserved exclusively for the experiments, this is to prevent divinations in the results.
    \item The networking will be disabled on the machines to ensure that these do not effect the results.
    \item The processes and the temperature of the machines will be measured before and after each experiment.
    \item After each batch the DUT will be restarted and performs it's setup before continuing to the next batch.
    \item The memory of the Test cases will not exceed the DUT's main memory to avoid memory swapping
    \item Some baseline power usage will be established by measurements before and after the experiments
\end{itemize}
Each batch will be tested on every DUT with a each type of OS these being Windows 10 and \todo[]{Insert some Linux version}
The specs for the different DUT varies, both in age, spec and manufacturer, this is done to get a wider scope of testes to see if certain measurements works better on certain systems. For the experiment one stationary workstation and two laptops will be used. The specs for the workstation can be seen in \cref(tab:Komplett), the surface pro 4 and surface books can be be seen in \cref{tab:surface} \todo[]{Insert specs when avilable}. Comparing these will help evaluate the differs measuring methods for the different system, and provide us with insight into why they may or may not differ in performance.
\begin{table}[]
    \begin{tabular}{ll}
    \hline
    \multicolumn{2}{|c|}{Komplett pc}           \\ \hline
    Processor:   & Intel i7-8700 CPU 3.20GHz    \\
    GPU:         & NVIDIA GeForce GTX 1060 6gb  \\
    Memory:      & DDR4 16G                     \\
    Disk:        & Samsung SSD 970 EVO Plus 1TB \\
    Motherboard: & TUF B360M-PLUS GAMING       
    \end{tabular}
    \caption{The spec for the Komplett pc}
    \end{table}
\begin{table}[H]
    \begin{tabular}{ll}
    \hline
    \multicolumn{2}{|c|}{Surface Pro 4}           \\ \hline
    Processor: & Intel i7-6650U CPU 2.20GHz (Intel Skylake) \\
    GPU:       & Intel iris Graphics 540          \\
    Memory:    & DDR3 16GB                         \\
    Disk:      & SAMSUNG MZFLV256HCHP-000MV 256GB   \\
    Ubuntu version:  & Ubuntu                            \\
    Linux kernel: & Linux        \\
    Windows version:& Windows 10 build: 19045.2251
    \end{tabular}
    \caption{The specifications for the Microsoft Surface Pro 4}
    \label{tab:surfacePro}
\end{table} 
Each DUT will have a slightly different setup depending on the hardware and the OS of the system. For the windows machines the varies unnecessary background processes will be disabled to ensure as little noise in the measurements as possible, the specific process disabled can be seen in \todo[]{Fine out which should be shut off}. The other difference between the DUT are the if they are stationary or mobile. The stationary DUT is where the system-level physical measurements are do, because of this it is desired for as little non-test case related power fluctuation, this is why the fans of the system will be set to max during all the measurement times. For the laptops this is not an issues as the measurements does not include the whole system. For the laptops some special controls are necessary for the measurements because measurement quality can become worse if the laptops are connected to the charger duration measurements\cite{E3Video}. To prevent this the power is shut of during the experiments, and then turned on afterwards, this cycle will repeat until all of the experiments are done.


\cite{sestoft2013microbenchmarks} \cite{fahad2019comparative} \cite{Bokhari2020r3}