\section{Measurement Methodology}\label{sec:Measurement_Methodology}

Now that the different components required to execute the experiment have been introduced, the following section will delve into the measurement methodology. When conducting the experiments three different kinds of measuring approaches are being utilized from the tools in \cref{sec:hardware}.
The first is hardware-based measurements, which are system-level physical measurements. Another approach is software-based measurements: RAPL, LHM, Intel Power Gadget and sometimes E3, where measurements are performed using on-chip sensors. Lastly, there are energy predictive models, which are utilized by E3, when there is no power meter chip in the DUT.\todo{We could remove this from here as it should be described in a previous section.} There is however some uncertainty about how E3 works, this is covered in \cref{sec:E3Experiments}. Of these three approaches, the most accurate is the system-level physical measurements which can provide highly accurate measurements for the DUT's energy consumption, but these measurements are for the entire DUT, so they cannot provide a view of the power consumption of applications on an individual level, which the E3 can do. It is however not necessary to get a per-application energy consumption, as only the test cases' energy consumption is to be calculated. In the following our framework is described as well as how to solve the concern of full DUT energy consumption measurements and e.g. only CPU measurements. Furthermore, the test cases on which the measurements are done is also presented and the programming languages used.


%A framework to runTo solve these issues and accommodate an  dynamic energy consumption can solve this, which is expanded upon in \cref{subsec:how_to_measure}



%\subsection{Measurement Methodology} \label[]{Measure_meth}
When conducting the experiments three different kinds of measuring approaches are being utilized from the tools in \cref{sec:hardware}.
The first is hardware-based measurements, which are system-level physical measurements. Another approach is software based measurements: RAPL, LHM, Intel Power Gadget and sometimes E3, where measurements are performed using on-chip sensors. Lastly, there are energy predictive models, which are utilized by E3, when there is no power meter chip in the DUT.\todo{We could remove this from here as it should be described in a previous section.} There is however some uncertainty about how E3 works, this is covered in \cref{sec:E3Experiments}. Of these three approaches, the most accurate is the system-level physical measurements which can provide highly accurate measurements for the DUT's energy consumption, but these measurements are for the entire DUT, so they cannot provide a view of the power consumption of applications on an individual level, which the E3 can do. It is however not necessary to get a per-application energy consumption, as only the test cases' energy consumption is to be calculated, dynamic energy consumption can solve this, which is expanded upon in \cref{subsec:how_to_measure}

%Dynamic energy consumption is an estimation of the application's power consumption. To calculate the dynamic energy consumption $E_D$, as presented by Fahad et al \cite{fahad2019comparative}, the following formula is used:
%$$E_D = E_T -(P_S * T_E)$$
%Where $E_T$ is the total energy consumption by the DUT when running the experiment, $T_E$ is the duration of the program execution and $P_S$ is the energy consumption when the DUT is idle. The dynamic energy consumption will then represent the energy consumption of the test case. To do this a certain amount of control over the DUT is required and some precautions are necessary. The procedures that will be used are different for the DUTs and OSes, but some will be the same for all DUTs as shown here:, and are a combination of procedures presented in \cref{sec:rw_measureing_methodology_setup} by Sestoft\cite*[]{sestoft2013microbenchmarks} and Fahad et al.\cite*[]{fahad2019comparative}:
%
%\begin{itemize}
%    \item The machines are reserved exclusively for the experiments, this is to prevent deviations in the results.
%    \item The networking will be disabled on the machines to ensure that these do not affect the results.
%    \item The processes and the temperature of the machines will be measured before and after each experiment.
%    \item After each batch, the DUT will be restarted and its setup phase is performed before continuing to the next batch.
%    \item The memory of the test cases\todo{Hvordan ved vi det?} will not exceed the DUT's main memory to avoid memory swapping
%    \item Some baseline power usage will be established by measurements before and after the experiments
%\end{itemize}

%Each batch will be tested on every DUT with each type of OS these being Windows 10 and Ubuntu.
%The specifications for the different DUT vary, both in age, hardware specifications and manufacturer, this is done to get a wider scope of tests to see if certain measurements work better on certain DUTs. For the experiment one desktop workstation and two laptops will be used. The specs for the workstation can be seen in \cref{tab:komplett}, the Surface Pro 4 can be seen in \cref{tab:surfacePro} and Surface Book can be seen in \cref{tab:surfaceBook}. Comparing these will help evaluate the different measuring methods for the different DUTs, and provide insight into why they may or may not differ in performance.
%
%
%Each DUT will have a slightly different setup depending on the hardware and the OS of the DUT. For the Windows and Linux machines, the various unnecessary background processes will be disabled to ensure as little noise in the measurements as possible\cite*[]{sestoft2013microbenchmarks}, the specific process disabled can be seen in \cref{tab:disabled_proc}.\todo{How did we decide this} There are no disabled processes on the Linux OS, this was choice was made on the general assumption that Ubuntu has less unnecessary background processes than Windows.\todo{Does a proper source exist for this statement} A difference between the DUTs is if whether they are desktops or laptops. On the desktop it is possible to perform system-level physical measurements because the measurement is system-wide it is important to have as little non test case related power fluctuation as possible, this is why the fans of the DUT will be set to max on the workstation. For the laptops, this is not an issue as the measurements do not include the whole system. For the laptops, some special controls are necessary for the measurements because measurement quality can become worse if the laptops are connected to the charger during measurements\cite{E3Video}. To prevent this the charger is shut off during the experiments, and then turned on to fully recharge the DUT during the setup phase, this cycle will repeat until all of the experiments are done.

%\begin{table}[ht]
    \centering
    \begin{tabular}{| l | l |}
    \hline
    \textbf{Windows}    & \textbf{Linux}    \\ \hline
    AsusUpdateCheck     &   \\ \hline
    AsusDownLoadLicense &   \\ \hline
    msedge              &   \\ \hline
    OneDrive            &   \\ \hline
    GitHubDesktop       &   \\ \hline
    Microsoft.Photos    &   \\ \hline
    SkypeApp            &   \\ \hline
    SkypeBackgroundHost &   \\ \hline
    \end{tabular}
    \caption{The background processes which are disabled when running a test case}
    \label{tab:disabled_proc}
\end{table}
\subsection{Framework}\label[subsec]{subsec:framework}

In order to answer the first part of \textbf{RQ1}, a framework is introduced. This framework should be able to systematically measure the energy consumption of test cases on different DUTs and OSs. The measurements should be conducted in a rotated round-robin fashion using R3-Validation.\newline

The first thing to consider, is the measuring instruments and test cases, as both of these need to be implemented in a generic fashion, to be able to alter between them during runtime. Because of this, both are implemented using interfaces, as this was concluded not to impact performance for Java in the work by Sestoft\cite[]{sestoft2013microbenchmarks}, where it is assumed a similar conclusion would be made for C\#. When considering the measuring instrument, the implementation can be seen in \cref{lst:measurement_instruments}, where the \texttt{IMeasuringInstrument} has four methods implemented. The \texttt{Start} and \texttt{Stop}, starts and stops the measuring instruments, where the input \texttt{date} in \texttt{Start} is used to name the file in which the results are saved. \texttt{GetName} is used to get the name of the current measuring instrument, also related to saving the data correctly. Lastly, \texttt{ParseCsv} is used to transform the data saved for a measuring instrument for one measurement, on a given time, into a \texttt{DtoRawData}, this being the format accepted by the database, which will be introduced in \cref{subsec:sql}.

\begin{lstlisting}[caption=The interface used to implement the measurement instruments, label={lst:measurement_instruments}]
public interface IMeasurementInstrument
{
    string GetName();
    DtoRawData ParseCsv(string path, int experimentId, 
        DateTime startTime);
    public void Start(DateTime date);
    public void Stop();
}
\end{lstlisting}

Next up, the implementation for the test case can be seen in \cref{lst:test_case}, where the \texttt{ITestCase} implements four methods. Here the \texttt{GetLanguage} and \texttt{GetName} return the language the test case is implemented in, and the name of it respectively, which is used when saving the results. The \texttt{GetProgram} returns a \texttt{DtoTestCase} object containing data about the test case.  The \texttt{DtoTestCase} is saved in the database, and the last method \texttt{Run} executes the test case.



\begin{lstlisting}[caption=The interface used to implement the test cases, label={lst:test_case}]
    public interface ITestCase
    {
        DtoTestCase GetProgram();
        string GetLanguage();
        void Run();
        public string GetName();
    }
\end{lstlisting}

To get a better context of how the \texttt{IMeasuringInstrument} and \texttt{ITestCase} is used in the framework, pseudo code illustrating this can be seen in \cref{lst:execute_async}. \texttt{ExecuteAsync} will in line 5 wait until a stable condition is reached, where a stable condition is defined as a condition where the battery is charged to a certain level, and the CPU is below some temperature. Each time \texttt{ExecuteAsync} is executed, one test case will be used, as expressed in line 8. In lines 10-12 conditions regarding whether the framework should keep running the test case, or the DUT should restart. This will be decided based on three conditions, the first one being the method \texttt{ShouldStopMeasurement()}. This method will return \texttt{true} if the temperature is above some limit, or the battery levels are too low. \texttt{ismMasurementValid} ensures the last measurement was valid, and \texttt{EnoughEntires} ensures the method will not run forever, by restarting the DUT every time each measuring instrument has measured a test case a set number of times. On each iteration of the \texttt{While}-loop, a new measuring instrument is chosen, as can be seen in line 14, where \texttt{GetNextMeasuringInstrument} will take the next measuring instrument, based on which one was used in the last iteration. On the first iteration, the measuring instrument starting is based on which measuring instrument started last time the DUT was running, which is saved in the database, as covered in \cref{subsec:sql}. In line 18, the measurement is performed, and before and after this, different dependencies are initialized or removed. This includes for example a connection to the database, which is not required during the measurements. Lastly, in lines 25-26, the results are saved, and the DUT is restarted. 


\begin{lstlisting}[caption=The method handeling dependencies and the validity of the results, label={lst:execute_async}]
public async Task ExecuteAsync()
{
    CreateNonexistingFolders();

    await WaitTillStableState();
    var isExpValid = true;

    var tc = GetTestCase();

    while (!ShouldStopExperiment() 
            && isExperimentValid 
            && !EnoughEntires())
    {
        var mi = GetNextMeasuringInstrument();

        RemoveDependencies();

        isExpValid = await RunExperiment(mi, tc);

        InitializeDependencies();

        await Task.Delay(MinutesBetweenExperiments);
    }

    await SaveProfilers();
    RestartComputer();
}
\end{lstlisting}

To get a better understanding of how the measurements are performed, pseudo-code for \texttt{RunMeasurement} from line 18 in \cref{lst:execute_async} can be seen in \cref{lst:run_measurements}. This method will count how many times the test case is executed during the the given duration and handle measurements of temperatures and battery levels before and after the measurements in lines 4, 6 and 19 respectively. In line 7 the wifi/ethernet is disabled before the measurements are performed, line 8 triggers the garbage collection, to ensure all removed dependencies are cleaned up, and in line 11 the stopwatch and measuring instrument will be started. One thing to consider here is the order in which the measuring instrument and the stopwatch are started, since if the stopwatch is started first, it will also measure the time it took for the measuring instrument to start and stop, and if the measuring instrument is started first, the energy consumption of starting and stopping the stopwatch will be included. In this work, the stopwatch will be started first, as this was the setup in the work by Pereira et al.\cite[]{Pereira2017}, where both versions of either starting the stopwatch or measuring instrument first were tested, and the impact was concluded to be insignificant. In addition to this, the impact will become even small if the test case executes multiple times between the start and stop of the measuring instrument. In lines 13, the \texttt{ITestCase} will execute, until some predefined number of minutes has passed. When the measurement is done, the wifi/ethernet will be enabled again in line 21, and the result will be handled.



\begin{lstlisting}[caption=The method running the test case, label={lst:run_experiments}]
public bool RunExperiment(IMeasuringInstrument mi, ITestCase tc)
{
    var stopwatch = new Stopwatch();
    var counter = 0;

    var (inTemp, inBat) = InitializeExperiment();
    await DisableWifi();

    RunGarbageCollection();

    var startTime = StartTimeAndProfiler();

    while (startTime.AddMinutes(Duration) > DateTime.UtcNow)
    {
        tc.Run();
        counter += 1;
    }

    var (stopTime, duration) = StopTimeAndProfiler();

    await EnableWifiAndDependencies();

    var (endTemp, endBat) = GetEndMeasurements();
    var experimentId = await EndExperiment();

    return await HandleResultsIfValid();
}
    \end{lstlisting}




% When running the experiments, what code is best to test a language

\section{Experimental Setup}

When running experiments on the energy consumption of hardware, it is not straightforward, as many factors can impact the measurements. This is covered in the work by Sestoft\cite[]{sestoft2013microbenchmarks}, where a framework to measure the execution time of microbenchmarks is created in an iterative manner where different pitfalls are uncovered. The reason why energy consumption measurements are not straightforward is because of a large number of factors which has been introduced in complex modern systems. This is especially clear on managed execution platforms like the Common Language Infrastructure e.g .Net from Microsoft or the Java Virtual Machine (JVM) where software in an intermediate form is compiled to real machine code at runtime by just-in-time (JIT) compilation, which affects the execution time for a couple  reasons. One thing is the start-up overhead of the JIT or the adaptive optimization of JIT compilation. What the optimization does is locate code executed only a few times, and code executed a lot, which results in a prioritization, where a lot of time is used to generate optimized code for code executed many times and quickly generated code is made when the code is executed only a few times. The JIT compiler also avoids using a lot of time on code analysis, which can result in cases where the generated code works well in simpler contexts, and in more complex contexts, the performance is not as good. Another factor the programmer cannot control is automatic memory management, which may decide to run the garbage collection during the experiments, resulting in unreliable results. The same can be said for the operating system, processors and memory management systems.

In the process of creating the framework for measuring execution time, Sestoft\cite*[]{sestoft2013microbenchmarks} uses a \texttt{multiply} method which performs 20 floating point multiplications, an integer bitwise "and" and a conversion from a 32-bit int to a 64-bit double. During the first phases, one observation is that when running the experiments for many iterations, the execution time drops to zero. This happens because the JIT compiler observes the result of the method is not used for anything, resulting in it skipping the method entirely. When measuring the runtime of the experiment, only measuring one execution is deemed too little as the results vary too much. This should rather be multiple runs, and then looking at the average runtime and the standard deviation. These multiple runs are in this study deemed to be until the execution time exceeds $0.25$ seconds, as this is long enough to avoid problems with virtual machine startup and clock resolution. In the end, some additional pitfalls are noted, including:

\begin{itemize}
    \item Shut down as many background services as possible, as this can impact performance
    \item The generation of logging and debug messages can use more than 90\% of execution time, so this should be disabled
    \item An IDE uses a lot of CPU time and causes slower execution time because of debugging code, so do not execute through IDEs
    \item Disable power saving schemes so the CPU does not reduce its speed during benchmark runs.
    \item Compile with relevant optimization options so the generated bytecode does not include code to accommodate debugging
    \item Different implementations of .NET and JVM have different characteristics and different garbage collectors
    \item Different CPU brands, versions (like desktop or mobile) and hardware (like ram) have different characteristics
    \item Reflect on results, and if they look slow/fast something might have been overlooked.
\end{itemize}

%% benchmarking c# for energy consumption ørsted nielsen


Bokhari et al\cite{Bokhari2020r3} found that when running benchmarks comparing different variants of the same program on Android systems, noise had an impact on the results. This was due to noise coming from several uncontrollable factors. Firstly the background processes could not be fully controlled during the execution of the experiments. Secondly the memory consumption of the Android system and background processes. Thirdly the battery voltage, because even though the system was fully charged at the starting point as well as the remaining voltage level after. To solve this they propose a method called \textit{R3-VALIDATION} which is a rotated round-robin approach to running the program variants, which ensures more fair execution conditions. In this approach the variants (A, B, C) are rotated as follows: setup, ABC, ABC, setup, BCA, BCA, setup, CAB, CAB. Where the setup phase is a restart, initialization and recharge of the system. They achieved more consistent system states from this approach.
\subsection{Test case languages}\label[subsec]{subsec:language}

When considering what languages to use in the implementation of the test cases, different approaches has been seen. One such approach is from the work by Pereira et al.\cite{Pereira2017} 27 different languages was chosen, where the aim was to compare similar implementations in different languages. The aim in this work is however a bit different, where measurement instruments are compared, as is mentioned in \textbf{RQ} 2. Because of this, the specific language is not expected to make a huge difference, which is why C\# is chosen, as this is the preferred language of the authors. 