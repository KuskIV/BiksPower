\section{Experimental/Test Setup}\label[subsection]{ExperimentalSetup}
After the different components required in order to execute the experiment has been introduced, it is now time to take a look into the experiment. This will be in regards to the languages used, how the data is saved, how many times the experiments are run ect.


\subsection{Sample size}
The amount of times to run the experiment are important to ensure enough tests are conducted for each benchmark to arrive at a representative mean and standard deviation, while also avoiding wasting resources by running an large amount of unnecessary test. There are several approaches used in the literature. As example some papers use what seems like an arbitrary number of samples\cite{Pereira2017,Koedijk2022diff,Georgiou2020}. Another method is to base the number of samples on how many times the experiment can be run within in a time-frame\cite{sestoft2013microbenchmarks}. In this report a formula will be used to calculate the sample size. One formula or family of formulas are Cochran's formula.\cite{Cochran, kotrlik2001organizational,israel1992determining},\todo{Other methods for calculating sample size?} which gives the minimum number of required observations for a performance metrics within a specified standard deviation. However there are several versions of the formula.
The first one in \cref{cochransEQ1}\footnote{Note the terminology used by Cochran is different to the one utilized in this report} is for categorical data when the population is large\cite{israel1992determining}:

\begin{equation}
    n_0 = \frac{Z^2*p*q}{e^2}
    \label[equation]{cochransEQ1}
\end{equation}

\begin{itemize}
    \item $n_0$ is the number of samples
    \item $Z$ is the abscissa of the normal curve which removes an area $\alpha$ at the tails of the distribution. (Z-score) Where $1 - \alpha$ is the desired confidence level.%is the z-value, which is found using the z-table and represents the confidence level
    \item $pq$ is an estimate of variance, where
    \begin{itemize}
        \item $p$ is the estimated proportion of an attribute in the data
        \item $q$ is $1 - p$
    \end{itemize}
    \item $e$ is desired level of precision (acceptable margin of error)
\end{itemize}


Then there is a correction step for smaller populations since then a smaller sample size is required, shown in \cref{cochransCorrection, kotrlik2001organizational}.

\begin{equation}
    n = \frac{n_0}{1+\frac{(n_0-1)}{N}}
    \label[equation]{cochransCorrection}
\end{equation}

Furthermore there is a simplified formula called Yamane's formula 


The other one in \cref{cochransEQ2} is for continuous data. However it should still be considered if categorical variables will play an important role in the data analysis, then \cref{cochransEQ1} should be used.

\begin{equation}
    n_0 = \frac{Z^2*\sigma^2}{e^2}
    \label[equation]{cochransEQ2}
\end{equation}

Where, instead of $pq$ we have $\sigma$ which is the standard deviation of the data.




For our data \cref{cochransEQ2} seem to most accurately fit the description and is therefore chosen. With Cochran's formula given a acceptable margin of error, desired confidence level and an estimate of the standard deviation a sample size can be calculated. So these variables are determined as following. 

The confidence level is how little the results must deviate. A confidence level at 95\% would represents 95\% of the data points would match 95\% of the times. If a confidence level of 95\% is desired and confidence level $= 1 - \alpha$ then $\alpha = 0.05$. 
A margin of error of 5\% is commonly chosen and is deemed acceptable for categorical data, but for continuous data a 3\% margin of error is deemed acceptable\cite{kotrlik2001organizational} and as such is chosen.

The Z-score reflects how many standard deviations a sample is from the mean. An approximate score can be found in a Z-table when there is a desired confidence level or $\alpha$. The most commonly used $\alpha$ is $0.05$ or $0.01$.\cite{kotrlik2001organizational} In this report $0.05$ is chosen which gives a confidence level of 95\%. \todo{I have not found a reason as to why yet.}From the Z-table the estimated Z-value is then $1.96$. 

Lastly an estimate of the standard deviation is also needed, which is not available. Cochran listed four methods for estimating the standard deviation in case it is not initially available:
\begin{enumerate}
    \item The sample is taken in two steps. The first step is used to determine how many further samples are required in step two, based on the standard deviation in the first set of samples.
    \item Utilizing results of a pilot study
    \item Utilizing results from similar study
    \item Come up with a guess assisted with logical mathematical results.
\end{enumerate}


Method one and two are both based on the same principle of getting an initial smaller sample. Method three is to the extend of our knowledge not feasible since no results from a study similar enough is available. Method four requires more knowledge than we posses to give a qualified estimate. Therefore method one is chosen. Now for step one when making the small sample it is required to know how many samples to acquire. The Central Limit Theorem says that the mean of a sample will become close to the mean of the overall population in corelation with an increase in the sample size. In general a sample size of 30 to 50 or above is enough for the central limit theorem to hold. Which means the distributions of the sample means are close to normally distributed.\todo{We need to pick an actual number and also find a better source than investiopedia}

% Influence of pragmming paradigms picks 100 based on 9781118805350.

From the initial sample an estimated standard deviation of each parameter measured can be acquired. Then the minimum number of samples required can be calculated for each parameter. Then the largest of the values and run the experiments for that amount of times to get the minimum required samples.


% Focus on report influence of programming paradigms.

% Probabply not normal distributed - Jeppe alcholt
