\subsection{SQL}\label[subsec]{subsec:sql}

When considering existing work as presented in \cref{ch:related_work}, most existing work save data in comma-separated files(CSV)\cite[]{Koedijk2022diff,Pereira2017}. This will however not be done in our work, as we argue SQL will improve the workflow and the ability to analyze the data later. The ability to analyze the data later in a convenient manner is argued to be important, as a lot of data will be recorded from the different DUTs, measuring instruments, and test cases. Relational data like this, where one test case, DUT, and measuring instrument can be used in zero or more experiments are an ideal use case for SQL, where it enables the user to process the data later in an optimized manner\cite[]{Silberschatz2019}.

\paragraph*{}
In this section, the notion of a configuration will also be introduced, as this is an important part of the experiments. A configuration will in this case save under which circumstances the experiments were executed, including what temperature and battery limits were used when reaching a stable condition and when to stop running the experiments in the framework, as introduced in \cref{subsec:framework}.

\paragraph*{}
In this case, the test case, measuring instrument, DUT, and configuration will be contained in their own tables, to avoid replicated data, where each table will have its respective primary key presented as an auto-incremented integer. An integer is chosen, as it is easy to generate the next primary key and the primary key will not be related to the data, thus it will never be changed. The primary key of a row, in a table, will also be used by other tables when referring to that specific row, as a foreign key. 

\paragraph*{}
When performing a measurement, it will refer to a test case, a DUT, a configuration, and a measuring instrument through foreign keys, in addition to the data saved. The data will in this case both include the temperature, battery level, and energy consumption. The energy consumption will be saved from each measurement as both the entire energy consumption from one measurement and a time series representing how the energy consumption evolved. The temperature and battery levels at the start and end of the measurements will be represented as zero to many relationships. The structure of the data can be seen in \cref{fig:uml_diagram} and is used for two things during the experiments. The relevant components for capturing results from \cref{fig:uml_diagram} are as following:

\begin{itemize}
    \item Test case: Represents a test program through a unique ID, name, and which language it was implemented in.
    \item Measuring instrument: Represents a measuring instrument through a unique ID and name
    \item DUT: Represents a DUT through a unique ID, name, operating system, and version\feetnote{The version is to separate different versions of the framework. Meaning each time a change is made to the framework, the version increments.}. 
    \item RawData: Represents the output from the measuring instrument. This is represented through a unique ID, a measurement ID, a value, and the date of execution. The value here is a serialized object representing the data, as the different measuring instruments had different formats in their output. The raw data will represent the sum over the entire measurement, meaning how much energy was consumed as a single value.
    \item TimeSeries: Also represents the output from the measuring instrument in a similar manner as in RawData. The difference is how the values will be a time series during the entire measurement, with some interval between the data points. This can be used to analyze during which parts of the test case the most energy is consumed.
    \item Configuration: Represents what configuration was used for a measurement. Here min/max temp and battery represent the limits the system had to exceed before the results were no longer useful, and a system restart and cooldown period are required. Between represents the cooldown period between two test case runs in minutes, and duration specifies the minimum duration the test case had to run for, denoted in minutes. This is also represented through a unique ID.
    \item Environment: Represents what conditions the DUT was under at the start and end of the experiments, including battery levels and temperatures. The environment is represented through a unique ID, an experiment ID, a time, a type, a name, and a value. Here the name is the sensor name e.g. 'CORE \#1', and the type is more general like 'CpuTemperature'.
    \item Measurement: The measurement ties all other tables together, where one measurement is represented with a unique ID, configuration ID, measuring instrument ID, DUT ID, test case ID and start/end times. This enables one measurement to several rows in Environment etc. In addition to this, runs represent how many times the test case was executed during the duration specified in the Configuration, and iteration represents how many times the test case has been measured at the given point in time, by the given measuring instrument since the last restart of the DUT. This is relevant for the R3-Validation, where for the first ABC, the iteration will be 1 for all measuring instruments, the second BCA will be 2, etc.
\end{itemize}

One thing of focus when creating the database, was to make it as generic as possible. This is why the Environment table is as it is, to facilitate different kinds of Environment measurements. This can also be seen when considering the test case, where it is noted what language it is implemented in, to make it possible to include different languages, if it is deemed important for the experiments.\newline


The second use case for the database is to save the state of the framework upon a restart of the DUT. This is relevant for R3-Validation, where after each restart, a new measuring instrument needs to start. The relevant tables for this are as follows:

\begin{itemize}
    \item Test case: Same definition as before
    \item DUT: Same definition as before
    \item Run: Represents what measuring instrument was the first measuring instrument last time an experiment was run on for a given test case on a given DUT. This is represented using a unique run ID, a test case ID, and a DUT ID. In addition to this is the value, which represents the different measuring instruments used in the experiment, and which one started last time.
\end{itemize}

\begin{figure}[H]
    \centering
    \begin{tikzpicture}
        \begin{object}[text width=4 cm]{TestCase}{0 ,1}
            \attribute{TestCaseId : INT}
            \attribute{Name : VARCHAR}
            \attribute{Language : VARCHAR}
        \end{object}
        \begin{object}[text width=4 cm]{Dut}{10,1}
            \attribute{DutId : INT}
            \attribute{Name : VARCHAR}
            \attribute{OS : VARCHAR}
            \attribute{Version : INT}
        \end{object}
        \begin{object}[text width=4 cm]{Run}{5,5}
            \attribute{RunId : INT}
            \attribute{DutId : INT}
            \attribute{TestCaseId : INT}
            \attribute{Value : VARCHAR}
        \end{object}
        \begin{object}[text width = 4 cm]{MeasuringInstrument}{5,1}
            \attribute{InstrumentId : INT}
            \attribute{Name : VARCHAR}
        \end{object}
        \begin{object}{Measurement}{6.5,-2.5}
            \attribute{MeasurementId : INT}
            \attribute{ConfigId : INT}
            \attribute{InstrumentId : INT}
            \attribute{DutId : INT}
            \attribute{TestCaseId : INT}
            \attribute{Runs : INT}
            \attribute{Iteration : INT}
            \attribute{FirstMeasuring : VARCHAR}
            \attribute{StartTime : DATETIME(6)}
            \attribute{EndTime : DATETIME(6)}
        \end{object}
        \begin{object}[text width=4 cm]{RawData}{0, -10}
            \attribute{RawDataId : INT}
            \attribute{MeasurementId : INT}
            \attribute{Value : VARCHAR}
            \attribute{Time : DATETIME(6)}
        \end{object}
        \begin{object}[text width=4 cm]{TimeSeries}{0, -4.5}
            \attribute{TimeSeriesId : INT}
            \attribute{MeasurementId : INT}
            \attribute{Value : VARCHAR}
            \attribute{Time : DATETIME(6)}
        \end{object}
        \begin{object}[text width = 4 cm]{Configuration}{5,-10}
            \attribute{ConfigurationId : INT}
            \attribute{MinTemp : INT}
            \attribute{MaxTemp : INT}
            \attribute{Between : INT}
            \attribute{Duration : INT}
            \attribute{MinBattery : INT}
            \attribute{MaxBattery : INT}
            \attribute{Version : INT}
        \end{object}
        \begin{object}[text width = 4 cm]{Environment}{10, -10}
            \attribute{EnvironmentId : INT}
            \attribute{MeasurementId : INT}
            \attribute{Time : DATETIME(6)}
            \attribute{Name : VARCHAR}
            \attribute{Value : INT}
            \attribute{Type : VARCHAR}
        \end{object}
        
        \association{Run}{}{0..*}{TestCase}{}{1}
        \association{Run}{}{0..*}{Dut}{}{1}
        \association{Measurement}{}{0..*}{Dut}{}{1}
        \association{Measurement}{}{0..*}{MeasuringInstrument}{}{1}
        \association{Measurement}{}{0..*}{Dut}{}{1}
        \association{Measurement}{}{1}{RawData}{}{1}
        \association{Measurement}{}{1}{TimeSeries}{}{1}
        \association{Measurement}{}{0..*}{Configuration}{}{1}
        \association{Measurement}{}{0..*}{Environment}{}{0..*}
        \association{Measurement}{}{0..*}{TestCase}{}{1}
    \end{tikzpicture}
    \caption{An UML diagram representing the tables in the SQL database} \label{fig:uml_diagram}
\end{figure}



