\subsection{Test Cases}\label[subsec]{subsec:test_cases}

After the language has been decided, another aspect to consider is which test cases make sense to run during the experiments. Based on what was found in \cref{ch:related_work}, two common repositories are used, either Rosetta Code\cite[]{rosetta_code} or the Computer Language Benchmark Game\cite[]{benchmark_game}. From Rosetta and the Computer Language Benchmark Game, different test cases are chosen in literature, depending on what part of the DUT the test case should target. In the end, the choices were made based on the work by Lima et al.\cite[]{greenland2016statistical} and Koedijk et al.\cite[]{Koedijk2022diff}, where the benchmarks tested either the CPU, memory, synchronization, or IO. The choices for the experiments conducted in this work can be seen in \cref{tab:benchmarks}. When choosing the specific implementation of each test case, it was done based on speed, where the fastest implementation was chosen for the newest version of C\#. Unsafe solutions were also disregarded, as we consider the other implementations to be more realistic.

\begin{table}[]
    \begin{tabular}{||c c c c||}
    \hline
    Source & Name & Lines & Target \\[0.5ex] 
    \hline\hline
    Benchmark game & Binary Trees &  & Memory \\
    & Reverse Complement &  & Memory \\
    & Fannkuch-Redux &  & CPU \\
    & Nbody &  & CPU \\
    & Fasta &  & CPU, Memory, Synchonization and IO \\
    Rosetta & Dining philosophers &  & Synchronization \\ [1ex] 
    \hline
    \end{tabular}
    \caption{The benchmarks chosen for the experiments.}
    \label{tab:benchmarks}
\end{table}
